% Gerado por https://pasdoc.github.io/PasDoc 0.16.0
\documentclass{report}
\usepackage{hyperref}
% WARNING: THIS SHOULD BE MODIFIED DEPENDING ON THE LETTER/A4 SIZE
\oddsidemargin 0cm
\evensidemargin 0cm
\marginparsep 0cm
\marginparwidth 0cm
\parindent 0cm
\textwidth 16.5cm

\ifpdf
  \usepackage[pdftex]{graphicx}
\else
  \usepackage[dvips]{graphicx}
\fi

% definitons for warning and note tag
\usepackage[most]{tcolorbox}
\newtcolorbox{tcbwarning}{
 breakable,
 enhanced jigsaw,
 top=0pt,
 bottom=0pt,
 titlerule=0pt,
 bottomtitle=0pt,
 rightrule=0pt,
 toprule=0pt,
 bottomrule=0pt,
 colback=white,
 arc=0pt,
 outer arc=0pt,
 title style={white},
 fonttitle=\color{black}\bfseries,
 left=8pt,
 colframe=red,
 title={Warning:},
}
\newtcolorbox{tcbnote}{
 breakable,
 enhanced jigsaw,
 top=0pt,
 bottom=0pt,
 titlerule=0pt,
 bottomtitle=0pt,
 rightrule=0pt,
 toprule=0pt,
 bottomrule=0pt,
 colback=white,
 arc=0pt,
 outer arc=0pt,
 title style={white},
 fonttitle=\color{black}\bfseries,
 left=8pt,
 colframe=yellow,
 title={Note:},
}

\begin{document}
% special variable used for calculating some widths.
\newlength{\tmplength}
\chapter{Unit Classes{\_}C}
\section{Visão Geral}
\begin{description}
\item[\texttt{\begin{ttfamily}TJSON{\_}BaseObject\end{ttfamily} Classe}]
\item[\texttt{\begin{ttfamily}TNSComponent\end{ttfamily} Classe}]
\item[\texttt{\begin{ttfamily}TClass\end{ttfamily} Classe}]
\end{description}
\begin{description}
\item[\texttt{Message}]
\item[\texttt{CloneComponent}]
\item[\texttt{StrJSonToJSONObject}]
\item[\texttt{JSONObjectToStrJSon}]
\item[\texttt{StrJSonToArrays}]
\item[\texttt{ArraysToJSONValue}]
\item[\texttt{IsValidPtr}]
\item[\texttt{DISCARD}]
\end{description}
\section{Classes, Interfaces, Objetos e Registros}
\subsection*{TJSON{\_}BaseObject Classe}
\subsubsection*{\large{\textbf{Hierarquia}}\normalsize\hspace{1ex}\hfill}
TJSON{\_}BaseObject {$>$} TObject
%%%%Descrição
\subsubsection*{\large{\textbf{Métodos}}\normalsize\hspace{1ex}\hfill}
\paragraph*{ObjectToJSON}\hspace*{\fill}

\begin{list}{}{
\settowidth{\tmplength}{\textbf{Declaração}}
\setlength{\itemindent}{0cm}
\setlength{\listparindent}{0cm}
\setlength{\leftmargin}{\evensidemargin}
\addtolength{\leftmargin}{\tmplength}
\settowidth{\labelsep}{X}
\addtolength{\leftmargin}{\labelsep}
\setlength{\labelwidth}{\tmplength}
}
\begin{flushleft}
\item[\textbf{Declaração}\hfill]
\begin{ttfamily}
public class function ObjectToJSON{$<$}T : class{$>$}(myObject: T): TJSONValue;\end{ttfamily}


\end{flushleft}
\end{list}
\paragraph*{JSONToObject}\hspace*{\fill}

\begin{list}{}{
\settowidth{\tmplength}{\textbf{Declaração}}
\setlength{\itemindent}{0cm}
\setlength{\listparindent}{0cm}
\setlength{\leftmargin}{\evensidemargin}
\addtolength{\leftmargin}{\tmplength}
\settowidth{\labelsep}{X}
\addtolength{\leftmargin}{\labelsep}
\setlength{\labelwidth}{\tmplength}
}
\begin{flushleft}
\item[\textbf{Declaração}\hfill]
\begin{ttfamily}
public class function JSONToObject{$<$}T : class{$>$}(json: TJSONValue): T;\end{ttfamily}


\end{flushleft}
\end{list}
\subsection*{TNSComponent Classe}
\subsubsection*{\large{\textbf{Hierarquia}}\normalsize\hspace{1ex}\hfill}
TNSComponent {$>$} TComponent
%%%%Descrição
\subsubsection*{\large{\textbf{Propriedades}}\normalsize\hspace{1ex}\hfill}
\paragraph*{Alias}\hspace*{\fill}

\begin{list}{}{
\settowidth{\tmplength}{\textbf{Declaração}}
\setlength{\itemindent}{0cm}
\setlength{\listparindent}{0cm}
\setlength{\leftmargin}{\evensidemargin}
\addtolength{\leftmargin}{\tmplength}
\settowidth{\labelsep}{X}
\addtolength{\leftmargin}{\labelsep}
\setlength{\labelwidth}{\tmplength}
}
\begin{flushleft}
\item[\textbf{Declaração}\hfill]
\begin{ttfamily}
published property Alias : AnsiString Read GetAlias Write SetAlias;\end{ttfamily}


\end{flushleft}
\end{list}
\paragraph*{Path}\hspace*{\fill}

\begin{list}{}{
\settowidth{\tmplength}{\textbf{Declaração}}
\setlength{\itemindent}{0cm}
\setlength{\listparindent}{0cm}
\setlength{\leftmargin}{\evensidemargin}
\addtolength{\leftmargin}{\tmplength}
\settowidth{\labelsep}{X}
\addtolength{\leftmargin}{\labelsep}
\setlength{\labelwidth}{\tmplength}
}
\begin{flushleft}
\item[\textbf{Declaração}\hfill]
\begin{ttfamily}
public property Path : AnsiString Read {\_}Path Write SetPath;\end{ttfamily}


\end{flushleft}
\end{list}
\paragraph*{ID{\_}Dynamic}\hspace*{\fill}

\begin{list}{}{
\settowidth{\tmplength}{\textbf{Declaração}}
\setlength{\itemindent}{0cm}
\setlength{\listparindent}{0cm}
\setlength{\leftmargin}{\evensidemargin}
\addtolength{\leftmargin}{\tmplength}
\settowidth{\labelsep}{X}
\addtolength{\leftmargin}{\labelsep}
\setlength{\labelwidth}{\tmplength}
}
\begin{flushleft}
\item[\textbf{Declaração}\hfill]
\begin{ttfamily}
public property ID{\_}Dynamic : AnsiString Read GetID{\_}Dynamic;\end{ttfamily}


\end{flushleft}
\end{list}
\paragraph*{RecPosition}\hspace*{\fill}

\begin{list}{}{
\settowidth{\tmplength}{\textbf{Declaração}}
\setlength{\itemindent}{0cm}
\setlength{\listparindent}{0cm}
\setlength{\leftmargin}{\evensidemargin}
\addtolength{\leftmargin}{\tmplength}
\settowidth{\labelsep}{X}
\addtolength{\leftmargin}{\labelsep}
\setlength{\labelwidth}{\tmplength}
}
\begin{flushleft}
\item[\textbf{Declaração}\hfill]
\begin{ttfamily}
public property RecPosition: Longint Read {\_}RecPosition   Write  SetRecPosition;\end{ttfamily}


\end{flushleft}
\end{list}
\paragraph*{CurrentRecord}\hspace*{\fill}

\begin{list}{}{
\settowidth{\tmplength}{\textbf{Declaração}}
\setlength{\itemindent}{0cm}
\setlength{\listparindent}{0cm}
\setlength{\leftmargin}{\evensidemargin}
\addtolength{\leftmargin}{\tmplength}
\settowidth{\labelsep}{X}
\addtolength{\leftmargin}{\labelsep}
\setlength{\labelwidth}{\tmplength}
}
\begin{flushleft}
\item[\textbf{Declaração}\hfill]
\begin{ttfamily}
public property CurrentRecord: Longint Read {\_}CurrentRecord   Write  SetCurrentRecord;\end{ttfamily}


\end{flushleft}
\end{list}
\paragraph*{Procedure{\_}GlobalStr}\hspace*{\fill}

\begin{list}{}{
\settowidth{\tmplength}{\textbf{Declaração}}
\setlength{\itemindent}{0cm}
\setlength{\listparindent}{0cm}
\setlength{\leftmargin}{\evensidemargin}
\addtolength{\leftmargin}{\tmplength}
\settowidth{\labelsep}{X}
\addtolength{\leftmargin}{\labelsep}
\setlength{\labelwidth}{\tmplength}
}
\begin{flushleft}
\item[\textbf{Declaração}\hfill]
\begin{ttfamily}
public property Procedure{\_}GlobalStr: AnsiString read GetProcedure{\_}GlobalStr write {\_}Procedure{\_}GlobalStr;\end{ttfamily}


\end{flushleft}
\end{list}
\paragraph*{Command}\hspace*{\fill}

\begin{list}{}{
\settowidth{\tmplength}{\textbf{Declaração}}
\setlength{\itemindent}{0cm}
\setlength{\listparindent}{0cm}
\setlength{\leftmargin}{\evensidemargin}
\addtolength{\leftmargin}{\tmplength}
\settowidth{\labelsep}{X}
\addtolength{\leftmargin}{\labelsep}
\setlength{\labelwidth}{\tmplength}
}
\begin{flushleft}
\item[\textbf{Declaração}\hfill]
\begin{ttfamily}
published property Command: Integer Read {\_}Command   Write  Set{\_}Command;\end{ttfamily}


\end{flushleft}
\end{list}
\paragraph*{Module}\hspace*{\fill}

\begin{list}{}{
\settowidth{\tmplength}{\textbf{Declaração}}
\setlength{\itemindent}{0cm}
\setlength{\listparindent}{0cm}
\setlength{\leftmargin}{\evensidemargin}
\addtolength{\leftmargin}{\tmplength}
\settowidth{\labelsep}{X}
\addtolength{\leftmargin}{\labelsep}
\setlength{\labelwidth}{\tmplength}
}
\begin{flushleft}
\item[\textbf{Declaração}\hfill]
\begin{ttfamily}
published property Module: Byte Read GetModule   Write  SetModule;\end{ttfamily}


\end{flushleft}
\end{list}
\paragraph*{HelpCtx{\_}StrModule}\hspace*{\fill}

\begin{list}{}{
\settowidth{\tmplength}{\textbf{Declaração}}
\setlength{\itemindent}{0cm}
\setlength{\listparindent}{0cm}
\setlength{\leftmargin}{\evensidemargin}
\addtolength{\leftmargin}{\tmplength}
\settowidth{\labelsep}{X}
\addtolength{\leftmargin}{\labelsep}
\setlength{\labelwidth}{\tmplength}
}
\begin{flushleft}
\item[\textbf{Declaração}\hfill]
\begin{ttfamily}
public property HelpCtx{\_}StrModule: AnsiString read GetHelpCtx{\_}StrModule write {\_}HelpCtx{\_}StrModule;\end{ttfamily}


\end{flushleft}
\end{list}
\paragraph*{HelpCtx{\_}StrCommand}\hspace*{\fill}

\begin{list}{}{
\settowidth{\tmplength}{\textbf{Declaração}}
\setlength{\itemindent}{0cm}
\setlength{\listparindent}{0cm}
\setlength{\leftmargin}{\evensidemargin}
\addtolength{\leftmargin}{\tmplength}
\settowidth{\labelsep}{X}
\addtolength{\leftmargin}{\labelsep}
\setlength{\labelwidth}{\tmplength}
}
\begin{flushleft}
\item[\textbf{Declaração}\hfill]
\begin{ttfamily}
public property HelpCtx{\_}StrCommand: AnsiString read GetHelpCtx{\_}StrCommand write {\_}HelpCtx{\_}StrCommand;\end{ttfamily}


\end{flushleft}
\end{list}
\paragraph*{HelpCtx{\_}StrCommand{\_}Topic}\hspace*{\fill}

\begin{list}{}{
\settowidth{\tmplength}{\textbf{Declaração}}
\setlength{\itemindent}{0cm}
\setlength{\listparindent}{0cm}
\setlength{\leftmargin}{\evensidemargin}
\addtolength{\leftmargin}{\tmplength}
\settowidth{\labelsep}{X}
\addtolength{\leftmargin}{\labelsep}
\setlength{\labelwidth}{\tmplength}
}
\begin{flushleft}
\item[\textbf{Declaração}\hfill]
\begin{ttfamily}
public property HelpCtx{\_}StrCommand{\_}Topic: AnsiString read GetHelpCtx{\_}StrCommand{\_}Topic write {\_}HelpCtx{\_}StrCommand{\_}Topic;\end{ttfamily}


\end{flushleft}
\end{list}
\paragraph*{HelpCtx{\_}StrCurrentModule}\hspace*{\fill}

\begin{list}{}{
\settowidth{\tmplength}{\textbf{Declaração}}
\setlength{\itemindent}{0cm}
\setlength{\listparindent}{0cm}
\setlength{\leftmargin}{\evensidemargin}
\addtolength{\leftmargin}{\tmplength}
\settowidth{\labelsep}{X}
\addtolength{\leftmargin}{\labelsep}
\setlength{\labelwidth}{\tmplength}
}
\begin{flushleft}
\item[\textbf{Declaração}\hfill]
\begin{ttfamily}
public property HelpCtx{\_}StrCurrentModule: AnsiString read GetHelpCtx{\_}StrCurrentModule write {\_}HelpCtx{\_}StrCurrentModule;\end{ttfamily}


\end{flushleft}
\end{list}
\paragraph*{HelpCtx{\_}StrCurrentCommand}\hspace*{\fill}

\begin{list}{}{
\settowidth{\tmplength}{\textbf{Declaração}}
\setlength{\itemindent}{0cm}
\setlength{\listparindent}{0cm}
\setlength{\leftmargin}{\evensidemargin}
\addtolength{\leftmargin}{\tmplength}
\settowidth{\labelsep}{X}
\addtolength{\leftmargin}{\labelsep}
\setlength{\labelwidth}{\tmplength}
}
\begin{flushleft}
\item[\textbf{Declaração}\hfill]
\begin{ttfamily}
public property HelpCtx{\_}StrCurrentCommand: AnsiString read GetHelpCtx{\_}StrCurrentCommand write {\_}HelpCtx{\_}StrCurrentCommand;\end{ttfamily}


\end{flushleft}
\end{list}
\paragraph*{HelpCtx{\_}StrCurrentCommand{\_}Opcao}\hspace*{\fill}

\begin{list}{}{
\settowidth{\tmplength}{\textbf{Declaração}}
\setlength{\itemindent}{0cm}
\setlength{\listparindent}{0cm}
\setlength{\leftmargin}{\evensidemargin}
\addtolength{\leftmargin}{\tmplength}
\settowidth{\labelsep}{X}
\addtolength{\leftmargin}{\labelsep}
\setlength{\labelwidth}{\tmplength}
}
\begin{flushleft}
\item[\textbf{Declaração}\hfill]
\begin{ttfamily}
public property HelpCtx{\_}StrCurrentCommand{\_}Opcao: AnsiString read GetHelpCtx{\_}StrCurrentCommand{\_}Opcao write {\_}HelpCtx{\_}StrCurrentCommand{\_}Opcao;\end{ttfamily}


\end{flushleft}
\end{list}
\paragraph*{HelpCtx{\_}StrCurrentCommand{\_}Topic}\hspace*{\fill}

\begin{list}{}{
\settowidth{\tmplength}{\textbf{Declaração}}
\setlength{\itemindent}{0cm}
\setlength{\listparindent}{0cm}
\setlength{\leftmargin}{\evensidemargin}
\addtolength{\leftmargin}{\tmplength}
\settowidth{\labelsep}{X}
\addtolength{\leftmargin}{\labelsep}
\setlength{\labelwidth}{\tmplength}
}
\begin{flushleft}
\item[\textbf{Declaração}\hfill]
\begin{ttfamily}
public property HelpCtx{\_}StrCurrentCommand{\_}Topic: AnsiString read GetHelpCtx{\_}StrCurrentCommand{\_}Topic write {\_}HelpCtx{\_}StrCurrentCommand{\_}Topic;\end{ttfamily}


\end{flushleft}
\end{list}
\paragraph*{HelpCtx{\_}StrCurrentCommand{\_}Topic{\_}Content{\_}Run}\hspace*{\fill}

\begin{list}{}{
\settowidth{\tmplength}{\textbf{Declaração}}
\setlength{\itemindent}{0cm}
\setlength{\listparindent}{0cm}
\setlength{\leftmargin}{\evensidemargin}
\addtolength{\leftmargin}{\tmplength}
\settowidth{\labelsep}{X}
\addtolength{\leftmargin}{\labelsep}
\setlength{\labelwidth}{\tmplength}
}
\begin{flushleft}
\item[\textbf{Declaração}\hfill]
\begin{ttfamily}
public property HelpCtx{\_}StrCurrentCommand{\_}Topic{\_}Content{\_}Run: TEnum{\_}HelpCtx{\_}StrCurrentCommand{\_}Topic{\_}Content{\_}run read GetHelpCtx{\_}StrCurrentCommand{\_}Topic{\_}Content{\_}Run write SetHelpCtx{\_}StrCurrentCommand{\_}Topic{\_}Content{\_}Run;\end{ttfamily}


\end{flushleft}
\end{list}
\paragraph*{Ok{\_}HelpCtx{\_}StrCurrentCommand{\_}Topic{\_}Content{\_}run{\_}Parameter{\_}File}\hspace*{\fill}

\begin{list}{}{
\settowidth{\tmplength}{\textbf{Declaração}}
\setlength{\itemindent}{0cm}
\setlength{\listparindent}{0cm}
\setlength{\leftmargin}{\evensidemargin}
\addtolength{\leftmargin}{\tmplength}
\settowidth{\labelsep}{X}
\addtolength{\leftmargin}{\labelsep}
\setlength{\labelwidth}{\tmplength}
}
\begin{flushleft}
\item[\textbf{Declaração}\hfill]
\begin{ttfamily}
public property Ok{\_}HelpCtx{\_}StrCurrentCommand{\_}Topic{\_}Content{\_}run{\_}Parameter{\_}File: Boolean read {\_}Ok{\_}HelpCtx{\_}StrCurrentCommand{\_}Topic{\_}Content{\_}run{\_}Parameter{\_}File write Set{\_}Ok{\_}HelpCtx{\_}StrCurrentCommand{\_}Topic{\_}Content{\_}run{\_}Parameter{\_}File;\end{ttfamily}


\end{flushleft}
\end{list}
\paragraph*{HelpCtx{\_}StrCurrentCommand{\_}Topic{\_}Content}\hspace*{\fill}

\begin{list}{}{
\settowidth{\tmplength}{\textbf{Declaração}}
\setlength{\itemindent}{0cm}
\setlength{\listparindent}{0cm}
\setlength{\leftmargin}{\evensidemargin}
\addtolength{\leftmargin}{\tmplength}
\settowidth{\labelsep}{X}
\addtolength{\leftmargin}{\labelsep}
\setlength{\labelwidth}{\tmplength}
}
\begin{flushleft}
\item[\textbf{Declaração}\hfill]
\begin{ttfamily}
public property HelpCtx{\_}StrCurrentCommand{\_}Topic{\_}Content: AnsiString read GetHelpCtx{\_}StrCurrentCommand{\_}Topic{\_}Content write SetHelpCtx{\_}StrCurrentCommand{\_}Topic{\_}Content;\end{ttfamily}


\end{flushleft}
\end{list}
\paragraph*{HelpCtx{\_}Hint}\hspace*{\fill}

\begin{list}{}{
\settowidth{\tmplength}{\textbf{Declaração}}
\setlength{\itemindent}{0cm}
\setlength{\listparindent}{0cm}
\setlength{\leftmargin}{\evensidemargin}
\addtolength{\leftmargin}{\tmplength}
\settowidth{\labelsep}{X}
\addtolength{\leftmargin}{\labelsep}
\setlength{\labelwidth}{\tmplength}
}
\begin{flushleft}
\item[\textbf{Declaração}\hfill]
\begin{ttfamily}
public property HelpCtx{\_}Hint : AnsiString read GetHelpCtx{\_}Hint write {\_}HelpCtx{\_}Hint;\end{ttfamily}


\end{flushleft}
\end{list}
\paragraph*{HelpCtx{\_}Historico}\hspace*{\fill}

\begin{list}{}{
\settowidth{\tmplength}{\textbf{Declaração}}
\setlength{\itemindent}{0cm}
\setlength{\listparindent}{0cm}
\setlength{\leftmargin}{\evensidemargin}
\addtolength{\leftmargin}{\tmplength}
\settowidth{\labelsep}{X}
\addtolength{\leftmargin}{\labelsep}
\setlength{\labelwidth}{\tmplength}
}
\begin{flushleft}
\item[\textbf{Declaração}\hfill]
\begin{ttfamily}
public property HelpCtx{\_}Historico : AnsiString read GetHelpCtx{\_}Historico write {\_}HelpCtx{\_}Historico;\end{ttfamily}


\end{flushleft}
\end{list}
\paragraph*{HelpCtx{\_}Porque}\hspace*{\fill}

\begin{list}{}{
\settowidth{\tmplength}{\textbf{Declaração}}
\setlength{\itemindent}{0cm}
\setlength{\listparindent}{0cm}
\setlength{\leftmargin}{\evensidemargin}
\addtolength{\leftmargin}{\tmplength}
\settowidth{\labelsep}{X}
\addtolength{\leftmargin}{\labelsep}
\setlength{\labelwidth}{\tmplength}
}
\begin{flushleft}
\item[\textbf{Declaração}\hfill]
\begin{ttfamily}
public property HelpCtx{\_}Porque : AnsiString read GetHelpCtx{\_}Porque write {\_}HelpCtx{\_}Porque;\end{ttfamily}


\end{flushleft}
\end{list}
\paragraph*{HelpCtx{\_}Onde}\hspace*{\fill}

\begin{list}{}{
\settowidth{\tmplength}{\textbf{Declaração}}
\setlength{\itemindent}{0cm}
\setlength{\listparindent}{0cm}
\setlength{\leftmargin}{\evensidemargin}
\addtolength{\leftmargin}{\tmplength}
\settowidth{\labelsep}{X}
\addtolength{\leftmargin}{\labelsep}
\setlength{\labelwidth}{\tmplength}
}
\begin{flushleft}
\item[\textbf{Declaração}\hfill]
\begin{ttfamily}
public property HelpCtx{\_}Onde : AnsiString read GetHelpCtx{\_}Onde write {\_}HelpCtx{\_}Onde;\end{ttfamily}


\end{flushleft}
\end{list}
\paragraph*{HelpCtx{\_}Como}\hspace*{\fill}

\begin{list}{}{
\settowidth{\tmplength}{\textbf{Declaração}}
\setlength{\itemindent}{0cm}
\setlength{\listparindent}{0cm}
\setlength{\leftmargin}{\evensidemargin}
\addtolength{\leftmargin}{\tmplength}
\settowidth{\labelsep}{X}
\addtolength{\leftmargin}{\labelsep}
\setlength{\labelwidth}{\tmplength}
}
\begin{flushleft}
\item[\textbf{Declaração}\hfill]
\begin{ttfamily}
public property HelpCtx{\_}Como : AnsiString read GetHelpCtx{\_}Como write {\_}HelpCtx{\_}Como;\end{ttfamily}


\end{flushleft}
\end{list}
\paragraph*{HelpCtx{\_}Quais}\hspace*{\fill}

\begin{list}{}{
\settowidth{\tmplength}{\textbf{Declaração}}
\setlength{\itemindent}{0cm}
\setlength{\listparindent}{0cm}
\setlength{\leftmargin}{\evensidemargin}
\addtolength{\leftmargin}{\tmplength}
\settowidth{\labelsep}{X}
\addtolength{\leftmargin}{\labelsep}
\setlength{\labelwidth}{\tmplength}
}
\begin{flushleft}
\item[\textbf{Declaração}\hfill]
\begin{ttfamily}
public property HelpCtx{\_}Quais : AnsiString read GetHelpCtx{\_}Quais write {\_}HelpCtx{\_}Quais;\end{ttfamily}


\end{flushleft}
\end{list}
\paragraph*{InstanceClass}\hspace*{\fill}

\begin{list}{}{
\settowidth{\tmplength}{\textbf{Declaração}}
\setlength{\itemindent}{0cm}
\setlength{\listparindent}{0cm}
\setlength{\leftmargin}{\evensidemargin}
\addtolength{\leftmargin}{\tmplength}
\settowidth{\labelsep}{X}
\addtolength{\leftmargin}{\labelsep}
\setlength{\labelwidth}{\tmplength}
}
\begin{flushleft}
\item[\textbf{Declaração}\hfill]
\begin{ttfamily}
published property InstanceClass : TComponentClass Read GetInstanceClass;\end{ttfamily}


\end{flushleft}
\end{list}
\paragraph*{OkCreate}\hspace*{\fill}

\begin{list}{}{
\settowidth{\tmplength}{\textbf{Declaração}}
\setlength{\itemindent}{0cm}
\setlength{\listparindent}{0cm}
\setlength{\leftmargin}{\evensidemargin}
\addtolength{\leftmargin}{\tmplength}
\settowidth{\labelsep}{X}
\addtolength{\leftmargin}{\labelsep}
\setlength{\labelwidth}{\tmplength}
}
\begin{flushleft}
\item[\textbf{Declaração}\hfill]
\begin{ttfamily}
public property OkCreate : Boolean Read {\_}okCreate Default false;\end{ttfamily}


\end{flushleft}
\end{list}
\paragraph*{Owner{\_}NSComponent}\hspace*{\fill}

\begin{list}{}{
\settowidth{\tmplength}{\textbf{Declaração}}
\setlength{\itemindent}{0cm}
\setlength{\listparindent}{0cm}
\setlength{\leftmargin}{\evensidemargin}
\addtolength{\leftmargin}{\tmplength}
\settowidth{\labelsep}{X}
\addtolength{\leftmargin}{\labelsep}
\setlength{\labelwidth}{\tmplength}
}
\begin{flushleft}
\item[\textbf{Declaração}\hfill]
\begin{ttfamily}
public property Owner{\_}NSComponent : TNSComponent read {\_}Owner{\_}NSComponent write SetOwner{\_}NSComponent;\end{ttfamily}


\end{flushleft}
\end{list}
\paragraph*{RecordSelected}\hspace*{\fill}

\begin{list}{}{
\settowidth{\tmplength}{\textbf{Declaração}}
\setlength{\itemindent}{0cm}
\setlength{\listparindent}{0cm}
\setlength{\leftmargin}{\evensidemargin}
\addtolength{\leftmargin}{\tmplength}
\settowidth{\labelsep}{X}
\addtolength{\leftmargin}{\labelsep}
\setlength{\labelwidth}{\tmplength}
}
\begin{flushleft}
\item[\textbf{Declaração}\hfill]
\begin{ttfamily}
public property RecordSelected  : boolean read GetRecordSelected  Write SetRecordSelected;\end{ttfamily}


\end{flushleft}
\end{list}
\paragraph*{FieldSelected}\hspace*{\fill}

\begin{list}{}{
\settowidth{\tmplength}{\textbf{Declaração}}
\setlength{\itemindent}{0cm}
\setlength{\listparindent}{0cm}
\setlength{\leftmargin}{\evensidemargin}
\addtolength{\leftmargin}{\tmplength}
\settowidth{\labelsep}{X}
\addtolength{\leftmargin}{\labelsep}
\setlength{\labelwidth}{\tmplength}
}
\begin{flushleft}
\item[\textbf{Declaração}\hfill]
\begin{ttfamily}
public property FieldSelected  : boolean read GetFieldSelected Write SetFieldSelected default false;\end{ttfamily}


\end{flushleft}
\end{list}
\paragraph*{HTMLContent}\hspace*{\fill}

\begin{list}{}{
\settowidth{\tmplength}{\textbf{Declaração}}
\setlength{\itemindent}{0cm}
\setlength{\listparindent}{0cm}
\setlength{\leftmargin}{\evensidemargin}
\addtolength{\leftmargin}{\tmplength}
\settowidth{\labelsep}{X}
\addtolength{\leftmargin}{\labelsep}
\setlength{\labelwidth}{\tmplength}
}
\begin{flushleft}
\item[\textbf{Declaração}\hfill]
\begin{ttfamily}
public property HTMLContent : AnsiString Read GetHTMLContent;\end{ttfamily}


\end{flushleft}
\end{list}
\paragraph*{OnHTMLTag}\hspace*{\fill}

\begin{list}{}{
\settowidth{\tmplength}{\textbf{Declaração}}
\setlength{\itemindent}{0cm}
\setlength{\listparindent}{0cm}
\setlength{\leftmargin}{\evensidemargin}
\addtolength{\leftmargin}{\tmplength}
\settowidth{\labelsep}{X}
\addtolength{\leftmargin}{\labelsep}
\setlength{\labelwidth}{\tmplength}
}
\begin{flushleft}
\item[\textbf{Declaração}\hfill]
\begin{ttfamily}
public property OnHTMLTag  : Boolean Read {\_}OnHTMLTag Write SetOnHTMLTag;\end{ttfamily}


\end{flushleft}
\end{list}
\paragraph*{HTMLDoc}\hspace*{\fill}

\begin{list}{}{
\settowidth{\tmplength}{\textbf{Declaração}}
\setlength{\itemindent}{0cm}
\setlength{\listparindent}{0cm}
\setlength{\leftmargin}{\evensidemargin}
\addtolength{\leftmargin}{\tmplength}
\settowidth{\labelsep}{X}
\addtolength{\leftmargin}{\labelsep}
\setlength{\labelwidth}{\tmplength}
}
\begin{flushleft}
\item[\textbf{Declaração}\hfill]
\begin{ttfamily}
public property HTMLDoc  : tStrings Read GetHTMLDoc Write SetHTMLDoc;\end{ttfamily}


\end{flushleft}
\end{list}
\paragraph*{HTMLFile}\hspace*{\fill}

\begin{list}{}{
\settowidth{\tmplength}{\textbf{Declaração}}
\setlength{\itemindent}{0cm}
\setlength{\listparindent}{0cm}
\setlength{\leftmargin}{\evensidemargin}
\addtolength{\leftmargin}{\tmplength}
\settowidth{\labelsep}{X}
\addtolength{\leftmargin}{\labelsep}
\setlength{\labelwidth}{\tmplength}
}
\begin{flushleft}
\item[\textbf{Declaração}\hfill]
\begin{ttfamily}
public property HTMLFile : TFileName read GetHTMLFile Write SetHTMLFile;\end{ttfamily}


\end{flushleft}
\end{list}
\paragraph*{PageProducer}\hspace*{\fill}

\begin{list}{}{
\settowidth{\tmplength}{\textbf{Declaração}}
\setlength{\itemindent}{0cm}
\setlength{\listparindent}{0cm}
\setlength{\leftmargin}{\evensidemargin}
\addtolength{\leftmargin}{\tmplength}
\settowidth{\labelsep}{X}
\addtolength{\leftmargin}{\labelsep}
\setlength{\labelwidth}{\tmplength}
}
\begin{flushleft}
\item[\textbf{Declaração}\hfill]
\begin{ttfamily}
public property PageProducer : TPageProducer read {\_}PageProducer write {\_}PageProducer;\end{ttfamily}


\end{flushleft}
\end{list}
\paragraph*{RecordAltered}\hspace*{\fill}

\begin{list}{}{
\settowidth{\tmplength}{\textbf{Declaração}}
\setlength{\itemindent}{0cm}
\setlength{\listparindent}{0cm}
\setlength{\leftmargin}{\evensidemargin}
\addtolength{\leftmargin}{\tmplength}
\settowidth{\labelsep}{X}
\addtolength{\leftmargin}{\labelsep}
\setlength{\labelwidth}{\tmplength}
}
\begin{flushleft}
\item[\textbf{Declaração}\hfill]
\begin{ttfamily}
public property RecordAltered   : Boolean read {\_}RecordAltered write Set{\_}RecordAltered;\end{ttfamily}


\end{flushleft}
\end{list}
\paragraph*{FieldAltered}\hspace*{\fill}

\begin{list}{}{
\settowidth{\tmplength}{\textbf{Declaração}}
\setlength{\itemindent}{0cm}
\setlength{\listparindent}{0cm}
\setlength{\leftmargin}{\evensidemargin}
\addtolength{\leftmargin}{\tmplength}
\settowidth{\labelsep}{X}
\addtolength{\leftmargin}{\labelsep}
\setlength{\labelwidth}{\tmplength}
}
\begin{flushleft}
\item[\textbf{Declaração}\hfill]
\begin{ttfamily}
public property FieldAltered  : Boolean read Get{\_}FieldAltered write Set{\_}FieldAltered;\end{ttfamily}


\end{flushleft}
\end{list}
\paragraph*{KeyAltered}\hspace*{\fill}

\begin{list}{}{
\settowidth{\tmplength}{\textbf{Declaração}}
\setlength{\itemindent}{0cm}
\setlength{\listparindent}{0cm}
\setlength{\leftmargin}{\evensidemargin}
\addtolength{\leftmargin}{\tmplength}
\settowidth{\labelsep}{X}
\addtolength{\leftmargin}{\labelsep}
\setlength{\labelwidth}{\tmplength}
}
\begin{flushleft}
\item[\textbf{Declaração}\hfill]
\begin{ttfamily}
public property KeyAltered      : Boolean read Get{\_}KeyAltered write Set{\_}KeyAltered;\end{ttfamily}


\end{flushleft}
\end{list}
\paragraph*{Appending}\hspace*{\fill}

\begin{list}{}{
\settowidth{\tmplength}{\textbf{Declaração}}
\setlength{\itemindent}{0cm}
\setlength{\listparindent}{0cm}
\setlength{\leftmargin}{\evensidemargin}
\addtolength{\leftmargin}{\tmplength}
\settowidth{\labelsep}{X}
\addtolength{\leftmargin}{\labelsep}
\setlength{\labelwidth}{\tmplength}
}
\begin{flushleft}
\item[\textbf{Declaração}\hfill]
\begin{ttfamily}
public property Appending : Boolean read Get{\_}Appending write Set{\_}Appending;\end{ttfamily}


\end{flushleft}
\end{list}
\paragraph*{Append}\hspace*{\fill}

\begin{list}{}{
\settowidth{\tmplength}{\textbf{Declaração}}
\setlength{\itemindent}{0cm}
\setlength{\listparindent}{0cm}
\setlength{\leftmargin}{\evensidemargin}
\addtolength{\leftmargin}{\tmplength}
\settowidth{\labelsep}{X}
\addtolength{\leftmargin}{\labelsep}
\setlength{\labelwidth}{\tmplength}
}
\begin{flushleft}
\item[\textbf{Declaração}\hfill]
\begin{ttfamily}
public property Append : Boolean Read {\_}Append write SetAppend;\end{ttfamily}


\end{flushleft}
\end{list}
\paragraph*{RecordLimit}\hspace*{\fill}

\begin{list}{}{
\settowidth{\tmplength}{\textbf{Declaração}}
\setlength{\itemindent}{0cm}
\setlength{\listparindent}{0cm}
\setlength{\leftmargin}{\evensidemargin}
\addtolength{\leftmargin}{\tmplength}
\settowidth{\labelsep}{X}
\addtolength{\leftmargin}{\labelsep}
\setlength{\labelwidth}{\tmplength}
}
\begin{flushleft}
\item[\textbf{Declaração}\hfill]
\begin{ttfamily}
public property RecordLimit : longint read Get{\_}RecordLimit;\end{ttfamily}


\end{flushleft}
\end{list}
\subsubsection*{\large{\textbf{Campos}}\normalsize\hspace{1ex}\hfill}
\paragraph*{Protected}\hspace*{\fill}

\begin{list}{}{
\settowidth{\tmplength}{\textbf{Declaração}}
\setlength{\itemindent}{0cm}
\setlength{\listparindent}{0cm}
\setlength{\leftmargin}{\evensidemargin}
\addtolength{\leftmargin}{\tmplength}
\settowidth{\labelsep}{X}
\addtolength{\leftmargin}{\labelsep}
\setlength{\labelwidth}{\tmplength}
}
\begin{flushleft}
\item[\textbf{Declaração}\hfill]
\begin{ttfamily}
public const Protected          {\_}EditViewHelpCtx{\_}Ok{\_}Create{\_}File{\_}HTML  : Boolean;\end{ttfamily}


\end{flushleft}
\end{list}
\paragraph*{State}\hspace*{\fill}

\begin{list}{}{
\settowidth{\tmplength}{\textbf{Declaração}}
\setlength{\itemindent}{0cm}
\setlength{\listparindent}{0cm}
\setlength{\leftmargin}{\evensidemargin}
\addtolength{\leftmargin}{\tmplength}
\settowidth{\labelsep}{X}
\addtolength{\leftmargin}{\labelsep}
\setlength{\labelwidth}{\tmplength}
}
\begin{flushleft}
\item[\textbf{Declaração}\hfill]
\begin{ttfamily}
public State: Int64;\end{ttfamily}


\end{flushleft}
\end{list}
\paragraph*{{\_}HTMLContent}\hspace*{\fill}

\begin{list}{}{
\settowidth{\tmplength}{\textbf{Declaração}}
\setlength{\itemindent}{0cm}
\setlength{\listparindent}{0cm}
\setlength{\leftmargin}{\evensidemargin}
\addtolength{\leftmargin}{\tmplength}
\settowidth{\labelsep}{X}
\addtolength{\leftmargin}{\labelsep}
\setlength{\labelwidth}{\tmplength}
}
\begin{flushleft}
\item[\textbf{Declaração}\hfill]
\begin{ttfamily}
public {\_}HTMLContent: AnsiString;\end{ttfamily}


\end{flushleft}
\end{list}
\subsubsection*{\large{\textbf{Métodos}}\normalsize\hspace{1ex}\hfill}
\paragraph*{QueryInterface}\hspace*{\fill}

\begin{list}{}{
\settowidth{\tmplength}{\textbf{Declaração}}
\setlength{\itemindent}{0cm}
\setlength{\listparindent}{0cm}
\setlength{\leftmargin}{\evensidemargin}
\addtolength{\leftmargin}{\tmplength}
\settowidth{\labelsep}{X}
\addtolength{\leftmargin}{\labelsep}
\setlength{\labelwidth}{\tmplength}
}
\begin{flushleft}
\item[\textbf{Declaração}\hfill]
\begin{ttfamily}
public function QueryInterface(const IID: TGUID; out Obj): Integer; stdcall;\end{ttfamily}


\end{flushleft}
\end{list}
\paragraph*{{\_}AddRef}\hspace*{\fill}

\begin{list}{}{
\settowidth{\tmplength}{\textbf{Declaração}}
\setlength{\itemindent}{0cm}
\setlength{\listparindent}{0cm}
\setlength{\leftmargin}{\evensidemargin}
\addtolength{\leftmargin}{\tmplength}
\settowidth{\labelsep}{X}
\addtolength{\leftmargin}{\labelsep}
\setlength{\labelwidth}{\tmplength}
}
\begin{flushleft}
\item[\textbf{Declaração}\hfill]
\begin{ttfamily}
public function {\_}AddRef: Integer; stdcall;\end{ttfamily}


\end{flushleft}
\end{list}
\paragraph*{{\_}Release}\hspace*{\fill}

\begin{list}{}{
\settowidth{\tmplength}{\textbf{Declaração}}
\setlength{\itemindent}{0cm}
\setlength{\listparindent}{0cm}
\setlength{\leftmargin}{\evensidemargin}
\addtolength{\leftmargin}{\tmplength}
\settowidth{\labelsep}{X}
\addtolength{\leftmargin}{\labelsep}
\setlength{\labelwidth}{\tmplength}
}
\begin{flushleft}
\item[\textbf{Declaração}\hfill]
\begin{ttfamily}
public function {\_}Release: Integer; stdcall;\end{ttfamily}


\end{flushleft}
\end{list}
\paragraph*{Owner{\_}Component}\hspace*{\fill}

\begin{list}{}{
\settowidth{\tmplength}{\textbf{Declaração}}
\setlength{\itemindent}{0cm}
\setlength{\listparindent}{0cm}
\setlength{\leftmargin}{\evensidemargin}
\addtolength{\leftmargin}{\tmplength}
\settowidth{\labelsep}{X}
\addtolength{\leftmargin}{\labelsep}
\setlength{\labelwidth}{\tmplength}
}
\begin{flushleft}
\item[\textbf{Declaração}\hfill]
\begin{ttfamily}
public Function Owner{\_}Component:TComponent;\end{ttfamily}


\end{flushleft}
\end{list}
\paragraph*{GetAlias}\hspace*{\fill}

\begin{list}{}{
\settowidth{\tmplength}{\textbf{Declaração}}
\setlength{\itemindent}{0cm}
\setlength{\listparindent}{0cm}
\setlength{\leftmargin}{\evensidemargin}
\addtolength{\leftmargin}{\tmplength}
\settowidth{\labelsep}{X}
\addtolength{\leftmargin}{\labelsep}
\setlength{\labelwidth}{\tmplength}
}
\begin{flushleft}
\item[\textbf{Declaração}\hfill]
\begin{ttfamily}
protected Function GetAlias:AnsiString; Virtual;\end{ttfamily}


\end{flushleft}
\end{list}
\paragraph*{SetAlias}\hspace*{\fill}

\begin{list}{}{
\settowidth{\tmplength}{\textbf{Declaração}}
\setlength{\itemindent}{0cm}
\setlength{\listparindent}{0cm}
\setlength{\leftmargin}{\evensidemargin}
\addtolength{\leftmargin}{\tmplength}
\settowidth{\labelsep}{X}
\addtolength{\leftmargin}{\labelsep}
\setlength{\labelwidth}{\tmplength}
}
\begin{flushleft}
\item[\textbf{Declaração}\hfill]
\begin{ttfamily}
protected Procedure SetAlias(Const aAlias:AnsiString); Virtual;\end{ttfamily}


\end{flushleft}
\end{list}
\paragraph*{SetPath}\hspace*{\fill}

\begin{list}{}{
\settowidth{\tmplength}{\textbf{Declaração}}
\setlength{\itemindent}{0cm}
\setlength{\listparindent}{0cm}
\setlength{\leftmargin}{\evensidemargin}
\addtolength{\leftmargin}{\tmplength}
\settowidth{\labelsep}{X}
\addtolength{\leftmargin}{\labelsep}
\setlength{\labelwidth}{\tmplength}
}
\begin{flushleft}
\item[\textbf{Declaração}\hfill]
\begin{ttfamily}
protected Procedure SetPath(Const aPath:AnsiString); virtual;\end{ttfamily}


\end{flushleft}
\end{list}
\paragraph*{GetID{\_}Dynamic}\hspace*{\fill}

\begin{list}{}{
\settowidth{\tmplength}{\textbf{Declaração}}
\setlength{\itemindent}{0cm}
\setlength{\listparindent}{0cm}
\setlength{\leftmargin}{\evensidemargin}
\addtolength{\leftmargin}{\tmplength}
\settowidth{\labelsep}{X}
\addtolength{\leftmargin}{\labelsep}
\setlength{\labelwidth}{\tmplength}
}
\begin{flushleft}
\item[\textbf{Declaração}\hfill]
\begin{ttfamily}
protected Function GetID{\_}Dynamic:AnsiString;\end{ttfamily}


\end{flushleft}
\end{list}
\paragraph*{GetCurrentField}\hspace*{\fill}

\begin{list}{}{
\settowidth{\tmplength}{\textbf{Declaração}}
\setlength{\itemindent}{0cm}
\setlength{\listparindent}{0cm}
\setlength{\leftmargin}{\evensidemargin}
\addtolength{\leftmargin}{\tmplength}
\settowidth{\labelsep}{X}
\addtolength{\leftmargin}{\labelsep}
\setlength{\labelwidth}{\tmplength}
}
\begin{flushleft}
\item[\textbf{Declaração}\hfill]
\begin{ttfamily}
public Function GetCurrentField:Pointer; overload; Virtual;\end{ttfamily}


\end{flushleft}
\end{list}
\paragraph*{GetCurrentField}\hspace*{\fill}

\begin{list}{}{
\settowidth{\tmplength}{\textbf{Declaração}}
\setlength{\itemindent}{0cm}
\setlength{\listparindent}{0cm}
\setlength{\leftmargin}{\evensidemargin}
\addtolength{\leftmargin}{\tmplength}
\settowidth{\labelsep}{X}
\addtolength{\leftmargin}{\labelsep}
\setlength{\labelwidth}{\tmplength}
}
\begin{flushleft}
\item[\textbf{Declaração}\hfill]
\begin{ttfamily}
public Function GetCurrentField(FieldNum:Longint):Pointer; overload; Virtual;\end{ttfamily}


\end{flushleft}
\end{list}
\paragraph*{TabIndex}\hspace*{\fill}

\begin{list}{}{
\settowidth{\tmplength}{\textbf{Declaração}}
\setlength{\itemindent}{0cm}
\setlength{\listparindent}{0cm}
\setlength{\leftmargin}{\evensidemargin}
\addtolength{\leftmargin}{\tmplength}
\settowidth{\labelsep}{X}
\addtolength{\leftmargin}{\labelsep}
\setlength{\labelwidth}{\tmplength}
}
\begin{flushleft}
\item[\textbf{Declaração}\hfill]
\begin{ttfamily}
protected Function TabIndex:Longint; Virtual;\end{ttfamily}


\end{flushleft}
\end{list}
\paragraph*{GetAcao}\hspace*{\fill}

\begin{list}{}{
\settowidth{\tmplength}{\textbf{Declaração}}
\setlength{\itemindent}{0cm}
\setlength{\listparindent}{0cm}
\setlength{\leftmargin}{\evensidemargin}
\addtolength{\leftmargin}{\tmplength}
\settowidth{\labelsep}{X}
\addtolength{\leftmargin}{\labelsep}
\setlength{\labelwidth}{\tmplength}
}
\begin{flushleft}
\item[\textbf{Declaração}\hfill]
\begin{ttfamily}
protected Function GetAcao():AnsiString; Virtual;\end{ttfamily}


\end{flushleft}
\end{list}
\paragraph*{DoOnHTMLTag{\_}tgLink}\hspace*{\fill}

\begin{list}{}{
\settowidth{\tmplength}{\textbf{Declaração}}
\setlength{\itemindent}{0cm}
\setlength{\listparindent}{0cm}
\setlength{\leftmargin}{\evensidemargin}
\addtolength{\leftmargin}{\tmplength}
\settowidth{\labelsep}{X}
\addtolength{\leftmargin}{\labelsep}
\setlength{\labelwidth}{\tmplength}
}
\begin{flushleft}
\item[\textbf{Declaração}\hfill]
\begin{ttfamily}
protected procedure DoOnHTMLTag{\_}tgLink(Sender: TObject; const TagString: String;TagParams: tStrings; var ReplaceText: String); Virtual;\end{ttfamily}


\end{flushleft}
\end{list}
\paragraph*{DoOnHTMLTag{\_}tgImage}\hspace*{\fill}

\begin{list}{}{
\settowidth{\tmplength}{\textbf{Declaração}}
\setlength{\itemindent}{0cm}
\setlength{\listparindent}{0cm}
\setlength{\leftmargin}{\evensidemargin}
\addtolength{\leftmargin}{\tmplength}
\settowidth{\labelsep}{X}
\addtolength{\leftmargin}{\labelsep}
\setlength{\labelwidth}{\tmplength}
}
\begin{flushleft}
\item[\textbf{Declaração}\hfill]
\begin{ttfamily}
protected procedure DoOnHTMLTag{\_}tgImage(Sender: TObject; const TagString: String;TagParams: tStrings; var ReplaceText: String); Virtual;\end{ttfamily}


\end{flushleft}
\end{list}
\paragraph*{DoOnHTMLTag{\_}tgTable}\hspace*{\fill}

\begin{list}{}{
\settowidth{\tmplength}{\textbf{Declaração}}
\setlength{\itemindent}{0cm}
\setlength{\listparindent}{0cm}
\setlength{\leftmargin}{\evensidemargin}
\addtolength{\leftmargin}{\tmplength}
\settowidth{\labelsep}{X}
\addtolength{\leftmargin}{\labelsep}
\setlength{\labelwidth}{\tmplength}
}
\begin{flushleft}
\item[\textbf{Declaração}\hfill]
\begin{ttfamily}
protected procedure DoOnHTMLTag{\_}tgTable(Sender: TObject; const TagString: String;TagParams: tStrings; var ReplaceText: String); Virtual;\end{ttfamily}


\end{flushleft}
\end{list}
\paragraph*{DoOnHTMLTag{\_}tgImageMap}\hspace*{\fill}

\begin{list}{}{
\settowidth{\tmplength}{\textbf{Declaração}}
\setlength{\itemindent}{0cm}
\setlength{\listparindent}{0cm}
\setlength{\leftmargin}{\evensidemargin}
\addtolength{\leftmargin}{\tmplength}
\settowidth{\labelsep}{X}
\addtolength{\leftmargin}{\labelsep}
\setlength{\labelwidth}{\tmplength}
}
\begin{flushleft}
\item[\textbf{Declaração}\hfill]
\begin{ttfamily}
protected procedure DoOnHTMLTag{\_}tgImageMap(Sender: TObject; const TagString: String;TagParams: tStrings; var ReplaceText: String); Virtual;\end{ttfamily}


\end{flushleft}
\end{list}
\paragraph*{DoOnHTMLTag{\_}tgObject}\hspace*{\fill}

\begin{list}{}{
\settowidth{\tmplength}{\textbf{Declaração}}
\setlength{\itemindent}{0cm}
\setlength{\listparindent}{0cm}
\setlength{\leftmargin}{\evensidemargin}
\addtolength{\leftmargin}{\tmplength}
\settowidth{\labelsep}{X}
\addtolength{\leftmargin}{\labelsep}
\setlength{\labelwidth}{\tmplength}
}
\begin{flushleft}
\item[\textbf{Declaração}\hfill]
\begin{ttfamily}
protected procedure DoOnHTMLTag{\_}tgObject(Sender: TObject; const TagString: String;TagParams: tStrings; var ReplaceText: String); Virtual;\end{ttfamily}


\end{flushleft}
\end{list}
\paragraph*{DoOnHTMLTag{\_}tgEmbed}\hspace*{\fill}

\begin{list}{}{
\settowidth{\tmplength}{\textbf{Declaração}}
\setlength{\itemindent}{0cm}
\setlength{\listparindent}{0cm}
\setlength{\leftmargin}{\evensidemargin}
\addtolength{\leftmargin}{\tmplength}
\settowidth{\labelsep}{X}
\addtolength{\leftmargin}{\labelsep}
\setlength{\labelwidth}{\tmplength}
}
\begin{flushleft}
\item[\textbf{Declaração}\hfill]
\begin{ttfamily}
protected procedure DoOnHTMLTag{\_}tgEmbed(Sender: TObject; const TagString: String;TagParams: tStrings; var ReplaceText: String); Virtual;\end{ttfamily}


\end{flushleft}
\end{list}
\paragraph*{DoOnHTMLTag{\_}tgCustom}\hspace*{\fill}

\begin{list}{}{
\settowidth{\tmplength}{\textbf{Declaração}}
\setlength{\itemindent}{0cm}
\setlength{\listparindent}{0cm}
\setlength{\leftmargin}{\evensidemargin}
\addtolength{\leftmargin}{\tmplength}
\settowidth{\labelsep}{X}
\addtolength{\leftmargin}{\labelsep}
\setlength{\labelwidth}{\tmplength}
}
\begin{flushleft}
\item[\textbf{Declaração}\hfill]
\begin{ttfamily}
protected procedure DoOnHTMLTag{\_}tgCustom(Sender: TObject; const TagString: String;TagParams: tStrings; var ReplaceText: String); Virtual;\end{ttfamily}


\end{flushleft}
\end{list}
\paragraph*{DoOnHTMLTag}\hspace*{\fill}

\begin{list}{}{
\settowidth{\tmplength}{\textbf{Declaração}}
\setlength{\itemindent}{0cm}
\setlength{\listparindent}{0cm}
\setlength{\leftmargin}{\evensidemargin}
\addtolength{\leftmargin}{\tmplength}
\settowidth{\labelsep}{X}
\addtolength{\leftmargin}{\labelsep}
\setlength{\labelwidth}{\tmplength}
}
\begin{flushleft}
\item[\textbf{Declaração}\hfill]
\begin{ttfamily}
public procedure DoOnHTMLTag(Sender: TObject; Tag: TTag; const TagString: String;TagParams: tStrings; var ReplaceText: String);\end{ttfamily}


\end{flushleft}
\end{list}
\paragraph*{GetHelpCtx{\_}Path}\hspace*{\fill}

\begin{list}{}{
\settowidth{\tmplength}{\textbf{Declaração}}
\setlength{\itemindent}{0cm}
\setlength{\listparindent}{0cm}
\setlength{\leftmargin}{\evensidemargin}
\addtolength{\leftmargin}{\tmplength}
\settowidth{\labelsep}{X}
\addtolength{\leftmargin}{\labelsep}
\setlength{\labelwidth}{\tmplength}
}
\begin{flushleft}
\item[\textbf{Declaração}\hfill]
\begin{ttfamily}
protected function GetHelpCtx{\_}Path: AnsiString; Virtual;\end{ttfamily}


\end{flushleft}
\end{list}
\paragraph*{GetHelpCtx{\_}Doc{\_}HTML}\hspace*{\fill}

\begin{list}{}{
\settowidth{\tmplength}{\textbf{Declaração}}
\setlength{\itemindent}{0cm}
\setlength{\listparindent}{0cm}
\setlength{\leftmargin}{\evensidemargin}
\addtolength{\leftmargin}{\tmplength}
\settowidth{\labelsep}{X}
\addtolength{\leftmargin}{\labelsep}
\setlength{\labelwidth}{\tmplength}
}
\begin{flushleft}
\item[\textbf{Declaração}\hfill]
\begin{ttfamily}
public function GetHelpCtx{\_}Doc{\_}HTML: AnsiString; Virtual;\end{ttfamily}


\end{flushleft}
\end{list}
\paragraph*{ExecViewHelpCtx{\_}F1}\hspace*{\fill}

\begin{list}{}{
\settowidth{\tmplength}{\textbf{Declaração}}
\setlength{\itemindent}{0cm}
\setlength{\listparindent}{0cm}
\setlength{\leftmargin}{\evensidemargin}
\addtolength{\leftmargin}{\tmplength}
\settowidth{\labelsep}{X}
\addtolength{\leftmargin}{\labelsep}
\setlength{\labelwidth}{\tmplength}
}
\begin{flushleft}
\item[\textbf{Declaração}\hfill]
\begin{ttfamily}
public function ExecViewHelpCtx{\_}F1:Word; Virtual;\end{ttfamily}


\end{flushleft}
\end{list}
\paragraph*{ExecViewHelpCtx{\_}Alt{\_}F1}\hspace*{\fill}

\begin{list}{}{
\settowidth{\tmplength}{\textbf{Declaração}}
\setlength{\itemindent}{0cm}
\setlength{\listparindent}{0cm}
\setlength{\leftmargin}{\evensidemargin}
\addtolength{\leftmargin}{\tmplength}
\settowidth{\labelsep}{X}
\addtolength{\leftmargin}{\labelsep}
\setlength{\labelwidth}{\tmplength}
}
\begin{flushleft}
\item[\textbf{Declaração}\hfill]
\begin{ttfamily}
public function ExecViewHelpCtx{\_}Alt{\_}F1:Word; Virtual;\end{ttfamily}


\end{flushleft}
\end{list}
\paragraph*{ExecViewHelpCtx{\_}Crtl{\_}F1}\hspace*{\fill}

\begin{list}{}{
\settowidth{\tmplength}{\textbf{Declaração}}
\setlength{\itemindent}{0cm}
\setlength{\listparindent}{0cm}
\setlength{\leftmargin}{\evensidemargin}
\addtolength{\leftmargin}{\tmplength}
\settowidth{\labelsep}{X}
\addtolength{\leftmargin}{\labelsep}
\setlength{\labelwidth}{\tmplength}
}
\begin{flushleft}
\item[\textbf{Declaração}\hfill]
\begin{ttfamily}
public function ExecViewHelpCtx{\_}Crtl{\_}F1:Word; Virtual;\end{ttfamily}


\end{flushleft}
\end{list}
\paragraph*{ExecViewHelpCtx}\hspace*{\fill}

\begin{list}{}{
\settowidth{\tmplength}{\textbf{Declaração}}
\setlength{\itemindent}{0cm}
\setlength{\listparindent}{0cm}
\setlength{\leftmargin}{\evensidemargin}
\addtolength{\leftmargin}{\tmplength}
\settowidth{\labelsep}{X}
\addtolength{\leftmargin}{\labelsep}
\setlength{\labelwidth}{\tmplength}
}
\begin{flushleft}
\item[\textbf{Declaração}\hfill]
\begin{ttfamily}
public function ExecViewHelpCtx:Word;\end{ttfamily}


\end{flushleft}
\end{list}
\paragraph*{EditViewHelpCtx}\hspace*{\fill}

\begin{list}{}{
\settowidth{\tmplength}{\textbf{Declaração}}
\setlength{\itemindent}{0cm}
\setlength{\listparindent}{0cm}
\setlength{\leftmargin}{\evensidemargin}
\addtolength{\leftmargin}{\tmplength}
\settowidth{\labelsep}{X}
\addtolength{\leftmargin}{\labelsep}
\setlength{\labelwidth}{\tmplength}
}
\begin{flushleft}
\item[\textbf{Declaração}\hfill]
\begin{ttfamily}
protected function EditViewHelpCtx:Word; Virtual;\end{ttfamily}


\end{flushleft}
\end{list}
\paragraph*{EventAvail}\hspace*{\fill}

\begin{list}{}{
\settowidth{\tmplength}{\textbf{Declaração}}
\setlength{\itemindent}{0cm}
\setlength{\listparindent}{0cm}
\setlength{\leftmargin}{\evensidemargin}
\addtolength{\leftmargin}{\tmplength}
\settowidth{\labelsep}{X}
\addtolength{\leftmargin}{\labelsep}
\setlength{\labelwidth}{\tmplength}
}
\begin{flushleft}
\item[\textbf{Declaração}\hfill]
\begin{ttfamily}
protected function EventAvail: Boolean; Virtual;\end{ttfamily}


\end{flushleft}
\end{list}
\paragraph*{GetEvent}\hspace*{\fill}

\begin{list}{}{
\settowidth{\tmplength}{\textbf{Declaração}}
\setlength{\itemindent}{0cm}
\setlength{\listparindent}{0cm}
\setlength{\leftmargin}{\evensidemargin}
\addtolength{\leftmargin}{\tmplength}
\settowidth{\labelsep}{X}
\addtolength{\leftmargin}{\labelsep}
\setlength{\labelwidth}{\tmplength}
}
\begin{flushleft}
\item[\textbf{Declaração}\hfill]
\begin{ttfamily}
protected procedure GetEvent(var Event: TEvent); Virtual;\end{ttfamily}


\end{flushleft}
\end{list}
\paragraph*{PutEvent}\hspace*{\fill}

\begin{list}{}{
\settowidth{\tmplength}{\textbf{Declaração}}
\setlength{\itemindent}{0cm}
\setlength{\listparindent}{0cm}
\setlength{\leftmargin}{\evensidemargin}
\addtolength{\leftmargin}{\tmplength}
\settowidth{\labelsep}{X}
\addtolength{\leftmargin}{\labelsep}
\setlength{\labelwidth}{\tmplength}
}
\begin{flushleft}
\item[\textbf{Declaração}\hfill]
\begin{ttfamily}
protected procedure PutEvent(var Event: TEvent); Virtual;\end{ttfamily}


\end{flushleft}
\end{list}
\paragraph*{GetOwner}\hspace*{\fill}

\begin{list}{}{
\settowidth{\tmplength}{\textbf{Declaração}}
\setlength{\itemindent}{0cm}
\setlength{\listparindent}{0cm}
\setlength{\leftmargin}{\evensidemargin}
\addtolength{\leftmargin}{\tmplength}
\settowidth{\labelsep}{X}
\addtolength{\leftmargin}{\labelsep}
\setlength{\labelwidth}{\tmplength}
}
\begin{flushleft}
\item[\textbf{Declaração}\hfill]
\begin{ttfamily}
protected function GetOwner: TPersistent; override;\end{ttfamily}


\end{flushleft}
\end{list}
\paragraph*{SetRecordAltered}\hspace*{\fill}

\begin{list}{}{
\settowidth{\tmplength}{\textbf{Declaração}}
\setlength{\itemindent}{0cm}
\setlength{\listparindent}{0cm}
\setlength{\leftmargin}{\evensidemargin}
\addtolength{\leftmargin}{\tmplength}
\settowidth{\labelsep}{X}
\addtolength{\leftmargin}{\labelsep}
\setlength{\labelwidth}{\tmplength}
}
\begin{flushleft}
\item[\textbf{Declaração}\hfill]
\begin{ttfamily}
protected Function SetRecordAltered(Const aRecordAltered: Boolean):Boolean; Virtual;\end{ttfamily}


\end{flushleft}
\end{list}
\paragraph*{ChangeMadeOnOff}\hspace*{\fill}

\begin{list}{}{
\settowidth{\tmplength}{\textbf{Declaração}}
\setlength{\itemindent}{0cm}
\setlength{\listparindent}{0cm}
\setlength{\leftmargin}{\evensidemargin}
\addtolength{\leftmargin}{\tmplength}
\settowidth{\labelsep}{X}
\addtolength{\leftmargin}{\labelsep}
\setlength{\labelwidth}{\tmplength}
}
\begin{flushleft}
\item[\textbf{Declaração}\hfill]
\begin{ttfamily}
protected procedure ChangeMadeOnOff(const aValue:Boolean); Virtual;\end{ttfamily}


\end{flushleft}
\end{list}
\paragraph*{SetRecPosition}\hspace*{\fill}

\begin{list}{}{
\settowidth{\tmplength}{\textbf{Declaração}}
\setlength{\itemindent}{0cm}
\setlength{\listparindent}{0cm}
\setlength{\leftmargin}{\evensidemargin}
\addtolength{\leftmargin}{\tmplength}
\settowidth{\labelsep}{X}
\addtolength{\leftmargin}{\labelsep}
\setlength{\labelwidth}{\tmplength}
}
\begin{flushleft}
\item[\textbf{Declaração}\hfill]
\begin{ttfamily}
protected Procedure SetRecPosition(aRecPosition : Longint ); Virtual;\end{ttfamily}


\end{flushleft}
\end{list}
\paragraph*{SetCurrentRecord}\hspace*{\fill}

\begin{list}{}{
\settowidth{\tmplength}{\textbf{Declaração}}
\setlength{\itemindent}{0cm}
\setlength{\listparindent}{0cm}
\setlength{\leftmargin}{\evensidemargin}
\addtolength{\leftmargin}{\tmplength}
\settowidth{\labelsep}{X}
\addtolength{\leftmargin}{\labelsep}
\setlength{\labelwidth}{\tmplength}
}
\begin{flushleft}
\item[\textbf{Declaração}\hfill]
\begin{ttfamily}
protected Procedure SetCurrentRecord(aCurrentRecord : Longint ); Virtual;\end{ttfamily}


\end{flushleft}
\end{list}
\paragraph*{GetProcedure{\_}GlobalStr}\hspace*{\fill}

\begin{list}{}{
\settowidth{\tmplength}{\textbf{Declaração}}
\setlength{\itemindent}{0cm}
\setlength{\listparindent}{0cm}
\setlength{\leftmargin}{\evensidemargin}
\addtolength{\leftmargin}{\tmplength}
\settowidth{\labelsep}{X}
\addtolength{\leftmargin}{\labelsep}
\setlength{\labelwidth}{\tmplength}
}
\begin{flushleft}
\item[\textbf{Declaração}\hfill]
\begin{ttfamily}
protected Function GetProcedure{\_}GlobalStr:AnsiString; virtual;\end{ttfamily}


\end{flushleft}
\end{list}
\paragraph*{Set{\_}Command}\hspace*{\fill}

\begin{list}{}{
\settowidth{\tmplength}{\textbf{Declaração}}
\setlength{\itemindent}{0cm}
\setlength{\listparindent}{0cm}
\setlength{\leftmargin}{\evensidemargin}
\addtolength{\leftmargin}{\tmplength}
\settowidth{\labelsep}{X}
\addtolength{\leftmargin}{\labelsep}
\setlength{\labelwidth}{\tmplength}
}
\begin{flushleft}
\item[\textbf{Declaração}\hfill]
\begin{ttfamily}
protected Procedure Set{\_}Command(a{\_}Command : Integer); Virtual;\end{ttfamily}


\end{flushleft}
\end{list}
\paragraph*{SetCommand}\hspace*{\fill}

\begin{list}{}{
\settowidth{\tmplength}{\textbf{Declaração}}
\setlength{\itemindent}{0cm}
\setlength{\listparindent}{0cm}
\setlength{\leftmargin}{\evensidemargin}
\addtolength{\leftmargin}{\tmplength}
\settowidth{\labelsep}{X}
\addtolength{\leftmargin}{\labelsep}
\setlength{\labelwidth}{\tmplength}
}
\begin{flushleft}
\item[\textbf{Declaração}\hfill]
\begin{ttfamily}
public Procedure SetCommand(aModule:Byte;aCommand : Integer;AStrModule,aStrCommand:AnsiString);\end{ttfamily}


\end{flushleft}
\end{list}
\paragraph*{GetModule}\hspace*{\fill}

\begin{list}{}{
\settowidth{\tmplength}{\textbf{Declaração}}
\setlength{\itemindent}{0cm}
\setlength{\listparindent}{0cm}
\setlength{\leftmargin}{\evensidemargin}
\addtolength{\leftmargin}{\tmplength}
\settowidth{\labelsep}{X}
\addtolength{\leftmargin}{\labelsep}
\setlength{\labelwidth}{\tmplength}
}
\begin{flushleft}
\item[\textbf{Declaração}\hfill]
\begin{ttfamily}
protected Function GetModule: Byte; Virtual;\end{ttfamily}


\end{flushleft}
\end{list}
\paragraph*{SetModule}\hspace*{\fill}

\begin{list}{}{
\settowidth{\tmplength}{\textbf{Declaração}}
\setlength{\itemindent}{0cm}
\setlength{\listparindent}{0cm}
\setlength{\leftmargin}{\evensidemargin}
\addtolength{\leftmargin}{\tmplength}
\settowidth{\labelsep}{X}
\addtolength{\leftmargin}{\labelsep}
\setlength{\labelwidth}{\tmplength}
}
\begin{flushleft}
\item[\textbf{Declaração}\hfill]
\begin{ttfamily}
protected Procedure SetModule(aModule : Byte ); Virtual;\end{ttfamily}


\end{flushleft}
\end{list}
\paragraph*{GetHelpCtx{\_}StrModule}\hspace*{\fill}

\begin{list}{}{
\settowidth{\tmplength}{\textbf{Declaração}}
\setlength{\itemindent}{0cm}
\setlength{\listparindent}{0cm}
\setlength{\leftmargin}{\evensidemargin}
\addtolength{\leftmargin}{\tmplength}
\settowidth{\labelsep}{X}
\addtolength{\leftmargin}{\labelsep}
\setlength{\labelwidth}{\tmplength}
}
\begin{flushleft}
\item[\textbf{Declaração}\hfill]
\begin{ttfamily}
protected Function GetHelpCtx{\_}StrModule:AnsiString; virtual;\end{ttfamily}


\end{flushleft}
\end{list}
\paragraph*{GetHelpCtx{\_}StrCommand}\hspace*{\fill}

\begin{list}{}{
\settowidth{\tmplength}{\textbf{Declaração}}
\setlength{\itemindent}{0cm}
\setlength{\listparindent}{0cm}
\setlength{\leftmargin}{\evensidemargin}
\addtolength{\leftmargin}{\tmplength}
\settowidth{\labelsep}{X}
\addtolength{\leftmargin}{\labelsep}
\setlength{\labelwidth}{\tmplength}
}
\begin{flushleft}
\item[\textbf{Declaração}\hfill]
\begin{ttfamily}
protected Function GetHelpCtx{\_}StrCommand: AnsiString; Virtual;\end{ttfamily}


\end{flushleft}
\end{list}
\paragraph*{GetHelpCtx{\_}StrCommand{\_}Topic}\hspace*{\fill}

\begin{list}{}{
\settowidth{\tmplength}{\textbf{Declaração}}
\setlength{\itemindent}{0cm}
\setlength{\listparindent}{0cm}
\setlength{\leftmargin}{\evensidemargin}
\addtolength{\leftmargin}{\tmplength}
\settowidth{\labelsep}{X}
\addtolength{\leftmargin}{\labelsep}
\setlength{\labelwidth}{\tmplength}
}
\begin{flushleft}
\item[\textbf{Declaração}\hfill]
\begin{ttfamily}
protected Function GetHelpCtx{\_}StrCommand{\_}Topic: AnsiString; virtual;\end{ttfamily}


\end{flushleft}
\end{list}
\paragraph*{GetHelpCtx{\_}StrCurrentModule}\hspace*{\fill}

\begin{list}{}{
\settowidth{\tmplength}{\textbf{Declaração}}
\setlength{\itemindent}{0cm}
\setlength{\listparindent}{0cm}
\setlength{\leftmargin}{\evensidemargin}
\addtolength{\leftmargin}{\tmplength}
\settowidth{\labelsep}{X}
\addtolength{\leftmargin}{\labelsep}
\setlength{\labelwidth}{\tmplength}
}
\begin{flushleft}
\item[\textbf{Declaração}\hfill]
\begin{ttfamily}
protected Function GetHelpCtx{\_}StrCurrentModule:AnsiString; virtual;\end{ttfamily}


\end{flushleft}
\end{list}
\paragraph*{GetHelpCtx{\_}StrCurrentCommand}\hspace*{\fill}

\begin{list}{}{
\settowidth{\tmplength}{\textbf{Declaração}}
\setlength{\itemindent}{0cm}
\setlength{\listparindent}{0cm}
\setlength{\leftmargin}{\evensidemargin}
\addtolength{\leftmargin}{\tmplength}
\settowidth{\labelsep}{X}
\addtolength{\leftmargin}{\labelsep}
\setlength{\labelwidth}{\tmplength}
}
\begin{flushleft}
\item[\textbf{Declaração}\hfill]
\begin{ttfamily}
protected Function GetHelpCtx{\_}StrCurrentCommand: AnsiString; Virtual;\end{ttfamily}


\end{flushleft}
\end{list}
\paragraph*{GetHelpCtx{\_}StrCurrentCommand{\_}Opcao}\hspace*{\fill}

\begin{list}{}{
\settowidth{\tmplength}{\textbf{Declaração}}
\setlength{\itemindent}{0cm}
\setlength{\listparindent}{0cm}
\setlength{\leftmargin}{\evensidemargin}
\addtolength{\leftmargin}{\tmplength}
\settowidth{\labelsep}{X}
\addtolength{\leftmargin}{\labelsep}
\setlength{\labelwidth}{\tmplength}
}
\begin{flushleft}
\item[\textbf{Declaração}\hfill]
\begin{ttfamily}
protected Function GetHelpCtx{\_}StrCurrentCommand{\_}Opcao: AnsiString; virtual;\end{ttfamily}


\end{flushleft}
\end{list}
\paragraph*{GetHelpCtx{\_}StrCurrentCommand{\_}Topic}\hspace*{\fill}

\begin{list}{}{
\settowidth{\tmplength}{\textbf{Declaração}}
\setlength{\itemindent}{0cm}
\setlength{\listparindent}{0cm}
\setlength{\leftmargin}{\evensidemargin}
\addtolength{\leftmargin}{\tmplength}
\settowidth{\labelsep}{X}
\addtolength{\leftmargin}{\labelsep}
\setlength{\labelwidth}{\tmplength}
}
\begin{flushleft}
\item[\textbf{Declaração}\hfill]
\begin{ttfamily}
protected Function GetHelpCtx{\_}StrCurrentCommand{\_}Topic: AnsiString; virtual;\end{ttfamily}


\end{flushleft}
\end{list}
\paragraph*{GetHelpCtx{\_}StrCurrentCommand{\_}Topic{\_}Content{\_}Run}\hspace*{\fill}

\begin{list}{}{
\settowidth{\tmplength}{\textbf{Declaração}}
\setlength{\itemindent}{0cm}
\setlength{\listparindent}{0cm}
\setlength{\leftmargin}{\evensidemargin}
\addtolength{\leftmargin}{\tmplength}
\settowidth{\labelsep}{X}
\addtolength{\leftmargin}{\labelsep}
\setlength{\labelwidth}{\tmplength}
}
\begin{flushleft}
\item[\textbf{Declaração}\hfill]
\begin{ttfamily}
protected Function GetHelpCtx{\_}StrCurrentCommand{\_}Topic{\_}Content{\_}Run: TEnum{\_}HelpCtx{\_}StrCurrentCommand{\_}Topic{\_}Content{\_}run; virtual;\end{ttfamily}


\end{flushleft}
\end{list}
\paragraph*{SetHelpCtx{\_}StrCurrentCommand{\_}Topic{\_}Content{\_}Run}\hspace*{\fill}

\begin{list}{}{
\settowidth{\tmplength}{\textbf{Declaração}}
\setlength{\itemindent}{0cm}
\setlength{\listparindent}{0cm}
\setlength{\leftmargin}{\evensidemargin}
\addtolength{\leftmargin}{\tmplength}
\settowidth{\labelsep}{X}
\addtolength{\leftmargin}{\labelsep}
\setlength{\labelwidth}{\tmplength}
}
\begin{flushleft}
\item[\textbf{Declaração}\hfill]
\begin{ttfamily}
protected Procedure SetHelpCtx{\_}StrCurrentCommand{\_}Topic{\_}Content{\_}Run(wHelpCtx{\_}StrCurrentCommand{\_}Topic{\_}Content{\_}Run: TEnum{\_}HelpCtx{\_}StrCurrentCommand{\_}Topic{\_}Content{\_}run); virtual;\end{ttfamily}


\end{flushleft}
\end{list}
\paragraph*{Get{\_}Ok{\_}HelpCtx{\_}StrCurrentCommand{\_}Topic{\_}Content{\_}run{\_}Parameter{\_}File}\hspace*{\fill}

\begin{list}{}{
\settowidth{\tmplength}{\textbf{Declaração}}
\setlength{\itemindent}{0cm}
\setlength{\listparindent}{0cm}
\setlength{\leftmargin}{\evensidemargin}
\addtolength{\leftmargin}{\tmplength}
\settowidth{\labelsep}{X}
\addtolength{\leftmargin}{\labelsep}
\setlength{\labelwidth}{\tmplength}
}
\begin{flushleft}
\item[\textbf{Declaração}\hfill]
\begin{ttfamily}
protected Function Get{\_}Ok{\_}HelpCtx{\_}StrCurrentCommand{\_}Topic{\_}Content{\_}run{\_}Parameter{\_}File:Boolean; Virtual;\end{ttfamily}


\end{flushleft}
\end{list}
\paragraph*{Set{\_}Ok{\_}HelpCtx{\_}StrCurrentCommand{\_}Topic{\_}Content{\_}run{\_}Parameter{\_}File}\hspace*{\fill}

\begin{list}{}{
\settowidth{\tmplength}{\textbf{Declaração}}
\setlength{\itemindent}{0cm}
\setlength{\listparindent}{0cm}
\setlength{\leftmargin}{\evensidemargin}
\addtolength{\leftmargin}{\tmplength}
\settowidth{\labelsep}{X}
\addtolength{\leftmargin}{\labelsep}
\setlength{\labelwidth}{\tmplength}
}
\begin{flushleft}
\item[\textbf{Declaração}\hfill]
\begin{ttfamily}
protected procedure Set{\_}Ok{\_}HelpCtx{\_}StrCurrentCommand{\_}Topic{\_}Content{\_}run{\_}Parameter{\_}File(a{\_}Ok{\_}HelpCtx{\_}StrCurrentCommand{\_}Topic{\_}Content{\_}run{\_}Parameter{\_}File: Boolean); Virtual;\end{ttfamily}


\end{flushleft}
\end{list}
\paragraph*{GetHelpCtx{\_}StrCurrentCommand{\_}Topic{\_}Content}\hspace*{\fill}

\begin{list}{}{
\settowidth{\tmplength}{\textbf{Declaração}}
\setlength{\itemindent}{0cm}
\setlength{\listparindent}{0cm}
\setlength{\leftmargin}{\evensidemargin}
\addtolength{\leftmargin}{\tmplength}
\settowidth{\labelsep}{X}
\addtolength{\leftmargin}{\labelsep}
\setlength{\labelwidth}{\tmplength}
}
\begin{flushleft}
\item[\textbf{Declaração}\hfill]
\begin{ttfamily}
protected Function GetHelpCtx{\_}StrCurrentCommand{\_}Topic{\_}Content: AnsiString; virtual;\end{ttfamily}


\end{flushleft}
\end{list}
\paragraph*{SetHelpCtx{\_}StrCurrentCommand{\_}Topic{\_}Content}\hspace*{\fill}

\begin{list}{}{
\settowidth{\tmplength}{\textbf{Declaração}}
\setlength{\itemindent}{0cm}
\setlength{\listparindent}{0cm}
\setlength{\leftmargin}{\evensidemargin}
\addtolength{\leftmargin}{\tmplength}
\settowidth{\labelsep}{X}
\addtolength{\leftmargin}{\labelsep}
\setlength{\labelwidth}{\tmplength}
}
\begin{flushleft}
\item[\textbf{Declaração}\hfill]
\begin{ttfamily}
protected Procedure SetHelpCtx{\_}StrCurrentCommand{\_}Topic{\_}Content(wHelpCtx{\_}StrCurrentCommand{\_}Topic{\_}Content:AnsiString); virtual;\end{ttfamily}


\end{flushleft}
\end{list}
\paragraph*{GetHelpCtx{\_}Hint}\hspace*{\fill}

\begin{list}{}{
\settowidth{\tmplength}{\textbf{Declaração}}
\setlength{\itemindent}{0cm}
\setlength{\listparindent}{0cm}
\setlength{\leftmargin}{\evensidemargin}
\addtolength{\leftmargin}{\tmplength}
\settowidth{\labelsep}{X}
\addtolength{\leftmargin}{\labelsep}
\setlength{\labelwidth}{\tmplength}
}
\begin{flushleft}
\item[\textbf{Declaração}\hfill]
\begin{ttfamily}
protected Function GetHelpCtx{\_}Hint: AnsiString; VIRTUAL;\end{ttfamily}


\end{flushleft}
\end{list}
\paragraph*{GetHelpCtx{\_}Historico}\hspace*{\fill}

\begin{list}{}{
\settowidth{\tmplength}{\textbf{Declaração}}
\setlength{\itemindent}{0cm}
\setlength{\listparindent}{0cm}
\setlength{\leftmargin}{\evensidemargin}
\addtolength{\leftmargin}{\tmplength}
\settowidth{\labelsep}{X}
\addtolength{\leftmargin}{\labelsep}
\setlength{\labelwidth}{\tmplength}
}
\begin{flushleft}
\item[\textbf{Declaração}\hfill]
\begin{ttfamily}
protected Function GetHelpCtx{\_}Historico: AnsiString; VIRTUAL;\end{ttfamily}


\end{flushleft}
\end{list}
\paragraph*{GetHelpCtx{\_}Porque}\hspace*{\fill}

\begin{list}{}{
\settowidth{\tmplength}{\textbf{Declaração}}
\setlength{\itemindent}{0cm}
\setlength{\listparindent}{0cm}
\setlength{\leftmargin}{\evensidemargin}
\addtolength{\leftmargin}{\tmplength}
\settowidth{\labelsep}{X}
\addtolength{\leftmargin}{\labelsep}
\setlength{\labelwidth}{\tmplength}
}
\begin{flushleft}
\item[\textbf{Declaração}\hfill]
\begin{ttfamily}
protected Function GetHelpCtx{\_}Porque: AnsiString; VIRTUAL;\end{ttfamily}


\end{flushleft}
\end{list}
\paragraph*{GetHelpCtx{\_}Onde}\hspace*{\fill}

\begin{list}{}{
\settowidth{\tmplength}{\textbf{Declaração}}
\setlength{\itemindent}{0cm}
\setlength{\listparindent}{0cm}
\setlength{\leftmargin}{\evensidemargin}
\addtolength{\leftmargin}{\tmplength}
\settowidth{\labelsep}{X}
\addtolength{\leftmargin}{\labelsep}
\setlength{\labelwidth}{\tmplength}
}
\begin{flushleft}
\item[\textbf{Declaração}\hfill]
\begin{ttfamily}
protected Function GetHelpCtx{\_}Onde: AnsiString; VIRTUAL;\end{ttfamily}


\end{flushleft}
\end{list}
\paragraph*{GetHelpCtx{\_}Como}\hspace*{\fill}

\begin{list}{}{
\settowidth{\tmplength}{\textbf{Declaração}}
\setlength{\itemindent}{0cm}
\setlength{\listparindent}{0cm}
\setlength{\leftmargin}{\evensidemargin}
\addtolength{\leftmargin}{\tmplength}
\settowidth{\labelsep}{X}
\addtolength{\leftmargin}{\labelsep}
\setlength{\labelwidth}{\tmplength}
}
\begin{flushleft}
\item[\textbf{Declaração}\hfill]
\begin{ttfamily}
protected Function GetHelpCtx{\_}Como: AnsiString; VIRTUAL;\end{ttfamily}


\end{flushleft}
\end{list}
\paragraph*{GetHelpCtx{\_}Quais}\hspace*{\fill}

\begin{list}{}{
\settowidth{\tmplength}{\textbf{Declaração}}
\setlength{\itemindent}{0cm}
\setlength{\listparindent}{0cm}
\setlength{\leftmargin}{\evensidemargin}
\addtolength{\leftmargin}{\tmplength}
\settowidth{\labelsep}{X}
\addtolength{\leftmargin}{\labelsep}
\setlength{\labelwidth}{\tmplength}
}
\begin{flushleft}
\item[\textbf{Declaração}\hfill]
\begin{ttfamily}
protected Function GetHelpCtx{\_}Quais: AnsiString; VIRTUAL;\end{ttfamily}


\end{flushleft}
\end{list}
\paragraph*{DoBeforeCreate}\hspace*{\fill}

\begin{list}{}{
\settowidth{\tmplength}{\textbf{Declaração}}
\setlength{\itemindent}{0cm}
\setlength{\listparindent}{0cm}
\setlength{\leftmargin}{\evensidemargin}
\addtolength{\leftmargin}{\tmplength}
\settowidth{\labelsep}{X}
\addtolength{\leftmargin}{\labelsep}
\setlength{\labelwidth}{\tmplength}
}
\begin{flushleft}
\item[\textbf{Declaração}\hfill]
\begin{ttfamily}
public Procedure DoBeforeCreate(); Virtual;\end{ttfamily}


\end{flushleft}
\end{list}
\paragraph*{DoAfterCreate}\hspace*{\fill}

\begin{list}{}{
\settowidth{\tmplength}{\textbf{Declaração}}
\setlength{\itemindent}{0cm}
\setlength{\listparindent}{0cm}
\setlength{\leftmargin}{\evensidemargin}
\addtolength{\leftmargin}{\tmplength}
\settowidth{\labelsep}{X}
\addtolength{\leftmargin}{\labelsep}
\setlength{\labelwidth}{\tmplength}
}
\begin{flushleft}
\item[\textbf{Declaração}\hfill]
\begin{ttfamily}
public Procedure DoAfterCreate; Virtual;\end{ttfamily}


\end{flushleft}
\end{list}
\paragraph*{GetSelf}\hspace*{\fill}

\begin{list}{}{
\settowidth{\tmplength}{\textbf{Declaração}}
\setlength{\itemindent}{0cm}
\setlength{\listparindent}{0cm}
\setlength{\leftmargin}{\evensidemargin}
\addtolength{\leftmargin}{\tmplength}
\settowidth{\labelsep}{X}
\addtolength{\leftmargin}{\labelsep}
\setlength{\labelwidth}{\tmplength}
}
\begin{flushleft}
\item[\textbf{Declaração}\hfill]
\begin{ttfamily}
public Function GetSelf: TNSComponent;\end{ttfamily}


\end{flushleft}
\end{list}
\paragraph*{BeforeOpen}\hspace*{\fill}

\begin{list}{}{
\settowidth{\tmplength}{\textbf{Declaração}}
\setlength{\itemindent}{0cm}
\setlength{\listparindent}{0cm}
\setlength{\leftmargin}{\evensidemargin}
\addtolength{\leftmargin}{\tmplength}
\settowidth{\labelsep}{X}
\addtolength{\leftmargin}{\labelsep}
\setlength{\labelwidth}{\tmplength}
}
\begin{flushleft}
\item[\textbf{Declaração}\hfill]
\begin{ttfamily}
public Function BeforeOpen(Const APath,AAlias : tString):Boolean; Virtual;\end{ttfamily}


\end{flushleft}
\end{list}
\paragraph*{AfterOpen}\hspace*{\fill}

\begin{list}{}{
\settowidth{\tmplength}{\textbf{Declaração}}
\setlength{\itemindent}{0cm}
\setlength{\listparindent}{0cm}
\setlength{\leftmargin}{\evensidemargin}
\addtolength{\leftmargin}{\tmplength}
\settowidth{\labelsep}{X}
\addtolength{\leftmargin}{\labelsep}
\setlength{\labelwidth}{\tmplength}
}
\begin{flushleft}
\item[\textbf{Declaração}\hfill]
\begin{ttfamily}
public Function AfterOpen:Boolean; Virtual;\end{ttfamily}


\end{flushleft}
\end{list}
\paragraph*{Create}\hspace*{\fill}

\begin{list}{}{
\settowidth{\tmplength}{\textbf{Declaração}}
\setlength{\itemindent}{0cm}
\setlength{\listparindent}{0cm}
\setlength{\leftmargin}{\evensidemargin}
\addtolength{\leftmargin}{\tmplength}
\settowidth{\labelsep}{X}
\addtolength{\leftmargin}{\labelsep}
\setlength{\labelwidth}{\tmplength}
}
\begin{flushleft}
\item[\textbf{Declaração}\hfill]
\begin{ttfamily}
public Constructor Create(AOwner: TComponent); Overload; override;\end{ttfamily}


\end{flushleft}
\end{list}
\paragraph*{CCreate}\hspace*{\fill}

\begin{list}{}{
\settowidth{\tmplength}{\textbf{Declaração}}
\setlength{\itemindent}{0cm}
\setlength{\listparindent}{0cm}
\setlength{\leftmargin}{\evensidemargin}
\addtolength{\leftmargin}{\tmplength}
\settowidth{\labelsep}{X}
\addtolength{\leftmargin}{\labelsep}
\setlength{\labelwidth}{\tmplength}
}
\begin{flushleft}
\item[\textbf{Declaração}\hfill]
\begin{ttfamily}
public Function CCreate(wInstanceClass : TComponentClass):TNSComponent; overload; Virtual;\end{ttfamily}


\end{flushleft}
\end{list}
\paragraph*{CloneComponent}\hspace*{\fill}

\begin{list}{}{
\settowidth{\tmplength}{\textbf{Declaração}}
\setlength{\itemindent}{0cm}
\setlength{\listparindent}{0cm}
\setlength{\leftmargin}{\evensidemargin}
\addtolength{\leftmargin}{\tmplength}
\settowidth{\labelsep}{X}
\addtolength{\leftmargin}{\labelsep}
\setlength{\labelwidth}{\tmplength}
}
\begin{flushleft}
\item[\textbf{Declaração}\hfill]
\begin{ttfamily}
public function CloneComponent(): TComponent; Virtual;\end{ttfamily}


\end{flushleft}
\end{list}
\paragraph*{Destroy}\hspace*{\fill}

\begin{list}{}{
\settowidth{\tmplength}{\textbf{Declaração}}
\setlength{\itemindent}{0cm}
\setlength{\listparindent}{0cm}
\setlength{\leftmargin}{\evensidemargin}
\addtolength{\leftmargin}{\tmplength}
\settowidth{\labelsep}{X}
\addtolength{\leftmargin}{\labelsep}
\setlength{\labelwidth}{\tmplength}
}
\begin{flushleft}
\item[\textbf{Declaração}\hfill]
\begin{ttfamily}
public Destructor Destroy; Override;\end{ttfamily}


\end{flushleft}
\end{list}
\paragraph*{GetState}\hspace*{\fill}

\begin{list}{}{
\settowidth{\tmplength}{\textbf{Declaração}}
\setlength{\itemindent}{0cm}
\setlength{\listparindent}{0cm}
\setlength{\leftmargin}{\evensidemargin}
\addtolength{\leftmargin}{\tmplength}
\settowidth{\labelsep}{X}
\addtolength{\leftmargin}{\labelsep}
\setlength{\labelwidth}{\tmplength}
}
\begin{flushleft}
\item[\textbf{Declaração}\hfill]
\begin{ttfamily}
public function GetState(Const AState: Longint): Boolean; Virtual;\end{ttfamily}


\end{flushleft}
\end{list}
\paragraph*{SetState}\hspace*{\fill}

\begin{list}{}{
\settowidth{\tmplength}{\textbf{Declaração}}
\setlength{\itemindent}{0cm}
\setlength{\listparindent}{0cm}
\setlength{\leftmargin}{\evensidemargin}
\addtolength{\leftmargin}{\tmplength}
\settowidth{\labelsep}{X}
\addtolength{\leftmargin}{\labelsep}
\setlength{\labelwidth}{\tmplength}
}
\begin{flushleft}
\item[\textbf{Declaração}\hfill]
\begin{ttfamily}
public Function SetState(Const AState: Int64; Const Enable: boolean):Boolean; overload; Virtual;\end{ttfamily}


\end{flushleft}
\end{list}
\paragraph*{Abort{\_}Create}\hspace*{\fill}

\begin{list}{}{
\settowidth{\tmplength}{\textbf{Declaração}}
\setlength{\itemindent}{0cm}
\setlength{\listparindent}{0cm}
\setlength{\leftmargin}{\evensidemargin}
\addtolength{\leftmargin}{\tmplength}
\settowidth{\labelsep}{X}
\addtolength{\leftmargin}{\labelsep}
\setlength{\labelwidth}{\tmplength}
}
\begin{flushleft}
\item[\textbf{Declaração}\hfill]
\begin{ttfamily}
public PROCEDURE Abort{\_}Create; Virtual;\end{ttfamily}


\end{flushleft}
\end{list}
\paragraph*{IsDB}\hspace*{\fill}

\begin{list}{}{
\settowidth{\tmplength}{\textbf{Declaração}}
\setlength{\itemindent}{0cm}
\setlength{\listparindent}{0cm}
\setlength{\leftmargin}{\evensidemargin}
\addtolength{\leftmargin}{\tmplength}
\settowidth{\labelsep}{X}
\addtolength{\leftmargin}{\labelsep}
\setlength{\labelwidth}{\tmplength}
}
\begin{flushleft}
\item[\textbf{Declaração}\hfill]
\begin{ttfamily}
public function IsDB:Boolean; Virtual;\end{ttfamily}


\end{flushleft}
\end{list}
\paragraph*{IsTable}\hspace*{\fill}

\begin{list}{}{
\settowidth{\tmplength}{\textbf{Declaração}}
\setlength{\itemindent}{0cm}
\setlength{\listparindent}{0cm}
\setlength{\leftmargin}{\evensidemargin}
\addtolength{\leftmargin}{\tmplength}
\settowidth{\labelsep}{X}
\addtolength{\leftmargin}{\labelsep}
\setlength{\labelwidth}{\tmplength}
}
\begin{flushleft}
\item[\textbf{Declaração}\hfill]
\begin{ttfamily}
public Function IsTable:ITable; Virtual;\end{ttfamily}


\end{flushleft}
\end{list}
\paragraph*{IsField}\hspace*{\fill}

\begin{list}{}{
\settowidth{\tmplength}{\textbf{Declaração}}
\setlength{\itemindent}{0cm}
\setlength{\listparindent}{0cm}
\setlength{\leftmargin}{\evensidemargin}
\addtolength{\leftmargin}{\tmplength}
\settowidth{\labelsep}{X}
\addtolength{\leftmargin}{\labelsep}
\setlength{\labelwidth}{\tmplength}
}
\begin{flushleft}
\item[\textbf{Declaração}\hfill]
\begin{ttfamily}
public Function IsField:IHTML{\_}Base; Virtual;\end{ttfamily}


\end{flushleft}
\end{list}
\paragraph*{IsGroup}\hspace*{\fill}

\begin{list}{}{
\settowidth{\tmplength}{\textbf{Declaração}}
\setlength{\itemindent}{0cm}
\setlength{\listparindent}{0cm}
\setlength{\leftmargin}{\evensidemargin}
\addtolength{\leftmargin}{\tmplength}
\settowidth{\labelsep}{X}
\addtolength{\leftmargin}{\labelsep}
\setlength{\labelwidth}{\tmplength}
}
\begin{flushleft}
\item[\textbf{Declaração}\hfill]
\begin{ttfamily}
public Function IsGroup:Boolean; Virtual;\end{ttfamily}


\end{flushleft}
\end{list}
\paragraph*{IsFrame}\hspace*{\fill}

\begin{list}{}{
\settowidth{\tmplength}{\textbf{Declaração}}
\setlength{\itemindent}{0cm}
\setlength{\listparindent}{0cm}
\setlength{\leftmargin}{\evensidemargin}
\addtolength{\leftmargin}{\tmplength}
\settowidth{\labelsep}{X}
\addtolength{\leftmargin}{\labelsep}
\setlength{\labelwidth}{\tmplength}
}
\begin{flushleft}
\item[\textbf{Declaração}\hfill]
\begin{ttfamily}
public Function IsFrame:IFrame; Virtual;\end{ttfamily}


\end{flushleft}
\end{list}
\paragraph*{IsDialog}\hspace*{\fill}

\begin{list}{}{
\settowidth{\tmplength}{\textbf{Declaração}}
\setlength{\itemindent}{0cm}
\setlength{\listparindent}{0cm}
\setlength{\leftmargin}{\evensidemargin}
\addtolength{\leftmargin}{\tmplength}
\settowidth{\labelsep}{X}
\addtolength{\leftmargin}{\labelsep}
\setlength{\labelwidth}{\tmplength}
}
\begin{flushleft}
\item[\textbf{Declaração}\hfill]
\begin{ttfamily}
public Function IsDialog:Boolean; Virtual;\end{ttfamily}


\end{flushleft}
\end{list}
\paragraph*{IsInputText}\hspace*{\fill}

\begin{list}{}{
\settowidth{\tmplength}{\textbf{Declaração}}
\setlength{\itemindent}{0cm}
\setlength{\listparindent}{0cm}
\setlength{\leftmargin}{\evensidemargin}
\addtolength{\leftmargin}{\tmplength}
\settowidth{\labelsep}{X}
\addtolength{\leftmargin}{\labelsep}
\setlength{\labelwidth}{\tmplength}
}
\begin{flushleft}
\item[\textbf{Declaração}\hfill]
\begin{ttfamily}
public function IsInputText:IInputText; Virtual;\end{ttfamily}


\end{flushleft}
\end{list}
\paragraph*{IsInputButton}\hspace*{\fill}

\begin{list}{}{
\settowidth{\tmplength}{\textbf{Declaração}}
\setlength{\itemindent}{0cm}
\setlength{\listparindent}{0cm}
\setlength{\leftmargin}{\evensidemargin}
\addtolength{\leftmargin}{\tmplength}
\settowidth{\labelsep}{X}
\addtolength{\leftmargin}{\labelsep}
\setlength{\labelwidth}{\tmplength}
}
\begin{flushleft}
\item[\textbf{Declaração}\hfill]
\begin{ttfamily}
public function IsInputButton:IInputButton; Virtual;\end{ttfamily}


\end{flushleft}
\end{list}
\paragraph*{IsInputRadio}\hspace*{\fill}

\begin{list}{}{
\settowidth{\tmplength}{\textbf{Declaração}}
\setlength{\itemindent}{0cm}
\setlength{\listparindent}{0cm}
\setlength{\leftmargin}{\evensidemargin}
\addtolength{\leftmargin}{\tmplength}
\settowidth{\labelsep}{X}
\addtolength{\leftmargin}{\labelsep}
\setlength{\labelwidth}{\tmplength}
}
\begin{flushleft}
\item[\textbf{Declaração}\hfill]
\begin{ttfamily}
public function IsInputRadio:IInputRadio; Virtual;\end{ttfamily}


\end{flushleft}
\end{list}
\paragraph*{IsInputCheckbox}\hspace*{\fill}

\begin{list}{}{
\settowidth{\tmplength}{\textbf{Declaração}}
\setlength{\itemindent}{0cm}
\setlength{\listparindent}{0cm}
\setlength{\leftmargin}{\evensidemargin}
\addtolength{\leftmargin}{\tmplength}
\settowidth{\labelsep}{X}
\addtolength{\leftmargin}{\labelsep}
\setlength{\labelwidth}{\tmplength}
}
\begin{flushleft}
\item[\textbf{Declaração}\hfill]
\begin{ttfamily}
public function IsInputCheckbox:IInputCheckbox; Virtual;\end{ttfamily}


\end{flushleft}
\end{list}
\paragraph*{isInputPassword}\hspace*{\fill}

\begin{list}{}{
\settowidth{\tmplength}{\textbf{Declaração}}
\setlength{\itemindent}{0cm}
\setlength{\listparindent}{0cm}
\setlength{\leftmargin}{\evensidemargin}
\addtolength{\leftmargin}{\tmplength}
\settowidth{\labelsep}{X}
\addtolength{\leftmargin}{\labelsep}
\setlength{\labelwidth}{\tmplength}
}
\begin{flushleft}
\item[\textbf{Declaração}\hfill]
\begin{ttfamily}
public function isInputPassword:IInputPassword; Virtual;\end{ttfamily}


\end{flushleft}
\end{list}
\paragraph*{isInputHidden}\hspace*{\fill}

\begin{list}{}{
\settowidth{\tmplength}{\textbf{Declaração}}
\setlength{\itemindent}{0cm}
\setlength{\listparindent}{0cm}
\setlength{\leftmargin}{\evensidemargin}
\addtolength{\leftmargin}{\tmplength}
\settowidth{\labelsep}{X}
\addtolength{\leftmargin}{\labelsep}
\setlength{\labelwidth}{\tmplength}
}
\begin{flushleft}
\item[\textbf{Declaração}\hfill]
\begin{ttfamily}
public function isInputHidden:IInputHidden; Virtual;\end{ttfamily}


\end{flushleft}
\end{list}
\paragraph*{IsSelect}\hspace*{\fill}

\begin{list}{}{
\settowidth{\tmplength}{\textbf{Declaração}}
\setlength{\itemindent}{0cm}
\setlength{\listparindent}{0cm}
\setlength{\leftmargin}{\evensidemargin}
\addtolength{\leftmargin}{\tmplength}
\settowidth{\labelsep}{X}
\addtolength{\leftmargin}{\labelsep}
\setlength{\labelwidth}{\tmplength}
}
\begin{flushleft}
\item[\textbf{Declaração}\hfill]
\begin{ttfamily}
public function IsSelect:ISelect; Virtual;\end{ttfamily}


\end{flushleft}
\end{list}
\paragraph*{IsComboBox}\hspace*{\fill}

\begin{list}{}{
\settowidth{\tmplength}{\textbf{Declaração}}
\setlength{\itemindent}{0cm}
\setlength{\listparindent}{0cm}
\setlength{\leftmargin}{\evensidemargin}
\addtolength{\leftmargin}{\tmplength}
\settowidth{\labelsep}{X}
\addtolength{\leftmargin}{\labelsep}
\setlength{\labelwidth}{\tmplength}
}
\begin{flushleft}
\item[\textbf{Declaração}\hfill]
\begin{ttfamily}
public function IsComboBox:Boolean; Virtual;\end{ttfamily}


\end{flushleft}
\end{list}
\paragraph*{IsMultiCheckBoxes}\hspace*{\fill}

\begin{list}{}{
\settowidth{\tmplength}{\textbf{Declaração}}
\setlength{\itemindent}{0cm}
\setlength{\listparindent}{0cm}
\setlength{\leftmargin}{\evensidemargin}
\addtolength{\leftmargin}{\tmplength}
\settowidth{\labelsep}{X}
\addtolength{\leftmargin}{\labelsep}
\setlength{\labelwidth}{\tmplength}
}
\begin{flushleft}
\item[\textbf{Declaração}\hfill]
\begin{ttfamily}
public function IsMultiCheckBoxes:Boolean; Virtual;\end{ttfamily}


\end{flushleft}
\end{list}
\paragraph*{IsListBox}\hspace*{\fill}

\begin{list}{}{
\settowidth{\tmplength}{\textbf{Declaração}}
\setlength{\itemindent}{0cm}
\setlength{\listparindent}{0cm}
\setlength{\leftmargin}{\evensidemargin}
\addtolength{\leftmargin}{\tmplength}
\settowidth{\labelsep}{X}
\addtolength{\leftmargin}{\labelsep}
\setlength{\labelwidth}{\tmplength}
}
\begin{flushleft}
\item[\textbf{Declaração}\hfill]
\begin{ttfamily}
public function IsListBox:Boolean; Virtual;\end{ttfamily}


\end{flushleft}
\end{list}
\paragraph*{IsStaticText}\hspace*{\fill}

\begin{list}{}{
\settowidth{\tmplength}{\textbf{Declaração}}
\setlength{\itemindent}{0cm}
\setlength{\listparindent}{0cm}
\setlength{\leftmargin}{\evensidemargin}
\addtolength{\leftmargin}{\tmplength}
\settowidth{\labelsep}{X}
\addtolength{\leftmargin}{\labelsep}
\setlength{\labelwidth}{\tmplength}
}
\begin{flushleft}
\item[\textbf{Declaração}\hfill]
\begin{ttfamily}
public function IsStaticText:Boolean; Virtual;\end{ttfamily}


\end{flushleft}
\end{list}
\paragraph*{IsLabel}\hspace*{\fill}

\begin{list}{}{
\settowidth{\tmplength}{\textbf{Declaração}}
\setlength{\itemindent}{0cm}
\setlength{\listparindent}{0cm}
\setlength{\leftmargin}{\evensidemargin}
\addtolength{\leftmargin}{\tmplength}
\settowidth{\labelsep}{X}
\addtolength{\leftmargin}{\labelsep}
\setlength{\labelwidth}{\tmplength}
}
\begin{flushleft}
\item[\textbf{Declaração}\hfill]
\begin{ttfamily}
public function IsLabel:Boolean; Virtual;\end{ttfamily}


\end{flushleft}
\end{list}
\paragraph*{IsWindow}\hspace*{\fill}

\begin{list}{}{
\settowidth{\tmplength}{\textbf{Declaração}}
\setlength{\itemindent}{0cm}
\setlength{\listparindent}{0cm}
\setlength{\leftmargin}{\evensidemargin}
\addtolength{\leftmargin}{\tmplength}
\settowidth{\labelsep}{X}
\addtolength{\leftmargin}{\labelsep}
\setlength{\labelwidth}{\tmplength}
}
\begin{flushleft}
\item[\textbf{Declaração}\hfill]
\begin{ttfamily}
public function IsWindow:Boolean; Virtual;\end{ttfamily}


\end{flushleft}
\end{list}
\paragraph*{IsHistoryWindow}\hspace*{\fill}

\begin{list}{}{
\settowidth{\tmplength}{\textbf{Declaração}}
\setlength{\itemindent}{0cm}
\setlength{\listparindent}{0cm}
\setlength{\leftmargin}{\evensidemargin}
\addtolength{\leftmargin}{\tmplength}
\settowidth{\labelsep}{X}
\addtolength{\leftmargin}{\labelsep}
\setlength{\labelwidth}{\tmplength}
}
\begin{flushleft}
\item[\textbf{Declaração}\hfill]
\begin{ttfamily}
public function IsHistoryWindow:Boolean; Virtual;\end{ttfamily}


\end{flushleft}
\end{list}
\paragraph*{IsHistory}\hspace*{\fill}

\begin{list}{}{
\settowidth{\tmplength}{\textbf{Declaração}}
\setlength{\itemindent}{0cm}
\setlength{\listparindent}{0cm}
\setlength{\leftmargin}{\evensidemargin}
\addtolength{\leftmargin}{\tmplength}
\settowidth{\labelsep}{X}
\addtolength{\leftmargin}{\labelsep}
\setlength{\labelwidth}{\tmplength}
}
\begin{flushleft}
\item[\textbf{Declaração}\hfill]
\begin{ttfamily}
public function IsHistory:Boolean; Virtual;\end{ttfamily}


\end{flushleft}
\end{list}
\paragraph*{IsScroller}\hspace*{\fill}

\begin{list}{}{
\settowidth{\tmplength}{\textbf{Declaração}}
\setlength{\itemindent}{0cm}
\setlength{\listparindent}{0cm}
\setlength{\leftmargin}{\evensidemargin}
\addtolength{\leftmargin}{\tmplength}
\settowidth{\labelsep}{X}
\addtolength{\leftmargin}{\labelsep}
\setlength{\labelwidth}{\tmplength}
}
\begin{flushleft}
\item[\textbf{Declaração}\hfill]
\begin{ttfamily}
public function IsScroller:Boolean; Virtual;\end{ttfamily}


\end{flushleft}
\end{list}
\paragraph*{IsGrid}\hspace*{\fill}

\begin{list}{}{
\settowidth{\tmplength}{\textbf{Declaração}}
\setlength{\itemindent}{0cm}
\setlength{\listparindent}{0cm}
\setlength{\leftmargin}{\evensidemargin}
\addtolength{\leftmargin}{\tmplength}
\settowidth{\labelsep}{X}
\addtolength{\leftmargin}{\labelsep}
\setlength{\labelwidth}{\tmplength}
}
\begin{flushleft}
\item[\textbf{Declaração}\hfill]
\begin{ttfamily}
public function IsGrid:Boolean; Virtual;\end{ttfamily}


\end{flushleft}
\end{list}
\paragraph*{IsScrollBar}\hspace*{\fill}

\begin{list}{}{
\settowidth{\tmplength}{\textbf{Declaração}}
\setlength{\itemindent}{0cm}
\setlength{\listparindent}{0cm}
\setlength{\leftmargin}{\evensidemargin}
\addtolength{\leftmargin}{\tmplength}
\settowidth{\labelsep}{X}
\addtolength{\leftmargin}{\labelsep}
\setlength{\labelwidth}{\tmplength}
}
\begin{flushleft}
\item[\textbf{Declaração}\hfill]
\begin{ttfamily}
public function IsScrollBar:Boolean; Virtual;\end{ttfamily}


\end{flushleft}
\end{list}
\paragraph*{HandleEvent}\hspace*{\fill}

\begin{list}{}{
\settowidth{\tmplength}{\textbf{Declaração}}
\setlength{\itemindent}{0cm}
\setlength{\listparindent}{0cm}
\setlength{\leftmargin}{\evensidemargin}
\addtolength{\leftmargin}{\tmplength}
\settowidth{\labelsep}{X}
\addtolength{\leftmargin}{\labelsep}
\setlength{\labelwidth}{\tmplength}
}
\begin{flushleft}
\item[\textbf{Declaração}\hfill]
\begin{ttfamily}
public procedure HandleEvent(var Event: TEvent); Virtual;\end{ttfamily}


\end{flushleft}
\end{list}
\paragraph*{ClearEvent}\hspace*{\fill}

\begin{list}{}{
\settowidth{\tmplength}{\textbf{Declaração}}
\setlength{\itemindent}{0cm}
\setlength{\listparindent}{0cm}
\setlength{\leftmargin}{\evensidemargin}
\addtolength{\leftmargin}{\tmplength}
\settowidth{\labelsep}{X}
\addtolength{\leftmargin}{\labelsep}
\setlength{\labelwidth}{\tmplength}
}
\begin{flushleft}
\item[\textbf{Declaração}\hfill]
\begin{ttfamily}
public procedure ClearEvent(var Event: TEvent); Virtual;\end{ttfamily}


\end{flushleft}
\end{list}
\paragraph*{ClearEvents}\hspace*{\fill}

\begin{list}{}{
\settowidth{\tmplength}{\textbf{Declaração}}
\setlength{\itemindent}{0cm}
\setlength{\listparindent}{0cm}
\setlength{\leftmargin}{\evensidemargin}
\addtolength{\leftmargin}{\tmplength}
\settowidth{\labelsep}{X}
\addtolength{\leftmargin}{\labelsep}
\setlength{\labelwidth}{\tmplength}
}
\begin{flushleft}
\item[\textbf{Declaração}\hfill]
\begin{ttfamily}
public Procedure ClearEvents;\end{ttfamily}


\end{flushleft}
\end{list}
\paragraph*{Top{\_}Owner{\_}NSComponent}\hspace*{\fill}

\begin{list}{}{
\settowidth{\tmplength}{\textbf{Declaração}}
\setlength{\itemindent}{0cm}
\setlength{\listparindent}{0cm}
\setlength{\leftmargin}{\evensidemargin}
\addtolength{\leftmargin}{\tmplength}
\settowidth{\labelsep}{X}
\addtolength{\leftmargin}{\labelsep}
\setlength{\labelwidth}{\tmplength}
}
\begin{flushleft}
\item[\textbf{Declaração}\hfill]
\begin{ttfamily}
public Function Top{\_}Owner{\_}NSComponent:TNSComponent;\end{ttfamily}


\end{flushleft}
\end{list}
\paragraph*{SetOwner{\_}NSComponent}\hspace*{\fill}

\begin{list}{}{
\settowidth{\tmplength}{\textbf{Declaração}}
\setlength{\itemindent}{0cm}
\setlength{\listparindent}{0cm}
\setlength{\leftmargin}{\evensidemargin}
\addtolength{\leftmargin}{\tmplength}
\settowidth{\labelsep}{X}
\addtolength{\leftmargin}{\labelsep}
\setlength{\labelwidth}{\tmplength}
}
\begin{flushleft}
\item[\textbf{Declaração}\hfill]
\begin{ttfamily}
protected Procedure SetOwner{\_}NSComponent(aOwner{\_}NSComponent : TNSComponent); Virtual;\end{ttfamily}


\end{flushleft}
\end{list}
\paragraph*{GetRecordSelected}\hspace*{\fill}

\begin{list}{}{
\settowidth{\tmplength}{\textbf{Declaração}}
\setlength{\itemindent}{0cm}
\setlength{\listparindent}{0cm}
\setlength{\leftmargin}{\evensidemargin}
\addtolength{\leftmargin}{\tmplength}
\settowidth{\labelsep}{X}
\addtolength{\leftmargin}{\labelsep}
\setlength{\labelwidth}{\tmplength}
}
\begin{flushleft}
\item[\textbf{Declaração}\hfill]
\begin{ttfamily}
protected Function GetRecordSelected: boolean; Virtual;\end{ttfamily}


\end{flushleft}
\end{list}
\paragraph*{SetRecordSelected}\hspace*{\fill}

\begin{list}{}{
\settowidth{\tmplength}{\textbf{Declaração}}
\setlength{\itemindent}{0cm}
\setlength{\listparindent}{0cm}
\setlength{\leftmargin}{\evensidemargin}
\addtolength{\leftmargin}{\tmplength}
\settowidth{\labelsep}{X}
\addtolength{\leftmargin}{\labelsep}
\setlength{\labelwidth}{\tmplength}
}
\begin{flushleft}
\item[\textbf{Declaração}\hfill]
\begin{ttfamily}
protected Procedure SetRecordSelected(a{\_}RecordSelected : boolean); Virtual;\end{ttfamily}


\end{flushleft}
\end{list}
\paragraph*{SetFieldSelected}\hspace*{\fill}

\begin{list}{}{
\settowidth{\tmplength}{\textbf{Declaração}}
\setlength{\itemindent}{0cm}
\setlength{\listparindent}{0cm}
\setlength{\leftmargin}{\evensidemargin}
\addtolength{\leftmargin}{\tmplength}
\settowidth{\labelsep}{X}
\addtolength{\leftmargin}{\labelsep}
\setlength{\labelwidth}{\tmplength}
}
\begin{flushleft}
\item[\textbf{Declaração}\hfill]
\begin{ttfamily}
protected Procedure SetFieldSelected(a{\_}FieldSelected : boolean); Overload; Virtual;\end{ttfamily}


\end{flushleft}
\end{list}
\paragraph*{GetFieldSelected}\hspace*{\fill}

\begin{list}{}{
\settowidth{\tmplength}{\textbf{Declaração}}
\setlength{\itemindent}{0cm}
\setlength{\listparindent}{0cm}
\setlength{\leftmargin}{\evensidemargin}
\addtolength{\leftmargin}{\tmplength}
\settowidth{\labelsep}{X}
\addtolength{\leftmargin}{\labelsep}
\setlength{\labelwidth}{\tmplength}
}
\begin{flushleft}
\item[\textbf{Declaração}\hfill]
\begin{ttfamily}
protected Function GetFieldSelected: boolean; Overload; Virtual;\end{ttfamily}


\end{flushleft}
\end{list}
\paragraph*{GetHTMLContent}\hspace*{\fill}

\begin{list}{}{
\settowidth{\tmplength}{\textbf{Declaração}}
\setlength{\itemindent}{0cm}
\setlength{\listparindent}{0cm}
\setlength{\leftmargin}{\evensidemargin}
\addtolength{\leftmargin}{\tmplength}
\settowidth{\labelsep}{X}
\addtolength{\leftmargin}{\labelsep}
\setlength{\labelwidth}{\tmplength}
}
\begin{flushleft}
\item[\textbf{Declaração}\hfill]
\begin{ttfamily}
protected Function GetHTMLContent: AnsiString; Virtual;\end{ttfamily}


\end{flushleft}
\end{list}
\paragraph*{SetHTMLFile}\hspace*{\fill}

\begin{list}{}{
\settowidth{\tmplength}{\textbf{Declaração}}
\setlength{\itemindent}{0cm}
\setlength{\listparindent}{0cm}
\setlength{\leftmargin}{\evensidemargin}
\addtolength{\leftmargin}{\tmplength}
\settowidth{\labelsep}{X}
\addtolength{\leftmargin}{\labelsep}
\setlength{\labelwidth}{\tmplength}
}
\begin{flushleft}
\item[\textbf{Declaração}\hfill]
\begin{ttfamily}
protected Procedure SetHTMLFile(Const aHTMLFile: TFileName ); Overload; Virtual;\end{ttfamily}


\end{flushleft}
\end{list}
\paragraph*{SetHTMLFile}\hspace*{\fill}

\begin{list}{}{
\settowidth{\tmplength}{\textbf{Declaração}}
\setlength{\itemindent}{0cm}
\setlength{\listparindent}{0cm}
\setlength{\leftmargin}{\evensidemargin}
\addtolength{\leftmargin}{\tmplength}
\settowidth{\labelsep}{X}
\addtolength{\leftmargin}{\labelsep}
\setlength{\labelwidth}{\tmplength}
}
\begin{flushleft}
\item[\textbf{Declaração}\hfill]
\begin{ttfamily}
public Procedure SetHTMLFile(); Overload; Virtual;\end{ttfamily}


\end{flushleft}
\end{list}
\paragraph*{SaveHTMLContentToFile}\hspace*{\fill}

\begin{list}{}{
\settowidth{\tmplength}{\textbf{Declaração}}
\setlength{\itemindent}{0cm}
\setlength{\listparindent}{0cm}
\setlength{\leftmargin}{\evensidemargin}
\addtolength{\leftmargin}{\tmplength}
\settowidth{\labelsep}{X}
\addtolength{\leftmargin}{\labelsep}
\setlength{\labelwidth}{\tmplength}
}
\begin{flushleft}
\item[\textbf{Declaração}\hfill]
\begin{ttfamily}
public Function SaveHTMLContentToFile(FileNameDest:AnsiString):Integer; overload; Virtual;\end{ttfamily}


\end{flushleft}
\end{list}
\paragraph*{SaveHTMLContentToFile}\hspace*{\fill}

\begin{list}{}{
\settowidth{\tmplength}{\textbf{Declaração}}
\setlength{\itemindent}{0cm}
\setlength{\listparindent}{0cm}
\setlength{\leftmargin}{\evensidemargin}
\addtolength{\leftmargin}{\tmplength}
\settowidth{\labelsep}{X}
\addtolength{\leftmargin}{\labelsep}
\setlength{\labelwidth}{\tmplength}
}
\begin{flushleft}
\item[\textbf{Declaração}\hfill]
\begin{ttfamily}
public Function SaveHTMLContentToFile:Integer; overload; Virtual;\end{ttfamily}


\end{flushleft}
\end{list}
\paragraph*{CreateHTML}\hspace*{\fill}

\begin{list}{}{
\settowidth{\tmplength}{\textbf{Declaração}}
\setlength{\itemindent}{0cm}
\setlength{\listparindent}{0cm}
\setlength{\leftmargin}{\evensidemargin}
\addtolength{\leftmargin}{\tmplength}
\settowidth{\labelsep}{X}
\addtolength{\leftmargin}{\labelsep}
\setlength{\labelwidth}{\tmplength}
}
\begin{flushleft}
\item[\textbf{Declaração}\hfill]
\begin{ttfamily}
protected function CreateHTML: AnsiString; Overload; Virtual;\end{ttfamily}


\end{flushleft}
\end{list}
\paragraph*{Set{\_}RecordAltered}\hspace*{\fill}

\begin{list}{}{
\settowidth{\tmplength}{\textbf{Declaração}}
\setlength{\itemindent}{0cm}
\setlength{\listparindent}{0cm}
\setlength{\leftmargin}{\evensidemargin}
\addtolength{\leftmargin}{\tmplength}
\settowidth{\labelsep}{X}
\addtolength{\leftmargin}{\labelsep}
\setlength{\labelwidth}{\tmplength}
}
\begin{flushleft}
\item[\textbf{Declaração}\hfill]
\begin{ttfamily}
protected Procedure Set{\_}RecordAltered(aSetRecordAltered:Boolean); VIRTUAL;\end{ttfamily}


\end{flushleft}
\end{list}
\paragraph*{Get{\_}FieldAltered}\hspace*{\fill}

\begin{list}{}{
\settowidth{\tmplength}{\textbf{Declaração}}
\setlength{\itemindent}{0cm}
\setlength{\listparindent}{0cm}
\setlength{\leftmargin}{\evensidemargin}
\addtolength{\leftmargin}{\tmplength}
\settowidth{\labelsep}{X}
\addtolength{\leftmargin}{\labelsep}
\setlength{\labelwidth}{\tmplength}
}
\begin{flushleft}
\item[\textbf{Declaração}\hfill]
\begin{ttfamily}
protected Function Get{\_}FieldAltered:Boolean; VIRTUAL;\end{ttfamily}


\end{flushleft}
\end{list}
\paragraph*{Set{\_}FieldAltered}\hspace*{\fill}

\begin{list}{}{
\settowidth{\tmplength}{\textbf{Declaração}}
\setlength{\itemindent}{0cm}
\setlength{\listparindent}{0cm}
\setlength{\leftmargin}{\evensidemargin}
\addtolength{\leftmargin}{\tmplength}
\settowidth{\labelsep}{X}
\addtolength{\leftmargin}{\labelsep}
\setlength{\labelwidth}{\tmplength}
}
\begin{flushleft}
\item[\textbf{Declaração}\hfill]
\begin{ttfamily}
protected Procedure Set{\_}FieldAltered(aFieldAltered:Boolean); VIRTUAL;\end{ttfamily}


\end{flushleft}
\end{list}
\paragraph*{Get{\_}KeyAltered}\hspace*{\fill}

\begin{list}{}{
\settowidth{\tmplength}{\textbf{Declaração}}
\setlength{\itemindent}{0cm}
\setlength{\listparindent}{0cm}
\setlength{\leftmargin}{\evensidemargin}
\addtolength{\leftmargin}{\tmplength}
\settowidth{\labelsep}{X}
\addtolength{\leftmargin}{\labelsep}
\setlength{\labelwidth}{\tmplength}
}
\begin{flushleft}
\item[\textbf{Declaração}\hfill]
\begin{ttfamily}
protected Function Get{\_}KeyAltered:Boolean; VIRTUAL;\end{ttfamily}


\end{flushleft}
\end{list}
\paragraph*{Set{\_}KeyAltered}\hspace*{\fill}

\begin{list}{}{
\settowidth{\tmplength}{\textbf{Declaração}}
\setlength{\itemindent}{0cm}
\setlength{\listparindent}{0cm}
\setlength{\leftmargin}{\evensidemargin}
\addtolength{\leftmargin}{\tmplength}
\settowidth{\labelsep}{X}
\addtolength{\leftmargin}{\labelsep}
\setlength{\labelwidth}{\tmplength}
}
\begin{flushleft}
\item[\textbf{Declaração}\hfill]
\begin{ttfamily}
protected Procedure Set{\_}KeyAltered(aKeyAltered:Boolean); VIRTUAL;\end{ttfamily}


\end{flushleft}
\end{list}
\paragraph*{Get{\_}Appending}\hspace*{\fill}

\begin{list}{}{
\settowidth{\tmplength}{\textbf{Declaração}}
\setlength{\itemindent}{0cm}
\setlength{\listparindent}{0cm}
\setlength{\leftmargin}{\evensidemargin}
\addtolength{\leftmargin}{\tmplength}
\settowidth{\labelsep}{X}
\addtolength{\leftmargin}{\labelsep}
\setlength{\labelwidth}{\tmplength}
}
\begin{flushleft}
\item[\textbf{Declaração}\hfill]
\begin{ttfamily}
protected Function Get{\_}Appending:Boolean; VIRTUAL;\end{ttfamily}


\end{flushleft}
\end{list}
\paragraph*{Set{\_}Appending}\hspace*{\fill}

\begin{list}{}{
\settowidth{\tmplength}{\textbf{Declaração}}
\setlength{\itemindent}{0cm}
\setlength{\listparindent}{0cm}
\setlength{\leftmargin}{\evensidemargin}
\addtolength{\leftmargin}{\tmplength}
\settowidth{\labelsep}{X}
\addtolength{\leftmargin}{\labelsep}
\setlength{\labelwidth}{\tmplength}
}
\begin{flushleft}
\item[\textbf{Declaração}\hfill]
\begin{ttfamily}
protected Procedure Set{\_}Appending(aAppending:Boolean); VIRTUAL;\end{ttfamily}


\end{flushleft}
\end{list}
\paragraph*{SetAppend}\hspace*{\fill}

\begin{list}{}{
\settowidth{\tmplength}{\textbf{Declaração}}
\setlength{\itemindent}{0cm}
\setlength{\listparindent}{0cm}
\setlength{\leftmargin}{\evensidemargin}
\addtolength{\leftmargin}{\tmplength}
\settowidth{\labelsep}{X}
\addtolength{\leftmargin}{\labelsep}
\setlength{\labelwidth}{\tmplength}
}
\begin{flushleft}
\item[\textbf{Declaração}\hfill]
\begin{ttfamily}
protected Procedure SetAppend(aAppend:Boolean); Virtual;\end{ttfamily}


\end{flushleft}
\end{list}
\paragraph*{Get{\_}RecordLimit}\hspace*{\fill}

\begin{list}{}{
\settowidth{\tmplength}{\textbf{Declaração}}
\setlength{\itemindent}{0cm}
\setlength{\listparindent}{0cm}
\setlength{\leftmargin}{\evensidemargin}
\addtolength{\leftmargin}{\tmplength}
\settowidth{\labelsep}{X}
\addtolength{\leftmargin}{\labelsep}
\setlength{\labelwidth}{\tmplength}
}
\begin{flushleft}
\item[\textbf{Declaração}\hfill]
\begin{ttfamily}
protected Function Get{\_}RecordLimit: longint; overload; Virtual;\end{ttfamily}


\end{flushleft}
\end{list}
\subsection*{TClass Classe}
\subsubsection*{\large{\textbf{Hierarquia}}\normalsize\hspace{1ex}\hfill}
TClass {$>$} TPersistent
%%%%Descrição
\subsubsection*{\large{\textbf{Propriedades}}\normalsize\hspace{1ex}\hfill}
\paragraph*{OkCreate}\hspace*{\fill}

\begin{list}{}{
\settowidth{\tmplength}{\textbf{Declaração}}
\setlength{\itemindent}{0cm}
\setlength{\listparindent}{0cm}
\setlength{\leftmargin}{\evensidemargin}
\addtolength{\leftmargin}{\tmplength}
\settowidth{\labelsep}{X}
\addtolength{\leftmargin}{\labelsep}
\setlength{\labelwidth}{\tmplength}
}
\begin{flushleft}
\item[\textbf{Declaração}\hfill]
\begin{ttfamily}
public property OkCreate : Boolean Read {\_}okCreate Write {\_}okCreate Default false;\end{ttfamily}


\end{flushleft}
\end{list}
\paragraph*{Alias}\hspace*{\fill}

\begin{list}{}{
\settowidth{\tmplength}{\textbf{Declaração}}
\setlength{\itemindent}{0cm}
\setlength{\listparindent}{0cm}
\setlength{\leftmargin}{\evensidemargin}
\addtolength{\leftmargin}{\tmplength}
\settowidth{\labelsep}{X}
\addtolength{\leftmargin}{\labelsep}
\setlength{\labelwidth}{\tmplength}
}
\begin{flushleft}
\item[\textbf{Declaração}\hfill]
\begin{ttfamily}
public property Alias : AnsiString Read GetAlias Write SetAlias;\end{ttfamily}


\end{flushleft}
\end{list}
\paragraph*{ClassName}\hspace*{\fill}

\begin{list}{}{
\settowidth{\tmplength}{\textbf{Declaração}}
\setlength{\itemindent}{0cm}
\setlength{\listparindent}{0cm}
\setlength{\leftmargin}{\evensidemargin}
\addtolength{\leftmargin}{\tmplength}
\settowidth{\labelsep}{X}
\addtolength{\leftmargin}{\labelsep}
\setlength{\labelwidth}{\tmplength}
}
\begin{flushleft}
\item[\textbf{Declaração}\hfill]
\begin{ttfamily}
public property ClassName : String read GetClassName ;\end{ttfamily}


\end{flushleft}
\end{list}
\subsubsection*{\large{\textbf{Campos}}\normalsize\hspace{1ex}\hfill}
\paragraph*{State}\hspace*{\fill}

\begin{list}{}{
\settowidth{\tmplength}{\textbf{Declaração}}
\setlength{\itemindent}{0cm}
\setlength{\listparindent}{0cm}
\setlength{\leftmargin}{\evensidemargin}
\addtolength{\leftmargin}{\tmplength}
\settowidth{\labelsep}{X}
\addtolength{\leftmargin}{\labelsep}
\setlength{\labelwidth}{\tmplength}
}
\begin{flushleft}
\item[\textbf{Declaração}\hfill]
\begin{ttfamily}
public State: Longint;\end{ttfamily}


\end{flushleft}
\end{list}
\subsubsection*{\large{\textbf{Métodos}}\normalsize\hspace{1ex}\hfill}
\paragraph*{GetSelf}\hspace*{\fill}

\begin{list}{}{
\settowidth{\tmplength}{\textbf{Declaração}}
\setlength{\itemindent}{0cm}
\setlength{\listparindent}{0cm}
\setlength{\leftmargin}{\evensidemargin}
\addtolength{\leftmargin}{\tmplength}
\settowidth{\labelsep}{X}
\addtolength{\leftmargin}{\labelsep}
\setlength{\labelwidth}{\tmplength}
}
\begin{flushleft}
\item[\textbf{Declaração}\hfill]
\begin{ttfamily}
public Function GetSelf: TClass;\end{ttfamily}


\end{flushleft}
\end{list}
\paragraph*{Create}\hspace*{\fill}

\begin{list}{}{
\settowidth{\tmplength}{\textbf{Declaração}}
\setlength{\itemindent}{0cm}
\setlength{\listparindent}{0cm}
\setlength{\leftmargin}{\evensidemargin}
\addtolength{\leftmargin}{\tmplength}
\settowidth{\labelsep}{X}
\addtolength{\leftmargin}{\labelsep}
\setlength{\labelwidth}{\tmplength}
}
\begin{flushleft}
\item[\textbf{Declaração}\hfill]
\begin{ttfamily}
public CONSTRUCTOR Create; Virtual;\end{ttfamily}


\end{flushleft}
\end{list}
\paragraph*{Free}\hspace*{\fill}

\begin{list}{}{
\settowidth{\tmplength}{\textbf{Declaração}}
\setlength{\itemindent}{0cm}
\setlength{\listparindent}{0cm}
\setlength{\leftmargin}{\evensidemargin}
\addtolength{\leftmargin}{\tmplength}
\settowidth{\labelsep}{X}
\addtolength{\leftmargin}{\labelsep}
\setlength{\labelwidth}{\tmplength}
}
\begin{flushleft}
\item[\textbf{Declaração}\hfill]
\begin{ttfamily}
public PROCEDURE Free;\end{ttfamily}


\end{flushleft}
\end{list}
\paragraph*{Abort{\_}Create}\hspace*{\fill}

\begin{list}{}{
\settowidth{\tmplength}{\textbf{Declaração}}
\setlength{\itemindent}{0cm}
\setlength{\listparindent}{0cm}
\setlength{\leftmargin}{\evensidemargin}
\addtolength{\leftmargin}{\tmplength}
\settowidth{\labelsep}{X}
\addtolength{\leftmargin}{\labelsep}
\setlength{\labelwidth}{\tmplength}
}
\begin{flushleft}
\item[\textbf{Declaração}\hfill]
\begin{ttfamily}
public PROCEDURE Abort{\_}Create; Virtual;\end{ttfamily}


\end{flushleft}
\end{list}
\paragraph*{Destroy}\hspace*{\fill}

\begin{list}{}{
\settowidth{\tmplength}{\textbf{Declaração}}
\setlength{\itemindent}{0cm}
\setlength{\listparindent}{0cm}
\setlength{\leftmargin}{\evensidemargin}
\addtolength{\leftmargin}{\tmplength}
\settowidth{\labelsep}{X}
\addtolength{\leftmargin}{\labelsep}
\setlength{\labelwidth}{\tmplength}
}
\begin{flushleft}
\item[\textbf{Declaração}\hfill]
\begin{ttfamily}
public DESTRUCTOR Destroy; Override;\end{ttfamily}


\end{flushleft}
\end{list}
\paragraph*{GetState}\hspace*{\fill}

\begin{list}{}{
\settowidth{\tmplength}{\textbf{Declaração}}
\setlength{\itemindent}{0cm}
\setlength{\listparindent}{0cm}
\setlength{\leftmargin}{\evensidemargin}
\addtolength{\leftmargin}{\tmplength}
\settowidth{\labelsep}{X}
\addtolength{\leftmargin}{\labelsep}
\setlength{\labelwidth}{\tmplength}
}
\begin{flushleft}
\item[\textbf{Declaração}\hfill]
\begin{ttfamily}
public function GetState(Const AState: Longint): Boolean; Virtual;\end{ttfamily}


\end{flushleft}
\end{list}
\paragraph*{SetState}\hspace*{\fill}

\begin{list}{}{
\settowidth{\tmplength}{\textbf{Declaração}}
\setlength{\itemindent}{0cm}
\setlength{\listparindent}{0cm}
\setlength{\leftmargin}{\evensidemargin}
\addtolength{\leftmargin}{\tmplength}
\settowidth{\labelsep}{X}
\addtolength{\leftmargin}{\labelsep}
\setlength{\labelwidth}{\tmplength}
}
\begin{flushleft}
\item[\textbf{Declaração}\hfill]
\begin{ttfamily}
public Function SetState(Const AState: Int64; Const Enable: boolean):Boolean; overload; Virtual;\end{ttfamily}


\end{flushleft}
\end{list}
\section{Funções e Procedimentos}
\subsection*{Message}
\begin{list}{}{
\settowidth{\tmplength}{\textbf{Declaração}}
\setlength{\itemindent}{0cm}
\setlength{\listparindent}{0cm}
\setlength{\leftmargin}{\evensidemargin}
\addtolength{\leftmargin}{\tmplength}
\settowidth{\labelsep}{X}
\addtolength{\leftmargin}{\labelsep}
\setlength{\labelwidth}{\tmplength}
}
\begin{flushleft}
\item[\textbf{Declaração}\hfill]
\begin{ttfamily}
function Message(Receiver: TNSComponent; What, Command: Word;InfoPtr: Pointer): Pointer;\end{ttfamily}


\end{flushleft}
\end{list}
\subsection*{CloneComponent}
\begin{list}{}{
\settowidth{\tmplength}{\textbf{Declaração}}
\setlength{\itemindent}{0cm}
\setlength{\listparindent}{0cm}
\setlength{\leftmargin}{\evensidemargin}
\addtolength{\leftmargin}{\tmplength}
\settowidth{\labelsep}{X}
\addtolength{\leftmargin}{\labelsep}
\setlength{\labelwidth}{\tmplength}
}
\begin{flushleft}
\item[\textbf{Declaração}\hfill]
\begin{ttfamily}
function CloneComponent(aComponent: TComponent): TComponent;\end{ttfamily}


\end{flushleft}
\end{list}
\subsection*{StrJSonToJSONObject}
\begin{list}{}{
\settowidth{\tmplength}{\textbf{Declaração}}
\setlength{\itemindent}{0cm}
\setlength{\listparindent}{0cm}
\setlength{\leftmargin}{\evensidemargin}
\addtolength{\leftmargin}{\tmplength}
\settowidth{\labelsep}{X}
\addtolength{\leftmargin}{\labelsep}
\setlength{\labelwidth}{\tmplength}
}
\begin{flushleft}
\item[\textbf{Declaração}\hfill]
\begin{ttfamily}
Function StrJSonToJSONObject(StrJSon: String):TJSONObject;\end{ttfamily}


\end{flushleft}
\end{list}
\subsection*{JSONObjectToStrJSon}
\begin{list}{}{
\settowidth{\tmplength}{\textbf{Declaração}}
\setlength{\itemindent}{0cm}
\setlength{\listparindent}{0cm}
\setlength{\leftmargin}{\evensidemargin}
\addtolength{\leftmargin}{\tmplength}
\settowidth{\labelsep}{X}
\addtolength{\leftmargin}{\labelsep}
\setlength{\labelwidth}{\tmplength}
}
\begin{flushleft}
\item[\textbf{Declaração}\hfill]
\begin{ttfamily}
Function JSONObjectToStrJSon(aJSONObject :TJSONObject): String;\end{ttfamily}


\end{flushleft}
\end{list}
\subsection*{StrJSonToArrays}
\begin{list}{}{
\settowidth{\tmplength}{\textbf{Declaração}}
\setlength{\itemindent}{0cm}
\setlength{\listparindent}{0cm}
\setlength{\leftmargin}{\evensidemargin}
\addtolength{\leftmargin}{\tmplength}
\settowidth{\labelsep}{X}
\addtolength{\leftmargin}{\labelsep}
\setlength{\labelwidth}{\tmplength}
}
\begin{flushleft}
\item[\textbf{Declaração}\hfill]
\begin{ttfamily}
procedure StrJSonToArrays(StrJSon: String;Var aNames,aValues: TArrayOpenVariant);\end{ttfamily}


\end{flushleft}
\end{list}
\subsection*{ArraysToJSONValue}
\begin{list}{}{
\settowidth{\tmplength}{\textbf{Declaração}}
\setlength{\itemindent}{0cm}
\setlength{\listparindent}{0cm}
\setlength{\leftmargin}{\evensidemargin}
\addtolength{\leftmargin}{\tmplength}
\settowidth{\labelsep}{X}
\addtolength{\leftmargin}{\labelsep}
\setlength{\labelwidth}{\tmplength}
}
\begin{flushleft}
\item[\textbf{Declaração}\hfill]
\begin{ttfamily}
Function ArraysToJSONValue(Const aNames,aValues: TArrayOpenVariant ):TJSONValue;\end{ttfamily}


\end{flushleft}
\end{list}
\subsection*{IsValidPtr}
\begin{list}{}{
\settowidth{\tmplength}{\textbf{Declaração}}
\setlength{\itemindent}{0cm}
\setlength{\listparindent}{0cm}
\setlength{\leftmargin}{\evensidemargin}
\addtolength{\leftmargin}{\tmplength}
\settowidth{\labelsep}{X}
\addtolength{\leftmargin}{\labelsep}
\setlength{\labelwidth}{\tmplength}
}
\begin{flushleft}
\item[\textbf{Declaração}\hfill]
\begin{ttfamily}
FUNCTION IsValidPtr( aClass:TNSComponent):BOOLEAN ; Overload;\end{ttfamily}


\end{flushleft}
\end{list}
\subsection*{DISCARD}
\begin{list}{}{
\settowidth{\tmplength}{\textbf{Declaração}}
\setlength{\itemindent}{0cm}
\setlength{\listparindent}{0cm}
\setlength{\leftmargin}{\evensidemargin}
\addtolength{\leftmargin}{\tmplength}
\settowidth{\labelsep}{X}
\addtolength{\leftmargin}{\labelsep}
\setlength{\labelwidth}{\tmplength}
}
\begin{flushleft}
\item[\textbf{Declaração}\hfill]
\begin{ttfamily}
PROCEDURE DISCARD(Var AClass);\end{ttfamily}


\end{flushleft}
\end{list}
\chapter{Program fpmake}
\section{Uses}
\begin{itemize}
\item \begin{ttfamily}fpmkunit\end{ttfamily}\end{itemize}
\chapter{Program httpproject1}
\section{Uses}
\begin{itemize}
\item \begin{ttfamily}fphttpapp\end{ttfamily}\item \begin{ttfamily}unit1\end{ttfamily}(\ref{Unit1})\end{itemize}
\chapter{Unit mi.rtl}
\section{Uses}
\begin{itemize}
\item \begin{ttfamily}mi.rtl.ApplicationAbstract\end{ttfamily}(\ref{mi.rtl.ApplicationAbstract})\item \begin{ttfamily}mi.rtl.Class{\_}Of{\_}Char\end{ttfamily}(\ref{mi.rtl.Class_Of_Char})\item \begin{ttfamily}mi.rtl.Types\end{ttfamily}(\ref{mi.rtl.Types})\item \begin{ttfamily}mi.rtl.Consts\end{ttfamily}(\ref{mi.rtl.Consts})\item \begin{ttfamily}mi.rtl.files\end{ttfamily}(\ref{mi.rtl.files})\item \begin{ttfamily}mi.rtl.Consts.StrError\end{ttfamily}(\ref{mi.rtl.Consts.StrError})\item \begin{ttfamily}mi.rtl.Consts.StringListBase\end{ttfamily}(\ref{mi.rtl.Consts.StringListBase})\item \begin{ttfamily}mi.rtl.Consts.StringList\end{ttfamily}(\ref{mi.rtl.Consts.StringList})\item \begin{ttfamily}mi.rtl.objects.types\end{ttfamily}(\ref{mi.rtl.objects.types})\item \begin{ttfamily}mi.rtl.Objects.Consts\end{ttfamily}(\ref{mi.rtl.Objects.Consts})\item \begin{ttfamily}mi.rtl.Objects.Consts.Logs\end{ttfamily}(\ref{mi.rtl.Objects.Consts.Logs})\item \begin{ttfamily}mi.rtl.Objects.Consts.Mi{\_}MsgBox\end{ttfamily}\item \begin{ttfamily}mi.rtl.Objects.Consts.ProgressDlg{\_}If\end{ttfamily}(\ref{mi.rtl.Objects.Consts.ProgressDlg_If})\item \begin{ttfamily}mi.rtl.Objects.Methods\end{ttfamily}(\ref{mi.rtl.Objects.Methods})\item \begin{ttfamily}mi.rtl.objects.Methods.dates\end{ttfamily}(\ref{mi.rtl.objects.Methods.dates})\item \begin{ttfamily}mi.rtl.Objects.Methods.Exception\end{ttfamily}(\ref{mi.rtl.Objects.Methods.Exception})\item \begin{ttfamily}mi.rtl.Objects.Methods.Paramexecucao\end{ttfamily}(\ref{mi.rtl.Objects.Methods.Paramexecucao})\item \begin{ttfamily}mi.rtl.Objects.Methods.Paramexecucao.Application\end{ttfamily}(\ref{mi.rtl.Objects.Methods.Paramexecucao.Application})\item \begin{ttfamily}mi.rtl.Objects.Methods.StreamBase\end{ttfamily}(\ref{mi.rtl.Objects.Methods.StreamBase})\item \begin{ttfamily}mi.rtl.Objects.Methods.StreamBase.Stream\end{ttfamily}(\ref{mi.rtl.Objects.Methods.StreamBase.Stream})\item \begin{ttfamily}mi.rtl.Objects.Methods.StreamBase.Stream.FileStream\end{ttfamily}(\ref{mi.rtl.Objects.Methods.StreamBase.Stream.FileStream})\item \begin{ttfamily}mi.rtl.Objects.Methods.StreamBase.Stream.MemoryStream\end{ttfamily}(\ref{mi.rtl.Objects.Methods.StreamBase.Stream.MemoryStream})\item \begin{ttfamily}mi.rtl.objects.methods.StreamBase.Stream.MemoryStream.BufferMemory\end{ttfamily}(\ref{mi.rtl.objects.methods.StreamBase.Stream.MemoryStream.BufferMemory})\item \begin{ttfamily}mi.rtl.Objects.Methods.Collection\end{ttfamily}(\ref{mi.rtl.Objects.Methods.Collection})\item \begin{ttfamily}mi.rtl.Objects.Methods.Collection.FilesStreams\end{ttfamily}(\ref{mi.rtl.Objects.Methods.Collection.FilesStreams})\item \begin{ttfamily}mi.rtl.Objects.Methods.System\end{ttfamily}(\ref{mi.rtl.Objects.Methods.System})\item \begin{ttfamily}mi.rtl.Objects.Methods.Collection.SortedCollection\end{ttfamily}(\ref{mi.rtl.Objects.Methods.Collection.SortedCollection})\item \begin{ttfamily}mi.rtl.Objects.Methods.Collection.SortedCollection.StringCollection\end{ttfamily}(\ref{mi.rtl.Objects.Methods.Collection.SortedCollection.StringCollection})\item \begin{ttfamily}mi.rtl.Objects.Methods.Collection.Sortedcollection.Stringcollection.Collectionstring\end{ttfamily}(\ref{mi.rtl.Objects.Methods.Collection.Sortedcollection.Stringcollection.Collectionstring})\item \begin{ttfamily}mi.rtl.Objects.Methods.Collection.SortedCollection.StrCollection\end{ttfamily}(\ref{mi.rtl.Objects.Methods.Collection.SortedCollection.StrCollection})\item \begin{ttfamily}mi.rtl.Objects.Methods.Db.Tb{\_}Access\end{ttfamily}(\ref{mi.rtl.Objects.Methods.Db.Tb_Access})\item \begin{ttfamily}mi.rtl.Objects.Methods.Db.Tb{\_}{\_}Access\end{ttfamily}(\ref{mi.rtl.Objects.Methods.Db.Tb__Access})\item \begin{ttfamily}mi.rtl.Objects.Methods.Db.Tb{\_}{\_}{\_}Access\end{ttfamily}(\ref{mi.rtl.Objects.Methods.Db.Tb___Access})\item \begin{ttfamily}mi.rtl.Objectss\end{ttfamily}(\ref{mi.rtl.Objectss})\item \begin{ttfamily}mi{\_}rtl{\_}ui{\_}types\end{ttfamily}(\ref{mi_rtl_ui_types})\item \begin{ttfamily}mi{\_}rtl{\_}ui{\_}consts\end{ttfamily}\item \begin{ttfamily}mi{\_}rtl{\_}ui{\_}methods\end{ttfamily}(\ref{mi_rtl_ui_methods})\item \begin{ttfamily}mi{\_}rtl{\_}ui{\_}DmxScroller{\_}Buttons\end{ttfamily}(\ref{mi_rtl_ui_DmxScroller_Buttons})\item \begin{ttfamily}mi{\_}rtl{\_}ui{\_}custom{\_}application\end{ttfamily}(\ref{mi_rtl_ui_custom_application})\item \begin{ttfamily}mi{\_}rtl{\_}ui{\_}Dmxscroller\end{ttfamily}(\ref{mi_rtl_ui_Dmxscroller})\item \begin{ttfamily}mi{\_}rtl{\_}ui{\_}dmxscroller{\_}form\end{ttfamily}(\ref{mi_rtl_ui_dmxscroller_form})\item \begin{ttfamily}LazarusPackageIntf\end{ttfamily}\end{itemize}
\chapter{Unit mi.rtl.ApplicationAbstract}
\section{Uses}
\begin{itemize}
\item \begin{ttfamily}Classes\end{ttfamily}\item \begin{ttfamily}SysUtils\end{ttfamily}\item \begin{ttfamily}CustApp\end{ttfamily}\end{itemize}
\section{Visão Geral}
\begin{description}
\item[\texttt{\begin{ttfamily}TApplicationAbstract\end{ttfamily} Classe}]
\end{description}
\section{Classes, Interfaces, Objetos e Registros}
\subsection*{TApplicationAbstract Classe}
\subsubsection*{\large{\textbf{Hierarquia}}\normalsize\hspace{1ex}\hfill}
TApplicationAbstract {$>$} TCustomApplication
%%%%Descrição
\chapter{Unit mi.rtl.Class{\_}Of{\_}Char}
\section{Uses}
\begin{itemize}
\item \begin{ttfamily}Classes\end{ttfamily}\item \begin{ttfamily}SysUtils\end{ttfamily}\end{itemize}
\section{Visão Geral}
\begin{description}
\item[\texttt{\begin{ttfamily}TClass{\_}Of{\_}Char\end{ttfamily} Classe}]
\end{description}
\section{Classes, Interfaces, Objetos e Registros}
\subsection*{TClass{\_}Of{\_}Char Classe}
\subsubsection*{\large{\textbf{Hierarquia}}\normalsize\hspace{1ex}\hfill}
TClass{\_}Of{\_}Char {$>$} TObject
\subsubsection*{\large{\textbf{Descrição}}\normalsize\hspace{1ex}\hfill}
A class \textbf{\begin{ttfamily}TClass{\_}Of{\_}Char\end{ttfamily}} é usada na tabela de caracter para conversão das letras com acentos\subsubsection*{\large{\textbf{Propriedades}}\normalsize\hspace{1ex}\hfill}
\paragraph*{Asc{\_}Ingles}\hspace*{\fill}

\begin{list}{}{
\settowidth{\tmplength}{\textbf{Declaração}}
\setlength{\itemindent}{0cm}
\setlength{\listparindent}{0cm}
\setlength{\leftmargin}{\evensidemargin}
\addtolength{\leftmargin}{\tmplength}
\settowidth{\labelsep}{X}
\addtolength{\leftmargin}{\labelsep}
\setlength{\labelwidth}{\tmplength}
}
\begin{flushleft}
\item[\textbf{Declaração}\hfill]
\begin{ttfamily}
public property Asc{\_}Ingles  : AnsiChar       Read {\_}Asc{\_}Ingles;\end{ttfamily}


\end{flushleft}
\end{list}
\paragraph*{Asc{\_}GUI}\hspace*{\fill}

\begin{list}{}{
\settowidth{\tmplength}{\textbf{Declaração}}
\setlength{\itemindent}{0cm}
\setlength{\listparindent}{0cm}
\setlength{\leftmargin}{\evensidemargin}
\addtolength{\leftmargin}{\tmplength}
\settowidth{\labelsep}{X}
\addtolength{\leftmargin}{\labelsep}
\setlength{\labelwidth}{\tmplength}
}
\begin{flushleft}
\item[\textbf{Declaração}\hfill]
\begin{ttfamily}
public property Asc{\_}GUI     : AnsiString       Read {\_}Asc{\_}GUI;\end{ttfamily}


\end{flushleft}
\end{list}
\paragraph*{Asc{\_}HTML}\hspace*{\fill}

\begin{list}{}{
\settowidth{\tmplength}{\textbf{Declaração}}
\setlength{\itemindent}{0cm}
\setlength{\listparindent}{0cm}
\setlength{\leftmargin}{\evensidemargin}
\addtolength{\leftmargin}{\tmplength}
\settowidth{\labelsep}{X}
\addtolength{\leftmargin}{\labelsep}
\setlength{\labelwidth}{\tmplength}
}
\begin{flushleft}
\item[\textbf{Declaração}\hfill]
\begin{ttfamily}
public property Asc{\_}HTML    : AnsiString Read {\_}Asc{\_}HTML;\end{ttfamily}


\end{flushleft}
\end{list}
\subsubsection*{\large{\textbf{Métodos}}\normalsize\hspace{1ex}\hfill}
\paragraph*{Create}\hspace*{\fill}

\begin{list}{}{
\settowidth{\tmplength}{\textbf{Declaração}}
\setlength{\itemindent}{0cm}
\setlength{\listparindent}{0cm}
\setlength{\leftmargin}{\evensidemargin}
\addtolength{\leftmargin}{\tmplength}
\settowidth{\labelsep}{X}
\addtolength{\leftmargin}{\labelsep}
\setlength{\labelwidth}{\tmplength}
}
\begin{flushleft}
\item[\textbf{Declaração}\hfill]
\begin{ttfamily}
public Constructor Create(aAsc{\_}Ingles : AnsiChar; aAsc{\_}GUI : AnsiString; aAsc{\_}HTML : AnsiString);\end{ttfamily}


\end{flushleft}
\end{list}
\chapter{Unit mi.rtl.Consts}
\section{Descrição}
\begin{itemize}
\item A Unit \textbf{\begin{ttfamily}mi.rtl.Consts\end{ttfamily}} reúne as constantes globais usados pelo pacote \textbf{\begin{ttfamily}mi.rtl\end{ttfamily}(\ref{mi.rtl})}. Esta unit foi testada nas plataformas: win32, win64 e linux.

\begin{itemize}
\item \textbf{VERSÃO} \begin{itemize}
\item Alpha {-} 0.5.0.687
\end{itemize}
\item \textbf{CÓDIGO FONTE}: \begin{itemize}
\item 
\end{itemize}
\item \textbf{HISTÓRICO} \begin{itemize}
\item Criado por: Paulo Sérgio da Silva Pacheco e{-}mail: paulosspacheco@yahoo.com.br \begin{itemize}
\item \textbf{13/11/2021} : Classe criada
\item \textbf{/16/11/2021} : \begin{itemize}
\item Em \textbf{TConsts.Initialization} executar: \begin{itemize}
\item \textbf{System.FileMode} := \textbf{\begin{ttfamily}TConsts.FileMode\end{ttfamily}(\ref{mi.rtl.Consts.TConsts-FileMode})};
\item Motivo: O mapa de Bits \textbf{System.FileMode} não permite acesso compartilhado.
\end{itemize}
\end{itemize}
\item \textbf{15/12/2021} \begin{itemize}
\item Criado a constante Identification = TIdentification.
\end{itemize}
\item \textbf{31/12/2021} \begin{itemize}
\item Criado a constante NRec e NRecAux para manter a compatibilidade com o passado.
\end{itemize}
\item **24/06/2022 \begin{itemize}
\item Criar constante FldLink
\end{itemize}
\end{itemize}
\end{itemize}
\end{itemize}
\end{itemize}
\section{Uses}
\begin{itemize}
\item \begin{ttfamily}Classes\end{ttfamily}\item \begin{ttfamily}SysUtils\end{ttfamily}\item \begin{ttfamily}process\end{ttfamily}\item \begin{ttfamily}mi.rtl.types\end{ttfamily}(\ref{mi.rtl.Types})\end{itemize}
\section{Visão Geral}
\begin{description}
\item[\texttt{\begin{ttfamily}TConsts\end{ttfamily} Classe}]
\end{description}
\section{Classes, Interfaces, Objetos e Registros}
\subsection*{TConsts Classe}
\subsubsection*{\large{\textbf{Hierarquia}}\normalsize\hspace{1ex}\hfill}
TConsts {$>$} \begin{ttfamily}TTypes\end{ttfamily}(\ref{mi.rtl.Types.TTypes}) {$>$} 
TComponent
\subsubsection*{\large{\textbf{Descrição}}\normalsize\hspace{1ex}\hfill}
A classe \textbf{\begin{ttfamily}TConsts\end{ttfamily}} declara todas as constantes globais do pacote MarIcarai\subsubsection*{\large{\textbf{Campos}}\normalsize\hspace{1ex}\hfill}
\paragraph*{ListaDeMsgErro}\hspace*{\fill}

\begin{list}{}{
\settowidth{\tmplength}{\textbf{Declaração}}
\setlength{\itemindent}{0cm}
\setlength{\listparindent}{0cm}
\setlength{\leftmargin}{\evensidemargin}
\addtolength{\leftmargin}{\tmplength}
\settowidth{\labelsep}{X}
\addtolength{\leftmargin}{\labelsep}
\setlength{\labelwidth}{\tmplength}
}
\begin{flushleft}
\item[\textbf{Declaração}\hfill]
\begin{ttfamily}
public const ListaDeMsgErro : TTypes.PSItem = nil;\end{ttfamily}


\end{flushleft}
\par
\item[\textbf{Descrição}]
Pilha com tStrings de erros.

\end{list}
\paragraph*{accNormal}\hspace*{\fill}

\begin{list}{}{
\settowidth{\tmplength}{\textbf{Declaração}}
\setlength{\itemindent}{0cm}
\setlength{\listparindent}{0cm}
\setlength{\leftmargin}{\evensidemargin}
\addtolength{\leftmargin}{\tmplength}
\settowidth{\labelsep}{X}
\addtolength{\leftmargin}{\labelsep}
\setlength{\labelwidth}{\tmplength}
}
\begin{flushleft}
\item[\textbf{Declaração}\hfill]
\begin{ttfamily}
public const accNormal      =    0;\end{ttfamily}


\end{flushleft}
\par
\item[\textbf{Descrição}]
A constante \textbf{\begin{ttfamily}accNormal\end{ttfamily}} (Const \begin{ttfamily}AccNormal\end{ttfamily} = 0;) é um mapa de bits usado para identificar o bit do campo \begin{ttfamily}TDmxFieldRec.access\end{ttfamily}(\ref{mi_rtl_ui_Dmxscroller.TDmxFieldRec-access}) que informa se que o campo pode ser editado.

\begin{itemize}
\item \textbf{EXEMPLO} \begin{itemize}
\item Como usar o mapa de bits \begin{ttfamily}accNormal\end{ttfamily} para saber se o campo pode ser editado.

\texttt{\\\nopagebreak[3]
\\\nopagebreak[3]
}\textbf{with}\texttt{~pDmxFieldRec{\^{}}~}\textbf{do}\texttt{\\\nopagebreak[3]
~~}\textbf{If}\texttt{~(access~}\textbf{and}\texttt{~accNormal~{$<$}{$>$}~0)\\\nopagebreak[3]
~~}\textbf{then}\texttt{~}\textbf{begin}\texttt{\\\nopagebreak[3]
~~~~~~~~~ShowMessage(Format('O~campo~{\%}s~pode~ser~editado'),[CharFieldName]);\\\nopagebreak[3]
~~~~~~~}\textbf{end}\texttt{;\\
}
\end{itemize}
\end{itemize}

\end{list}
\paragraph*{accReadOnly}\hspace*{\fill}

\begin{list}{}{
\settowidth{\tmplength}{\textbf{Declaração}}
\setlength{\itemindent}{0cm}
\setlength{\listparindent}{0cm}
\setlength{\leftmargin}{\evensidemargin}
\addtolength{\leftmargin}{\tmplength}
\settowidth{\labelsep}{X}
\addtolength{\leftmargin}{\labelsep}
\setlength{\labelwidth}{\tmplength}
}
\begin{flushleft}
\item[\textbf{Declaração}\hfill]
\begin{ttfamily}
public const accReadOnly  =  {\$}1;\end{ttfamily}


\end{flushleft}
\par
\item[\textbf{Descrição}]
A constante \textbf{\begin{ttfamily}accReadOnly\end{ttfamily}} (Const ReadOnly = 1;) é um mapa de bits usado para identificar o bit do campo \begin{ttfamily}TDmxFieldRec.access\end{ttfamily}(\ref{mi_rtl_ui_Dmxscroller.TDmxFieldRec-access}) que informa se o campo é somente para leitura.

\begin{itemize}
\item \textbf{EXEMPLO} \begin{itemize}
\item Como usar o mapa de bits ReadOnly para saber se o campo não pode ser editado.

\texttt{\\\nopagebreak[3]
\\\nopagebreak[3]
}\textbf{with}\texttt{~pDmxFieldRec{\^{}}~}\textbf{do}\texttt{\\\nopagebreak[3]
~~}\textbf{If}\texttt{~(access~}\textbf{and}\texttt{~ReadOnly~{$<$}{$>$}~0)\\\nopagebreak[3]
~~}\textbf{then}\texttt{~}\textbf{begin}\texttt{\\\nopagebreak[3]
~~~~~~~~~ShowMessage(Format('O~campo~{\%}s~não~pode~ser~editado'),[CharFieldName]);\\\nopagebreak[3]
~~~~~~~}\textbf{end}\texttt{;\\
}
\end{itemize}
\end{itemize}

\end{list}
\paragraph*{accHidden}\hspace*{\fill}

\begin{list}{}{
\settowidth{\tmplength}{\textbf{Declaração}}
\setlength{\itemindent}{0cm}
\setlength{\listparindent}{0cm}
\setlength{\leftmargin}{\evensidemargin}
\addtolength{\leftmargin}{\tmplength}
\settowidth{\labelsep}{X}
\addtolength{\leftmargin}{\labelsep}
\setlength{\labelwidth}{\tmplength}
}
\begin{flushleft}
\item[\textbf{Declaração}\hfill]
\begin{ttfamily}
public const accHidden      =  {\$}2;\end{ttfamily}


\end{flushleft}
\par
\item[\textbf{Descrição}]
A constante \textbf{\begin{ttfamily}accHidden\end{ttfamily}} (Const \begin{ttfamily}accHidden\end{ttfamily} = 2;) é um mapa de bits usado para identificar o bit do campo \begin{ttfamily}TDmxFieldRec.access\end{ttfamily}(\ref{mi_rtl_ui_Dmxscroller.TDmxFieldRec-access}) que informa se o mesmo é invisível.

\begin{itemize}
\item \textbf{EXEMPLO} \begin{itemize}
\item Como usar o mapa de bits \begin{ttfamily}accHidden\end{ttfamily} para saber se o campo é invisível.

\texttt{\\\nopagebreak[3]
\\\nopagebreak[3]
}\textbf{with}\texttt{~pDmxFieldRec{\^{}}~}\textbf{do}\texttt{\\\nopagebreak[3]
~~}\textbf{If}\texttt{~(access~}\textbf{and}\texttt{~accHidden~{$<$}{$>$}~0)\\\nopagebreak[3]
~~}\textbf{then}\texttt{~}\textbf{begin}\texttt{\\\nopagebreak[3]
~~~~~~ShowMessage(Format('O~campo~{\%}s~está~invisível'),[CharFieldName]);~\\\nopagebreak[3]
~~~~~~}\textbf{end}\texttt{;\\
}
\end{itemize}
\end{itemize}

\end{list}
\paragraph*{accSkip}\hspace*{\fill}

\begin{list}{}{
\settowidth{\tmplength}{\textbf{Declaração}}
\setlength{\itemindent}{0cm}
\setlength{\listparindent}{0cm}
\setlength{\leftmargin}{\evensidemargin}
\addtolength{\leftmargin}{\tmplength}
\settowidth{\labelsep}{X}
\addtolength{\leftmargin}{\labelsep}
\setlength{\labelwidth}{\tmplength}
}
\begin{flushleft}
\item[\textbf{Declaração}\hfill]
\begin{ttfamily}
public const accSkip  =  {\$}4;\end{ttfamily}


\end{flushleft}
\par
\item[\textbf{Descrição}]
A constante \textbf{\begin{ttfamily}accSkip\end{ttfamily}} (Const \begin{ttfamily}accSkip\end{ttfamily} = 4;) é um mapa de bits usado para identificar o bit do campo \begin{ttfamily}TDmxFieldRec.access\end{ttfamily}(\ref{mi_rtl_ui_Dmxscroller.TDmxFieldRec-access}) que informa se o campo pode receber o focus.

\begin{itemize}
\item \textbf{EXEMPLO} \begin{itemize}
\item Como usar o mapa de bits \begin{ttfamily}accSkip\end{ttfamily} para saber se o campo não pode receber o focus.

\texttt{\\\nopagebreak[3]
\\\nopagebreak[3]
}\textbf{with}\texttt{~pDmxFieldRec{\^{}}~}\textbf{do}\texttt{\\\nopagebreak[3]
~~}\textbf{If}\texttt{~(access~}\textbf{and}\texttt{~accSkip~{$<$}{$>$}~0)\\\nopagebreak[3]
~~}\textbf{then}\texttt{~}\textbf{begin}\texttt{\\\nopagebreak[3]
~~~~~~~~~ShowMessage(Format('O~campo~{\%}s~não~pode~receber~o~focus'),[CharFieldName]);\\\nopagebreak[3]
~~~~~~~}\textbf{end}\texttt{;\\
}
\end{itemize}
\end{itemize}

\end{list}
\paragraph*{accDelimiter}\hspace*{\fill}

\begin{list}{}{
\settowidth{\tmplength}{\textbf{Declaração}}
\setlength{\itemindent}{0cm}
\setlength{\listparindent}{0cm}
\setlength{\leftmargin}{\evensidemargin}
\addtolength{\leftmargin}{\tmplength}
\settowidth{\labelsep}{X}
\addtolength{\leftmargin}{\labelsep}
\setlength{\labelwidth}{\tmplength}
}
\begin{flushleft}
\item[\textbf{Declaração}\hfill]
\begin{ttfamily}
public const accDelimiter   =  {\$}8;\end{ttfamily}


\end{flushleft}
\par
\item[\textbf{Descrição}]
A constante \textbf{\begin{ttfamily}accDelimiter\end{ttfamily}} informa que o campo é delimitador de campos no Template.

\end{list}
\paragraph*{accExternal}\hspace*{\fill}

\begin{list}{}{
\settowidth{\tmplength}{\textbf{Declaração}}
\setlength{\itemindent}{0cm}
\setlength{\listparindent}{0cm}
\setlength{\leftmargin}{\evensidemargin}
\addtolength{\leftmargin}{\tmplength}
\settowidth{\labelsep}{X}
\addtolength{\leftmargin}{\labelsep}
\setlength{\labelwidth}{\tmplength}
}
\begin{flushleft}
\item[\textbf{Declaração}\hfill]
\begin{ttfamily}
public const accExternal    =  {\$}10;\end{ttfamily}


\end{flushleft}
\end{list}
\paragraph*{accSpecA}\hspace*{\fill}

\begin{list}{}{
\settowidth{\tmplength}{\textbf{Declaração}}
\setlength{\itemindent}{0cm}
\setlength{\listparindent}{0cm}
\setlength{\leftmargin}{\evensidemargin}
\addtolength{\leftmargin}{\tmplength}
\settowidth{\labelsep}{X}
\addtolength{\leftmargin}{\labelsep}
\setlength{\labelwidth}{\tmplength}
}
\begin{flushleft}
\item[\textbf{Declaração}\hfill]
\begin{ttfamily}
public const accSpecA       =  {\$}20;\end{ttfamily}


\end{flushleft}
\end{list}
\paragraph*{accSpecB}\hspace*{\fill}

\begin{list}{}{
\settowidth{\tmplength}{\textbf{Declaração}}
\setlength{\itemindent}{0cm}
\setlength{\listparindent}{0cm}
\setlength{\leftmargin}{\evensidemargin}
\addtolength{\leftmargin}{\tmplength}
\settowidth{\labelsep}{X}
\addtolength{\leftmargin}{\labelsep}
\setlength{\labelwidth}{\tmplength}
}
\begin{flushleft}
\item[\textbf{Declaração}\hfill]
\begin{ttfamily}
public const accSpecB       =  {\$}40;\end{ttfamily}


\end{flushleft}
\end{list}
\paragraph*{accSpecC}\hspace*{\fill}

\begin{list}{}{
\settowidth{\tmplength}{\textbf{Declaração}}
\setlength{\itemindent}{0cm}
\setlength{\listparindent}{0cm}
\setlength{\leftmargin}{\evensidemargin}
\addtolength{\leftmargin}{\tmplength}
\settowidth{\labelsep}{X}
\addtolength{\leftmargin}{\labelsep}
\setlength{\labelwidth}{\tmplength}
}
\begin{flushleft}
\item[\textbf{Declaração}\hfill]
\begin{ttfamily}
public const accSpecC       =  {\$}80;\end{ttfamily}


\end{flushleft}
\end{list}
\paragraph*{fldStr}\hspace*{\fill}

\begin{list}{}{
\settowidth{\tmplength}{\textbf{Declaração}}
\setlength{\itemindent}{0cm}
\setlength{\listparindent}{0cm}
\setlength{\leftmargin}{\evensidemargin}
\addtolength{\leftmargin}{\tmplength}
\settowidth{\labelsep}{X}
\addtolength{\leftmargin}{\labelsep}
\setlength{\labelwidth}{\tmplength}
}
\begin{flushleft}
\item[\textbf{Declaração}\hfill]
\begin{ttfamily}
public const fldStr              =   'S';\end{ttfamily}


\end{flushleft}
\par
\item[\textbf{Descrição}]
A constante \textbf{\begin{ttfamily}fldStr\end{ttfamily}} (Const \begin{ttfamily}fldStr\end{ttfamily} = 'S') usado na máscara do Template, informa ao componente \textbf{\begin{ttfamily}TUiDmxScroller\end{ttfamily}(\ref{mi_rtl_ui_Dmxscroller.TUiDmxScroller})} que a sequência de caracteres 'S' após o caractere \textbf{"{\textbackslash}"} representa no buffer do formulário um tipo ShortString que só aceita caractere maiúsculo.

\begin{itemize}
\item \textbf{EXEMPLO} \begin{itemize}
\item Representação de um string de 10 dígitos em um buffer de 11 bytes onde o byte zero contém o tamanho da string;

\texttt{\\\nopagebreak[3]
\\\nopagebreak[3]
}\textbf{Const}\texttt{\\\nopagebreak[3]
~~Nome~:=~'{\textbackslash}SSSSSSSSSS'\\
}
\end{itemize}
\end{itemize}

\end{list}
\paragraph*{fldS}\hspace*{\fill}

\begin{list}{}{
\settowidth{\tmplength}{\textbf{Declaração}}
\setlength{\itemindent}{0cm}
\setlength{\listparindent}{0cm}
\setlength{\leftmargin}{\evensidemargin}
\addtolength{\leftmargin}{\tmplength}
\settowidth{\labelsep}{X}
\addtolength{\leftmargin}{\labelsep}
\setlength{\labelwidth}{\tmplength}
}
\begin{flushleft}
\item[\textbf{Declaração}\hfill]
\begin{ttfamily}
public const fldS = fldStr;\end{ttfamily}


\end{flushleft}
\end{list}
\paragraph*{fldSTR{\_}Minuscula}\hspace*{\fill}

\begin{list}{}{
\settowidth{\tmplength}{\textbf{Declaração}}
\setlength{\itemindent}{0cm}
\setlength{\listparindent}{0cm}
\setlength{\leftmargin}{\evensidemargin}
\addtolength{\leftmargin}{\tmplength}
\settowidth{\labelsep}{X}
\addtolength{\leftmargin}{\labelsep}
\setlength{\labelwidth}{\tmplength}
}
\begin{flushleft}
\item[\textbf{Declaração}\hfill]
\begin{ttfamily}
public const fldSTR{\_}Minuscula    =   's';\end{ttfamily}


\end{flushleft}
\par
\item[\textbf{Descrição}]
A constante \textbf{\begin{ttfamily}fldSTR{\_}Minuscula\end{ttfamily}} (Const \begin{ttfamily}fldSTR{\_}Minuscula\end{ttfamily} = 's') usado na máscara do Template, informa ao componente \textbf{\begin{ttfamily}TUiDmxScroller\end{ttfamily}(\ref{mi_rtl_ui_Dmxscroller.TUiDmxScroller})} que a sequência de caracteres 's' após o caractere \textbf{"{\textbackslash}"} representa no buffer do formulário um tipo ShortString que só aceita caractere minúscula.

\begin{itemize}
\item \textbf{EXEMPLO} \begin{itemize}
\item Representação de um string de 10 dígitos em um buffer de 11 bytes onde o byte zero contém o tamanho da string;

\texttt{\\\nopagebreak[3]
\\\nopagebreak[3]
}\textbf{Const}\texttt{\\\nopagebreak[3]
~~Nome~:=~'{\textbackslash}ssssssssss'~\textit{//paulosergi}\\\nopagebreak[3]
~~Nome~:=~'{\textbackslash}Ssssssssss'~\textit{//Paulo~serg}\\
}
\end{itemize}
\end{itemize}

\end{list}
\paragraph*{fldSMi}\hspace*{\fill}

\begin{list}{}{
\settowidth{\tmplength}{\textbf{Declaração}}
\setlength{\itemindent}{0cm}
\setlength{\listparindent}{0cm}
\setlength{\leftmargin}{\evensidemargin}
\addtolength{\leftmargin}{\tmplength}
\settowidth{\labelsep}{X}
\addtolength{\leftmargin}{\labelsep}
\setlength{\labelwidth}{\tmplength}
}
\begin{flushleft}
\item[\textbf{Declaração}\hfill]
\begin{ttfamily}
public const fldSMi = fldSTR{\_}Minuscula;\end{ttfamily}


\end{flushleft}
\end{list}
\paragraph*{fldSTRNUM}\hspace*{\fill}

\begin{list}{}{
\settowidth{\tmplength}{\textbf{Declaração}}
\setlength{\itemindent}{0cm}
\setlength{\listparindent}{0cm}
\setlength{\leftmargin}{\evensidemargin}
\addtolength{\leftmargin}{\tmplength}
\settowidth{\labelsep}{X}
\addtolength{\leftmargin}{\labelsep}
\setlength{\labelwidth}{\tmplength}
}
\begin{flushleft}
\item[\textbf{Declaração}\hfill]
\begin{ttfamily}
public const fldSTRNUM           =   '{\#}';\end{ttfamily}


\end{flushleft}
\par
\item[\textbf{Descrição}]
A constante \textbf{\begin{ttfamily}fldSTRNUM\end{ttfamily}} (Const \begin{ttfamily}fldSTRNUM\end{ttfamily} = '{\#}') usado na máscara do Template, informa ao componente \textbf{\begin{ttfamily}TUiDmxScroller\end{ttfamily}(\ref{mi_rtl_ui_Dmxscroller.TUiDmxScroller})} que a sequência de caracteres '{\#}' após o caractere \textbf{"{\textbackslash}"} representa no buffer do formulário um tipo ShortString que só aceita caractere numérico.

\begin{itemize}
\item \textbf{EXEMPLO} \begin{itemize}
\item Representação de um string de 11 dígitos em um buffer de 12 bytes onde o byte zero contém o tamanho da string;

\texttt{\\\nopagebreak[3]
\\\nopagebreak[3]
}\textbf{Const}\texttt{\\\nopagebreak[3]
~~telefone~:=~'{\textbackslash}({\#}{\#})~{\#}~{\#}{\#}{\#}{\#}-{\#}{\#}{\#}{\#}'~\textit{//85~9~9702~4498}\\
}
\end{itemize}
\end{itemize}

\end{list}
\paragraph*{fldSN}\hspace*{\fill}

\begin{list}{}{
\settowidth{\tmplength}{\textbf{Declaração}}
\setlength{\itemindent}{0cm}
\setlength{\listparindent}{0cm}
\setlength{\leftmargin}{\evensidemargin}
\addtolength{\leftmargin}{\tmplength}
\settowidth{\labelsep}{X}
\addtolength{\leftmargin}{\labelsep}
\setlength{\labelwidth}{\tmplength}
}
\begin{flushleft}
\item[\textbf{Declaração}\hfill]
\begin{ttfamily}
public const fldSN = fldSTRNUM;\end{ttfamily}


\end{flushleft}
\end{list}
\paragraph*{fldAnsiChar}\hspace*{\fill}

\begin{list}{}{
\settowidth{\tmplength}{\textbf{Declaração}}
\setlength{\itemindent}{0cm}
\setlength{\listparindent}{0cm}
\setlength{\leftmargin}{\evensidemargin}
\addtolength{\leftmargin}{\tmplength}
\settowidth{\labelsep}{X}
\addtolength{\leftmargin}{\labelsep}
\setlength{\labelwidth}{\tmplength}
}
\begin{flushleft}
\item[\textbf{Declaração}\hfill]
\begin{ttfamily}
public const fldAnsiChar             =   'C';\end{ttfamily}


\end{flushleft}
\par
\item[\textbf{Descrição}]
A constante \textbf{\begin{ttfamily}fldAnsiChar\end{ttfamily}} (Const \begin{ttfamily}fldAnsiChar\end{ttfamily} = 'C') usado na máscara do Template, informa ao componente \textbf{\begin{ttfamily}TUiDmxScroller\end{ttfamily}(\ref{mi_rtl_ui_Dmxscroller.TUiDmxScroller})} que a sequência de caracteres 'C' após o caractere \textbf{"{\textbackslash}"} representa no buffer do formulário um tipo AnsiString que só aceita caractere maiúsculo.

\begin{itemize}
\item \textbf{EXEMPLO} \begin{itemize}
\item Representação de um AnsiString de 10 dígitos em um buffer de 11 bytes onde o ultimo byte contém o caractere {\#}0 informando o fim da string;

\texttt{\\\nopagebreak[3]
\\\nopagebreak[3]
}\textbf{Const}\texttt{\\\nopagebreak[3]
~~Nome~:=~'{\textbackslash}CCCCCCCCCC';~\textit{//PAULO~SÉRG}\\
}
\end{itemize}
\end{itemize}

\end{list}
\paragraph*{fldAC}\hspace*{\fill}

\begin{list}{}{
\settowidth{\tmplength}{\textbf{Declaração}}
\setlength{\itemindent}{0cm}
\setlength{\listparindent}{0cm}
\setlength{\leftmargin}{\evensidemargin}
\addtolength{\leftmargin}{\tmplength}
\settowidth{\labelsep}{X}
\addtolength{\leftmargin}{\labelsep}
\setlength{\labelwidth}{\tmplength}
}
\begin{flushleft}
\item[\textbf{Declaração}\hfill]
\begin{ttfamily}
public const fldAC = fldAnsiChar;\end{ttfamily}


\end{flushleft}
\end{list}
\paragraph*{fldAnsiChar{\_}Minuscula}\hspace*{\fill}

\begin{list}{}{
\settowidth{\tmplength}{\textbf{Declaração}}
\setlength{\itemindent}{0cm}
\setlength{\listparindent}{0cm}
\setlength{\leftmargin}{\evensidemargin}
\addtolength{\leftmargin}{\tmplength}
\settowidth{\labelsep}{X}
\addtolength{\leftmargin}{\labelsep}
\setlength{\labelwidth}{\tmplength}
}
\begin{flushleft}
\item[\textbf{Declaração}\hfill]
\begin{ttfamily}
public const fldAnsiChar{\_}Minuscula   =   'c';\end{ttfamily}


\end{flushleft}
\par
\item[\textbf{Descrição}]
A constante \textbf{\begin{ttfamily}fldAnsiChar{\_}Minuscula\end{ttfamily}} (Const \begin{ttfamily}fldAnsiChar\end{ttfamily}(\ref{mi.rtl.Consts.TConsts-fldAnsiChar}) = 'c') usado na máscara do Template, informa ao componente \textbf{\begin{ttfamily}TUiDmxScroller\end{ttfamily}(\ref{mi_rtl_ui_Dmxscroller.TUiDmxScroller})} que a sequência de caracteres 'c' após o caractere \textbf{"{\textbackslash}"} representa no buffer do formulário um tipo AnsiString que só aceita caractere minúsculo.

\begin{itemize}
\item \textbf{EXEMPLO} \begin{itemize}
\item Representação de um AnsiString de 10 dígitos em um buffer de 11 bytes onde o ultimo byte contém o caractere {\#}0 informando o fim da string;

\texttt{\\\nopagebreak[3]
\\\nopagebreak[3]
}\textbf{Const}\texttt{\\\nopagebreak[3]
~~Nome~:=~'{\textbackslash}cccccccccc';~\textit{//paulo~Sérg}\\\nopagebreak[3]
~~Nome~:=~'{\textbackslash}Cccccccccc';~\textit{//Paulo~Sérg}\\
}
\end{itemize}
\end{itemize}

\end{list}
\paragraph*{fldACMi}\hspace*{\fill}

\begin{list}{}{
\settowidth{\tmplength}{\textbf{Declaração}}
\setlength{\itemindent}{0cm}
\setlength{\listparindent}{0cm}
\setlength{\leftmargin}{\evensidemargin}
\addtolength{\leftmargin}{\tmplength}
\settowidth{\labelsep}{X}
\addtolength{\leftmargin}{\labelsep}
\setlength{\labelwidth}{\tmplength}
}
\begin{flushleft}
\item[\textbf{Declaração}\hfill]
\begin{ttfamily}
public const fldACMi = fldAnsiChar{\_}Minuscula;\end{ttfamily}


\end{flushleft}
\end{list}
\paragraph*{fldAnsiCharNUM}\hspace*{\fill}

\begin{list}{}{
\settowidth{\tmplength}{\textbf{Declaração}}
\setlength{\itemindent}{0cm}
\setlength{\listparindent}{0cm}
\setlength{\leftmargin}{\evensidemargin}
\addtolength{\leftmargin}{\tmplength}
\settowidth{\labelsep}{X}
\addtolength{\leftmargin}{\labelsep}
\setlength{\labelwidth}{\tmplength}
}
\begin{flushleft}
\item[\textbf{Declaração}\hfill]
\begin{ttfamily}
public const fldAnsiCharNUM          =   '0';\end{ttfamily}


\end{flushleft}
\par
\item[\textbf{Descrição}]
A constante \textbf{\begin{ttfamily}fldAnsiCharNUM\end{ttfamily}} (Const \begin{ttfamily}fldAnsiChar\end{ttfamily}(\ref{mi.rtl.Consts.TConsts-fldAnsiChar}) = '0') usado na máscara do Template, informa ao componente \textbf{\begin{ttfamily}TUiDmxScroller\end{ttfamily}(\ref{mi_rtl_ui_Dmxscroller.TUiDmxScroller})} que a sequência de caracteres '0' após o caractere \textbf{"{\textbackslash}"} representa no buffer do formulário um tipo AnsiString que só aceita caractere numérico ['0'..'9']] .

\begin{itemize}
\item \textbf{EXEMPLO} \begin{itemize}
\item Representação de um AnsiString de 11 dígitos em um buffer de 12 bytes onde o ultimo byte contém o caractere {\#}0 informando o fim da string;

\texttt{\\\nopagebreak[3]
\\\nopagebreak[3]
}\textbf{Const}\texttt{\\\nopagebreak[3]
\\\nopagebreak[3]
~~telefone~:=~'{\textbackslash}(00)~0~0000-0000'~\textit{//85~9~9702~4498}\\
}
\end{itemize}
\end{itemize}

\end{list}
\paragraph*{fldACN}\hspace*{\fill}

\begin{list}{}{
\settowidth{\tmplength}{\textbf{Declaração}}
\setlength{\itemindent}{0cm}
\setlength{\listparindent}{0cm}
\setlength{\leftmargin}{\evensidemargin}
\addtolength{\leftmargin}{\tmplength}
\settowidth{\labelsep}{X}
\addtolength{\leftmargin}{\labelsep}
\setlength{\labelwidth}{\tmplength}
}
\begin{flushleft}
\item[\textbf{Declaração}\hfill]
\begin{ttfamily}
public const fldACN = fldAnsiCharNUM;\end{ttfamily}


\end{flushleft}
\end{list}
\paragraph*{fldAnsiCharVAL}\hspace*{\fill}

\begin{list}{}{
\settowidth{\tmplength}{\textbf{Declaração}}
\setlength{\itemindent}{0cm}
\setlength{\listparindent}{0cm}
\setlength{\leftmargin}{\evensidemargin}
\addtolength{\leftmargin}{\tmplength}
\settowidth{\labelsep}{X}
\addtolength{\leftmargin}{\labelsep}
\setlength{\labelwidth}{\tmplength}
}
\begin{flushleft}
\item[\textbf{Declaração}\hfill]
\begin{ttfamily}
public const fldAnsiCharVAL          =   'N';\end{ttfamily}


\end{flushleft}
\par
\item[\textbf{Descrição}]
A constante \textbf{\begin{ttfamily}fldAnsiCharVAL\end{ttfamily}} (Const \begin{ttfamily}fldAnsiChar\end{ttfamily}(\ref{mi.rtl.Consts.TConsts-fldAnsiChar}) = '0') usado na máscara do Template, informa ao componente \textbf{\begin{ttfamily}TUiDmxScroller\end{ttfamily}(\ref{mi_rtl_ui_Dmxscroller.TUiDmxScroller})} que a sequência de caracteres '0' após o caractere \textbf{"{\textbackslash}"} representa no buffer do formulário um tipo AnsiString que só aceita caractere numérico ['0'..'9']] com formatação dbase.

\begin{itemize}
\item \textbf{EXEMPLO} \begin{itemize}
\item Representação de um AnsiString de 11 dígitos em um buffer de 12 bytes onde o ultimo byte contém o caractere {\#}0 informando o fim da string;

\texttt{\\\nopagebreak[3]
\\\nopagebreak[3]
}\textbf{Const}\texttt{\\\nopagebreak[3]
\\\nopagebreak[3]
~~telefone~:=~'{\textbackslash}(NN)~N~NNNN-NNNN'~\textit{//85~9~9702~4498}\\
}
\end{itemize}
\end{itemize}

\end{list}
\paragraph*{fldBYTE}\hspace*{\fill}

\begin{list}{}{
\settowidth{\tmplength}{\textbf{Declaração}}
\setlength{\itemindent}{0cm}
\setlength{\listparindent}{0cm}
\setlength{\leftmargin}{\evensidemargin}
\addtolength{\leftmargin}{\tmplength}
\settowidth{\labelsep}{X}
\addtolength{\leftmargin}{\labelsep}
\setlength{\labelwidth}{\tmplength}
}
\begin{flushleft}
\item[\textbf{Declaração}\hfill]
\begin{ttfamily}
public const fldBYTE             =   'B';\end{ttfamily}


\end{flushleft}
\par
\item[\textbf{Descrição}]
byte Field

\end{list}
\paragraph*{fldSHORTINT}\hspace*{\fill}

\begin{list}{}{
\settowidth{\tmplength}{\textbf{Declaração}}
\setlength{\itemindent}{0cm}
\setlength{\listparindent}{0cm}
\setlength{\leftmargin}{\evensidemargin}
\addtolength{\leftmargin}{\tmplength}
\settowidth{\labelsep}{X}
\addtolength{\leftmargin}{\labelsep}
\setlength{\labelwidth}{\tmplength}
}
\begin{flushleft}
\item[\textbf{Declaração}\hfill]
\begin{ttfamily}
public const fldSHORTINT         =   'J';\end{ttfamily}


\end{flushleft}
\par
\item[\textbf{Descrição}]
shortint Field

\end{list}
\paragraph*{fldSmallWORD}\hspace*{\fill}

\begin{list}{}{
\settowidth{\tmplength}{\textbf{Declaração}}
\setlength{\itemindent}{0cm}
\setlength{\listparindent}{0cm}
\setlength{\leftmargin}{\evensidemargin}
\addtolength{\leftmargin}{\tmplength}
\settowidth{\labelsep}{X}
\addtolength{\leftmargin}{\labelsep}
\setlength{\labelwidth}{\tmplength}
}
\begin{flushleft}
\item[\textbf{Declaração}\hfill]
\begin{ttfamily}
public const fldSmallWORD        =   'W';\end{ttfamily}


\end{flushleft}
\par
\item[\textbf{Descrição}]
\begin{ttfamily}word\end{ttfamily}(\ref{mi.rtl.Types.TTypes-Word}) Field NortSoft

\end{list}
\paragraph*{fldSmallInt}\hspace*{\fill}

\begin{list}{}{
\settowidth{\tmplength}{\textbf{Declaração}}
\setlength{\itemindent}{0cm}
\setlength{\listparindent}{0cm}
\setlength{\leftmargin}{\evensidemargin}
\addtolength{\leftmargin}{\tmplength}
\settowidth{\labelsep}{X}
\addtolength{\leftmargin}{\labelsep}
\setlength{\labelwidth}{\tmplength}
}
\begin{flushleft}
\item[\textbf{Declaração}\hfill]
\begin{ttfamily}
public const fldSmallInt         =   'I';\end{ttfamily}


\end{flushleft}
\par
\item[\textbf{Descrição}]
\begin{ttfamily}integer\end{ttfamily}(\ref{mi.rtl.Types.TTypes-Integer}) Field NortSoft

\end{list}
\paragraph*{fldLONGINT}\hspace*{\fill}

\begin{list}{}{
\settowidth{\tmplength}{\textbf{Declaração}}
\setlength{\itemindent}{0cm}
\setlength{\listparindent}{0cm}
\setlength{\leftmargin}{\evensidemargin}
\addtolength{\leftmargin}{\tmplength}
\settowidth{\labelsep}{X}
\addtolength{\leftmargin}{\labelsep}
\setlength{\labelwidth}{\tmplength}
}
\begin{flushleft}
\item[\textbf{Declaração}\hfill]
\begin{ttfamily}
public const fldLONGINT          =   'L';\end{ttfamily}


\end{flushleft}
\par
\item[\textbf{Descrição}]
\begin{ttfamily}longint\end{ttfamily}(\ref{mi.rtl.Types.TTypes-LongInt}) Field

\end{list}
\paragraph*{fldRealNum}\hspace*{\fill}

\begin{list}{}{
\settowidth{\tmplength}{\textbf{Declaração}}
\setlength{\itemindent}{0cm}
\setlength{\listparindent}{0cm}
\setlength{\leftmargin}{\evensidemargin}
\addtolength{\leftmargin}{\tmplength}
\settowidth{\labelsep}{X}
\addtolength{\leftmargin}{\labelsep}
\setlength{\labelwidth}{\tmplength}
}
\begin{flushleft}
\item[\textbf{Declaração}\hfill]
\begin{ttfamily}
public const fldRealNum          =   'R';\end{ttfamily}


\end{flushleft}
\par
\item[\textbf{Descrição}]
\begin{ttfamily}real\end{ttfamily}(\ref{mi.rtl.Types.TTypes-Real}) number Field (uses \begin{ttfamily}TRealNum\end{ttfamily}(\ref{mi.rtl.Types.TTypes-TRealNum}))

\end{list}
\paragraph*{fldRealNum{\_}Positivo}\hspace*{\fill}

\begin{list}{}{
\settowidth{\tmplength}{\textbf{Declaração}}
\setlength{\itemindent}{0cm}
\setlength{\listparindent}{0cm}
\setlength{\leftmargin}{\evensidemargin}
\addtolength{\leftmargin}{\tmplength}
\settowidth{\labelsep}{X}
\addtolength{\leftmargin}{\labelsep}
\setlength{\labelwidth}{\tmplength}
}
\begin{flushleft}
\item[\textbf{Declaração}\hfill]
\begin{ttfamily}
public const fldRealNum{\_}Positivo =   'r';\end{ttfamily}


\end{flushleft}
\par
\item[\textbf{Descrição}]
\begin{ttfamily}real\end{ttfamily}(\ref{mi.rtl.Types.TTypes-Real}) number Field positive (uses \begin{ttfamily}TRealNum\end{ttfamily}(\ref{mi.rtl.Types.TTypes-TRealNum}))

\end{list}
\paragraph*{fldBoolean}\hspace*{\fill}

\begin{list}{}{
\settowidth{\tmplength}{\textbf{Declaração}}
\setlength{\itemindent}{0cm}
\setlength{\listparindent}{0cm}
\setlength{\leftmargin}{\evensidemargin}
\addtolength{\leftmargin}{\tmplength}
\settowidth{\labelsep}{X}
\addtolength{\leftmargin}{\labelsep}
\setlength{\labelwidth}{\tmplength}
}
\begin{flushleft}
\item[\textbf{Declaração}\hfill]
\begin{ttfamily}
public const fldBoolean          =   'X';\end{ttfamily}


\end{flushleft}
\par
\item[\textbf{Descrição}]
A constante \textbf{\begin{ttfamily}fldBoolean\end{ttfamily}} (\begin{ttfamily}fldBoolean\end{ttfamily} = 'X') indica que o campo é do tipo byte e só pode ter dois valores.

\begin{itemize}
\item \textbf{NOTA} \begin{itemize}
\item Valores possíveis: \begin{itemize}
\item 0 {-} False; não
\item 1 = True; sim
\end{itemize}
\item A forma de editá{-}los deve ser com o componente checkbox.
\end{itemize}
\item \textbf{EXEMPLO}

\texttt{\\\nopagebreak[3]
\\\nopagebreak[3]
}\textbf{Resourcestring}\texttt{\\\nopagebreak[3]
~~tmp{\_}Aceita~=~'{\textbackslash}X~Aceita~o~contrato~+ChFN+'Aceita{\_}contrato'+CharHint+'Aceita~os~termos~}\textbf{do}\texttt{~contrato?';\\\nopagebreak[3]
~~Template~=~tmp{\_}Aceita+'~Aceita~os~termos~}\textbf{do}\texttt{~contrato~\\
}
\end{itemize}

\end{list}
\paragraph*{fldHexValue}\hspace*{\fill}

\begin{list}{}{
\settowidth{\tmplength}{\textbf{Declaração}}
\setlength{\itemindent}{0cm}
\setlength{\listparindent}{0cm}
\setlength{\leftmargin}{\evensidemargin}
\addtolength{\leftmargin}{\tmplength}
\settowidth{\labelsep}{X}
\addtolength{\leftmargin}{\labelsep}
\setlength{\labelwidth}{\tmplength}
}
\begin{flushleft}
\item[\textbf{Declaração}\hfill]
\begin{ttfamily}
public const fldHexValue         =   'H';\end{ttfamily}


\end{flushleft}
\par
\item[\textbf{Descrição}]
hexadecimal numeric entry

\end{list}
\paragraph*{CharUpperlimit}\hspace*{\fill}

\begin{list}{}{
\settowidth{\tmplength}{\textbf{Declaração}}
\setlength{\itemindent}{0cm}
\setlength{\listparindent}{0cm}
\setlength{\leftmargin}{\evensidemargin}
\addtolength{\leftmargin}{\tmplength}
\settowidth{\labelsep}{X}
\addtolength{\leftmargin}{\labelsep}
\setlength{\labelwidth}{\tmplength}
}
\begin{flushleft}
\item[\textbf{Declaração}\hfill]
\begin{ttfamily}
public const CharUpperlimit       =   {\^{}}U ;\end{ttfamily}


\end{flushleft}
\par
\item[\textbf{Descrição}]
Limite superior do campo (Somente 1 a 255)

\end{list}
\paragraph*{fldENUM}\hspace*{\fill}

\begin{list}{}{
\settowidth{\tmplength}{\textbf{Declaração}}
\setlength{\itemindent}{0cm}
\setlength{\listparindent}{0cm}
\setlength{\leftmargin}{\evensidemargin}
\addtolength{\leftmargin}{\tmplength}
\settowidth{\labelsep}{X}
\addtolength{\leftmargin}{\labelsep}
\setlength{\labelwidth}{\tmplength}
}
\begin{flushleft}
\item[\textbf{Declaração}\hfill]
\begin{ttfamily}
public const fldENUM      =  {\^{}}E;\end{ttfamily}


\end{flushleft}
\par
\item[\textbf{Descrição}]
enumerated Field

\end{list}
\paragraph*{fldBLOb}\hspace*{\fill}

\begin{list}{}{
\settowidth{\tmplength}{\textbf{Declaração}}
\setlength{\itemindent}{0cm}
\setlength{\listparindent}{0cm}
\setlength{\leftmargin}{\evensidemargin}
\addtolength{\leftmargin}{\tmplength}
\settowidth{\labelsep}{X}
\addtolength{\leftmargin}{\labelsep}
\setlength{\labelwidth}{\tmplength}
}
\begin{flushleft}
\item[\textbf{Declaração}\hfill]
\begin{ttfamily}
public const fldBLOb             =   {\^{}}M;\end{ttfamily}


\end{flushleft}
\par
\item[\textbf{Descrição}]
A constante \textbf{\begin{ttfamily}fldBLOb\end{ttfamily}} indica que o campo é não formatado podendo ser um Record, porém a edição do mesmo será feito por outros meios.

\begin{itemize}
\item \textbf{NOTA} \begin{itemize}
\item Para informar ao buffer do registro que o campo é \textbf{\begin{ttfamily}fldBLOb\end{ttfamily}}, a função \textbf{CreateBlobField} é necessário.
\item A \textbf{class function \begin{ttfamily}TUiMethods.CreateBlobField\end{ttfamily}(\ref{mi_rtl_ui_methods.TUiMethods-CreateBlobField})(Len: \begin{ttfamily}integer\end{ttfamily}(\ref{mi.rtl.Types.TTypes-Integer}); AccMode,Default: byte) : DmxIDstr;} reserva espaço para o mesmo.
\item Pendência: Preciso criar um exemplo de uso deste tipo de informação.
\end{itemize}
\end{itemize}

\end{list}
\paragraph*{FldRadioButton}\hspace*{\fill}

\begin{list}{}{
\settowidth{\tmplength}{\textbf{Declaração}}
\setlength{\itemindent}{0cm}
\setlength{\listparindent}{0cm}
\setlength{\leftmargin}{\evensidemargin}
\addtolength{\leftmargin}{\tmplength}
\settowidth{\labelsep}{X}
\addtolength{\leftmargin}{\labelsep}
\setlength{\labelwidth}{\tmplength}
}
\begin{flushleft}
\item[\textbf{Declaração}\hfill]
\begin{ttfamily}
public const FldRadioButton      =  'K';\end{ttfamily}


\end{flushleft}
\par
\item[\textbf{Descrição}]
O tipo do campo \textbf{\begin{ttfamily}FldRadioButton\end{ttfamily}} é um campo tipo \begin{ttfamily}TCluster\end{ttfamily}(\ref{mi.rtl.Types.TTypes-TCluster}) e é representado no Template em um controle TRadioButton

\begin{itemize}
\item \textbf{NOTAS} \begin{itemize}
\item Um Template pode conter vários campos do tipo cluster e o mesmo é identificado após a sequencia {\textbackslash}K? onde ? indica que a informação pertence ao campo ? \begin{itemize}
\item Exemplo: \begin{itemize}
\item SEXO \begin{itemize}
\item {\textbackslash}Ka Masculino
\item {\textbackslash}Ka Feminino
\item {\textbackslash}Ka Indefinido
\end{itemize}
\end{itemize}
\end{itemize}
\item Os campos clusteres possuem o mesmo número do campo e na primeira ocorrência contém o nome do campo na lista \begin{ttfamily}pDmxFieldRec\end{ttfamily}(\ref{mi_rtl_ui_Dmxscroller-pDmxFieldRec}).
\end{itemize}
\item \textbf{EXEMPLO}

\texttt{\\\nopagebreak[3]
Result~:=\\\nopagebreak[3]
~~NewSItem('~~~SEXO~',\\\nopagebreak[3]
~~NewSItem('~~~~{\textbackslash}Ka~Masculino',\\\nopagebreak[3]
~~NewSItem('~~~~{\textbackslash}Ka~Feminino',\\\nopagebreak[3]
~~NewSItem('~~~~{\textbackslash}Ka~Indefinido',\\\nopagebreak[3]
~~NewSItem('~~~ESTADO~CIVIL~',\\\nopagebreak[3]
~~NewSItem('~~~~{\textbackslash}Kb~Solteiro',\\\nopagebreak[3]
~~NewSItem('~~~~{\textbackslash}Kb~Casado',\\\nopagebreak[3]
~~NewSItem('~~~~{\textbackslash}Kb~Divorciado',\\\nopagebreak[3]
~~}\textbf{nil}\texttt{))))))))\\
}
\end{itemize}

\end{list}
\paragraph*{FldDbRadioButton}\hspace*{\fill}

\begin{list}{}{
\settowidth{\tmplength}{\textbf{Declaração}}
\setlength{\itemindent}{0cm}
\setlength{\listparindent}{0cm}
\setlength{\leftmargin}{\evensidemargin}
\addtolength{\leftmargin}{\tmplength}
\settowidth{\labelsep}{X}
\addtolength{\leftmargin}{\labelsep}
\setlength{\labelwidth}{\tmplength}
}
\begin{flushleft}
\item[\textbf{Declaração}\hfill]
\begin{ttfamily}
public const FldDbRadioButton    =  'k';\end{ttfamily}


\end{flushleft}
\par
\item[\textbf{Descrição}]
O tipo do campo \textbf{\begin{ttfamily}FldDbRadioButton\end{ttfamily}} é um campo tipo String e é representado no Template em controle TDbRadioButton

\begin{itemize}
\item \textbf{NOTAS} \begin{itemize}
\item Um Template pode conter vários campos do tipo DbRaidoButton e o mesmo é identificado após a sequencia {\textbackslash}k? onde ? indica que a informação pertence ao campo ? \begin{itemize}
\item Exemplo: \begin{itemize}
\item SEXO \begin{itemize}
\item {\textbackslash}ka Masculino
\item {\textbackslash}ka Feminino
\item {\textbackslash}ka Indefinido
\end{itemize}
\end{itemize}
\end{itemize}
\item Os campos DbRadioButton possuem o mesmo número do campo e na primeira ocorrência contém o nome do campo na lista \begin{ttfamily}pDmxFieldRec\end{ttfamily}(\ref{mi_rtl_ui_Dmxscroller-pDmxFieldRec}).
\item O motivo pelo qual \textbf{\begin{ttfamily}FldDbRadioButton\end{ttfamily}} foi criado é que o banco de dados do freepascal reconhece esse tipo como string com o nome do caption selecionado.
\item O tamanho da string deve ser o tamanho da maior string da lista de opções.
\end{itemize}
\end{itemize}

\end{list}
\paragraph*{fldZEROMOD}\hspace*{\fill}

\begin{list}{}{
\settowidth{\tmplength}{\textbf{Declaração}}
\setlength{\itemindent}{0cm}
\setlength{\listparindent}{0cm}
\setlength{\leftmargin}{\evensidemargin}
\addtolength{\leftmargin}{\tmplength}
\settowidth{\labelsep}{X}
\addtolength{\leftmargin}{\labelsep}
\setlength{\labelwidth}{\tmplength}
}
\begin{flushleft}
\item[\textbf{Declaração}\hfill]
\begin{ttfamily}
public const fldZEROMOD          =   'Z';\end{ttfamily}


\end{flushleft}
\par
\item[\textbf{Descrição}]
zero modifier

\end{list}
\paragraph*{fldCONTRACTION}\hspace*{\fill}

\begin{list}{}{
\settowidth{\tmplength}{\textbf{Declaração}}
\setlength{\itemindent}{0cm}
\setlength{\listparindent}{0cm}
\setlength{\leftmargin}{\evensidemargin}
\addtolength{\leftmargin}{\tmplength}
\settowidth{\labelsep}{X}
\addtolength{\leftmargin}{\labelsep}
\setlength{\labelwidth}{\tmplength}
}
\begin{flushleft}
\item[\textbf{Declaração}\hfill]
\begin{ttfamily}
public const fldCONTRACTION      =   '`';\end{ttfamily}


\end{flushleft}
\par
\item[\textbf{Descrição}]
A constante \textbf{\begin{ttfamily}fldCONTRACTION\end{ttfamily}} omite da visão do usuário a parte do campo que não precisa ser mostrado, ou seja: limita a parte visível do texto permitindo scroll lateral do mesmo.

\end{list}
\paragraph*{fldAPPEND}\hspace*{\fill}

\begin{list}{}{
\settowidth{\tmplength}{\textbf{Declaração}}
\setlength{\itemindent}{0cm}
\setlength{\listparindent}{0cm}
\setlength{\leftmargin}{\evensidemargin}
\addtolength{\leftmargin}{\tmplength}
\settowidth{\labelsep}{X}
\addtolength{\leftmargin}{\labelsep}
\setlength{\labelwidth}{\tmplength}
}
\begin{flushleft}
\item[\textbf{Declaração}\hfill]
\begin{ttfamily}
public const fldAPPEND           =   {\^{}}G;\end{ttfamily}


\end{flushleft}
\par
\item[\textbf{Descrição}]
A constante \textbf{\begin{ttfamily}fldAPPEND\end{ttfamily}} é usada para concatenar duas listas do tipo \begin{ttfamily}PSItem\end{ttfamily}(\ref{mi.rtl.Types.TTypes-PSItem}).

\begin{itemize}
\item A constante \textbf{\begin{ttfamily}fldAPPEND\end{ttfamily}} é necessário porque DmxScroller trabalha com string curta e a mesma tem um tamanho de 255 caracteres, onde o tamanho está na posição 0.
\item Como usar a constante \textbf{\begin{ttfamily}fldAPPEND\end{ttfamily}}:

\begin{itemize}
\item A função \textbf{CreateAppendFields} retorna a constante \textbf{\begin{ttfamily}fldAPPEND\end{ttfamily}} mais o endereço da string a ser concatenada.

\begin{itemize}
\item \textbf{EXEMPLO}

\texttt{\\\nopagebreak[3]
\\\nopagebreak[3]
}\textbf{procedure}\texttt{~Template~:~ShortString;\\\nopagebreak[3]
~~}\textbf{Var}\texttt{\\\nopagebreak[3]
~~~~S1,s2,Template~:~TString;\\\nopagebreak[3]
}\textbf{begin}\texttt{\\\nopagebreak[3]
~~S1~:=~'~Nome~do~Aluno....:~{\textbackslash}ssssssssssssssssssssssssssssssssss';\\\nopagebreak[3]
~~s2~:=~'~Endereço~do~aluno:~{\textbackslash}sssssssssssssssssssssssss';\\\nopagebreak[3]
~~result~:=~S1+CreateAppendFields(s2);\\\nopagebreak[3]
}\textbf{end}\texttt{;\\
}
\item \textbf{NOTA} \begin{itemize}
\item A contante \textbf{\begin{ttfamily}fldAPPEND\end{ttfamily}} foi criada porque o projeto inicial foi para turbo pascal e ambiente console.
\item A versão atual podemos usar AnsiString visto que o limite do mesmo é a memória.
\item Para usar AnsiString é necessário converter para \begin{ttfamily}PSitem\end{ttfamily}(\ref{mi.rtl.Types.TTypes-PSItem}) com a função: \textbf{StringToSItem}.

\begin{itemize}
\item \textbf{EXEMPLO:}

\texttt{\\\nopagebreak[3]
\\\nopagebreak[3]
}\textbf{function}\texttt{~TMI{\_}UI{\_}InputBox.DmxScroller{\_}Form1GetTemplate(aNext:~PSItem):~PSItem;\\\nopagebreak[3]
}\textbf{begin}\texttt{\\\nopagebreak[3]
~~}\textbf{with}\texttt{~DmxScroller{\_}Form1~}\textbf{do}\texttt{\\\nopagebreak[3]
~~}\textbf{begin}\texttt{\\\nopagebreak[3]
~~~~}\textbf{if}\texttt{~{\_}Template~~{$<$}{$>$}~''\\\nopagebreak[3]
~~~~}\textbf{then}\texttt{~Result~:=~StringToSItem({\_}Template,~80);\\\nopagebreak[3]
\\\nopagebreak[3]
\textit{//~~~~Result~:=~StringToSItem({\_}Template,~40,TObjectsTypes.TAlinhamento.Alinhamento{\_}Esquerda)}\\\nopagebreak[3]
\textit{//~~~~Result~:=~StringToSItem({\_}Template,~40,TObjectsTypes.TAlinhamento.Alinhamento{\_}Central)}\\\nopagebreak[3]
\textit{//~~~~Result~:=~StringToSItem({\_}Template,~40,TObjectsTypes.TAlinhamento.Alinhamento{\_}Direita)}\\\nopagebreak[3]
\textit{//~~~~Result~:=~StringToSItem({\_}Template,~80,TObjectsTypes.TAlinhamento.Alinhamento{\_}Justificado)}\\\nopagebreak[3]
\\\nopagebreak[3]
~~~~}\textbf{else}\texttt{~result~:=~}\textbf{nil}\texttt{;\\\nopagebreak[3]
~~}\textbf{end}\texttt{;\\\nopagebreak[3]
}\textbf{end}\texttt{;\\
}
\end{itemize}
\end{itemize}
\end{itemize}
\end{itemize}
\end{itemize}

\end{list}
\paragraph*{fldSItems}\hspace*{\fill}

\begin{list}{}{
\settowidth{\tmplength}{\textbf{Declaração}}
\setlength{\itemindent}{0cm}
\setlength{\listparindent}{0cm}
\setlength{\leftmargin}{\evensidemargin}
\addtolength{\leftmargin}{\tmplength}
\settowidth{\labelsep}{X}
\addtolength{\leftmargin}{\labelsep}
\setlength{\labelwidth}{\tmplength}
}
\begin{flushleft}
\item[\textbf{Declaração}\hfill]
\begin{ttfamily}
public const fldSItems           =   {\^{}}I;\end{ttfamily}


\end{flushleft}
\par
\item[\textbf{Descrição}]
link to chain of \begin{ttfamily}TSItem\end{ttfamily}(\ref{mi.rtl.Types.TTypes-TSItem}) Templates

\end{list}
\paragraph*{fldExtended}\hspace*{\fill}

\begin{list}{}{
\settowidth{\tmplength}{\textbf{Declaração}}
\setlength{\itemindent}{0cm}
\setlength{\listparindent}{0cm}
\setlength{\leftmargin}{\evensidemargin}
\addtolength{\leftmargin}{\tmplength}
\settowidth{\labelsep}{X}
\addtolength{\leftmargin}{\labelsep}
\setlength{\labelwidth}{\tmplength}
}
\begin{flushleft}
\item[\textbf{Declaração}\hfill]
\begin{ttfamily}
public const fldExtended       = 'E';\end{ttfamily}


\end{flushleft}
\par
\item[\textbf{Descrição}]
\begin{ttfamily}Real\end{ttfamily}(\ref{mi.rtl.Types.TTypes-Real}) 10 bytes

\end{list}
\paragraph*{fldReal4}\hspace*{\fill}

\begin{list}{}{
\settowidth{\tmplength}{\textbf{Declaração}}
\setlength{\itemindent}{0cm}
\setlength{\listparindent}{0cm}
\setlength{\leftmargin}{\evensidemargin}
\addtolength{\leftmargin}{\tmplength}
\settowidth{\labelsep}{X}
\addtolength{\leftmargin}{\labelsep}
\setlength{\labelwidth}{\tmplength}
}
\begin{flushleft}
\item[\textbf{Declaração}\hfill]
\begin{ttfamily}
public const fldReal4          = 'O';\end{ttfamily}


\end{flushleft}
\par
\item[\textbf{Descrição}]
\begin{ttfamily}Real\end{ttfamily}(\ref{mi.rtl.Types.TTypes-Real}) 4 Byte positivos e negativos

\end{list}
\paragraph*{fldReal4Positivo}\hspace*{\fill}

\begin{list}{}{
\settowidth{\tmplength}{\textbf{Declaração}}
\setlength{\itemindent}{0cm}
\setlength{\listparindent}{0cm}
\setlength{\leftmargin}{\evensidemargin}
\addtolength{\leftmargin}{\tmplength}
\settowidth{\labelsep}{X}
\addtolength{\leftmargin}{\labelsep}
\setlength{\labelwidth}{\tmplength}
}
\begin{flushleft}
\item[\textbf{Declaração}\hfill]
\begin{ttfamily}
public const fldReal4Positivo  = 'o';\end{ttfamily}


\end{flushleft}
\par
\item[\textbf{Descrição}]
\begin{ttfamily}Real\end{ttfamily}(\ref{mi.rtl.Types.TTypes-Real}) 4 Byte positivos

\end{list}
\paragraph*{fldReal4P}\hspace*{\fill}

\begin{list}{}{
\settowidth{\tmplength}{\textbf{Declaração}}
\setlength{\itemindent}{0cm}
\setlength{\listparindent}{0cm}
\setlength{\leftmargin}{\evensidemargin}
\addtolength{\leftmargin}{\tmplength}
\settowidth{\labelsep}{X}
\addtolength{\leftmargin}{\labelsep}
\setlength{\labelwidth}{\tmplength}
}
\begin{flushleft}
\item[\textbf{Declaração}\hfill]
\begin{ttfamily}
public const fldReal4P         = 'P';\end{ttfamily}


\end{flushleft}
\par
\item[\textbf{Descrição}]
P = \begin{ttfamily}Real\end{ttfamily}(\ref{mi.rtl.Types.TTypes-Real}) de mostrado x por 100 positivos e negativos

\end{list}
\paragraph*{fldReal4PPositivo}\hspace*{\fill}

\begin{list}{}{
\settowidth{\tmplength}{\textbf{Declaração}}
\setlength{\itemindent}{0cm}
\setlength{\listparindent}{0cm}
\setlength{\leftmargin}{\evensidemargin}
\addtolength{\leftmargin}{\tmplength}
\settowidth{\labelsep}{X}
\addtolength{\leftmargin}{\labelsep}
\setlength{\labelwidth}{\tmplength}
}
\begin{flushleft}
\item[\textbf{Declaração}\hfill]
\begin{ttfamily}
public const fldReal4PPositivo = 'p';\end{ttfamily}


\end{flushleft}
\par
\item[\textbf{Descrição}]
P = \begin{ttfamily}Real\end{ttfamily}(\ref{mi.rtl.Types.TTypes-Real}) de mostrado x por 100 positivos

\end{list}
\paragraph*{FldLink}\hspace*{\fill}

\begin{list}{}{
\settowidth{\tmplength}{\textbf{Declaração}}
\setlength{\itemindent}{0cm}
\setlength{\listparindent}{0cm}
\setlength{\leftmargin}{\evensidemargin}
\addtolength{\leftmargin}{\tmplength}
\settowidth{\labelsep}{X}
\addtolength{\leftmargin}{\labelsep}
\setlength{\labelwidth}{\tmplength}
}
\begin{flushleft}
\item[\textbf{Declaração}\hfill]
\begin{ttfamily}
public const FldLink       = {\^{}}L;\end{ttfamily}


\end{flushleft}
\par
\item[\textbf{Descrição}]
A constante \textbf{\begin{ttfamily}FldLink\end{ttfamily}} indica que o campo contem um campo com 255 posições que contém um endereço para um página html ou não:

\begin{itemize}
\item \textbf{LINKS POSSÍVEIS:} \begin{itemize}
\item {\^{}}L+1 = Endereço de uma página na web a ser acessada pelo browser.
\item {\^{}}L+2 = Nome de uma ação da lista actionItens.
\end{itemize}
\end{itemize}

\end{list}
\paragraph*{FldlinkUrl}\hspace*{\fill}

\begin{list}{}{
\settowidth{\tmplength}{\textbf{Declaração}}
\setlength{\itemindent}{0cm}
\setlength{\listparindent}{0cm}
\setlength{\leftmargin}{\evensidemargin}
\addtolength{\leftmargin}{\tmplength}
\settowidth{\labelsep}{X}
\addtolength{\leftmargin}{\labelsep}
\setlength{\labelwidth}{\tmplength}
}
\begin{flushleft}
\item[\textbf{Declaração}\hfill]
\begin{ttfamily}
public const FldlinkUrl    = {\^{}}L+'1';\end{ttfamily}


\end{flushleft}
\par
\item[\textbf{Descrição}]
Endereço de uma página na web a ser acessada pelo browser.

\end{list}
\paragraph*{FldlinkAction}\hspace*{\fill}

\begin{list}{}{
\settowidth{\tmplength}{\textbf{Declaração}}
\setlength{\itemindent}{0cm}
\setlength{\listparindent}{0cm}
\setlength{\leftmargin}{\evensidemargin}
\addtolength{\leftmargin}{\tmplength}
\settowidth{\labelsep}{X}
\addtolength{\leftmargin}{\labelsep}
\setlength{\labelwidth}{\tmplength}
}
\begin{flushleft}
\item[\textbf{Declaração}\hfill]
\begin{ttfamily}
public const FldlinkAction = {\^{}}L+'2';\end{ttfamily}


\end{flushleft}
\par
\item[\textbf{Descrição}]
Nome de uma ação da lista actionItens.

\end{list}
\paragraph*{fldData}\hspace*{\fill}

\begin{list}{}{
\settowidth{\tmplength}{\textbf{Declaração}}
\setlength{\itemindent}{0cm}
\setlength{\listparindent}{0cm}
\setlength{\leftmargin}{\evensidemargin}
\addtolength{\leftmargin}{\tmplength}
\settowidth{\labelsep}{X}
\addtolength{\leftmargin}{\labelsep}
\setlength{\labelwidth}{\tmplength}
}
\begin{flushleft}
\item[\textbf{Declaração}\hfill]
\begin{ttfamily}
public const fldData           = 'D';\end{ttfamily}


\end{flushleft}
\par
\item[\textbf{Descrição}]
D = TipoData DD/DD/DD

\end{list}
\paragraph*{fld{\_}LData}\hspace*{\fill}

\begin{list}{}{
\settowidth{\tmplength}{\textbf{Declaração}}
\setlength{\itemindent}{0cm}
\setlength{\listparindent}{0cm}
\setlength{\leftmargin}{\evensidemargin}
\addtolength{\leftmargin}{\tmplength}
\settowidth{\labelsep}{X}
\addtolength{\leftmargin}{\labelsep}
\setlength{\labelwidth}{\tmplength}
}
\begin{flushleft}
\item[\textbf{Declaração}\hfill]
\begin{ttfamily}
public const fld{\_}LData         = 'd' ;\end{ttfamily}


\end{flushleft}
\par
\item[\textbf{Descrição}]
d = TDataTime;Guarda a data compactada 'dd/dd/dd'

\end{list}
\paragraph*{fldLData}\hspace*{\fill}

\begin{list}{}{
\settowidth{\tmplength}{\textbf{Declaração}}
\setlength{\itemindent}{0cm}
\setlength{\listparindent}{0cm}
\setlength{\leftmargin}{\evensidemargin}
\addtolength{\leftmargin}{\tmplength}
\settowidth{\labelsep}{X}
\addtolength{\leftmargin}{\labelsep}
\setlength{\labelwidth}{\tmplength}
}
\begin{flushleft}
\item[\textbf{Declaração}\hfill]
\begin{ttfamily}
public const fldLData          = {\#}1  ;\end{ttfamily}


\end{flushleft}
\par
\item[\textbf{Descrição}]
{\#}1 = TDataTime;Guarda a data compactada '{\#}{\#}/{\#}{\#}/{\#}{\#}'

\end{list}
\paragraph*{FldSData}\hspace*{\fill}

\begin{list}{}{
\settowidth{\tmplength}{\textbf{Declaração}}
\setlength{\itemindent}{0cm}
\setlength{\listparindent}{0cm}
\setlength{\leftmargin}{\evensidemargin}
\addtolength{\leftmargin}{\tmplength}
\settowidth{\labelsep}{X}
\addtolength{\leftmargin}{\labelsep}
\setlength{\labelwidth}{\tmplength}
}
\begin{flushleft}
\item[\textbf{Declaração}\hfill]
\begin{ttfamily}
public const FldSData          = '{\#}{\#}/{\#}{\#}/{\#}{\#}';\end{ttfamily}


\end{flushleft}
\end{list}
\paragraph*{fldLHora}\hspace*{\fill}

\begin{list}{}{
\settowidth{\tmplength}{\textbf{Declaração}}
\setlength{\itemindent}{0cm}
\setlength{\listparindent}{0cm}
\setlength{\leftmargin}{\evensidemargin}
\addtolength{\leftmargin}{\tmplength}
\settowidth{\labelsep}{X}
\addtolength{\leftmargin}{\labelsep}
\setlength{\labelwidth}{\tmplength}
}
\begin{flushleft}
\item[\textbf{Declaração}\hfill]
\begin{ttfamily}
public const fldLHora          = {\#}2 ;\end{ttfamily}


\end{flushleft}
\par
\item[\textbf{Descrição}]
{\#}2 = \begin{ttfamily}Longint\end{ttfamily}(\ref{mi.rtl.Types.TTypes-LongInt});Guarda a hora compactada {\#}{\#}:{\#}{\#}:{\#}{\#}

\end{list}
\paragraph*{FldSHora}\hspace*{\fill}

\begin{list}{}{
\settowidth{\tmplength}{\textbf{Declaração}}
\setlength{\itemindent}{0cm}
\setlength{\listparindent}{0cm}
\setlength{\leftmargin}{\evensidemargin}
\addtolength{\leftmargin}{\tmplength}
\settowidth{\labelsep}{X}
\addtolength{\leftmargin}{\labelsep}
\setlength{\labelwidth}{\tmplength}
}
\begin{flushleft}
\item[\textbf{Declaração}\hfill]
\begin{ttfamily}
public const FldSHora          = '{\#}{\#}:{\#}{\#}:{\#}{\#}';\end{ttfamily}


\end{flushleft}
\end{list}
\paragraph*{fld{\_}LHora}\hspace*{\fill}

\begin{list}{}{
\settowidth{\tmplength}{\textbf{Declaração}}
\setlength{\itemindent}{0cm}
\setlength{\listparindent}{0cm}
\setlength{\leftmargin}{\evensidemargin}
\addtolength{\leftmargin}{\tmplength}
\settowidth{\labelsep}{X}
\addtolength{\leftmargin}{\labelsep}
\setlength{\labelwidth}{\tmplength}
}
\begin{flushleft}
\item[\textbf{Declaração}\hfill]
\begin{ttfamily}
public const fld{\_}LHora         = 'h';\end{ttfamily}


\end{flushleft}
\par
\item[\textbf{Descrição}]
h = \begin{ttfamily}Longint\end{ttfamily}(\ref{mi.rtl.Types.TTypes-LongInt});Guarda a hora compactada hh:hh:hh

\end{list}
\paragraph*{FldOperador}\hspace*{\fill}

\begin{list}{}{
\settowidth{\tmplength}{\textbf{Declaração}}
\setlength{\itemindent}{0cm}
\setlength{\listparindent}{0cm}
\setlength{\leftmargin}{\evensidemargin}
\addtolength{\leftmargin}{\tmplength}
\settowidth{\labelsep}{X}
\addtolength{\leftmargin}{\labelsep}
\setlength{\labelwidth}{\tmplength}
}
\begin{flushleft}
\item[\textbf{Declaração}\hfill]
\begin{ttfamily}
public const FldOperador       = {\#}3;\end{ttfamily}


\end{flushleft}
\par
\item[\textbf{Descrição}]
{\#}3 = Byte indica que o campo é um operador matemático

\end{list}
\paragraph*{FldDateTimeDos}\hspace*{\fill}

\begin{list}{}{
\settowidth{\tmplength}{\textbf{Declaração}}
\setlength{\itemindent}{0cm}
\setlength{\listparindent}{0cm}
\setlength{\leftmargin}{\evensidemargin}
\addtolength{\leftmargin}{\tmplength}
\settowidth{\labelsep}{X}
\addtolength{\leftmargin}{\labelsep}
\setlength{\labelwidth}{\tmplength}
}
\begin{flushleft}
\item[\textbf{Declaração}\hfill]
\begin{ttfamily}
public const FldDateTimeDos    = {\#}4 ;\end{ttfamily}


\end{flushleft}
\par
\item[\textbf{Descrição}]
{\#}4 = \begin{ttfamily}Longint\end{ttfamily}(\ref{mi.rtl.Types.TTypes-LongInt});Guarda a data e hora compactada {\#}{\#}/{\#}{\#}/{\#}{\#} {\#}{\#}:{\#}{\#}:{\#}{\#} e o ano não pode ser menor que 1980.

\end{list}
\paragraph*{FldSDateTimeDos}\hspace*{\fill}

\begin{list}{}{
\settowidth{\tmplength}{\textbf{Declaração}}
\setlength{\itemindent}{0cm}
\setlength{\listparindent}{0cm}
\setlength{\leftmargin}{\evensidemargin}
\addtolength{\leftmargin}{\tmplength}
\settowidth{\labelsep}{X}
\addtolength{\leftmargin}{\labelsep}
\setlength{\labelwidth}{\tmplength}
}
\begin{flushleft}
\item[\textbf{Declaração}\hfill]
\begin{ttfamily}
public const FldSDateTimeDos   = '{\#}{\#}/{\#}{\#}/{\#}{\#} {\#}{\#}:{\#}{\#}:{\#}{\#}';\end{ttfamily}


\end{flushleft}
\end{list}
\paragraph*{CharShowPassword}\hspace*{\fill}

\begin{list}{}{
\settowidth{\tmplength}{\textbf{Declaração}}
\setlength{\itemindent}{0cm}
\setlength{\listparindent}{0cm}
\setlength{\leftmargin}{\evensidemargin}
\addtolength{\leftmargin}{\tmplength}
\settowidth{\labelsep}{X}
\addtolength{\leftmargin}{\labelsep}
\setlength{\labelwidth}{\tmplength}
}
\begin{flushleft}
\item[\textbf{Declaração}\hfill]
\begin{ttfamily}
public const CharShowPassword      = {\^{}}W;\end{ttfamily}


\end{flushleft}
\par
\item[\textbf{Descrição}]
Usado para omitir os caracteres que estão sendo digitados em qualquer tipo de campo

\end{list}
\paragraph*{ChSP}\hspace*{\fill}

\begin{list}{}{
\settowidth{\tmplength}{\textbf{Declaração}}
\setlength{\itemindent}{0cm}
\setlength{\listparindent}{0cm}
\setlength{\leftmargin}{\evensidemargin}
\addtolength{\leftmargin}{\tmplength}
\settowidth{\labelsep}{X}
\addtolength{\leftmargin}{\labelsep}
\setlength{\labelwidth}{\tmplength}
}
\begin{flushleft}
\item[\textbf{Declaração}\hfill]
\begin{ttfamily}
public const ChSP = CharShowPassword;\end{ttfamily}


\end{flushleft}
\par
\item[\textbf{Descrição}]
A contante \textbf{\begin{ttfamily}ChSP\end{ttfamily}} é igual \begin{ttfamily}CharShowPassword\end{ttfamily}(\ref{mi.rtl.Consts.TConsts-CharShowPassword}).

\end{list}
\paragraph*{CharShowPasswordChar}\hspace*{\fill}

\begin{list}{}{
\settowidth{\tmplength}{\textbf{Declaração}}
\setlength{\itemindent}{0cm}
\setlength{\listparindent}{0cm}
\setlength{\leftmargin}{\evensidemargin}
\addtolength{\leftmargin}{\tmplength}
\settowidth{\labelsep}{X}
\addtolength{\leftmargin}{\labelsep}
\setlength{\labelwidth}{\tmplength}
}
\begin{flushleft}
\item[\textbf{Declaração}\hfill]
\begin{ttfamily}
public const CharShowPasswordChar  =   '*';\end{ttfamily}


\end{flushleft}
\par
\item[\textbf{Descrição}]
Caractere a ser mostrado quando \begin{ttfamily}CharShowPassword\end{ttfamily}(\ref{mi.rtl.Consts.TConsts-CharShowPassword}) em fldField for igual = {\^{}}W

\end{list}
\paragraph*{CharExecAction}\hspace*{\fill}

\begin{list}{}{
\settowidth{\tmplength}{\textbf{Declaração}}
\setlength{\itemindent}{0cm}
\setlength{\listparindent}{0cm}
\setlength{\leftmargin}{\evensidemargin}
\addtolength{\leftmargin}{\tmplength}
\settowidth{\labelsep}{X}
\addtolength{\leftmargin}{\labelsep}
\setlength{\labelwidth}{\tmplength}
}
\begin{flushleft}
\item[\textbf{Declaração}\hfill]
\begin{ttfamily}
public const CharExecAction       = {\^{}}T;\end{ttfamily}


\end{flushleft}
\par
\item[\textbf{Descrição}]
A contante \textbf{\begin{ttfamily}CharExecAction\end{ttfamily}} é usado para associar ao campo atual uma classe \textbf{TAction}.

\begin{itemize}
\item \textbf{NOTA} \begin{itemize}
\item O interpretador de Templates associa a ação do Template ao corrente campo.
\end{itemize}
\end{itemize}

\end{list}
\paragraph*{ChEA}\hspace*{\fill}

\begin{list}{}{
\settowidth{\tmplength}{\textbf{Declaração}}
\setlength{\itemindent}{0cm}
\setlength{\listparindent}{0cm}
\setlength{\leftmargin}{\evensidemargin}
\addtolength{\leftmargin}{\tmplength}
\settowidth{\labelsep}{X}
\addtolength{\leftmargin}{\labelsep}
\setlength{\labelwidth}{\tmplength}
}
\begin{flushleft}
\item[\textbf{Declaração}\hfill]
\begin{ttfamily}
public const ChEA  = CharExecAction;\end{ttfamily}


\end{flushleft}
\par
\item[\textbf{Descrição}]
A contante \textbf{\begin{ttfamily}ChEA\end{ttfamily}} é igual \begin{ttfamily}CharExecAction\end{ttfamily}(\ref{mi.rtl.Consts.TConsts-CharExecAction}).

\end{list}
\paragraph*{CharLupa{\_}Left}\hspace*{\fill}

\begin{list}{}{
\settowidth{\tmplength}{\textbf{Declaração}}
\setlength{\itemindent}{0cm}
\setlength{\listparindent}{0cm}
\setlength{\leftmargin}{\evensidemargin}
\addtolength{\leftmargin}{\tmplength}
\settowidth{\labelsep}{X}
\addtolength{\leftmargin}{\labelsep}
\setlength{\labelwidth}{\tmplength}
}
\begin{flushleft}
\item[\textbf{Declaração}\hfill]
\begin{ttfamily}
public const CharLupa{\_}Left           = '🔍';\end{ttfamily}


\end{flushleft}
\end{list}
\paragraph*{CharLupa{\_}Right}\hspace*{\fill}

\begin{list}{}{
\settowidth{\tmplength}{\textbf{Declaração}}
\setlength{\itemindent}{0cm}
\setlength{\listparindent}{0cm}
\setlength{\leftmargin}{\evensidemargin}
\addtolength{\leftmargin}{\tmplength}
\settowidth{\labelsep}{X}
\addtolength{\leftmargin}{\labelsep}
\setlength{\labelwidth}{\tmplength}
}
\begin{flushleft}
\item[\textbf{Declaração}\hfill]
\begin{ttfamily}
public const CharLupa{\_}Right          = '🔎';\end{ttfamily}


\end{flushleft}
\end{list}
\paragraph*{Char{\_}Seta{\_}para{\_}Cima}\hspace*{\fill}

\begin{list}{}{
\settowidth{\tmplength}{\textbf{Declaração}}
\setlength{\itemindent}{0cm}
\setlength{\listparindent}{0cm}
\setlength{\leftmargin}{\evensidemargin}
\addtolength{\leftmargin}{\tmplength}
\settowidth{\labelsep}{X}
\addtolength{\leftmargin}{\labelsep}
\setlength{\labelwidth}{\tmplength}
}
\begin{flushleft}
\item[\textbf{Declaração}\hfill]
\begin{ttfamily}
public const Char{\_}Seta{\_}para{\_}Cima     = '⬆️';\end{ttfamily}


\end{flushleft}
\end{list}
\paragraph*{Char{\_}Seta{\_}para{\_}Baixo}\hspace*{\fill}

\begin{list}{}{
\settowidth{\tmplength}{\textbf{Declaração}}
\setlength{\itemindent}{0cm}
\setlength{\listparindent}{0cm}
\setlength{\leftmargin}{\evensidemargin}
\addtolength{\leftmargin}{\tmplength}
\settowidth{\labelsep}{X}
\addtolength{\leftmargin}{\labelsep}
\setlength{\labelwidth}{\tmplength}
}
\begin{flushleft}
\item[\textbf{Declaração}\hfill]
\begin{ttfamily}
public const Char{\_}Seta{\_}para{\_}Baixo    = '⬇️️';\end{ttfamily}


\end{flushleft}
\end{list}
\paragraph*{Char{\_}Seta{\_}para{\_}direita}\hspace*{\fill}

\begin{list}{}{
\settowidth{\tmplength}{\textbf{Declaração}}
\setlength{\itemindent}{0cm}
\setlength{\listparindent}{0cm}
\setlength{\leftmargin}{\evensidemargin}
\addtolength{\leftmargin}{\tmplength}
\settowidth{\labelsep}{X}
\addtolength{\leftmargin}{\labelsep}
\setlength{\labelwidth}{\tmplength}
}
\begin{flushleft}
\item[\textbf{Declaração}\hfill]
\begin{ttfamily}
public const Char{\_}Seta{\_}para{\_}direita  : AnsiString = '➡️';\end{ttfamily}


\end{flushleft}
\end{list}
\paragraph*{Char{\_}Seta{\_}para{\_}esquerda}\hspace*{\fill}

\begin{list}{}{
\settowidth{\tmplength}{\textbf{Declaração}}
\setlength{\itemindent}{0cm}
\setlength{\listparindent}{0cm}
\setlength{\leftmargin}{\evensidemargin}
\addtolength{\leftmargin}{\tmplength}
\settowidth{\labelsep}{X}
\addtolength{\leftmargin}{\labelsep}
\setlength{\labelwidth}{\tmplength}
}
\begin{flushleft}
\item[\textbf{Declaração}\hfill]
\begin{ttfamily}
public const Char{\_}Seta{\_}para{\_}esquerda : AnsiString = '⬅️️';\end{ttfamily}


\end{flushleft}
\end{list}
\paragraph*{Char{\_}Seta{\_}Back}\hspace*{\fill}

\begin{list}{}{
\settowidth{\tmplength}{\textbf{Declaração}}
\setlength{\itemindent}{0cm}
\setlength{\listparindent}{0cm}
\setlength{\leftmargin}{\evensidemargin}
\addtolength{\leftmargin}{\tmplength}
\settowidth{\labelsep}{X}
\addtolength{\leftmargin}{\labelsep}
\setlength{\labelwidth}{\tmplength}
}
\begin{flushleft}
\item[\textbf{Declaração}\hfill]
\begin{ttfamily}
public const Char{\_}Seta{\_}Back = '🔙';\end{ttfamily}


\end{flushleft}
\end{list}
\paragraph*{Char{\_}Seta{\_}End}\hspace*{\fill}

\begin{list}{}{
\settowidth{\tmplength}{\textbf{Declaração}}
\setlength{\itemindent}{0cm}
\setlength{\listparindent}{0cm}
\setlength{\leftmargin}{\evensidemargin}
\addtolength{\leftmargin}{\tmplength}
\settowidth{\labelsep}{X}
\addtolength{\leftmargin}{\labelsep}
\setlength{\labelwidth}{\tmplength}
}
\begin{flushleft}
\item[\textbf{Declaração}\hfill]
\begin{ttfamily}
public const Char{\_}Seta{\_}End = '🔚';\end{ttfamily}


\end{flushleft}
\end{list}
\paragraph*{Char{\_}Seta{\_}On}\hspace*{\fill}

\begin{list}{}{
\settowidth{\tmplength}{\textbf{Declaração}}
\setlength{\itemindent}{0cm}
\setlength{\listparindent}{0cm}
\setlength{\leftmargin}{\evensidemargin}
\addtolength{\leftmargin}{\tmplength}
\settowidth{\labelsep}{X}
\addtolength{\leftmargin}{\labelsep}
\setlength{\labelwidth}{\tmplength}
}
\begin{flushleft}
\item[\textbf{Declaração}\hfill]
\begin{ttfamily}
public const Char{\_}Seta{\_}On  = '🔛';\end{ttfamily}


\end{flushleft}
\end{list}
\paragraph*{Char{\_}Seta{\_}Em{\_}breve{\_}Flecha}\hspace*{\fill}

\begin{list}{}{
\settowidth{\tmplength}{\textbf{Declaração}}
\setlength{\itemindent}{0cm}
\setlength{\listparindent}{0cm}
\setlength{\leftmargin}{\evensidemargin}
\addtolength{\leftmargin}{\tmplength}
\settowidth{\labelsep}{X}
\addtolength{\leftmargin}{\labelsep}
\setlength{\labelwidth}{\tmplength}
}
\begin{flushleft}
\item[\textbf{Declaração}\hfill]
\begin{ttfamily}
public const Char{\_}Seta{\_}Em{\_}breve{\_}Flecha  = '🔜';\end{ttfamily}


\end{flushleft}
\end{list}
\paragraph*{Char{\_}Seta{\_}Top}\hspace*{\fill}

\begin{list}{}{
\settowidth{\tmplength}{\textbf{Declaração}}
\setlength{\itemindent}{0cm}
\setlength{\listparindent}{0cm}
\setlength{\leftmargin}{\evensidemargin}
\addtolength{\leftmargin}{\tmplength}
\settowidth{\labelsep}{X}
\addtolength{\leftmargin}{\labelsep}
\setlength{\labelwidth}{\tmplength}
}
\begin{flushleft}
\item[\textbf{Declaração}\hfill]
\begin{ttfamily}
public const Char{\_}Seta{\_}Top              = '🔝';\end{ttfamily}


\end{flushleft}
\end{list}
\paragraph*{Char{\_}Seta{\_}Cicle}\hspace*{\fill}

\begin{list}{}{
\settowidth{\tmplength}{\textbf{Declaração}}
\setlength{\itemindent}{0cm}
\setlength{\listparindent}{0cm}
\setlength{\leftmargin}{\evensidemargin}
\addtolength{\leftmargin}{\tmplength}
\settowidth{\labelsep}{X}
\addtolength{\leftmargin}{\labelsep}
\setlength{\labelwidth}{\tmplength}
}
\begin{flushleft}
\item[\textbf{Declaração}\hfill]
\begin{ttfamily}
public const Char{\_}Seta{\_}Cicle            = '🔃';\end{ttfamily}


\end{flushleft}
\end{list}
\paragraph*{Char{\_}Bandeira{\_}triangular}\hspace*{\fill}

\begin{list}{}{
\settowidth{\tmplength}{\textbf{Declaração}}
\setlength{\itemindent}{0cm}
\setlength{\listparindent}{0cm}
\setlength{\leftmargin}{\evensidemargin}
\addtolength{\leftmargin}{\tmplength}
\settowidth{\labelsep}{X}
\addtolength{\leftmargin}{\labelsep}
\setlength{\labelwidth}{\tmplength}
}
\begin{flushleft}
\item[\textbf{Declaração}\hfill]
\begin{ttfamily}
public const Char{\_}Bandeira{\_}triangular   = '🚩 ';\end{ttfamily}


\end{flushleft}
\end{list}
\paragraph*{Char{\_}Ponto{\_}Interrogacao}\hspace*{\fill}

\begin{list}{}{
\settowidth{\tmplength}{\textbf{Declaração}}
\setlength{\itemindent}{0cm}
\setlength{\listparindent}{0cm}
\setlength{\leftmargin}{\evensidemargin}
\addtolength{\leftmargin}{\tmplength}
\settowidth{\labelsep}{X}
\addtolength{\leftmargin}{\labelsep}
\setlength{\labelwidth}{\tmplength}
}
\begin{flushleft}
\item[\textbf{Declaração}\hfill]
\begin{ttfamily}
public const Char{\_}Ponto{\_}Interrogacao    = '❓';\end{ttfamily}


\end{flushleft}
\end{list}
\paragraph*{Char{\_}Ponto{\_}Exclamacao}\hspace*{\fill}

\begin{list}{}{
\settowidth{\tmplength}{\textbf{Declaração}}
\setlength{\itemindent}{0cm}
\setlength{\listparindent}{0cm}
\setlength{\leftmargin}{\evensidemargin}
\addtolength{\leftmargin}{\tmplength}
\settowidth{\labelsep}{X}
\addtolength{\leftmargin}{\labelsep}
\setlength{\labelwidth}{\tmplength}
}
\begin{flushleft}
\item[\textbf{Declaração}\hfill]
\begin{ttfamily}
public const Char{\_}Ponto{\_}Exclamacao      = '❗';\end{ttfamily}


\end{flushleft}
\end{list}
\paragraph*{Char{\_}Dedo{\_}Direita}\hspace*{\fill}

\begin{list}{}{
\settowidth{\tmplength}{\textbf{Declaração}}
\setlength{\itemindent}{0cm}
\setlength{\listparindent}{0cm}
\setlength{\leftmargin}{\evensidemargin}
\addtolength{\leftmargin}{\tmplength}
\settowidth{\labelsep}{X}
\addtolength{\leftmargin}{\labelsep}
\setlength{\labelwidth}{\tmplength}
}
\begin{flushleft}
\item[\textbf{Declaração}\hfill]
\begin{ttfamily}
public const Char{\_}Dedo{\_}Direita          = '👉';\end{ttfamily}


\end{flushleft}
\end{list}
\paragraph*{Char{\_}Proxima{\_}Faixa}\hspace*{\fill}

\begin{list}{}{
\settowidth{\tmplength}{\textbf{Declaração}}
\setlength{\itemindent}{0cm}
\setlength{\listparindent}{0cm}
\setlength{\leftmargin}{\evensidemargin}
\addtolength{\leftmargin}{\tmplength}
\settowidth{\labelsep}{X}
\addtolength{\leftmargin}{\labelsep}
\setlength{\labelwidth}{\tmplength}
}
\begin{flushleft}
\item[\textbf{Declaração}\hfill]
\begin{ttfamily}
public const Char{\_}Proxima{\_}Faixa     = '⏭️';\end{ttfamily}


\end{flushleft}
\end{list}
\paragraph*{Char{\_}AvancoRapido}\hspace*{\fill}

\begin{list}{}{
\settowidth{\tmplength}{\textbf{Declaração}}
\setlength{\itemindent}{0cm}
\setlength{\listparindent}{0cm}
\setlength{\leftmargin}{\evensidemargin}
\addtolength{\leftmargin}{\tmplength}
\settowidth{\labelsep}{X}
\addtolength{\leftmargin}{\labelsep}
\setlength{\labelwidth}{\tmplength}
}
\begin{flushleft}
\item[\textbf{Declaração}\hfill]
\begin{ttfamily}
public const Char{\_}AvancoRapido      = '⏩';\end{ttfamily}


\end{flushleft}
\end{list}
\paragraph*{Char{\_}Retrocesso{\_}Rapido}\hspace*{\fill}

\begin{list}{}{
\settowidth{\tmplength}{\textbf{Declaração}}
\setlength{\itemindent}{0cm}
\setlength{\listparindent}{0cm}
\setlength{\leftmargin}{\evensidemargin}
\addtolength{\leftmargin}{\tmplength}
\settowidth{\labelsep}{X}
\addtolength{\leftmargin}{\labelsep}
\setlength{\labelwidth}{\tmplength}
}
\begin{flushleft}
\item[\textbf{Declaração}\hfill]
\begin{ttfamily}
public const Char{\_}Retrocesso{\_}Rapido = '⏪';\end{ttfamily}


\end{flushleft}
\end{list}
\paragraph*{Char{\_}Ultima{\_}Faixa}\hspace*{\fill}

\begin{list}{}{
\settowidth{\tmplength}{\textbf{Declaração}}
\setlength{\itemindent}{0cm}
\setlength{\listparindent}{0cm}
\setlength{\leftmargin}{\evensidemargin}
\addtolength{\leftmargin}{\tmplength}
\settowidth{\labelsep}{X}
\addtolength{\leftmargin}{\labelsep}
\setlength{\labelwidth}{\tmplength}
}
\begin{flushleft}
\item[\textbf{Declaração}\hfill]
\begin{ttfamily}
public const Char{\_}Ultima{\_}Faixa      = '⏮️';\end{ttfamily}


\end{flushleft}
\end{list}
\paragraph*{Char{\_}GoBof}\hspace*{\fill}

\begin{list}{}{
\settowidth{\tmplength}{\textbf{Declaração}}
\setlength{\itemindent}{0cm}
\setlength{\listparindent}{0cm}
\setlength{\leftmargin}{\evensidemargin}
\addtolength{\leftmargin}{\tmplength}
\settowidth{\labelsep}{X}
\addtolength{\leftmargin}{\labelsep}
\setlength{\labelwidth}{\tmplength}
}
\begin{flushleft}
\item[\textbf{Declaração}\hfill]
\begin{ttfamily}
public const Char{\_}GoBof             = Char{\_}Proxima{\_}Faixa;\end{ttfamily}


\end{flushleft}
\end{list}
\paragraph*{Char{\_}Next}\hspace*{\fill}

\begin{list}{}{
\settowidth{\tmplength}{\textbf{Declaração}}
\setlength{\itemindent}{0cm}
\setlength{\listparindent}{0cm}
\setlength{\leftmargin}{\evensidemargin}
\addtolength{\leftmargin}{\tmplength}
\settowidth{\labelsep}{X}
\addtolength{\leftmargin}{\labelsep}
\setlength{\labelwidth}{\tmplength}
}
\begin{flushleft}
\item[\textbf{Declaração}\hfill]
\begin{ttfamily}
public const Char{\_}Next              = Char{\_}AvancoRapido;\end{ttfamily}


\end{flushleft}
\end{list}
\paragraph*{Char{\_}Prev}\hspace*{\fill}

\begin{list}{}{
\settowidth{\tmplength}{\textbf{Declaração}}
\setlength{\itemindent}{0cm}
\setlength{\listparindent}{0cm}
\setlength{\leftmargin}{\evensidemargin}
\addtolength{\leftmargin}{\tmplength}
\settowidth{\labelsep}{X}
\addtolength{\leftmargin}{\labelsep}
\setlength{\labelwidth}{\tmplength}
}
\begin{flushleft}
\item[\textbf{Declaração}\hfill]
\begin{ttfamily}
public const Char{\_}Prev              = Char{\_}Retrocesso{\_}Rapido;\end{ttfamily}


\end{flushleft}
\end{list}
\paragraph*{Char{\_}GoEof}\hspace*{\fill}

\begin{list}{}{
\settowidth{\tmplength}{\textbf{Declaração}}
\setlength{\itemindent}{0cm}
\setlength{\listparindent}{0cm}
\setlength{\leftmargin}{\evensidemargin}
\addtolength{\leftmargin}{\tmplength}
\settowidth{\labelsep}{X}
\addtolength{\leftmargin}{\labelsep}
\setlength{\labelwidth}{\tmplength}
}
\begin{flushleft}
\item[\textbf{Declaração}\hfill]
\begin{ttfamily}
public const Char{\_}GoEof             = Char{\_}Ultima{\_}Faixa;\end{ttfamily}


\end{flushleft}
\end{list}
\paragraph*{Char{\_}Refresh}\hspace*{\fill}

\begin{list}{}{
\settowidth{\tmplength}{\textbf{Declaração}}
\setlength{\itemindent}{0cm}
\setlength{\listparindent}{0cm}
\setlength{\leftmargin}{\evensidemargin}
\addtolength{\leftmargin}{\tmplength}
\settowidth{\labelsep}{X}
\addtolength{\leftmargin}{\labelsep}
\setlength{\labelwidth}{\tmplength}
}
\begin{flushleft}
\item[\textbf{Declaração}\hfill]
\begin{ttfamily}
public const Char{\_}Refresh           = Char{\_}Seta{\_}Cicle;\end{ttfamily}


\end{flushleft}
\end{list}
\paragraph*{CharFieldName}\hspace*{\fill}

\begin{list}{}{
\settowidth{\tmplength}{\textbf{Declaração}}
\setlength{\itemindent}{0cm}
\setlength{\listparindent}{0cm}
\setlength{\leftmargin}{\evensidemargin}
\addtolength{\leftmargin}{\tmplength}
\settowidth{\labelsep}{X}
\addtolength{\leftmargin}{\labelsep}
\setlength{\labelwidth}{\tmplength}
}
\begin{flushleft}
\item[\textbf{Declaração}\hfill]
\begin{ttfamily}
public const CharFieldName     = {\^{}}B;\end{ttfamily}


\end{flushleft}
\par
\item[\textbf{Descrição}]
A constante \textbf{\begin{ttfamily}CharFieldName\end{ttfamily}} informa o nome do campo no Template. O nome do campo é passado após {\^{}}B e o mesmo não pode conter espaço em branco.

\begin{itemize}
\item \textbf{EXEMPLO DE USO}

\texttt{\\\nopagebreak[3]
\\\nopagebreak[3]
NewSitem(~Nome~}\textbf{do}\texttt{~produto:~~~SSSSSSSSSSSSSSSS`SSSSSSS{\^{}}BNome{\_}do{\_}Produto,}\textbf{nil}\texttt{);\\
}
\end{itemize}

\end{list}
\paragraph*{ChFN}\hspace*{\fill}

\begin{list}{}{
\settowidth{\tmplength}{\textbf{Declaração}}
\setlength{\itemindent}{0cm}
\setlength{\listparindent}{0cm}
\setlength{\leftmargin}{\evensidemargin}
\addtolength{\leftmargin}{\tmplength}
\settowidth{\labelsep}{X}
\addtolength{\leftmargin}{\labelsep}
\setlength{\labelwidth}{\tmplength}
}
\begin{flushleft}
\item[\textbf{Declaração}\hfill]
\begin{ttfamily}
public const ChFN = CharFieldName;\end{ttfamily}


\end{flushleft}
\par
\item[\textbf{Descrição}]
A constante \textbf{\begin{ttfamily}ChFN\end{ttfamily}} é igual a \begin{ttfamily}CharFieldName\end{ttfamily}(\ref{mi.rtl.Consts.TConsts-CharFieldName}), foi criada para facilitar seu uso.

\end{list}
\paragraph*{CharListComboBox}\hspace*{\fill}

\begin{list}{}{
\settowidth{\tmplength}{\textbf{Declaração}}
\setlength{\itemindent}{0cm}
\setlength{\listparindent}{0cm}
\setlength{\leftmargin}{\evensidemargin}
\addtolength{\leftmargin}{\tmplength}
\settowidth{\labelsep}{X}
\addtolength{\leftmargin}{\labelsep}
\setlength{\labelwidth}{\tmplength}
}
\begin{flushleft}
\item[\textbf{Declaração}\hfill]
\begin{ttfamily}
public const CharListComboBox  = {\^{}}C;\end{ttfamily}


\end{flushleft}
\par
\item[\textbf{Descrição}]
A contante \textbf{\begin{ttfamily}CharListComboBox\end{ttfamily}} indica que o campo corrente possuem uma lista de opções do mesmo tipo campo.

\begin{itemize}
\item \textbf{EXEMPLO DE USO}

\texttt{\\\nopagebreak[3]
\\\nopagebreak[3]
NewSItem('~Dia~de~vencimento:~{\textbackslash}sssssssssss'+ChFN+'Dia'+\\\nopagebreak[3]
~~~~~~~~~~CreateOptions(2,NewSItem('Dia~10',\\\nopagebreak[3]
~~~~~~~~~~~~~~~~~~~~~~~~~~NewSItem('Dia~15',\\\nopagebreak[3]
~~~~~~~~~~~~~~~~~~~~~~~~~~NewSItem('Dia~20',\\\nopagebreak[3]
~~~~~~~~~~~~~~~~~~~~~~~~~~NewSItem('Dia~25~e~26',\\\nopagebreak[3]
~~~~~~~~~~~~~~~~~~~~~~~~}\textbf{nil}\texttt{)))))+\\\nopagebreak[3]
~~~~~~~~~~CharHint+'O~template~do~campo~deve~ser~do~tamanho~do~maior~item~da~lista.'~+\\\nopagebreak[3]
~~~~~~~~~~'~~dias~',\\\nopagebreak[3]
~~~~~~~~}\textbf{nil}\texttt{);\\
}
\item \textbf{NOTA} \begin{itemize}
\item O template do campo deve ser do tamanho do maior item da lista.
\end{itemize}
\end{itemize}

\end{list}
\paragraph*{ChLCB}\hspace*{\fill}

\begin{list}{}{
\settowidth{\tmplength}{\textbf{Declaração}}
\setlength{\itemindent}{0cm}
\setlength{\listparindent}{0cm}
\setlength{\leftmargin}{\evensidemargin}
\addtolength{\leftmargin}{\tmplength}
\settowidth{\labelsep}{X}
\addtolength{\leftmargin}{\labelsep}
\setlength{\labelwidth}{\tmplength}
}
\begin{flushleft}
\item[\textbf{Declaração}\hfill]
\begin{ttfamily}
public const ChLCB  = CharListComboBox;\end{ttfamily}


\end{flushleft}
\par
\item[\textbf{Descrição}]
A contante \textbf{\begin{ttfamily}ChLCB\end{ttfamily}} é igual a \begin{ttfamily}CharListComboBox\end{ttfamily}(\ref{mi.rtl.Consts.TConsts-CharListComboBox})

\end{list}
\paragraph*{TypeDate}\hspace*{\fill}

\begin{list}{}{
\settowidth{\tmplength}{\textbf{Declaração}}
\setlength{\itemindent}{0cm}
\setlength{\listparindent}{0cm}
\setlength{\leftmargin}{\evensidemargin}
\addtolength{\leftmargin}{\tmplength}
\settowidth{\labelsep}{X}
\addtolength{\leftmargin}{\labelsep}
\setlength{\labelwidth}{\tmplength}
}
\begin{flushleft}
\item[\textbf{Declaração}\hfill]
\begin{ttfamily}
public const TypeDate        = '{\textbackslash} ZB'+{\^{}}F+{\^{}}U+AnsiChar(31)+{\#}0+'/'+'ZB'+{\^{}}U+AnsiChar(12)+{\#}0+'/'+'ZB'+{\#}0+{\^{}}F;\end{ttfamily}


\end{flushleft}
\par
\item[\textbf{Descrição}]
the same Date Field with a Day/Month/Year sequence

\end{list}
\paragraph*{{\_}TypeDate}\hspace*{\fill}

\begin{list}{}{
\settowidth{\tmplength}{\textbf{Declaração}}
\setlength{\itemindent}{0cm}
\setlength{\listparindent}{0cm}
\setlength{\leftmargin}{\evensidemargin}
\addtolength{\leftmargin}{\tmplength}
\settowidth{\labelsep}{X}
\addtolength{\leftmargin}{\labelsep}
\setlength{\labelwidth}{\tmplength}
}
\begin{flushleft}
\item[\textbf{Declaração}\hfill]
\begin{ttfamily}
public const {\_}TypeDate       = '{\textbackslash} ZB'+{\^{}}F+{\^{}}U+AnsiChar(31)+{\#}0+'/'+'ZB'{\^{}}U+AnsiChar(12)+{\#}0+'/'+'ZB'+{\^{}}F;\end{ttfamily}


\end{flushleft}
\end{list}
\paragraph*{TypeHora}\hspace*{\fill}

\begin{list}{}{
\settowidth{\tmplength}{\textbf{Declaração}}
\setlength{\itemindent}{0cm}
\setlength{\listparindent}{0cm}
\setlength{\leftmargin}{\evensidemargin}
\addtolength{\leftmargin}{\tmplength}
\settowidth{\labelsep}{X}
\addtolength{\leftmargin}{\labelsep}
\setlength{\labelwidth}{\tmplength}
}
\begin{flushleft}
\item[\textbf{Declaração}\hfill]
\begin{ttfamily}
public const TypeHora        = '{\textbackslash} ZB'+{\^{}}F+{\^{}}U+AnsiChar(24)+{\#}0+':'+'ZB'{\^{}}U+AnsiChar(60)+{\#}0+':'+'ZB'{\^{}}U+AnsiChar(60)+{\#}0+{\^{}}F;\end{ttfamily}


\end{flushleft}
\end{list}
\paragraph*{FldMemo}\hspace*{\fill}

\begin{list}{}{
\settowidth{\tmplength}{\textbf{Declaração}}
\setlength{\itemindent}{0cm}
\setlength{\listparindent}{0cm}
\setlength{\leftmargin}{\evensidemargin}
\addtolength{\leftmargin}{\tmplength}
\settowidth{\labelsep}{X}
\addtolength{\leftmargin}{\labelsep}
\setlength{\labelwidth}{\tmplength}
}
\begin{flushleft}
\item[\textbf{Declaração}\hfill]
\begin{ttfamily}
public const FldMemo         = 'M';\end{ttfamily}


\end{flushleft}
\end{list}
\paragraph*{TypeMemo}\hspace*{\fill}

\begin{list}{}{
\settowidth{\tmplength}{\textbf{Declaração}}
\setlength{\itemindent}{0cm}
\setlength{\listparindent}{0cm}
\setlength{\leftmargin}{\evensidemargin}
\addtolength{\leftmargin}{\tmplength}
\settowidth{\labelsep}{X}
\addtolength{\leftmargin}{\labelsep}
\setlength{\labelwidth}{\tmplength}
}
\begin{flushleft}
\item[\textbf{Declaração}\hfill]
\begin{ttfamily}
public const TypeMemo        = '{\textbackslash}ZB'+{\^{}}F+{\#}0'ssssssssss'{\#}0'ZZZZZZZL'{\#}0'ZZZZW'{\#}0'ZZZZW'+{\#}0+{\^{}}F;\end{ttfamily}


\end{flushleft}
\par
\item[\textbf{Descrição}]
Usado em conjunto com \begin{ttfamily}FldBLob\end{ttfamily}(\ref{mi.rtl.Consts.TConsts-fldBLOb})

\end{list}
\paragraph*{CTypeReal}\hspace*{\fill}

\begin{list}{}{
\settowidth{\tmplength}{\textbf{Declaração}}
\setlength{\itemindent}{0cm}
\setlength{\listparindent}{0cm}
\setlength{\leftmargin}{\evensidemargin}
\addtolength{\leftmargin}{\tmplength}
\settowidth{\labelsep}{X}
\addtolength{\leftmargin}{\labelsep}
\setlength{\labelwidth}{\tmplength}
}
\begin{flushleft}
\item[\textbf{Declaração}\hfill]
\begin{ttfamily}
public const CTypeReal       =  [fldRealNum,fldReal4,fldReal4P,fldRealNum{\_}Positivo,fldExtended];\end{ttfamily}


\end{flushleft}
\end{list}
\paragraph*{CTypeAnsiChar}\hspace*{\fill}

\begin{list}{}{
\settowidth{\tmplength}{\textbf{Declaração}}
\setlength{\itemindent}{0cm}
\setlength{\listparindent}{0cm}
\setlength{\leftmargin}{\evensidemargin}
\addtolength{\leftmargin}{\tmplength}
\settowidth{\labelsep}{X}
\addtolength{\leftmargin}{\labelsep}
\setlength{\labelwidth}{\tmplength}
}
\begin{flushleft}
\item[\textbf{Declaração}\hfill]
\begin{ttfamily}
public const CTypeAnsiChar   =  [fldAnsiChar,fldAnsiChar{\_}Minuscula,fldAnsiCharVAL];\end{ttfamily}


\end{flushleft}
\end{list}
\paragraph*{CTypeString}\hspace*{\fill}

\begin{list}{}{
\settowidth{\tmplength}{\textbf{Declaração}}
\setlength{\itemindent}{0cm}
\setlength{\listparindent}{0cm}
\setlength{\leftmargin}{\evensidemargin}
\addtolength{\leftmargin}{\tmplength}
\settowidth{\labelsep}{X}
\addtolength{\leftmargin}{\labelsep}
\setlength{\labelwidth}{\tmplength}
}
\begin{flushleft}
\item[\textbf{Declaração}\hfill]
\begin{ttfamily}
public const CTypeString     =  [fldSTRNUM,fldSTR,fldSTR{\_}Minuscula];\end{ttfamily}


\end{flushleft}
\end{list}
\paragraph*{CTypeInteger}\hspace*{\fill}

\begin{list}{}{
\settowidth{\tmplength}{\textbf{Declaração}}
\setlength{\itemindent}{0cm}
\setlength{\listparindent}{0cm}
\setlength{\leftmargin}{\evensidemargin}
\addtolength{\leftmargin}{\tmplength}
\settowidth{\labelsep}{X}
\addtolength{\leftmargin}{\labelsep}
\setlength{\labelwidth}{\tmplength}
}
\begin{flushleft}
\item[\textbf{Declaração}\hfill]
\begin{ttfamily}
public const CTypeInteger    =  [fldENUM,fldBOOLEAN,fldBYTE,fldSHORTINT,fldSmallWORD,fldSmallInt,fldLONGINT,FldRadioButton];\end{ttfamily}


\end{flushleft}
\end{list}
\paragraph*{CTypeDate}\hspace*{\fill}

\begin{list}{}{
\settowidth{\tmplength}{\textbf{Declaração}}
\setlength{\itemindent}{0cm}
\setlength{\listparindent}{0cm}
\setlength{\leftmargin}{\evensidemargin}
\addtolength{\leftmargin}{\tmplength}
\settowidth{\labelsep}{X}
\addtolength{\leftmargin}{\labelsep}
\setlength{\labelwidth}{\tmplength}
}
\begin{flushleft}
\item[\textbf{Declaração}\hfill]
\begin{ttfamily}
public const CTypeDate       =  [fldData,fldLData,fld{\_}LData,FldDateTimeDos];\end{ttfamily}


\end{flushleft}
\end{list}
\paragraph*{CTypeHour}\hspace*{\fill}

\begin{list}{}{
\settowidth{\tmplength}{\textbf{Declaração}}
\setlength{\itemindent}{0cm}
\setlength{\listparindent}{0cm}
\setlength{\leftmargin}{\evensidemargin}
\addtolength{\leftmargin}{\tmplength}
\settowidth{\labelsep}{X}
\addtolength{\leftmargin}{\labelsep}
\setlength{\labelwidth}{\tmplength}
}
\begin{flushleft}
\item[\textbf{Declaração}\hfill]
\begin{ttfamily}
public const CTypeHour       =  [fldLHora,fld{\_}LHora];\end{ttfamily}


\end{flushleft}
\end{list}
\paragraph*{CTypeBlob}\hspace*{\fill}

\begin{list}{}{
\settowidth{\tmplength}{\textbf{Declaração}}
\setlength{\itemindent}{0cm}
\setlength{\listparindent}{0cm}
\setlength{\leftmargin}{\evensidemargin}
\addtolength{\leftmargin}{\tmplength}
\settowidth{\labelsep}{X}
\addtolength{\leftmargin}{\labelsep}
\setlength{\labelwidth}{\tmplength}
}
\begin{flushleft}
\item[\textbf{Declaração}\hfill]
\begin{ttfamily}
public const CTypeBlob       =  [FldMemo,fldBLOb];\end{ttfamily}


\end{flushleft}
\end{list}
\paragraph*{CTypeOperator}\hspace*{\fill}

\begin{list}{}{
\settowidth{\tmplength}{\textbf{Declaração}}
\setlength{\itemindent}{0cm}
\setlength{\listparindent}{0cm}
\setlength{\leftmargin}{\evensidemargin}
\addtolength{\leftmargin}{\tmplength}
\settowidth{\labelsep}{X}
\addtolength{\leftmargin}{\labelsep}
\setlength{\labelwidth}{\tmplength}
}
\begin{flushleft}
\item[\textbf{Declaração}\hfill]
\begin{ttfamily}
public const CTypeOperator   =  [FldOperador];\end{ttfamily}


\end{flushleft}
\end{list}
\paragraph*{CTypeKnown}\hspace*{\fill}

\begin{list}{}{
\settowidth{\tmplength}{\textbf{Declaração}}
\setlength{\itemindent}{0cm}
\setlength{\listparindent}{0cm}
\setlength{\leftmargin}{\evensidemargin}
\addtolength{\leftmargin}{\tmplength}
\settowidth{\labelsep}{X}
\addtolength{\leftmargin}{\labelsep}
\setlength{\labelwidth}{\tmplength}
}
\begin{flushleft}
\item[\textbf{Declaração}\hfill]
\begin{ttfamily}
public const CTypeKnown      : AnsiCharSet = CTypeReal
                                              + CTypeAnsiChar
                                              + CTypeString
                                              + CTypeInteger
                                              + CTypeDate
                                              + CTypeHour
                                              + CTypeBlob
                                              + CTypeOperator;\end{ttfamily}


\end{flushleft}
\end{list}
\paragraph*{efSync}\hspace*{\fill}

\begin{list}{}{
\settowidth{\tmplength}{\textbf{Declaração}}
\setlength{\itemindent}{0cm}
\setlength{\listparindent}{0cm}
\setlength{\leftmargin}{\evensidemargin}
\addtolength{\leftmargin}{\tmplength}
\settowidth{\labelsep}{X}
\addtolength{\leftmargin}{\labelsep}
\setlength{\labelwidth}{\tmplength}
}
\begin{flushleft}
\item[\textbf{Declaração}\hfill]
\begin{ttfamily}
public const efSync  = 0;\end{ttfamily}


\end{flushleft}
\end{list}
\paragraph*{efAsync}\hspace*{\fill}

\begin{list}{}{
\settowidth{\tmplength}{\textbf{Declaração}}
\setlength{\itemindent}{0cm}
\setlength{\listparindent}{0cm}
\setlength{\leftmargin}{\evensidemargin}
\addtolength{\leftmargin}{\tmplength}
\settowidth{\labelsep}{X}
\addtolength{\leftmargin}{\labelsep}
\setlength{\labelwidth}{\tmplength}
}
\begin{flushleft}
\item[\textbf{Declaração}\hfill]
\begin{ttfamily}
public const efAsync = 1;\end{ttfamily}


\end{flushleft}
\end{list}
\paragraph*{SW{\_}SHOWNORMAL}\hspace*{\fill}

\begin{list}{}{
\settowidth{\tmplength}{\textbf{Declaração}}
\setlength{\itemindent}{0cm}
\setlength{\listparindent}{0cm}
\setlength{\leftmargin}{\evensidemargin}
\addtolength{\leftmargin}{\tmplength}
\settowidth{\labelsep}{X}
\addtolength{\leftmargin}{\labelsep}
\setlength{\labelwidth}{\tmplength}
}
\begin{flushleft}
\item[\textbf{Declaração}\hfill]
\begin{ttfamily}
public const SW{\_}SHOWNORMAL : integer = ord(swoShow);\end{ttfamily}


\end{flushleft}
\end{list}
\paragraph*{Password{\_}Admin}\hspace*{\fill}

\begin{list}{}{
\settowidth{\tmplength}{\textbf{Declaração}}
\setlength{\itemindent}{0cm}
\setlength{\listparindent}{0cm}
\setlength{\leftmargin}{\evensidemargin}
\addtolength{\leftmargin}{\tmplength}
\settowidth{\labelsep}{X}
\addtolength{\leftmargin}{\labelsep}
\setlength{\labelwidth}{\tmplength}
}
\begin{flushleft}
\item[\textbf{Declaração}\hfill]
\begin{ttfamily}
public const Password{\_}Admin : string = '123456';\end{ttfamily}


\end{flushleft}
\par
\item[\textbf{Descrição}]
0 = Indica uso normal do produto; 1= Indica que a \begin{ttfamily}Password{\_}admin\end{ttfamily} esta logado

\end{list}
\paragraph*{Admin{\_}Logado}\hspace*{\fill}

\begin{list}{}{
\settowidth{\tmplength}{\textbf{Declaração}}
\setlength{\itemindent}{0cm}
\setlength{\listparindent}{0cm}
\setlength{\leftmargin}{\evensidemargin}
\addtolength{\leftmargin}{\tmplength}
\settowidth{\labelsep}{X}
\addtolength{\leftmargin}{\labelsep}
\setlength{\labelwidth}{\tmplength}
}
\begin{flushleft}
\item[\textbf{Declaração}\hfill]
\begin{ttfamily}
public const Admin{\_}Logado : SmallWord = 0;\end{ttfamily}


\end{flushleft}
\end{list}
\paragraph*{FileModeDenyALL}\hspace*{\fill}

\begin{list}{}{
\settowidth{\tmplength}{\textbf{Declaração}}
\setlength{\itemindent}{0cm}
\setlength{\listparindent}{0cm}
\setlength{\leftmargin}{\evensidemargin}
\addtolength{\leftmargin}{\tmplength}
\settowidth{\labelsep}{X}
\addtolength{\leftmargin}{\labelsep}
\setlength{\labelwidth}{\tmplength}
}
\begin{flushleft}
\item[\textbf{Declaração}\hfill]
\begin{ttfamily}
public const FileModeDenyALL : Boolean = False;\end{ttfamily}


\end{flushleft}
\par
\item[\textbf{Descrição}]
Indica se o arquivo e exclusivo. Usado em Set{\_}FileModeDenyALL

\end{list}
\paragraph*{FlushBuffer}\hspace*{\fill}

\begin{list}{}{
\settowidth{\tmplength}{\textbf{Declaração}}
\setlength{\itemindent}{0cm}
\setlength{\listparindent}{0cm}
\setlength{\leftmargin}{\evensidemargin}
\addtolength{\leftmargin}{\tmplength}
\settowidth{\labelsep}{X}
\addtolength{\leftmargin}{\labelsep}
\setlength{\labelwidth}{\tmplength}
}
\begin{flushleft}
\item[\textbf{Declaração}\hfill]
\begin{ttfamily}
public const FlushBuffer     : Boolean = true;\end{ttfamily}


\end{flushleft}
\par
\item[\textbf{Descrição}]
\begin{itemize}
\item A constante \begin{ttfamily}FlushBuffer\end{ttfamily} dar opção para usar cache de disco ou não.

\begin{itemize}
\item \textbf{NOTA} \begin{itemize}
\item Se \textbf{True} então executa executa \textbf{FlushDOSFile} apos atualização dos arquivos.
\end{itemize}
\end{itemize}
\end{itemize}

\end{list}
\paragraph*{FlushBuffer{\_}Disk}\hspace*{\fill}

\begin{list}{}{
\settowidth{\tmplength}{\textbf{Declaração}}
\setlength{\itemindent}{0cm}
\setlength{\listparindent}{0cm}
\setlength{\leftmargin}{\evensidemargin}
\addtolength{\leftmargin}{\tmplength}
\settowidth{\labelsep}{X}
\addtolength{\leftmargin}{\labelsep}
\setlength{\labelwidth}{\tmplength}
}
\begin{flushleft}
\item[\textbf{Declaração}\hfill]
\begin{ttfamily}
public const FlushBuffer{\_}Disk : Boolean = False;\end{ttfamily}


\end{flushleft}
\par
\item[\textbf{Descrição}]
\begin{itemize}
\item A constante \begin{ttfamily}FlushBuffer{\_}Disk\end{ttfamily} é usado para indicar a banco de dados MarIcarai se deve usar cache de disco ou não. \begin{itemize}
\item \textbf{NOTA} \begin{itemize}
\item Se \textbf{True} executa executa \textbf{SysFileFlushBuffers} após atualização dos arquivos.
\end{itemize}
\end{itemize}
\end{itemize}

\end{list}
\paragraph*{FlushBuffer{\_}Disk{\_}Transaction}\hspace*{\fill}

\begin{list}{}{
\settowidth{\tmplength}{\textbf{Declaração}}
\setlength{\itemindent}{0cm}
\setlength{\listparindent}{0cm}
\setlength{\leftmargin}{\evensidemargin}
\addtolength{\leftmargin}{\tmplength}
\settowidth{\labelsep}{X}
\addtolength{\leftmargin}{\labelsep}
\setlength{\labelwidth}{\tmplength}
}
\begin{flushleft}
\item[\textbf{Declaração}\hfill]
\begin{ttfamily}
public const FlushBuffer{\_}Disk{\_}Transaction : Boolean = False;\end{ttfamily}


\end{flushleft}
\par
\item[\textbf{Descrição}]
\begin{itemize}
\item A constante \begin{ttfamily}FlushBuffer{\_}Disk{\_}Transaction\end{ttfamily} é usado para indicar ao banco de dados \textbf{MarIcarai} se deve usar cache de disco ou não. \begin{itemize}
\item \textbf{NOTAS} \begin{itemize}
\item False = habilita cache de gravação das transações
\item True = Desabilita cache de gravação das transações
\end{itemize}
\end{itemize}
\end{itemize}

\end{list}
\paragraph*{OkTempoDeTentativas}\hspace*{\fill}

\begin{list}{}{
\settowidth{\tmplength}{\textbf{Declaração}}
\setlength{\itemindent}{0cm}
\setlength{\listparindent}{0cm}
\setlength{\leftmargin}{\evensidemargin}
\addtolength{\leftmargin}{\tmplength}
\settowidth{\labelsep}{X}
\addtolength{\leftmargin}{\labelsep}
\setlength{\labelwidth}{\tmplength}
}
\begin{flushleft}
\item[\textbf{Declaração}\hfill]
\begin{ttfamily}
public const OkTempoDeTentativas : Boolean = true;\end{ttfamily}


\end{flushleft}
\par
\item[\textbf{Descrição}]
\begin{itemize}
\item A constante \begin{ttfamily}OkTempoDeTentativas\end{ttfamily} habilita o loop \begin{ttfamily}TempoDeTentativas\end{ttfamily}(\ref{mi.rtl.Consts.TConsts-TempoDeTentativas}) nas leitura e escritas ao arquivo.
\end{itemize}

\end{list}
\paragraph*{TempoDeTentativas}\hspace*{\fill}

\begin{list}{}{
\settowidth{\tmplength}{\textbf{Declaração}}
\setlength{\itemindent}{0cm}
\setlength{\listparindent}{0cm}
\setlength{\leftmargin}{\evensidemargin}
\addtolength{\leftmargin}{\tmplength}
\settowidth{\labelsep}{X}
\addtolength{\leftmargin}{\labelsep}
\setlength{\labelwidth}{\tmplength}
}
\begin{flushleft}
\item[\textbf{Declaração}\hfill]
\begin{ttfamily}
public const TempoDeTentativas   : Longint = 30;\end{ttfamily}


\end{flushleft}
\par
\item[\textbf{Descrição}]
\begin{itemize}
\item A constante \begin{ttfamily}TempoDeTentativas\end{ttfamily} é o tempo em segundos de tentativos nos processos de abertura, leitura e gravação de arquivos.
\end{itemize}

\end{list}
\paragraph*{UAnsiChar}\hspace*{\fill}

\begin{list}{}{
\settowidth{\tmplength}{\textbf{Declaração}}
\setlength{\itemindent}{0cm}
\setlength{\listparindent}{0cm}
\setlength{\leftmargin}{\evensidemargin}
\addtolength{\leftmargin}{\tmplength}
\settowidth{\labelsep}{X}
\addtolength{\leftmargin}{\labelsep}
\setlength{\labelwidth}{\tmplength}
}
\begin{flushleft}
\item[\textbf{Declaração}\hfill]
\begin{ttfamily}
public const UAnsiChar : AnsiChar = ' ';\end{ttfamily}


\end{flushleft}
\par
\item[\textbf{Descrição}]
Último caractere digitado

\end{list}
\paragraph*{TeclaF}\hspace*{\fill}

\begin{list}{}{
\settowidth{\tmplength}{\textbf{Declaração}}
\setlength{\itemindent}{0cm}
\setlength{\listparindent}{0cm}
\setlength{\leftmargin}{\evensidemargin}
\addtolength{\leftmargin}{\tmplength}
\settowidth{\labelsep}{X}
\addtolength{\leftmargin}{\labelsep}
\setlength{\labelwidth}{\tmplength}
}
\begin{flushleft}
\item[\textbf{Declaração}\hfill]
\begin{ttfamily}
public const TeclaF    : SmallInt = 0;\end{ttfamily}


\end{flushleft}
\par
\item[\textbf{Descrição}]
Usado em readKey para capturar as Teclas Alt, Ctrl, Shift etc.

\end{list}
\paragraph*{Identification}\hspace*{\fill}

\begin{list}{}{
\settowidth{\tmplength}{\textbf{Declaração}}
\setlength{\itemindent}{0cm}
\setlength{\listparindent}{0cm}
\setlength{\leftmargin}{\evensidemargin}
\addtolength{\leftmargin}{\tmplength}
\settowidth{\labelsep}{X}
\addtolength{\leftmargin}{\labelsep}
\setlength{\labelwidth}{\tmplength}
}
\begin{flushleft}
\item[\textbf{Declaração}\hfill]
\begin{ttfamily}
public const Identification: TIdentification = (
                                              
                                              Id{\_}branch       : 0;
                                              
                                              Id{\_}user         : 1;
                                              
                                              UserName        : '';
                                              
                                              FullUserName    : '';
                                              
                                              password        : '';
                                            );\end{ttfamily}


\end{flushleft}
\par
\item[\textbf{Descrição}]
\begin{itemize}
\item A constante \textbf{\begin{ttfamily}Identification\end{ttfamily}} é usada para manter os dados do usuário logado ao sistema.

\begin{itemize}
\item Id{\_}branch : 0; //Número da filial do usuário logado
\item Id{\_}user : 1; // Número do usuário Logado;
\item UserName : 'PauloSSPacheco'; // Nome do usuário logado
\item FullUserName : ''; //: Nome completo do usuário logado
\item password : ''; //: Password do usuário logado
\end{itemize}
\end{itemize}

\end{list}
\paragraph*{TimeCmTime}\hspace*{\fill}

\begin{list}{}{
\settowidth{\tmplength}{\textbf{Declaração}}
\setlength{\itemindent}{0cm}
\setlength{\listparindent}{0cm}
\setlength{\leftmargin}{\evensidemargin}
\addtolength{\leftmargin}{\tmplength}
\settowidth{\labelsep}{X}
\addtolength{\leftmargin}{\labelsep}
\setlength{\labelwidth}{\tmplength}
}
\begin{flushleft}
\item[\textbf{Declaração}\hfill]
\begin{ttfamily}
public const TimeCmTime : Longint = 10;\end{ttfamily}


\end{flushleft}
\par
\item[\textbf{Descrição}]
Número da filial do usuário logado Número do usuário Logado; Nome do usuário logado Nome completo do usuário logado Password do usuário logado

\end{list}
\paragraph*{CTRL{\_}SLEEP{\_}ENABLE}\hspace*{\fill}

\begin{list}{}{
\settowidth{\tmplength}{\textbf{Declaração}}
\setlength{\itemindent}{0cm}
\setlength{\listparindent}{0cm}
\setlength{\leftmargin}{\evensidemargin}
\addtolength{\leftmargin}{\tmplength}
\settowidth{\labelsep}{X}
\addtolength{\leftmargin}{\labelsep}
\setlength{\labelwidth}{\tmplength}
}
\begin{flushleft}
\item[\textbf{Declaração}\hfill]
\begin{ttfamily}
public const CTRL{\_}SLEEP{\_}ENABLE : Boolean = True;\end{ttfamily}


\end{flushleft}
\par
\item[\textbf{Descrição}]
\begin{itemize}
\item A constante pública global \textbf{\begin{ttfamily}CTRL{\_}SLEEP{\_}ENABLE\end{ttfamily}} indica se o sistema deve executar a aplicação central caso a rotina atual tiver em loop aguardando alguma ação. \begin{itemize}
\item Exemplo: Tentando abrir um arquivo onde o mesmo se encontra dentro de uma transação.
\end{itemize}
\end{itemize}

\end{list}
\paragraph*{FORMS{\_}APPLICATION{\_}PROCESS{\_}MESSAGES}\hspace*{\fill}

\begin{list}{}{
\settowidth{\tmplength}{\textbf{Declaração}}
\setlength{\itemindent}{0cm}
\setlength{\listparindent}{0cm}
\setlength{\leftmargin}{\evensidemargin}
\addtolength{\leftmargin}{\tmplength}
\settowidth{\labelsep}{X}
\addtolength{\leftmargin}{\labelsep}
\setlength{\labelwidth}{\tmplength}
}
\begin{flushleft}
\item[\textbf{Declaração}\hfill]
\begin{ttfamily}
public const FORMS{\_}APPLICATION{\_}PROCESS{\_}MESSAGES : Boolean = false;\end{ttfamily}


\end{flushleft}
\par
\item[\textbf{Descrição}]
\begin{itemize}
\item A contante \textbf{\begin{ttfamily}FORMS{\_}APPLICATION{\_}PROCESS{\_}MESSAGES\end{ttfamily}} indica se deve ou não executar a aplicação principal; \begin{enumerate}
\setcounter{enumi}{0} \setcounter{enumii}{0} \setcounter{enumiii}{0} \setcounter{enumiv}{0} 
\item True : Processa as mensagem da aplicação gráfica quando necessário;
\setcounter{enumi}{1} \setcounter{enumii}{1} \setcounter{enumiii}{1} \setcounter{enumiv}{1} 
\item False : ignora.
\end{enumerate}
\end{itemize}

\end{list}
\paragraph*{FORMS{\_}APPLICATION{\_}SHOW{\_}MODAL}\hspace*{\fill}

\begin{list}{}{
\settowidth{\tmplength}{\textbf{Declaração}}
\setlength{\itemindent}{0cm}
\setlength{\listparindent}{0cm}
\setlength{\leftmargin}{\evensidemargin}
\addtolength{\leftmargin}{\tmplength}
\settowidth{\labelsep}{X}
\addtolength{\leftmargin}{\labelsep}
\setlength{\labelwidth}{\tmplength}
}
\begin{flushleft}
\item[\textbf{Declaração}\hfill]
\begin{ttfamily}
public const FORMS{\_}APPLICATION{\_}SHOW{\_}MODAL       : Boolean = false;\end{ttfamily}


\end{flushleft}
\par
\item[\textbf{Descrição}]
\begin{itemize}
\item Se a constante \textbf{\begin{ttfamily}FORMS{\_}APPLICATION{\_}SHOW{\_}MODAL\end{ttfamily}=True} indica que \textbf{\begin{ttfamily}FORMS{\_}APPLICATION{\_}PROCESS{\_}MESSAGES\end{ttfamily}(\ref{mi.rtl.Consts.TConsts-FORMS_APPLICATION_PROCESS_MESSAGES})=true} e o método \textbf{TForm{\_}VCL{\_}DmxEditor.ShowModal} está em execução.
\end{itemize}

\end{list}
\paragraph*{HANDLE{\_}INVALID}\hspace*{\fill}

\begin{list}{}{
\settowidth{\tmplength}{\textbf{Declaração}}
\setlength{\itemindent}{0cm}
\setlength{\listparindent}{0cm}
\setlength{\leftmargin}{\evensidemargin}
\addtolength{\leftmargin}{\tmplength}
\settowidth{\labelsep}{X}
\addtolength{\leftmargin}{\labelsep}
\setlength{\labelwidth}{\tmplength}
}
\begin{flushleft}
\item[\textbf{Declaração}\hfill]
\begin{ttfamily}
public const HANDLE{\_}INVALID = high(THandle);\end{ttfamily}


\end{flushleft}
\par
\item[\textbf{Descrição}]
\begin{itemize}
\item A constante \textbf{\begin{ttfamily}HANDLE{\_}INVALID\end{ttfamily}} é usada para checar se um handle de um dispositivo é válido ou não
\end{itemize}

\end{list}
\paragraph*{LF}\hspace*{\fill}

\begin{list}{}{
\settowidth{\tmplength}{\textbf{Declaração}}
\setlength{\itemindent}{0cm}
\setlength{\listparindent}{0cm}
\setlength{\leftmargin}{\evensidemargin}
\addtolength{\leftmargin}{\tmplength}
\settowidth{\labelsep}{X}
\addtolength{\leftmargin}{\labelsep}
\setlength{\labelwidth}{\tmplength}
}
\begin{flushleft}
\item[\textbf{Declaração}\hfill]
\begin{ttfamily}
public const LF = {\#}10;\end{ttfamily}


\end{flushleft}
\end{list}
\paragraph*{CR}\hspace*{\fill}

\begin{list}{}{
\settowidth{\tmplength}{\textbf{Declaração}}
\setlength{\itemindent}{0cm}
\setlength{\listparindent}{0cm}
\setlength{\leftmargin}{\evensidemargin}
\addtolength{\leftmargin}{\tmplength}
\settowidth{\labelsep}{X}
\addtolength{\leftmargin}{\labelsep}
\setlength{\labelwidth}{\tmplength}
}
\begin{flushleft}
\item[\textbf{Declaração}\hfill]
\begin{ttfamily}
public const CR = {\#}13;\end{ttfamily}


\end{flushleft}
\end{list}
\paragraph*{New{\_}Line}\hspace*{\fill}

\begin{list}{}{
\settowidth{\tmplength}{\textbf{Declaração}}
\setlength{\itemindent}{0cm}
\setlength{\listparindent}{0cm}
\setlength{\leftmargin}{\evensidemargin}
\addtolength{\leftmargin}{\tmplength}
\settowidth{\labelsep}{X}
\addtolength{\leftmargin}{\labelsep}
\setlength{\labelwidth}{\tmplength}
}
\begin{flushleft}
\item[\textbf{Declaração}\hfill]
\begin{ttfamily}
public const New{\_}Line =  ;\end{ttfamily}


\end{flushleft}
\par
\item[\textbf{Descrição}]
\begin{itemize}
\item A constantes \textbf{\begin{ttfamily}New{\_}Line\end{ttfamily}} é usado em \textbf{writeln} para passar a linha.
\end{itemize}

\end{list}
\paragraph*{fmOpenRead}\hspace*{\fill}

\begin{list}{}{
\settowidth{\tmplength}{\textbf{Declaração}}
\setlength{\itemindent}{0cm}
\setlength{\listparindent}{0cm}
\setlength{\leftmargin}{\evensidemargin}
\addtolength{\leftmargin}{\tmplength}
\settowidth{\labelsep}{X}
\addtolength{\leftmargin}{\labelsep}
\setlength{\labelwidth}{\tmplength}
}
\begin{flushleft}
\item[\textbf{Declaração}\hfill]
\begin{ttfamily}
public const fmOpenRead  = SysUtils.fmOpenRead  ;\end{ttfamily}


\end{flushleft}
\par
\item[\textbf{Descrição}]
\begin{itemize}
\item Abre um arquivo com acesso somente leitura.

\begin{itemize}
\item Mapa de bits : 0 = Bit 0 desligado.
\item REFERÊNCIA: \begin{itemize}
\item [\begin{ttfamily}fmopenread\end{ttfamily}](https://www.freepascal.org/docs-html/rtl/sysutils/fmopenread.html)
\end{itemize}
\end{itemize}
\end{itemize}

\end{list}
\paragraph*{fmOpenWrite}\hspace*{\fill}

\begin{list}{}{
\settowidth{\tmplength}{\textbf{Declaração}}
\setlength{\itemindent}{0cm}
\setlength{\listparindent}{0cm}
\setlength{\leftmargin}{\evensidemargin}
\addtolength{\leftmargin}{\tmplength}
\settowidth{\labelsep}{X}
\addtolength{\leftmargin}{\labelsep}
\setlength{\labelwidth}{\tmplength}
}
\begin{flushleft}
\item[\textbf{Declaração}\hfill]
\begin{ttfamily}
public const fmOpenWrite = SysUtils.fmOpenWrite    ;\end{ttfamily}


\end{flushleft}
\par
\item[\textbf{Descrição}]
\begin{itemize}
\item Abre um arquivo com acesso somente gravação.

\begin{itemize}
\item Mapa de bits : 1 = Bit 0 ligado
\item REFERÊNCIA: \begin{itemize}
\item [\begin{ttfamily}fmopenwrite\end{ttfamily}](https://www.freepascal.org/docs-html/rtl/sysutils/fmopenwrite.html)
\end{itemize}
\end{itemize}
\end{itemize}

\end{list}
\paragraph*{fmOpenReadWrite}\hspace*{\fill}

\begin{list}{}{
\settowidth{\tmplength}{\textbf{Declaração}}
\setlength{\itemindent}{0cm}
\setlength{\listparindent}{0cm}
\setlength{\leftmargin}{\evensidemargin}
\addtolength{\leftmargin}{\tmplength}
\settowidth{\labelsep}{X}
\addtolength{\leftmargin}{\labelsep}
\setlength{\labelwidth}{\tmplength}
}
\begin{flushleft}
\item[\textbf{Declaração}\hfill]
\begin{ttfamily}
public const fmOpenReadWrite  = SysUtils.fmOpenReadWrite;\end{ttfamily}


\end{flushleft}
\par
\item[\textbf{Descrição}]
\begin{itemize}
\item Abre um arquivo com acesso de leitura e gravação

\begin{itemize}
\item Mapa de bits : 10 = Bit 1 ligado
\item REFERÊNCIA: \begin{itemize}
\item [\begin{ttfamily}fmopenreadwrite\end{ttfamily}](https://www.freepascal.org/docs-html/rtl/sysutils/fmopenreadwrite.html)
\end{itemize}
\end{itemize}
\end{itemize}

\end{list}
\paragraph*{fmShareCompat}\hspace*{\fill}

\begin{list}{}{
\settowidth{\tmplength}{\textbf{Declaração}}
\setlength{\itemindent}{0cm}
\setlength{\listparindent}{0cm}
\setlength{\leftmargin}{\evensidemargin}
\addtolength{\leftmargin}{\tmplength}
\settowidth{\labelsep}{X}
\addtolength{\leftmargin}{\labelsep}
\setlength{\labelwidth}{\tmplength}
}
\begin{flushleft}
\item[\textbf{Declaração}\hfill]
\begin{ttfamily}
public const fmShareCompat    = SysUtils.fmShareCompat   ;\end{ttfamily}


\end{flushleft}
\par
\item[\textbf{Descrição}]
\begin{itemize}
\item A Constante \textbf{\begin{ttfamily}fmShareCompat\end{ttfamily}} é usada na abertura do arquivo indicando no modo de compatibilidade com o DOS

\begin{itemize}
\item Mapa de bits: 0 = bit 0(zero) desligado
\item REFERÊNCIA: \begin{itemize}
\item [\begin{ttfamily}fmshareexclusive\end{ttfamily}(\ref{mi.rtl.Consts.TConsts-fmShareExclusive})](https://www.freepascal.org/docs-html/rtl/sysutils/fmsharecompat.html)
\end{itemize}
\end{itemize}
\end{itemize}

\end{list}
\paragraph*{fmShareExclusive}\hspace*{\fill}

\begin{list}{}{
\settowidth{\tmplength}{\textbf{Declaração}}
\setlength{\itemindent}{0cm}
\setlength{\listparindent}{0cm}
\setlength{\leftmargin}{\evensidemargin}
\addtolength{\leftmargin}{\tmplength}
\settowidth{\labelsep}{X}
\addtolength{\leftmargin}{\labelsep}
\setlength{\labelwidth}{\tmplength}
}
\begin{flushleft}
\item[\textbf{Declaração}\hfill]
\begin{ttfamily}
public const fmShareExclusive = SysUtils.fmShareExclusive;\end{ttfamily}


\end{flushleft}
\par
\item[\textbf{Descrição}]
\begin{itemize}
\item Flag usado para tornar acesso ao arquivo no modo exclusivo.

\begin{itemize}
\item \textbf{NOTA} \begin{itemize}
\item Binário: 10000 = Bit 4 ligado
\item As contantes usadas para abertura de arquivo da \textbf{unit SysUtils} é totalmente diferente das constantes usadas na \textbf{unit system}, por isso o exemplo abaixo não funciona.
\end{itemize}
\item \textbf{REFERÊNCIA}: \begin{itemize}
\item [\begin{ttfamily}fmshareexclusive\end{ttfamily}](https://www.freepascal.org/docs-html/rtl/sysutils/fmshareexclusive.html)
\end{itemize}
\item EXEMPLO:

\texttt{\\\nopagebreak[3]
\\\nopagebreak[3]
~}\textbf{function}\texttt{~TFormTests.fTest{\_}Reset(}\textbf{Var}\texttt{~f~:~}\textbf{file}\texttt{~~):longint;\\\nopagebreak[3]
~}\textbf{Begin}\texttt{\\\nopagebreak[3]
~~~AssignFile(f,'./doc/index.html');\\\nopagebreak[3]
\\\nopagebreak[3]
~\textit{{\{}{\$}i-{\}}}\\\nopagebreak[3]
~Reset(f);\\\nopagebreak[3]
\textit{{\{}{\$}i+{\}}}\\\nopagebreak[3]
~Result~:=~IoResult;\\\nopagebreak[3]
~~~}\textbf{If}\texttt{~Result~{$<$}{$>$}~0\\\nopagebreak[3]
~~~}\textbf{then}\texttt{~ShowMessage('Error:~'+ErrorMessage(result));\\\nopagebreak[3]
~}\textbf{end}\texttt{;\\\nopagebreak[3]
\\\nopagebreak[3]
~}\textbf{procedure}\texttt{~TFormTests.Button{\_}Test{\_}ResetClick(Sender:~TObject);\\\nopagebreak[3]
~~~}\textbf{Var}\texttt{\\\nopagebreak[3]
~~~~~f1,f2~:~}\textbf{file}\texttt{;\\\nopagebreak[3]
~}\textbf{begin}\texttt{\\\nopagebreak[3]
~~~fileMode~:=~fmOpenReadWrite~}\textbf{or}\texttt{~fmShareExclusive~~}\textbf{or}\texttt{~fmShareCompat\\\nopagebreak[3]
\\\nopagebreak[3]
~~~ShowMessage(IntToStr(fileMode));\\\nopagebreak[3]
\\\nopagebreak[3]
~~~}\textbf{if}\texttt{~fTest{\_}Reset(f1)~=~0\\\nopagebreak[3]
~~~}\textbf{Then}\texttt{~fTest{\_}Reset(f2);\\\nopagebreak[3]
\\\nopagebreak[3]
~~~closeFile(f1);\\\nopagebreak[3]
~~~closeFile(f2);\\\nopagebreak[3]
~}\textbf{end}\texttt{;\\
}
\end{itemize}
\end{itemize}

\end{list}
\paragraph*{fmShareDenyWrite}\hspace*{\fill}

\begin{list}{}{
\settowidth{\tmplength}{\textbf{Declaração}}
\setlength{\itemindent}{0cm}
\setlength{\listparindent}{0cm}
\setlength{\leftmargin}{\evensidemargin}
\addtolength{\leftmargin}{\tmplength}
\settowidth{\labelsep}{X}
\addtolength{\leftmargin}{\labelsep}
\setlength{\labelwidth}{\tmplength}
}
\begin{flushleft}
\item[\textbf{Declaração}\hfill]
\begin{ttfamily}
public const fmShareDenyWrite = SysUtils.fmShareDenyWrite;\end{ttfamily}


\end{flushleft}
\par
\item[\textbf{Descrição}]
\begin{itemize}
\item Bloqueie o arquivo para que outros processos possam apenas ler.

\begin{itemize}
\item Mapa de Bit: 100000 = Bit 5 ligado.
\item REFERÊNCIA: \begin{itemize}
\item [\begin{ttfamily}fmsharedenywrite\end{ttfamily}](https://www.freepascal.org/docs-html/rtl/sysutils/fmsharedenywrite.html)
\end{itemize}
\end{itemize}
\end{itemize}

\end{list}
\paragraph*{fmShareDenyRead}\hspace*{\fill}

\begin{list}{}{
\settowidth{\tmplength}{\textbf{Declaração}}
\setlength{\itemindent}{0cm}
\setlength{\listparindent}{0cm}
\setlength{\leftmargin}{\evensidemargin}
\addtolength{\leftmargin}{\tmplength}
\settowidth{\labelsep}{X}
\addtolength{\leftmargin}{\labelsep}
\setlength{\labelwidth}{\tmplength}
}
\begin{flushleft}
\item[\textbf{Declaração}\hfill]
\begin{ttfamily}
public const fmShareDenyRead  = SysUtils.fmShareDenyRead ;\end{ttfamily}


\end{flushleft}
\par
\item[\textbf{Descrição}]
\begin{itemize}
\item Bloqueie o arquivo para que outros processos não possam ler.

\begin{itemize}
\item Mapa de bits: 110000 = Bit 4 e 5 ligado.
\item REFERÊNCIA: \begin{itemize}
\item [\begin{ttfamily}fmsharedenyread\end{ttfamily}](https://www.freepascal.org/docs-html/rtl/sysutils/fmsharedenyread.html)
\end{itemize}
\end{itemize}
\end{itemize}

\end{list}
\paragraph*{fmShareDenyNone}\hspace*{\fill}

\begin{list}{}{
\settowidth{\tmplength}{\textbf{Declaração}}
\setlength{\itemindent}{0cm}
\setlength{\listparindent}{0cm}
\setlength{\leftmargin}{\evensidemargin}
\addtolength{\leftmargin}{\tmplength}
\settowidth{\labelsep}{X}
\addtolength{\leftmargin}{\labelsep}
\setlength{\labelwidth}{\tmplength}
}
\begin{flushleft}
\item[\textbf{Declaração}\hfill]
\begin{ttfamily}
public const fmShareDenyNone  = SysUtils.fmShareDenyNone ;\end{ttfamily}


\end{flushleft}
\par
\item[\textbf{Descrição}]
\begin{itemize}
\item Não bloqueie o arquivo.

\begin{itemize}
\item Mapa de bits: 1000000 = Bit 6 ligado
\item REFERÊNCIA: \begin{itemize}
\item [\begin{ttfamily}fmsharedenynone\end{ttfamily}](https://www.freepascal.org/docs-html/rtl/sysutils/fmsharedenynone.html)
\end{itemize}
\end{itemize}
\end{itemize}

\end{list}
\paragraph*{GLOBAL{\_}RIGHTS}\hspace*{\fill}

\begin{list}{}{
\settowidth{\tmplength}{\textbf{Declaração}}
\setlength{\itemindent}{0cm}
\setlength{\listparindent}{0cm}
\setlength{\leftmargin}{\evensidemargin}
\addtolength{\leftmargin}{\tmplength}
\settowidth{\labelsep}{X}
\addtolength{\leftmargin}{\labelsep}
\setlength{\labelwidth}{\tmplength}
}
\begin{flushleft}
\item[\textbf{Declaração}\hfill]
\begin{ttfamily}
public const GLOBAL{\_}RIGHTS = 0;\end{ttfamily}


\end{flushleft}
\par
\item[\textbf{Descrição}]
\begin{itemize}
\item A constante \begin{ttfamily}GLOBAL{\_}RIGHTS\end{ttfamily} é usada em FileCreate se o sistema for linux.

\begin{itemize}
\item \textbf{REFERÊNCIA} \begin{itemize}
\item https://www.gnu.org/software/libc/manual/html{\_}node/Permission-Bits.html
\end{itemize}
\end{itemize}
\end{itemize}

\end{list}
\paragraph*{faReadOnly}\hspace*{\fill}

\begin{list}{}{
\settowidth{\tmplength}{\textbf{Declaração}}
\setlength{\itemindent}{0cm}
\setlength{\listparindent}{0cm}
\setlength{\leftmargin}{\evensidemargin}
\addtolength{\leftmargin}{\tmplength}
\settowidth{\labelsep}{X}
\addtolength{\leftmargin}{\labelsep}
\setlength{\labelwidth}{\tmplength}
}
\begin{flushleft}
\item[\textbf{Declaração}\hfill]
\begin{ttfamily}
public const faReadOnly   = SysUtils.faReadOnly  ;\end{ttfamily}


\end{flushleft}
\par
\item[\textbf{Descrição}]
\begin{itemize}
\item O atributo \textbf{\begin{ttfamily}faReadOnly\end{ttfamily}} indica que o arquivo é somente para leitura.

\begin{itemize}
\item \textbf{REFERÊNCIA} \begin{itemize}
\item https://www.freepascal.org/docs-html/rtl/sysutils/fareadonly.html
\item https://www.freepascal.org/docs-html/rtl/sysutils/findfirst.html
\end{itemize}
\item \textbf{NOTA} \begin{itemize}
\item Usado em TSearchRec e FindFirst
\end{itemize}
\end{itemize}
\end{itemize}

\end{list}
\paragraph*{faDirectory}\hspace*{\fill}

\begin{list}{}{
\settowidth{\tmplength}{\textbf{Declaração}}
\setlength{\itemindent}{0cm}
\setlength{\listparindent}{0cm}
\setlength{\leftmargin}{\evensidemargin}
\addtolength{\leftmargin}{\tmplength}
\settowidth{\labelsep}{X}
\addtolength{\leftmargin}{\labelsep}
\setlength{\labelwidth}{\tmplength}
}
\begin{flushleft}
\item[\textbf{Declaração}\hfill]
\begin{ttfamily}
public const faDirectory  = SysUtils.faDirectory ;\end{ttfamily}


\end{flushleft}
\par
\item[\textbf{Descrição}]
\begin{itemize}
\item O atributo \textbf{\begin{ttfamily}faDirectory\end{ttfamily}} indica que o arquivo é um diretório.

\begin{itemize}
\item \textbf{REFERÊNCIA} \begin{itemize}
\item https://www.freepascal.org/docs-html/rtl/sysutils/fadirectory.html
\item https://www.freepascal.org/docs-html/rtl/sysutils/findfirst.html
\end{itemize}
\item \textbf{NOTA} \begin{itemize}
\item Usado em TSearchRec e FindFirst
\end{itemize}
\item \textbf{EXEMPLO}

\texttt{\\\nopagebreak[3]
\\\nopagebreak[3]
}\textbf{procedure}\texttt{~TFormTests.Button{\_}GetInfoFileClick(Sender:~TObject);\\\nopagebreak[3]
\\\nopagebreak[3]
~~}\textbf{function}\texttt{~GetInfoFile(FileName:}\textbf{string}\texttt{~;~}\textbf{out}\texttt{~info~:~TSearchRec):~Integer;\\\nopagebreak[3]
\\\nopagebreak[3]
~~}\textbf{begin}\texttt{\\\nopagebreak[3]
~~~~~Result~:=~FindFirst(FileName,faDirectory,Info);\\\nopagebreak[3]
~~~~~}\textbf{if}\texttt{~Result~=~0\\\nopagebreak[3]
~~~~~}\textbf{then}\texttt{~}\textbf{Begin}\texttt{\\\nopagebreak[3]
~~~~~~~~~~~~ShowMessage('O~Diretório~'+fileName+'~encontrado.');\\\nopagebreak[3]
~~~~~~~~~~}\textbf{end}\texttt{\\\nopagebreak[3]
~~~~~}\textbf{else}\texttt{~}\textbf{begin}\texttt{\\\nopagebreak[3]
~~~~~~~~~~~~ShowMessage('O~diretório~'+fileName+'~não~encontrado.');\\\nopagebreak[3]
~~~~~~~~~~}\textbf{end}\texttt{;\\\nopagebreak[3]
~~}\textbf{end}\texttt{;\\\nopagebreak[3]
\\\nopagebreak[3]
~~}\textbf{var}\texttt{\\\nopagebreak[3]
~~~Info:~TSearchRec;\\\nopagebreak[3]
~~~err~:~integer;\\\nopagebreak[3]
}\textbf{begin}\texttt{\\\nopagebreak[3]
~~err~:=~GetInfoFile(ExpandFileName('..'),info);\\\nopagebreak[3]
~~}\textbf{if}\texttt{~err~=~0~}\textbf{then}\texttt{\\\nopagebreak[3]
~~}\textbf{Begin}\texttt{\\\nopagebreak[3]
~~~~showMessage(intToStr(info.Attr));\\\nopagebreak[3]
~~~~FindClose(Info);\\\nopagebreak[3]
~~}\textbf{end}\texttt{;\\\nopagebreak[3]
}\textbf{end}\texttt{;\\
}
\end{itemize}
\end{itemize}

\end{list}
\paragraph*{faNormal}\hspace*{\fill}

\begin{list}{}{
\settowidth{\tmplength}{\textbf{Declaração}}
\setlength{\itemindent}{0cm}
\setlength{\listparindent}{0cm}
\setlength{\leftmargin}{\evensidemargin}
\addtolength{\leftmargin}{\tmplength}
\settowidth{\labelsep}{X}
\addtolength{\leftmargin}{\labelsep}
\setlength{\labelwidth}{\tmplength}
}
\begin{flushleft}
\item[\textbf{Declaração}\hfill]
\begin{ttfamily}
public const faNormal     = SysUtils.faNormal    ;\end{ttfamily}


\end{flushleft}
\par
\item[\textbf{Descrição}]
\begin{itemize}
\item O atributo \textbf{\begin{ttfamily}faNormal\end{ttfamily}} indica que é uma arquivo normal.

\begin{itemize}
\item \textbf{REFERÊNCIA} \begin{itemize}
\item https://www.freepascal.org/docs-html/rtl/sysutils/fanormal.html
\end{itemize}
\item \textbf{NOTA} \begin{itemize}
\item Usado em FindFirst para indicar que arquivos normais devem ser incluídos no resultado.
\end{itemize}
\end{itemize}
\end{itemize}

\end{list}
\paragraph*{faAnyFile}\hspace*{\fill}

\begin{list}{}{
\settowidth{\tmplength}{\textbf{Declaração}}
\setlength{\itemindent}{0cm}
\setlength{\listparindent}{0cm}
\setlength{\leftmargin}{\evensidemargin}
\addtolength{\leftmargin}{\tmplength}
\settowidth{\labelsep}{X}
\addtolength{\leftmargin}{\labelsep}
\setlength{\labelwidth}{\tmplength}
}
\begin{flushleft}
\item[\textbf{Declaração}\hfill]
\begin{ttfamily}
public const faAnyFile    = SysUtils.faAnyFile   ;\end{ttfamily}


\end{flushleft}
\par
\item[\textbf{Descrição}]
\begin{itemize}
\item O atributo \textbf{\begin{ttfamily}faAnyFile\end{ttfamily}} indica que corresponder a qualquer arquivo

\begin{itemize}
\item \textbf{REFERÊNCIA} \begin{itemize}
\item https://www.freepascal.org/docs-html/rtl/sysutils/faanyfile.html
\end{itemize}
\item \textbf{NOTA} \begin{itemize}
\item Use este atributo na chamada FindFirst para localizar todos os arquivos correspondentes.
\end{itemize}
\end{itemize}
\end{itemize}

\end{list}
\paragraph*{faArchive}\hspace*{\fill}

\begin{list}{}{
\settowidth{\tmplength}{\textbf{Declaração}}
\setlength{\itemindent}{0cm}
\setlength{\listparindent}{0cm}
\setlength{\leftmargin}{\evensidemargin}
\addtolength{\leftmargin}{\tmplength}
\settowidth{\labelsep}{X}
\addtolength{\leftmargin}{\labelsep}
\setlength{\labelwidth}{\tmplength}
}
\begin{flushleft}
\item[\textbf{Declaração}\hfill]
\begin{ttfamily}
public const faArchive    = SysUtils.faArchive   ;\end{ttfamily}


\end{flushleft}
\par
\item[\textbf{Descrição}]
\begin{itemize}
\item O atributo \textbf{\begin{ttfamily}faArchive\end{ttfamily}} indica que o bit do arquivo está definido. \begin{itemize}
\item \textbf{REFERÊNCIA} \begin{itemize}
\item https://www.freepascal.org/docs-html/rtl/sysutils/faarchive.html
\end{itemize}
\item \textbf{NOTA} \begin{itemize}
\item Significa que o arquivo tem o conjunto de bits de arquivo. Usado em TSearchRec e FindFirst
\end{itemize}
\end{itemize}
\end{itemize}

\end{list}
\paragraph*{fsFromBeginning}\hspace*{\fill}

\begin{list}{}{
\settowidth{\tmplength}{\textbf{Declaração}}
\setlength{\itemindent}{0cm}
\setlength{\listparindent}{0cm}
\setlength{\leftmargin}{\evensidemargin}
\addtolength{\leftmargin}{\tmplength}
\settowidth{\labelsep}{X}
\addtolength{\leftmargin}{\labelsep}
\setlength{\labelwidth}{\tmplength}
}
\begin{flushleft}
\item[\textbf{Declaração}\hfill]
\begin{ttfamily}
public const fsFromBeginning = SysUtils.fsFromBeginning;\end{ttfamily}


\end{flushleft}
\par
\item[\textbf{Descrição}]
\begin{itemize}
\item O mapa de bits \textbf{\begin{ttfamily}fsFromBeginning\end{ttfamily}} indica ao \textbf{\begin{ttfamily}TFiles.FileSeek\end{ttfamily}(\ref{mi.rtl.files.TFiles-FileSeek})} que o deslocamento é relativo ao primeiro byte do arquivo. Esta posição é baseada em zero. ou seja, o primeiro byte está no deslocamento 0 (zero).
\end{itemize}

\end{list}
\paragraph*{fsFromCurrent}\hspace*{\fill}

\begin{list}{}{
\settowidth{\tmplength}{\textbf{Declaração}}
\setlength{\itemindent}{0cm}
\setlength{\listparindent}{0cm}
\setlength{\leftmargin}{\evensidemargin}
\addtolength{\leftmargin}{\tmplength}
\settowidth{\labelsep}{X}
\addtolength{\leftmargin}{\labelsep}
\setlength{\labelwidth}{\tmplength}
}
\begin{flushleft}
\item[\textbf{Declaração}\hfill]
\begin{ttfamily}
public const fsFromCurrent   = SysUtils.fsFromCurrent  ;\end{ttfamily}


\end{flushleft}
\par
\item[\textbf{Descrição}]
\begin{itemize}
\item O mapa de bits \textbf{\begin{ttfamily}fsFromCurrent\end{ttfamily}} indica ao \textbf{\begin{ttfamily}TFiles.FileSeek\end{ttfamily}(\ref{mi.rtl.files.TFiles-FileSeek})} que o deslocamento é relativo à posição atual.
\end{itemize}

\end{list}
\paragraph*{fsFromEnd}\hspace*{\fill}

\begin{list}{}{
\settowidth{\tmplength}{\textbf{Declaração}}
\setlength{\itemindent}{0cm}
\setlength{\listparindent}{0cm}
\setlength{\leftmargin}{\evensidemargin}
\addtolength{\leftmargin}{\tmplength}
\settowidth{\labelsep}{X}
\addtolength{\leftmargin}{\labelsep}
\setlength{\labelwidth}{\tmplength}
}
\begin{flushleft}
\item[\textbf{Declaração}\hfill]
\begin{ttfamily}
public const fsFromEnd       = SysUtils.fsFromEnd      ;\end{ttfamily}


\end{flushleft}
\par
\item[\textbf{Descrição}]
\begin{itemize}
\item O mapa de bits \textbf{\begin{ttfamily}fsFromEnd\end{ttfamily}} indica ao \textbf{\begin{ttfamily}TFiles.FileSeek\end{ttfamily}(\ref{mi.rtl.files.TFiles-FileSeek})} que o deslocamento é relativo ao final do arquivo. Isso significa que o deslocamento só pode ser zero ou negativo neste caso.
\end{itemize}

\end{list}
\paragraph*{ArquivoNaoEncontrado2}\hspace*{\fill}

\begin{list}{}{
\settowidth{\tmplength}{\textbf{Declaração}}
\setlength{\itemindent}{0cm}
\setlength{\listparindent}{0cm}
\setlength{\leftmargin}{\evensidemargin}
\addtolength{\leftmargin}{\tmplength}
\settowidth{\labelsep}{X}
\addtolength{\leftmargin}{\labelsep}
\setlength{\labelwidth}{\tmplength}
}
\begin{flushleft}
\item[\textbf{Declaração}\hfill]
\begin{ttfamily}
public const ArquivoNaoEncontrado2                   = 002;\end{ttfamily}


\end{flushleft}
\par
\item[\textbf{Descrição}]
\textbf{}\textbf{}\textbf{}\textbf{}\textbf{}\textbf{}\textbf{}\textbf{}\textbf{}\textbf{}\textbf{}\textbf{}\textbf{}\textbf{}\textbf{}\textbf{}\textbf{}\textbf{* }\textbf{}\textbf{}\textbf{}\textbf{}\textbf{}\textbf{}\textbf{}\textbf{}\textbf{}\textbf{}\textbf{}\textbf{}\textbf{}\textbf{}\textbf{}\textbf{}\textbf{}*

\end{list}
\paragraph*{PathNaoEncontrado}\hspace*{\fill}

\begin{list}{}{
\settowidth{\tmplength}{\textbf{Declaração}}
\setlength{\itemindent}{0cm}
\setlength{\listparindent}{0cm}
\setlength{\leftmargin}{\evensidemargin}
\addtolength{\leftmargin}{\tmplength}
\settowidth{\labelsep}{X}
\addtolength{\leftmargin}{\labelsep}
\setlength{\labelwidth}{\tmplength}
}
\begin{flushleft}
\item[\textbf{Declaração}\hfill]
\begin{ttfamily}
public const PathNaoEncontrado                       = 003;\end{ttfamily}


\end{flushleft}
\end{list}
\paragraph*{muitosArquivosAbertoSimultaneamente}\hspace*{\fill}

\begin{list}{}{
\settowidth{\tmplength}{\textbf{Declaração}}
\setlength{\itemindent}{0cm}
\setlength{\listparindent}{0cm}
\setlength{\leftmargin}{\evensidemargin}
\addtolength{\leftmargin}{\tmplength}
\settowidth{\labelsep}{X}
\addtolength{\leftmargin}{\labelsep}
\setlength{\labelwidth}{\tmplength}
}
\begin{flushleft}
\item[\textbf{Declaração}\hfill]
\begin{ttfamily}
public const muitosArquivosAbertoSimultaneamente     = 004;\end{ttfamily}


\end{flushleft}
\end{list}
\paragraph*{AcessoNegado5}\hspace*{\fill}

\begin{list}{}{
\settowidth{\tmplength}{\textbf{Declaração}}
\setlength{\itemindent}{0cm}
\setlength{\listparindent}{0cm}
\setlength{\leftmargin}{\evensidemargin}
\addtolength{\leftmargin}{\tmplength}
\settowidth{\labelsep}{X}
\addtolength{\leftmargin}{\labelsep}
\setlength{\labelwidth}{\tmplength}
}
\begin{flushleft}
\item[\textbf{Declaração}\hfill]
\begin{ttfamily}
public const AcessoNegado5                          = 005;\end{ttfamily}


\end{flushleft}
\end{list}
\paragraph*{Seek{\_}fora{\_}da{\_}faixa{\_}do{\_}arquivo}\hspace*{\fill}

\begin{list}{}{
\settowidth{\tmplength}{\textbf{Declaração}}
\setlength{\itemindent}{0cm}
\setlength{\listparindent}{0cm}
\setlength{\leftmargin}{\evensidemargin}
\addtolength{\leftmargin}{\tmplength}
\settowidth{\labelsep}{X}
\addtolength{\leftmargin}{\labelsep}
\setlength{\labelwidth}{\tmplength}
}
\begin{flushleft}
\item[\textbf{Declaração}\hfill]
\begin{ttfamily}
public const Seek{\_}fora{\_}da{\_}faixa{\_}do{\_}arquivo           = 007;\end{ttfamily}


\end{flushleft}
\end{list}
\paragraph*{ErroDeMemoria}\hspace*{\fill}

\begin{list}{}{
\settowidth{\tmplength}{\textbf{Declaração}}
\setlength{\itemindent}{0cm}
\setlength{\listparindent}{0cm}
\setlength{\leftmargin}{\evensidemargin}
\addtolength{\leftmargin}{\tmplength}
\settowidth{\labelsep}{X}
\addtolength{\leftmargin}{\labelsep}
\setlength{\labelwidth}{\tmplength}
}
\begin{flushleft}
\item[\textbf{Declaração}\hfill]
\begin{ttfamily}
public const ErroDeMemoria                           = 008;\end{ttfamily}


\end{flushleft}
\end{list}
\paragraph*{ErroFormatoInvalido}\hspace*{\fill}

\begin{list}{}{
\settowidth{\tmplength}{\textbf{Declaração}}
\setlength{\itemindent}{0cm}
\setlength{\listparindent}{0cm}
\setlength{\leftmargin}{\evensidemargin}
\addtolength{\leftmargin}{\tmplength}
\settowidth{\labelsep}{X}
\addtolength{\leftmargin}{\labelsep}
\setlength{\labelwidth}{\tmplength}
}
\begin{flushleft}
\item[\textbf{Declaração}\hfill]
\begin{ttfamily}
public const ErroFormatoInvalido                     = 011;\end{ttfamily}


\end{flushleft}
\end{list}
\paragraph*{NaoPodeExecutarTrocaDeNomesEntreDiscos}\hspace*{\fill}

\begin{list}{}{
\settowidth{\tmplength}{\textbf{Declaração}}
\setlength{\itemindent}{0cm}
\setlength{\listparindent}{0cm}
\setlength{\leftmargin}{\evensidemargin}
\addtolength{\leftmargin}{\tmplength}
\settowidth{\labelsep}{X}
\addtolength{\leftmargin}{\labelsep}
\setlength{\labelwidth}{\tmplength}
}
\begin{flushleft}
\item[\textbf{Declaração}\hfill]
\begin{ttfamily}
public const NaoPodeExecutarTrocaDeNomesEntreDiscos  = 017;\end{ttfamily}


\end{flushleft}
\end{list}
\paragraph*{ArquivoNaoEncontrado18}\hspace*{\fill}

\begin{list}{}{
\settowidth{\tmplength}{\textbf{Declaração}}
\setlength{\itemindent}{0cm}
\setlength{\listparindent}{0cm}
\setlength{\leftmargin}{\evensidemargin}
\addtolength{\leftmargin}{\tmplength}
\settowidth{\labelsep}{X}
\addtolength{\leftmargin}{\labelsep}
\setlength{\labelwidth}{\tmplength}
}
\begin{flushleft}
\item[\textbf{Declaração}\hfill]
\begin{ttfamily}
public const ArquivoNaoEncontrado18          = 018;\end{ttfamily}


\end{flushleft}
\end{list}
\paragraph*{DiscoProtegidoContraEscrita}\hspace*{\fill}

\begin{list}{}{
\settowidth{\tmplength}{\textbf{Declaração}}
\setlength{\itemindent}{0cm}
\setlength{\listparindent}{0cm}
\setlength{\leftmargin}{\evensidemargin}
\addtolength{\leftmargin}{\tmplength}
\settowidth{\labelsep}{X}
\addtolength{\leftmargin}{\labelsep}
\setlength{\labelwidth}{\tmplength}
}
\begin{flushleft}
\item[\textbf{Declaração}\hfill]
\begin{ttfamily}
public const DiscoProtegidoContraEscrita     = 019;\end{ttfamily}


\end{flushleft}
\end{list}
\paragraph*{UnidadeDesconhecida}\hspace*{\fill}

\begin{list}{}{
\settowidth{\tmplength}{\textbf{Declaração}}
\setlength{\itemindent}{0cm}
\setlength{\listparindent}{0cm}
\setlength{\leftmargin}{\evensidemargin}
\addtolength{\leftmargin}{\tmplength}
\settowidth{\labelsep}{X}
\addtolength{\leftmargin}{\labelsep}
\setlength{\labelwidth}{\tmplength}
}
\begin{flushleft}
\item[\textbf{Declaração}\hfill]
\begin{ttfamily}
public const UnidadeDesconhecida             = 020;\end{ttfamily}


\end{flushleft}
\end{list}
\paragraph*{DriveNaoEstaPronto}\hspace*{\fill}

\begin{list}{}{
\settowidth{\tmplength}{\textbf{Declaração}}
\setlength{\itemindent}{0cm}
\setlength{\listparindent}{0cm}
\setlength{\leftmargin}{\evensidemargin}
\addtolength{\leftmargin}{\tmplength}
\settowidth{\labelsep}{X}
\addtolength{\leftmargin}{\labelsep}
\setlength{\labelwidth}{\tmplength}
}
\begin{flushleft}
\item[\textbf{Declaração}\hfill]
\begin{ttfamily}
public const DriveNaoEstaPronto              = 021;\end{ttfamily}


\end{flushleft}
\end{list}
\paragraph*{ErroDeDadosCRC}\hspace*{\fill}

\begin{list}{}{
\settowidth{\tmplength}{\textbf{Declaração}}
\setlength{\itemindent}{0cm}
\setlength{\listparindent}{0cm}
\setlength{\leftmargin}{\evensidemargin}
\addtolength{\leftmargin}{\tmplength}
\settowidth{\labelsep}{X}
\addtolength{\leftmargin}{\labelsep}
\setlength{\labelwidth}{\tmplength}
}
\begin{flushleft}
\item[\textbf{Declaração}\hfill]
\begin{ttfamily}
public const ErroDeDadosCRC                  = 023;\end{ttfamily}


\end{flushleft}
\end{list}
\paragraph*{Falha{\_}Geral}\hspace*{\fill}

\begin{list}{}{
\settowidth{\tmplength}{\textbf{Declaração}}
\setlength{\itemindent}{0cm}
\setlength{\listparindent}{0cm}
\setlength{\leftmargin}{\evensidemargin}
\addtolength{\leftmargin}{\tmplength}
\settowidth{\labelsep}{X}
\addtolength{\leftmargin}{\labelsep}
\setlength{\labelwidth}{\tmplength}
}
\begin{flushleft}
\item[\textbf{Declaração}\hfill]
\begin{ttfamily}
public const Falha{\_}Geral                     = 031;\end{ttfamily}


\end{flushleft}
\end{list}
\paragraph*{AcessoNegado32}\hspace*{\fill}

\begin{list}{}{
\settowidth{\tmplength}{\textbf{Declaração}}
\setlength{\itemindent}{0cm}
\setlength{\listparindent}{0cm}
\setlength{\leftmargin}{\evensidemargin}
\addtolength{\leftmargin}{\tmplength}
\settowidth{\labelsep}{X}
\addtolength{\leftmargin}{\labelsep}
\setlength{\labelwidth}{\tmplength}
}
\begin{flushleft}
\item[\textbf{Declaração}\hfill]
\begin{ttfamily}
public const AcessoNegado32                  = 032;\end{ttfamily}


\end{flushleft}
\end{list}
\paragraph*{ErroViolacaoDeLacre}\hspace*{\fill}

\begin{list}{}{
\settowidth{\tmplength}{\textbf{Declaração}}
\setlength{\itemindent}{0cm}
\setlength{\listparindent}{0cm}
\setlength{\leftmargin}{\evensidemargin}
\addtolength{\leftmargin}{\tmplength}
\settowidth{\labelsep}{X}
\addtolength{\leftmargin}{\labelsep}
\setlength{\labelwidth}{\tmplength}
}
\begin{flushleft}
\item[\textbf{Declaração}\hfill]
\begin{ttfamily}
public const ErroViolacaoDeLacre             = 033;\end{ttfamily}


\end{flushleft}
\end{list}
\paragraph*{MudancaDeDiscoInvalida}\hspace*{\fill}

\begin{list}{}{
\settowidth{\tmplength}{\textbf{Declaração}}
\setlength{\itemindent}{0cm}
\setlength{\listparindent}{0cm}
\setlength{\leftmargin}{\evensidemargin}
\addtolength{\leftmargin}{\tmplength}
\settowidth{\labelsep}{X}
\addtolength{\leftmargin}{\labelsep}
\setlength{\labelwidth}{\tmplength}
}
\begin{flushleft}
\item[\textbf{Declaração}\hfill]
\begin{ttfamily}
public const MudancaDeDiscoInvalida          = 034;\end{ttfamily}


\end{flushleft}
\end{list}
\paragraph*{Campo{\_}nao{\_}existe{\_}no{\_}registro{\_}do{\_}arquivo}\hspace*{\fill}

\begin{list}{}{
\settowidth{\tmplength}{\textbf{Declaração}}
\setlength{\itemindent}{0cm}
\setlength{\listparindent}{0cm}
\setlength{\leftmargin}{\evensidemargin}
\addtolength{\leftmargin}{\tmplength}
\settowidth{\labelsep}{X}
\addtolength{\leftmargin}{\labelsep}
\setlength{\labelwidth}{\tmplength}
}
\begin{flushleft}
\item[\textbf{Declaração}\hfill]
\begin{ttfamily}
public const Campo{\_}nao{\_}existe{\_}no{\_}registro{\_}do{\_}arquivo = 037;\end{ttfamily}


\end{flushleft}
\end{list}
\paragraph*{Tipo{\_}em{\_}memoria{\_}incompativel{\_}com{\_}o{\_}tipo{\_}do{\_}campo{\_}no{\_}arquivo}\hspace*{\fill}

\begin{list}{}{
\settowidth{\tmplength}{\textbf{Declaração}}
\setlength{\itemindent}{0cm}
\setlength{\listparindent}{0cm}
\setlength{\leftmargin}{\evensidemargin}
\addtolength{\leftmargin}{\tmplength}
\settowidth{\labelsep}{X}
\addtolength{\leftmargin}{\labelsep}
\setlength{\labelwidth}{\tmplength}
}
\begin{flushleft}
\item[\textbf{Declaração}\hfill]
\begin{ttfamily}
public const Tipo{\_}em{\_}memoria{\_}incompativel{\_}com{\_}o{\_}tipo{\_}do{\_}campo{\_}no{\_}arquivo = 038;\end{ttfamily}


\end{flushleft}
\end{list}
\paragraph*{Erro{\_}de{\_}sintaxe{\_}na{\_}expressao}\hspace*{\fill}

\begin{list}{}{
\settowidth{\tmplength}{\textbf{Declaração}}
\setlength{\itemindent}{0cm}
\setlength{\listparindent}{0cm}
\setlength{\leftmargin}{\evensidemargin}
\addtolength{\leftmargin}{\tmplength}
\settowidth{\labelsep}{X}
\addtolength{\leftmargin}{\labelsep}
\setlength{\labelwidth}{\tmplength}
}
\begin{flushleft}
\item[\textbf{Declaração}\hfill]
\begin{ttfamily}
public const Erro{\_}de{\_}sintaxe{\_}na{\_}expressao    = 039;\end{ttfamily}


\end{flushleft}
\end{list}
\paragraph*{Tipos{\_}de{\_}campos{\_}incompativeis}\hspace*{\fill}

\begin{list}{}{
\settowidth{\tmplength}{\textbf{Declaração}}
\setlength{\itemindent}{0cm}
\setlength{\listparindent}{0cm}
\setlength{\leftmargin}{\evensidemargin}
\addtolength{\leftmargin}{\tmplength}
\settowidth{\labelsep}{X}
\addtolength{\leftmargin}{\labelsep}
\setlength{\labelwidth}{\tmplength}
}
\begin{flushleft}
\item[\textbf{Declaração}\hfill]
\begin{ttfamily}
public const Tipos{\_}de{\_}campos{\_}incompativeis   = 040;\end{ttfamily}


\end{flushleft}
\end{list}
\paragraph*{Tipos{\_}de{\_}campo{\_}nao{\_}conhecido}\hspace*{\fill}

\begin{list}{}{
\settowidth{\tmplength}{\textbf{Declaração}}
\setlength{\itemindent}{0cm}
\setlength{\listparindent}{0cm}
\setlength{\leftmargin}{\evensidemargin}
\addtolength{\leftmargin}{\tmplength}
\settowidth{\labelsep}{X}
\addtolength{\leftmargin}{\labelsep}
\setlength{\labelwidth}{\tmplength}
}
\begin{flushleft}
\item[\textbf{Declaração}\hfill]
\begin{ttfamily}
public const Tipos{\_}de{\_}campo{\_}nao{\_}conhecido    = 041;\end{ttfamily}


\end{flushleft}
\end{list}
\paragraph*{Campo{\_}em{\_}duplicidade{\_}na{\_}estrutura{\_}da{\_}tabela}\hspace*{\fill}

\begin{list}{}{
\settowidth{\tmplength}{\textbf{Declaração}}
\setlength{\itemindent}{0cm}
\setlength{\listparindent}{0cm}
\setlength{\leftmargin}{\evensidemargin}
\addtolength{\leftmargin}{\tmplength}
\settowidth{\labelsep}{X}
\addtolength{\leftmargin}{\labelsep}
\setlength{\labelwidth}{\tmplength}
}
\begin{flushleft}
\item[\textbf{Declaração}\hfill]
\begin{ttfamily}
public const Campo{\_}em{\_}duplicidade{\_}na{\_}estrutura{\_}da{\_}tabela = 42;\end{ttfamily}


\end{flushleft}
\end{list}
\paragraph*{Arquivo{\_}ja{\_}existe}\hspace*{\fill}

\begin{list}{}{
\settowidth{\tmplength}{\textbf{Declaração}}
\setlength{\itemindent}{0cm}
\setlength{\listparindent}{0cm}
\setlength{\leftmargin}{\evensidemargin}
\addtolength{\leftmargin}{\tmplength}
\settowidth{\labelsep}{X}
\addtolength{\leftmargin}{\labelsep}
\setlength{\labelwidth}{\tmplength}
}
\begin{flushleft}
\item[\textbf{Declaração}\hfill]
\begin{ttfamily}
public const Arquivo{\_}ja{\_}existe               = 080;\end{ttfamily}


\end{flushleft}
\end{list}
\paragraph*{NaoPodeCriarDiretorio}\hspace*{\fill}

\begin{list}{}{
\settowidth{\tmplength}{\textbf{Declaração}}
\setlength{\itemindent}{0cm}
\setlength{\listparindent}{0cm}
\setlength{\leftmargin}{\evensidemargin}
\addtolength{\leftmargin}{\tmplength}
\settowidth{\labelsep}{X}
\addtolength{\leftmargin}{\labelsep}
\setlength{\labelwidth}{\tmplength}
}
\begin{flushleft}
\item[\textbf{Declaração}\hfill]
\begin{ttfamily}
public const NaoPodeCriarDiretorio           = 082;\end{ttfamily}


\end{flushleft}
\end{list}
\paragraph*{ParametroInvalido}\hspace*{\fill}

\begin{list}{}{
\settowidth{\tmplength}{\textbf{Declaração}}
\setlength{\itemindent}{0cm}
\setlength{\listparindent}{0cm}
\setlength{\leftmargin}{\evensidemargin}
\addtolength{\leftmargin}{\tmplength}
\settowidth{\labelsep}{X}
\addtolength{\leftmargin}{\labelsep}
\setlength{\labelwidth}{\tmplength}
}
\begin{flushleft}
\item[\textbf{Declaração}\hfill]
\begin{ttfamily}
public const ParametroInvalido               = 087;\end{ttfamily}


\end{flushleft}
\end{list}
\paragraph*{ErroDeLeituraEmDisco}\hspace*{\fill}

\begin{list}{}{
\settowidth{\tmplength}{\textbf{Declaração}}
\setlength{\itemindent}{0cm}
\setlength{\listparindent}{0cm}
\setlength{\leftmargin}{\evensidemargin}
\addtolength{\leftmargin}{\tmplength}
\settowidth{\labelsep}{X}
\addtolength{\leftmargin}{\labelsep}
\setlength{\labelwidth}{\tmplength}
}
\begin{flushleft}
\item[\textbf{Declaração}\hfill]
\begin{ttfamily}
public const ErroDeLeituraEmDisco            = 100;\end{ttfamily}


\end{flushleft}
\end{list}
\paragraph*{ErroDeGravacaoEmDisco}\hspace*{\fill}

\begin{list}{}{
\settowidth{\tmplength}{\textbf{Declaração}}
\setlength{\itemindent}{0cm}
\setlength{\listparindent}{0cm}
\setlength{\leftmargin}{\evensidemargin}
\addtolength{\leftmargin}{\tmplength}
\settowidth{\labelsep}{X}
\addtolength{\leftmargin}{\labelsep}
\setlength{\labelwidth}{\tmplength}
}
\begin{flushleft}
\item[\textbf{Declaração}\hfill]
\begin{ttfamily}
public const ErroDeGravacaoEmDisco           = 101;\end{ttfamily}


\end{flushleft}
\end{list}
\paragraph*{ErroArquivoFechado}\hspace*{\fill}

\begin{list}{}{
\settowidth{\tmplength}{\textbf{Declaração}}
\setlength{\itemindent}{0cm}
\setlength{\listparindent}{0cm}
\setlength{\leftmargin}{\evensidemargin}
\addtolength{\leftmargin}{\tmplength}
\settowidth{\labelsep}{X}
\addtolength{\leftmargin}{\labelsep}
\setlength{\labelwidth}{\tmplength}
}
\begin{flushleft}
\item[\textbf{Declaração}\hfill]
\begin{ttfamily}
public const ErroArquivoFechado              = 103;\end{ttfamily}


\end{flushleft}
\end{list}
\paragraph*{ErroArquivoFechadoParaEntrada}\hspace*{\fill}

\begin{list}{}{
\settowidth{\tmplength}{\textbf{Declaração}}
\setlength{\itemindent}{0cm}
\setlength{\listparindent}{0cm}
\setlength{\leftmargin}{\evensidemargin}
\addtolength{\leftmargin}{\tmplength}
\settowidth{\labelsep}{X}
\addtolength{\leftmargin}{\labelsep}
\setlength{\labelwidth}{\tmplength}
}
\begin{flushleft}
\item[\textbf{Declaração}\hfill]
\begin{ttfamily}
public const ErroArquivoFechadoParaEntrada   = 104;\end{ttfamily}


\end{flushleft}
\end{list}
\paragraph*{ErrorArquivoFechadoParaSaida}\hspace*{\fill}

\begin{list}{}{
\settowidth{\tmplength}{\textbf{Declaração}}
\setlength{\itemindent}{0cm}
\setlength{\listparindent}{0cm}
\setlength{\leftmargin}{\evensidemargin}
\addtolength{\leftmargin}{\tmplength}
\settowidth{\labelsep}{X}
\addtolength{\leftmargin}{\labelsep}
\setlength{\labelwidth}{\tmplength}
}
\begin{flushleft}
\item[\textbf{Declaração}\hfill]
\begin{ttfamily}
public const ErrorArquivoFechadoParaSaida    = 105;\end{ttfamily}


\end{flushleft}
\end{list}
\paragraph*{Formato{\_}numerico{\_}invalido{\_}ou{\_}incompativel}\hspace*{\fill}

\begin{list}{}{
\settowidth{\tmplength}{\textbf{Declaração}}
\setlength{\itemindent}{0cm}
\setlength{\listparindent}{0cm}
\setlength{\leftmargin}{\evensidemargin}
\addtolength{\leftmargin}{\tmplength}
\settowidth{\labelsep}{X}
\addtolength{\leftmargin}{\labelsep}
\setlength{\labelwidth}{\tmplength}
}
\begin{flushleft}
\item[\textbf{Declaração}\hfill]
\begin{ttfamily}
public const Formato{\_}numerico{\_}invalido{\_}ou{\_}incompativel   = 106;\end{ttfamily}


\end{flushleft}
\end{list}
\paragraph*{DiscoCheio}\hspace*{\fill}

\begin{list}{}{
\settowidth{\tmplength}{\textbf{Declaração}}
\setlength{\itemindent}{0cm}
\setlength{\listparindent}{0cm}
\setlength{\leftmargin}{\evensidemargin}
\addtolength{\leftmargin}{\tmplength}
\settowidth{\labelsep}{X}
\addtolength{\leftmargin}{\labelsep}
\setlength{\labelwidth}{\tmplength}
}
\begin{flushleft}
\item[\textbf{Declaração}\hfill]
\begin{ttfamily}
public const DiscoCheio                                  = 107;\end{ttfamily}


\end{flushleft}
\end{list}
\paragraph*{ErroDeEscritaNoDispositivoDeSaidaImpressora}\hspace*{\fill}

\begin{list}{}{
\settowidth{\tmplength}{\textbf{Declaração}}
\setlength{\itemindent}{0cm}
\setlength{\listparindent}{0cm}
\setlength{\leftmargin}{\evensidemargin}
\addtolength{\leftmargin}{\tmplength}
\settowidth{\labelsep}{X}
\addtolength{\leftmargin}{\labelsep}
\setlength{\labelwidth}{\tmplength}
}
\begin{flushleft}
\item[\textbf{Declaração}\hfill]
\begin{ttfamily}
public const ErroDeEscritaNoDispositivoDeSaidaImpressora = 160;\end{ttfamily}


\end{flushleft}
\end{list}
\paragraph*{ErroFaltaHardware}\hspace*{\fill}

\begin{list}{}{
\settowidth{\tmplength}{\textbf{Declaração}}
\setlength{\itemindent}{0cm}
\setlength{\listparindent}{0cm}
\setlength{\leftmargin}{\evensidemargin}
\addtolength{\leftmargin}{\tmplength}
\settowidth{\labelsep}{X}
\addtolength{\leftmargin}{\labelsep}
\setlength{\labelwidth}{\tmplength}
}
\begin{flushleft}
\item[\textbf{Declaração}\hfill]
\begin{ttfamily}
public const ErroFaltaHardware                               = 162;\end{ttfamily}


\end{flushleft}
\end{list}
\paragraph*{Err{\_}Division{\_}by{\_}zero}\hspace*{\fill}

\begin{list}{}{
\settowidth{\tmplength}{\textbf{Declaração}}
\setlength{\itemindent}{0cm}
\setlength{\listparindent}{0cm}
\setlength{\leftmargin}{\evensidemargin}
\addtolength{\leftmargin}{\tmplength}
\settowidth{\labelsep}{X}
\addtolength{\leftmargin}{\labelsep}
\setlength{\labelwidth}{\tmplength}
}
\begin{flushleft}
\item[\textbf{Declaração}\hfill]
\begin{ttfamily}
public const Err{\_}Division{\_}by{\_}zero                            = 200;\end{ttfamily}


\end{flushleft}
\end{list}
\paragraph*{ErrorNaChecagemDeFaixa}\hspace*{\fill}

\begin{list}{}{
\settowidth{\tmplength}{\textbf{Declaração}}
\setlength{\itemindent}{0cm}
\setlength{\listparindent}{0cm}
\setlength{\leftmargin}{\evensidemargin}
\addtolength{\leftmargin}{\tmplength}
\settowidth{\labelsep}{X}
\addtolength{\leftmargin}{\labelsep}
\setlength{\labelwidth}{\tmplength}
}
\begin{flushleft}
\item[\textbf{Declaração}\hfill]
\begin{ttfamily}
public const ErrorNaChecagemDeFaixa                          = 201;\end{ttfamily}


\end{flushleft}
\end{list}
\paragraph*{Objeto{\_}Nao{\_}Inicializado}\hspace*{\fill}

\begin{list}{}{
\settowidth{\tmplength}{\textbf{Declaração}}
\setlength{\itemindent}{0cm}
\setlength{\listparindent}{0cm}
\setlength{\leftmargin}{\evensidemargin}
\addtolength{\leftmargin}{\tmplength}
\settowidth{\labelsep}{X}
\addtolength{\leftmargin}{\labelsep}
\setlength{\labelwidth}{\tmplength}
}
\begin{flushleft}
\item[\textbf{Declaração}\hfill]
\begin{ttfamily}
public const Objeto{\_}Nao{\_}Inicializado                         = 210;\end{ttfamily}


\end{flushleft}
\end{list}
\paragraph*{Chamada{\_}a{\_}um{\_}Metodo{\_}Abstrato}\hspace*{\fill}

\begin{list}{}{
\settowidth{\tmplength}{\textbf{Declaração}}
\setlength{\itemindent}{0cm}
\setlength{\listparindent}{0cm}
\setlength{\leftmargin}{\evensidemargin}
\addtolength{\leftmargin}{\tmplength}
\settowidth{\labelsep}{X}
\addtolength{\leftmargin}{\labelsep}
\setlength{\labelwidth}{\tmplength}
}
\begin{flushleft}
\item[\textbf{Declaração}\hfill]
\begin{ttfamily}
public const Chamada{\_}a{\_}um{\_}Metodo{\_}Abstrato                    = 211;\end{ttfamily}


\end{flushleft}
\end{list}
\paragraph*{Stream{\_}Registration{\_}error}\hspace*{\fill}

\begin{list}{}{
\settowidth{\tmplength}{\textbf{Declaração}}
\setlength{\itemindent}{0cm}
\setlength{\listparindent}{0cm}
\setlength{\leftmargin}{\evensidemargin}
\addtolength{\leftmargin}{\tmplength}
\settowidth{\labelsep}{X}
\addtolength{\leftmargin}{\labelsep}
\setlength{\labelwidth}{\tmplength}
}
\begin{flushleft}
\item[\textbf{Declaração}\hfill]
\begin{ttfamily}
public const Stream{\_}Registration{\_}error                       = 212;\end{ttfamily}


\end{flushleft}
\end{list}
\paragraph*{Collection{\_}Index{\_}Out{\_}of{\_}range}\hspace*{\fill}

\begin{list}{}{
\settowidth{\tmplength}{\textbf{Declaração}}
\setlength{\itemindent}{0cm}
\setlength{\listparindent}{0cm}
\setlength{\leftmargin}{\evensidemargin}
\addtolength{\leftmargin}{\tmplength}
\settowidth{\labelsep}{X}
\addtolength{\leftmargin}{\labelsep}
\setlength{\labelwidth}{\tmplength}
}
\begin{flushleft}
\item[\textbf{Declaração}\hfill]
\begin{ttfamily}
public const Collection{\_}Index{\_}Out{\_}of{\_}range                   = 213;\end{ttfamily}


\end{flushleft}
\end{list}
\paragraph*{ErrorTentativa{\_}de{\_}abrir{\_}um{\_}arquivo{\_}aberto}\hspace*{\fill}

\begin{list}{}{
\settowidth{\tmplength}{\textbf{Declaração}}
\setlength{\itemindent}{0cm}
\setlength{\listparindent}{0cm}
\setlength{\leftmargin}{\evensidemargin}
\addtolength{\leftmargin}{\tmplength}
\settowidth{\labelsep}{X}
\addtolength{\leftmargin}{\labelsep}
\setlength{\labelwidth}{\tmplength}
}
\begin{flushleft}
\item[\textbf{Declaração}\hfill]
\begin{ttfamily}
public const ErrorTentativa{\_}de{\_}abrir{\_}um{\_}arquivo{\_}aberto       = 217;\end{ttfamily}


\end{flushleft}
\end{list}
\paragraph*{Erro{\_}Tentativa{\_}de{\_}excluir{\_}um{\_}registro{\_}excluido}\hspace*{\fill}

\begin{list}{}{
\settowidth{\tmplength}{\textbf{Declaração}}
\setlength{\itemindent}{0cm}
\setlength{\listparindent}{0cm}
\setlength{\leftmargin}{\evensidemargin}
\addtolength{\leftmargin}{\tmplength}
\settowidth{\labelsep}{X}
\addtolength{\leftmargin}{\labelsep}
\setlength{\labelwidth}{\tmplength}
}
\begin{flushleft}
\item[\textbf{Declaração}\hfill]
\begin{ttfamily}
public const Erro{\_}Tentativa{\_}de{\_}excluir{\_}um{\_}registro{\_}excluido  = 218;\end{ttfamily}


\end{flushleft}
\end{list}
\paragraph*{Erro{\_}Tentativa{\_}de{\_}ler{\_}um{\_}registro{\_}excluido}\hspace*{\fill}

\begin{list}{}{
\settowidth{\tmplength}{\textbf{Declaração}}
\setlength{\itemindent}{0cm}
\setlength{\listparindent}{0cm}
\setlength{\leftmargin}{\evensidemargin}
\addtolength{\leftmargin}{\tmplength}
\settowidth{\labelsep}{X}
\addtolength{\leftmargin}{\labelsep}
\setlength{\labelwidth}{\tmplength}
}
\begin{flushleft}
\item[\textbf{Declaração}\hfill]
\begin{ttfamily}
public const Erro{\_}Tentativa{\_}de{\_}ler{\_}um{\_}registro{\_}excluido      = 219;\end{ttfamily}


\end{flushleft}
\end{list}
\paragraph*{Erro{\_}outro{\_}usuario{\_}da{\_}rede{\_}alterou{\_}o{\_}registro}\hspace*{\fill}

\begin{list}{}{
\settowidth{\tmplength}{\textbf{Declaração}}
\setlength{\itemindent}{0cm}
\setlength{\listparindent}{0cm}
\setlength{\leftmargin}{\evensidemargin}
\addtolength{\leftmargin}{\tmplength}
\settowidth{\labelsep}{X}
\addtolength{\leftmargin}{\labelsep}
\setlength{\labelwidth}{\tmplength}
}
\begin{flushleft}
\item[\textbf{Declaração}\hfill]
\begin{ttfamily}
public const Erro{\_}outro{\_}usuario{\_}da{\_}rede{\_}alterou{\_}o{\_}registro   = 220;\end{ttfamily}


\end{flushleft}
\end{list}
\paragraph*{Estrutura{\_}da{\_}tabela{\_}esta{\_}danificada}\hspace*{\fill}

\begin{list}{}{
\settowidth{\tmplength}{\textbf{Declaração}}
\setlength{\itemindent}{0cm}
\setlength{\listparindent}{0cm}
\setlength{\leftmargin}{\evensidemargin}
\addtolength{\leftmargin}{\tmplength}
\settowidth{\labelsep}{X}
\addtolength{\leftmargin}{\labelsep}
\setlength{\labelwidth}{\tmplength}
}
\begin{flushleft}
\item[\textbf{Declaração}\hfill]
\begin{ttfamily}
public const Estrutura{\_}da{\_}tabela{\_}esta{\_}danificada             = 221;\end{ttfamily}


\end{flushleft}
\end{list}
\paragraph*{Tentativa{\_}de{\_}gravar{\_}em{\_}um{\_}registro{\_}compartilhado{\_}sem{\_}que{\_}o{\_}mesmo{\_}esteja{\_}travado}\hspace*{\fill}

\begin{list}{}{
\settowidth{\tmplength}{\textbf{Declaração}}
\setlength{\itemindent}{0cm}
\setlength{\listparindent}{0cm}
\setlength{\leftmargin}{\evensidemargin}
\addtolength{\leftmargin}{\tmplength}
\settowidth{\labelsep}{X}
\addtolength{\leftmargin}{\labelsep}
\setlength{\labelwidth}{\tmplength}
}
\begin{flushleft}
\item[\textbf{Declaração}\hfill]
\begin{ttfamily}
public const Tentativa{\_}de{\_}gravar{\_}em{\_}um{\_}registro{\_}compartilhado{\_}sem{\_}que{\_}o{\_}mesmo{\_}esteja{\_}travado = 222;\end{ttfamily}


\end{flushleft}
\end{list}
\paragraph*{AppCli{\_}Evento{\_}Executado{\_}Por{\_}Outro{\_}Processo}\hspace*{\fill}

\begin{list}{}{
\settowidth{\tmplength}{\textbf{Declaração}}
\setlength{\itemindent}{0cm}
\setlength{\listparindent}{0cm}
\setlength{\leftmargin}{\evensidemargin}
\addtolength{\leftmargin}{\tmplength}
\settowidth{\labelsep}{X}
\addtolength{\leftmargin}{\labelsep}
\setlength{\labelwidth}{\tmplength}
}
\begin{flushleft}
\item[\textbf{Declaração}\hfill]
\begin{ttfamily}
public const AppCli{\_}Evento{\_}Executado{\_}Por{\_}Outro{\_}Processo = 223;\end{ttfamily}


\end{flushleft}
\end{list}
\paragraph*{AppCLi{\_}Svr{\_}Api{\_}Nao{\_}Instalado}\hspace*{\fill}

\begin{list}{}{
\settowidth{\tmplength}{\textbf{Declaração}}
\setlength{\itemindent}{0cm}
\setlength{\listparindent}{0cm}
\setlength{\leftmargin}{\evensidemargin}
\addtolength{\leftmargin}{\tmplength}
\settowidth{\labelsep}{X}
\addtolength{\leftmargin}{\labelsep}
\setlength{\labelwidth}{\tmplength}
}
\begin{flushleft}
\item[\textbf{Declaração}\hfill]
\begin{ttfamily}
public const AppCLi{\_}Svr{\_}Api{\_}Nao{\_}Instalado               = 224;\end{ttfamily}


\end{flushleft}
\end{list}
\paragraph*{TTransaction{\_}Commit{\_}esperado}\hspace*{\fill}

\begin{list}{}{
\settowidth{\tmplength}{\textbf{Declaração}}
\setlength{\itemindent}{0cm}
\setlength{\listparindent}{0cm}
\setlength{\leftmargin}{\evensidemargin}
\addtolength{\leftmargin}{\tmplength}
\settowidth{\labelsep}{X}
\addtolength{\leftmargin}{\labelsep}
\setlength{\labelwidth}{\tmplength}
}
\begin{flushleft}
\item[\textbf{Declaração}\hfill]
\begin{ttfamily}
public const TTransaction{\_}Commit{\_}esperado               = 225;\end{ttfamily}


\end{flushleft}
\end{list}
\paragraph*{Erro{\_}Expression{\_}is{\_}not{\_}valid}\hspace*{\fill}

\begin{list}{}{
\settowidth{\tmplength}{\textbf{Declaração}}
\setlength{\itemindent}{0cm}
\setlength{\listparindent}{0cm}
\setlength{\leftmargin}{\evensidemargin}
\addtolength{\leftmargin}{\tmplength}
\settowidth{\labelsep}{X}
\addtolength{\leftmargin}{\labelsep}
\setlength{\labelwidth}{\tmplength}
}
\begin{flushleft}
\item[\textbf{Declaração}\hfill]
\begin{ttfamily}
public const Erro{\_}Expression{\_}is{\_}not{\_}valid               = 226;\end{ttfamily}


\end{flushleft}
\end{list}
\paragraph*{Erro{\_}Many{\_}Parenthesis}\hspace*{\fill}

\begin{list}{}{
\settowidth{\tmplength}{\textbf{Declaração}}
\setlength{\itemindent}{0cm}
\setlength{\listparindent}{0cm}
\setlength{\leftmargin}{\evensidemargin}
\addtolength{\leftmargin}{\tmplength}
\settowidth{\labelsep}{X}
\addtolength{\leftmargin}{\labelsep}
\setlength{\labelwidth}{\tmplength}
}
\begin{flushleft}
\item[\textbf{Declaração}\hfill]
\begin{ttfamily}
public const Erro{\_}Many{\_}Parenthesis                      = 227;\end{ttfamily}


\end{flushleft}
\end{list}
\paragraph*{Erro{\_}Many{\_}operators}\hspace*{\fill}

\begin{list}{}{
\settowidth{\tmplength}{\textbf{Declaração}}
\setlength{\itemindent}{0cm}
\setlength{\listparindent}{0cm}
\setlength{\leftmargin}{\evensidemargin}
\addtolength{\leftmargin}{\tmplength}
\settowidth{\labelsep}{X}
\addtolength{\leftmargin}{\labelsep}
\setlength{\labelwidth}{\tmplength}
}
\begin{flushleft}
\item[\textbf{Declaração}\hfill]
\begin{ttfamily}
public const Erro{\_}Many{\_}operators                        = 228;\end{ttfamily}


\end{flushleft}
\end{list}
\paragraph*{Erro{\_}Operador{\_}aritmetico{\_}esperado}\hspace*{\fill}

\begin{list}{}{
\settowidth{\tmplength}{\textbf{Declaração}}
\setlength{\itemindent}{0cm}
\setlength{\listparindent}{0cm}
\setlength{\leftmargin}{\evensidemargin}
\addtolength{\leftmargin}{\tmplength}
\settowidth{\labelsep}{X}
\addtolength{\leftmargin}{\labelsep}
\setlength{\labelwidth}{\tmplength}
}
\begin{flushleft}
\item[\textbf{Declaração}\hfill]
\begin{ttfamily}
public const Erro{\_}Operador{\_}aritmetico{\_}esperado          = 229;\end{ttfamily}


\end{flushleft}
\end{list}
\paragraph*{Err{\_}CalcVal{\_}Not{\_}Ready{\_}Number}\hspace*{\fill}

\begin{list}{}{
\settowidth{\tmplength}{\textbf{Declaração}}
\setlength{\itemindent}{0cm}
\setlength{\listparindent}{0cm}
\setlength{\leftmargin}{\evensidemargin}
\addtolength{\leftmargin}{\tmplength}
\settowidth{\labelsep}{X}
\addtolength{\leftmargin}{\labelsep}
\setlength{\labelwidth}{\tmplength}
}
\begin{flushleft}
\item[\textbf{Declaração}\hfill]
\begin{ttfamily}
public const Err{\_}CalcVal{\_}Not{\_}Ready{\_}Number               = 230;\end{ttfamily}


\end{flushleft}
\end{list}
\paragraph*{REC{\_}TOO{\_}LARGE}\hspace*{\fill}

\begin{list}{}{
\settowidth{\tmplength}{\textbf{Declaração}}
\setlength{\itemindent}{0cm}
\setlength{\listparindent}{0cm}
\setlength{\leftmargin}{\evensidemargin}
\addtolength{\leftmargin}{\tmplength}
\settowidth{\labelsep}{X}
\addtolength{\leftmargin}{\labelsep}
\setlength{\labelwidth}{\tmplength}
}
\begin{flushleft}
\item[\textbf{Declaração}\hfill]
\begin{ttfamily}
public const REC{\_}TOO{\_}LARGE                              = 231;\end{ttfamily}


\end{flushleft}
\end{list}
\paragraph*{REC{\_}TOO{\_}SMALL}\hspace*{\fill}

\begin{list}{}{
\settowidth{\tmplength}{\textbf{Declaração}}
\setlength{\itemindent}{0cm}
\setlength{\listparindent}{0cm}
\setlength{\leftmargin}{\evensidemargin}
\addtolength{\leftmargin}{\tmplength}
\settowidth{\labelsep}{X}
\addtolength{\leftmargin}{\labelsep}
\setlength{\labelwidth}{\tmplength}
}
\begin{flushleft}
\item[\textbf{Declaração}\hfill]
\begin{ttfamily}
public const REC{\_}TOO{\_}SMALL                              = 232;\end{ttfamily}


\end{flushleft}
\end{list}
\paragraph*{KeyTooLarge}\hspace*{\fill}

\begin{list}{}{
\settowidth{\tmplength}{\textbf{Declaração}}
\setlength{\itemindent}{0cm}
\setlength{\listparindent}{0cm}
\setlength{\leftmargin}{\evensidemargin}
\addtolength{\leftmargin}{\tmplength}
\settowidth{\labelsep}{X}
\addtolength{\leftmargin}{\labelsep}
\setlength{\labelwidth}{\tmplength}
}
\begin{flushleft}
\item[\textbf{Declaração}\hfill]
\begin{ttfamily}
public const KeyTooLarge                                = 233;\end{ttfamily}


\end{flushleft}
\end{list}
\paragraph*{RecSizeMismatch}\hspace*{\fill}

\begin{list}{}{
\settowidth{\tmplength}{\textbf{Declaração}}
\setlength{\itemindent}{0cm}
\setlength{\listparindent}{0cm}
\setlength{\leftmargin}{\evensidemargin}
\addtolength{\leftmargin}{\tmplength}
\settowidth{\labelsep}{X}
\addtolength{\leftmargin}{\labelsep}
\setlength{\labelwidth}{\tmplength}
}
\begin{flushleft}
\item[\textbf{Declaração}\hfill]
\begin{ttfamily}
public const RecSizeMismatch                            = 234;\end{ttfamily}


\end{flushleft}
\end{list}
\paragraph*{KeySizeMismatch}\hspace*{\fill}

\begin{list}{}{
\settowidth{\tmplength}{\textbf{Declaração}}
\setlength{\itemindent}{0cm}
\setlength{\listparindent}{0cm}
\setlength{\leftmargin}{\evensidemargin}
\addtolength{\leftmargin}{\tmplength}
\settowidth{\labelsep}{X}
\addtolength{\leftmargin}{\labelsep}
\setlength{\labelwidth}{\tmplength}
}
\begin{flushleft}
\item[\textbf{Declaração}\hfill]
\begin{ttfamily}
public const KeySizeMismatch                            = 235;\end{ttfamily}


\end{flushleft}
\end{list}
\paragraph*{MemOverflow}\hspace*{\fill}

\begin{list}{}{
\settowidth{\tmplength}{\textbf{Declaração}}
\setlength{\itemindent}{0cm}
\setlength{\listparindent}{0cm}
\setlength{\leftmargin}{\evensidemargin}
\addtolength{\leftmargin}{\tmplength}
\settowidth{\labelsep}{X}
\addtolength{\leftmargin}{\labelsep}
\setlength{\labelwidth}{\tmplength}
}
\begin{flushleft}
\item[\textbf{Declaração}\hfill]
\begin{ttfamily}
public const MemOverflow                                = 236;\end{ttfamily}


\end{flushleft}
\end{list}
\paragraph*{ArqIndexInconsistente}\hspace*{\fill}

\begin{list}{}{
\settowidth{\tmplength}{\textbf{Declaração}}
\setlength{\itemindent}{0cm}
\setlength{\listparindent}{0cm}
\setlength{\leftmargin}{\evensidemargin}
\addtolength{\leftmargin}{\tmplength}
\settowidth{\labelsep}{X}
\addtolength{\leftmargin}{\labelsep}
\setlength{\labelwidth}{\tmplength}
}
\begin{flushleft}
\item[\textbf{Declaração}\hfill]
\begin{ttfamily}
public const ArqIndexInconsistente                      = 237;\end{ttfamily}


\end{flushleft}
\par
\item[\textbf{Descrição}]
turbo access. Erros Db e DaAccess

\end{list}
\paragraph*{O{\_}gerente{\_}de{\_}transacoes{\_}esta{\_}inativo}\hspace*{\fill}

\begin{list}{}{
\settowidth{\tmplength}{\textbf{Declaração}}
\setlength{\itemindent}{0cm}
\setlength{\listparindent}{0cm}
\setlength{\leftmargin}{\evensidemargin}
\addtolength{\leftmargin}{\tmplength}
\settowidth{\labelsep}{X}
\addtolength{\leftmargin}{\labelsep}
\setlength{\labelwidth}{\tmplength}
}
\begin{flushleft}
\item[\textbf{Declaração}\hfill]
\begin{ttfamily}
public const O{\_}gerente{\_}de{\_}transacoes{\_}esta{\_}inativo       = 238;\end{ttfamily}


\end{flushleft}
\end{list}
\paragraph*{Erro{\_}Excecao{\_}inesperada}\hspace*{\fill}

\begin{list}{}{
\settowidth{\tmplength}{\textbf{Declaração}}
\setlength{\itemindent}{0cm}
\setlength{\listparindent}{0cm}
\setlength{\leftmargin}{\evensidemargin}
\addtolength{\leftmargin}{\tmplength}
\settowidth{\labelsep}{X}
\addtolength{\leftmargin}{\labelsep}
\setlength{\labelwidth}{\tmplength}
}
\begin{flushleft}
\item[\textbf{Declaração}\hfill]
\begin{ttfamily}
public const Erro{\_}Excecao{\_}inesperada                    = 239;\end{ttfamily}


\end{flushleft}
\end{list}
\paragraph*{Acesso{\_}negado{\_}ao{\_}arquivo{\_}por{\_}falta{\_}de{\_}autorizacao{\_}de{\_}seu{\_}superior{\_}imediato}\hspace*{\fill}

\begin{list}{}{
\settowidth{\tmplength}{\textbf{Declaração}}
\setlength{\itemindent}{0cm}
\setlength{\listparindent}{0cm}
\setlength{\leftmargin}{\evensidemargin}
\addtolength{\leftmargin}{\tmplength}
\settowidth{\labelsep}{X}
\addtolength{\leftmargin}{\labelsep}
\setlength{\labelwidth}{\tmplength}
}
\begin{flushleft}
\item[\textbf{Declaração}\hfill]
\begin{ttfamily}
public const Acesso{\_}negado{\_}ao{\_}arquivo{\_}por{\_}falta{\_}de{\_}autorizacao{\_}de{\_}seu{\_}superior{\_}imediato = 240;\end{ttfamily}


\end{flushleft}
\end{list}
\paragraph*{Registro{\_}nao{\_}localizado}\hspace*{\fill}

\begin{list}{}{
\settowidth{\tmplength}{\textbf{Declaração}}
\setlength{\itemindent}{0cm}
\setlength{\listparindent}{0cm}
\setlength{\leftmargin}{\evensidemargin}
\addtolength{\leftmargin}{\tmplength}
\settowidth{\labelsep}{X}
\addtolength{\leftmargin}{\labelsep}
\setlength{\labelwidth}{\tmplength}
}
\begin{flushleft}
\item[\textbf{Declaração}\hfill]
\begin{ttfamily}
public const Registro{\_}nao{\_}localizado                    = 241;\end{ttfamily}


\end{flushleft}
\end{list}
\paragraph*{O{\_}Evento{\_}OnEnter{\_}Retornou{\_}falso}\hspace*{\fill}

\begin{list}{}{
\settowidth{\tmplength}{\textbf{Declaração}}
\setlength{\itemindent}{0cm}
\setlength{\listparindent}{0cm}
\setlength{\leftmargin}{\evensidemargin}
\addtolength{\leftmargin}{\tmplength}
\settowidth{\labelsep}{X}
\addtolength{\leftmargin}{\labelsep}
\setlength{\labelwidth}{\tmplength}
}
\begin{flushleft}
\item[\textbf{Declaração}\hfill]
\begin{ttfamily}
public const O{\_}Evento{\_}OnEnter{\_}Retornou{\_}falso            = 242;\end{ttfamily}


\end{flushleft}
\par
\item[\textbf{Descrição}]
{$>$} Erro retornados nas buscas de registros

\end{list}
\paragraph*{O{\_}Evento{\_}OnExit{\_}Retornou{\_}falso}\hspace*{\fill}

\begin{list}{}{
\settowidth{\tmplength}{\textbf{Declaração}}
\setlength{\itemindent}{0cm}
\setlength{\listparindent}{0cm}
\setlength{\leftmargin}{\evensidemargin}
\addtolength{\leftmargin}{\tmplength}
\settowidth{\labelsep}{X}
\addtolength{\leftmargin}{\labelsep}
\setlength{\labelwidth}{\tmplength}
}
\begin{flushleft}
\item[\textbf{Declaração}\hfill]
\begin{ttfamily}
public const O{\_}Evento{\_}OnExit{\_}Retornou{\_}falso             = 243;\end{ttfamily}


\end{flushleft}
\end{list}
\paragraph*{Attempt{\_}to{\_}insert{\_}record{\_}without{\_}is{\_}selected}\hspace*{\fill}

\begin{list}{}{
\settowidth{\tmplength}{\textbf{Declaração}}
\setlength{\itemindent}{0cm}
\setlength{\listparindent}{0cm}
\setlength{\leftmargin}{\evensidemargin}
\addtolength{\leftmargin}{\tmplength}
\settowidth{\labelsep}{X}
\addtolength{\leftmargin}{\labelsep}
\setlength{\labelwidth}{\tmplength}
}
\begin{flushleft}
\item[\textbf{Declaração}\hfill]
\begin{ttfamily}
public const Attempt{\_}to{\_}insert{\_}record{\_}without{\_}is{\_}selected = 244;\end{ttfamily}


\end{flushleft}
\par
\item[\textbf{Descrição}]
Erro gerado ao tentar incluir um registro sem que o mesmo não esteja no modo appending

\end{list}
\paragraph*{attempt{\_}to{\_}edit{\_}a{\_}record{\_}not{\_}selecting}\hspace*{\fill}

\begin{list}{}{
\settowidth{\tmplength}{\textbf{Declaração}}
\setlength{\itemindent}{0cm}
\setlength{\listparindent}{0cm}
\setlength{\leftmargin}{\evensidemargin}
\addtolength{\leftmargin}{\tmplength}
\settowidth{\labelsep}{X}
\addtolength{\leftmargin}{\labelsep}
\setlength{\labelwidth}{\tmplength}
}
\begin{flushleft}
\item[\textbf{Declaração}\hfill]
\begin{ttfamily}
public const attempt{\_}to{\_}edit{\_}a{\_}record{\_}not{\_}selecting       = 245;\end{ttfamily}


\end{flushleft}
\end{list}
\paragraph*{LastError}\hspace*{\fill}

\begin{list}{}{
\settowidth{\tmplength}{\textbf{Declaração}}
\setlength{\itemindent}{0cm}
\setlength{\listparindent}{0cm}
\setlength{\leftmargin}{\evensidemargin}
\addtolength{\leftmargin}{\tmplength}
\settowidth{\labelsep}{X}
\addtolength{\leftmargin}{\labelsep}
\setlength{\labelwidth}{\tmplength}
}
\begin{flushleft}
\item[\textbf{Declaração}\hfill]
\begin{ttfamily}
public const LastError : SmallInt =  0;\end{ttfamily}


\end{flushleft}
\par
\item[\textbf{Descrição}]
\begin{itemize}
\item Após uma chamada ao sistema operacional a variável publica global \textbf{\begin{ttfamily}LastError\end{ttfamily}} guarda \textbf{0 (zero)} se sucesso ou o \textbf{código do erro} se houve fracasso. \begin{itemize}
\item A função \textbf{\begin{ttfamily}LastError\end{ttfamily}} é atualizada em SetResult.
\end{itemize}
\end{itemize}

\end{list}
\paragraph*{TaStatus}\hspace*{\fill}

\begin{list}{}{
\settowidth{\tmplength}{\textbf{Declaração}}
\setlength{\itemindent}{0cm}
\setlength{\listparindent}{0cm}
\setlength{\leftmargin}{\evensidemargin}
\addtolength{\leftmargin}{\tmplength}
\settowidth{\labelsep}{X}
\addtolength{\leftmargin}{\labelsep}
\setlength{\labelwidth}{\tmplength}
}
\begin{flushleft}
\item[\textbf{Declaração}\hfill]
\begin{ttfamily}
public const TaStatus     : SmallInt = 0;\end{ttfamily}


\end{flushleft}
\par
\item[\textbf{Descrição}]
\begin{itemize}
\item A variável pública global \textbf{\begin{ttfamily}TaStatus\end{ttfamily}} indica o status da última operação de acesso ao banco de dados Turbo Access. \begin{itemize}
\item Nota: Sua função é semelhante a \textbf{\begin{ttfamily}LastError\end{ttfamily}(\ref{mi.rtl.Consts.TConsts-LastError})};
\end{itemize}
\end{itemize}

\end{list}
\paragraph*{OK}\hspace*{\fill}

\begin{list}{}{
\settowidth{\tmplength}{\textbf{Declaração}}
\setlength{\itemindent}{0cm}
\setlength{\listparindent}{0cm}
\setlength{\leftmargin}{\evensidemargin}
\addtolength{\leftmargin}{\tmplength}
\settowidth{\labelsep}{X}
\addtolength{\leftmargin}{\labelsep}
\setlength{\labelwidth}{\tmplength}
}
\begin{flushleft}
\item[\textbf{Declaração}\hfill]
\begin{ttfamily}
public const OK  : Boolean = True;\end{ttfamily}


\end{flushleft}
\par
\item[\textbf{Descrição}]
\begin{itemize}
\item A variável pública global \textbf{\begin{ttfamily}OK\end{ttfamily}} indica se houver erro na última ação. \begin{itemize}
\item Nota: Atualizada em \textbf{SetResult} onde \textbf{\begin{ttfamily}OK\end{ttfamily}=true} houve sucesso e \textbf{\begin{ttfamily}OK\end{ttfamily}=false} houve fracasso.
\end{itemize}
\end{itemize}

\end{list}
\paragraph*{FileMode}\hspace*{\fill}

\begin{list}{}{
\settowidth{\tmplength}{\textbf{Declaração}}
\setlength{\itemindent}{0cm}
\setlength{\listparindent}{0cm}
\setlength{\leftmargin}{\evensidemargin}
\addtolength{\leftmargin}{\tmplength}
\settowidth{\labelsep}{X}
\addtolength{\leftmargin}{\labelsep}
\setlength{\labelwidth}{\tmplength}
}
\begin{flushleft}
\item[\textbf{Declaração}\hfill]
\begin{ttfamily}
public const FileMode : word    = fmOpenReadWrite ;\end{ttfamily}


\end{flushleft}
\par
\item[\textbf{Descrição}]
A constante pública \textbf{\begin{ttfamily}FileMode\end{ttfamily}} guarda o modo padrão de abertura dos arquivos;

\begin{itemize}
\item \textbf{NOTAS}: \begin{itemize}
\item Usada em \textbf{FileOpen} e \textbf{FileCreate}.
\item O mapa de bits usado \textbf{\begin{ttfamily}FileMode\end{ttfamily}} é inicializado com: \begin{itemize}
\item Const \textbf{\begin{ttfamily}FileMode\end{ttfamily}} : \begin{ttfamily}word\end{ttfamily}(\ref{mi.rtl.Types.TTypes-Word}) = \begin{ttfamily}fmOpenReadWrite\end{ttfamily}(\ref{mi.rtl.Consts.TConsts-fmOpenReadWrite});
\end{itemize}
\end{itemize}
\item \textbf{EXEMPLO}

\texttt{\\\nopagebreak[3]
}\textbf{procedure}\texttt{~TMi{\_}Rtl{\_}Tests.Action{\_}test{\_}FileCreateExecute(Sender:~TObject);\\\nopagebreak[3]
\\\nopagebreak[3]
~~}\textbf{var}\texttt{\\\nopagebreak[3]
~~~~aHandle,aHandle2,aHandle3~:~~TMI{\_}ui{\_}types.THandle;\\\nopagebreak[3]
~~~~err:integer;\\\nopagebreak[3]
~~~~s~:~AnsiString;\\\nopagebreak[3]
}\textbf{begin}\texttt{\\\nopagebreak[3]
~~}\textbf{with}\texttt{~TMI{\_}ui{\_}types~}\textbf{do}\texttt{\\\nopagebreak[3]
~~}\textbf{begin}\texttt{\\\nopagebreak[3]
~~~~err~:=~FileCreate('text.txt',fileMode,~ShareMode~,aHandle);\\\nopagebreak[3]
~~~~}\textbf{if}\texttt{~err~=~0\\\nopagebreak[3]
~~~~}\textbf{then}\texttt{~}\textbf{begin}\texttt{\\\nopagebreak[3]
~~~~~~~~~~~SysMessageBox('Arquivo~text.txt~criado~na~pasta~corrente.','Action{\_}test{\_}FileCreateExecute',false);\\\nopagebreak[3]
~~~~~~~~~~~s~:=~ExpandFileName('text.txt');\\\nopagebreak[3]
\\\nopagebreak[3]
~~~~~~~~~~~FileMode~:=~fmOpenReadWrite;\\\nopagebreak[3]
~~~~~~~~~~~ShareMode~:=~fmShareCompat~}\textbf{or}\texttt{~fmShareDenyNone;\\\nopagebreak[3]
\\\nopagebreak[3]
~~~~~~~~~~~err~:=~FileOpen(s,FileMode,~shareMode,aHandle2);\\\nopagebreak[3]
~~~~~~~~~~~}\textbf{if}\texttt{~err~=~0\\\nopagebreak[3]
~~~~~~~~~~~}\textbf{Then}\texttt{~}\textbf{begin}\texttt{\\\nopagebreak[3]
~~~~~~~~~~~~~~~~~~SysMessageBox('Arquivo~text.txt~aberto~com~o~modo~fmOpenReadWrite~e~atributo~fmShareCompat~or~fmShareDenyNone.','Action{\_}test{\_}FileCreateExecute',false);\\\nopagebreak[3]
~~~~~~~~~~~~~~~~~~FileClose(aHandle2);\\\nopagebreak[3]
~~~~~~~~~~~~~~~~}\textbf{end}\texttt{\\\nopagebreak[3]
~~~~~~~~~~~}\textbf{else}\texttt{~SysMessageBox(TStrError.ErrorMessage(err),'Action{\_}test{\_}FileCreateExecute',true);\\\nopagebreak[3]
\\\nopagebreak[3]
~~~~~~~~~~~ShareMode~:=~fmShareCompat~}\textbf{or}\texttt{~fmShareExclusive;\\\nopagebreak[3]
~~~~~~~~~~~err~:=~FileOpen(s,fileMode,ShareMode~,aHandle3);\\\nopagebreak[3]
~~~~~~~~~~~}\textbf{if}\texttt{~err~=~0\\\nopagebreak[3]
~~~~~~~~~~~}\textbf{Then}\texttt{~}\textbf{begin}\texttt{\\\nopagebreak[3]
~~~~~~~~~~~~~~~~~~SysMessageBox('Arquivo~text.txt~aberto~com~o~modo~fmOpenReadWrite~e~atributo~fmShareCompat~or~fmShareExclusive.','Action{\_}test{\_}FileCreateExecute',false);\\\nopagebreak[3]
~~~~~~~~~~~~~~~~~~FileClose(aHandle3);\\\nopagebreak[3]
~~~~~~~~~~~~~~~~}\textbf{end}\texttt{\\\nopagebreak[3]
~~~~~~~~~~~}\textbf{else}\texttt{~SysMessageBox(TStrError.ErrorMessage(err),'Action{\_}test{\_}FileCreateExecute',true);\\\nopagebreak[3]
\\\nopagebreak[3]
\\\nopagebreak[3]
~~~~~~~~~~~FileClose(aHandle);\\\nopagebreak[3]
~~~~~~~~~}\textbf{end}\texttt{\\\nopagebreak[3]
~~~~~~~~~}\textbf{else}\texttt{~SysMessageBox(TStrError.ErrorMessage(err),'Action{\_}test{\_}FileCreateExecute',true);\\\nopagebreak[3]
~~}\textbf{end}\texttt{;\\\nopagebreak[3]
}\textbf{end}\texttt{;\\
}
\end{itemize}

\end{list}
\paragraph*{FileModeAnt}\hspace*{\fill}

\begin{list}{}{
\settowidth{\tmplength}{\textbf{Declaração}}
\setlength{\itemindent}{0cm}
\setlength{\listparindent}{0cm}
\setlength{\leftmargin}{\evensidemargin}
\addtolength{\leftmargin}{\tmplength}
\settowidth{\labelsep}{X}
\addtolength{\leftmargin}{\labelsep}
\setlength{\labelwidth}{\tmplength}
}
\begin{flushleft}
\item[\textbf{Declaração}\hfill]
\begin{ttfamily}
public const FileModeAnt : Word  = 0;\end{ttfamily}


\end{flushleft}
\par
\item[\textbf{Descrição}]
\begin{itemize}
\item A function \textbf{SetFileMode} salva a variável \textbf{\begin{ttfamily}FileModeAnt\end{ttfamily}} atual antes de modificar \textbf{\begin{ttfamily}fileMode\end{ttfamily}(\ref{mi.rtl.Consts.TConsts-FileMode})}.;
\end{itemize}

\end{list}
\paragraph*{ShareMode}\hspace*{\fill}

\begin{list}{}{
\settowidth{\tmplength}{\textbf{Declaração}}
\setlength{\itemindent}{0cm}
\setlength{\listparindent}{0cm}
\setlength{\leftmargin}{\evensidemargin}
\addtolength{\leftmargin}{\tmplength}
\settowidth{\labelsep}{X}
\addtolength{\leftmargin}{\labelsep}
\setlength{\labelwidth}{\tmplength}
}
\begin{flushleft}
\item[\textbf{Declaração}\hfill]
\begin{ttfamily}
public const ShareMode : Cardinal = fmShareCompat or fmShareDenyNone;\end{ttfamily}


\end{flushleft}
\par
\item[\textbf{Descrição}]
A constante pública \textbf{\begin{ttfamily}ShareMode\end{ttfamily}} guarda o modo padrão de compartilhamento na abertura dos arquivos;

\begin{itemize}
\item \textbf{NOTAS}: \begin{itemize}
\item Usada em \textbf{FileOpen} e \textbf{FileCreate}.
\item O mapa de bits usado \textbf{\begin{ttfamily}ShareMode\end{ttfamily}} é inicializado com: \begin{itemize}
\item Const \textbf{\begin{ttfamily}ShareMode\end{ttfamily}} : cardinal = \begin{ttfamily}fmShareCompat\end{ttfamily}(\ref{mi.rtl.Consts.TConsts-fmShareCompat}) or \begin{ttfamily}fmShareDenyNone\end{ttfamily}(\ref{mi.rtl.Consts.TConsts-fmShareDenyNone});
\end{itemize}
\end{itemize}
\item \textbf{EXEMPLO}

\texttt{\\\nopagebreak[3]
\\\nopagebreak[3]
}\textbf{procedure}\texttt{~TMi{\_}Rtl{\_}Tests.Action{\_}test{\_}FileCreateExecute(Sender:~TObject);\\\nopagebreak[3]
\\\nopagebreak[3]
~~}\textbf{var}\texttt{\\\nopagebreak[3]
~~~~aHandle,aHandle2,aHandle3~:~~TMI{\_}ui{\_}types.THandle;\\\nopagebreak[3]
~~~~err:integer;\\\nopagebreak[3]
~~~~s~:~AnsiString;\\\nopagebreak[3]
}\textbf{begin}\texttt{\\\nopagebreak[3]
~~}\textbf{with}\texttt{~TMI{\_}ui{\_}types~}\textbf{do}\texttt{\\\nopagebreak[3]
~~}\textbf{begin}\texttt{\\\nopagebreak[3]
~~~~err~:=~FileCreate('text.txt',fileMode,~ShareMode~,aHandle);\\\nopagebreak[3]
~~~~}\textbf{if}\texttt{~err~=~0\\\nopagebreak[3]
~~~~}\textbf{then}\texttt{~}\textbf{begin}\texttt{\\\nopagebreak[3]
~~~~~~~~~~~SysMessageBox('Arquivo~text.txt~criado~na~pasta~corrente.','Action{\_}test{\_}FileCreateExecute',false);\\\nopagebreak[3]
~~~~~~~~~~~s~:=~ExpandFileName('text.txt');\\\nopagebreak[3]
\\\nopagebreak[3]
~~~~~~~~~~~FileMode~:=~fmOpenReadWrite;\\\nopagebreak[3]
~~~~~~~~~~~ShareMode~:=~fmShareCompat~}\textbf{or}\texttt{~fmShareDenyNone;\\\nopagebreak[3]
\\\nopagebreak[3]
~~~~~~~~~~~err~:=~FileOpen(s,FileMode,~shareMode,aHandle2);\\\nopagebreak[3]
~~~~~~~~~~~}\textbf{if}\texttt{~err~=~0\\\nopagebreak[3]
~~~~~~~~~~~}\textbf{Then}\texttt{~}\textbf{begin}\texttt{\\\nopagebreak[3]
~~~~~~~~~~~~~~~~~~SysMessageBox('Arquivo~text.txt~aberto~com~o~modo~fmOpenReadWrite~e~atributo~fmShareCompat~or~fmShareDenyNone.','Action{\_}test{\_}FileCreateExecute',false);\\\nopagebreak[3]
~~~~~~~~~~~~~~~~~~FileClose(aHandle2);\\\nopagebreak[3]
~~~~~~~~~~~~~~~~}\textbf{end}\texttt{\\\nopagebreak[3]
~~~~~~~~~~~}\textbf{else}\texttt{~SysMessageBox(TStrError.ErrorMessage(err),'Action{\_}test{\_}FileCreateExecute',true);\\\nopagebreak[3]
\\\nopagebreak[3]
~~~~~~~~~~~ShareMode~:=~fmShareCompat~}\textbf{or}\texttt{~fmShareExclusive;\\\nopagebreak[3]
~~~~~~~~~~~err~:=~FileOpen(s,fileMode,ShareMode~,aHandle3);\\\nopagebreak[3]
~~~~~~~~~~~}\textbf{if}\texttt{~err~=~0\\\nopagebreak[3]
~~~~~~~~~~~}\textbf{Then}\texttt{~}\textbf{begin}\texttt{\\\nopagebreak[3]
~~~~~~~~~~~~~~~~~~SysMessageBox('Arquivo~text.txt~aberto~com~o~modo~fmOpenReadWrite~e~atributo~fmShareCompat~or~fmShareExclusive.','Action{\_}test{\_}FileCreateExecute',false);\\\nopagebreak[3]
~~~~~~~~~~~~~~~~~~FileClose(aHandle3);\\\nopagebreak[3]
~~~~~~~~~~~~~~~~}\textbf{end}\texttt{\\\nopagebreak[3]
~~~~~~~~~~~}\textbf{else}\texttt{~SysMessageBox(TStrError.ErrorMessage(err),'Action{\_}test{\_}FileCreateExecute',true);\\\nopagebreak[3]
\\\nopagebreak[3]
\\\nopagebreak[3]
~~~~~~~~~~~FileClose(aHandle);\\\nopagebreak[3]
~~~~~~~~~}\textbf{end}\texttt{\\\nopagebreak[3]
~~~~~~~~~}\textbf{else}\texttt{~SysMessageBox(TStrError.ErrorMessage(err),'Action{\_}test{\_}FileCreateExecute',true);\\\nopagebreak[3]
~~}\textbf{end}\texttt{;\\\nopagebreak[3]
}\textbf{end}\texttt{;\\
}
\end{itemize}

\end{list}
\paragraph*{MaxDirSizeFat32}\hspace*{\fill}

\begin{list}{}{
\settowidth{\tmplength}{\textbf{Declaração}}
\setlength{\itemindent}{0cm}
\setlength{\listparindent}{0cm}
\setlength{\leftmargin}{\evensidemargin}
\addtolength{\leftmargin}{\tmplength}
\settowidth{\labelsep}{X}
\addtolength{\leftmargin}{\labelsep}
\setlength{\labelwidth}{\tmplength}
}
\begin{flushleft}
\item[\textbf{Declaração}\hfill]
\begin{ttfamily}
public const MaxDirSizeFat32  = 65534;\end{ttfamily}


\end{flushleft}
\end{list}
\paragraph*{MaxDirSizeNTFS}\hspace*{\fill}

\begin{list}{}{
\settowidth{\tmplength}{\textbf{Declaração}}
\setlength{\itemindent}{0cm}
\setlength{\listparindent}{0cm}
\setlength{\leftmargin}{\evensidemargin}
\addtolength{\leftmargin}{\tmplength}
\settowidth{\labelsep}{X}
\addtolength{\leftmargin}{\labelsep}
\setlength{\labelwidth}{\tmplength}
}
\begin{flushleft}
\item[\textbf{Declaração}\hfill]
\begin{ttfamily}
public const MaxDirSizeNTFS   = 2294967295;\end{ttfamily}


\end{flushleft}
\end{list}
\paragraph*{MaxDirSizeLinux}\hspace*{\fill}

\begin{list}{}{
\settowidth{\tmplength}{\textbf{Declaração}}
\setlength{\itemindent}{0cm}
\setlength{\listparindent}{0cm}
\setlength{\leftmargin}{\evensidemargin}
\addtolength{\leftmargin}{\tmplength}
\settowidth{\labelsep}{X}
\addtolength{\leftmargin}{\labelsep}
\setlength{\labelwidth}{\tmplength}
}
\begin{flushleft}
\item[\textbf{Declaração}\hfill]
\begin{ttfamily}
public const MaxDirSizeLinux  = MaxDirSizeNTFS;\end{ttfamily}


\end{flushleft}
\par
\item[\textbf{Descrição}]
\begin{itemize}
\item Máximo de pastas dos sistemas de arquivo do linux é limitada ao espaço em disco.
\end{itemize}

\end{list}
\paragraph*{MaxDirSize}\hspace*{\fill}

\begin{list}{}{
\settowidth{\tmplength}{\textbf{Declaração}}
\setlength{\itemindent}{0cm}
\setlength{\listparindent}{0cm}
\setlength{\leftmargin}{\evensidemargin}
\addtolength{\leftmargin}{\tmplength}
\settowidth{\labelsep}{X}
\addtolength{\leftmargin}{\labelsep}
\setlength{\labelwidth}{\tmplength}
}
\begin{flushleft}
\item[\textbf{Declaração}\hfill]
\begin{ttfamily}
public const MaxDirSize : word = MaxDirSizeLinux;\end{ttfamily}


\end{flushleft}
\end{list}
\paragraph*{Auto{\_}Add{\_}Line{\_}Default}\hspace*{\fill}

\begin{list}{}{
\settowidth{\tmplength}{\textbf{Declaração}}
\setlength{\itemindent}{0cm}
\setlength{\listparindent}{0cm}
\setlength{\leftmargin}{\evensidemargin}
\addtolength{\leftmargin}{\tmplength}
\settowidth{\labelsep}{X}
\addtolength{\leftmargin}{\labelsep}
\setlength{\labelwidth}{\tmplength}
}
\begin{flushleft}
\item[\textbf{Declaração}\hfill]
\begin{ttfamily}
public const Auto{\_}Add{\_}Line{\_}Default:Boolean = false;\end{ttfamily}


\end{flushleft}
\par
\item[\textbf{Descrição}]
\begin{itemize}
\item A contante \textbf{\begin{ttfamily}Auto{\_}Add{\_}Line{\_}Default\end{ttfamily}} é usada na construção de formulários de entrada automaticamente.

\begin{itemize}
\item \textbf{NOTA} \begin{itemize}
\item \textbf{true} o formulário de entrada de dados insere uma linha em branco automaticamente.
\end{itemize}
\end{itemize}
\end{itemize}

\end{list}
\paragraph*{Comma}\hspace*{\fill}

\begin{list}{}{
\settowidth{\tmplength}{\textbf{Declaração}}
\setlength{\itemindent}{0cm}
\setlength{\listparindent}{0cm}
\setlength{\leftmargin}{\evensidemargin}
\addtolength{\leftmargin}{\tmplength}
\settowidth{\labelsep}{X}
\addtolength{\leftmargin}{\labelsep}
\setlength{\labelwidth}{\tmplength}
}
\begin{flushleft}
\item[\textbf{Declaração}\hfill]
\begin{ttfamily}
public const Comma     : char  =   ',';\end{ttfamily}


\end{flushleft}
\par
\item[\textbf{Descrição}]
\begin{itemize}
\item Separador de milhar nas mascaras internas ao campo.
\end{itemize}

\end{list}
\paragraph*{showComma}\hspace*{\fill}

\begin{list}{}{
\settowidth{\tmplength}{\textbf{Declaração}}
\setlength{\itemindent}{0cm}
\setlength{\listparindent}{0cm}
\setlength{\leftmargin}{\evensidemargin}
\addtolength{\leftmargin}{\tmplength}
\settowidth{\labelsep}{X}
\addtolength{\leftmargin}{\labelsep}
\setlength{\labelwidth}{\tmplength}
}
\begin{flushleft}
\item[\textbf{Declaração}\hfill]
\begin{ttfamily}
public const showComma : char  =   '.';\end{ttfamily}


\end{flushleft}
\par
\item[\textbf{Descrição}]
\begin{itemize}
\item Separador de números na visualização {\}}
\end{itemize}

\end{list}
\paragraph*{DecPt}\hspace*{\fill}

\begin{list}{}{
\settowidth{\tmplength}{\textbf{Declaração}}
\setlength{\itemindent}{0cm}
\setlength{\listparindent}{0cm}
\setlength{\leftmargin}{\evensidemargin}
\addtolength{\leftmargin}{\tmplength}
\settowidth{\labelsep}{X}
\addtolength{\leftmargin}{\labelsep}
\setlength{\labelwidth}{\tmplength}
}
\begin{flushleft}
\item[\textbf{Declaração}\hfill]
\begin{ttfamily}
public const DecPt      : char =   '.';\end{ttfamily}


\end{flushleft}
\par
\item[\textbf{Descrição}]
\begin{itemize}
\item Ponto decimal usado nas mascaras internas ao campo.
\end{itemize}

\end{list}
\paragraph*{showDecPt}\hspace*{\fill}

\begin{list}{}{
\settowidth{\tmplength}{\textbf{Declaração}}
\setlength{\itemindent}{0cm}
\setlength{\listparindent}{0cm}
\setlength{\leftmargin}{\evensidemargin}
\addtolength{\leftmargin}{\tmplength}
\settowidth{\labelsep}{X}
\addtolength{\leftmargin}{\labelsep}
\setlength{\labelwidth}{\tmplength}
}
\begin{flushleft}
\item[\textbf{Declaração}\hfill]
\begin{ttfamily}
public const showDecPt  : Char =   ',';\end{ttfamily}


\end{flushleft}
\par
\item[\textbf{Descrição}]
\begin{itemize}
\item \begin{ttfamily}Char\end{ttfamily}(\ref{mi.rtl.Types.TTypes-Char}) decimal point display {\}}
\end{itemize}

\end{list}
\paragraph*{CloseParenthesis}\hspace*{\fill}

\begin{list}{}{
\settowidth{\tmplength}{\textbf{Declaração}}
\setlength{\itemindent}{0cm}
\setlength{\listparindent}{0cm}
\setlength{\leftmargin}{\evensidemargin}
\addtolength{\leftmargin}{\tmplength}
\settowidth{\labelsep}{X}
\addtolength{\leftmargin}{\labelsep}
\setlength{\labelwidth}{\tmplength}
}
\begin{flushleft}
\item[\textbf{Declaração}\hfill]
\begin{ttfamily}
public const CloseParenthesis = ')';\end{ttfamily}


\end{flushleft}
\end{list}
\paragraph*{OpenParenthesis}\hspace*{\fill}

\begin{list}{}{
\settowidth{\tmplength}{\textbf{Declaração}}
\setlength{\itemindent}{0cm}
\setlength{\listparindent}{0cm}
\setlength{\leftmargin}{\evensidemargin}
\addtolength{\leftmargin}{\tmplength}
\settowidth{\labelsep}{X}
\addtolength{\leftmargin}{\labelsep}
\setlength{\labelwidth}{\tmplength}
}
\begin{flushleft}
\item[\textbf{Declaração}\hfill]
\begin{ttfamily}
public const OpenParenthesis  = '(';\end{ttfamily}


\end{flushleft}
\end{list}
\paragraph*{CharDelimiter{\_}0}\hspace*{\fill}

\begin{list}{}{
\settowidth{\tmplength}{\textbf{Declaração}}
\setlength{\itemindent}{0cm}
\setlength{\listparindent}{0cm}
\setlength{\leftmargin}{\evensidemargin}
\addtolength{\leftmargin}{\tmplength}
\settowidth{\labelsep}{X}
\addtolength{\leftmargin}{\labelsep}
\setlength{\labelwidth}{\tmplength}
}
\begin{flushleft}
\item[\textbf{Declaração}\hfill]
\begin{ttfamily}
public const CharDelimiter{\_}0 = {\#}0;\end{ttfamily}


\end{flushleft}
\par
\item[\textbf{Descrição}]
A constante \textbf{\begin{ttfamily}CharDelimiter{\_}0\end{ttfamily}} indica qua a sequẽncia seguite é um campo de dados

\end{list}
\paragraph*{CharDelimiter{\_}1}\hspace*{\fill}

\begin{list}{}{
\settowidth{\tmplength}{\textbf{Declaração}}
\setlength{\itemindent}{0cm}
\setlength{\listparindent}{0cm}
\setlength{\leftmargin}{\evensidemargin}
\addtolength{\leftmargin}{\tmplength}
\settowidth{\labelsep}{X}
\addtolength{\leftmargin}{\labelsep}
\setlength{\labelwidth}{\tmplength}
}
\begin{flushleft}
\item[\textbf{Declaração}\hfill]
\begin{ttfamily}
public const CharDelimiter{\_}1 = '{\textbackslash}';\end{ttfamily}


\end{flushleft}
\par
\item[\textbf{Descrição}]
A constante \textbf{\begin{ttfamily}CharDelimiter{\_}1\end{ttfamily}} indica qua a sequẽncia seguite é um campo de dados

\end{list}
\paragraph*{CharDelimiter{\_}2}\hspace*{\fill}

\begin{list}{}{
\settowidth{\tmplength}{\textbf{Declaração}}
\setlength{\itemindent}{0cm}
\setlength{\listparindent}{0cm}
\setlength{\leftmargin}{\evensidemargin}
\addtolength{\leftmargin}{\tmplength}
\settowidth{\labelsep}{X}
\addtolength{\leftmargin}{\labelsep}
\setlength{\labelwidth}{\tmplength}
}
\begin{flushleft}
\item[\textbf{Declaração}\hfill]
\begin{ttfamily}
public const CharDelimiter{\_}2 = '|';\end{ttfamily}


\end{flushleft}
\par
\item[\textbf{Descrição}]
A constante \textbf{\begin{ttfamily}CharDelimiter{\_}2\end{ttfamily}} separa nome da tabela do nome do campo

\end{list}
\paragraph*{CharDelimiter{\_}3}\hspace*{\fill}

\begin{list}{}{
\settowidth{\tmplength}{\textbf{Declaração}}
\setlength{\itemindent}{0cm}
\setlength{\listparindent}{0cm}
\setlength{\leftmargin}{\evensidemargin}
\addtolength{\leftmargin}{\tmplength}
\settowidth{\labelsep}{X}
\addtolength{\leftmargin}{\labelsep}
\setlength{\labelwidth}{\tmplength}
}
\begin{flushleft}
\item[\textbf{Declaração}\hfill]
\begin{ttfamily}
public const CharDelimiter{\_}3  = '~';\end{ttfamily}


\end{flushleft}
\end{list}
\paragraph*{CharShowzeroes}\hspace*{\fill}

\begin{list}{}{
\settowidth{\tmplength}{\textbf{Declaração}}
\setlength{\itemindent}{0cm}
\setlength{\listparindent}{0cm}
\setlength{\leftmargin}{\evensidemargin}
\addtolength{\leftmargin}{\tmplength}
\settowidth{\labelsep}{X}
\addtolength{\leftmargin}{\labelsep}
\setlength{\labelwidth}{\tmplength}
}
\begin{flushleft}
\item[\textbf{Declaração}\hfill]
\begin{ttfamily}
public const CharShowzeroes = {\^{}}Z;\end{ttfamily}


\end{flushleft}
\par
\item[\textbf{Descrição}]
A constante \textbf{\begin{ttfamily}CharShowzeroes\end{ttfamily}} inicializa o registro com zeros

\end{list}
\paragraph*{CharFillvalue}\hspace*{\fill}

\begin{list}{}{
\settowidth{\tmplength}{\textbf{Declaração}}
\setlength{\itemindent}{0cm}
\setlength{\listparindent}{0cm}
\setlength{\leftmargin}{\evensidemargin}
\addtolength{\leftmargin}{\tmplength}
\settowidth{\labelsep}{X}
\addtolength{\leftmargin}{\labelsep}
\setlength{\labelwidth}{\tmplength}
}
\begin{flushleft}
\item[\textbf{Declaração}\hfill]
\begin{ttfamily}
public const CharFillvalue = {\^{}}V;\end{ttfamily}


\end{flushleft}
\par
\item[\textbf{Descrição}]
Se o campo for numérico, preencha com '{\#}0'(\begin{ttfamily}AccNormal\end{ttfamily}(\ref{mi.rtl.Consts.TConsts-accNormal})) se for alfanumérico, preencha com ' ' \begin{ttfamily}AccNormal\end{ttfamily}(\ref{mi.rtl.Consts.TConsts-accNormal})

\end{list}
\paragraph*{CharAccHidden}\hspace*{\fill}

\begin{list}{}{
\settowidth{\tmplength}{\textbf{Declaração}}
\setlength{\itemindent}{0cm}
\setlength{\listparindent}{0cm}
\setlength{\leftmargin}{\evensidemargin}
\addtolength{\leftmargin}{\tmplength}
\settowidth{\labelsep}{X}
\addtolength{\leftmargin}{\labelsep}
\setlength{\labelwidth}{\tmplength}
}
\begin{flushleft}
\item[\textbf{Declaração}\hfill]
\begin{ttfamily}
public const CharAccHidden = {\^{}}H;\end{ttfamily}


\end{flushleft}
\par
\item[\textbf{Descrição}]
A contante \textbf{\begin{ttfamily}CharAccHidden\end{ttfamily}} torna o campo invisível

\end{list}
\paragraph*{ChAH}\hspace*{\fill}

\begin{list}{}{
\settowidth{\tmplength}{\textbf{Declaração}}
\setlength{\itemindent}{0cm}
\setlength{\listparindent}{0cm}
\setlength{\leftmargin}{\evensidemargin}
\addtolength{\leftmargin}{\tmplength}
\settowidth{\labelsep}{X}
\addtolength{\leftmargin}{\labelsep}
\setlength{\labelwidth}{\tmplength}
}
\begin{flushleft}
\item[\textbf{Declaração}\hfill]
\begin{ttfamily}
public const ChAH          = CharAccHidden;\end{ttfamily}


\end{flushleft}
\end{list}
\paragraph*{CharAccSkip}\hspace*{\fill}

\begin{list}{}{
\settowidth{\tmplength}{\textbf{Declaração}}
\setlength{\itemindent}{0cm}
\setlength{\listparindent}{0cm}
\setlength{\leftmargin}{\evensidemargin}
\addtolength{\leftmargin}{\tmplength}
\settowidth{\labelsep}{X}
\addtolength{\leftmargin}{\labelsep}
\setlength{\labelwidth}{\tmplength}
}
\begin{flushleft}
\item[\textbf{Declaração}\hfill]
\begin{ttfamily}
public const CharAccSkip   = {\^{}}S;\end{ttfamily}


\end{flushleft}
\par
\item[\textbf{Descrição}]
A constante \textbf{\begin{ttfamily}CharAccSkip\end{ttfamily}} indica que o campo não pode receber o foco.

\end{list}
\paragraph*{ChAS}\hspace*{\fill}

\begin{list}{}{
\settowidth{\tmplength}{\textbf{Declaração}}
\setlength{\itemindent}{0cm}
\setlength{\listparindent}{0cm}
\setlength{\leftmargin}{\evensidemargin}
\addtolength{\leftmargin}{\tmplength}
\settowidth{\labelsep}{X}
\addtolength{\leftmargin}{\labelsep}
\setlength{\labelwidth}{\tmplength}
}
\begin{flushleft}
\item[\textbf{Declaração}\hfill]
\begin{ttfamily}
public const ChAS   = CharAccSkip;\end{ttfamily}


\end{flushleft}
\par
\item[\textbf{Descrição}]
A constante \textbf{\begin{ttfamily}ChAS\end{ttfamily}} é igual a \begin{ttfamily}CharAccSkip\end{ttfamily}(\ref{mi.rtl.Consts.TConsts-CharAccSkip})

\end{list}
\paragraph*{CharAccReadOnly}\hspace*{\fill}

\begin{list}{}{
\settowidth{\tmplength}{\textbf{Declaração}}
\setlength{\itemindent}{0cm}
\setlength{\listparindent}{0cm}
\setlength{\leftmargin}{\evensidemargin}
\addtolength{\leftmargin}{\tmplength}
\settowidth{\labelsep}{X}
\addtolength{\leftmargin}{\labelsep}
\setlength{\labelwidth}{\tmplength}
}
\begin{flushleft}
\item[\textbf{Declaração}\hfill]
\begin{ttfamily}
public const CharAccReadOnly = {\^{}}R;\end{ttfamily}


\end{flushleft}
\par
\item[\textbf{Descrição}]
A constante \textit{\begin{ttfamily}CharAccReadOnly\end{ttfamily}}* informa que o tipo de acesso ao campo é somente para leitura e não pode ser editado.

\end{list}
\paragraph*{ChARO}\hspace*{\fill}

\begin{list}{}{
\settowidth{\tmplength}{\textbf{Declaração}}
\setlength{\itemindent}{0cm}
\setlength{\listparindent}{0cm}
\setlength{\leftmargin}{\evensidemargin}
\addtolength{\leftmargin}{\tmplength}
\settowidth{\labelsep}{X}
\addtolength{\leftmargin}{\labelsep}
\setlength{\labelwidth}{\tmplength}
}
\begin{flushleft}
\item[\textbf{Declaração}\hfill]
\begin{ttfamily}
public const ChARO   = CharAccReadOnly;\end{ttfamily}


\end{flushleft}
\par
\item[\textbf{Descrição}]
A constante \textbf{\begin{ttfamily}ChARO\end{ttfamily}} é igual a \begin{ttfamily}CharAccReadOnly\end{ttfamily}(\ref{mi.rtl.Consts.TConsts-CharAccReadOnly})

\end{list}
\paragraph*{CharAllZeroes}\hspace*{\fill}

\begin{list}{}{
\settowidth{\tmplength}{\textbf{Declaração}}
\setlength{\itemindent}{0cm}
\setlength{\listparindent}{0cm}
\setlength{\leftmargin}{\evensidemargin}
\addtolength{\leftmargin}{\tmplength}
\settowidth{\labelsep}{X}
\addtolength{\leftmargin}{\labelsep}
\setlength{\labelwidth}{\tmplength}
}
\begin{flushleft}
\item[\textbf{Declaração}\hfill]
\begin{ttfamily}
public const CharAllZeroes = {\^{}}A;\end{ttfamily}


\end{flushleft}
\par
\item[\textbf{Descrição}]
A constante \textbf{\begin{ttfamily}CharAllZeroes\end{ttfamily}} avisa para iniciar com {\#}0 todos os campos

\end{list}
\paragraph*{CharProviderFlag}\hspace*{\fill}

\begin{list}{}{
\settowidth{\tmplength}{\textbf{Declaração}}
\setlength{\itemindent}{0cm}
\setlength{\listparindent}{0cm}
\setlength{\leftmargin}{\evensidemargin}
\addtolength{\leftmargin}{\tmplength}
\settowidth{\labelsep}{X}
\addtolength{\leftmargin}{\labelsep}
\setlength{\labelwidth}{\tmplength}
}
\begin{flushleft}
\item[\textbf{Declaração}\hfill]
\begin{ttfamily}
public const CharProviderFlag  = {\^{}}P;\end{ttfamily}


\end{flushleft}
\par
\item[\textbf{Descrição}]
O caractere de controle \textbf{\begin{ttfamily}CharProviderFlag\end{ttfamily}} é usado pelo método \textbf{TUiDmxScroller{\_}sql.CreateTables} para indicar que o caractere seguinte tem um sinalizador usado para criar tabelas no banco de dados.

\begin{itemize}
\item \textbf{SINALIZADORES} \begin{itemize}
\item 0 = \textbf{pfInUpdate} : As alterações no campo devem ser propagadas para o banco de dados..
\item 1 = \textbf{pfInWhere} : O campo deve ser usado na cláusula WHERE de uma instrução de atualização no caso de upWhereChanged.
\item 2 = \textbf{pfInKey} : Campo é um campo chave e usado na cláusula WHERE de uma instrução de atualização.
\item 3 = \textbf{pfHidden} : O valor deste campo deve ser atualizado após a inserção.
\item 4 = \textbf{pfRefreshOnInsert} : O valor deste campo deve ser atualizado após a inserção.
\item 5 = \textbf{pfRefreshOnUpdate} : O valor deste campo deve ser atualizado após a atualização.
\item 6 = \textbf{pfInKeyPrimary} : Campo é um campo chave primária e usado na cláusula WHERE de uma instrução de atualização.
\item 7 = \textbf{pfInAutoIncrement} : Campo é um campo autoincremental e usado em uma instrução de atualização.
\end{itemize}
\item \textbf{NOTAS} \begin{itemize}
\item O campos com access = {\^{}}S automaticmanente o atributo MIProviderFlag terá [pfHidden]
\item O valor defaults de MiProviderFlags := [pfInUpdate,pfInWhere];
\item \textbf{Campos de chave primária} \begin{itemize}
\item Ao atualizar registros, TSQLQuery precisa saber quais campos compõem a chave primária que pode ser usada para atualizar o registro e quais campos devem ser atualizados: com base nessas informações, ele constrói um comando SQL UPDATE, INSERT ou DELETE.
\item A construção da instrução SQL é controlada pela propriedade UsePrimaryKeyAsKey e pelas propriedades ProviderFlags .
\item A propriedade Providerflags é um conjunto de 3 sinalizadores: \begin{itemize}
\item pfInkey : O campo faz parte da chave primária
\item pfInWhere : O campo deve ser utilizado na cláusula WHERE das instruções SQL.
\item pfInUpdate : Atualizações ou inserções devem incluir este campo. Por padrão, ProviderFlags consiste apenas em pfInUpdate .
\end{itemize}
\item \textbf{REFERÊNCIA} \begin{itemize}
\item [Working{\_}With{\_}TSQLQuery](https://wiki.freepascal.org/Working{\_}With{\_}TSQLQuery)
\end{itemize}
\end{itemize}
\end{itemize}
\end{itemize}

\end{list}
\paragraph*{CharpfInUpdate}\hspace*{\fill}

\begin{list}{}{
\settowidth{\tmplength}{\textbf{Declaração}}
\setlength{\itemindent}{0cm}
\setlength{\listparindent}{0cm}
\setlength{\leftmargin}{\evensidemargin}
\addtolength{\leftmargin}{\tmplength}
\settowidth{\labelsep}{X}
\addtolength{\leftmargin}{\labelsep}
\setlength{\labelwidth}{\tmplength}
}
\begin{flushleft}
\item[\textbf{Declaração}\hfill]
\begin{ttfamily}
public const CharpfInUpdate = {\^{}}P'0';\end{ttfamily}


\end{flushleft}
\par
\item[\textbf{Descrição}]
As alterações no campo devem ser propagadas para o banco de dados..

\end{list}
\paragraph*{CharpfInWhere}\hspace*{\fill}

\begin{list}{}{
\settowidth{\tmplength}{\textbf{Declaração}}
\setlength{\itemindent}{0cm}
\setlength{\listparindent}{0cm}
\setlength{\leftmargin}{\evensidemargin}
\addtolength{\leftmargin}{\tmplength}
\settowidth{\labelsep}{X}
\addtolength{\leftmargin}{\labelsep}
\setlength{\labelwidth}{\tmplength}
}
\begin{flushleft}
\item[\textbf{Declaração}\hfill]
\begin{ttfamily}
public const CharpfInWhere  = {\^{}}P'1';\end{ttfamily}


\end{flushleft}
\par
\item[\textbf{Descrição}]
O campo deve ser usado na cláusula WHERE de uma instrução de atualização no caso de upWhereChanged.

\end{list}
\paragraph*{CharPfInKey}\hspace*{\fill}

\begin{list}{}{
\settowidth{\tmplength}{\textbf{Declaração}}
\setlength{\itemindent}{0cm}
\setlength{\listparindent}{0cm}
\setlength{\leftmargin}{\evensidemargin}
\addtolength{\leftmargin}{\tmplength}
\settowidth{\labelsep}{X}
\addtolength{\leftmargin}{\labelsep}
\setlength{\labelwidth}{\tmplength}
}
\begin{flushleft}
\item[\textbf{Declaração}\hfill]
\begin{ttfamily}
public const CharPfInKey    = {\^{}}P'2';\end{ttfamily}


\end{flushleft}
\par
\item[\textbf{Descrição}]
Campo é um campo chave e usado na cláusula WHERE de uma instrução de atualização.

\end{list}
\paragraph*{CharPfHidden}\hspace*{\fill}

\begin{list}{}{
\settowidth{\tmplength}{\textbf{Declaração}}
\setlength{\itemindent}{0cm}
\setlength{\listparindent}{0cm}
\setlength{\leftmargin}{\evensidemargin}
\addtolength{\leftmargin}{\tmplength}
\settowidth{\labelsep}{X}
\addtolength{\leftmargin}{\labelsep}
\setlength{\labelwidth}{\tmplength}
}
\begin{flushleft}
\item[\textbf{Declaração}\hfill]
\begin{ttfamily}
public const CharPfHidden   = {\^{}}P'3';\end{ttfamily}


\end{flushleft}
\par
\item[\textbf{Descrição}]
O valor deste campo deve ser atualizado após a inserção.

\end{list}
\paragraph*{CharPfRefreshOnInsert}\hspace*{\fill}

\begin{list}{}{
\settowidth{\tmplength}{\textbf{Declaração}}
\setlength{\itemindent}{0cm}
\setlength{\listparindent}{0cm}
\setlength{\leftmargin}{\evensidemargin}
\addtolength{\leftmargin}{\tmplength}
\settowidth{\labelsep}{X}
\addtolength{\leftmargin}{\labelsep}
\setlength{\labelwidth}{\tmplength}
}
\begin{flushleft}
\item[\textbf{Declaração}\hfill]
\begin{ttfamily}
public const CharPfRefreshOnInsert = {\^{}}P'4';\end{ttfamily}


\end{flushleft}
\par
\item[\textbf{Descrição}]
O valor deste campo deve ser atualizado após a inserção.

\end{list}
\paragraph*{CharPfRefreshOnUpdate}\hspace*{\fill}

\begin{list}{}{
\settowidth{\tmplength}{\textbf{Declaração}}
\setlength{\itemindent}{0cm}
\setlength{\listparindent}{0cm}
\setlength{\leftmargin}{\evensidemargin}
\addtolength{\leftmargin}{\tmplength}
\settowidth{\labelsep}{X}
\addtolength{\leftmargin}{\labelsep}
\setlength{\labelwidth}{\tmplength}
}
\begin{flushleft}
\item[\textbf{Declaração}\hfill]
\begin{ttfamily}
public const CharPfRefreshOnUpdate = {\^{}}P'5';\end{ttfamily}


\end{flushleft}
\par
\item[\textbf{Descrição}]
O valor deste campo deve ser atualizado após a atualização.

\end{list}
\paragraph*{CharPfInKeyPrimary}\hspace*{\fill}

\begin{list}{}{
\settowidth{\tmplength}{\textbf{Declaração}}
\setlength{\itemindent}{0cm}
\setlength{\listparindent}{0cm}
\setlength{\leftmargin}{\evensidemargin}
\addtolength{\leftmargin}{\tmplength}
\settowidth{\labelsep}{X}
\addtolength{\leftmargin}{\labelsep}
\setlength{\labelwidth}{\tmplength}
}
\begin{flushleft}
\item[\textbf{Declaração}\hfill]
\begin{ttfamily}
public const CharPfInKeyPrimary    = {\^{}}P'6';\end{ttfamily}


\end{flushleft}
\par
\item[\textbf{Descrição}]
Campo é um campo chave primária e usado na cláusula WHERE de uma instrução de atualização.

\end{list}
\paragraph*{CharPfInKeyPrimaryAutoIncrement}\hspace*{\fill}

\begin{list}{}{
\settowidth{\tmplength}{\textbf{Declaração}}
\setlength{\itemindent}{0cm}
\setlength{\listparindent}{0cm}
\setlength{\leftmargin}{\evensidemargin}
\addtolength{\leftmargin}{\tmplength}
\settowidth{\labelsep}{X}
\addtolength{\leftmargin}{\labelsep}
\setlength{\labelwidth}{\tmplength}
}
\begin{flushleft}
\item[\textbf{Declaração}\hfill]
\begin{ttfamily}
public const CharPfInKeyPrimaryAutoIncrement = {\^{}}P'7';\end{ttfamily}


\end{flushleft}
\par
\item[\textbf{Descrição}]
Campo é um campo autoincremental e usado em uma instrução de atualização.

\end{list}
\paragraph*{CharForeignKey}\hspace*{\fill}

\begin{list}{}{
\settowidth{\tmplength}{\textbf{Declaração}}
\setlength{\itemindent}{0cm}
\setlength{\listparindent}{0cm}
\setlength{\leftmargin}{\evensidemargin}
\addtolength{\leftmargin}{\tmplength}
\settowidth{\labelsep}{X}
\addtolength{\leftmargin}{\labelsep}
\setlength{\labelwidth}{\tmplength}
}
\begin{flushleft}
\item[\textbf{Declaração}\hfill]
\begin{ttfamily}
public const CharForeignKey  = {\^{}}F ;\end{ttfamily}


\end{flushleft}
\par
\item[\textbf{Descrição}]
Produz um erro indicando que a exclusão ou atualização criaria uma violação de restrição de chave estrangeira. Se a restrição for adiada, esse erro será produzido no momento da verificação da restrição se ainda existirem linhas de referência. Esta é a ação padrão.

\end{list}
\paragraph*{CharFk{\_}No{\_}Action}\hspace*{\fill}

\begin{list}{}{
\settowidth{\tmplength}{\textbf{Declaração}}
\setlength{\itemindent}{0cm}
\setlength{\listparindent}{0cm}
\setlength{\leftmargin}{\evensidemargin}
\addtolength{\leftmargin}{\tmplength}
\settowidth{\labelsep}{X}
\addtolength{\leftmargin}{\labelsep}
\setlength{\labelwidth}{\tmplength}
}
\begin{flushleft}
\item[\textbf{Declaração}\hfill]
\begin{ttfamily}
public const CharFk{\_}No{\_}Action = {\^{}}F'0' ;\end{ttfamily}


\end{flushleft}
\par
\item[\textbf{Descrição}]
Produz um erro indicando que a exclusão ou atualização criaria uma violação de restrição de chave estrangeira. Isso é o mesmo que, \textbf{NO ACTION} exceto que o cheque não é adiável.

\end{list}
\paragraph*{CharFk{\_}Restrict}\hspace*{\fill}

\begin{list}{}{
\settowidth{\tmplength}{\textbf{Declaração}}
\setlength{\itemindent}{0cm}
\setlength{\listparindent}{0cm}
\setlength{\leftmargin}{\evensidemargin}
\addtolength{\leftmargin}{\tmplength}
\settowidth{\labelsep}{X}
\addtolength{\leftmargin}{\labelsep}
\setlength{\labelwidth}{\tmplength}
}
\begin{flushleft}
\item[\textbf{Declaração}\hfill]
\begin{ttfamily}
public const CharFk{\_}Restrict = {\^{}}F'1' ;\end{ttfamily}


\end{flushleft}
\par
\item[\textbf{Descrição}]
Exclua todas as linhas que fazem referência à linha excluída ou atualize os valores das colunas de referência para os novos valores das colunas referenciadas, respectivamente.

\end{list}
\paragraph*{CharFk{\_}Cascade}\hspace*{\fill}

\begin{list}{}{
\settowidth{\tmplength}{\textbf{Declaração}}
\setlength{\itemindent}{0cm}
\setlength{\listparindent}{0cm}
\setlength{\leftmargin}{\evensidemargin}
\addtolength{\leftmargin}{\tmplength}
\settowidth{\labelsep}{X}
\addtolength{\leftmargin}{\labelsep}
\setlength{\labelwidth}{\tmplength}
}
\begin{flushleft}
\item[\textbf{Declaração}\hfill]
\begin{ttfamily}
public const CharFk{\_}Cascade  = {\^{}}F'2' ;\end{ttfamily}


\end{flushleft}
\par
\item[\textbf{Descrição}]
Defina a(s) coluna(s) de referência como nula.

\end{list}
\paragraph*{CharFk{\_}Set{\_}Null}\hspace*{\fill}

\begin{list}{}{
\settowidth{\tmplength}{\textbf{Declaração}}
\setlength{\itemindent}{0cm}
\setlength{\listparindent}{0cm}
\setlength{\leftmargin}{\evensidemargin}
\addtolength{\leftmargin}{\tmplength}
\settowidth{\labelsep}{X}
\addtolength{\leftmargin}{\labelsep}
\setlength{\labelwidth}{\tmplength}
}
\begin{flushleft}
\item[\textbf{Declaração}\hfill]
\begin{ttfamily}
public const CharFk{\_}Set{\_}Null  = {\^{}}F'3' ;\end{ttfamily}


\end{flushleft}
\par
\item[\textbf{Descrição}]
A contante \textbf{\begin{ttfamily}CharFk{\_}Set{\_}Null\end{ttfamily}} defina a(s) coluna(s) de referência para seus valores padrão. (Deve haver uma linha na tabela referenciada que corresponda aos valores padrão, se eles não forem nulos, ou a operação falhará.

\end{list}
\paragraph*{CharFk{\_}Set{\_}Default}\hspace*{\fill}

\begin{list}{}{
\settowidth{\tmplength}{\textbf{Declaração}}
\setlength{\itemindent}{0cm}
\setlength{\listparindent}{0cm}
\setlength{\leftmargin}{\evensidemargin}
\addtolength{\leftmargin}{\tmplength}
\settowidth{\labelsep}{X}
\addtolength{\leftmargin}{\labelsep}
\setlength{\labelwidth}{\tmplength}
}
\begin{flushleft}
\item[\textbf{Declaração}\hfill]
\begin{ttfamily}
public const CharFk{\_}Set{\_}Default  = {\^{}}F'4' ;\end{ttfamily}


\end{flushleft}
\end{list}
\paragraph*{CharHint}\hspace*{\fill}

\begin{list}{}{
\settowidth{\tmplength}{\textbf{Declaração}}
\setlength{\itemindent}{0cm}
\setlength{\listparindent}{0cm}
\setlength{\leftmargin}{\evensidemargin}
\addtolength{\leftmargin}{\tmplength}
\settowidth{\labelsep}{X}
\addtolength{\leftmargin}{\labelsep}
\setlength{\labelwidth}{\tmplength}
}
\begin{flushleft}
\item[\textbf{Declaração}\hfill]
\begin{ttfamily}
public const CharHint       = '{\^{}}';\end{ttfamily}


\end{flushleft}
\par
\item[\textbf{Descrição}]
O A constante \textbf{\begin{ttfamily}CharHint\end{ttfamily}} é usado para documentar o campo e indica que todo o texto até o próximo caractere de controle será o conteúdo do campo HelpCtx{\_}Hint. \begin{itemize}
\item \textbf{EXEMPLO} ```pascal

Resourcestring \begin{ttfamily}tmp{\_}Alunos{\_}Idade\end{ttfamily}(\ref{uDmxScroller_Form_Lcl_test-tmp_Alunos_Idade}) = '{\textbackslash}BB'+\begin{ttfamily}ChFN\end{ttfamily}(\ref{mi.rtl.Consts.TConsts-ChFN})+'idade'+\begin{ttfamily}CharUpperlimit\end{ttfamily}(\ref{mi.rtl.Consts.TConsts-CharUpperlimit})+{\#}64+ \begin{ttfamily}CharHint\end{ttfamily}+'A idade do aluno. Valores válidos 1 a 64'+ \begin{ttfamily}CharHintPorque\end{ttfamily}(\ref{mi.rtl.Consts.TConsts-CharHintPorque})+'Este campo é necessário para que se agrupe o alunos baseado em sua faixa etária'+ \begin{ttfamily}CharHintOnde\end{ttfamily}(\ref{mi.rtl.Consts.TConsts-CharHintOnde})+'Ele será usado pelo coordenador ao classificar a turma';

\begin{ttfamily}tmp{\_}Alunos{\_}Matricula\end{ttfamily}(\ref{uDmxScroller_Form_Lcl_test-tmp_Alunos_Matricula}) = {\textbackslash}IIII'+\begin{ttfamily}ChFN\end{ttfamily}(\ref{mi.rtl.Consts.TConsts-ChFN})+'matricula'+\begin{ttfamily}CharHint\end{ttfamily}+'A matricula do aluno é um campo sequencial e calculado ao incluir o registro';

\begin{ttfamily}tmp{\_}Alunos\end{ttfamily}(\ref{uDmxScroller_Form_Lcl_test-tmp_Alunos}) = '~ Idade:~'+\begin{ttfamily}tmp{\_}Alunos{\_}Idade\end{ttfamily}(\ref{uDmxScroller_Form_Lcl_test-tmp_Alunos_Idade})+\begin{ttfamily}lf\end{ttfamily}(\ref{mi.rtl.Consts.TConsts-LF})+ '~ Matricula:~'+\begin{ttfamily}tmp{\_}Alunos{\_}Matricula\end{ttfamily}(\ref{uDmxScroller_Form_Lcl_test-tmp_Alunos_Matricula})+\begin{ttfamily}lf\end{ttfamily}(\ref{mi.rtl.Consts.TConsts-LF});
\end{itemize}

\end{list}
\paragraph*{ChH}\hspace*{\fill}

\begin{list}{}{
\settowidth{\tmplength}{\textbf{Declaração}}
\setlength{\itemindent}{0cm}
\setlength{\listparindent}{0cm}
\setlength{\leftmargin}{\evensidemargin}
\addtolength{\leftmargin}{\tmplength}
\settowidth{\labelsep}{X}
\addtolength{\leftmargin}{\labelsep}
\setlength{\labelwidth}{\tmplength}
}
\begin{flushleft}
\item[\textbf{Declaração}\hfill]
\begin{ttfamily}
public const ChH = CharHint;\end{ttfamily}


\end{flushleft}
\end{list}
\paragraph*{CharHintPorque}\hspace*{\fill}

\begin{list}{}{
\settowidth{\tmplength}{\textbf{Declaração}}
\setlength{\itemindent}{0cm}
\setlength{\listparindent}{0cm}
\setlength{\leftmargin}{\evensidemargin}
\addtolength{\leftmargin}{\tmplength}
\settowidth{\labelsep}{X}
\addtolength{\leftmargin}{\labelsep}
\setlength{\labelwidth}{\tmplength}
}
\begin{flushleft}
\item[\textbf{Declaração}\hfill]
\begin{ttfamily}
public const CharHintPorque = '{\^{}}0';\end{ttfamily}


\end{flushleft}
\par
\item[\textbf{Descrição}]
A contante \textbf{\begin{ttfamily}CharHintPorque\end{ttfamily}} informa que todo texto até o próximo delimitador contém informações para o campo HelpCtx{\_}Porque

\end{list}
\paragraph*{CharHintOnde}\hspace*{\fill}

\begin{list}{}{
\settowidth{\tmplength}{\textbf{Declaração}}
\setlength{\itemindent}{0cm}
\setlength{\listparindent}{0cm}
\setlength{\leftmargin}{\evensidemargin}
\addtolength{\leftmargin}{\tmplength}
\settowidth{\labelsep}{X}
\addtolength{\leftmargin}{\labelsep}
\setlength{\labelwidth}{\tmplength}
}
\begin{flushleft}
\item[\textbf{Declaração}\hfill]
\begin{ttfamily}
public const CharHintOnde   = '{\^{}}1';\end{ttfamily}


\end{flushleft}
\par
\item[\textbf{Descrição}]
A contante \textbf{\begin{ttfamily}CharHintOnde\end{ttfamily}} informa que todo texto até o próximo delimitador contém informações para o campo HelpCtx{\_}Onde

\end{list}
\paragraph*{Delimiters}\hspace*{\fill}

\begin{list}{}{
\settowidth{\tmplength}{\textbf{Declaração}}
\setlength{\itemindent}{0cm}
\setlength{\listparindent}{0cm}
\setlength{\leftmargin}{\evensidemargin}
\addtolength{\leftmargin}{\tmplength}
\settowidth{\labelsep}{X}
\addtolength{\leftmargin}{\labelsep}
\setlength{\labelwidth}{\tmplength}
}
\begin{flushleft}
\item[\textbf{Declaração}\hfill]
\begin{ttfamily}
public const Delimiters : AnsiCharSet = [CharDelimiter{\_}0,
                                            CharDelimiter{\_}1,
                                            CharDelimiter{\_}2,
                                            CharDelimiter{\_}3,
                                            CharExecAction,
                                            CharFieldName,
                                            CharUpperlimit,
                                            CharAccHidden,
                                            CharAccSkip,
                                            CharAccReadOnly,
                                            CharAllZeroes,
                                            CharProviderFlag,
                                            CharForeignKey,
                                            CharHint,
                                            fldAPPEND,
                                            fldSItems,
                                            CharListComboBox
                                            ];\end{ttfamily}


\end{flushleft}
\end{list}
\paragraph*{SinalDireita}\hspace*{\fill}

\begin{list}{}{
\settowidth{\tmplength}{\textbf{Declaração}}
\setlength{\itemindent}{0cm}
\setlength{\listparindent}{0cm}
\setlength{\leftmargin}{\evensidemargin}
\addtolength{\leftmargin}{\tmplength}
\settowidth{\labelsep}{X}
\addtolength{\leftmargin}{\labelsep}
\setlength{\labelwidth}{\tmplength}
}
\begin{flushleft}
\item[\textbf{Declaração}\hfill]
\begin{ttfamily}
public const SinalDireita   : Boolean = False;\end{ttfamily}


\end{flushleft}
\end{list}
\paragraph*{SinalDeMaisAtivo}\hspace*{\fill}

\begin{list}{}{
\settowidth{\tmplength}{\textbf{Declaração}}
\setlength{\itemindent}{0cm}
\setlength{\listparindent}{0cm}
\setlength{\leftmargin}{\evensidemargin}
\addtolength{\leftmargin}{\tmplength}
\settowidth{\labelsep}{X}
\addtolength{\leftmargin}{\labelsep}
\setlength{\labelwidth}{\tmplength}
}
\begin{flushleft}
\item[\textbf{Declaração}\hfill]
\begin{ttfamily}
public const SinalDeMaisAtivo : Boolean = False;\end{ttfamily}


\end{flushleft}
\par
\item[\textbf{Descrição}]
\begin{itemize}
\item Mostra o sinal de + a direita dos campos numéricos
\end{itemize}

\end{list}
\paragraph*{MaskIsNumber}\hspace*{\fill}

\begin{list}{}{
\settowidth{\tmplength}{\textbf{Declaração}}
\setlength{\itemindent}{0cm}
\setlength{\listparindent}{0cm}
\setlength{\leftmargin}{\evensidemargin}
\addtolength{\leftmargin}{\tmplength}
\settowidth{\labelsep}{X}
\addtolength{\leftmargin}{\labelsep}
\setlength{\labelwidth}{\tmplength}
}
\begin{flushleft}
\item[\textbf{Declaração}\hfill]
\begin{ttfamily}
public const MaskIsNumber : TAnsiCharSet = [];\end{ttfamily}


\end{flushleft}
\end{list}
\paragraph*{Delta{\_}Locate}\hspace*{\fill}

\begin{list}{}{
\settowidth{\tmplength}{\textbf{Declaração}}
\setlength{\itemindent}{0cm}
\setlength{\listparindent}{0cm}
\setlength{\leftmargin}{\evensidemargin}
\addtolength{\leftmargin}{\tmplength}
\settowidth{\labelsep}{X}
\addtolength{\leftmargin}{\labelsep}
\setlength{\labelwidth}{\tmplength}
}
\begin{flushleft}
\item[\textbf{Declaração}\hfill]
\begin{ttfamily}
public const Delta{\_}Locate   : Longint  = 100;\end{ttfamily}


\end{flushleft}
\par
\item[\textbf{Descrição}]
\textbf{}\textbf{}\textbf{}\textbf{}\textbf{}\textbf{}\textbf{}\textbf{}\textbf{}\textbf{}\textbf{}\textbf{}\textbf{}\textbf{}\textbf{}\textbf{}\textbf{}\textbf{}

\end{list}
\paragraph*{ConvertIdioma{\_}Nil}\hspace*{\fill}

\begin{list}{}{
\settowidth{\tmplength}{\textbf{Declaração}}
\setlength{\itemindent}{0cm}
\setlength{\listparindent}{0cm}
\setlength{\leftmargin}{\evensidemargin}
\addtolength{\leftmargin}{\tmplength}
\settowidth{\labelsep}{X}
\addtolength{\leftmargin}{\labelsep}
\setlength{\labelwidth}{\tmplength}
}
\begin{flushleft}
\item[\textbf{Declaração}\hfill]
\begin{ttfamily}
public const ConvertIdioma{\_}Nil : TConvertIdioma = nil;\end{ttfamily}


\end{flushleft}
\end{list}
\paragraph*{Html{\_}Nivel1}\hspace*{\fill}

\begin{list}{}{
\settowidth{\tmplength}{\textbf{Declaração}}
\setlength{\itemindent}{0cm}
\setlength{\listparindent}{0cm}
\setlength{\leftmargin}{\evensidemargin}
\addtolength{\leftmargin}{\tmplength}
\settowidth{\labelsep}{X}
\addtolength{\leftmargin}{\labelsep}
\setlength{\labelwidth}{\tmplength}
}
\begin{flushleft}
\item[\textbf{Declaração}\hfill]
\begin{ttfamily}
public const Html{\_}Nivel1 = '{$<$}font size="6"{$>$}{\&}{\#}9642;{$<$}/font{$>$}';\end{ttfamily}


\end{flushleft}
\end{list}
\paragraph*{Html{\_}Nivel2}\hspace*{\fill}

\begin{list}{}{
\settowidth{\tmplength}{\textbf{Declaração}}
\setlength{\itemindent}{0cm}
\setlength{\listparindent}{0cm}
\setlength{\leftmargin}{\evensidemargin}
\addtolength{\leftmargin}{\tmplength}
\settowidth{\labelsep}{X}
\addtolength{\leftmargin}{\labelsep}
\setlength{\labelwidth}{\tmplength}
}
\begin{flushleft}
\item[\textbf{Declaração}\hfill]
\begin{ttfamily}
public const Html{\_}Nivel2 = '{$<$}font size="5"{$>$}{\&}{\#}9642;{$<$}/font{$>$}';\end{ttfamily}


\end{flushleft}
\end{list}
\paragraph*{Html{\_}Nivel3}\hspace*{\fill}

\begin{list}{}{
\settowidth{\tmplength}{\textbf{Declaração}}
\setlength{\itemindent}{0cm}
\setlength{\listparindent}{0cm}
\setlength{\leftmargin}{\evensidemargin}
\addtolength{\leftmargin}{\tmplength}
\settowidth{\labelsep}{X}
\addtolength{\leftmargin}{\labelsep}
\setlength{\labelwidth}{\tmplength}
}
\begin{flushleft}
\item[\textbf{Declaração}\hfill]
\begin{ttfamily}
public const Html{\_}Nivel3 = '{$<$}font size="4"{$>$}{\&}{\#}9642;{$<$}/font{$>$}';\end{ttfamily}


\end{flushleft}
\end{list}
\paragraph*{Html{\_}Nivel4}\hspace*{\fill}

\begin{list}{}{
\settowidth{\tmplength}{\textbf{Declaração}}
\setlength{\itemindent}{0cm}
\setlength{\listparindent}{0cm}
\setlength{\leftmargin}{\evensidemargin}
\addtolength{\leftmargin}{\tmplength}
\settowidth{\labelsep}{X}
\addtolength{\leftmargin}{\labelsep}
\setlength{\labelwidth}{\tmplength}
}
\begin{flushleft}
\item[\textbf{Declaração}\hfill]
\begin{ttfamily}
public const Html{\_}Nivel4 = '{$<$}font size="2"{$>$}{\&}{\#}9642;{$<$}/font{$>$}';\end{ttfamily}


\end{flushleft}
\end{list}
\paragraph*{Char{\_}Nivel1}\hspace*{\fill}

\begin{list}{}{
\settowidth{\tmplength}{\textbf{Declaração}}
\setlength{\itemindent}{0cm}
\setlength{\listparindent}{0cm}
\setlength{\leftmargin}{\evensidemargin}
\addtolength{\leftmargin}{\tmplength}
\settowidth{\labelsep}{X}
\addtolength{\leftmargin}{\labelsep}
\setlength{\labelwidth}{\tmplength}
}
\begin{flushleft}
\item[\textbf{Declaração}\hfill]
\begin{ttfamily}
public const Char{\_}Nivel1 = Ansichar(254);\end{ttfamily}


\end{flushleft}
\end{list}
\paragraph*{Char{\_}Nivel2}\hspace*{\fill}

\begin{list}{}{
\settowidth{\tmplength}{\textbf{Declaração}}
\setlength{\itemindent}{0cm}
\setlength{\listparindent}{0cm}
\setlength{\leftmargin}{\evensidemargin}
\addtolength{\leftmargin}{\tmplength}
\settowidth{\labelsep}{X}
\addtolength{\leftmargin}{\labelsep}
\setlength{\labelwidth}{\tmplength}
}
\begin{flushleft}
\item[\textbf{Declaração}\hfill]
\begin{ttfamily}
public const Char{\_}Nivel2 = Ansichar(207);\end{ttfamily}


\end{flushleft}
\end{list}
\paragraph*{Char{\_}Nivel3}\hspace*{\fill}

\begin{list}{}{
\settowidth{\tmplength}{\textbf{Declaração}}
\setlength{\itemindent}{0cm}
\setlength{\listparindent}{0cm}
\setlength{\leftmargin}{\evensidemargin}
\addtolength{\leftmargin}{\tmplength}
\settowidth{\labelsep}{X}
\addtolength{\leftmargin}{\labelsep}
\setlength{\labelwidth}{\tmplength}
}
\begin{flushleft}
\item[\textbf{Declaração}\hfill]
\begin{ttfamily}
public const Char{\_}Nivel3 = Ansichar(248);\end{ttfamily}


\end{flushleft}
\end{list}
\paragraph*{Char{\_}Nivel4}\hspace*{\fill}

\begin{list}{}{
\settowidth{\tmplength}{\textbf{Declaração}}
\setlength{\itemindent}{0cm}
\setlength{\listparindent}{0cm}
\setlength{\leftmargin}{\evensidemargin}
\addtolength{\leftmargin}{\tmplength}
\settowidth{\labelsep}{X}
\addtolength{\leftmargin}{\labelsep}
\setlength{\labelwidth}{\tmplength}
}
\begin{flushleft}
\item[\textbf{Declaração}\hfill]
\begin{ttfamily}
public const Char{\_}Nivel4 = Ansichar(250);\end{ttfamily}


\end{flushleft}
\end{list}
\paragraph*{Array{\_}Of{\_}Char}\hspace*{\fill}

\begin{list}{}{
\settowidth{\tmplength}{\textbf{Declaração}}
\setlength{\itemindent}{0cm}
\setlength{\listparindent}{0cm}
\setlength{\leftmargin}{\evensidemargin}
\addtolength{\leftmargin}{\tmplength}
\settowidth{\labelsep}{X}
\addtolength{\leftmargin}{\labelsep}
\setlength{\labelwidth}{\tmplength}
}
\begin{flushleft}
\item[\textbf{Declaração}\hfill]
\begin{ttfamily}
public const Array{\_}Of{\_}Char : TArray{\_}Of{\_}Char =
       (
        (Asc{\_}Ingles :'a';Asc{\_}GUI :'á';Asc{\_}HTML :'{\&}aacute;'), 
        (Asc{\_}Ingles :'a';Asc{\_}GUI :'â';Asc{\_}HTML :'{\&}acirc;'),  
        (Asc{\_}Ingles :'a';Asc{\_}GUI :'à';Asc{\_}HTML :'{\&}agrave;'), 
        (Asc{\_}Ingles :'a';Asc{\_}GUI :'ã';Asc{\_}HTML :'{\&}atilde;'), 
        (Asc{\_}Ingles :'A';Asc{\_}GUI :'Á';Asc{\_}HTML :'{\&}Aacute;'), 
        (Asc{\_}Ingles :'A';Asc{\_}GUI :'À';Asc{\_}HTML :'{\&}Agrave;'), 
        (Asc{\_}Ingles :'A';Asc{\_}GUI :'Â';Asc{\_}HTML :'{\&}Acirc;'),  
        (Asc{\_}Ingles :'A';Asc{\_}GUI :'Ã';Asc{\_}HTML:'{\&}Atilde;'),  
        (Asc{\_}Ingles :'c';Asc{\_}GUI :'ç';Asc{\_}HTML :'{\&}ccedil;'), 
        (Asc{\_}Ingles :'C';Asc{\_}GUI :'Ç';Asc{\_}HTML :'{\&}Ccedil;'), 
        (Asc{\_}Ingles :'e';Asc{\_}GUI :'é';Asc{\_}HTML :'{\&}eacute;'), 
        (Asc{\_}Ingles :'e';Asc{\_}GUI :'ê';Asc{\_}HTML :'ê'),        
        (Asc{\_}Ingles :'E';Asc{\_}GUI :'É';Asc{\_}HTML :'{\&}Eacute;'), 
        (Asc{\_}Ingles :'E';Asc{\_}GUI :'Ê';Asc{\_}HTML :'{\&}Ecirc;'),  
        (Asc{\_}Ingles :'i';Asc{\_}GUI :'í';Asc{\_}HTML :'{\&}iacute;'), 
        (Asc{\_}Ingles :'I';Asc{\_}GUI :'Í';Asc{\_}HTML :'{\&}Iacute;'), 
        (Asc{\_}Ingles :'o';Asc{\_}GUI :'ó';Asc{\_}HTML :'ó'),        
        (Asc{\_}Ingles :'O';Asc{\_}GUI :'Ó';Asc{\_}HTML :'{\&}Oacute;'), 
        (Asc{\_}Ingles :'o';Asc{\_}GUI :'õ';Asc{\_}HTML :'õ'),        
        (Asc{\_}Ingles :'O';Asc{\_}GUI :'Õ';Asc{\_}HTML :'Õ'),        
        (Asc{\_}Ingles :'o';Asc{\_}GUI :'ô';Asc{\_}HTML :'ô'),        
        (Asc{\_}Ingles :'O';Asc{\_}GUI :'Ô';Asc{\_}HTML :'Ô'),        
        (Asc{\_}Ingles :'u';Asc{\_}GUI :'ú';Asc{\_}HTML :'ú'),        
        (Asc{\_}Ingles :'U';Asc{\_}GUI :'Ú';Asc{\_}HTML :'{\&}Uacute;'), 
        (Asc{\_}Ingles :'u';Asc{\_}GUI :'ü';Asc{\_}HTML :'{\&}{\#}252;'),   
        (Asc{\_}Ingles :'U';Asc{\_}GUI :'Ü';Asc{\_}HTML :'{\&}{\#}220;'),   
        (Asc{\_}Ingles :'o';Asc{\_}GUI :'º';Asc{\_}HTML     :'º')     
       );\end{ttfamily}


\end{flushleft}
\end{list}
\paragraph*{PortaDaImpressora}\hspace*{\fill}

\begin{list}{}{
\settowidth{\tmplength}{\textbf{Declaração}}
\setlength{\itemindent}{0cm}
\setlength{\listparindent}{0cm}
\setlength{\leftmargin}{\evensidemargin}
\addtolength{\leftmargin}{\tmplength}
\settowidth{\labelsep}{X}
\addtolength{\leftmargin}{\labelsep}
\setlength{\labelwidth}{\tmplength}
}
\begin{flushleft}
\item[\textbf{Declaração}\hfill]
\begin{ttfamily}
public const PortaDaImpressora : tString = 'prn';\end{ttfamily}


\end{flushleft}
\end{list}
\paragraph*{opcaoRedireciona}\hspace*{\fill}

\begin{list}{}{
\settowidth{\tmplength}{\textbf{Declaração}}
\setlength{\itemindent}{0cm}
\setlength{\listparindent}{0cm}
\setlength{\leftmargin}{\evensidemargin}
\addtolength{\leftmargin}{\tmplength}
\settowidth{\labelsep}{X}
\addtolength{\leftmargin}{\labelsep}
\setlength{\labelwidth}{\tmplength}
}
\begin{flushleft}
\item[\textbf{Declaração}\hfill]
\begin{ttfamily}
public const opcaoRedireciona : AnsiChar = 'I';\end{ttfamily}


\end{flushleft}
\end{list}
\paragraph*{RedirecionaImpressora}\hspace*{\fill}

\begin{list}{}{
\settowidth{\tmplength}{\textbf{Declaração}}
\setlength{\itemindent}{0cm}
\setlength{\listparindent}{0cm}
\setlength{\leftmargin}{\evensidemargin}
\addtolength{\leftmargin}{\tmplength}
\settowidth{\labelsep}{X}
\addtolength{\leftmargin}{\labelsep}
\setlength{\labelwidth}{\tmplength}
}
\begin{flushleft}
\item[\textbf{Declaração}\hfill]
\begin{ttfamily}
public const RedirecionaImpressora  : boolean = false;\end{ttfamily}


\end{flushleft}
\end{list}
\paragraph*{redirecionaImpNul}\hspace*{\fill}

\begin{list}{}{
\settowidth{\tmplength}{\textbf{Declaração}}
\setlength{\itemindent}{0cm}
\setlength{\listparindent}{0cm}
\setlength{\leftmargin}{\evensidemargin}
\addtolength{\leftmargin}{\tmplength}
\settowidth{\labelsep}{X}
\addtolength{\leftmargin}{\labelsep}
\setlength{\labelwidth}{\tmplength}
}
\begin{flushleft}
\item[\textbf{Declaração}\hfill]
\begin{ttfamily}
public const redirecionaImpNul      : Boolean = False;\end{ttfamily}


\end{flushleft}
\end{list}
\paragraph*{NomeRedireciona}\hspace*{\fill}

\begin{list}{}{
\settowidth{\tmplength}{\textbf{Declaração}}
\setlength{\itemindent}{0cm}
\setlength{\listparindent}{0cm}
\setlength{\leftmargin}{\evensidemargin}
\addtolength{\leftmargin}{\tmplength}
\settowidth{\labelsep}{X}
\addtolength{\leftmargin}{\labelsep}
\setlength{\labelwidth}{\tmplength}
}
\begin{flushleft}
\item[\textbf{Declaração}\hfill]
\begin{ttfamily}
public const NomeRedireciona        : PathStr = 'C:{\textbackslash}Maricarai.Lst';\end{ttfamily}


\end{flushleft}
\end{list}
\paragraph*{ApartirDeQuePagina}\hspace*{\fill}

\begin{list}{}{
\settowidth{\tmplength}{\textbf{Declaração}}
\setlength{\itemindent}{0cm}
\setlength{\listparindent}{0cm}
\setlength{\leftmargin}{\evensidemargin}
\addtolength{\leftmargin}{\tmplength}
\settowidth{\labelsep}{X}
\addtolength{\leftmargin}{\labelsep}
\setlength{\labelwidth}{\tmplength}
}
\begin{flushleft}
\item[\textbf{Declaração}\hfill]
\begin{ttfamily}
public const ApartirDeQuePagina     : Longint= 1;\end{ttfamily}


\end{flushleft}
\par
\item[\textbf{Descrição}]
Caso \begin{ttfamily}ApartirDeQuePagina\end{ttfamily} {$>$} 1 então redireciona para NUL todas as paginas dos relatórios ate que \begin{ttfamily}ContaPagina\end{ttfamily}(\ref{mi.rtl.Consts.TConsts-contaPagina}) seja = \begin{ttfamily}ApartirDeQuePagina\end{ttfamily}

\end{list}
\paragraph*{PaginaInicial}\hspace*{\fill}

\begin{list}{}{
\settowidth{\tmplength}{\textbf{Declaração}}
\setlength{\itemindent}{0cm}
\setlength{\listparindent}{0cm}
\setlength{\leftmargin}{\evensidemargin}
\addtolength{\leftmargin}{\tmplength}
\settowidth{\labelsep}{X}
\addtolength{\leftmargin}{\labelsep}
\setlength{\labelwidth}{\tmplength}
}
\begin{flushleft}
\item[\textbf{Declaração}\hfill]
\begin{ttfamily}
public const PaginaInicial: Longint= 1;\end{ttfamily}


\end{flushleft}
\par
\item[\textbf{Descrição}]
Pagina inicial na listagem

\end{list}
\paragraph*{contalinha}\hspace*{\fill}

\begin{list}{}{
\settowidth{\tmplength}{\textbf{Declaração}}
\setlength{\itemindent}{0cm}
\setlength{\listparindent}{0cm}
\setlength{\leftmargin}{\evensidemargin}
\addtolength{\leftmargin}{\tmplength}
\settowidth{\labelsep}{X}
\addtolength{\leftmargin}{\labelsep}
\setlength{\labelwidth}{\tmplength}
}
\begin{flushleft}
\item[\textbf{Declaração}\hfill]
\begin{ttfamily}
public const contalinha   : Longint = 0;\end{ttfamily}


\end{flushleft}
\end{list}
\paragraph*{contaPagina}\hspace*{\fill}

\begin{list}{}{
\settowidth{\tmplength}{\textbf{Declaração}}
\setlength{\itemindent}{0cm}
\setlength{\listparindent}{0cm}
\setlength{\leftmargin}{\evensidemargin}
\addtolength{\leftmargin}{\tmplength}
\settowidth{\labelsep}{X}
\addtolength{\leftmargin}{\labelsep}
\setlength{\labelwidth}{\tmplength}
}
\begin{flushleft}
\item[\textbf{Declaração}\hfill]
\begin{ttfamily}
public const contaPagina  : Longint= 1;\end{ttfamily}


\end{flushleft}
\end{list}
\paragraph*{CmNulo}\hspace*{\fill}

\begin{list}{}{
\settowidth{\tmplength}{\textbf{Declaração}}
\setlength{\itemindent}{0cm}
\setlength{\listparindent}{0cm}
\setlength{\leftmargin}{\evensidemargin}
\addtolength{\leftmargin}{\tmplength}
\settowidth{\labelsep}{X}
\addtolength{\leftmargin}{\labelsep}
\setlength{\labelwidth}{\tmplength}
}
\begin{flushleft}
\item[\textbf{Declaração}\hfill]
\begin{ttfamily}
public const CmNulo                =          100 ;\end{ttfamily}


\end{flushleft}
\end{list}
\paragraph*{CmDbNextRec}\hspace*{\fill}

\begin{list}{}{
\settowidth{\tmplength}{\textbf{Declaração}}
\setlength{\itemindent}{0cm}
\setlength{\listparindent}{0cm}
\setlength{\leftmargin}{\evensidemargin}
\addtolength{\leftmargin}{\tmplength}
\settowidth{\labelsep}{X}
\addtolength{\leftmargin}{\labelsep}
\setlength{\labelwidth}{\tmplength}
}
\begin{flushleft}
\item[\textbf{Declaração}\hfill]
\begin{ttfamily}
public const CmDbNextRec           = CmNulo + 01;\end{ttfamily}


\end{flushleft}
\end{list}
\paragraph*{CmDbPrevRec}\hspace*{\fill}

\begin{list}{}{
\settowidth{\tmplength}{\textbf{Declaração}}
\setlength{\itemindent}{0cm}
\setlength{\listparindent}{0cm}
\setlength{\leftmargin}{\evensidemargin}
\addtolength{\leftmargin}{\tmplength}
\settowidth{\labelsep}{X}
\addtolength{\leftmargin}{\labelsep}
\setlength{\labelwidth}{\tmplength}
}
\begin{flushleft}
\item[\textbf{Declaração}\hfill]
\begin{ttfamily}
public const CmDbPrevRec           = CmNulo + 02;\end{ttfamily}


\end{flushleft}
\end{list}
\paragraph*{CmDbNextRecValid}\hspace*{\fill}

\begin{list}{}{
\settowidth{\tmplength}{\textbf{Declaração}}
\setlength{\itemindent}{0cm}
\setlength{\listparindent}{0cm}
\setlength{\leftmargin}{\evensidemargin}
\addtolength{\leftmargin}{\tmplength}
\settowidth{\labelsep}{X}
\addtolength{\leftmargin}{\labelsep}
\setlength{\labelwidth}{\tmplength}
}
\begin{flushleft}
\item[\textbf{Declaração}\hfill]
\begin{ttfamily}
public const CmDbNextRecValid      = CmNulo + 03;\end{ttfamily}


\end{flushleft}
\end{list}
\paragraph*{CmDbPrevRecValid}\hspace*{\fill}

\begin{list}{}{
\settowidth{\tmplength}{\textbf{Declaração}}
\setlength{\itemindent}{0cm}
\setlength{\listparindent}{0cm}
\setlength{\leftmargin}{\evensidemargin}
\addtolength{\leftmargin}{\tmplength}
\settowidth{\labelsep}{X}
\addtolength{\leftmargin}{\labelsep}
\setlength{\labelwidth}{\tmplength}
}
\begin{flushleft}
\item[\textbf{Declaração}\hfill]
\begin{ttfamily}
public const CmDbPrevRecValid      = CmNulo + 04;\end{ttfamily}


\end{flushleft}
\end{list}
\paragraph*{CmDbFindRec}\hspace*{\fill}

\begin{list}{}{
\settowidth{\tmplength}{\textbf{Declaração}}
\setlength{\itemindent}{0cm}
\setlength{\listparindent}{0cm}
\setlength{\leftmargin}{\evensidemargin}
\addtolength{\leftmargin}{\tmplength}
\settowidth{\labelsep}{X}
\addtolength{\leftmargin}{\labelsep}
\setlength{\labelwidth}{\tmplength}
}
\begin{flushleft}
\item[\textbf{Declaração}\hfill]
\begin{ttfamily}
public const CmDbFindRec           = CmNulo + 05;\end{ttfamily}


\end{flushleft}
\end{list}
\paragraph*{CmDbSearchRec}\hspace*{\fill}

\begin{list}{}{
\settowidth{\tmplength}{\textbf{Declaração}}
\setlength{\itemindent}{0cm}
\setlength{\listparindent}{0cm}
\setlength{\leftmargin}{\evensidemargin}
\addtolength{\leftmargin}{\tmplength}
\settowidth{\labelsep}{X}
\addtolength{\leftmargin}{\labelsep}
\setlength{\labelwidth}{\tmplength}
}
\begin{flushleft}
\item[\textbf{Declaração}\hfill]
\begin{ttfamily}
public const CmDbSearchRec         = CmNulo + 06;\end{ttfamily}


\end{flushleft}
\end{list}
\paragraph*{CmDbGoEof}\hspace*{\fill}

\begin{list}{}{
\settowidth{\tmplength}{\textbf{Declaração}}
\setlength{\itemindent}{0cm}
\setlength{\listparindent}{0cm}
\setlength{\leftmargin}{\evensidemargin}
\addtolength{\leftmargin}{\tmplength}
\settowidth{\labelsep}{X}
\addtolength{\leftmargin}{\labelsep}
\setlength{\labelwidth}{\tmplength}
}
\begin{flushleft}
\item[\textbf{Declaração}\hfill]
\begin{ttfamily}
public const CmDbGoEof             = CmNulo + 07;\end{ttfamily}


\end{flushleft}
\end{list}
\paragraph*{CmDbGoBof}\hspace*{\fill}

\begin{list}{}{
\settowidth{\tmplength}{\textbf{Declaração}}
\setlength{\itemindent}{0cm}
\setlength{\listparindent}{0cm}
\setlength{\leftmargin}{\evensidemargin}
\addtolength{\leftmargin}{\tmplength}
\settowidth{\labelsep}{X}
\addtolength{\leftmargin}{\labelsep}
\setlength{\labelwidth}{\tmplength}
}
\begin{flushleft}
\item[\textbf{Declaração}\hfill]
\begin{ttfamily}
public const CmDbGoBof             = CmNulo + 08;\end{ttfamily}


\end{flushleft}
\end{list}
\paragraph*{CmDbLocaliza}\hspace*{\fill}

\begin{list}{}{
\settowidth{\tmplength}{\textbf{Declaração}}
\setlength{\itemindent}{0cm}
\setlength{\listparindent}{0cm}
\setlength{\leftmargin}{\evensidemargin}
\addtolength{\leftmargin}{\tmplength}
\settowidth{\labelsep}{X}
\addtolength{\leftmargin}{\labelsep}
\setlength{\labelwidth}{\tmplength}
}
\begin{flushleft}
\item[\textbf{Declaração}\hfill]
\begin{ttfamily}
public const CmDbLocaliza          = CmNulo + 09;\end{ttfamily}


\end{flushleft}
\end{list}
\paragraph*{CmNewRecord}\hspace*{\fill}

\begin{list}{}{
\settowidth{\tmplength}{\textbf{Declaração}}
\setlength{\itemindent}{0cm}
\setlength{\listparindent}{0cm}
\setlength{\leftmargin}{\evensidemargin}
\addtolength{\leftmargin}{\tmplength}
\settowidth{\labelsep}{X}
\addtolength{\leftmargin}{\labelsep}
\setlength{\labelwidth}{\tmplength}
}
\begin{flushleft}
\item[\textbf{Declaração}\hfill]
\begin{ttfamily}
public const CmNewRecord           = CmNulo + 10;\end{ttfamily}


\end{flushleft}
\end{list}
\paragraph*{CmZeroizeRecord}\hspace*{\fill}

\begin{list}{}{
\settowidth{\tmplength}{\textbf{Declaração}}
\setlength{\itemindent}{0cm}
\setlength{\listparindent}{0cm}
\setlength{\leftmargin}{\evensidemargin}
\addtolength{\leftmargin}{\tmplength}
\settowidth{\labelsep}{X}
\addtolength{\leftmargin}{\labelsep}
\setlength{\labelwidth}{\tmplength}
}
\begin{flushleft}
\item[\textbf{Declaração}\hfill]
\begin{ttfamily}
public const CmZeroizeRecord       = CmNulo + 11;\end{ttfamily}


\end{flushleft}
\end{list}
\paragraph*{CmEvaluateRecord}\hspace*{\fill}

\begin{list}{}{
\settowidth{\tmplength}{\textbf{Declaração}}
\setlength{\itemindent}{0cm}
\setlength{\listparindent}{0cm}
\setlength{\leftmargin}{\evensidemargin}
\addtolength{\leftmargin}{\tmplength}
\settowidth{\labelsep}{X}
\addtolength{\leftmargin}{\labelsep}
\setlength{\labelwidth}{\tmplength}
}
\begin{flushleft}
\item[\textbf{Declaração}\hfill]
\begin{ttfamily}
public const CmEvaluateRecord      = CmNulo + 12;\end{ttfamily}


\end{flushleft}
\end{list}
\paragraph*{CmEditDlg}\hspace*{\fill}

\begin{list}{}{
\settowidth{\tmplength}{\textbf{Declaração}}
\setlength{\itemindent}{0cm}
\setlength{\listparindent}{0cm}
\setlength{\leftmargin}{\evensidemargin}
\addtolength{\leftmargin}{\tmplength}
\settowidth{\labelsep}{X}
\addtolength{\leftmargin}{\labelsep}
\setlength{\labelwidth}{\tmplength}
}
\begin{flushleft}
\item[\textbf{Declaração}\hfill]
\begin{ttfamily}
public const CmEditDlg             = CmNulo + 13;\end{ttfamily}


\end{flushleft}
\end{list}
\paragraph*{cmMyOK}\hspace*{\fill}

\begin{list}{}{
\settowidth{\tmplength}{\textbf{Declaração}}
\setlength{\itemindent}{0cm}
\setlength{\listparindent}{0cm}
\setlength{\leftmargin}{\evensidemargin}
\addtolength{\leftmargin}{\tmplength}
\settowidth{\labelsep}{X}
\addtolength{\leftmargin}{\labelsep}
\setlength{\labelwidth}{\tmplength}
}
\begin{flushleft}
\item[\textbf{Declaração}\hfill]
\begin{ttfamily}
public const cmMyOK                = CmNulo + 14;\end{ttfamily}


\end{flushleft}
\end{list}
\paragraph*{cmMyCancel}\hspace*{\fill}

\begin{list}{}{
\settowidth{\tmplength}{\textbf{Declaração}}
\setlength{\itemindent}{0cm}
\setlength{\listparindent}{0cm}
\setlength{\leftmargin}{\evensidemargin}
\addtolength{\leftmargin}{\tmplength}
\settowidth{\labelsep}{X}
\addtolength{\leftmargin}{\labelsep}
\setlength{\labelwidth}{\tmplength}
}
\begin{flushleft}
\item[\textbf{Declaração}\hfill]
\begin{ttfamily}
public const cmMyCancel            = CmNulo + 15;\end{ttfamily}


\end{flushleft}
\end{list}
\paragraph*{cmPrint}\hspace*{\fill}

\begin{list}{}{
\settowidth{\tmplength}{\textbf{Declaração}}
\setlength{\itemindent}{0cm}
\setlength{\listparindent}{0cm}
\setlength{\leftmargin}{\evensidemargin}
\addtolength{\leftmargin}{\tmplength}
\settowidth{\labelsep}{X}
\addtolength{\leftmargin}{\labelsep}
\setlength{\labelwidth}{\tmplength}
}
\begin{flushleft}
\item[\textbf{Declaração}\hfill]
\begin{ttfamily}
public const cmPrint               = CmNulo + 16;\end{ttfamily}


\end{flushleft}
\end{list}
\paragraph*{CmImport}\hspace*{\fill}

\begin{list}{}{
\settowidth{\tmplength}{\textbf{Declaração}}
\setlength{\itemindent}{0cm}
\setlength{\listparindent}{0cm}
\setlength{\leftmargin}{\evensidemargin}
\addtolength{\leftmargin}{\tmplength}
\settowidth{\labelsep}{X}
\addtolength{\leftmargin}{\labelsep}
\setlength{\labelwidth}{\tmplength}
}
\begin{flushleft}
\item[\textbf{Declaração}\hfill]
\begin{ttfamily}
public const CmImport              = CmNulo + 17;\end{ttfamily}


\end{flushleft}
\end{list}
\paragraph*{CmProcess}\hspace*{\fill}

\begin{list}{}{
\settowidth{\tmplength}{\textbf{Declaração}}
\setlength{\itemindent}{0cm}
\setlength{\listparindent}{0cm}
\setlength{\leftmargin}{\evensidemargin}
\addtolength{\leftmargin}{\tmplength}
\settowidth{\labelsep}{X}
\addtolength{\leftmargin}{\labelsep}
\setlength{\labelwidth}{\tmplength}
}
\begin{flushleft}
\item[\textbf{Declaração}\hfill]
\begin{ttfamily}
public const CmProcess             = CmNulo + 18;\end{ttfamily}


\end{flushleft}
\end{list}
\paragraph*{CmExecEndProc}\hspace*{\fill}

\begin{list}{}{
\settowidth{\tmplength}{\textbf{Declaração}}
\setlength{\itemindent}{0cm}
\setlength{\listparindent}{0cm}
\setlength{\leftmargin}{\evensidemargin}
\addtolength{\leftmargin}{\tmplength}
\settowidth{\labelsep}{X}
\addtolength{\leftmargin}{\labelsep}
\setlength{\labelwidth}{\tmplength}
}
\begin{flushleft}
\item[\textbf{Declaração}\hfill]
\begin{ttfamily}
public const CmExecEndProc         = CmNulo + 19;\end{ttfamily}


\end{flushleft}
\par
\item[\textbf{Descrição}]
Usado para acessar a pesquisa associado ao campo

\end{list}
\paragraph*{CmExecComboBox}\hspace*{\fill}

\begin{list}{}{
\settowidth{\tmplength}{\textbf{Declaração}}
\setlength{\itemindent}{0cm}
\setlength{\listparindent}{0cm}
\setlength{\leftmargin}{\evensidemargin}
\addtolength{\leftmargin}{\tmplength}
\settowidth{\labelsep}{X}
\addtolength{\leftmargin}{\labelsep}
\setlength{\labelwidth}{\tmplength}
}
\begin{flushleft}
\item[\textbf{Declaração}\hfill]
\begin{ttfamily}
public const CmExecComboBox        = CmNulo + 20;\end{ttfamily}


\end{flushleft}
\par
\item[\textbf{Descrição}]
Usado para acessar a visao associada ao campo. Usado para visualizar CamposEnumerado e lista de forma geral

\end{list}
\paragraph*{CmExecCommand}\hspace*{\fill}

\begin{list}{}{
\settowidth{\tmplength}{\textbf{Declaração}}
\setlength{\itemindent}{0cm}
\setlength{\listparindent}{0cm}
\setlength{\leftmargin}{\evensidemargin}
\addtolength{\leftmargin}{\tmplength}
\settowidth{\labelsep}{X}
\addtolength{\leftmargin}{\labelsep}
\setlength{\labelwidth}{\tmplength}
}
\begin{flushleft}
\item[\textbf{Declaração}\hfill]
\begin{ttfamily}
public const CmExecCommand         = CmNulo + 21;\end{ttfamily}


\end{flushleft}
\par
\item[\textbf{Descrição}]
O comando vinculado ao campo focado e disparado para apliication.HanleEvent() se

\end{list}
\paragraph*{CmCreate{\_}Shortcut}\hspace*{\fill}

\begin{list}{}{
\settowidth{\tmplength}{\textbf{Declaração}}
\setlength{\itemindent}{0cm}
\setlength{\listparindent}{0cm}
\setlength{\leftmargin}{\evensidemargin}
\addtolength{\leftmargin}{\tmplength}
\settowidth{\labelsep}{X}
\addtolength{\leftmargin}{\labelsep}
\setlength{\labelwidth}{\tmplength}
}
\begin{flushleft}
\item[\textbf{Declaração}\hfill]
\begin{ttfamily}
public const CmCreate{\_}Shortcut     = CmNulo + 22;\end{ttfamily}


\end{flushleft}
\end{list}
\paragraph*{CmVisualizar}\hspace*{\fill}

\begin{list}{}{
\settowidth{\tmplength}{\textbf{Declaração}}
\setlength{\itemindent}{0cm}
\setlength{\listparindent}{0cm}
\setlength{\leftmargin}{\evensidemargin}
\addtolength{\leftmargin}{\tmplength}
\settowidth{\labelsep}{X}
\addtolength{\leftmargin}{\labelsep}
\setlength{\labelwidth}{\tmplength}
}
\begin{flushleft}
\item[\textbf{Declaração}\hfill]
\begin{ttfamily}
public const CmVisualizar          = CmNulo + 23;\end{ttfamily}


\end{flushleft}
\end{list}
\paragraph*{CmExport{\_}Stru}\hspace*{\fill}

\begin{list}{}{
\settowidth{\tmplength}{\textbf{Declaração}}
\setlength{\itemindent}{0cm}
\setlength{\listparindent}{0cm}
\setlength{\leftmargin}{\evensidemargin}
\addtolength{\leftmargin}{\tmplength}
\settowidth{\labelsep}{X}
\addtolength{\leftmargin}{\labelsep}
\setlength{\labelwidth}{\tmplength}
}
\begin{flushleft}
\item[\textbf{Declaração}\hfill]
\begin{ttfamily}
public const CmExport{\_}Stru         = CmNulo + 24;\end{ttfamily}


\end{flushleft}
\end{list}
\paragraph*{CmExport}\hspace*{\fill}

\begin{list}{}{
\settowidth{\tmplength}{\textbf{Declaração}}
\setlength{\itemindent}{0cm}
\setlength{\listparindent}{0cm}
\setlength{\leftmargin}{\evensidemargin}
\addtolength{\leftmargin}{\tmplength}
\settowidth{\labelsep}{X}
\addtolength{\leftmargin}{\labelsep}
\setlength{\labelwidth}{\tmplength}
}
\begin{flushleft}
\item[\textbf{Declaração}\hfill]
\begin{ttfamily}
public const CmExport              = CmNulo + 25;\end{ttfamily}


\end{flushleft}
\end{list}
\paragraph*{CmInt}\hspace*{\fill}

\begin{list}{}{
\settowidth{\tmplength}{\textbf{Declaração}}
\setlength{\itemindent}{0cm}
\setlength{\listparindent}{0cm}
\setlength{\leftmargin}{\evensidemargin}
\addtolength{\leftmargin}{\tmplength}
\settowidth{\labelsep}{X}
\addtolength{\leftmargin}{\labelsep}
\setlength{\labelwidth}{\tmplength}
}
\begin{flushleft}
\item[\textbf{Declaração}\hfill]
\begin{ttfamily}
public const CmInt                 = CmNulo + 29;\end{ttfamily}


\end{flushleft}
\end{list}
\paragraph*{TCmLivre}\hspace*{\fill}

\begin{list}{}{
\settowidth{\tmplength}{\textbf{Declaração}}
\setlength{\itemindent}{0cm}
\setlength{\listparindent}{0cm}
\setlength{\leftmargin}{\evensidemargin}
\addtolength{\leftmargin}{\tmplength}
\settowidth{\labelsep}{X}
\addtolength{\leftmargin}{\labelsep}
\setlength{\labelwidth}{\tmplength}
}
\begin{flushleft}
\item[\textbf{Declaração}\hfill]
\begin{ttfamily}
public const TCmLivre    = [CmVisualizar..CmInt];\end{ttfamily}


\end{flushleft}
\end{list}
\paragraph*{TCmCommands}\hspace*{\fill}

\begin{list}{}{
\settowidth{\tmplength}{\textbf{Declaração}}
\setlength{\itemindent}{0cm}
\setlength{\listparindent}{0cm}
\setlength{\leftmargin}{\evensidemargin}
\addtolength{\leftmargin}{\tmplength}
\settowidth{\labelsep}{X}
\addtolength{\leftmargin}{\labelsep}
\setlength{\labelwidth}{\tmplength}
}
\begin{flushleft}
\item[\textbf{Declaração}\hfill]
\begin{ttfamily}
public const TCmCommands = [CmInt..255];\end{ttfamily}


\end{flushleft}
\end{list}
\paragraph*{TCmDb}\hspace*{\fill}

\begin{list}{}{
\settowidth{\tmplength}{\textbf{Declaração}}
\setlength{\itemindent}{0cm}
\setlength{\listparindent}{0cm}
\setlength{\leftmargin}{\evensidemargin}
\addtolength{\leftmargin}{\tmplength}
\settowidth{\labelsep}{X}
\addtolength{\leftmargin}{\labelsep}
\setlength{\labelwidth}{\tmplength}
}
\begin{flushleft}
\item[\textbf{Declaração}\hfill]
\begin{ttfamily}
public const TCmDb       = [CmDbNextRec    ,
                             CmDbPrevRec    ,
                             CmDbNextRecValid,
                             CmDbPrevRecValid,
                             CmDbFindRec    ,
                             CmDbSearchRec  ,
                             CmDbGoEof      ,
                             CmDbGoBof      ,
                             CmDbLocaliza
                            ];\end{ttfamily}


\end{flushleft}
\end{list}
\paragraph*{TCmDbView}\hspace*{\fill}

\begin{list}{}{
\settowidth{\tmplength}{\textbf{Declaração}}
\setlength{\itemindent}{0cm}
\setlength{\listparindent}{0cm}
\setlength{\leftmargin}{\evensidemargin}
\addtolength{\leftmargin}{\tmplength}
\settowidth{\labelsep}{X}
\addtolength{\leftmargin}{\labelsep}
\setlength{\labelwidth}{\tmplength}
}
\begin{flushleft}
\item[\textbf{Declaração}\hfill]
\begin{ttfamily}
public const TCmDbView    = [cmMyOK                  ,
                              cmMyCancel               ,
                              CmEditDlg                ,
                              CmEvaluateRecord         ,
                              cmZeroizeRecord          ,
                              CmNewRecord,
                              CmProcess,
                              cmPrint,
                              CmExecEndProc,
                              CmExecComboBox
                           ];\end{ttfamily}


\end{flushleft}
\end{list}
\paragraph*{TCmOutros}\hspace*{\fill}

\begin{list}{}{
\settowidth{\tmplength}{\textbf{Declaração}}
\setlength{\itemindent}{0cm}
\setlength{\listparindent}{0cm}
\setlength{\leftmargin}{\evensidemargin}
\addtolength{\leftmargin}{\tmplength}
\settowidth{\labelsep}{X}
\addtolength{\leftmargin}{\labelsep}
\setlength{\labelwidth}{\tmplength}
}
\begin{flushleft}
\item[\textbf{Declaração}\hfill]
\begin{ttfamily}
public const TCmOutros        =[cmPrint];\end{ttfamily}


\end{flushleft}
\end{list}
\paragraph*{CmNortSoft}\hspace*{\fill}

\begin{list}{}{
\settowidth{\tmplength}{\textbf{Declaração}}
\setlength{\itemindent}{0cm}
\setlength{\listparindent}{0cm}
\setlength{\leftmargin}{\evensidemargin}
\addtolength{\leftmargin}{\tmplength}
\settowidth{\labelsep}{X}
\addtolength{\leftmargin}{\labelsep}
\setlength{\labelwidth}{\tmplength}
}
\begin{flushleft}
\item[\textbf{Declaração}\hfill]
\begin{ttfamily}
public const CmNortSoft           = 50000;\end{ttfamily}


\end{flushleft}
\par
\item[\textbf{Descrição}]
\textbf{}\textbf{}\textbf{}\textbf{}\textbf{}\textbf{}\textbf{}\textbf{}\textbf{}\textbf{}\textbf{}\textbf{}\textbf{}\textbf{}\textbf{}\textbf{}**

\end{list}
\paragraph*{CmDbAddRec}\hspace*{\fill}

\begin{list}{}{
\settowidth{\tmplength}{\textbf{Declaração}}
\setlength{\itemindent}{0cm}
\setlength{\listparindent}{0cm}
\setlength{\leftmargin}{\evensidemargin}
\addtolength{\leftmargin}{\tmplength}
\settowidth{\labelsep}{X}
\addtolength{\leftmargin}{\labelsep}
\setlength{\labelwidth}{\tmplength}
}
\begin{flushleft}
\item[\textbf{Declaração}\hfill]
\begin{ttfamily}
public const CmDbAddRec           = CmNortSoft + 001;\end{ttfamily}


\end{flushleft}
\end{list}
\paragraph*{CmDbDeleteRec}\hspace*{\fill}

\begin{list}{}{
\settowidth{\tmplength}{\textbf{Declaração}}
\setlength{\itemindent}{0cm}
\setlength{\listparindent}{0cm}
\setlength{\leftmargin}{\evensidemargin}
\addtolength{\leftmargin}{\tmplength}
\settowidth{\labelsep}{X}
\addtolength{\leftmargin}{\labelsep}
\setlength{\labelwidth}{\tmplength}
}
\begin{flushleft}
\item[\textbf{Declaração}\hfill]
\begin{ttfamily}
public const CmDbDeleteRec        = CmNortSoft + 002;\end{ttfamily}


\end{flushleft}
\end{list}
\paragraph*{CmDbGetRec}\hspace*{\fill}

\begin{list}{}{
\settowidth{\tmplength}{\textbf{Declaração}}
\setlength{\itemindent}{0cm}
\setlength{\listparindent}{0cm}
\setlength{\leftmargin}{\evensidemargin}
\addtolength{\leftmargin}{\tmplength}
\settowidth{\labelsep}{X}
\addtolength{\leftmargin}{\labelsep}
\setlength{\labelwidth}{\tmplength}
}
\begin{flushleft}
\item[\textbf{Declaração}\hfill]
\begin{ttfamily}
public const CmDbGetRec           = CmNortSoft + 003;\end{ttfamily}


\end{flushleft}
\end{list}
\paragraph*{CmDbPutRec}\hspace*{\fill}

\begin{list}{}{
\settowidth{\tmplength}{\textbf{Declaração}}
\setlength{\itemindent}{0cm}
\setlength{\listparindent}{0cm}
\setlength{\leftmargin}{\evensidemargin}
\addtolength{\leftmargin}{\tmplength}
\settowidth{\labelsep}{X}
\addtolength{\leftmargin}{\labelsep}
\setlength{\labelwidth}{\tmplength}
}
\begin{flushleft}
\item[\textbf{Declaração}\hfill]
\begin{ttfamily}
public const CmDbPutRec           = CmNortSoft + 004;\end{ttfamily}


\end{flushleft}
\end{list}
\paragraph*{CmDbUpdateRec}\hspace*{\fill}

\begin{list}{}{
\settowidth{\tmplength}{\textbf{Declaração}}
\setlength{\itemindent}{0cm}
\setlength{\listparindent}{0cm}
\setlength{\leftmargin}{\evensidemargin}
\addtolength{\leftmargin}{\tmplength}
\settowidth{\labelsep}{X}
\addtolength{\leftmargin}{\labelsep}
\setlength{\labelwidth}{\tmplength}
}
\begin{flushleft}
\item[\textbf{Declaração}\hfill]
\begin{ttfamily}
public const CmDbUpdateRec        = CmNortSoft + 005;\end{ttfamily}


\end{flushleft}
\end{list}
\paragraph*{CmDbSearchTop}\hspace*{\fill}

\begin{list}{}{
\settowidth{\tmplength}{\textbf{Declaração}}
\setlength{\itemindent}{0cm}
\setlength{\listparindent}{0cm}
\setlength{\leftmargin}{\evensidemargin}
\addtolength{\leftmargin}{\tmplength}
\settowidth{\labelsep}{X}
\addtolength{\leftmargin}{\labelsep}
\setlength{\labelwidth}{\tmplength}
}
\begin{flushleft}
\item[\textbf{Declaração}\hfill]
\begin{ttfamily}
public const CmDbSearchTop        = CmNortSoft + 006;\end{ttfamily}


\end{flushleft}
\end{list}
\paragraph*{CmDbSearchKey}\hspace*{\fill}

\begin{list}{}{
\settowidth{\tmplength}{\textbf{Declaração}}
\setlength{\itemindent}{0cm}
\setlength{\listparindent}{0cm}
\setlength{\leftmargin}{\evensidemargin}
\addtolength{\leftmargin}{\tmplength}
\settowidth{\labelsep}{X}
\addtolength{\leftmargin}{\labelsep}
\setlength{\labelwidth}{\tmplength}
}
\begin{flushleft}
\item[\textbf{Declaração}\hfill]
\begin{ttfamily}
public const CmDbSearchKey        = CmNortSoft + 007;\end{ttfamily}


\end{flushleft}
\end{list}
\paragraph*{CmDbUsedRecs{\_}Valid}\hspace*{\fill}

\begin{list}{}{
\settowidth{\tmplength}{\textbf{Declaração}}
\setlength{\itemindent}{0cm}
\setlength{\listparindent}{0cm}
\setlength{\leftmargin}{\evensidemargin}
\addtolength{\leftmargin}{\tmplength}
\settowidth{\labelsep}{X}
\addtolength{\leftmargin}{\labelsep}
\setlength{\labelwidth}{\tmplength}
}
\begin{flushleft}
\item[\textbf{Declaração}\hfill]
\begin{ttfamily}
public const CmDbUsedRecs{\_}Valid    = CmNortSoft + 008;\end{ttfamily}


\end{flushleft}
\end{list}
\paragraph*{CmOkEscrevaParametrosDosRelatorios}\hspace*{\fill}

\begin{list}{}{
\settowidth{\tmplength}{\textbf{Declaração}}
\setlength{\itemindent}{0cm}
\setlength{\listparindent}{0cm}
\setlength{\leftmargin}{\evensidemargin}
\addtolength{\leftmargin}{\tmplength}
\settowidth{\labelsep}{X}
\addtolength{\leftmargin}{\labelsep}
\setlength{\labelwidth}{\tmplength}
}
\begin{flushleft}
\item[\textbf{Declaração}\hfill]
\begin{ttfamily}
public const CmOkEscrevaParametrosDosRelatorios = CmNortSoft + 009;\end{ttfamily}


\end{flushleft}
\end{list}
\paragraph*{CmDbSelecionaIndice}\hspace*{\fill}

\begin{list}{}{
\settowidth{\tmplength}{\textbf{Declaração}}
\setlength{\itemindent}{0cm}
\setlength{\listparindent}{0cm}
\setlength{\leftmargin}{\evensidemargin}
\addtolength{\leftmargin}{\tmplength}
\settowidth{\labelsep}{X}
\addtolength{\leftmargin}{\labelsep}
\setlength{\labelwidth}{\tmplength}
}
\begin{flushleft}
\item[\textbf{Declaração}\hfill]
\begin{ttfamily}
public const CmDbSelecionaIndice   = CmNortSoft + 010;\end{ttfamily}


\end{flushleft}
\end{list}
\paragraph*{LivreCmVisualisa}\hspace*{\fill}

\begin{list}{}{
\settowidth{\tmplength}{\textbf{Declaração}}
\setlength{\itemindent}{0cm}
\setlength{\listparindent}{0cm}
\setlength{\leftmargin}{\evensidemargin}
\addtolength{\leftmargin}{\tmplength}
\settowidth{\labelsep}{X}
\addtolength{\leftmargin}{\labelsep}
\setlength{\labelwidth}{\tmplength}
}
\begin{flushleft}
\item[\textbf{Declaração}\hfill]
\begin{ttfamily}
public const LivreCmVisualisa      = CmNortSoft + 011;\end{ttfamily}


\end{flushleft}
\end{list}
\paragraph*{CmQuitInterno}\hspace*{\fill}

\begin{list}{}{
\settowidth{\tmplength}{\textbf{Declaração}}
\setlength{\itemindent}{0cm}
\setlength{\listparindent}{0cm}
\setlength{\leftmargin}{\evensidemargin}
\addtolength{\leftmargin}{\tmplength}
\settowidth{\labelsep}{X}
\addtolength{\leftmargin}{\labelsep}
\setlength{\labelwidth}{\tmplength}
}
\begin{flushleft}
\item[\textbf{Declaração}\hfill]
\begin{ttfamily}
public const CmQuitInterno         = CmNortSoft + 012;\end{ttfamily}


\end{flushleft}
\end{list}
\paragraph*{CmSobre}\hspace*{\fill}

\begin{list}{}{
\settowidth{\tmplength}{\textbf{Declaração}}
\setlength{\itemindent}{0cm}
\setlength{\listparindent}{0cm}
\setlength{\leftmargin}{\evensidemargin}
\addtolength{\leftmargin}{\tmplength}
\settowidth{\labelsep}{X}
\addtolength{\leftmargin}{\labelsep}
\setlength{\labelwidth}{\tmplength}
}
\begin{flushleft}
\item[\textbf{Declaração}\hfill]
\begin{ttfamily}
public const CmSobre               = CmNortSoft + 013;\end{ttfamily}


\end{flushleft}
\end{list}
\paragraph*{CmDbOnEnter}\hspace*{\fill}

\begin{list}{}{
\settowidth{\tmplength}{\textbf{Declaração}}
\setlength{\itemindent}{0cm}
\setlength{\listparindent}{0cm}
\setlength{\leftmargin}{\evensidemargin}
\addtolength{\leftmargin}{\tmplength}
\settowidth{\labelsep}{X}
\addtolength{\leftmargin}{\labelsep}
\setlength{\labelwidth}{\tmplength}
}
\begin{flushleft}
\item[\textbf{Declaração}\hfill]
\begin{ttfamily}
public const CmDbOnEnter           = CmNortSoft + 014;\end{ttfamily}


\end{flushleft}
\end{list}
\paragraph*{CmDbOnExit}\hspace*{\fill}

\begin{list}{}{
\settowidth{\tmplength}{\textbf{Declaração}}
\setlength{\itemindent}{0cm}
\setlength{\listparindent}{0cm}
\setlength{\leftmargin}{\evensidemargin}
\addtolength{\leftmargin}{\tmplength}
\settowidth{\labelsep}{X}
\addtolength{\leftmargin}{\labelsep}
\setlength{\labelwidth}{\tmplength}
}
\begin{flushleft}
\item[\textbf{Declaração}\hfill]
\begin{ttfamily}
public const CmDbOnExit            = CmNortSoft + 015;\end{ttfamily}


\end{flushleft}
\end{list}
\paragraph*{cmCores}\hspace*{\fill}

\begin{list}{}{
\settowidth{\tmplength}{\textbf{Declaração}}
\setlength{\itemindent}{0cm}
\setlength{\listparindent}{0cm}
\setlength{\leftmargin}{\evensidemargin}
\addtolength{\leftmargin}{\tmplength}
\settowidth{\labelsep}{X}
\addtolength{\leftmargin}{\labelsep}
\setlength{\labelwidth}{\tmplength}
}
\begin{flushleft}
\item[\textbf{Declaração}\hfill]
\begin{ttfamily}
public const cmCores               = CmNortSoft + 016;\end{ttfamily}


\end{flushleft}
\end{list}
\paragraph*{CmF7}\hspace*{\fill}

\begin{list}{}{
\settowidth{\tmplength}{\textbf{Declaração}}
\setlength{\itemindent}{0cm}
\setlength{\listparindent}{0cm}
\setlength{\leftmargin}{\evensidemargin}
\addtolength{\leftmargin}{\tmplength}
\settowidth{\labelsep}{X}
\addtolength{\leftmargin}{\labelsep}
\setlength{\labelwidth}{\tmplength}
}
\begin{flushleft}
\item[\textbf{Declaração}\hfill]
\begin{ttfamily}
public const CmF7                  = CmNortSoft + 017;\end{ttfamily}


\end{flushleft}
\end{list}
\paragraph*{CmDbLabel{\_}DoubleClick}\hspace*{\fill}

\begin{list}{}{
\settowidth{\tmplength}{\textbf{Declaração}}
\setlength{\itemindent}{0cm}
\setlength{\listparindent}{0cm}
\setlength{\leftmargin}{\evensidemargin}
\addtolength{\leftmargin}{\tmplength}
\settowidth{\labelsep}{X}
\addtolength{\leftmargin}{\labelsep}
\setlength{\labelwidth}{\tmplength}
}
\begin{flushleft}
\item[\textbf{Declaração}\hfill]
\begin{ttfamily}
public const CmDbLabel{\_}DoubleClick = CmNortSoft + 018;\end{ttfamily}


\end{flushleft}
\end{list}
\paragraph*{cmDbView{\_}DoubleClick}\hspace*{\fill}

\begin{list}{}{
\settowidth{\tmplength}{\textbf{Declaração}}
\setlength{\itemindent}{0cm}
\setlength{\listparindent}{0cm}
\setlength{\leftmargin}{\evensidemargin}
\addtolength{\leftmargin}{\tmplength}
\settowidth{\labelsep}{X}
\addtolength{\leftmargin}{\labelsep}
\setlength{\labelwidth}{\tmplength}
}
\begin{flushleft}
\item[\textbf{Declaração}\hfill]
\begin{ttfamily}
public const cmDbView{\_}DoubleClick  = CmNortSoft + 019;\end{ttfamily}


\end{flushleft}
\end{list}
\paragraph*{CmDbOrdemCressante}\hspace*{\fill}

\begin{list}{}{
\settowidth{\tmplength}{\textbf{Declaração}}
\setlength{\itemindent}{0cm}
\setlength{\listparindent}{0cm}
\setlength{\leftmargin}{\evensidemargin}
\addtolength{\leftmargin}{\tmplength}
\settowidth{\labelsep}{X}
\addtolength{\leftmargin}{\labelsep}
\setlength{\labelwidth}{\tmplength}
}
\begin{flushleft}
\item[\textbf{Declaração}\hfill]
\begin{ttfamily}
public const CmDbOrdemCressante    = CmNortSoft + 020;\end{ttfamily}


\end{flushleft}
\end{list}
\paragraph*{CmDbOrdemDecrescente}\hspace*{\fill}

\begin{list}{}{
\settowidth{\tmplength}{\textbf{Declaração}}
\setlength{\itemindent}{0cm}
\setlength{\listparindent}{0cm}
\setlength{\leftmargin}{\evensidemargin}
\addtolength{\leftmargin}{\tmplength}
\settowidth{\labelsep}{X}
\addtolength{\leftmargin}{\labelsep}
\setlength{\labelwidth}{\tmplength}
}
\begin{flushleft}
\item[\textbf{Declaração}\hfill]
\begin{ttfamily}
public const CmDbOrdemDecrescente  = CmNortSoft + 021;\end{ttfamily}


\end{flushleft}
\end{list}
\paragraph*{CmDbSelecColunaAtual}\hspace*{\fill}

\begin{list}{}{
\settowidth{\tmplength}{\textbf{Declaração}}
\setlength{\itemindent}{0cm}
\setlength{\listparindent}{0cm}
\setlength{\leftmargin}{\evensidemargin}
\addtolength{\leftmargin}{\tmplength}
\settowidth{\labelsep}{X}
\addtolength{\leftmargin}{\labelsep}
\setlength{\labelwidth}{\tmplength}
}
\begin{flushleft}
\item[\textbf{Declaração}\hfill]
\begin{ttfamily}
public const CmDbSelecColunaAtual  = CmNortSoft + 022;\end{ttfamily}


\end{flushleft}
\end{list}
\paragraph*{CmMouseDownmbRightButton}\hspace*{\fill}

\begin{list}{}{
\settowidth{\tmplength}{\textbf{Declaração}}
\setlength{\itemindent}{0cm}
\setlength{\listparindent}{0cm}
\setlength{\leftmargin}{\evensidemargin}
\addtolength{\leftmargin}{\tmplength}
\settowidth{\labelsep}{X}
\addtolength{\leftmargin}{\labelsep}
\setlength{\labelwidth}{\tmplength}
}
\begin{flushleft}
\item[\textbf{Declaração}\hfill]
\begin{ttfamily}
public const CmMouseDownmbRightButton = CmNortSoft + 023;\end{ttfamily}


\end{flushleft}
\end{list}
\paragraph*{CmReindex}\hspace*{\fill}

\begin{list}{}{
\settowidth{\tmplength}{\textbf{Declaração}}
\setlength{\itemindent}{0cm}
\setlength{\listparindent}{0cm}
\setlength{\leftmargin}{\evensidemargin}
\addtolength{\leftmargin}{\tmplength}
\settowidth{\labelsep}{X}
\addtolength{\leftmargin}{\labelsep}
\setlength{\labelwidth}{\tmplength}
}
\begin{flushleft}
\item[\textbf{Declaração}\hfill]
\begin{ttfamily}
public const CmReindex             = CmNortSoft + 024;\end{ttfamily}


\end{flushleft}
\end{list}
\paragraph*{CmCadastraImpressoraRede}\hspace*{\fill}

\begin{list}{}{
\settowidth{\tmplength}{\textbf{Declaração}}
\setlength{\itemindent}{0cm}
\setlength{\listparindent}{0cm}
\setlength{\leftmargin}{\evensidemargin}
\addtolength{\leftmargin}{\tmplength}
\settowidth{\labelsep}{X}
\addtolength{\leftmargin}{\labelsep}
\setlength{\labelwidth}{\tmplength}
}
\begin{flushleft}
\item[\textbf{Declaração}\hfill]
\begin{ttfamily}
public const CmCadastraImpressoraRede  = CmNortSoft + 025;\end{ttfamily}


\end{flushleft}
\end{list}
\paragraph*{CmInfoSystem}\hspace*{\fill}

\begin{list}{}{
\settowidth{\tmplength}{\textbf{Declaração}}
\setlength{\itemindent}{0cm}
\setlength{\listparindent}{0cm}
\setlength{\leftmargin}{\evensidemargin}
\addtolength{\leftmargin}{\tmplength}
\settowidth{\labelsep}{X}
\addtolength{\leftmargin}{\labelsep}
\setlength{\labelwidth}{\tmplength}
}
\begin{flushleft}
\item[\textbf{Declaração}\hfill]
\begin{ttfamily}
public const CmInfoSystem          = CmNortSoft + 026;\end{ttfamily}


\end{flushleft}
\end{list}
\paragraph*{cmPrintSemFormatar}\hspace*{\fill}

\begin{list}{}{
\settowidth{\tmplength}{\textbf{Declaração}}
\setlength{\itemindent}{0cm}
\setlength{\listparindent}{0cm}
\setlength{\leftmargin}{\evensidemargin}
\addtolength{\leftmargin}{\tmplength}
\settowidth{\labelsep}{X}
\addtolength{\leftmargin}{\labelsep}
\setlength{\labelwidth}{\tmplength}
}
\begin{flushleft}
\item[\textbf{Declaração}\hfill]
\begin{ttfamily}
public const cmPrintSemFormatar    = CmNortSoft + 027;\end{ttfamily}


\end{flushleft}
\end{list}
\paragraph*{CmDbDoBeforeInsert}\hspace*{\fill}

\begin{list}{}{
\settowidth{\tmplength}{\textbf{Declaração}}
\setlength{\itemindent}{0cm}
\setlength{\listparindent}{0cm}
\setlength{\leftmargin}{\evensidemargin}
\addtolength{\leftmargin}{\tmplength}
\settowidth{\labelsep}{X}
\addtolength{\leftmargin}{\labelsep}
\setlength{\labelwidth}{\tmplength}
}
\begin{flushleft}
\item[\textbf{Declaração}\hfill]
\begin{ttfamily}
public const CmDbDoBeforeInsert      = CmNortSoft + 028;\end{ttfamily}


\end{flushleft}
\end{list}
\paragraph*{CmDbDoBeforePost}\hspace*{\fill}

\begin{list}{}{
\settowidth{\tmplength}{\textbf{Declaração}}
\setlength{\itemindent}{0cm}
\setlength{\listparindent}{0cm}
\setlength{\leftmargin}{\evensidemargin}
\addtolength{\leftmargin}{\tmplength}
\settowidth{\labelsep}{X}
\addtolength{\leftmargin}{\labelsep}
\setlength{\labelwidth}{\tmplength}
}
\begin{flushleft}
\item[\textbf{Declaração}\hfill]
\begin{ttfamily}
public const CmDbDoBeforePost        = CmNortSoft + 029;\end{ttfamily}


\end{flushleft}
\end{list}
\paragraph*{CmDbDoBeforeDelete}\hspace*{\fill}

\begin{list}{}{
\settowidth{\tmplength}{\textbf{Declaração}}
\setlength{\itemindent}{0cm}
\setlength{\listparindent}{0cm}
\setlength{\leftmargin}{\evensidemargin}
\addtolength{\leftmargin}{\tmplength}
\settowidth{\labelsep}{X}
\addtolength{\leftmargin}{\labelsep}
\setlength{\labelwidth}{\tmplength}
}
\begin{flushleft}
\item[\textbf{Declaração}\hfill]
\begin{ttfamily}
public const CmDbDoBeforeDelete      = CmNortSoft + 030;\end{ttfamily}


\end{flushleft}
\end{list}
\paragraph*{CmDbDoAfterInsert}\hspace*{\fill}

\begin{list}{}{
\settowidth{\tmplength}{\textbf{Declaração}}
\setlength{\itemindent}{0cm}
\setlength{\listparindent}{0cm}
\setlength{\leftmargin}{\evensidemargin}
\addtolength{\leftmargin}{\tmplength}
\settowidth{\labelsep}{X}
\addtolength{\leftmargin}{\labelsep}
\setlength{\labelwidth}{\tmplength}
}
\begin{flushleft}
\item[\textbf{Declaração}\hfill]
\begin{ttfamily}
public const CmDbDoAfterInsert       = CmNortSoft + 031;\end{ttfamily}


\end{flushleft}
\end{list}
\paragraph*{CmDbDoAfterPost}\hspace*{\fill}

\begin{list}{}{
\settowidth{\tmplength}{\textbf{Declaração}}
\setlength{\itemindent}{0cm}
\setlength{\listparindent}{0cm}
\setlength{\leftmargin}{\evensidemargin}
\addtolength{\leftmargin}{\tmplength}
\settowidth{\labelsep}{X}
\addtolength{\leftmargin}{\labelsep}
\setlength{\labelwidth}{\tmplength}
}
\begin{flushleft}
\item[\textbf{Declaração}\hfill]
\begin{ttfamily}
public const CmDbDoAfterPost         = CmNortSoft + 032;\end{ttfamily}


\end{flushleft}
\end{list}
\paragraph*{CmDbDoAfterDelete}\hspace*{\fill}

\begin{list}{}{
\settowidth{\tmplength}{\textbf{Declaração}}
\setlength{\itemindent}{0cm}
\setlength{\listparindent}{0cm}
\setlength{\leftmargin}{\evensidemargin}
\addtolength{\leftmargin}{\tmplength}
\settowidth{\labelsep}{X}
\addtolength{\leftmargin}{\labelsep}
\setlength{\labelwidth}{\tmplength}
}
\begin{flushleft}
\item[\textbf{Declaração}\hfill]
\begin{ttfamily}
public const CmDbDoAfterDelete       = CmNortSoft + 033;\end{ttfamily}


\end{flushleft}
\end{list}
\paragraph*{CmTb{\_}SelectRefCruzadaResume}\hspace*{\fill}

\begin{list}{}{
\settowidth{\tmplength}{\textbf{Declaração}}
\setlength{\itemindent}{0cm}
\setlength{\listparindent}{0cm}
\setlength{\leftmargin}{\evensidemargin}
\addtolength{\leftmargin}{\tmplength}
\settowidth{\labelsep}{X}
\addtolength{\leftmargin}{\labelsep}
\setlength{\labelwidth}{\tmplength}
}
\begin{flushleft}
\item[\textbf{Declaração}\hfill]
\begin{ttfamily}
public const CmTb{\_}SelectRefCruzadaResume = CmNortSoft + 034;\end{ttfamily}


\end{flushleft}
\end{list}
\paragraph*{CmTb{\_}SelectSelect}\hspace*{\fill}

\begin{list}{}{
\settowidth{\tmplength}{\textbf{Declaração}}
\setlength{\itemindent}{0cm}
\setlength{\listparindent}{0cm}
\setlength{\leftmargin}{\evensidemargin}
\addtolength{\leftmargin}{\tmplength}
\settowidth{\labelsep}{X}
\addtolength{\leftmargin}{\labelsep}
\setlength{\labelwidth}{\tmplength}
}
\begin{flushleft}
\item[\textbf{Declaração}\hfill]
\begin{ttfamily}
public const CmTb{\_}SelectSelect           = CmNortSoft + 035;\end{ttfamily}


\end{flushleft}
\end{list}
\paragraph*{CmTb{\_}SelectResume}\hspace*{\fill}

\begin{list}{}{
\settowidth{\tmplength}{\textbf{Declaração}}
\setlength{\itemindent}{0cm}
\setlength{\listparindent}{0cm}
\setlength{\leftmargin}{\evensidemargin}
\addtolength{\leftmargin}{\tmplength}
\settowidth{\labelsep}{X}
\addtolength{\leftmargin}{\labelsep}
\setlength{\labelwidth}{\tmplength}
}
\begin{flushleft}
\item[\textbf{Declaração}\hfill]
\begin{ttfamily}
public const CmTb{\_}SelectResume           = CmNortSoft + 036;\end{ttfamily}


\end{flushleft}
\end{list}
\paragraph*{CmRegistroValido}\hspace*{\fill}

\begin{list}{}{
\settowidth{\tmplength}{\textbf{Declaração}}
\setlength{\itemindent}{0cm}
\setlength{\listparindent}{0cm}
\setlength{\leftmargin}{\evensidemargin}
\addtolength{\leftmargin}{\tmplength}
\settowidth{\labelsep}{X}
\addtolength{\leftmargin}{\labelsep}
\setlength{\labelwidth}{\tmplength}
}
\begin{flushleft}
\item[\textbf{Declaração}\hfill]
\begin{ttfamily}
public const CmRegistroValido            = CmNortSoft + 037;\end{ttfamily}


\end{flushleft}
\end{list}
\paragraph*{CmCopyTo}\hspace*{\fill}

\begin{list}{}{
\settowidth{\tmplength}{\textbf{Declaração}}
\setlength{\itemindent}{0cm}
\setlength{\listparindent}{0cm}
\setlength{\leftmargin}{\evensidemargin}
\addtolength{\leftmargin}{\tmplength}
\settowidth{\labelsep}{X}
\addtolength{\leftmargin}{\labelsep}
\setlength{\labelwidth}{\tmplength}
}
\begin{flushleft}
\item[\textbf{Declaração}\hfill]
\begin{ttfamily}
public const CmCopyTo                    = CmNortSoft + 038;\end{ttfamily}


\end{flushleft}
\end{list}
\paragraph*{CmCadastraImpressoraLocal}\hspace*{\fill}

\begin{list}{}{
\settowidth{\tmplength}{\textbf{Declaração}}
\setlength{\itemindent}{0cm}
\setlength{\listparindent}{0cm}
\setlength{\leftmargin}{\evensidemargin}
\addtolength{\leftmargin}{\tmplength}
\settowidth{\labelsep}{X}
\addtolength{\leftmargin}{\labelsep}
\setlength{\labelwidth}{\tmplength}
}
\begin{flushleft}
\item[\textbf{Declaração}\hfill]
\begin{ttfamily}
public const CmCadastraImpressoraLocal   = CmNortSoft + 039;\end{ttfamily}


\end{flushleft}
\end{list}
\paragraph*{CmSetAppending}\hspace*{\fill}

\begin{list}{}{
\settowidth{\tmplength}{\textbf{Declaração}}
\setlength{\itemindent}{0cm}
\setlength{\listparindent}{0cm}
\setlength{\leftmargin}{\evensidemargin}
\addtolength{\leftmargin}{\tmplength}
\settowidth{\labelsep}{X}
\addtolength{\leftmargin}{\labelsep}
\setlength{\labelwidth}{\tmplength}
}
\begin{flushleft}
\item[\textbf{Declaração}\hfill]
\begin{ttfamily}
public const CmSetAppending              = CmNortSoft + 040;\end{ttfamily}


\end{flushleft}
\end{list}
\paragraph*{CmStartTransaction}\hspace*{\fill}

\begin{list}{}{
\settowidth{\tmplength}{\textbf{Declaração}}
\setlength{\itemindent}{0cm}
\setlength{\listparindent}{0cm}
\setlength{\leftmargin}{\evensidemargin}
\addtolength{\leftmargin}{\tmplength}
\settowidth{\labelsep}{X}
\addtolength{\leftmargin}{\labelsep}
\setlength{\labelwidth}{\tmplength}
}
\begin{flushleft}
\item[\textbf{Declaração}\hfill]
\begin{ttfamily}
public const CmStartTransaction          = CmNortSoft + 041;\end{ttfamily}


\end{flushleft}
\end{list}
\paragraph*{CmCommit}\hspace*{\fill}

\begin{list}{}{
\settowidth{\tmplength}{\textbf{Declaração}}
\setlength{\itemindent}{0cm}
\setlength{\listparindent}{0cm}
\setlength{\leftmargin}{\evensidemargin}
\addtolength{\leftmargin}{\tmplength}
\settowidth{\labelsep}{X}
\addtolength{\leftmargin}{\labelsep}
\setlength{\labelwidth}{\tmplength}
}
\begin{flushleft}
\item[\textbf{Declaração}\hfill]
\begin{ttfamily}
public const CmCommit                    = CmNortSoft + 042;\end{ttfamily}


\end{flushleft}
\end{list}
\paragraph*{CmRollback}\hspace*{\fill}

\begin{list}{}{
\settowidth{\tmplength}{\textbf{Declaração}}
\setlength{\itemindent}{0cm}
\setlength{\listparindent}{0cm}
\setlength{\leftmargin}{\evensidemargin}
\addtolength{\leftmargin}{\tmplength}
\settowidth{\labelsep}{X}
\addtolength{\leftmargin}{\labelsep}
\setlength{\labelwidth}{\tmplength}
}
\begin{flushleft}
\item[\textbf{Declaração}\hfill]
\begin{ttfamily}
public const CmRollback                  = CmNortSoft + 043;\end{ttfamily}


\end{flushleft}
\end{list}
\paragraph*{CmOnCalcRecord{\_}All}\hspace*{\fill}

\begin{list}{}{
\settowidth{\tmplength}{\textbf{Declaração}}
\setlength{\itemindent}{0cm}
\setlength{\listparindent}{0cm}
\setlength{\leftmargin}{\evensidemargin}
\addtolength{\leftmargin}{\tmplength}
\settowidth{\labelsep}{X}
\addtolength{\leftmargin}{\labelsep}
\setlength{\labelwidth}{\tmplength}
}
\begin{flushleft}
\item[\textbf{Declaração}\hfill]
\begin{ttfamily}
public const CmOnCalcRecord{\_}All          = CmNortSoft + 044;\end{ttfamily}


\end{flushleft}
\end{list}
\paragraph*{CmTime}\hspace*{\fill}

\begin{list}{}{
\settowidth{\tmplength}{\textbf{Declaração}}
\setlength{\itemindent}{0cm}
\setlength{\listparindent}{0cm}
\setlength{\leftmargin}{\evensidemargin}
\addtolength{\leftmargin}{\tmplength}
\settowidth{\labelsep}{X}
\addtolength{\leftmargin}{\labelsep}
\setlength{\labelwidth}{\tmplength}
}
\begin{flushleft}
\item[\textbf{Declaração}\hfill]
\begin{ttfamily}
public const CmTime                      = CmNortSoft + 045;\end{ttfamily}


\end{flushleft}
\end{list}
\paragraph*{cmEditaCores}\hspace*{\fill}

\begin{list}{}{
\settowidth{\tmplength}{\textbf{Declaração}}
\setlength{\itemindent}{0cm}
\setlength{\listparindent}{0cm}
\setlength{\leftmargin}{\evensidemargin}
\addtolength{\leftmargin}{\tmplength}
\settowidth{\labelsep}{X}
\addtolength{\leftmargin}{\labelsep}
\setlength{\labelwidth}{\tmplength}
}
\begin{flushleft}
\item[\textbf{Declaração}\hfill]
\begin{ttfamily}
public const cmEditaCores                = CmNortSoft + 046;\end{ttfamily}


\end{flushleft}
\end{list}
\paragraph*{cmSalvaCores}\hspace*{\fill}

\begin{list}{}{
\settowidth{\tmplength}{\textbf{Declaração}}
\setlength{\itemindent}{0cm}
\setlength{\listparindent}{0cm}
\setlength{\leftmargin}{\evensidemargin}
\addtolength{\leftmargin}{\tmplength}
\settowidth{\labelsep}{X}
\addtolength{\leftmargin}{\labelsep}
\setlength{\labelwidth}{\tmplength}
}
\begin{flushleft}
\item[\textbf{Declaração}\hfill]
\begin{ttfamily}
public const cmSalvaCores                = CmNortSoft + 047;\end{ttfamily}


\end{flushleft}
\end{list}
\paragraph*{cmHomePage}\hspace*{\fill}

\begin{list}{}{
\settowidth{\tmplength}{\textbf{Declaração}}
\setlength{\itemindent}{0cm}
\setlength{\listparindent}{0cm}
\setlength{\leftmargin}{\evensidemargin}
\addtolength{\leftmargin}{\tmplength}
\settowidth{\labelsep}{X}
\addtolength{\leftmargin}{\labelsep}
\setlength{\labelwidth}{\tmplength}
}
\begin{flushleft}
\item[\textbf{Declaração}\hfill]
\begin{ttfamily}
public const cmHomePage                  = CmNortSoft + 048;\end{ttfamily}


\end{flushleft}
\end{list}
\paragraph*{CmDbPack}\hspace*{\fill}

\begin{list}{}{
\settowidth{\tmplength}{\textbf{Declaração}}
\setlength{\itemindent}{0cm}
\setlength{\listparindent}{0cm}
\setlength{\leftmargin}{\evensidemargin}
\addtolength{\leftmargin}{\tmplength}
\settowidth{\labelsep}{X}
\addtolength{\leftmargin}{\labelsep}
\setlength{\labelwidth}{\tmplength}
}
\begin{flushleft}
\item[\textbf{Declaração}\hfill]
\begin{ttfamily}
public const CmDbPack                    = CmNortSoft + 049;\end{ttfamily}


\end{flushleft}
\end{list}
\paragraph*{FirstCmdNum}\hspace*{\fill}

\begin{list}{}{
\settowidth{\tmplength}{\textbf{Declaração}}
\setlength{\itemindent}{0cm}
\setlength{\listparindent}{0cm}
\setlength{\leftmargin}{\evensidemargin}
\addtolength{\leftmargin}{\tmplength}
\settowidth{\labelsep}{X}
\addtolength{\leftmargin}{\labelsep}
\setlength{\labelwidth}{\tmplength}
}
\begin{flushleft}
\item[\textbf{Declaração}\hfill]
\begin{ttfamily}
public const FirstCmdNum   =  4400;\end{ttfamily}


\end{flushleft}
\end{list}
\paragraph*{cmDMX}\hspace*{\fill}

\begin{list}{}{
\settowidth{\tmplength}{\textbf{Declaração}}
\setlength{\itemindent}{0cm}
\setlength{\listparindent}{0cm}
\setlength{\leftmargin}{\evensidemargin}
\addtolength{\leftmargin}{\tmplength}
\settowidth{\labelsep}{X}
\addtolength{\leftmargin}{\labelsep}
\setlength{\labelwidth}{\tmplength}
}
\begin{flushleft}
\item[\textbf{Declaração}\hfill]
\begin{ttfamily}
public const cmDMX               = FirstCmdNum;\end{ttfamily}


\end{flushleft}
\end{list}
\paragraph*{cmDMX{\_}RollCall}\hspace*{\fill}

\begin{list}{}{
\settowidth{\tmplength}{\textbf{Declaração}}
\setlength{\itemindent}{0cm}
\setlength{\listparindent}{0cm}
\setlength{\leftmargin}{\evensidemargin}
\addtolength{\leftmargin}{\tmplength}
\settowidth{\labelsep}{X}
\addtolength{\leftmargin}{\labelsep}
\setlength{\labelwidth}{\tmplength}
}
\begin{flushleft}
\item[\textbf{Declaração}\hfill]
\begin{ttfamily}
public const cmDMX{\_}RollCall      = cmDMX +  1;\end{ttfamily}


\end{flushleft}
\end{list}
\paragraph*{cmDMX{\_}Ack}\hspace*{\fill}

\begin{list}{}{
\settowidth{\tmplength}{\textbf{Declaração}}
\setlength{\itemindent}{0cm}
\setlength{\listparindent}{0cm}
\setlength{\leftmargin}{\evensidemargin}
\addtolength{\leftmargin}{\tmplength}
\settowidth{\labelsep}{X}
\addtolength{\leftmargin}{\labelsep}
\setlength{\labelwidth}{\tmplength}
}
\begin{flushleft}
\item[\textbf{Declaração}\hfill]
\begin{ttfamily}
public const cmDMX{\_}Ack           = cmDMX +  2;\end{ttfamily}


\end{flushleft}
\end{list}
\paragraph*{cmDMX{\_}FieldAltered}\hspace*{\fill}

\begin{list}{}{
\settowidth{\tmplength}{\textbf{Declaração}}
\setlength{\itemindent}{0cm}
\setlength{\listparindent}{0cm}
\setlength{\leftmargin}{\evensidemargin}
\addtolength{\leftmargin}{\tmplength}
\settowidth{\labelsep}{X}
\addtolength{\leftmargin}{\labelsep}
\setlength{\labelwidth}{\tmplength}
}
\begin{flushleft}
\item[\textbf{Declaração}\hfill]
\begin{ttfamily}
public const cmDMX{\_}FieldAltered  = cmDMX +  3;\end{ttfamily}


\end{flushleft}
\end{list}
\paragraph*{cmDMX{\_}Draw}\hspace*{\fill}

\begin{list}{}{
\settowidth{\tmplength}{\textbf{Declaração}}
\setlength{\itemindent}{0cm}
\setlength{\listparindent}{0cm}
\setlength{\leftmargin}{\evensidemargin}
\addtolength{\leftmargin}{\tmplength}
\settowidth{\labelsep}{X}
\addtolength{\leftmargin}{\labelsep}
\setlength{\labelwidth}{\tmplength}
}
\begin{flushleft}
\item[\textbf{Declaração}\hfill]
\begin{ttfamily}
public const cmDMX{\_}Draw          = cmDMX +  4;\end{ttfamily}


\end{flushleft}
\end{list}
\paragraph*{cmDMX{\_}DrawData}\hspace*{\fill}

\begin{list}{}{
\settowidth{\tmplength}{\textbf{Declaração}}
\setlength{\itemindent}{0cm}
\setlength{\listparindent}{0cm}
\setlength{\leftmargin}{\evensidemargin}
\addtolength{\leftmargin}{\tmplength}
\settowidth{\labelsep}{X}
\addtolength{\leftmargin}{\labelsep}
\setlength{\labelwidth}{\tmplength}
}
\begin{flushleft}
\item[\textbf{Declaração}\hfill]
\begin{ttfamily}
public const cmDMX{\_}DrawData      = cmDMX +  5;\end{ttfamily}


\end{flushleft}
\end{list}
\paragraph*{cmDMX{\_}Lock}\hspace*{\fill}

\begin{list}{}{
\settowidth{\tmplength}{\textbf{Declaração}}
\setlength{\itemindent}{0cm}
\setlength{\listparindent}{0cm}
\setlength{\leftmargin}{\evensidemargin}
\addtolength{\leftmargin}{\tmplength}
\settowidth{\labelsep}{X}
\addtolength{\leftmargin}{\labelsep}
\setlength{\labelwidth}{\tmplength}
}
\begin{flushleft}
\item[\textbf{Declaração}\hfill]
\begin{ttfamily}
public const cmDMX{\_}Lock          = cmDMX +  6;\end{ttfamily}


\end{flushleft}
\end{list}
\paragraph*{cmDMX{\_}LockData}\hspace*{\fill}

\begin{list}{}{
\settowidth{\tmplength}{\textbf{Declaração}}
\setlength{\itemindent}{0cm}
\setlength{\listparindent}{0cm}
\setlength{\leftmargin}{\evensidemargin}
\addtolength{\leftmargin}{\tmplength}
\settowidth{\labelsep}{X}
\addtolength{\leftmargin}{\labelsep}
\setlength{\labelwidth}{\tmplength}
}
\begin{flushleft}
\item[\textbf{Declaração}\hfill]
\begin{ttfamily}
public const cmDMX{\_}LockData      = cmDMX +  7;\end{ttfamily}


\end{flushleft}
\end{list}
\paragraph*{cmDMX{\_}Unlock}\hspace*{\fill}

\begin{list}{}{
\settowidth{\tmplength}{\textbf{Declaração}}
\setlength{\itemindent}{0cm}
\setlength{\listparindent}{0cm}
\setlength{\leftmargin}{\evensidemargin}
\addtolength{\leftmargin}{\tmplength}
\settowidth{\labelsep}{X}
\addtolength{\leftmargin}{\labelsep}
\setlength{\labelwidth}{\tmplength}
}
\begin{flushleft}
\item[\textbf{Declaração}\hfill]
\begin{ttfamily}
public const cmDMX{\_}Unlock        = cmDMX +  8;\end{ttfamily}


\end{flushleft}
\end{list}
\paragraph*{cmDMX{\_}UnlockData}\hspace*{\fill}

\begin{list}{}{
\settowidth{\tmplength}{\textbf{Declaração}}
\setlength{\itemindent}{0cm}
\setlength{\listparindent}{0cm}
\setlength{\leftmargin}{\evensidemargin}
\addtolength{\leftmargin}{\tmplength}
\settowidth{\labelsep}{X}
\addtolength{\leftmargin}{\labelsep}
\setlength{\labelwidth}{\tmplength}
}
\begin{flushleft}
\item[\textbf{Declaração}\hfill]
\begin{ttfamily}
public const cmDMX{\_}UnlockData    = cmDMX +  9;\end{ttfamily}


\end{flushleft}
\end{list}
\paragraph*{cmDMX{\_}FixSize}\hspace*{\fill}

\begin{list}{}{
\settowidth{\tmplength}{\textbf{Declaração}}
\setlength{\itemindent}{0cm}
\setlength{\listparindent}{0cm}
\setlength{\leftmargin}{\evensidemargin}
\addtolength{\leftmargin}{\tmplength}
\settowidth{\labelsep}{X}
\addtolength{\leftmargin}{\labelsep}
\setlength{\labelwidth}{\tmplength}
}
\begin{flushleft}
\item[\textbf{Declaração}\hfill]
\begin{ttfamily}
public const cmDMX{\_}FixSize       = cmDMX + 10;\end{ttfamily}


\end{flushleft}
\end{list}
\paragraph*{cmDMX{\_}SetupRecord}\hspace*{\fill}

\begin{list}{}{
\settowidth{\tmplength}{\textbf{Declaração}}
\setlength{\itemindent}{0cm}
\setlength{\listparindent}{0cm}
\setlength{\leftmargin}{\evensidemargin}
\addtolength{\leftmargin}{\tmplength}
\settowidth{\labelsep}{X}
\addtolength{\leftmargin}{\labelsep}
\setlength{\labelwidth}{\tmplength}
}
\begin{flushleft}
\item[\textbf{Declaração}\hfill]
\begin{ttfamily}
public const cmDMX{\_}SetupRecord   = cmDMX + 11;\end{ttfamily}


\end{flushleft}
\end{list}
\paragraph*{cmDMX{\_}WrongKey}\hspace*{\fill}

\begin{list}{}{
\settowidth{\tmplength}{\textbf{Declaração}}
\setlength{\itemindent}{0cm}
\setlength{\listparindent}{0cm}
\setlength{\leftmargin}{\evensidemargin}
\addtolength{\leftmargin}{\tmplength}
\settowidth{\labelsep}{X}
\addtolength{\leftmargin}{\labelsep}
\setlength{\labelwidth}{\tmplength}
}
\begin{flushleft}
\item[\textbf{Declaração}\hfill]
\begin{ttfamily}
public const cmDMX{\_}WrongKey      = cmDMX + 12;\end{ttfamily}


\end{flushleft}
\end{list}
\paragraph*{cmDMX{\_}ZeroizeField}\hspace*{\fill}

\begin{list}{}{
\settowidth{\tmplength}{\textbf{Declaração}}
\setlength{\itemindent}{0cm}
\setlength{\listparindent}{0cm}
\setlength{\leftmargin}{\evensidemargin}
\addtolength{\leftmargin}{\tmplength}
\settowidth{\labelsep}{X}
\addtolength{\leftmargin}{\labelsep}
\setlength{\labelwidth}{\tmplength}
}
\begin{flushleft}
\item[\textbf{Declaração}\hfill]
\begin{ttfamily}
public const cmDMX{\_}ZeroizeField  = cmDMX + 13;\end{ttfamily}


\end{flushleft}
\end{list}
\paragraph*{cmDMX{\_}ZeroizeRecord}\hspace*{\fill}

\begin{list}{}{
\settowidth{\tmplength}{\textbf{Declaração}}
\setlength{\itemindent}{0cm}
\setlength{\listparindent}{0cm}
\setlength{\leftmargin}{\evensidemargin}
\addtolength{\leftmargin}{\tmplength}
\settowidth{\labelsep}{X}
\addtolength{\leftmargin}{\labelsep}
\setlength{\labelwidth}{\tmplength}
}
\begin{flushleft}
\item[\textbf{Declaração}\hfill]
\begin{ttfamily}
public const cmDMX{\_}ZeroizeRecord = cmDMX + 14;\end{ttfamily}


\end{flushleft}
\end{list}
\paragraph*{cmDMX{\_}Enter}\hspace*{\fill}

\begin{list}{}{
\settowidth{\tmplength}{\textbf{Declaração}}
\setlength{\itemindent}{0cm}
\setlength{\listparindent}{0cm}
\setlength{\leftmargin}{\evensidemargin}
\addtolength{\leftmargin}{\tmplength}
\settowidth{\labelsep}{X}
\addtolength{\leftmargin}{\labelsep}
\setlength{\labelwidth}{\tmplength}
}
\begin{flushleft}
\item[\textbf{Declaração}\hfill]
\begin{ttfamily}
public const cmDMX{\_}Enter         = cmDMX + 15;\end{ttfamily}


\end{flushleft}
\end{list}
\paragraph*{cmDMX{\_}Left}\hspace*{\fill}

\begin{list}{}{
\settowidth{\tmplength}{\textbf{Declaração}}
\setlength{\itemindent}{0cm}
\setlength{\listparindent}{0cm}
\setlength{\leftmargin}{\evensidemargin}
\addtolength{\leftmargin}{\tmplength}
\settowidth{\labelsep}{X}
\addtolength{\leftmargin}{\labelsep}
\setlength{\labelwidth}{\tmplength}
}
\begin{flushleft}
\item[\textbf{Declaração}\hfill]
\begin{ttfamily}
public const cmDMX{\_}Left          = cmDMX + 16;\end{ttfamily}


\end{flushleft}
\end{list}
\paragraph*{cmDMX{\_}Right}\hspace*{\fill}

\begin{list}{}{
\settowidth{\tmplength}{\textbf{Declaração}}
\setlength{\itemindent}{0cm}
\setlength{\listparindent}{0cm}
\setlength{\leftmargin}{\evensidemargin}
\addtolength{\leftmargin}{\tmplength}
\settowidth{\labelsep}{X}
\addtolength{\leftmargin}{\labelsep}
\setlength{\labelwidth}{\tmplength}
}
\begin{flushleft}
\item[\textbf{Declaração}\hfill]
\begin{ttfamily}
public const cmDMX{\_}Right         = cmDMX + 17;\end{ttfamily}


\end{flushleft}
\end{list}
\paragraph*{cmDMX{\_}Home}\hspace*{\fill}

\begin{list}{}{
\settowidth{\tmplength}{\textbf{Declaração}}
\setlength{\itemindent}{0cm}
\setlength{\listparindent}{0cm}
\setlength{\leftmargin}{\evensidemargin}
\addtolength{\leftmargin}{\tmplength}
\settowidth{\labelsep}{X}
\addtolength{\leftmargin}{\labelsep}
\setlength{\labelwidth}{\tmplength}
}
\begin{flushleft}
\item[\textbf{Declaração}\hfill]
\begin{ttfamily}
public const cmDMX{\_}Home          = cmDMX + 18;\end{ttfamily}


\end{flushleft}
\end{list}
\paragraph*{cmDMX{\_}End}\hspace*{\fill}

\begin{list}{}{
\settowidth{\tmplength}{\textbf{Declaração}}
\setlength{\itemindent}{0cm}
\setlength{\listparindent}{0cm}
\setlength{\leftmargin}{\evensidemargin}
\addtolength{\leftmargin}{\tmplength}
\settowidth{\labelsep}{X}
\addtolength{\leftmargin}{\labelsep}
\setlength{\labelwidth}{\tmplength}
}
\begin{flushleft}
\item[\textbf{Declaração}\hfill]
\begin{ttfamily}
public const cmDMX{\_}End           = cmDMX + 19;\end{ttfamily}


\end{flushleft}
\end{list}
\paragraph*{cmDMX{\_}goto}\hspace*{\fill}

\begin{list}{}{
\settowidth{\tmplength}{\textbf{Declaração}}
\setlength{\itemindent}{0cm}
\setlength{\listparindent}{0cm}
\setlength{\leftmargin}{\evensidemargin}
\addtolength{\leftmargin}{\tmplength}
\settowidth{\labelsep}{X}
\addtolength{\leftmargin}{\labelsep}
\setlength{\labelwidth}{\tmplength}
}
\begin{flushleft}
\item[\textbf{Declaração}\hfill]
\begin{ttfamily}
public const cmDMX{\_}goto          = cmDMX + 20;\end{ttfamily}


\end{flushleft}
\end{list}
\paragraph*{cmDMX{\_}NextRow}\hspace*{\fill}

\begin{list}{}{
\settowidth{\tmplength}{\textbf{Declaração}}
\setlength{\itemindent}{0cm}
\setlength{\listparindent}{0cm}
\setlength{\leftmargin}{\evensidemargin}
\addtolength{\leftmargin}{\tmplength}
\settowidth{\labelsep}{X}
\addtolength{\leftmargin}{\labelsep}
\setlength{\labelwidth}{\tmplength}
}
\begin{flushleft}
\item[\textbf{Declaração}\hfill]
\begin{ttfamily}
public const cmDMX{\_}NextRow       = cmDMX + 21;\end{ttfamily}


\end{flushleft}
\end{list}
\paragraph*{cmDMX{\_}Up}\hspace*{\fill}

\begin{list}{}{
\settowidth{\tmplength}{\textbf{Declaração}}
\setlength{\itemindent}{0cm}
\setlength{\listparindent}{0cm}
\setlength{\leftmargin}{\evensidemargin}
\addtolength{\leftmargin}{\tmplength}
\settowidth{\labelsep}{X}
\addtolength{\leftmargin}{\labelsep}
\setlength{\labelwidth}{\tmplength}
}
\begin{flushleft}
\item[\textbf{Declaração}\hfill]
\begin{ttfamily}
public const cmDMX{\_}Up            = cmDMX + 22;\end{ttfamily}


\end{flushleft}
\end{list}
\paragraph*{cmDMX{\_}Down}\hspace*{\fill}

\begin{list}{}{
\settowidth{\tmplength}{\textbf{Declaração}}
\setlength{\itemindent}{0cm}
\setlength{\listparindent}{0cm}
\setlength{\leftmargin}{\evensidemargin}
\addtolength{\leftmargin}{\tmplength}
\settowidth{\labelsep}{X}
\addtolength{\leftmargin}{\labelsep}
\setlength{\labelwidth}{\tmplength}
}
\begin{flushleft}
\item[\textbf{Declaração}\hfill]
\begin{ttfamily}
public const cmDMX{\_}Down          = cmDMX + 23;\end{ttfamily}


\end{flushleft}
\end{list}
\paragraph*{cmDMX{\_}PgUp}\hspace*{\fill}

\begin{list}{}{
\settowidth{\tmplength}{\textbf{Declaração}}
\setlength{\itemindent}{0cm}
\setlength{\listparindent}{0cm}
\setlength{\leftmargin}{\evensidemargin}
\addtolength{\leftmargin}{\tmplength}
\settowidth{\labelsep}{X}
\addtolength{\leftmargin}{\labelsep}
\setlength{\labelwidth}{\tmplength}
}
\begin{flushleft}
\item[\textbf{Declaração}\hfill]
\begin{ttfamily}
public const cmDMX{\_}PgUp          = cmDMX + 24;\end{ttfamily}


\end{flushleft}
\end{list}
\paragraph*{cmDMX{\_}PgDn}\hspace*{\fill}

\begin{list}{}{
\settowidth{\tmplength}{\textbf{Declaração}}
\setlength{\itemindent}{0cm}
\setlength{\listparindent}{0cm}
\setlength{\leftmargin}{\evensidemargin}
\addtolength{\leftmargin}{\tmplength}
\settowidth{\labelsep}{X}
\addtolength{\leftmargin}{\labelsep}
\setlength{\labelwidth}{\tmplength}
}
\begin{flushleft}
\item[\textbf{Declaração}\hfill]
\begin{ttfamily}
public const cmDMX{\_}PgDn          = cmDMX + 25;\end{ttfamily}


\end{flushleft}
\end{list}
\paragraph*{cmDMX{\_}ScreenTop}\hspace*{\fill}

\begin{list}{}{
\settowidth{\tmplength}{\textbf{Declaração}}
\setlength{\itemindent}{0cm}
\setlength{\listparindent}{0cm}
\setlength{\leftmargin}{\evensidemargin}
\addtolength{\leftmargin}{\tmplength}
\settowidth{\labelsep}{X}
\addtolength{\leftmargin}{\labelsep}
\setlength{\labelwidth}{\tmplength}
}
\begin{flushleft}
\item[\textbf{Declaração}\hfill]
\begin{ttfamily}
public const cmDMX{\_}ScreenTop     = cmDMX + 26;\end{ttfamily}


\end{flushleft}
\end{list}
\paragraph*{cmDMX{\_}ScreenBottom}\hspace*{\fill}

\begin{list}{}{
\settowidth{\tmplength}{\textbf{Declaração}}
\setlength{\itemindent}{0cm}
\setlength{\listparindent}{0cm}
\setlength{\leftmargin}{\evensidemargin}
\addtolength{\leftmargin}{\tmplength}
\settowidth{\labelsep}{X}
\addtolength{\leftmargin}{\labelsep}
\setlength{\labelwidth}{\tmplength}
}
\begin{flushleft}
\item[\textbf{Declaração}\hfill]
\begin{ttfamily}
public const cmDMX{\_}ScreenBottom  = cmDMX + 27;\end{ttfamily}


\end{flushleft}
\end{list}
\paragraph*{cmDMX{\_}Top}\hspace*{\fill}

\begin{list}{}{
\settowidth{\tmplength}{\textbf{Declaração}}
\setlength{\itemindent}{0cm}
\setlength{\listparindent}{0cm}
\setlength{\leftmargin}{\evensidemargin}
\addtolength{\leftmargin}{\tmplength}
\settowidth{\labelsep}{X}
\addtolength{\leftmargin}{\labelsep}
\setlength{\labelwidth}{\tmplength}
}
\begin{flushleft}
\item[\textbf{Declaração}\hfill]
\begin{ttfamily}
public const cmDMX{\_}Top           = cmDMX + 28;\end{ttfamily}


\end{flushleft}
\end{list}
\paragraph*{cmDMX{\_}Bottom}\hspace*{\fill}

\begin{list}{}{
\settowidth{\tmplength}{\textbf{Declaração}}
\setlength{\itemindent}{0cm}
\setlength{\listparindent}{0cm}
\setlength{\leftmargin}{\evensidemargin}
\addtolength{\leftmargin}{\tmplength}
\settowidth{\labelsep}{X}
\addtolength{\leftmargin}{\labelsep}
\setlength{\labelwidth}{\tmplength}
}
\begin{flushleft}
\item[\textbf{Declaração}\hfill]
\begin{ttfamily}
public const cmDMX{\_}Bottom        = cmDMX + 29;\end{ttfamily}


\end{flushleft}
\end{list}
\paragraph*{cmDMX{\_}DoubleClick}\hspace*{\fill}

\begin{list}{}{
\settowidth{\tmplength}{\textbf{Declaração}}
\setlength{\itemindent}{0cm}
\setlength{\listparindent}{0cm}
\setlength{\leftmargin}{\evensidemargin}
\addtolength{\leftmargin}{\tmplength}
\settowidth{\labelsep}{X}
\addtolength{\leftmargin}{\labelsep}
\setlength{\labelwidth}{\tmplength}
}
\begin{flushleft}
\item[\textbf{Declaração}\hfill]
\begin{ttfamily}
public const cmDMX{\_}DoubleClick   = cmDMX + 30;\end{ttfamily}


\end{flushleft}
\end{list}
\paragraph*{cmDMX{\_}RecIndClicked}\hspace*{\fill}

\begin{list}{}{
\settowidth{\tmplength}{\textbf{Declaração}}
\setlength{\itemindent}{0cm}
\setlength{\listparindent}{0cm}
\setlength{\leftmargin}{\evensidemargin}
\addtolength{\leftmargin}{\tmplength}
\settowidth{\labelsep}{X}
\addtolength{\leftmargin}{\labelsep}
\setlength{\labelwidth}{\tmplength}
}
\begin{flushleft}
\item[\textbf{Declaração}\hfill]
\begin{ttfamily}
public const cmDMX{\_}RecIndClicked = cmDMX + 31;\end{ttfamily}


\end{flushleft}
\end{list}
\paragraph*{cmDMX{\_}Reset}\hspace*{\fill}

\begin{list}{}{
\settowidth{\tmplength}{\textbf{Declaração}}
\setlength{\itemindent}{0cm}
\setlength{\listparindent}{0cm}
\setlength{\leftmargin}{\evensidemargin}
\addtolength{\leftmargin}{\tmplength}
\settowidth{\labelsep}{X}
\addtolength{\leftmargin}{\labelsep}
\setlength{\labelwidth}{\tmplength}
}
\begin{flushleft}
\item[\textbf{Declaração}\hfill]
\begin{ttfamily}
public const cmDMX{\_}Reset         = cmDMX + 32;\end{ttfamily}


\end{flushleft}
\end{list}
\paragraph*{cmDMX{\_}ScrollBarChanged}\hspace*{\fill}

\begin{list}{}{
\settowidth{\tmplength}{\textbf{Declaração}}
\setlength{\itemindent}{0cm}
\setlength{\listparindent}{0cm}
\setlength{\leftmargin}{\evensidemargin}
\addtolength{\leftmargin}{\tmplength}
\settowidth{\labelsep}{X}
\addtolength{\leftmargin}{\labelsep}
\setlength{\labelwidth}{\tmplength}
}
\begin{flushleft}
\item[\textbf{Declaração}\hfill]
\begin{ttfamily}
public const cmDMX{\_}ScrollBarChanged =cmDMX+33;\end{ttfamily}


\end{flushleft}
\end{list}
\paragraph*{cmDMX{\_}InsertRec}\hspace*{\fill}

\begin{list}{}{
\settowidth{\tmplength}{\textbf{Declaração}}
\setlength{\itemindent}{0cm}
\setlength{\listparindent}{0cm}
\setlength{\leftmargin}{\evensidemargin}
\addtolength{\leftmargin}{\tmplength}
\settowidth{\labelsep}{X}
\addtolength{\leftmargin}{\labelsep}
\setlength{\labelwidth}{\tmplength}
}
\begin{flushleft}
\item[\textbf{Declaração}\hfill]
\begin{ttfamily}
public const cmDMX{\_}InsertRec     = cmDMX + 34;\end{ttfamily}


\end{flushleft}
\end{list}
\paragraph*{cmPRN{\_}NewPage}\hspace*{\fill}

\begin{list}{}{
\settowidth{\tmplength}{\textbf{Declaração}}
\setlength{\itemindent}{0cm}
\setlength{\listparindent}{0cm}
\setlength{\leftmargin}{\evensidemargin}
\addtolength{\leftmargin}{\tmplength}
\settowidth{\labelsep}{X}
\addtolength{\leftmargin}{\labelsep}
\setlength{\labelwidth}{\tmplength}
}
\begin{flushleft}
\item[\textbf{Declaração}\hfill]
\begin{ttfamily}
public const cmPRN{\_}NewPage       = cmDMX + 40;\end{ttfamily}


\end{flushleft}
\end{list}
\paragraph*{cmPRN{\_}EndPage}\hspace*{\fill}

\begin{list}{}{
\settowidth{\tmplength}{\textbf{Declaração}}
\setlength{\itemindent}{0cm}
\setlength{\listparindent}{0cm}
\setlength{\leftmargin}{\evensidemargin}
\addtolength{\leftmargin}{\tmplength}
\settowidth{\labelsep}{X}
\addtolength{\leftmargin}{\labelsep}
\setlength{\labelwidth}{\tmplength}
}
\begin{flushleft}
\item[\textbf{Declaração}\hfill]
\begin{ttfamily}
public const cmPRN{\_}EndPage       = cmDMX + 41;\end{ttfamily}


\end{flushleft}
\end{list}
\paragraph*{cmPRN{\_}SetOptions}\hspace*{\fill}

\begin{list}{}{
\settowidth{\tmplength}{\textbf{Declaração}}
\setlength{\itemindent}{0cm}
\setlength{\listparindent}{0cm}
\setlength{\leftmargin}{\evensidemargin}
\addtolength{\leftmargin}{\tmplength}
\settowidth{\labelsep}{X}
\addtolength{\leftmargin}{\labelsep}
\setlength{\labelwidth}{\tmplength}
}
\begin{flushleft}
\item[\textbf{Declaração}\hfill]
\begin{ttfamily}
public const cmPRN{\_}SetOptions    = cmDMX + 42;\end{ttfamily}


\end{flushleft}
\end{list}
\paragraph*{cmPRN{\_}LineFeed}\hspace*{\fill}

\begin{list}{}{
\settowidth{\tmplength}{\textbf{Declaração}}
\setlength{\itemindent}{0cm}
\setlength{\listparindent}{0cm}
\setlength{\leftmargin}{\evensidemargin}
\addtolength{\leftmargin}{\tmplength}
\settowidth{\labelsep}{X}
\addtolength{\leftmargin}{\labelsep}
\setlength{\labelwidth}{\tmplength}
}
\begin{flushleft}
\item[\textbf{Declaração}\hfill]
\begin{ttfamily}
public const cmPRN{\_}LineFeed      = cmDMX + 43;\end{ttfamily}


\end{flushleft}
\end{list}
\paragraph*{cmPRN{\_}FormFeed}\hspace*{\fill}

\begin{list}{}{
\settowidth{\tmplength}{\textbf{Declaração}}
\setlength{\itemindent}{0cm}
\setlength{\listparindent}{0cm}
\setlength{\leftmargin}{\evensidemargin}
\addtolength{\leftmargin}{\tmplength}
\settowidth{\labelsep}{X}
\addtolength{\leftmargin}{\labelsep}
\setlength{\labelwidth}{\tmplength}
}
\begin{flushleft}
\item[\textbf{Declaração}\hfill]
\begin{ttfamily}
public const cmPRN{\_}FormFeed      = cmDMX + 44;\end{ttfamily}


\end{flushleft}
\end{list}
\paragraph*{cmPRN{\_}Reset}\hspace*{\fill}

\begin{list}{}{
\settowidth{\tmplength}{\textbf{Declaração}}
\setlength{\itemindent}{0cm}
\setlength{\listparindent}{0cm}
\setlength{\leftmargin}{\evensidemargin}
\addtolength{\leftmargin}{\tmplength}
\settowidth{\labelsep}{X}
\addtolength{\leftmargin}{\labelsep}
\setlength{\labelwidth}{\tmplength}
}
\begin{flushleft}
\item[\textbf{Declaração}\hfill]
\begin{ttfamily}
public const cmPRN{\_}Reset         = cmDMX + 45;\end{ttfamily}


\end{flushleft}
\end{list}
\paragraph*{cmUserScreen}\hspace*{\fill}

\begin{list}{}{
\settowidth{\tmplength}{\textbf{Declaração}}
\setlength{\itemindent}{0cm}
\setlength{\listparindent}{0cm}
\setlength{\leftmargin}{\evensidemargin}
\addtolength{\leftmargin}{\tmplength}
\settowidth{\labelsep}{X}
\addtolength{\leftmargin}{\labelsep}
\setlength{\labelwidth}{\tmplength}
}
\begin{flushleft}
\item[\textbf{Declaração}\hfill]
\begin{ttfamily}
public const cmUserScreen        = cmDMX + 51;\end{ttfamily}


\end{flushleft}
\end{list}
\paragraph*{cmToggleSound}\hspace*{\fill}

\begin{list}{}{
\settowidth{\tmplength}{\textbf{Declaração}}
\setlength{\itemindent}{0cm}
\setlength{\listparindent}{0cm}
\setlength{\leftmargin}{\evensidemargin}
\addtolength{\leftmargin}{\tmplength}
\settowidth{\labelsep}{X}
\addtolength{\leftmargin}{\labelsep}
\setlength{\labelwidth}{\tmplength}
}
\begin{flushleft}
\item[\textbf{Declaração}\hfill]
\begin{ttfamily}
public const cmToggleSound       = cmDMX + 52;\end{ttfamily}


\end{flushleft}
\end{list}
\paragraph*{cmToggleVideo}\hspace*{\fill}

\begin{list}{}{
\settowidth{\tmplength}{\textbf{Declaração}}
\setlength{\itemindent}{0cm}
\setlength{\listparindent}{0cm}
\setlength{\leftmargin}{\evensidemargin}
\addtolength{\leftmargin}{\tmplength}
\settowidth{\labelsep}{X}
\addtolength{\leftmargin}{\labelsep}
\setlength{\labelwidth}{\tmplength}
}
\begin{flushleft}
\item[\textbf{Declaração}\hfill]
\begin{ttfamily}
public const cmToggleVideo       = cmDMX + 53;\end{ttfamily}


\end{flushleft}
\end{list}
\paragraph*{cmBeep}\hspace*{\fill}

\begin{list}{}{
\settowidth{\tmplength}{\textbf{Declaração}}
\setlength{\itemindent}{0cm}
\setlength{\listparindent}{0cm}
\setlength{\leftmargin}{\evensidemargin}
\addtolength{\leftmargin}{\tmplength}
\settowidth{\labelsep}{X}
\addtolength{\leftmargin}{\labelsep}
\setlength{\labelwidth}{\tmplength}
}
\begin{flushleft}
\item[\textbf{Declaração}\hfill]
\begin{ttfamily}
public const cmBeep              = cmDMX + 54;\end{ttfamily}


\end{flushleft}
\end{list}
\paragraph*{cmChime}\hspace*{\fill}

\begin{list}{}{
\settowidth{\tmplength}{\textbf{Declaração}}
\setlength{\itemindent}{0cm}
\setlength{\listparindent}{0cm}
\setlength{\leftmargin}{\evensidemargin}
\addtolength{\leftmargin}{\tmplength}
\settowidth{\labelsep}{X}
\addtolength{\leftmargin}{\labelsep}
\setlength{\labelwidth}{\tmplength}
}
\begin{flushleft}
\item[\textbf{Declaração}\hfill]
\begin{ttfamily}
public const cmChime             = cmDMX + 55;\end{ttfamily}


\end{flushleft}
\end{list}
\paragraph*{cmPromptMsg}\hspace*{\fill}

\begin{list}{}{
\settowidth{\tmplength}{\textbf{Declaração}}
\setlength{\itemindent}{0cm}
\setlength{\listparindent}{0cm}
\setlength{\leftmargin}{\evensidemargin}
\addtolength{\leftmargin}{\tmplength}
\settowidth{\labelsep}{X}
\addtolength{\leftmargin}{\labelsep}
\setlength{\labelwidth}{\tmplength}
}
\begin{flushleft}
\item[\textbf{Declaração}\hfill]
\begin{ttfamily}
public const cmPromptMsg         = cmDMX + 56;\end{ttfamily}


\end{flushleft}
\end{list}
\paragraph*{cmBlinkMsg}\hspace*{\fill}

\begin{list}{}{
\settowidth{\tmplength}{\textbf{Declaração}}
\setlength{\itemindent}{0cm}
\setlength{\listparindent}{0cm}
\setlength{\leftmargin}{\evensidemargin}
\addtolength{\leftmargin}{\tmplength}
\settowidth{\labelsep}{X}
\addtolength{\leftmargin}{\labelsep}
\setlength{\labelwidth}{\tmplength}
}
\begin{flushleft}
\item[\textbf{Declaração}\hfill]
\begin{ttfamily}
public const cmBlinkMsg          = cmDMX + 57;\end{ttfamily}


\end{flushleft}
\end{list}
\paragraph*{cmDbMX{\_}GetBuffer}\hspace*{\fill}

\begin{list}{}{
\settowidth{\tmplength}{\textbf{Declaração}}
\setlength{\itemindent}{0cm}
\setlength{\listparindent}{0cm}
\setlength{\leftmargin}{\evensidemargin}
\addtolength{\leftmargin}{\tmplength}
\settowidth{\labelsep}{X}
\addtolength{\leftmargin}{\labelsep}
\setlength{\labelwidth}{\tmplength}
}
\begin{flushleft}
\item[\textbf{Declaração}\hfill]
\begin{ttfamily}
public const cmDbMX{\_}GetBuffer    = cmDMX + 58;\end{ttfamily}


\end{flushleft}
\end{list}
\paragraph*{cmDbMX{\_}PutBuffer}\hspace*{\fill}

\begin{list}{}{
\settowidth{\tmplength}{\textbf{Declaração}}
\setlength{\itemindent}{0cm}
\setlength{\listparindent}{0cm}
\setlength{\leftmargin}{\evensidemargin}
\addtolength{\leftmargin}{\tmplength}
\settowidth{\labelsep}{X}
\addtolength{\leftmargin}{\labelsep}
\setlength{\labelwidth}{\tmplength}
}
\begin{flushleft}
\item[\textbf{Declaração}\hfill]
\begin{ttfamily}
public const cmDbMX{\_}PutBuffer    = cmDMX + 59;\end{ttfamily}


\end{flushleft}
\end{list}
\paragraph*{DirectorySeparator}\hspace*{\fill}

\begin{list}{}{
\settowidth{\tmplength}{\textbf{Declaração}}
\setlength{\itemindent}{0cm}
\setlength{\listparindent}{0cm}
\setlength{\leftmargin}{\evensidemargin}
\addtolength{\leftmargin}{\tmplength}
\settowidth{\labelsep}{X}
\addtolength{\leftmargin}{\labelsep}
\setlength{\labelwidth}{\tmplength}
}
\begin{flushleft}
\item[\textbf{Declaração}\hfill]
\begin{ttfamily}
public const DirectorySeparator :char = system.DirectorySeparator;\end{ttfamily}


\end{flushleft}
\par
\item[\textbf{Descrição}]
A contante \textbf{\begin{ttfamily}DirectorySeparator\end{ttfamily}} contém o caractere separador de diretório.

\end{list}
\paragraph*{Lst}\hspace*{\fill}

\begin{list}{}{
\settowidth{\tmplength}{\textbf{Declaração}}
\setlength{\itemindent}{0cm}
\setlength{\listparindent}{0cm}
\setlength{\leftmargin}{\evensidemargin}
\addtolength{\leftmargin}{\tmplength}
\settowidth{\labelsep}{X}
\addtolength{\leftmargin}{\labelsep}
\setlength{\labelwidth}{\tmplength}
}
\begin{flushleft}
\item[\textbf{Declaração}\hfill]
\begin{ttfamily}
public var Lst: text ; static;\end{ttfamily}


\end{flushleft}
\end{list}
\paragraph*{onProcessMessages}\hspace*{\fill}

\begin{list}{}{
\settowidth{\tmplength}{\textbf{Declaração}}
\setlength{\itemindent}{0cm}
\setlength{\listparindent}{0cm}
\setlength{\leftmargin}{\evensidemargin}
\addtolength{\leftmargin}{\tmplength}
\settowidth{\labelsep}{X}
\addtolength{\leftmargin}{\labelsep}
\setlength{\labelwidth}{\tmplength}
}
\begin{flushleft}
\item[\textbf{Declaração}\hfill]
\begin{ttfamily}
public const onProcessMessages : TOnProcedure = nil;\end{ttfamily}


\end{flushleft}
\par
\item[\textbf{Descrição}]
\begin{itemize}
\item O evento \begin{ttfamily}onProcessMessages\end{ttfamily} é executado em CtrlSleep e deve ser iniciado para que possa processar as mensagens dos widgets que usão essa classe.
\end{itemize}

\end{list}
\paragraph*{kbNoKey}\hspace*{\fill}

\begin{list}{}{
\settowidth{\tmplength}{\textbf{Declaração}}
\setlength{\itemindent}{0cm}
\setlength{\listparindent}{0cm}
\setlength{\leftmargin}{\evensidemargin}
\addtolength{\leftmargin}{\tmplength}
\settowidth{\labelsep}{X}
\addtolength{\leftmargin}{\labelsep}
\setlength{\labelwidth}{\tmplength}
}
\begin{flushleft}
\item[\textbf{Declaração}\hfill]
\begin{ttfamily}
public const kbNoKey = 0;\end{ttfamily}


\end{flushleft}
\end{list}
\paragraph*{MessageBoxOff}\hspace*{\fill}

\begin{list}{}{
\settowidth{\tmplength}{\textbf{Declaração}}
\setlength{\itemindent}{0cm}
\setlength{\listparindent}{0cm}
\setlength{\leftmargin}{\evensidemargin}
\addtolength{\leftmargin}{\tmplength}
\settowidth{\labelsep}{X}
\addtolength{\leftmargin}{\labelsep}
\setlength{\labelwidth}{\tmplength}
}
\begin{flushleft}
\item[\textbf{Declaração}\hfill]
\begin{ttfamily}
public const MessageBoxOff    : Boolean = false;\end{ttfamily}


\end{flushleft}
\par
\item[\textbf{Descrição}]
Se \begin{ttfamily}MessageBoxOff\end{ttfamily} = true então não mostra o dialogo e torna o comando defaust

\begin{itemize}
\item Usada quando se quer despresar a ação do usuário e que ler os erros de um arquivo de erros. Normalmente deve ser usado nos programas controlados em linha de comando.
\end{itemize}

\end{list}
\subsubsection*{\large{\textbf{Métodos}}\normalsize\hspace{1ex}\hfill}
\paragraph*{CreateEnumField}\hspace*{\fill}

\begin{list}{}{
\settowidth{\tmplength}{\textbf{Declaração}}
\setlength{\itemindent}{0cm}
\setlength{\listparindent}{0cm}
\setlength{\leftmargin}{\evensidemargin}
\addtolength{\leftmargin}{\tmplength}
\settowidth{\labelsep}{X}
\addtolength{\leftmargin}{\labelsep}
\setlength{\labelwidth}{\tmplength}
}
\begin{flushleft}
\item[\textbf{Declaração}\hfill]
\begin{ttfamily}
public class function CreateEnumField(ShowZ: boolean; AccMode,Default: LongInt;AItems: PSItem) : tString;\end{ttfamily}


\end{flushleft}
\end{list}
\paragraph*{CreateTSItemFields}\hspace*{\fill}

\begin{list}{}{
\settowidth{\tmplength}{\textbf{Declaração}}
\setlength{\itemindent}{0cm}
\setlength{\listparindent}{0cm}
\setlength{\leftmargin}{\evensidemargin}
\addtolength{\leftmargin}{\tmplength}
\settowidth{\labelsep}{X}
\addtolength{\leftmargin}{\labelsep}
\setlength{\labelwidth}{\tmplength}
}
\begin{flushleft}
\item[\textbf{Declaração}\hfill]
\begin{ttfamily}
public class function CreateTSItemFields(ATemplates: PSItem) : tString;\end{ttfamily}


\end{flushleft}
\end{list}
\section{Constantes}
\subsection*{SCmDbNextRec}
\begin{list}{}{
\settowidth{\tmplength}{\textbf{Declaração}}
\setlength{\itemindent}{0cm}
\setlength{\listparindent}{0cm}
\setlength{\leftmargin}{\evensidemargin}
\addtolength{\leftmargin}{\tmplength}
\settowidth{\labelsep}{X}
\addtolength{\leftmargin}{\labelsep}
\setlength{\labelwidth}{\tmplength}
}
\begin{flushleft}
\item[\textbf{Declaração}\hfill]
\begin{ttfamily}
SCmDbNextRec          = 'Próximo registro';\end{ttfamily}


\end{flushleft}
\end{list}
\subsection*{SCmDbPrevRec}
\begin{list}{}{
\settowidth{\tmplength}{\textbf{Declaração}}
\setlength{\itemindent}{0cm}
\setlength{\listparindent}{0cm}
\setlength{\leftmargin}{\evensidemargin}
\addtolength{\leftmargin}{\tmplength}
\settowidth{\labelsep}{X}
\addtolength{\leftmargin}{\labelsep}
\setlength{\labelwidth}{\tmplength}
}
\begin{flushleft}
\item[\textbf{Declaração}\hfill]
\begin{ttfamily}
SCmDbPrevRec          = 'Registro Anterior';\end{ttfamily}


\end{flushleft}
\end{list}
\subsection*{SCmDbNextRecValid}
\begin{list}{}{
\settowidth{\tmplength}{\textbf{Declaração}}
\setlength{\itemindent}{0cm}
\setlength{\listparindent}{0cm}
\setlength{\leftmargin}{\evensidemargin}
\addtolength{\leftmargin}{\tmplength}
\settowidth{\labelsep}{X}
\addtolength{\leftmargin}{\labelsep}
\setlength{\labelwidth}{\tmplength}
}
\begin{flushleft}
\item[\textbf{Declaração}\hfill]
\begin{ttfamily}
SCmDbNextRecValid     = 'Próximo registro válido';\end{ttfamily}


\end{flushleft}
\end{list}
\subsection*{SCmDbPrevRecValid}
\begin{list}{}{
\settowidth{\tmplength}{\textbf{Declaração}}
\setlength{\itemindent}{0cm}
\setlength{\listparindent}{0cm}
\setlength{\leftmargin}{\evensidemargin}
\addtolength{\leftmargin}{\tmplength}
\settowidth{\labelsep}{X}
\addtolength{\leftmargin}{\labelsep}
\setlength{\labelwidth}{\tmplength}
}
\begin{flushleft}
\item[\textbf{Declaração}\hfill]
\begin{ttfamily}
SCmDbPrevRecValid     = 'Registro válido anterior';\end{ttfamily}


\end{flushleft}
\end{list}
\subsection*{SCmDbFindRec}
\begin{list}{}{
\settowidth{\tmplength}{\textbf{Declaração}}
\setlength{\itemindent}{0cm}
\setlength{\listparindent}{0cm}
\setlength{\leftmargin}{\evensidemargin}
\addtolength{\leftmargin}{\tmplength}
\settowidth{\labelsep}{X}
\addtolength{\leftmargin}{\labelsep}
\setlength{\labelwidth}{\tmplength}
}
\begin{flushleft}
\item[\textbf{Declaração}\hfill]
\begin{ttfamily}
SCmDbFindRec          = 'Atualiza o registro atual';\end{ttfamily}


\end{flushleft}
\end{list}
\subsection*{SCmDbSearchRec}
\begin{list}{}{
\settowidth{\tmplength}{\textbf{Declaração}}
\setlength{\itemindent}{0cm}
\setlength{\listparindent}{0cm}
\setlength{\leftmargin}{\evensidemargin}
\addtolength{\leftmargin}{\tmplength}
\settowidth{\labelsep}{X}
\addtolength{\leftmargin}{\labelsep}
\setlength{\labelwidth}{\tmplength}
}
\begin{flushleft}
\item[\textbf{Declaração}\hfill]
\begin{ttfamily}
SCmDbSearchRec        = 'SCmDbSearchRec';\end{ttfamily}


\end{flushleft}
\end{list}
\subsection*{SCmDbGoEof}
\begin{list}{}{
\settowidth{\tmplength}{\textbf{Declaração}}
\setlength{\itemindent}{0cm}
\setlength{\listparindent}{0cm}
\setlength{\leftmargin}{\evensidemargin}
\addtolength{\leftmargin}{\tmplength}
\settowidth{\labelsep}{X}
\addtolength{\leftmargin}{\labelsep}
\setlength{\labelwidth}{\tmplength}
}
\begin{flushleft}
\item[\textbf{Declaração}\hfill]
\begin{ttfamily}
SCmDbGoEof            = 'Último registro';\end{ttfamily}


\end{flushleft}
\end{list}
\subsection*{SCmDbGoBof}
\begin{list}{}{
\settowidth{\tmplength}{\textbf{Declaração}}
\setlength{\itemindent}{0cm}
\setlength{\listparindent}{0cm}
\setlength{\leftmargin}{\evensidemargin}
\addtolength{\leftmargin}{\tmplength}
\settowidth{\labelsep}{X}
\addtolength{\leftmargin}{\labelsep}
\setlength{\labelwidth}{\tmplength}
}
\begin{flushleft}
\item[\textbf{Declaração}\hfill]
\begin{ttfamily}
SCmDbGoBof            = 'Primeiro registro';\end{ttfamily}


\end{flushleft}
\end{list}
\subsection*{SCmDbLocaliza}
\begin{list}{}{
\settowidth{\tmplength}{\textbf{Declaração}}
\setlength{\itemindent}{0cm}
\setlength{\listparindent}{0cm}
\setlength{\leftmargin}{\evensidemargin}
\addtolength{\leftmargin}{\tmplength}
\settowidth{\labelsep}{X}
\addtolength{\leftmargin}{\labelsep}
\setlength{\labelwidth}{\tmplength}
}
\begin{flushleft}
\item[\textbf{Declaração}\hfill]
\begin{ttfamily}
SCmDbLocaliza         = 'Localiza registro';\end{ttfamily}


\end{flushleft}
\end{list}
\subsection*{SCmNewRecord}
\begin{list}{}{
\settowidth{\tmplength}{\textbf{Declaração}}
\setlength{\itemindent}{0cm}
\setlength{\listparindent}{0cm}
\setlength{\leftmargin}{\evensidemargin}
\addtolength{\leftmargin}{\tmplength}
\settowidth{\labelsep}{X}
\addtolength{\leftmargin}{\labelsep}
\setlength{\labelwidth}{\tmplength}
}
\begin{flushleft}
\item[\textbf{Declaração}\hfill]
\begin{ttfamily}
SCmNewRecord          = 'Novo registro';\end{ttfamily}


\end{flushleft}
\end{list}
\subsection*{SCmZeroizeRecord}
\begin{list}{}{
\settowidth{\tmplength}{\textbf{Declaração}}
\setlength{\itemindent}{0cm}
\setlength{\listparindent}{0cm}
\setlength{\leftmargin}{\evensidemargin}
\addtolength{\leftmargin}{\tmplength}
\settowidth{\labelsep}{X}
\addtolength{\leftmargin}{\labelsep}
\setlength{\labelwidth}{\tmplength}
}
\begin{flushleft}
\item[\textbf{Declaração}\hfill]
\begin{ttfamily}
SCmZeroizeRecord      = 'Apaga o registro atual';\end{ttfamily}


\end{flushleft}
\end{list}
\subsection*{SCmEvaluateRecord}
\begin{list}{}{
\settowidth{\tmplength}{\textbf{Declaração}}
\setlength{\itemindent}{0cm}
\setlength{\listparindent}{0cm}
\setlength{\leftmargin}{\evensidemargin}
\addtolength{\leftmargin}{\tmplength}
\settowidth{\labelsep}{X}
\addtolength{\leftmargin}{\labelsep}
\setlength{\labelwidth}{\tmplength}
}
\begin{flushleft}
\item[\textbf{Declaração}\hfill]
\begin{ttfamily}
SCmEvaluateRecord     = 'Grava o registro atual';\end{ttfamily}


\end{flushleft}
\end{list}
\subsection*{SCmEditDlg}
\begin{list}{}{
\settowidth{\tmplength}{\textbf{Declaração}}
\setlength{\itemindent}{0cm}
\setlength{\listparindent}{0cm}
\setlength{\leftmargin}{\evensidemargin}
\addtolength{\leftmargin}{\tmplength}
\settowidth{\labelsep}{X}
\addtolength{\leftmargin}{\labelsep}
\setlength{\labelwidth}{\tmplength}
}
\begin{flushleft}
\item[\textbf{Declaração}\hfill]
\begin{ttfamily}
SCmEditDlg            = 'Edita o registro atual';\end{ttfamily}


\end{flushleft}
\end{list}
\subsection*{ScmMyOK}
\begin{list}{}{
\settowidth{\tmplength}{\textbf{Declaração}}
\setlength{\itemindent}{0cm}
\setlength{\listparindent}{0cm}
\setlength{\leftmargin}{\evensidemargin}
\addtolength{\leftmargin}{\tmplength}
\settowidth{\labelsep}{X}
\addtolength{\leftmargin}{\labelsep}
\setlength{\labelwidth}{\tmplength}
}
\begin{flushleft}
\item[\textbf{Declaração}\hfill]
\begin{ttfamily}
ScmMyOK               = 'Ok';\end{ttfamily}


\end{flushleft}
\end{list}
\subsection*{ScmMyCancel}
\begin{list}{}{
\settowidth{\tmplength}{\textbf{Declaração}}
\setlength{\itemindent}{0cm}
\setlength{\listparindent}{0cm}
\setlength{\leftmargin}{\evensidemargin}
\addtolength{\leftmargin}{\tmplength}
\settowidth{\labelsep}{X}
\addtolength{\leftmargin}{\labelsep}
\setlength{\labelwidth}{\tmplength}
}
\begin{flushleft}
\item[\textbf{Declaração}\hfill]
\begin{ttfamily}
ScmMyCancel           = 'Cancelar';\end{ttfamily}


\end{flushleft}
\end{list}
\subsection*{ScmPrint}
\begin{list}{}{
\settowidth{\tmplength}{\textbf{Declaração}}
\setlength{\itemindent}{0cm}
\setlength{\listparindent}{0cm}
\setlength{\leftmargin}{\evensidemargin}
\addtolength{\leftmargin}{\tmplength}
\settowidth{\labelsep}{X}
\addtolength{\leftmargin}{\labelsep}
\setlength{\labelwidth}{\tmplength}
}
\begin{flushleft}
\item[\textbf{Declaração}\hfill]
\begin{ttfamily}
ScmPrint              = 'Imprimir';\end{ttfamily}


\end{flushleft}
\end{list}
\subsection*{SCmImport}
\begin{list}{}{
\settowidth{\tmplength}{\textbf{Declaração}}
\setlength{\itemindent}{0cm}
\setlength{\listparindent}{0cm}
\setlength{\leftmargin}{\evensidemargin}
\addtolength{\leftmargin}{\tmplength}
\settowidth{\labelsep}{X}
\addtolength{\leftmargin}{\labelsep}
\setlength{\labelwidth}{\tmplength}
}
\begin{flushleft}
\item[\textbf{Declaração}\hfill]
\begin{ttfamily}
SCmImport             = 'Importar';\end{ttfamily}


\end{flushleft}
\end{list}
\subsection*{SCmProcess}
\begin{list}{}{
\settowidth{\tmplength}{\textbf{Declaração}}
\setlength{\itemindent}{0cm}
\setlength{\listparindent}{0cm}
\setlength{\leftmargin}{\evensidemargin}
\addtolength{\leftmargin}{\tmplength}
\settowidth{\labelsep}{X}
\addtolength{\leftmargin}{\labelsep}
\setlength{\labelwidth}{\tmplength}
}
\begin{flushleft}
\item[\textbf{Declaração}\hfill]
\begin{ttfamily}
SCmProcess            = 'Processa';\end{ttfamily}


\end{flushleft}
\end{list}
\subsection*{SCmExecEndProc}
\begin{list}{}{
\settowidth{\tmplength}{\textbf{Declaração}}
\setlength{\itemindent}{0cm}
\setlength{\listparindent}{0cm}
\setlength{\leftmargin}{\evensidemargin}
\addtolength{\leftmargin}{\tmplength}
\settowidth{\labelsep}{X}
\addtolength{\leftmargin}{\labelsep}
\setlength{\labelwidth}{\tmplength}
}
\begin{flushleft}
\item[\textbf{Declaração}\hfill]
\begin{ttfamily}
SCmExecEndProc        = 'SCmExecEndProc';\end{ttfamily}


\end{flushleft}
\par
\item[\textbf{Descrição}]
Usado para acessar a pesquisa associado ao campo

\end{list}
\subsection*{SCmExecComboBox}
\begin{list}{}{
\settowidth{\tmplength}{\textbf{Declaração}}
\setlength{\itemindent}{0cm}
\setlength{\listparindent}{0cm}
\setlength{\leftmargin}{\evensidemargin}
\addtolength{\leftmargin}{\tmplength}
\settowidth{\labelsep}{X}
\addtolength{\leftmargin}{\labelsep}
\setlength{\labelwidth}{\tmplength}
}
\begin{flushleft}
\item[\textbf{Declaração}\hfill]
\begin{ttfamily}
SCmExecComboBox       = 'SCmExecComboBox';\end{ttfamily}


\end{flushleft}
\par
\item[\textbf{Descrição}]
Usado para acessar a visao associada ao campo. Usado para visualizar CamposEnumerado e lista de forma geral

\end{list}
\subsection*{SCmExecCommand}
\begin{list}{}{
\settowidth{\tmplength}{\textbf{Declaração}}
\setlength{\itemindent}{0cm}
\setlength{\listparindent}{0cm}
\setlength{\leftmargin}{\evensidemargin}
\addtolength{\leftmargin}{\tmplength}
\settowidth{\labelsep}{X}
\addtolength{\leftmargin}{\labelsep}
\setlength{\labelwidth}{\tmplength}
}
\begin{flushleft}
\item[\textbf{Declaração}\hfill]
\begin{ttfamily}
SCmExecCommand        = 'SCmExecCommand';\end{ttfamily}


\end{flushleft}
\par
\item[\textbf{Descrição}]
O comando vinculado ao campo focado e disparado para apliication.HanleEvent() se

\end{list}
\subsection*{SCmCreate{\_}Shortcut}
\begin{list}{}{
\settowidth{\tmplength}{\textbf{Declaração}}
\setlength{\itemindent}{0cm}
\setlength{\listparindent}{0cm}
\setlength{\leftmargin}{\evensidemargin}
\addtolength{\leftmargin}{\tmplength}
\settowidth{\labelsep}{X}
\addtolength{\leftmargin}{\labelsep}
\setlength{\labelwidth}{\tmplength}
}
\begin{flushleft}
\item[\textbf{Declaração}\hfill]
\begin{ttfamily}
SCmCreate{\_}Shortcut    = 'Cria atalho no desktop do windows';\end{ttfamily}


\end{flushleft}
\end{list}
\subsection*{SCmVisualizar}
\begin{list}{}{
\settowidth{\tmplength}{\textbf{Declaração}}
\setlength{\itemindent}{0cm}
\setlength{\listparindent}{0cm}
\setlength{\leftmargin}{\evensidemargin}
\addtolength{\leftmargin}{\tmplength}
\settowidth{\labelsep}{X}
\addtolength{\leftmargin}{\labelsep}
\setlength{\labelwidth}{\tmplength}
}
\begin{flushleft}
\item[\textbf{Declaração}\hfill]
\begin{ttfamily}
SCmVisualizar         = 'Visualizar';\end{ttfamily}


\end{flushleft}
\end{list}
\subsection*{SCmExport{\_}Stru}
\begin{list}{}{
\settowidth{\tmplength}{\textbf{Declaração}}
\setlength{\itemindent}{0cm}
\setlength{\listparindent}{0cm}
\setlength{\leftmargin}{\evensidemargin}
\addtolength{\leftmargin}{\tmplength}
\settowidth{\labelsep}{X}
\addtolength{\leftmargin}{\labelsep}
\setlength{\labelwidth}{\tmplength}
}
\begin{flushleft}
\item[\textbf{Declaração}\hfill]
\begin{ttfamily}
SCmExport{\_}Stru        = 'Exportar estrutura da tabela';\end{ttfamily}


\end{flushleft}
\par
\item[\textbf{Descrição}]
Exporta a estrutura das consultas para o arquivo Schema.ini

\end{list}
\subsection*{SCmExport}
\begin{list}{}{
\settowidth{\tmplength}{\textbf{Declaração}}
\setlength{\itemindent}{0cm}
\setlength{\listparindent}{0cm}
\setlength{\leftmargin}{\evensidemargin}
\addtolength{\leftmargin}{\tmplength}
\settowidth{\labelsep}{X}
\addtolength{\leftmargin}{\labelsep}
\setlength{\labelwidth}{\tmplength}
}
\begin{flushleft}
\item[\textbf{Declaração}\hfill]
\begin{ttfamily}
SCmExport             = 'Exporta';\end{ttfamily}


\end{flushleft}
\par
\item[\textbf{Descrição}]
Exporta a consulta seleciona para varios formatos de arquivos a serem implementados

\end{list}
\subsection*{SCmDbAddRec}
\begin{list}{}{
\settowidth{\tmplength}{\textbf{Declaração}}
\setlength{\itemindent}{0cm}
\setlength{\listparindent}{0cm}
\setlength{\leftmargin}{\evensidemargin}
\addtolength{\leftmargin}{\tmplength}
\settowidth{\labelsep}{X}
\addtolength{\leftmargin}{\labelsep}
\setlength{\labelwidth}{\tmplength}
}
\begin{flushleft}
\item[\textbf{Declaração}\hfill]
\begin{ttfamily}
SCmDbAddRec             = 'Adicionar registro';\end{ttfamily}


\end{flushleft}
\end{list}
\subsection*{SCmDbDeleteRec}
\begin{list}{}{
\settowidth{\tmplength}{\textbf{Declaração}}
\setlength{\itemindent}{0cm}
\setlength{\listparindent}{0cm}
\setlength{\leftmargin}{\evensidemargin}
\addtolength{\leftmargin}{\tmplength}
\settowidth{\labelsep}{X}
\addtolength{\leftmargin}{\labelsep}
\setlength{\labelwidth}{\tmplength}
}
\begin{flushleft}
\item[\textbf{Declaração}\hfill]
\begin{ttfamily}
SCmDbDeleteRec          = 'Apagar registro selecionado';\end{ttfamily}


\end{flushleft}
\end{list}
\subsection*{SCmDbGetRec}
\begin{list}{}{
\settowidth{\tmplength}{\textbf{Declaração}}
\setlength{\itemindent}{0cm}
\setlength{\listparindent}{0cm}
\setlength{\leftmargin}{\evensidemargin}
\addtolength{\leftmargin}{\tmplength}
\settowidth{\labelsep}{X}
\addtolength{\leftmargin}{\labelsep}
\setlength{\labelwidth}{\tmplength}
}
\begin{flushleft}
\item[\textbf{Declaração}\hfill]
\begin{ttfamily}
SCmDbGetRec             = 'Ler registro selecionado';\end{ttfamily}


\end{flushleft}
\end{list}
\subsection*{SCmDbPutRec}
\begin{list}{}{
\settowidth{\tmplength}{\textbf{Declaração}}
\setlength{\itemindent}{0cm}
\setlength{\listparindent}{0cm}
\setlength{\leftmargin}{\evensidemargin}
\addtolength{\leftmargin}{\tmplength}
\settowidth{\labelsep}{X}
\addtolength{\leftmargin}{\labelsep}
\setlength{\labelwidth}{\tmplength}
}
\begin{flushleft}
\item[\textbf{Declaração}\hfill]
\begin{ttfamily}
SCmDbPutRec             = 'Gravar registro selecionado';\end{ttfamily}


\end{flushleft}
\end{list}
\subsection*{SCmDbUpdateRec}
\begin{list}{}{
\settowidth{\tmplength}{\textbf{Declaração}}
\setlength{\itemindent}{0cm}
\setlength{\listparindent}{0cm}
\setlength{\leftmargin}{\evensidemargin}
\addtolength{\leftmargin}{\tmplength}
\settowidth{\labelsep}{X}
\addtolength{\leftmargin}{\labelsep}
\setlength{\labelwidth}{\tmplength}
}
\begin{flushleft}
\item[\textbf{Declaração}\hfill]
\begin{ttfamily}
SCmDbUpdateRec          = 'Atualizar registro selecionado caso tenha sido alterado';\end{ttfamily}


\end{flushleft}
\end{list}
\subsection*{SCmDbSearchTop}
\begin{list}{}{
\settowidth{\tmplength}{\textbf{Declaração}}
\setlength{\itemindent}{0cm}
\setlength{\listparindent}{0cm}
\setlength{\leftmargin}{\evensidemargin}
\addtolength{\leftmargin}{\tmplength}
\settowidth{\labelsep}{X}
\addtolength{\leftmargin}{\labelsep}
\setlength{\labelwidth}{\tmplength}
}
\begin{flushleft}
\item[\textbf{Declaração}\hfill]
\begin{ttfamily}
SCmDbSearchTop          = 'Pesquisar primeira ocorrência a partir do topo da tabela';\end{ttfamily}


\end{flushleft}
\end{list}
\subsection*{SCmDbSearchKey}
\begin{list}{}{
\settowidth{\tmplength}{\textbf{Declaração}}
\setlength{\itemindent}{0cm}
\setlength{\listparindent}{0cm}
\setlength{\leftmargin}{\evensidemargin}
\addtolength{\leftmargin}{\tmplength}
\settowidth{\labelsep}{X}
\addtolength{\leftmargin}{\labelsep}
\setlength{\labelwidth}{\tmplength}
}
\begin{flushleft}
\item[\textbf{Declaração}\hfill]
\begin{ttfamily}
SCmDbSearchKey          = 'Pesquisar primeira ocorrência a partir do inicio da tabela';\end{ttfamily}


\end{flushleft}
\end{list}
\subsection*{SCmDbUsedRecs{\_}Valid}
\begin{list}{}{
\settowidth{\tmplength}{\textbf{Declaração}}
\setlength{\itemindent}{0cm}
\setlength{\listparindent}{0cm}
\setlength{\leftmargin}{\evensidemargin}
\addtolength{\leftmargin}{\tmplength}
\settowidth{\labelsep}{X}
\addtolength{\leftmargin}{\labelsep}
\setlength{\labelwidth}{\tmplength}
}
\begin{flushleft}
\item[\textbf{Declaração}\hfill]
\begin{ttfamily}
SCmDbUsedRecs{\_}Valid     = 'CmDbUsedRecs{\_}Valid';\end{ttfamily}


\end{flushleft}
\end{list}
\subsection*{SCmOkEscrevaParametrosDosRelatorios}
\begin{list}{}{
\settowidth{\tmplength}{\textbf{Declaração}}
\setlength{\itemindent}{0cm}
\setlength{\listparindent}{0cm}
\setlength{\leftmargin}{\evensidemargin}
\addtolength{\leftmargin}{\tmplength}
\settowidth{\labelsep}{X}
\addtolength{\leftmargin}{\labelsep}
\setlength{\labelwidth}{\tmplength}
}
\begin{flushleft}
\item[\textbf{Declaração}\hfill]
\begin{ttfamily}
SCmOkEscrevaParametrosDosRelatorios = 'Escreva parâmetros dos relatórios';\end{ttfamily}


\end{flushleft}
\end{list}
\subsection*{SCmDbSelecionaIndice}
\begin{list}{}{
\settowidth{\tmplength}{\textbf{Declaração}}
\setlength{\itemindent}{0cm}
\setlength{\listparindent}{0cm}
\setlength{\leftmargin}{\evensidemargin}
\addtolength{\leftmargin}{\tmplength}
\settowidth{\labelsep}{X}
\addtolength{\leftmargin}{\labelsep}
\setlength{\labelwidth}{\tmplength}
}
\begin{flushleft}
\item[\textbf{Declaração}\hfill]
\begin{ttfamily}
SCmDbSelecionaIndice    = 'Selecionar indice';\end{ttfamily}


\end{flushleft}
\end{list}
\subsection*{SLivreCmVisualisa}
\begin{list}{}{
\settowidth{\tmplength}{\textbf{Declaração}}
\setlength{\itemindent}{0cm}
\setlength{\listparindent}{0cm}
\setlength{\leftmargin}{\evensidemargin}
\addtolength{\leftmargin}{\tmplength}
\settowidth{\labelsep}{X}
\addtolength{\leftmargin}{\labelsep}
\setlength{\labelwidth}{\tmplength}
}
\begin{flushleft}
\item[\textbf{Declaração}\hfill]
\begin{ttfamily}
SLivreCmVisualisa       = 'CmLivreCmVisualisa';\end{ttfamily}


\end{flushleft}
\end{list}
\subsection*{SCmQuitInterno}
\begin{list}{}{
\settowidth{\tmplength}{\textbf{Declaração}}
\setlength{\itemindent}{0cm}
\setlength{\listparindent}{0cm}
\setlength{\leftmargin}{\evensidemargin}
\addtolength{\leftmargin}{\tmplength}
\settowidth{\labelsep}{X}
\addtolength{\leftmargin}{\labelsep}
\setlength{\labelwidth}{\tmplength}
}
\begin{flushleft}
\item[\textbf{Declaração}\hfill]
\begin{ttfamily}
SCmQuitInterno          = 'Quit interno';\end{ttfamily}


\end{flushleft}
\end{list}
\subsection*{SCmSobre}
\begin{list}{}{
\settowidth{\tmplength}{\textbf{Declaração}}
\setlength{\itemindent}{0cm}
\setlength{\listparindent}{0cm}
\setlength{\leftmargin}{\evensidemargin}
\addtolength{\leftmargin}{\tmplength}
\settowidth{\labelsep}{X}
\addtolength{\leftmargin}{\labelsep}
\setlength{\labelwidth}{\tmplength}
}
\begin{flushleft}
\item[\textbf{Declaração}\hfill]
\begin{ttfamily}
SCmSobre                = 'Sobre';\end{ttfamily}


\end{flushleft}
\end{list}
\subsection*{SCmDbOnEnter}
\begin{list}{}{
\settowidth{\tmplength}{\textbf{Declaração}}
\setlength{\itemindent}{0cm}
\setlength{\listparindent}{0cm}
\setlength{\leftmargin}{\evensidemargin}
\addtolength{\leftmargin}{\tmplength}
\settowidth{\labelsep}{X}
\addtolength{\leftmargin}{\labelsep}
\setlength{\labelwidth}{\tmplength}
}
\begin{flushleft}
\item[\textbf{Declaração}\hfill]
\begin{ttfamily}
SCmDbOnEnter            = 'CmDbOnEnter';\end{ttfamily}


\end{flushleft}
\end{list}
\subsection*{SCmDbOnExit}
\begin{list}{}{
\settowidth{\tmplength}{\textbf{Declaração}}
\setlength{\itemindent}{0cm}
\setlength{\listparindent}{0cm}
\setlength{\leftmargin}{\evensidemargin}
\addtolength{\leftmargin}{\tmplength}
\settowidth{\labelsep}{X}
\addtolength{\leftmargin}{\labelsep}
\setlength{\labelwidth}{\tmplength}
}
\begin{flushleft}
\item[\textbf{Declaração}\hfill]
\begin{ttfamily}
SCmDbOnExit             = 'CmDbOnExit';\end{ttfamily}


\end{flushleft}
\end{list}
\subsection*{ScmCores}
\begin{list}{}{
\settowidth{\tmplength}{\textbf{Declaração}}
\setlength{\itemindent}{0cm}
\setlength{\listparindent}{0cm}
\setlength{\leftmargin}{\evensidemargin}
\addtolength{\leftmargin}{\tmplength}
\settowidth{\labelsep}{X}
\addtolength{\leftmargin}{\labelsep}
\setlength{\labelwidth}{\tmplength}
}
\begin{flushleft}
\item[\textbf{Declaração}\hfill]
\begin{ttfamily}
ScmCores                = 'cmCores';\end{ttfamily}


\end{flushleft}
\end{list}
\subsection*{SCmF7}
\begin{list}{}{
\settowidth{\tmplength}{\textbf{Declaração}}
\setlength{\itemindent}{0cm}
\setlength{\listparindent}{0cm}
\setlength{\leftmargin}{\evensidemargin}
\addtolength{\leftmargin}{\tmplength}
\settowidth{\labelsep}{X}
\addtolength{\leftmargin}{\labelsep}
\setlength{\labelwidth}{\tmplength}
}
\begin{flushleft}
\item[\textbf{Declaração}\hfill]
\begin{ttfamily}
SCmF7                   = 'Seleciona as opções para o campo selecionado';\end{ttfamily}


\end{flushleft}
\end{list}
\subsection*{SCmDbLabel{\_}DoubleClick}
\begin{list}{}{
\settowidth{\tmplength}{\textbf{Declaração}}
\setlength{\itemindent}{0cm}
\setlength{\listparindent}{0cm}
\setlength{\leftmargin}{\evensidemargin}
\addtolength{\leftmargin}{\tmplength}
\settowidth{\labelsep}{X}
\addtolength{\leftmargin}{\labelsep}
\setlength{\labelwidth}{\tmplength}
}
\begin{flushleft}
\item[\textbf{Declaração}\hfill]
\begin{ttfamily}
SCmDbLabel{\_}DoubleClick  = 'CmDbLabel{\_}DoubleClick';\end{ttfamily}


\end{flushleft}
\end{list}
\subsection*{ScmDbView{\_}DoubleClick}
\begin{list}{}{
\settowidth{\tmplength}{\textbf{Declaração}}
\setlength{\itemindent}{0cm}
\setlength{\listparindent}{0cm}
\setlength{\leftmargin}{\evensidemargin}
\addtolength{\leftmargin}{\tmplength}
\settowidth{\labelsep}{X}
\addtolength{\leftmargin}{\labelsep}
\setlength{\labelwidth}{\tmplength}
}
\begin{flushleft}
\item[\textbf{Declaração}\hfill]
\begin{ttfamily}
ScmDbView{\_}DoubleClick   = 'cmDbView{\_}DoubleClick';\end{ttfamily}


\end{flushleft}
\end{list}
\subsection*{SCmDbOrdemCressante}
\begin{list}{}{
\settowidth{\tmplength}{\textbf{Declaração}}
\setlength{\itemindent}{0cm}
\setlength{\listparindent}{0cm}
\setlength{\leftmargin}{\evensidemargin}
\addtolength{\leftmargin}{\tmplength}
\settowidth{\labelsep}{X}
\addtolength{\leftmargin}{\labelsep}
\setlength{\labelwidth}{\tmplength}
}
\begin{flushleft}
\item[\textbf{Declaração}\hfill]
\begin{ttfamily}
SCmDbOrdemCressante     = 'Ordem cressante';\end{ttfamily}


\end{flushleft}
\end{list}
\subsection*{SCmDbOrdemDecrescente}
\begin{list}{}{
\settowidth{\tmplength}{\textbf{Declaração}}
\setlength{\itemindent}{0cm}
\setlength{\listparindent}{0cm}
\setlength{\leftmargin}{\evensidemargin}
\addtolength{\leftmargin}{\tmplength}
\settowidth{\labelsep}{X}
\addtolength{\leftmargin}{\labelsep}
\setlength{\labelwidth}{\tmplength}
}
\begin{flushleft}
\item[\textbf{Declaração}\hfill]
\begin{ttfamily}
SCmDbOrdemDecrescente   = 'Ordem decrescente';\end{ttfamily}


\end{flushleft}
\end{list}
\subsection*{SCmDbSelecColunaAtual}
\begin{list}{}{
\settowidth{\tmplength}{\textbf{Declaração}}
\setlength{\itemindent}{0cm}
\setlength{\listparindent}{0cm}
\setlength{\leftmargin}{\evensidemargin}
\addtolength{\leftmargin}{\tmplength}
\settowidth{\labelsep}{X}
\addtolength{\leftmargin}{\labelsep}
\setlength{\labelwidth}{\tmplength}
}
\begin{flushleft}
\item[\textbf{Declaração}\hfill]
\begin{ttfamily}
SCmDbSelecColunaAtual   = 'CmDbSelecColunaAtual';\end{ttfamily}


\end{flushleft}
\end{list}
\subsection*{SCmMouseDownmbRightButton}
\begin{list}{}{
\settowidth{\tmplength}{\textbf{Declaração}}
\setlength{\itemindent}{0cm}
\setlength{\listparindent}{0cm}
\setlength{\leftmargin}{\evensidemargin}
\addtolength{\leftmargin}{\tmplength}
\settowidth{\labelsep}{X}
\addtolength{\leftmargin}{\labelsep}
\setlength{\labelwidth}{\tmplength}
}
\begin{flushleft}
\item[\textbf{Declaração}\hfill]
\begin{ttfamily}
SCmMouseDownmbRightButton = 'CmMouseDownmbRightButton';\end{ttfamily}


\end{flushleft}
\end{list}
\subsection*{SCmReindex}
\begin{list}{}{
\settowidth{\tmplength}{\textbf{Declaração}}
\setlength{\itemindent}{0cm}
\setlength{\listparindent}{0cm}
\setlength{\leftmargin}{\evensidemargin}
\addtolength{\leftmargin}{\tmplength}
\settowidth{\labelsep}{X}
\addtolength{\leftmargin}{\labelsep}
\setlength{\labelwidth}{\tmplength}
}
\begin{flushleft}
\item[\textbf{Declaração}\hfill]
\begin{ttfamily}
SCmReindex                = 'Cria indices dos arquivos';\end{ttfamily}


\end{flushleft}
\end{list}
\subsection*{SCmCadastraImpressoraRede}
\begin{list}{}{
\settowidth{\tmplength}{\textbf{Declaração}}
\setlength{\itemindent}{0cm}
\setlength{\listparindent}{0cm}
\setlength{\leftmargin}{\evensidemargin}
\addtolength{\leftmargin}{\tmplength}
\settowidth{\labelsep}{X}
\addtolength{\leftmargin}{\labelsep}
\setlength{\labelwidth}{\tmplength}
}
\begin{flushleft}
\item[\textbf{Declaração}\hfill]
\begin{ttfamily}
SCmCadastraImpressoraRede  = 'Cadastra impressora da rede';\end{ttfamily}


\end{flushleft}
\end{list}
\subsection*{SCmInfoSystem}
\begin{list}{}{
\settowidth{\tmplength}{\textbf{Declaração}}
\setlength{\itemindent}{0cm}
\setlength{\listparindent}{0cm}
\setlength{\leftmargin}{\evensidemargin}
\addtolength{\leftmargin}{\tmplength}
\settowidth{\labelsep}{X}
\addtolength{\leftmargin}{\labelsep}
\setlength{\labelwidth}{\tmplength}
}
\begin{flushleft}
\item[\textbf{Declaração}\hfill]
\begin{ttfamily}
SCmInfoSystem            = 'Informações do sistema';\end{ttfamily}


\end{flushleft}
\end{list}
\subsection*{ScmPrintSemFormatar}
\begin{list}{}{
\settowidth{\tmplength}{\textbf{Declaração}}
\setlength{\itemindent}{0cm}
\setlength{\listparindent}{0cm}
\setlength{\leftmargin}{\evensidemargin}
\addtolength{\leftmargin}{\tmplength}
\settowidth{\labelsep}{X}
\addtolength{\leftmargin}{\labelsep}
\setlength{\labelwidth}{\tmplength}
}
\begin{flushleft}
\item[\textbf{Declaração}\hfill]
\begin{ttfamily}
ScmPrintSemFormatar      = 'cmPrintSemFormatar';\end{ttfamily}


\end{flushleft}
\end{list}
\subsection*{SCmDbDoBeforeInsert}
\begin{list}{}{
\settowidth{\tmplength}{\textbf{Declaração}}
\setlength{\itemindent}{0cm}
\setlength{\listparindent}{0cm}
\setlength{\leftmargin}{\evensidemargin}
\addtolength{\leftmargin}{\tmplength}
\settowidth{\labelsep}{X}
\addtolength{\leftmargin}{\labelsep}
\setlength{\labelwidth}{\tmplength}
}
\begin{flushleft}
\item[\textbf{Declaração}\hfill]
\begin{ttfamily}
SCmDbDoBeforeInsert      = 'CmDbDoBeforeInsert';\end{ttfamily}


\end{flushleft}
\end{list}
\subsection*{SCmDbDoBeforePost}
\begin{list}{}{
\settowidth{\tmplength}{\textbf{Declaração}}
\setlength{\itemindent}{0cm}
\setlength{\listparindent}{0cm}
\setlength{\leftmargin}{\evensidemargin}
\addtolength{\leftmargin}{\tmplength}
\settowidth{\labelsep}{X}
\addtolength{\leftmargin}{\labelsep}
\setlength{\labelwidth}{\tmplength}
}
\begin{flushleft}
\item[\textbf{Declaração}\hfill]
\begin{ttfamily}
SCmDbDoBeforePost        = 'CmDbDoBeforePost';\end{ttfamily}


\end{flushleft}
\end{list}
\subsection*{SCmDbDoBeforeDelete}
\begin{list}{}{
\settowidth{\tmplength}{\textbf{Declaração}}
\setlength{\itemindent}{0cm}
\setlength{\listparindent}{0cm}
\setlength{\leftmargin}{\evensidemargin}
\addtolength{\leftmargin}{\tmplength}
\settowidth{\labelsep}{X}
\addtolength{\leftmargin}{\labelsep}
\setlength{\labelwidth}{\tmplength}
}
\begin{flushleft}
\item[\textbf{Declaração}\hfill]
\begin{ttfamily}
SCmDbDoBeforeDelete      = 'CmDbDoBeforeDelete';\end{ttfamily}


\end{flushleft}
\end{list}
\subsection*{SCmDbDoAfterInsert}
\begin{list}{}{
\settowidth{\tmplength}{\textbf{Declaração}}
\setlength{\itemindent}{0cm}
\setlength{\listparindent}{0cm}
\setlength{\leftmargin}{\evensidemargin}
\addtolength{\leftmargin}{\tmplength}
\settowidth{\labelsep}{X}
\addtolength{\leftmargin}{\labelsep}
\setlength{\labelwidth}{\tmplength}
}
\begin{flushleft}
\item[\textbf{Declaração}\hfill]
\begin{ttfamily}
SCmDbDoAfterInsert       = 'CmDbDoAfterInsert';\end{ttfamily}


\end{flushleft}
\end{list}
\subsection*{SCmDbDoAfterPost}
\begin{list}{}{
\settowidth{\tmplength}{\textbf{Declaração}}
\setlength{\itemindent}{0cm}
\setlength{\listparindent}{0cm}
\setlength{\leftmargin}{\evensidemargin}
\addtolength{\leftmargin}{\tmplength}
\settowidth{\labelsep}{X}
\addtolength{\leftmargin}{\labelsep}
\setlength{\labelwidth}{\tmplength}
}
\begin{flushleft}
\item[\textbf{Declaração}\hfill]
\begin{ttfamily}
SCmDbDoAfterPost         = 'CmDbDoAfterPost';\end{ttfamily}


\end{flushleft}
\end{list}
\subsection*{SCmDbDoAfterDelete}
\begin{list}{}{
\settowidth{\tmplength}{\textbf{Declaração}}
\setlength{\itemindent}{0cm}
\setlength{\listparindent}{0cm}
\setlength{\leftmargin}{\evensidemargin}
\addtolength{\leftmargin}{\tmplength}
\settowidth{\labelsep}{X}
\addtolength{\leftmargin}{\labelsep}
\setlength{\labelwidth}{\tmplength}
}
\begin{flushleft}
\item[\textbf{Declaração}\hfill]
\begin{ttfamily}
SCmDbDoAfterDelete       = 'CmDbDoAfterDelete';\end{ttfamily}


\end{flushleft}
\end{list}
\subsection*{SCmTb{\_}SelectRefCruzadaResume}
\begin{list}{}{
\settowidth{\tmplength}{\textbf{Declaração}}
\setlength{\itemindent}{0cm}
\setlength{\listparindent}{0cm}
\setlength{\leftmargin}{\evensidemargin}
\addtolength{\leftmargin}{\tmplength}
\settowidth{\labelsep}{X}
\addtolength{\leftmargin}{\labelsep}
\setlength{\labelwidth}{\tmplength}
}
\begin{flushleft}
\item[\textbf{Declaração}\hfill]
\begin{ttfamily}
SCmTb{\_}SelectRefCruzadaResume = 'CmTb{\_}SelectRefCruzadaResume';\end{ttfamily}


\end{flushleft}
\end{list}
\subsection*{SCmTb{\_}SelectSelect}
\begin{list}{}{
\settowidth{\tmplength}{\textbf{Declaração}}
\setlength{\itemindent}{0cm}
\setlength{\listparindent}{0cm}
\setlength{\leftmargin}{\evensidemargin}
\addtolength{\leftmargin}{\tmplength}
\settowidth{\labelsep}{X}
\addtolength{\leftmargin}{\labelsep}
\setlength{\labelwidth}{\tmplength}
}
\begin{flushleft}
\item[\textbf{Declaração}\hfill]
\begin{ttfamily}
SCmTb{\_}SelectSelect           = 'CmTb{\_}SelectSelect';\end{ttfamily}


\end{flushleft}
\end{list}
\subsection*{SCmTb{\_}SelectResume}
\begin{list}{}{
\settowidth{\tmplength}{\textbf{Declaração}}
\setlength{\itemindent}{0cm}
\setlength{\listparindent}{0cm}
\setlength{\leftmargin}{\evensidemargin}
\addtolength{\leftmargin}{\tmplength}
\settowidth{\labelsep}{X}
\addtolength{\leftmargin}{\labelsep}
\setlength{\labelwidth}{\tmplength}
}
\begin{flushleft}
\item[\textbf{Declaração}\hfill]
\begin{ttfamily}
SCmTb{\_}SelectResume           = 'CmTb{\_}SelectResume';\end{ttfamily}


\end{flushleft}
\end{list}
\subsection*{SCmRegistroValido}
\begin{list}{}{
\settowidth{\tmplength}{\textbf{Declaração}}
\setlength{\itemindent}{0cm}
\setlength{\listparindent}{0cm}
\setlength{\leftmargin}{\evensidemargin}
\addtolength{\leftmargin}{\tmplength}
\settowidth{\labelsep}{X}
\addtolength{\leftmargin}{\labelsep}
\setlength{\labelwidth}{\tmplength}
}
\begin{flushleft}
\item[\textbf{Declaração}\hfill]
\begin{ttfamily}
SCmRegistroValido            = 'Registro válido';\end{ttfamily}


\end{flushleft}
\end{list}
\subsection*{SCmCopyTo}
\begin{list}{}{
\settowidth{\tmplength}{\textbf{Declaração}}
\setlength{\itemindent}{0cm}
\setlength{\listparindent}{0cm}
\setlength{\leftmargin}{\evensidemargin}
\addtolength{\leftmargin}{\tmplength}
\settowidth{\labelsep}{X}
\addtolength{\leftmargin}{\labelsep}
\setlength{\labelwidth}{\tmplength}
}
\begin{flushleft}
\item[\textbf{Declaração}\hfill]
\begin{ttfamily}
SCmCopyTo                    = 'Copiar para';\end{ttfamily}


\end{flushleft}
\end{list}
\subsection*{SCmCadastraImpressoraLocal}
\begin{list}{}{
\settowidth{\tmplength}{\textbf{Declaração}}
\setlength{\itemindent}{0cm}
\setlength{\listparindent}{0cm}
\setlength{\leftmargin}{\evensidemargin}
\addtolength{\leftmargin}{\tmplength}
\settowidth{\labelsep}{X}
\addtolength{\leftmargin}{\labelsep}
\setlength{\labelwidth}{\tmplength}
}
\begin{flushleft}
\item[\textbf{Declaração}\hfill]
\begin{ttfamily}
SCmCadastraImpressoraLocal   = 'Cadastra impressora local';\end{ttfamily}


\end{flushleft}
\end{list}
\subsection*{SCmSetAppending}
\begin{list}{}{
\settowidth{\tmplength}{\textbf{Declaração}}
\setlength{\itemindent}{0cm}
\setlength{\listparindent}{0cm}
\setlength{\leftmargin}{\evensidemargin}
\addtolength{\leftmargin}{\tmplength}
\settowidth{\labelsep}{X}
\addtolength{\leftmargin}{\labelsep}
\setlength{\labelwidth}{\tmplength}
}
\begin{flushleft}
\item[\textbf{Declaração}\hfill]
\begin{ttfamily}
SCmSetAppending              = 'CmSetAppending';\end{ttfamily}


\end{flushleft}
\end{list}
\subsection*{SCmStartTransaction}
\begin{list}{}{
\settowidth{\tmplength}{\textbf{Declaração}}
\setlength{\itemindent}{0cm}
\setlength{\listparindent}{0cm}
\setlength{\leftmargin}{\evensidemargin}
\addtolength{\leftmargin}{\tmplength}
\settowidth{\labelsep}{X}
\addtolength{\leftmargin}{\labelsep}
\setlength{\labelwidth}{\tmplength}
}
\begin{flushleft}
\item[\textbf{Declaração}\hfill]
\begin{ttfamily}
SCmStartTransaction          = 'Inicia uma transação';\end{ttfamily}


\end{flushleft}
\end{list}
\subsection*{SCmCommit}
\begin{list}{}{
\settowidth{\tmplength}{\textbf{Declaração}}
\setlength{\itemindent}{0cm}
\setlength{\listparindent}{0cm}
\setlength{\leftmargin}{\evensidemargin}
\addtolength{\leftmargin}{\tmplength}
\settowidth{\labelsep}{X}
\addtolength{\leftmargin}{\labelsep}
\setlength{\labelwidth}{\tmplength}
}
\begin{flushleft}
\item[\textbf{Declaração}\hfill]
\begin{ttfamily}
SCmCommit                    = 'Confirma transação';\end{ttfamily}


\end{flushleft}
\end{list}
\subsection*{SCmRollback}
\begin{list}{}{
\settowidth{\tmplength}{\textbf{Declaração}}
\setlength{\itemindent}{0cm}
\setlength{\listparindent}{0cm}
\setlength{\leftmargin}{\evensidemargin}
\addtolength{\leftmargin}{\tmplength}
\settowidth{\labelsep}{X}
\addtolength{\leftmargin}{\labelsep}
\setlength{\labelwidth}{\tmplength}
}
\begin{flushleft}
\item[\textbf{Declaração}\hfill]
\begin{ttfamily}
SCmRollback                  = 'CmRollback';\end{ttfamily}


\end{flushleft}
\end{list}
\subsection*{SCmOnCalcRecord{\_}All}
\begin{list}{}{
\settowidth{\tmplength}{\textbf{Declaração}}
\setlength{\itemindent}{0cm}
\setlength{\listparindent}{0cm}
\setlength{\leftmargin}{\evensidemargin}
\addtolength{\leftmargin}{\tmplength}
\settowidth{\labelsep}{X}
\addtolength{\leftmargin}{\labelsep}
\setlength{\labelwidth}{\tmplength}
}
\begin{flushleft}
\item[\textbf{Declaração}\hfill]
\begin{ttfamily}
SCmOnCalcRecord{\_}All          = 'Calcula todos os registros';\end{ttfamily}


\end{flushleft}
\end{list}
\subsection*{SCmTime}
\begin{list}{}{
\settowidth{\tmplength}{\textbf{Declaração}}
\setlength{\itemindent}{0cm}
\setlength{\listparindent}{0cm}
\setlength{\leftmargin}{\evensidemargin}
\addtolength{\leftmargin}{\tmplength}
\settowidth{\labelsep}{X}
\addtolength{\leftmargin}{\labelsep}
\setlength{\labelwidth}{\tmplength}
}
\begin{flushleft}
\item[\textbf{Declaração}\hfill]
\begin{ttfamily}
SCmTime                      = 'Time';\end{ttfamily}


\end{flushleft}
\end{list}
\subsection*{ScmEditaCores}
\begin{list}{}{
\settowidth{\tmplength}{\textbf{Declaração}}
\setlength{\itemindent}{0cm}
\setlength{\listparindent}{0cm}
\setlength{\leftmargin}{\evensidemargin}
\addtolength{\leftmargin}{\tmplength}
\settowidth{\labelsep}{X}
\addtolength{\leftmargin}{\labelsep}
\setlength{\labelwidth}{\tmplength}
}
\begin{flushleft}
\item[\textbf{Declaração}\hfill]
\begin{ttfamily}
ScmEditaCores                = 'Edita cores';\end{ttfamily}


\end{flushleft}
\end{list}
\subsection*{ScmSalvaCores}
\begin{list}{}{
\settowidth{\tmplength}{\textbf{Declaração}}
\setlength{\itemindent}{0cm}
\setlength{\listparindent}{0cm}
\setlength{\leftmargin}{\evensidemargin}
\addtolength{\leftmargin}{\tmplength}
\settowidth{\labelsep}{X}
\addtolength{\leftmargin}{\labelsep}
\setlength{\labelwidth}{\tmplength}
}
\begin{flushleft}
\item[\textbf{Declaração}\hfill]
\begin{ttfamily}
ScmSalvaCores                = 'Salva cores';\end{ttfamily}


\end{flushleft}
\end{list}
\subsection*{ScmHomePage}
\begin{list}{}{
\settowidth{\tmplength}{\textbf{Declaração}}
\setlength{\itemindent}{0cm}
\setlength{\listparindent}{0cm}
\setlength{\leftmargin}{\evensidemargin}
\addtolength{\leftmargin}{\tmplength}
\settowidth{\labelsep}{X}
\addtolength{\leftmargin}{\labelsep}
\setlength{\labelwidth}{\tmplength}
}
\begin{flushleft}
\item[\textbf{Declaração}\hfill]
\begin{ttfamily}
ScmHomePage                  = 'Gera documento no formato HTML do formulário atual';\end{ttfamily}


\end{flushleft}
\end{list}
\subsection*{SCmDbPack}
\begin{list}{}{
\settowidth{\tmplength}{\textbf{Declaração}}
\setlength{\itemindent}{0cm}
\setlength{\listparindent}{0cm}
\setlength{\leftmargin}{\evensidemargin}
\addtolength{\leftmargin}{\tmplength}
\settowidth{\labelsep}{X}
\addtolength{\leftmargin}{\labelsep}
\setlength{\labelwidth}{\tmplength}
}
\begin{flushleft}
\item[\textbf{Declaração}\hfill]
\begin{ttfamily}
SCmDbPack                    = 'Pack';\end{ttfamily}


\end{flushleft}
\end{list}
\subsection*{sErr201}
\begin{list}{}{
\settowidth{\tmplength}{\textbf{Declaração}}
\setlength{\itemindent}{0cm}
\setlength{\listparindent}{0cm}
\setlength{\leftmargin}{\evensidemargin}
\addtolength{\leftmargin}{\tmplength}
\settowidth{\labelsep}{X}
\addtolength{\leftmargin}{\labelsep}
\setlength{\labelwidth}{\tmplength}
}
\begin{flushleft}
\item[\textbf{Declaração}\hfill]
\begin{ttfamily}
sErr201 = 'Erro: {\%}d - O valor {\%}s está fora da faixa permitida para o campo. Faixa: [{\%}d .. {\%}d ]';\end{ttfamily}


\end{flushleft}
\end{list}
\chapter{Unit mi.rtl.Consts.StrError}
\section{Descrição}
{-}A unit \textbf{\begin{ttfamily}mi.rtl.Consts.StrError\end{ttfamily}} implementa a classe \begin{ttfamily}TStrError\end{ttfamily}(\ref{mi.rtl.Consts.StrError.TStrError}) do pacote \begin{ttfamily}mi.rtl\end{ttfamily}(\ref{mi.rtl}).

\begin{itemize}
\item \textbf{VERSÃO}: \begin{itemize}
\item Alpha {-} 0.5.0.687
\end{itemize}
\item \textbf{CÓDIGO FONTE}: \begin{itemize}
\item 
\end{itemize}
\item \textbf{HISTÓRICO} \begin{itemize}
\item Criado por: Paulo Sérgio da Silva Pacheco e{-}mail: paulosspacheco@yahoo.com.br \begin{itemize}
\item 2021{-}12{-}02 {-}08:00 a 22:15 : Criado a unit \textbf{\begin{ttfamily}mi.rtl.Consts.StrError\end{ttfamily}} e implementação da classe \textbf{\begin{ttfamily}TStrError\end{ttfamily}(\ref{mi.rtl.Consts.StrError.TStrError})}
\item 2021{-}12{-}15 : Ajuste do método \begin{ttfamily}TStrError.ErrorMessage4\end{ttfamily}(\ref{mi.rtl.Consts.StrError.TStrError-ErrorMessage4});
\end{itemize}
\end{itemize}
\end{itemize}
\section{Uses}
\begin{itemize}
\item \begin{ttfamily}Classes\end{ttfamily}\item \begin{ttfamily}SysUtils\end{ttfamily}\item \begin{ttfamily}mi.rtl.Consts\end{ttfamily}(\ref{mi.rtl.Consts})\end{itemize}
\section{Visão Geral}
\begin{description}
\item[\texttt{\begin{ttfamily}TStrError\end{ttfamily} Classe}]
\end{description}
\section{Classes, Interfaces, Objetos e Registros}
\subsection*{TStrError Classe}
\subsubsection*{\large{\textbf{Hierarquia}}\normalsize\hspace{1ex}\hfill}
TStrError {$>$} \begin{ttfamily}TConsts\end{ttfamily}(\ref{mi.rtl.Consts.TConsts}) {$>$} \begin{ttfamily}TTypes\end{ttfamily}(\ref{mi.rtl.Types.TTypes}) {$>$} 
TComponent
\subsubsection*{\large{\textbf{Descrição}}\normalsize\hspace{1ex}\hfill}
A classe \textbf{\begin{ttfamily}TStrError\end{ttfamily}} é usada para produzir texto informativo sobre o local onde o erro ocorreu.\subsubsection*{\large{\textbf{Métodos}}\normalsize\hspace{1ex}\hfill}
\paragraph*{ErrorMessage}\hspace*{\fill}

\begin{list}{}{
\settowidth{\tmplength}{\textbf{Declaração}}
\setlength{\itemindent}{0cm}
\setlength{\listparindent}{0cm}
\setlength{\leftmargin}{\evensidemargin}
\addtolength{\leftmargin}{\tmplength}
\settowidth{\labelsep}{X}
\addtolength{\leftmargin}{\labelsep}
\setlength{\labelwidth}{\tmplength}
}
\begin{flushleft}
\item[\textbf{Declaração}\hfill]
\begin{ttfamily}
public class function ErrorMessage(Const ErrorCode : SmallWord) : AnsiString; overload;\end{ttfamily}


\end{flushleft}
\par
\item[\textbf{Descrição}]
A função \textbf{\begin{ttfamily}ErrorMessage\end{ttfamily}} retorna um texto com o nome do erro passado por ErrorCode

\begin{itemize}
\item \textbf{PARÂMETRO} \begin{itemize}
\item \textbf{ErrorCode} {-} Número do erro
\end{itemize}
\item \textbf{RETORNA} \begin{itemize}
\item Nome do erro
\end{itemize}
\item \textbf{NOTA} \begin{itemize}
\item A versão abaixo é usada quando o sistema é windows, a mesma tem o nome do erro no linux e o nome do erro no windows separada com a palavra \textbf{ou}.
\item Quando a plataforma é linux a mensagem que aparece é os erros no linux.
\end{itemize}
\item \textbf{LISTA DOS ERROS POSSÍVEIS} \begin{itemize}
\item [ErrorMessage.inc](./units/include/windows/ErrorMessage.inc)
\item [ErrorMessage.inc](./units/include/linux/ErrorMessage.inc)
\end{itemize} \texttt{\\\nopagebreak[3]
\\\nopagebreak[3]
0:~Result~:=~'000:~Chamada~inválida~a~função~Result.';\\\nopagebreak[3]
\\\nopagebreak[3]
\textit{{\{}1..99~RESERVADO~PARA~ERROS~DO~DOS{\}}}\\\nopagebreak[3]
1:~Result~:=~'EPERM~1:~Operação~não~permitida';\\\nopagebreak[3]
2:~Result~:=~'ENOENT~2~FileName~ou~diretório~inexistente';\\\nopagebreak[3]
3:~Result~:=~'ESRCH~3~Processo~inexistente';\\\nopagebreak[3]
4:~Result~:=~'EINTR~4~Chamada~de~sistema~interrompida';\\\nopagebreak[3]
5:~Result~:=~'EIO~5~Erro~de~entrada/saída';\\\nopagebreak[3]
6:~Result~:=~'ENXIO~6~Endereço~ou~dispositivo~inexistente';\\\nopagebreak[3]
7:~Result~:=~'E2BIG~7~Lista~de~argumentos~muito~longa';\\\nopagebreak[3]
8:~Result~:=~'ENOEXEC~8~Erro~no~formato~exec';\\\nopagebreak[3]
9:~Result~:=~'EBADF~9~Descritor~de~FileName~inválido';\\\nopagebreak[3]
10:~Result~:=~'ECHILD~10~Não~há~processos~filhos';\\\nopagebreak[3]
11:~Result~:=~'EAGAIN~11~Recurso~temporariamente~indisponível';\\\nopagebreak[3]
12:~Result~:=~'ENOMEM~12~Não~foi~possível~alocar~memória';\\\nopagebreak[3]
13:~Result~:=~'EACCES~13~Permissão~negada';\\\nopagebreak[3]
14:~Result~:=~'EFAULT~14~Endereço~inválido';\\\nopagebreak[3]
15:~Result~:=~'ENOTBLK~15~Dispositivo~de~bloco~requerido';\\\nopagebreak[3]
16:~Result~:=~'EBUSY~16~Dispositivo~ou~recurso~está~ocupado';\\\nopagebreak[3]
17:~Result~:=~'EEXIST~17~FileName~existe';\\\nopagebreak[3]
18:~Result~:=~'EXDEV~18~Link~entre~dispositivos~inválido';\\\nopagebreak[3]
19:~Result~:=~'ENODEV~19~Dispositivo~inexistente';\\\nopagebreak[3]
20:~Result~:=~'ENOTDIR~20~Não~é~um~diretório';\\\nopagebreak[3]
21:~Result~:=~'EISDIR~21~É~um~diretório';\\\nopagebreak[3]
22:~Result~:=~'EINVAL~22~Argumento~inválido';\\\nopagebreak[3]
23:~Result~:=~'ENFILE~23~Muitos~FileNames~abertos~no~sistema';\\\nopagebreak[3]
24:~Result~:=~'EMFILE~24~Muitos~FileNames~abertos';\\\nopagebreak[3]
25:~Result~:=~'ENOTTY~25~ioctl~inapropriado~para~dispositivo';\\\nopagebreak[3]
26:~Result~:=~'ETXTBSY~26~Área~de~texto~ocupada';\\\nopagebreak[3]
27:~Result~:=~'EFBIG~27~FileName~muito~grande';\\\nopagebreak[3]
28:~Result~:=~'ENOSPC~28~Não~há~espaço~disponível~no~dispositivo';\\\nopagebreak[3]
29:~Result~:=~'ESPIPE~29~Procura~ilegal';\\\nopagebreak[3]
30:~Result~:=~'EROFS~30~Sistema~de~FileNames~somente~para~leitura';\\\nopagebreak[3]
31:~Result~:=~'EMLINK~31~Muitos~links';\\\nopagebreak[3]
32:~Result~:=~'EPIPE~32~Pipe~quebrado';\\\nopagebreak[3]
33:~Result~:=~'EDOM~33~Argumento~numérico~fora~de~domínio';\\\nopagebreak[3]
34:~Result~:=~'ERANGE~34~Resultado~numérico~fora~de~alcance';\\\nopagebreak[3]
35:~Result~:=~'EDEADLK~35~Evitado~deadlock~de~recurso';\\\nopagebreak[3]
36:~Result~:=~'ENAMETOOLONG~36~Nome~de~FileName~muito~longo';\\\nopagebreak[3]
37:~Result~:=~'ENOLCK~37~Não~há~travas~disponíveis';\\\nopagebreak[3]
38:~Result~:=~'ENOSYS~38~Função~não~implementada';\\\nopagebreak[3]
39:~Result~:=~'ENOTEMPTY~39~Diretório~não~vazio';\\\nopagebreak[3]
40:~Result~:=~'ELOOP~40~Muitos~níveis~de~links~simbólicos';\\\nopagebreak[3]
41:~Result~:=~'EWOULDBLOCK~41~Recurso~temporariamente~indisponível';\\\nopagebreak[3]
42:~Result~:=~'ENOMSG~42~Não~há~mensagens~do~tipo~desejado';\\\nopagebreak[3]
43:~Result~:=~'EIDRM~43~Identificador~removido';\\\nopagebreak[3]
44:~Result~:=~'ECHRNG~44~Número~do~canal~fora~do~intervalo';\\\nopagebreak[3]
45:~Result~:=~'EL2NSYNC~45~Nível~2~não~sincronizado';\\\nopagebreak[3]
46:~Result~:=~'EL3HLT~46~Nível~3~parado';\\\nopagebreak[3]
47:~Result~:=~'EL3RST~47~Nível~3~reiniciado';\\\nopagebreak[3]
48:~Result~:=~'ELNRNG~48~Número~de~link~fora~da~faixa';\\\nopagebreak[3]
49:~Result~:=~'EUNATCH~49~Driver~de~protocolo~não~anexado';\\\nopagebreak[3]
50:~Result~:=~'ENOCSI~50~Não~há~estrutura~CSI~disponível';\\\nopagebreak[3]
51:~Result~:=~'EL2HLT~51~Parada~de~sistema~nível~2';\\\nopagebreak[3]
52:~Result~:=~'EBADE~52~Troca~inválida';\\\nopagebreak[3]
53:~Result~:=~'EBADR~53~Descritor~de~requisição~inválido';\\\nopagebreak[3]
54:~Result~:=~'EXFULL~54~Troca~completa';\\\nopagebreak[3]
55:~Result~:=~'ENOANO~55~Sem~anode';\\\nopagebreak[3]
56:~Result~:=~'EBADRQC~56~Código~de~requisição~inválido';\\\nopagebreak[3]
57:~Result~:=~'EBADSLT~57~Slot~inválido';\\\nopagebreak[3]
58:~Result~:=~'EDEADLOCK~35~Evitado~deadlock~de~recurso';\\\nopagebreak[3]
59:~Result~:=~'EBFONT~59~Formato~do~FileName~fonte~inválido';\\\nopagebreak[3]
60:~Result~:=~'ENOSTR~60~Dispositivo~não~é~um~stream';\\\nopagebreak[3]
61:~Result~:=~'ENODATA~61~Não~há~dados~disponíveis';\\\nopagebreak[3]
62:~Result~:=~'ETIME~62~Tempo~expirado';\\\nopagebreak[3]
63:~Result~:=~'ENOSR~63~Sem~recursos~de~streams';\\\nopagebreak[3]
64:~Result~:=~'ENOSR~64~Sem~recursos~de~streams';\\\nopagebreak[3]
65:~Result~:=~'ENOPKG~65~Pacote~não~instalado';\\\nopagebreak[3]
66:~Result~:=~'EREMOTE~66~ObjectName~é~remoto';\\\nopagebreak[3]
67:~Result~:=~'ENOLINK~67~Link~foi~cortado';\\\nopagebreak[3]
68:~Result~:=~'EADV~68~Erro~de~aviso';\\\nopagebreak[3]
69:~Result~:=~'ESRMNT~69~Erro~de~sr~mount';\\\nopagebreak[3]
70:~Result~:=~'COMM~70~Erro~de~comunicação~ao~enviar';\\\nopagebreak[3]
71:~Result~:=~'EPROTO~71~Erro~de~protocolo';\\\nopagebreak[3]
72:~Result~:=~'EMULTIHOP~72~Tentativa~de~Multi~hop';\\\nopagebreak[3]
73:~Result~:=~'EDOTDOT~73~Erro~específico~de~RFS';\\\nopagebreak[3]
74:~Result~:=~'EBADMSG~74~Mensagem~inválida';\\\nopagebreak[3]
75:~Result~:=~'EOVERFLOW~75~Valor~muito~grande~para~o~tipo~de~dados~definido';\\\nopagebreak[3]
76:~Result~:=~'ENOTUNIQ~76~O~nome~não~é~único~na~rede';\\\nopagebreak[3]
77:~Result~:=~'EBADFD~77~Descritor~de~FileName~em~mal~estado';\\\nopagebreak[3]
78:~Result~:=~'EREMCHG~78~Endereço~remoto~alterado';\\\nopagebreak[3]
79:~Result~:=~'ELIBACC~79~Não~foi~possível~acessar~uma~biblioteca~compartilhada';\\\nopagebreak[3]
80:~Result~:=~'ELIBBAD~80~Acessando~uma~biblioteca~compartilhado~corrompida';\\\nopagebreak[3]
81:~Result~:=~'ELIBSCN~81~Seção~.lib~corrompida~em~a.out';\\\nopagebreak[3]
82:~Result~:=~'ELIBMAX~82~Tentando~vincular~em~muitas~bibliotecas~compartilhadas';\\\nopagebreak[3]
83:~Result~:=~'ELIBEXEC~83~Não~foi~possível~executar~uma~biblioteca~compartilhado~diretamente';\\\nopagebreak[3]
84:~Result~:=~'EILSEQ~84~Multi~byte~ou~caractere~largo~inválido';\\\nopagebreak[3]
85:~Result~:=~'ERESTART~85~Chamada~de~sistema~interrompida~deve~ser~reiniciada';\\\nopagebreak[3]
86:~Result~:=~'ESTRPIPE~86~Erro~de~fluxos~de~pipe';\\\nopagebreak[3]
87:~Result~:=~'EUSERS~87~Muitos~usuários';\\\nopagebreak[3]
88:~Result~:=~'ENOTSOCK~88~Operação~socket~em~um~FileName~não-socket';\\\nopagebreak[3]
89:~Result~:=~'EDESTADDRREQ~89~Endereço~de~destino~requerido';\\\nopagebreak[3]
90:~Result~:=~'EMSGSIZE~90~Mensagem~muito~longa';\\\nopagebreak[3]
91:~Result~:=~'EPROTOTYPE~91~Tipo~errado~de~protocolo~para~socket';\\\nopagebreak[3]
92:~Result~:=~'ENOPROTOOPT~92~Protocolo~não~disponível';\\\nopagebreak[3]
93:~Result~:=~'EPROTONOSUPPORT~93~Protocolo~sem~suporte';\\\nopagebreak[3]
94:~Result~:=~'ESOCKTNOSUPPORT~94~Tipo~socket~sem~suporte';\\\nopagebreak[3]
95:~Result~:=~'EOPNOTSUPP~95~Operação~sem~suporte';\\\nopagebreak[3]
96:~Result~:=~'EPFNOSUPPORT~96~Família~de~protocolo~sem~suporte';\\\nopagebreak[3]
97:~Result~:=~'EAFNOSUPPORT~97~Família~de~endereços~sem~suporte~pelo~protocolo';\\\nopagebreak[3]
98:~Result~:=~'EADDRINUSE~98~Endereço~já~em~uso';\\\nopagebreak[3]
99:~Result~:=~'EADDRNOTAVAIL~99~Não~foi~possível~acessar~o~endereço~requisitado';\\\nopagebreak[3]
\\\nopagebreak[3]
\textit{{\{}100~A~149~ERROS~DE~ENTRADA~E~SAIDA~(I/O){\}}}\\\nopagebreak[3]
100:~Result~:=~'100:~Erro~ao~ler~o~disco';\textit{//EndOfFile~no~delphi}\\\nopagebreak[3]
101:~Result~:=~'101:~Erro~ao~gravar~no~disco';\textit{//DiskFull~no~delphi}\\\nopagebreak[3]
102:~Result~:=~'102:~FileName~não~assinalado~(falta~ASSIGN)~';\\\nopagebreak[3]
103:~Result~:=~'103:~FileName~fechado';\\\nopagebreak[3]
104:~Result~:=~'104:~O~FileName~fechado~para~entrada';\\\nopagebreak[3]
105:~Result~:=~'105:~O~FileName~fechado~para~saída';\\\nopagebreak[3]
106:~Result~:=~'106:~Formato~numérico~inválido~ou~incompatÍvel';\\\nopagebreak[3]
107:~Result~:=~'107:~Disco~cheio';\\\nopagebreak[3]
108..149:\\\nopagebreak[3]
~~~~Result~:=~'108..149:~Reservado~para~I/O~';\\\nopagebreak[3]
\\\nopagebreak[3]
\textit{{\{}150..199~RESERVADO~PARA~ERROS~CRITICOS{\}}}\\\nopagebreak[3]
150:~Result~:=~'150:~Disco~protegido';\\\nopagebreak[3]
151:~Result~:=~'151:~UNIT~desconhecida';\\\nopagebreak[3]
152:~Result~:=~'152:~Drive~não~esta~pronto';\\\nopagebreak[3]
153:~Result~:=~'153:~Comando~desconhecido';\\\nopagebreak[3]
154:~Result~:=~'154:~Erro~na~CRC~de~dados';\\\nopagebreak[3]
155:~Result~:=~'155:~Erro~no~drive~solicitado~pelo~tamanho';\\\nopagebreak[3]
156:~Result~:=~'156:~Erro~no~posicionamento~de~disco';\\\nopagebreak[3]
157:~Result~:=~'157:~Tipo~de~meio~desconhecido';\\\nopagebreak[3]
158:~Result~:=~'158:~Setor~não~encontrado';\\\nopagebreak[3]
159:~Result~:=~'159:~Impressora~sem~papel';\\\nopagebreak[3]
160:~Result~:=~'160:~Erro~de~escrita~no~dispositivo~de~saída~(~Impressora~)';\\\nopagebreak[3]
161:~Result~:=~'161:~Falta~dispositivo~de~entrada~(~Leitura~)';\\\nopagebreak[3]
162:~Result~:=~'162:~Falta~hardware~(~equipamento~)';\\\nopagebreak[3]
163..199:\\\nopagebreak[3]
~~~~Result~:=~'163..199:~RESERVADO~PARA~ERROS~CRITICOS';\\\nopagebreak[3]
\\\nopagebreak[3]
\textit{{\{}200..255~RESERVADO~PARA~ERROS~FATAL{\}}}\\\nopagebreak[3]
200:~Result~:=~'200:~Divisão~por~zero';\\\nopagebreak[3]
201:~Result~:=~'201:~Error~na~checagem~de~faixa';\\\nopagebreak[3]
202:~Result~:=~'202:~Estouro~no~stack~de~memória';\\\nopagebreak[3]
203:~Result~:=~'203:~Estouro~no~heap~de~memória';\\\nopagebreak[3]
204:~Result~:=~'204:~Operação~de~pointer~inválida';\\\nopagebreak[3]
205:~Result~:=~'205:~Estouro~em~operação~com~ponto~flutuante';\\\nopagebreak[3]
206:~Result~:=~'206:~Erro~de~underflow~com~ponto-flutuante~(Somente~com~8087)~';\\\nopagebreak[3]
207:~Result~:=~'207:~Operação~inválida~com~ponto~flutuante';\\\nopagebreak[3]
208:~Result~:=~'208:~Gerenciador~de~Overlay~não~instalado';\\\nopagebreak[3]
209:~Result~:=~'209:~Erro~da~leitura~no~FileName~de~overlay';\\\nopagebreak[3]
210:~Result~:=~'210:~ObjectName~não~inicializado';\\\nopagebreak[3]
211:~Result~:=~'211:~Chamada~a~um~MethodName~abstrato';\\\nopagebreak[3]
212:~Result~:=~'212:~Stream~registration~error';\\\nopagebreak[3]
\\\nopagebreak[3]
213:~Result~:=~'213:~Collection~index~out~of~range';\\\nopagebreak[3]
214:~Result~:=~'214:~Collection~overflow~error';\\\nopagebreak[3]
215:~Result~:=~'215:~Arithmetic~overflow~error';\\\nopagebreak[3]
216:~Result~:=~'216:~General~Protection~fault';\\\nopagebreak[3]
\\\nopagebreak[3]
\textit{{\{}MarIcarai~217..255{\}}}\\\nopagebreak[3]
217:~Result~:=~'217:~Tentativa~de~abrir~um~FileName~aberto.';\\\nopagebreak[3]
218:~Result~:=~'218:~Tentativa~de~excluir~um~registro~excluido';\\\nopagebreak[3]
219:~Result~:=~'219:~Tentativa~de~ler~um~registro~excluído';\\\nopagebreak[3]
220:~Result~:=~'220:~Outro~usuário~da~rede~alterou~o~registro';\\\nopagebreak[3]
221:~Result~:=~'221:~Estrutura~da~tabela~esta~danificada';\\\nopagebreak[3]
222:~Result~:=~'222:~Tentativa~de~gravar~em~um~registro~compartilhado~sem~que~o~mesmo~esteja~travado';\\\nopagebreak[3]
\\\nopagebreak[3]
\textit{{\{}ApCLiSvr.pas{\}}}\\\nopagebreak[3]
223:~Result~:=~'223:~Evento~executado~por~outro~processo';\\\nopagebreak[3]
224:~Result~:=~'224:~Servidor~de~API~não~instalado';\\\nopagebreak[3]
\\\nopagebreak[3]
225:~Result~:=~'225:~TTransaction.Commit~esperado.';\textit{//TTransaction}\\\nopagebreak[3]
226:~Result~:=~'226:~Não~é~uma~expressão~válida';\\\nopagebreak[3]
227:~Result~:=~'227:~Muitos~parentese~na~expressão';\\\nopagebreak[3]
228:~Result~:=~'228:~Muitos~operadores~na~expressão';\\\nopagebreak[3]
229:~Result~:=~'229:~Operador~aritmético~esperado~na~expressão';\\\nopagebreak[3]
230:~Result~:=~'230:~O~Número~não~pode~ser~lido~na~expressão';\\\nopagebreak[3]
\\\nopagebreak[3]
REC{\_}TOO{\_}LARGE~~~:~~Result~~:=~'Tamanho~do~registro~em~memória~maior~que~o~permitido~(MaxDataRecSize)';\\\nopagebreak[3]
REC{\_}TOO{\_}SMALL~~~:~~Result~~:=~'Tamanho~do~Registro~e~muito~longo';\\\nopagebreak[3]
KeyTooLarge~~~~~:~~Result~~:=~'Tamanho~da~chave~maior~do~que~o~maximo~permitido~(MaxKeyLen)~';\\\nopagebreak[3]
RecSizeMismatch~:~~Result~~:=~'Registro~de~dados~incompatÍvel~com~a~estrutura~corrente.~';\\\nopagebreak[3]
KeySizeMismatch~:~~Result~~:=~'Tamanho~da~pagina~ou~chave~incompatÍvel~com~a~estrutura~corrente';\\\nopagebreak[3]
MemOverflow~~~~~:~~Result~~:=~'Não~ha~memória~para~os~indice~dos~FileNames';\\\nopagebreak[3]
ArqIndexInconsistente\\\nopagebreak[3]
~~~~~~~~~~~~~~~~:~~Result~~:=~'FileName~de~indice~inconsistente.';\\\nopagebreak[3]
238~~~~~~~~~~~~~:~~Result~:=~'238:~O~gerente~de~transacoes~esta~inativo';~\textit{//TTransaction}\\\nopagebreak[3]
239~~~~~~~~~~~~~:~~Result~:=~'239:~Excecao~inesperada';\\\nopagebreak[3]
240:~Result~~~~~:=~'240:~Acesso~negado~ao~FileName~por~falta~de~autorizacao~de~seu~superior~imediato';\\\nopagebreak[3]
241:~Result~~~~~:=~'241:~Registro~não~localizado';~\textit{//~Erros~retornados~nas~buscas~de~registros}\\\nopagebreak[3]
242:~Result~~~~~:=~'242:~O~evento~OnEnter~Retornou~falso';\\\nopagebreak[3]
243:~Result~~~~~:=~'242:~O~evento~OnExit~Retornou~falso';\\\nopagebreak[3]
244\\\nopagebreak[3]
245~~~~~~~~~~~~~:=~'245~attempt{\_}to{\_}edit{\_}a{\_}record{\_}not{\_}selecting'\\\nopagebreak[3]
..\\\nopagebreak[3]
254:~Result~:=~'238..254:~RESERVADO~PARA~ERROS~FATAIS';\\\nopagebreak[3]
255:~Result~:=~'255:~{\^{}}C.~Sistema~abortado.';\\\nopagebreak[3]
\\\nopagebreak[3]
}\textbf{ELSE}\texttt{~Result~:=~'Erro~indefinido~Maior~que~255';\\
}
\item \textbf{REFERÊNCIAS}: \begin{itemize}
\item Lista de erros do Lazarus: https://wiki.lazarus.freepascal.org/RunError
\item FreePascal {-} acesso a FileNames. : https://www.freepascal.org/docs-html/rtl/system/ioresult.html
\item windows: https://docs.microsoft.com/pt-br/cpp/c-runtime-library/errno-constants?view=msvc-170
\item DOS: https://docs.microsoft.com/pt-br/cpp/c-runtime-library/errno-doserrno-sys-errlist-and-sys-nerr?view=msvc-170
\item Linux: https://man7.org/linux/man-pages/man3/errno.3.html
\end{itemize}
\end{itemize}

\end{list}
\paragraph*{ErrorMessage}\hspace*{\fill}

\begin{list}{}{
\settowidth{\tmplength}{\textbf{Declaração}}
\setlength{\itemindent}{0cm}
\setlength{\listparindent}{0cm}
\setlength{\leftmargin}{\evensidemargin}
\addtolength{\leftmargin}{\tmplength}
\settowidth{\labelsep}{X}
\addtolength{\leftmargin}{\labelsep}
\setlength{\labelwidth}{\tmplength}
}
\begin{flushleft}
\item[\textbf{Declaração}\hfill]
\begin{ttfamily}
public class function ErrorMessage(Const ErrorMsg : AnsiString) : AnsiString; overload;\end{ttfamily}


\end{flushleft}
\par
\item[\textbf{Descrição}]
O método \textbf{\begin{ttfamily}ErrorMessage\end{ttfamily}} receber uma mensagem e retorna a mesagem formatada.

\end{list}
\paragraph*{ErrorMessage}\hspace*{\fill}

\begin{list}{}{
\settowidth{\tmplength}{\textbf{Declaração}}
\setlength{\itemindent}{0cm}
\setlength{\listparindent}{0cm}
\setlength{\leftmargin}{\evensidemargin}
\addtolength{\leftmargin}{\tmplength}
\settowidth{\labelsep}{X}
\addtolength{\leftmargin}{\labelsep}
\setlength{\labelwidth}{\tmplength}
}
\begin{flushleft}
\item[\textbf{Declaração}\hfill]
\begin{ttfamily}
public class function ErrorMessage(const Sender: TObject; Const ErrorMsg : AnsiString) : AnsiString; overload;\end{ttfamily}


\end{flushleft}
\end{list}
\paragraph*{ErrorMessage}\hspace*{\fill}

\begin{list}{}{
\settowidth{\tmplength}{\textbf{Declaração}}
\setlength{\itemindent}{0cm}
\setlength{\listparindent}{0cm}
\setlength{\leftmargin}{\evensidemargin}
\addtolength{\leftmargin}{\tmplength}
\settowidth{\labelsep}{X}
\addtolength{\leftmargin}{\labelsep}
\setlength{\labelwidth}{\tmplength}
}
\begin{flushleft}
\item[\textbf{Declaração}\hfill]
\begin{ttfamily}
public class function ErrorMessage(const Sender: TObject;Const aMethodName, aFileName, AFieldName:AnsiString;aMsg:AnsiString):AnsiString; Overload;\end{ttfamily}


\end{flushleft}
\end{list}
\paragraph*{ErrorMessage}\hspace*{\fill}

\begin{list}{}{
\settowidth{\tmplength}{\textbf{Declaração}}
\setlength{\itemindent}{0cm}
\setlength{\listparindent}{0cm}
\setlength{\leftmargin}{\evensidemargin}
\addtolength{\leftmargin}{\tmplength}
\settowidth{\labelsep}{X}
\addtolength{\leftmargin}{\labelsep}
\setlength{\labelwidth}{\tmplength}
}
\begin{flushleft}
\item[\textbf{Declaração}\hfill]
\begin{ttfamily}
public class function ErrorMessage(const Sender: TObject;Const aMethodName, aFileName, AFieldName:AnsiString;aCodError:SmallInt):AnsiString; Overload;\end{ttfamily}


\end{flushleft}
\end{list}
\paragraph*{ErrorMessage8}\hspace*{\fill}

\begin{list}{}{
\settowidth{\tmplength}{\textbf{Declaração}}
\setlength{\itemindent}{0cm}
\setlength{\listparindent}{0cm}
\setlength{\leftmargin}{\evensidemargin}
\addtolength{\leftmargin}{\tmplength}
\settowidth{\labelsep}{X}
\addtolength{\leftmargin}{\labelsep}
\setlength{\labelwidth}{\tmplength}
}
\begin{flushleft}
\item[\textbf{Declaração}\hfill]
\begin{ttfamily}
public class function ErrorMessage8(Const aModule, aUnit, aObjectName, aMethodName, aFileName, AFieldName, aMessage, aProcedure{\_}or{\_}Function :AnsiString):AnsiString; Overload;\end{ttfamily}


\end{flushleft}
\end{list}
\paragraph*{ErrorMessage7}\hspace*{\fill}

\begin{list}{}{
\settowidth{\tmplength}{\textbf{Declaração}}
\setlength{\itemindent}{0cm}
\setlength{\listparindent}{0cm}
\setlength{\leftmargin}{\evensidemargin}
\addtolength{\leftmargin}{\tmplength}
\settowidth{\labelsep}{X}
\addtolength{\leftmargin}{\labelsep}
\setlength{\labelwidth}{\tmplength}
}
\begin{flushleft}
\item[\textbf{Declaração}\hfill]
\begin{ttfamily}
public class function ErrorMessage7(aModule, aUnit, aObjectName, aMethodName, aFileName, AFieldName:AnsiString; aCodError:SmallInt):AnsiString; Overload;\end{ttfamily}


\end{flushleft}
\end{list}
\paragraph*{ErrorMessage7}\hspace*{\fill}

\begin{list}{}{
\settowidth{\tmplength}{\textbf{Declaração}}
\setlength{\itemindent}{0cm}
\setlength{\listparindent}{0cm}
\setlength{\leftmargin}{\evensidemargin}
\addtolength{\leftmargin}{\tmplength}
\settowidth{\labelsep}{X}
\addtolength{\leftmargin}{\labelsep}
\setlength{\labelwidth}{\tmplength}
}
\begin{flushleft}
\item[\textbf{Declaração}\hfill]
\begin{ttfamily}
public class function ErrorMessage7(aModule, aUnit, aObjectName, aMethodName, aFileName, AFieldName:AnsiString; aMessage:AnsiString):AnsiString; Overload;\end{ttfamily}


\end{flushleft}
\end{list}
\paragraph*{ErrorMessage6}\hspace*{\fill}

\begin{list}{}{
\settowidth{\tmplength}{\textbf{Declaração}}
\setlength{\itemindent}{0cm}
\setlength{\listparindent}{0cm}
\setlength{\leftmargin}{\evensidemargin}
\addtolength{\leftmargin}{\tmplength}
\settowidth{\labelsep}{X}
\addtolength{\leftmargin}{\labelsep}
\setlength{\labelwidth}{\tmplength}
}
\begin{flushleft}
\item[\textbf{Declaração}\hfill]
\begin{ttfamily}
public class function ErrorMessage6(aModule, aObjectName, aMethodName, aFileName, AFieldName:AnsiString; aCodError:SmallInt):AnsiString; Overload;\end{ttfamily}


\end{flushleft}
\end{list}
\paragraph*{ErrorMessage5}\hspace*{\fill}

\begin{list}{}{
\settowidth{\tmplength}{\textbf{Declaração}}
\setlength{\itemindent}{0cm}
\setlength{\listparindent}{0cm}
\setlength{\leftmargin}{\evensidemargin}
\addtolength{\leftmargin}{\tmplength}
\settowidth{\labelsep}{X}
\addtolength{\leftmargin}{\labelsep}
\setlength{\labelwidth}{\tmplength}
}
\begin{flushleft}
\item[\textbf{Declaração}\hfill]
\begin{ttfamily}
public class function ErrorMessage5(aModule, aUnit, aObjectName, aMethodName :AnsiString; aCodError:SmallInt):AnsiString; Overload;\end{ttfamily}


\end{flushleft}
\end{list}
\paragraph*{ErrorMessage5}\hspace*{\fill}

\begin{list}{}{
\settowidth{\tmplength}{\textbf{Declaração}}
\setlength{\itemindent}{0cm}
\setlength{\listparindent}{0cm}
\setlength{\leftmargin}{\evensidemargin}
\addtolength{\leftmargin}{\tmplength}
\settowidth{\labelsep}{X}
\addtolength{\leftmargin}{\labelsep}
\setlength{\labelwidth}{\tmplength}
}
\begin{flushleft}
\item[\textbf{Declaração}\hfill]
\begin{ttfamily}
public class function ErrorMessage5(aModule, aUnit, aObjectName, aMethodName :AnsiString; aMsgError:AnsiString):AnsiString; Overload;\end{ttfamily}


\end{flushleft}
\end{list}
\paragraph*{ErrorMessage4}\hspace*{\fill}

\begin{list}{}{
\settowidth{\tmplength}{\textbf{Declaração}}
\setlength{\itemindent}{0cm}
\setlength{\listparindent}{0cm}
\setlength{\leftmargin}{\evensidemargin}
\addtolength{\leftmargin}{\tmplength}
\settowidth{\labelsep}{X}
\addtolength{\leftmargin}{\labelsep}
\setlength{\labelwidth}{\tmplength}
}
\begin{flushleft}
\item[\textbf{Declaração}\hfill]
\begin{ttfamily}
public class function ErrorMessage4(Const aModule, aUnit, Procedure{\_}or{\_}function:AnsiString; aCodError:SmallInt):AnsiString; Overload;\end{ttfamily}


\end{flushleft}
\par
\item[\textbf{Descrição}]
A class \textbf{\begin{ttfamily}ErrorMessage4\end{ttfamily}} formata um texto com os parãmetros passado

\end{list}
\paragraph*{ErrorMessage4}\hspace*{\fill}

\begin{list}{}{
\settowidth{\tmplength}{\textbf{Declaração}}
\setlength{\itemindent}{0cm}
\setlength{\listparindent}{0cm}
\setlength{\leftmargin}{\evensidemargin}
\addtolength{\leftmargin}{\tmplength}
\settowidth{\labelsep}{X}
\addtolength{\leftmargin}{\labelsep}
\setlength{\labelwidth}{\tmplength}
}
\begin{flushleft}
\item[\textbf{Declaração}\hfill]
\begin{ttfamily}
public class function ErrorMessage4(Const aModule, aUnit, Procedure{\_}or{\_}function:AnsiString; aMessage:AnsiString):AnsiString; Overload;\end{ttfamily}


\end{flushleft}
\par
\item[\textbf{Descrição}]
A class \textbf{\begin{ttfamily}ErrorMessage4\end{ttfamily}} formata um texto com os parãmetros passado

\end{list}
\chapter{Unit mi.rtl.Consts.StringList}
\section{Uses}
\begin{itemize}
\item \begin{ttfamily}Classes\end{ttfamily}\item \begin{ttfamily}SysUtils\end{ttfamily}\item \begin{ttfamily}mi.rtl.consts.StringListBase\end{ttfamily}(\ref{mi.rtl.Consts.StringListBase})\item \begin{ttfamily}mi.rtl.Consts\end{ttfamily}(\ref{mi.rtl.Consts})\end{itemize}
\section{Visão Geral}
\begin{description}
\item[\texttt{\begin{ttfamily}TMiStringList\end{ttfamily} Classe}]
\end{description}
\section{Classes, Interfaces, Objetos e Registros}
\subsection*{TMiStringList Classe}
\subsubsection*{\large{\textbf{Hierarquia}}\normalsize\hspace{1ex}\hfill}
TMiStringList {$>$} \begin{ttfamily}TStringListBase\end{ttfamily}(\ref{mi.rtl.Consts.StringListBase.TStringListBase}) {$>$} 
TStringList
\subsubsection*{\large{\textbf{Descrição}}\normalsize\hspace{1ex}\hfill}
A class \textbf{\begin{ttfamily}TMiStringList\end{ttfamily}} implementa a navegação como se tivesse navegando em arquivos usando os métodos \begin{ttfamily}NextKey\end{ttfamily}(\ref{mi.rtl.Consts.StringListBase.TStringListBase-NextKey}),\begin{ttfamily}Prevkey\end{ttfamily}(\ref{mi.rtl.Consts.StringList.TMiStringList-PrevKey}) etc...

\begin{itemize}
\item \textbf{NOTA} \begin{itemize}
\item Usando quando quero manter uma lista de registros ordenada.
\end{itemize}
\end{itemize}\subsubsection*{\large{\textbf{Métodos}}\normalsize\hspace{1ex}\hfill}
\paragraph*{AddKey}\hspace*{\fill}

\begin{list}{}{
\settowidth{\tmplength}{\textbf{Declaração}}
\setlength{\itemindent}{0cm}
\setlength{\listparindent}{0cm}
\setlength{\leftmargin}{\evensidemargin}
\addtolength{\leftmargin}{\tmplength}
\settowidth{\labelsep}{X}
\addtolength{\leftmargin}{\labelsep}
\setlength{\labelwidth}{\tmplength}
}
\begin{flushleft}
\item[\textbf{Declaração}\hfill]
\begin{ttfamily}
public Function AddKey(WKey:String;wNr:Longint):Boolean;\end{ttfamily}


\end{flushleft}
\end{list}
\paragraph*{BofKey}\hspace*{\fill}

\begin{list}{}{
\settowidth{\tmplength}{\textbf{Declaração}}
\setlength{\itemindent}{0cm}
\setlength{\listparindent}{0cm}
\setlength{\leftmargin}{\evensidemargin}
\addtolength{\leftmargin}{\tmplength}
\settowidth{\labelsep}{X}
\addtolength{\leftmargin}{\labelsep}
\setlength{\labelwidth}{\tmplength}
}
\begin{flushleft}
\item[\textbf{Declaração}\hfill]
\begin{ttfamily}
public Function BofKey: Boolean;\end{ttfamily}


\end{flushleft}
\par
\item[\textbf{Descrição}]
Posiciona no inicio do bloco de registro do tipo default

\end{list}
\paragraph*{LastKey}\hspace*{\fill}

\begin{list}{}{
\settowidth{\tmplength}{\textbf{Declaração}}
\setlength{\itemindent}{0cm}
\setlength{\listparindent}{0cm}
\setlength{\leftmargin}{\evensidemargin}
\addtolength{\leftmargin}{\tmplength}
\settowidth{\labelsep}{X}
\addtolength{\leftmargin}{\labelsep}
\setlength{\labelwidth}{\tmplength}
}
\begin{flushleft}
\item[\textbf{Declaração}\hfill]
\begin{ttfamily}
public Function LastKey: Boolean;\end{ttfamily}


\end{flushleft}
\par
\item[\textbf{Descrição}]
Posiciona no fin do bloco de registro do tipo default

\end{list}
\paragraph*{EofKey}\hspace*{\fill}

\begin{list}{}{
\settowidth{\tmplength}{\textbf{Declaração}}
\setlength{\itemindent}{0cm}
\setlength{\listparindent}{0cm}
\setlength{\leftmargin}{\evensidemargin}
\addtolength{\leftmargin}{\tmplength}
\settowidth{\labelsep}{X}
\addtolength{\leftmargin}{\labelsep}
\setlength{\labelwidth}{\tmplength}
}
\begin{flushleft}
\item[\textbf{Declaração}\hfill]
\begin{ttfamily}
public Function EofKey: Boolean;\end{ttfamily}


\end{flushleft}
\par
\item[\textbf{Descrição}]
Posiciona no fin do bloco de registro do tipo default

\end{list}
\paragraph*{PrevKey}\hspace*{\fill}

\begin{list}{}{
\settowidth{\tmplength}{\textbf{Declaração}}
\setlength{\itemindent}{0cm}
\setlength{\listparindent}{0cm}
\setlength{\leftmargin}{\evensidemargin}
\addtolength{\leftmargin}{\tmplength}
\settowidth{\labelsep}{X}
\addtolength{\leftmargin}{\labelsep}
\setlength{\labelwidth}{\tmplength}
}
\begin{flushleft}
\item[\textbf{Declaração}\hfill]
\begin{ttfamily}
public Function PrevKey: Boolean;\end{ttfamily}


\end{flushleft}
\par
\item[\textbf{Descrição}]
Posiciona no registro anterior do tipo default

\end{list}
\chapter{Unit mi.rtl.Consts.StringListBase}
\section{Descrição}
{-}A unit \textbf{\begin{ttfamily}mi.rtl.Consts.StringListBase\end{ttfamily}} implementa a classe \begin{ttfamily}TStringListBase\end{ttfamily}(\ref{mi.rtl.Consts.StringListBase.TStringListBase}) do pacote \begin{ttfamily}mi.rtl\end{ttfamily}(\ref{mi.rtl}).

\begin{itemize}
\item \textbf{VERSÃO}: \begin{itemize}
\item Alpha {-} 0.5.0.687
\end{itemize}
\item \textbf{CÓDIGO FONTE}: \begin{itemize}
\item 
\end{itemize}
\item \textbf{HISTÓRICO} \begin{itemize}
\item Criado por: Paulo Sérgio da Silva Pacheco e{-}mail: paulosspacheco@yahoo.com.br \begin{itemize}
\item \textbf{2022{-}01{-}25} {-}08:00 a 12:00 {-} Criado a unit \textbf{\begin{ttfamily}mi.rtl.Consts.StringListBase\end{ttfamily}} e implementação da classe \textbf{\begin{ttfamily}TStringListBase\end{ttfamily}(\ref{mi.rtl.Consts.StringListBase.TStringListBase})}
\end{itemize}
\item \textbf{2022{-}05{-}17} \begin{itemize}
\item T12 Criar método CopyFrom
\end{itemize}
\end{itemize}
\end{itemize}
\section{Uses}
\begin{itemize}
\item \begin{ttfamily}Classes\end{ttfamily}\item \begin{ttfamily}SysUtils\end{ttfamily}\item \begin{ttfamily}mi.rtl.Consts\end{ttfamily}(\ref{mi.rtl.Consts})\end{itemize}
\section{Visão Geral}
\begin{description}
\item[\texttt{\begin{ttfamily}TStringListBase\end{ttfamily} Classe}]
\end{description}
\section{Classes, Interfaces, Objetos e Registros}
\subsection*{TStringListBase Classe}
\subsubsection*{\large{\textbf{Hierarquia}}\normalsize\hspace{1ex}\hfill}
TStringListBase {$>$} TStringList
%%%%Descrição
\subsubsection*{\large{\textbf{Propriedades}}\normalsize\hspace{1ex}\hfill}
\paragraph*{KeyMaster}\hspace*{\fill}

\begin{list}{}{
\settowidth{\tmplength}{\textbf{Declaração}}
\setlength{\itemindent}{0cm}
\setlength{\listparindent}{0cm}
\setlength{\leftmargin}{\evensidemargin}
\addtolength{\leftmargin}{\tmplength}
\settowidth{\labelsep}{X}
\addtolength{\leftmargin}{\labelsep}
\setlength{\labelwidth}{\tmplength}
}
\begin{flushleft}
\item[\textbf{Declaração}\hfill]
\begin{ttfamily}
public property KeyMaster : String Read {\_}KeyMaster Write SetKeyMaster;\end{ttfamily}


\end{flushleft}
\end{list}
\paragraph*{OkDestroy{\_}Object}\hspace*{\fill}

\begin{list}{}{
\settowidth{\tmplength}{\textbf{Declaração}}
\setlength{\itemindent}{0cm}
\setlength{\listparindent}{0cm}
\setlength{\leftmargin}{\evensidemargin}
\addtolength{\leftmargin}{\tmplength}
\settowidth{\labelsep}{X}
\addtolength{\leftmargin}{\labelsep}
\setlength{\labelwidth}{\tmplength}
}
\begin{flushleft}
\item[\textbf{Declaração}\hfill]
\begin{ttfamily}
public property OkDestroy{\_}Object : Boolean Write {\_}OkDestroy{\_}Object;\end{ttfamily}


\end{flushleft}
\par
\item[\textbf{Descrição}]
Redefini para poder deletar o objeto associando

\end{list}
\subsubsection*{\large{\textbf{Campos}}\normalsize\hspace{1ex}\hfill}
\paragraph*{Index{\_}Currente}\hspace*{\fill}

\begin{list}{}{
\settowidth{\tmplength}{\textbf{Declaração}}
\setlength{\itemindent}{0cm}
\setlength{\listparindent}{0cm}
\setlength{\leftmargin}{\evensidemargin}
\addtolength{\leftmargin}{\tmplength}
\settowidth{\labelsep}{X}
\addtolength{\leftmargin}{\labelsep}
\setlength{\labelwidth}{\tmplength}
}
\begin{flushleft}
\item[\textbf{Declaração}\hfill]
\begin{ttfamily}
protected Index{\_}Currente: Integer;\end{ttfamily}


\end{flushleft}
\end{list}
\paragraph*{OkBof}\hspace*{\fill}

\begin{list}{}{
\settowidth{\tmplength}{\textbf{Declaração}}
\setlength{\itemindent}{0cm}
\setlength{\listparindent}{0cm}
\setlength{\leftmargin}{\evensidemargin}
\addtolength{\leftmargin}{\tmplength}
\settowidth{\labelsep}{X}
\addtolength{\leftmargin}{\labelsep}
\setlength{\labelwidth}{\tmplength}
}
\begin{flushleft}
\item[\textbf{Declaração}\hfill]
\begin{ttfamily}
protected OkBof: Boolean;\end{ttfamily}


\end{flushleft}
\par
\item[\textbf{Descrição}]
Inicio de arquivo

\end{list}
\paragraph*{okEof}\hspace*{\fill}

\begin{list}{}{
\settowidth{\tmplength}{\textbf{Declaração}}
\setlength{\itemindent}{0cm}
\setlength{\listparindent}{0cm}
\setlength{\leftmargin}{\evensidemargin}
\addtolength{\leftmargin}{\tmplength}
\settowidth{\labelsep}{X}
\addtolength{\leftmargin}{\labelsep}
\setlength{\labelwidth}{\tmplength}
}
\begin{flushleft}
\item[\textbf{Declaração}\hfill]
\begin{ttfamily}
protected okEof: Boolean;\end{ttfamily}


\end{flushleft}
\par
\item[\textbf{Descrição}]
Fim de arquivo

\end{list}
\paragraph*{Nr}\hspace*{\fill}

\begin{list}{}{
\settowidth{\tmplength}{\textbf{Declaração}}
\setlength{\itemindent}{0cm}
\setlength{\listparindent}{0cm}
\setlength{\leftmargin}{\evensidemargin}
\addtolength{\leftmargin}{\tmplength}
\settowidth{\labelsep}{X}
\addtolength{\leftmargin}{\labelsep}
\setlength{\labelwidth}{\tmplength}
}
\begin{flushleft}
\item[\textbf{Declaração}\hfill]
\begin{ttfamily}
public Nr: PtrInt;\end{ttfamily}


\end{flushleft}
\par
\item[\textbf{Descrição}]
Número do registro corrente

\end{list}
\subsubsection*{\large{\textbf{Métodos}}\normalsize\hspace{1ex}\hfill}
\paragraph*{Create}\hspace*{\fill}

\begin{list}{}{
\settowidth{\tmplength}{\textbf{Declaração}}
\setlength{\itemindent}{0cm}
\setlength{\listparindent}{0cm}
\setlength{\leftmargin}{\evensidemargin}
\addtolength{\leftmargin}{\tmplength}
\settowidth{\labelsep}{X}
\addtolength{\leftmargin}{\labelsep}
\setlength{\labelwidth}{\tmplength}
}
\begin{flushleft}
\item[\textbf{Declaração}\hfill]
\begin{ttfamily}
public constructor Create(ALimit: longint;AOrdem:Boolean ); overload; virtual;\end{ttfamily}


\end{flushleft}
\end{list}
\paragraph*{ClearKey}\hspace*{\fill}

\begin{list}{}{
\settowidth{\tmplength}{\textbf{Declaração}}
\setlength{\itemindent}{0cm}
\setlength{\listparindent}{0cm}
\setlength{\leftmargin}{\evensidemargin}
\addtolength{\leftmargin}{\tmplength}
\settowidth{\labelsep}{X}
\addtolength{\leftmargin}{\labelsep}
\setlength{\labelwidth}{\tmplength}
}
\begin{flushleft}
\item[\textbf{Declaração}\hfill]
\begin{ttfamily}
public Function ClearKey: Boolean;\end{ttfamily}


\end{flushleft}
\end{list}
\paragraph*{IndexOf}\hspace*{\fill}

\begin{list}{}{
\settowidth{\tmplength}{\textbf{Declaração}}
\setlength{\itemindent}{0cm}
\setlength{\listparindent}{0cm}
\setlength{\leftmargin}{\evensidemargin}
\addtolength{\leftmargin}{\tmplength}
\settowidth{\labelsep}{X}
\addtolength{\leftmargin}{\labelsep}
\setlength{\labelwidth}{\tmplength}
}
\begin{flushleft}
\item[\textbf{Declaração}\hfill]
\begin{ttfamily}
public function IndexOf(const S: string): Integer; override;\end{ttfamily}


\end{flushleft}
\par
\item[\textbf{Descrição}]
Redefini porque a instância anterior não funciona com caractere {\#}254

\end{list}
\paragraph*{Delete}\hspace*{\fill}

\begin{list}{}{
\settowidth{\tmplength}{\textbf{Declaração}}
\setlength{\itemindent}{0cm}
\setlength{\listparindent}{0cm}
\setlength{\leftmargin}{\evensidemargin}
\addtolength{\leftmargin}{\tmplength}
\settowidth{\labelsep}{X}
\addtolength{\leftmargin}{\labelsep}
\setlength{\labelwidth}{\tmplength}
}
\begin{flushleft}
\item[\textbf{Declaração}\hfill]
\begin{ttfamily}
public procedure Delete(Index: Integer); override;\end{ttfamily}


\end{flushleft}
\par
\item[\textbf{Descrição}]
Redefini para poder deletar o objeto associando

\end{list}
\paragraph*{Destroy}\hspace*{\fill}

\begin{list}{}{
\settowidth{\tmplength}{\textbf{Declaração}}
\setlength{\itemindent}{0cm}
\setlength{\listparindent}{0cm}
\setlength{\leftmargin}{\evensidemargin}
\addtolength{\leftmargin}{\tmplength}
\settowidth{\labelsep}{X}
\addtolength{\leftmargin}{\labelsep}
\setlength{\labelwidth}{\tmplength}
}
\begin{flushleft}
\item[\textbf{Declaração}\hfill]
\begin{ttfamily}
public destructor Destroy; override;\end{ttfamily}


\end{flushleft}
\end{list}
\paragraph*{Find}\hspace*{\fill}

\begin{list}{}{
\settowidth{\tmplength}{\textbf{Declaração}}
\setlength{\itemindent}{0cm}
\setlength{\listparindent}{0cm}
\setlength{\leftmargin}{\evensidemargin}
\addtolength{\leftmargin}{\tmplength}
\settowidth{\labelsep}{X}
\addtolength{\leftmargin}{\labelsep}
\setlength{\labelwidth}{\tmplength}
}
\begin{flushleft}
\item[\textbf{Declaração}\hfill]
\begin{ttfamily}
public function Find(const S: string; Out Index: Integer): Boolean; override;\end{ttfamily}


\end{flushleft}
\end{list}
\paragraph*{FindKey}\hspace*{\fill}

\begin{list}{}{
\settowidth{\tmplength}{\textbf{Declaração}}
\setlength{\itemindent}{0cm}
\setlength{\listparindent}{0cm}
\setlength{\leftmargin}{\evensidemargin}
\addtolength{\leftmargin}{\tmplength}
\settowidth{\labelsep}{X}
\addtolength{\leftmargin}{\labelsep}
\setlength{\labelwidth}{\tmplength}
}
\begin{flushleft}
\item[\textbf{Declaração}\hfill]
\begin{ttfamily}
public Function FindKey(WKey:String):Boolean;\end{ttfamily}


\end{flushleft}
\end{list}
\paragraph*{NextKey}\hspace*{\fill}

\begin{list}{}{
\settowidth{\tmplength}{\textbf{Declaração}}
\setlength{\itemindent}{0cm}
\setlength{\listparindent}{0cm}
\setlength{\leftmargin}{\evensidemargin}
\addtolength{\leftmargin}{\tmplength}
\settowidth{\labelsep}{X}
\addtolength{\leftmargin}{\labelsep}
\setlength{\labelwidth}{\tmplength}
}
\begin{flushleft}
\item[\textbf{Declaração}\hfill]
\begin{ttfamily}
public Function NextKey: Boolean;\end{ttfamily}


\end{flushleft}
\end{list}
\paragraph*{SearchKey}\hspace*{\fill}

\begin{list}{}{
\settowidth{\tmplength}{\textbf{Declaração}}
\setlength{\itemindent}{0cm}
\setlength{\listparindent}{0cm}
\setlength{\leftmargin}{\evensidemargin}
\addtolength{\leftmargin}{\tmplength}
\settowidth{\labelsep}{X}
\addtolength{\leftmargin}{\labelsep}
\setlength{\labelwidth}{\tmplength}
}
\begin{flushleft}
\item[\textbf{Declaração}\hfill]
\begin{ttfamily}
public Function SearchKey(WKey:String):Boolean;\end{ttfamily}


\end{flushleft}
\end{list}
\paragraph*{NewStr}\hspace*{\fill}

\begin{list}{}{
\settowidth{\tmplength}{\textbf{Declaração}}
\setlength{\itemindent}{0cm}
\setlength{\listparindent}{0cm}
\setlength{\leftmargin}{\evensidemargin}
\addtolength{\leftmargin}{\tmplength}
\settowidth{\labelsep}{X}
\addtolength{\leftmargin}{\labelsep}
\setlength{\labelwidth}{\tmplength}
}
\begin{flushleft}
\item[\textbf{Declaração}\hfill]
\begin{ttfamily}
public Function NewStr(S : String):Boolean; overload;\end{ttfamily}


\end{flushleft}
\end{list}
\paragraph*{Append}\hspace*{\fill}

\begin{list}{}{
\settowidth{\tmplength}{\textbf{Declaração}}
\setlength{\itemindent}{0cm}
\setlength{\listparindent}{0cm}
\setlength{\leftmargin}{\evensidemargin}
\addtolength{\leftmargin}{\tmplength}
\settowidth{\labelsep}{X}
\addtolength{\leftmargin}{\labelsep}
\setlength{\labelwidth}{\tmplength}
}
\begin{flushleft}
\item[\textbf{Declaração}\hfill]
\begin{ttfamily}
public Function Append(S : String):Boolean;\end{ttfamily}


\end{flushleft}
\end{list}
\paragraph*{AddSItem}\hspace*{\fill}

\begin{list}{}{
\settowidth{\tmplength}{\textbf{Declaração}}
\setlength{\itemindent}{0cm}
\setlength{\listparindent}{0cm}
\setlength{\leftmargin}{\evensidemargin}
\addtolength{\leftmargin}{\tmplength}
\settowidth{\labelsep}{X}
\addtolength{\leftmargin}{\labelsep}
\setlength{\labelwidth}{\tmplength}
}
\begin{flushleft}
\item[\textbf{Declaração}\hfill]
\begin{ttfamily}
public procedure AddSItem(P : TConsts.PSItem; ConvertIdioma : TConsts.TConvertIdioma; OkDisposeSItems:Boolean); Overload;\end{ttfamily}


\end{flushleft}
\end{list}
\paragraph*{AddSItem}\hspace*{\fill}

\begin{list}{}{
\settowidth{\tmplength}{\textbf{Declaração}}
\setlength{\itemindent}{0cm}
\setlength{\listparindent}{0cm}
\setlength{\leftmargin}{\evensidemargin}
\addtolength{\leftmargin}{\tmplength}
\settowidth{\labelsep}{X}
\addtolength{\leftmargin}{\labelsep}
\setlength{\labelwidth}{\tmplength}
}
\begin{flushleft}
\item[\textbf{Declaração}\hfill]
\begin{ttfamily}
public procedure AddSItem(P : TConsts.PSItem); Overload;\end{ttfamily}


\end{flushleft}
\par
\item[\textbf{Descrição}]
Adiciona a lista passada por aSItem e desaloca a lista se OkDisposeSItems = true;

\end{list}
\paragraph*{CloneSItems}\hspace*{\fill}

\begin{list}{}{
\settowidth{\tmplength}{\textbf{Declaração}}
\setlength{\itemindent}{0cm}
\setlength{\listparindent}{0cm}
\setlength{\leftmargin}{\evensidemargin}
\addtolength{\leftmargin}{\tmplength}
\settowidth{\labelsep}{X}
\addtolength{\leftmargin}{\labelsep}
\setlength{\labelwidth}{\tmplength}
}
\begin{flushleft}
\item[\textbf{Declaração}\hfill]
\begin{ttfamily}
public Function CloneSItems(Const Items: TConsts.PSItem):TConsts.PSItem;\end{ttfamily}


\end{flushleft}
\end{list}
\paragraph*{CopyTemplateFrom}\hspace*{\fill}

\begin{list}{}{
\settowidth{\tmplength}{\textbf{Declaração}}
\setlength{\itemindent}{0cm}
\setlength{\listparindent}{0cm}
\setlength{\leftmargin}{\evensidemargin}
\addtolength{\leftmargin}{\tmplength}
\settowidth{\labelsep}{X}
\addtolength{\leftmargin}{\labelsep}
\setlength{\labelwidth}{\tmplength}
}
\begin{flushleft}
\item[\textbf{Declaração}\hfill]
\begin{ttfamily}
public Function CopyTemplateFrom(Const aTemplate:TConsts.tString): TConsts.tString;\end{ttfamily}


\end{flushleft}
\end{list}
\paragraph*{PListSItem}\hspace*{\fill}

\begin{list}{}{
\settowidth{\tmplength}{\textbf{Declaração}}
\setlength{\itemindent}{0cm}
\setlength{\listparindent}{0cm}
\setlength{\leftmargin}{\evensidemargin}
\addtolength{\leftmargin}{\tmplength}
\settowidth{\labelsep}{X}
\addtolength{\leftmargin}{\labelsep}
\setlength{\labelwidth}{\tmplength}
}
\begin{flushleft}
\item[\textbf{Declaração}\hfill]
\begin{ttfamily}
public Function PListSItem: TConsts.PSItem;\end{ttfamily}


\end{flushleft}
\end{list}
\chapter{Unit mi.rtl.files}
\section{Descrição}
\begin{itemize}
\item A Unit \textbf{\begin{ttfamily}mi.rtl.files\end{ttfamily}} contém as funções https://techlib.wiki/definition/wrapper.html (Wrapper) para os sistemas operacionais \textbf{Win32} \textbf{Win64} e \textbf{Linux x86{\_}64} reconhecidos pelo free pascal.

\begin{itemize}
\item \textbf{OBJETIVO}: \begin{itemize}
\item Evitar de alterar todos os códigos escritos para a plataforma windows e por isso mantenho o mesmo comportamento do windows.
\end{itemize}
\item \textbf{VERSÃO}: \begin{itemize}
\item Alpha {-} 0.5.0.687
\end{itemize}
\item \textbf{NOTA}: \begin{itemize}
\item https://wiki.freepascal.org/Writing{\_}portable{\_}code{\_}regarding{\_}the{\_}processor{\_}architecture (Veja o link de como escrever código portátil em relação à arquitetura do processador?);
\item Só devo usar units https://man7.org/linux/man-pages/man2/syscalls.2.html (syscalls) do Linux ou https://docs.microsoft.com/pt-br/windows/win32/apiindex/windows-api-list (Windows) caso não encontre a mesma pronta nos projetos lazarus ou Free Pascal.
\end{itemize}
\item \textbf{REFERÊNCIA} \begin{itemize}
\item https://wiki.freepascal.org/Multiplatform{\_}Programming{\_}Guide (Guia de programação multiplataforma);
\item [Reference for unit 'System': procedures and functions](https://www.freepascal.org/docs-html/rtl/system/index-5.html)
\end{itemize}
\item \textbf{CÓDIGO FONTE}: \begin{itemize}
\item 
\end{itemize}
\item \textbf{HISTÓRICO} \begin{itemize}
\item Criado por: Paulo Sérgio da Silva Pacheco e{-}mail: paulosspacheco@yahoo.com.br \begin{itemize}
\item 2021{-}10{-}21 08:00 {-} Data em que essa unit \textbf{\begin{ttfamily}mi.rtl.files\end{ttfamily}} foi criada.
\item 2021{-}11{-}02 15:42 {-} Escolha do projeto \textbf{pasdoc} para criar documento do pacote \textbf{\begin{ttfamily}mi.rtl\end{ttfamily}(\ref{mi.rtl})}.
\item 2021{-}11{-}04 08:37 {-} Implementação da função \textbf{SysFileOpen()}
\item 2021{-}11{-}04 14:00 {-} Implementação da função \textbf{SysSetResult()}
\item 2021{-}11{-}04 14:30 {-} Implementação da função \textbf{SetFileMode()}
\item 2021{-}11{-}04 15:30 {-} Documentar a unit \textbf{\begin{ttfamily}mi.rtl.files\end{ttfamily}} e organizar sessão de constantes, variáveis e funções.
\item 2021{-}11{-}04 21:00 {-} Criar exemplo de uso das funções \textbf{SysFileOpen} e \textbf{SysFileClose}.
\item 2021{-}11{-}05 21:30 {-} Revisar documentação desta nas funções: \textbf{SysSetResult}, ...
\item 2021{-}11{-}12 08:56 {-} Procurar bug da função \textbf{SysFileOpen} na máquina windows. \begin{itemize}
\item Eureca. Resolvi o problema da função \textbf{SysFileOpen}.
\item O problema da função \textbf{SysFileOpen} estava na forma como no windows a função SysUtils.fileOpen trabalha. \begin{itemize}
\item Caso ocorra um erro a função \textbf{SysUtils.fileOpen} retorna high(THandle).
\item Para corrigir precisei modificar a função \textbf{SysSetResult}.
\end{itemize}
\end{itemize}
\item 2021{-}11{-}12 16:56 {-} Documentar a unit \textbf{\begin{ttfamily}mi.rtl.files\end{ttfamily}} e criar a função \textbf{CopyFile()}.
\item 2021{-}11{-}12 18:05 {-} Criar a função SysFileSetSize para truncar o arquivo e documenta{-}la.
\item 2021{-}11{-}13 \begin{itemize}
\item Criado a class \textbf{\begin{ttfamily}TFiles\end{ttfamily}(\ref{mi.rtl.files.TFiles})} herdade de \begin{ttfamily}TConsts\end{ttfamily}(\ref{mi.rtl.Consts.TConsts}) com propósito em encapsular as funções de acesso ao sistema operacional.
\item Criar método \textbf{TFiles.ErrorMessage()};
\item Criar método \textbf{\begin{ttfamily}TFiles.SetLastError\end{ttfamily}(\ref{mi.rtl.files.TFiles-SetLastError})()};
\item Criar método \textbf{\begin{ttfamily}TFiles.SetResult\end{ttfamily}(\ref{mi.rtl.files.TFiles-SetResult})()}
\item Criar método \textbf{\begin{ttfamily}TFiles.CopyFile\end{ttfamily}(\ref{mi.rtl.files.TFiles-CopyFile})()}
\item Criar método \textbf{\begin{ttfamily}TFiles.SetFileMode\end{ttfamily}(\ref{mi.rtl.files.TFiles-SetFileMode})()}
\item Criar método \textbf{\begin{ttfamily}TFiles.FileOpen\end{ttfamily}(\ref{mi.rtl.files.TFiles-FileOpen})(5 parametro)}
\item Criar método \textbf{\begin{ttfamily}TFiles.FileOpen\end{ttfamily}(\ref{mi.rtl.files.TFiles-FileOpen})(3 parâmetro)}
\item Criar método \textbf{\begin{ttfamily}TFiles.FileClose\end{ttfamily}(\ref{mi.rtl.files.TFiles-FileClose})()}
\item Criar método \textbf{\begin{ttfamily}TFiles.FileTruncate\end{ttfamily}(\ref{mi.rtl.files.TFiles-FileTruncate})()}
\item Criar método \textbf{\begin{ttfamily}TFiles.FileCreate\end{ttfamily}(\ref{mi.rtl.files.TFiles-FileCreate})()}
\end{itemize}
\item 2021{-}11{-}15

\begin{itemize}
\item O método \textbf{\begin{ttfamily}TFiles.FileCreate\end{ttfamily}(\ref{mi.rtl.files.TFiles-FileCreate})()} não está obedecendo o mapa de bits FileMode() checar o porque: \begin{itemize}
\item Solução: \begin{itemize}
\item A função SysUtils.FileCreate precisa do fmCreate na criação do arquivo.
\item Após criar o arquivo o mesmo deve ser fechado e aberto novamente com o mapa de bits \textbf{mode} e \textbf{shareMode} passado no parâmetro.
\end{itemize}
\end{itemize}
\item Criar método \textbf{\begin{ttfamily}TFiles.FileSeek\end{ttfamily}(\ref{mi.rtl.files.TFiles-FileSeek})()}
\item Criar método \textbf{\begin{ttfamily}TFiles.FileRead\end{ttfamily}(\ref{mi.rtl.files.TFiles-FileRead})()}
\end{itemize}
\item 2021{-}11{-}16 \begin{itemize}
\item O método SysFileSeek não gerar erro se o ponteiro do arquivo for inválido. \begin{itemize}
\item Para contornar devo fazer a crítica se o ponteiro é maior que zero e menor que fileSize.
\item Essa solução não atende porque não fileSeek não tem o nome do arquivo.
\item Entendendo porque SysUtils.FileSeek não dar erro quando se tenta posicionar além do fim do arquivo: \begin{itemize}
\item https://man7.org/linux/man-pages/man2/lseek.2.html
\item No linux \textbf{lseek()} permite que o deslocamento do arquivo seja definido além do final do arquivo (mas isso não altera o tamanho do arquivo). Se os dados forem posteriormente escrito neste ponto, leituras subsequentes dos dados no gap (um "buraco") retorna bytes nulos ('{\textbackslash} 0') até que os dados sejam realmente escrito na lacuna.
\end{itemize}
\end{itemize}
\item Criar método \textbf{\begin{ttfamily}TFiles.FileSize\end{ttfamily}(\ref{mi.rtl.files.TFiles-FileSize})()}
\item Criar exemplo de uso do método \textbf{\begin{ttfamily}TFiles.FileSize\end{ttfamily}(\ref{mi.rtl.files.TFiles-FileSize})()}
\item Criar exemplo de uso do método \textbf{\begin{ttfamily}TFiles.FileSeek\end{ttfamily}(\ref{mi.rtl.files.TFiles-FileSeek})()}
\end{itemize}
\item 2021{-}11{-}17 \begin{itemize}
\item 08:30 a 10:38 {-} Criar exemplo de uso do método \textbf{\begin{ttfamily}TFiles.FileRead\end{ttfamily}(\ref{mi.rtl.files.TFiles-FileRead})()}
\item 11:36 a 11:48 {-} Criar método \textbf{\begin{ttfamily}TFiles.FileWrite\end{ttfamily}(\ref{mi.rtl.files.TFiles-FileWrite})()}
\item 11:50 a 12:21 {-} Criar exemplo de uso do método \textbf{\begin{ttfamily}TFiles.FileWrite\end{ttfamily}(\ref{mi.rtl.files.TFiles-FileWrite})()} {-} Falta testar.
\item 14:15 a 15:23 {-} Testar o exemplo de uso da função \begin{ttfamily}TFiles.FileWrite\end{ttfamily}(\ref{mi.rtl.files.TFiles-FileWrite}). ok.
\end{itemize}
\item 2021{-}11{-}21 \begin{itemize}
\item 10:30 a 11:10 \begin{itemize}
\item Criar classe método {-} \textbf{\begin{ttfamily}TFiles.FileExists\end{ttfamily}(\ref{mi.rtl.files.TFiles-FileExists})()}
\item Criar classe método {-} \textbf{\begin{ttfamily}TFiles.DirectoryExists\end{ttfamily}(\ref{mi.rtl.files.TFiles-DirectoryExists})()}
\end{itemize}
\end{itemize}
\item 2021{-}11{-}22 \begin{itemize}
\item 10:00 a 10:06 {-} Criar classe método : \textbf{\begin{ttfamily}TFiles.CreateDir\end{ttfamily}(\ref{mi.rtl.files.TFiles-CreateDir})()}
\item 10:29 a 11:10 {-} Criar classe método : \textbf{TFiles.SysGetDriveType(aPath : AnsiString): TDriveType;}
\item 11:11 a 12:02 {-} Criar classe método : **DuplicateHandle(hSourceHandle: LongInt;Var lpTargeTHandle: Longint) : Longint);
\item 14:10 a 15:16 {-} Criar classe método : \textbf{FileFlushBuffers(Handle: THandle): Longint;}
\item 15:44 a 17:32 {-} Criar classe método : \textbf{LockFile({\_}Handle:THandle; {\_}LockStart, {\_}LockLength: Int64): LongInt;} \begin{itemize}
\item Não encontrei no linux o equivalente ao Windows.LockFile
\end{itemize}
\item 15:44 a 17:32 {-} Criar classe método : \textbf{UnLockFile({\_}Handle:THandle; {\_}LockStart, {\_}LockLength: Int64): LongInt;} \begin{itemize}
\item Não encontrei no linux o equivalente ao Windows.unLockFile
\end{itemize}
\item 2021{-}12{-}01 \begin{itemize}
\item 11:42 a ??:?? : Implementar a função Is{\_}TFileOpen
\end{itemize}
\item 2021{-}12{-}02 \begin{itemize}
\item 20:00 a 22:15 : Implementei a classe \begin{ttfamily}TStrError\end{ttfamily}(\ref{mi.rtl.Consts.StrError.TStrError})
\end{itemize}
\item 20211230 \begin{itemize}
\item 15:30 a 16:10 : Criar método GetTempFileName
\end{itemize}
\item 20220111 \begin{itemize}
\item 17:10 {-} Criar método shellExecute
\end{itemize}
\end{itemize}
\end{itemize}
\end{itemize}
\end{itemize}
\end{itemize}
\section{Uses}
\begin{itemize}
\item \begin{ttfamily}Classes\end{ttfamily}\item \begin{ttfamily}dos\end{ttfamily}\item \begin{ttfamily}SysUtils\end{ttfamily}\item \begin{ttfamily}crt\end{ttfamily}\item \begin{ttfamily}FileUtil\end{ttfamily}\item \begin{ttfamily}mi.rtl.types\end{ttfamily}(\ref{mi.rtl.Types})\item \begin{ttfamily}mi.rtl.Consts\end{ttfamily}(\ref{mi.rtl.Consts})\item \begin{ttfamily}mi.rtl.Consts.StrError\end{ttfamily}(\ref{mi.rtl.Consts.StrError})\end{itemize}
\section{Visão Geral}
\begin{description}
\item[\texttt{\begin{ttfamily}TFiles\end{ttfamily} Classe}]
\end{description}
\section{Classes, Interfaces, Objetos e Registros}
\subsection*{TFiles Classe}
\subsubsection*{\large{\textbf{Hierarquia}}\normalsize\hspace{1ex}\hfill}
TFiles {$>$} \begin{ttfamily}TConsts\end{ttfamily}(\ref{mi.rtl.Consts.TConsts}) {$>$} \begin{ttfamily}TTypes\end{ttfamily}(\ref{mi.rtl.Types.TTypes}) {$>$} 
TComponent
\subsubsection*{\large{\textbf{Descrição}}\normalsize\hspace{1ex}\hfill}
no description available, TConsts description followsA classe \textbf{\begin{ttfamily}TConsts\end{ttfamily}} declara todas as constantes globais do pacote MarIcarai\subsubsection*{\large{\textbf{Métodos}}\normalsize\hspace{1ex}\hfill}
\paragraph*{IoResult}\hspace*{\fill}

\begin{list}{}{
\settowidth{\tmplength}{\textbf{Declaração}}
\setlength{\itemindent}{0cm}
\setlength{\listparindent}{0cm}
\setlength{\leftmargin}{\evensidemargin}
\addtolength{\leftmargin}{\tmplength}
\settowidth{\labelsep}{X}
\addtolength{\leftmargin}{\labelsep}
\setlength{\labelwidth}{\tmplength}
}
\begin{flushleft}
\item[\textbf{Declaração}\hfill]
\begin{ttfamily}
public class Function IoResult: Integer;\end{ttfamily}


\end{flushleft}
\par
\item[\textbf{Descrição}]
A função \textbf{\begin{ttfamily}IoResult\end{ttfamily}} captura o estados de system.ioResult e atualiza \begin{ttfamily}TaStatus\end{ttfamily}(\ref{mi.rtl.Consts.TConsts-TaStatus})

\end{list}
\paragraph*{CtrlSleep}\hspace*{\fill}

\begin{list}{}{
\settowidth{\tmplength}{\textbf{Declaração}}
\setlength{\itemindent}{0cm}
\setlength{\listparindent}{0cm}
\setlength{\leftmargin}{\evensidemargin}
\addtolength{\leftmargin}{\tmplength}
\settowidth{\labelsep}{X}
\addtolength{\leftmargin}{\labelsep}
\setlength{\labelwidth}{\tmplength}
}
\begin{flushleft}
\item[\textbf{Declaração}\hfill]
\begin{ttfamily}
public class procedure CtrlSleep(Const Delay: Cardinal);\end{ttfamily}


\end{flushleft}
\par
\item[\textbf{Descrição}]
\begin{itemize}
\item A função \textbf{\begin{ttfamily}CtrlSleep\end{ttfamily}} dar opção para que a aplicação execute outras tarefas caso ela se encontre em estado de espera.

\begin{itemize}
\item \textbf{PARÂMETRO} \begin{itemize}
\item \textbf{Delay} {-} Tempo em milissegundo que deve aguardar**.
\end{itemize}
\end{itemize}
\end{itemize}

\end{list}
\paragraph*{Set{\_}CTRL{\_}SLEEP{\_}ENABLE}\hspace*{\fill}

\begin{list}{}{
\settowidth{\tmplength}{\textbf{Declaração}}
\setlength{\itemindent}{0cm}
\setlength{\listparindent}{0cm}
\setlength{\leftmargin}{\evensidemargin}
\addtolength{\leftmargin}{\tmplength}
\settowidth{\labelsep}{X}
\addtolength{\leftmargin}{\labelsep}
\setlength{\labelwidth}{\tmplength}
}
\begin{flushleft}
\item[\textbf{Declaração}\hfill]
\begin{ttfamily}
public class Function Set{\_}CTRL{\_}SLEEP{\_}ENABLE(Const aEnable: Boolean):Boolean;\end{ttfamily}


\end{flushleft}
\par
\item[\textbf{Descrição}]
\begin{itemize}
\item A função \textbf{\begin{ttfamily}Set{\_}CTRL{\_}SLEEP{\_}ENABLE\end{ttfamily}} habilita ou não a função \textbf{\begin{ttfamily}CtrlSleep\end{ttfamily}(\ref{mi.rtl.files.TFiles-CtrlSleep})}.

\begin{itemize}
\item \textbf{PARÂMETRO} \begin{itemize}
\item \textbf{aEnable} \begin{itemize}
\item Se \textbf{True} habilita \begin{ttfamily}CtrlSleep\end{ttfamily}(\ref{mi.rtl.files.TFiles-CtrlSleep});
\item Se \textbf{false} desabilita o método \begin{ttfamily}CtrlSleep\end{ttfamily}(\ref{mi.rtl.files.TFiles-CtrlSleep});
\end{itemize}
\end{itemize}
\item \textbf{RETORNA} \begin{itemize}
\item o valor anterior da variável \begin{ttfamily}CTRL{\_}SLEEP{\_}ENABLE\end{ttfamily}(\ref{mi.rtl.Consts.TConsts-CTRL_SLEEP_ENABLE});
\end{itemize}
\end{itemize}
\end{itemize}

\end{list}
\paragraph*{ReadKey}\hspace*{\fill}

\begin{list}{}{
\settowidth{\tmplength}{\textbf{Declaração}}
\setlength{\itemindent}{0cm}
\setlength{\listparindent}{0cm}
\setlength{\leftmargin}{\evensidemargin}
\addtolength{\leftmargin}{\tmplength}
\settowidth{\labelsep}{X}
\addtolength{\leftmargin}{\labelsep}
\setlength{\labelwidth}{\tmplength}
}
\begin{flushleft}
\item[\textbf{Declaração}\hfill]
\begin{ttfamily}
public class function ReadKey: AnsiChar;\end{ttfamily}


\end{flushleft}
\end{list}
\paragraph*{SetLastError}\hspace*{\fill}

\begin{list}{}{
\settowidth{\tmplength}{\textbf{Declaração}}
\setlength{\itemindent}{0cm}
\setlength{\listparindent}{0cm}
\setlength{\leftmargin}{\evensidemargin}
\addtolength{\leftmargin}{\tmplength}
\settowidth{\labelsep}{X}
\addtolength{\leftmargin}{\labelsep}
\setlength{\labelwidth}{\tmplength}
}
\begin{flushleft}
\item[\textbf{Declaração}\hfill]
\begin{ttfamily}
public class procedure SetLastError(aCodeError: integer);\end{ttfamily}


\end{flushleft}
\par
\item[\textbf{Descrição}]
\begin{itemize}
\item A procedure \textbf{\begin{ttfamily}SetLastError\end{ttfamily}} atualiza as variáveis globais \textbf{\begin{ttfamily}LastError\end{ttfamily}(\ref{mi.rtl.Consts.TConsts-LastError})} e \textbf{\begin{ttfamily}OK\end{ttfamily}(\ref{mi.rtl.Consts.TConsts-OK})}

\begin{itemize}
\item \textbf{PARÂMETROS} \begin{itemize}
\item aCodeError:\begin{ttfamily}integer\end{ttfamily}(\ref{mi.rtl.Types.TTypes-Integer}) {-} Código do erro.
\end{itemize}
\item \textbf{EXEMPLO}

\texttt{\\\nopagebreak[3]
\\\nopagebreak[3]
}\textbf{procedure}\texttt{~tes{\_}SetLastError();\\\nopagebreak[3]
}\textbf{Begin}\texttt{\\\nopagebreak[3]
~~\textit{//Executa~a~procedure~SetLastError}\\\nopagebreak[3]
\\\nopagebreak[3]
~~SetLastError(2);\\\nopagebreak[3]
~~showMessage(ErrorMessage(LastError));\\\nopagebreak[3]
\\\nopagebreak[3]
}\textbf{end}\texttt{;\\
}
\end{itemize}
\end{itemize}

\end{list}
\paragraph*{SetResult}\hspace*{\fill}

\begin{list}{}{
\settowidth{\tmplength}{\textbf{Declaração}}
\setlength{\itemindent}{0cm}
\setlength{\listparindent}{0cm}
\setlength{\leftmargin}{\evensidemargin}
\addtolength{\leftmargin}{\tmplength}
\settowidth{\labelsep}{X}
\addtolength{\leftmargin}{\labelsep}
\setlength{\labelwidth}{\tmplength}
}
\begin{flushleft}
\item[\textbf{Declaração}\hfill]
\begin{ttfamily}
public class function SetResult(aHandle: THandle ; aSuccess: Boolean): Longint; overload;\end{ttfamily}


\end{flushleft}
\par
\item[\textbf{Descrição}]
\begin{itemize}
\item A função \textbf{\begin{ttfamily}SetResult\end{ttfamily}} captura o último erro se o parâmetro \textbf{aSucesso=false} ou o \textbf{aHandle for inválido} e retorna \textbf{0 (zero)} se \textbf{aSucesso=true} e o \textit{aHandle for válido}.

\begin{itemize}
\item A função \textbf{\begin{ttfamily}SetResult\end{ttfamily}} atualiza a variável global \textbf{\begin{ttfamily}LastError\end{ttfamily}(\ref{mi.rtl.Consts.TConsts-LastError})} e a variável global \textbf{\begin{ttfamily}ok\end{ttfamily}(\ref{mi.rtl.Consts.TConsts-OK})}:
\item \textbf{Plataformas testadas} \begin{itemize}
\item win32
\item win64
\item linux
\end{itemize}
\item \textbf{PARÂMETROS} \begin{itemize}
\item \textbf{aHandle} {-} O handle do arquivo retornado pela ultima chamada ao sistema operacional.
\item \textbf{aSucesso} {-} Recebe \textbf{true} se sucesso ou \textbf{false} se fracasso na última chamada ao sistema operacional.
\end{itemize}
\item \textbf{RETORNA} \begin{itemize}
\item O conteúdo da variável global \textbf{\begin{ttfamily}LastError\end{ttfamily}(\ref{mi.rtl.Consts.TConsts-LastError})}.
\end{itemize}
\item \textbf{NOTA} \begin{itemize}
\item No windows, quando ocorre um erro o handle é igual = high(\begin{ttfamily}THandle\end{ttfamily}(\ref{mi.rtl.Types.TTypes-THandle})) por isso é necessário passar o handle do arquivo na chamada a \begin{ttfamily}SetResult\end{ttfamily}().
\end{itemize}
\end{itemize}
\end{itemize}

\end{list}
\paragraph*{CopyFile}\hspace*{\fill}

\begin{list}{}{
\settowidth{\tmplength}{\textbf{Declaração}}
\setlength{\itemindent}{0cm}
\setlength{\listparindent}{0cm}
\setlength{\leftmargin}{\evensidemargin}
\addtolength{\leftmargin}{\tmplength}
\settowidth{\labelsep}{X}
\addtolength{\leftmargin}{\labelsep}
\setlength{\labelwidth}{\tmplength}
}
\begin{flushleft}
\item[\textbf{Declaração}\hfill]
\begin{ttfamily}
public class function CopyFile(lpExistingFileName, lpNewFileName: AnsiString; aExceptionOnError: boolean): Integer;\end{ttfamily}


\end{flushleft}
\par
\item[\textbf{Descrição}]
\begin{itemize}
\item A função \textbf{\begin{ttfamily}CopyFile\end{ttfamily}} copia o arquivo passado por \textbf{alpExistingFileName} para o arquivo passo por \textbf{lpNewFileName}.

\begin{itemize}
\item \textbf{PARÂMETROS} \begin{itemize}
\item \textbf{alpExistingFileName}:AnsiString {-} Nome do arquivo a ser copiado;
\item \textbf{lpNewFileName}:AnsiString {-} Nome do arquivo destino da cópia;
\item \textbf{aExceptionOnError}:boolean {-} \textbf{True} se o sistema deve gera exceção ou \textbf{false} se o sistema não deve gera exceção.
\end{itemize}
\item \textbf{RETORNO} \begin{itemize}
\item \textbf{\begin{ttfamily}Integer\end{ttfamily}(\ref{mi.rtl.Types.TTypes-Integer})} {-} Código do erro ou 0 (zero) se a cópia for feita com sucesso.
\item Caso o parâmetro \textbf{aExceptionOnError} = true então a exceção deve ser tratada pela rotina que o chamou.
\end{itemize}
\item \textbf{EXEMPLO}

\texttt{\\\nopagebreak[3]
\\\nopagebreak[3]
}\textbf{procedure}\texttt{~TFormTests.Button{\_}tes{\_}CopyFileClick(Sender:~TObject);\\\nopagebreak[3]
\\\nopagebreak[3]
~~\textit{//~Este~procedimento~faz~duas~cópia~do~arquivo~index.html}\\\nopagebreak[3]
\\\nopagebreak[3]
~~}\textbf{Var}\texttt{\\\nopagebreak[3]
~~~~err~:~TFiles.integer;\\\nopagebreak[3]
}\textbf{Begin}\texttt{\\\nopagebreak[3]
~~}\textbf{with}\texttt{~TFiles~}\textbf{do}\texttt{\\\nopagebreak[3]
~~~~err~:=~CopyFile('index.html','index.bak1',false);\\\nopagebreak[3]
\\\nopagebreak[3]
~~}\textbf{with}\texttt{~TFiles~}\textbf{do}\texttt{\\\nopagebreak[3]
~~}\textbf{if}\texttt{~err~=~0\\\nopagebreak[3]
~~}\textbf{Then}\texttt{~showMessage('Copia~1~feita~com~sucesso.')\\\nopagebreak[3]
~~}\textbf{else}\texttt{~showMessage(ErrorMessage(err));\\\nopagebreak[3]
\\\nopagebreak[3]
~~}\textbf{with}\texttt{~TFiles~}\textbf{do}\texttt{\\\nopagebreak[3]
~~}\textbf{try}\texttt{\\\nopagebreak[3]
~~~~CopyFile('index.html','index.bak2',true);\\\nopagebreak[3]
~~~~showMessage('Copia~2~feita~com~sucesso.')~;\\\nopagebreak[3]
~~}\textbf{Except}\texttt{\\\nopagebreak[3]
~~~~}\textbf{on}\texttt{~E:Exception~}\textbf{do}\texttt{\\\nopagebreak[3]
~~~~~ShowMessage(e.}\textbf{Message}\texttt{);\\\nopagebreak[3]
~~}\textbf{end}\texttt{;\\\nopagebreak[3]
}\textbf{end}\texttt{;\\
}
\end{itemize}
\end{itemize}

\end{list}
\paragraph*{SetFileMode}\hspace*{\fill}

\begin{list}{}{
\settowidth{\tmplength}{\textbf{Declaração}}
\setlength{\itemindent}{0cm}
\setlength{\listparindent}{0cm}
\setlength{\leftmargin}{\evensidemargin}
\addtolength{\leftmargin}{\tmplength}
\settowidth{\labelsep}{X}
\addtolength{\leftmargin}{\labelsep}
\setlength{\labelwidth}{\tmplength}
}
\begin{flushleft}
\item[\textbf{Declaração}\hfill]
\begin{ttfamily}
public Class function SetFileMode(aFileMode:word):word;\end{ttfamily}


\end{flushleft}
\par
\item[\textbf{Descrição}]
\begin{itemize}
\item A função \textbf{\begin{ttfamily}SetFileMode\end{ttfamily}} modifica o valor de \begin{ttfamily}FileMode\end{ttfamily}(\ref{mi.rtl.Consts.TConsts-FileMode}) e retorna o valor do \textbf{\begin{ttfamily}FileMode\end{ttfamily}(\ref{mi.rtl.Consts.TConsts-FileMode})} anterior; \begin{itemize}
\item \textbf{PARÂMETROS} \begin{itemize}
\item \textbf{aFileMode} é o modo de abertura do arquivo.
\end{itemize}
\item \textbf{RETORNA} \begin{itemize}
\item O \begin{ttfamily}FileModeAnt\end{ttfamily}(\ref{mi.rtl.Consts.TConsts-FileModeAnt});
\end{itemize}
\item \textbf{NOTA} \begin{itemize}
\item A variável pública \begin{ttfamily}FileModeAnt\end{ttfamily}(\ref{mi.rtl.Consts.TConsts-FileModeAnt}) é igual ao resultado desta função.
\end{itemize}
\end{itemize}
\end{itemize}

\end{list}
\paragraph*{SetStateFileMode}\hspace*{\fill}

\begin{list}{}{
\settowidth{\tmplength}{\textbf{Declaração}}
\setlength{\itemindent}{0cm}
\setlength{\listparindent}{0cm}
\setlength{\leftmargin}{\evensidemargin}
\addtolength{\leftmargin}{\tmplength}
\settowidth{\labelsep}{X}
\addtolength{\leftmargin}{\labelsep}
\setlength{\labelwidth}{\tmplength}
}
\begin{flushleft}
\item[\textbf{Declaração}\hfill]
\begin{ttfamily}
public class Function SetStateFileMode(Const AState: Longint; Const Enable: boolean):Boolean;\end{ttfamily}


\end{flushleft}
\par
\item[\textbf{Descrição}]
\begin{itemize}
\item Seta \begin{ttfamily}FileMode\end{ttfamily}(\ref{mi.rtl.Consts.TConsts-FileMode}) e retorna o estado anterior do Mapa de bits passado por aState
\end{itemize}

\end{list}
\paragraph*{GetStateFileMode}\hspace*{\fill}

\begin{list}{}{
\settowidth{\tmplength}{\textbf{Declaração}}
\setlength{\itemindent}{0cm}
\setlength{\listparindent}{0cm}
\setlength{\leftmargin}{\evensidemargin}
\addtolength{\leftmargin}{\tmplength}
\settowidth{\labelsep}{X}
\addtolength{\leftmargin}{\labelsep}
\setlength{\labelwidth}{\tmplength}
}
\begin{flushleft}
\item[\textbf{Declaração}\hfill]
\begin{ttfamily}
public class function GetStateFileMode(Const AState: Longint): Boolean;\end{ttfamily}


\end{flushleft}
\par
\item[\textbf{Descrição}]
\begin{itemize}
\item Ler o estado do File Mode
\end{itemize}

\end{list}
\paragraph*{FileOpen}\hspace*{\fill}

\begin{list}{}{
\settowidth{\tmplength}{\textbf{Declaração}}
\setlength{\itemindent}{0cm}
\setlength{\listparindent}{0cm}
\setlength{\leftmargin}{\evensidemargin}
\addtolength{\leftmargin}{\tmplength}
\settowidth{\labelsep}{X}
\addtolength{\leftmargin}{\labelsep}
\setlength{\labelwidth}{\tmplength}
}
\begin{flushleft}
\item[\textbf{Declaração}\hfill]
\begin{ttfamily}
public class function FileOpen(const FileName: AnsiString; const Mode: Longint; const ShareMode : Cardinal; out Handle: THandle): Longint; Overload;\end{ttfamily}


\end{flushleft}
\par
\item[\textbf{Descrição}]
\begin{itemize}
\item \textbf{Abre o arquivo passado pelo parâmetro FileName}

\begin{itemize}
\item \textbf{PARÂMETROS} \begin{itemize}
\item \textbf{FileName} {-} Nome do arquivo a ser aberto;
\item \textbf{mode} {-} Modo de abertura. Valor possível veja \begin{ttfamily}TFileMode\end{ttfamily}(\ref{mi.rtl.Types.TTypes-TFileMode});
\item \textbf{attribute} {-} Atributo de abertura do arquivo;
\item \textbf{Flags} {-} flag de abertura do arquivo;
\item \textbf{Handle} {-} Se tiver sucesso retorna nesta variável o número do arquivo aberto;
\end{itemize}
\item \textbf{RETORNA} \begin{itemize}
\item \begin{ttfamily}LongInt\end{ttfamily}(\ref{mi.rtl.Types.TTypes-LongInt}) Zero se sucesso ou o código do erro se fracasso.
\end{itemize}
\item \textbf{NOTA}: \begin{itemize}
\item Possiveis erros pode ser visto na função ErrorMessage();
\end{itemize}
\item \textbf{EXEMPLOS DE USO}

\texttt{\\\nopagebreak[3]
\\\nopagebreak[3]
}\textbf{procedure}\texttt{~TFormTests.Button{\_}Test{\_}OpenFile{\_}exclusive{\_}modeClick(Sender:~TObject);\\\nopagebreak[3]
\\\nopagebreak[3]
~~}\textbf{procedure}\texttt{~Test{\_}OpenFile{\_}exclusive(aFileName:AnsiString);\\\nopagebreak[3]
~~}\textbf{Var}\texttt{\\\nopagebreak[3]
~~~~Err~:~TFiles.integer;\\\nopagebreak[3]
~~~~h,\\\nopagebreak[3]
~~~~h1~:~TFiles.THandle;\\\nopagebreak[3]
~~}\textbf{Begin}\texttt{\\\nopagebreak[3]
~~~~}\textbf{with}\texttt{~TFiles~}\textbf{do}\texttt{\\\nopagebreak[3]
~~~~~~Err~:=~FileOpen~(aFileName,~FmReadWrite~}\textbf{or}\texttt{~FmDenyALL~~}\textbf{or}\texttt{~fmShareCompat~~,h);\\\nopagebreak[3]
\\\nopagebreak[3]
~~~~}\textbf{with}\texttt{~TFiles~}\textbf{do}\texttt{\\\nopagebreak[3]
~~~~}\textbf{if}\texttt{~Err~=~0\\\nopagebreak[3]
~~~~}\textbf{Then}\texttt{~}\textbf{Begin}\texttt{\\\nopagebreak[3]
~~~~~~~~~~~ShowMessage('Teste~da~função~SysFileOpen~retornou~true');\\\nopagebreak[3]
\\\nopagebreak[3]
~~~~~~~~~~~Err~:=~FileOpen~(aFileName,FmReadWrite~}\textbf{or}\texttt{~FmDenyALL~~}\textbf{or}\texttt{~fmShareCompat~~,h1);\\\nopagebreak[3]
~~~~~~~~~~~}\textbf{if}\texttt{~Err~=~0\\\nopagebreak[3]
~~~~~~~~~~~}\textbf{Then}\texttt{~}\textbf{Begin}\texttt{\\\nopagebreak[3]
~~~~~~~~~~~~~~~~~~FileClose(h1);\\\nopagebreak[3]
~~~~~~~~~~~~~~~~~~ShowMessage('Teste~da~função~SysFileOpen~retornou~true')\\\nopagebreak[3]
~~~~~~~~~~~~~~~~}\textbf{end}\texttt{\\\nopagebreak[3]
~~~~~~~~~~~}\textbf{else}\texttt{~ShowMessage('Error:~'+ErrorMessage(Err));\\\nopagebreak[3]
\\\nopagebreak[3]
~~~~~~~~~~~FileClose(h);\\\nopagebreak[3]
~~~~}\textbf{end}\texttt{\\\nopagebreak[3]
~~~~}\textbf{else}\texttt{~ShowMessage('Error:~'+ErrorMessage(Err));\\\nopagebreak[3]
\\\nopagebreak[3]
~~}\textbf{End}\texttt{;\\\nopagebreak[3]
\\\nopagebreak[3]
}\textbf{begin}\texttt{\\\nopagebreak[3]
~~Test{\_}OpenFile{\_}exclusive('index.html');\\\nopagebreak[3]
}\textbf{end}\texttt{;\\
}
\item \textbf{REFERÊNCIAS} \begin{itemize}
\item https://www.freepascal.org/docs-html/rtl/sysutils/fileopen.html (FileOpen);
\item \begin{ttfamily}fmOpenRead\end{ttfamily}(\ref{mi.rtl.Consts.TConsts-fmOpenRead});
\item \begin{ttfamily}FileClose\end{ttfamily}(\ref{mi.rtl.files.TFiles-FileClose});
\item \begin{ttfamily}THandle\end{ttfamily}(\ref{mi.rtl.Types.TTypes-THandle}).
\end{itemize}
\end{itemize}
\end{itemize}

\end{list}
\paragraph*{FileOpen}\hspace*{\fill}

\begin{list}{}{
\settowidth{\tmplength}{\textbf{Declaração}}
\setlength{\itemindent}{0cm}
\setlength{\listparindent}{0cm}
\setlength{\leftmargin}{\evensidemargin}
\addtolength{\leftmargin}{\tmplength}
\settowidth{\labelsep}{X}
\addtolength{\leftmargin}{\labelsep}
\setlength{\labelwidth}{\tmplength}
}
\begin{flushleft}
\item[\textbf{Declaração}\hfill]
\begin{ttfamily}
public Class function FileOpen(const FileName: AnsiString; out Handle: THandle): Longint; Overload;\end{ttfamily}


\end{flushleft}
\par
\item[\textbf{Descrição}]
\begin{itemize}
\item \textbf{Abre o arquivo passado pelo parâmetro FileName}

\begin{itemize}
\item \textbf{PARÂMETROS} \begin{itemize}
\item FileName \textbf{Nome do arquivo a ser aberto;}
\item \textbf{mode} {-} Modo de abertura;
\item \textbf{Handle} {-} Se result = 0 o Handle contém o número do arquivo aberto, caso contrário, retorna \begin{ttfamily}HANDLE{\_}INVALID\end{ttfamily}(\ref{mi.rtl.Consts.TConsts-HANDLE_INVALID})**;
\end{itemize}
\item \textbf{RETORNA} \begin{itemize}
\item 0 (zero) se sucesso ou o código do erro se fracasso;
\end{itemize}
\item \textbf{NOTA}: \begin{itemize}
\item Possiveis erros pode ser visto na função ErrorMessage();
\end{itemize}
\item \textbf{EXEMPLO DE USO}

\texttt{\\\nopagebreak[3]
\\\nopagebreak[3]
}\textbf{procedure}\texttt{~TFormTests.TestSysOpenFileClick(Sender:~TObject);\\\nopagebreak[3]
}\textbf{Var}\texttt{\\\nopagebreak[3]
~~Err~:~TFiles.Longint;\\\nopagebreak[3]
~~h~:~TFiles.THandle;\\\nopagebreak[3]
}\textbf{Begin}\texttt{\\\nopagebreak[3]
~~}\textbf{with}\texttt{~TFiles~}\textbf{do}\texttt{\\\nopagebreak[3]
~~~~Err~:=~FileOpen~('index.html',fmOpenRead,h);\\\nopagebreak[3]
\\\nopagebreak[3]
~~}\textbf{with}\texttt{~TFiles~}\textbf{do}\texttt{\\\nopagebreak[3]
~~}\textbf{if}\texttt{~Err~=~0~}\textbf{Then}\texttt{\\\nopagebreak[3]
~~}\textbf{Begin}\texttt{\\\nopagebreak[3]
~~~~FileClose(h);\\\nopagebreak[3]
~~~~ShowMessage('Teste~da~função~FileOpen~retornou~true')\\\nopagebreak[3]
~~}\textbf{end}\texttt{\\\nopagebreak[3]
~~}\textbf{else}\texttt{~ShowMessage('Error:~'+ErrorMessage(Err));\\\nopagebreak[3]
}\textbf{End}\texttt{;\\
}
\item \textbf{REFERÊNCIAS} \begin{itemize}
\item https://www.freepascal.org/docs-html/rtl/sysutils/fileopen.html (FileOpen);
\item \begin{ttfamily}fmOpenRead\end{ttfamily}(\ref{mi.rtl.Consts.TConsts-fmOpenRead});
\item \begin{ttfamily}FileClose\end{ttfamily}(\ref{mi.rtl.files.TFiles-FileClose});
\item \begin{ttfamily}THandle\end{ttfamily}(\ref{mi.rtl.Types.TTypes-THandle}).
\end{itemize}
\end{itemize}
\end{itemize}

\end{list}
\paragraph*{FileClose}\hspace*{\fill}

\begin{list}{}{
\settowidth{\tmplength}{\textbf{Declaração}}
\setlength{\itemindent}{0cm}
\setlength{\listparindent}{0cm}
\setlength{\leftmargin}{\evensidemargin}
\addtolength{\leftmargin}{\tmplength}
\settowidth{\labelsep}{X}
\addtolength{\leftmargin}{\labelsep}
\setlength{\labelwidth}{\tmplength}
}
\begin{flushleft}
\item[\textbf{Declaração}\hfill]
\begin{ttfamily}
public Class function FileClose(Handle: THandle): Longint;\end{ttfamily}


\end{flushleft}
\par
\item[\textbf{Descrição}]
\begin{itemize}
\item A função \textbf{\begin{ttfamily}FileClose\end{ttfamily}} fecha o arquivo passando por Handle. \begin{itemize}
\item \textbf{PARÂMETRO} \begin{itemize}
\item \textbf{Handle} {-} Número do arquivo aberto por \begin{ttfamily}FileOpen\end{ttfamily}(\ref{mi.rtl.files.TFiles-FileOpen}).
\end{itemize}
\item \textbf{RETORNA} \begin{itemize}
\item Zero de tiver sucesso ou o código do erro se não conseguir fechar o arquivo.
\end{itemize}
\item \textbf{REFERÊNCIAS} \begin{itemize}
\item https://www.freepascal.org/docs-html/rtl/sysutils/FileClose.html (FileClose);
\end{itemize}
\end{itemize}
\end{itemize}

\end{list}
\paragraph*{FileTruncate}\hspace*{\fill}

\begin{list}{}{
\settowidth{\tmplength}{\textbf{Declaração}}
\setlength{\itemindent}{0cm}
\setlength{\listparindent}{0cm}
\setlength{\leftmargin}{\evensidemargin}
\addtolength{\leftmargin}{\tmplength}
\settowidth{\labelsep}{X}
\addtolength{\leftmargin}{\labelsep}
\setlength{\labelwidth}{\tmplength}
}
\begin{flushleft}
\item[\textbf{Declaração}\hfill]
\begin{ttfamily}
public class function FileTruncate(Handle:THandle;NewSize:Int64 ): Longint;\end{ttfamily}


\end{flushleft}
\par
\item[\textbf{Descrição}]
\begin{itemize}
\item A função \textbf{\begin{ttfamily}FileTruncate\end{ttfamily}} reduz o tamanho do arquivo para o tamanho passado pelo parâmetro \textbf{NewSize}.

\begin{itemize}
\item \textbf{PARÂMETROS} \begin{itemize}
\item Handle:\begin{ttfamily}THandle\end{ttfamily}(\ref{mi.rtl.Types.TTypes-THandle}) {-} Handle do arquivo a ser truncado.
\item NewSize:\begin{ttfamily}Int64\end{ttfamily}(\ref{mi.rtl.Types.TTypes-int64}) {-} Tamanho do arquivo a se truncado.
\end{itemize}
\item \textbf{RETORNO} \begin{itemize}
\item \textbf{\begin{ttfamily}Longint\end{ttfamily}(\ref{mi.rtl.Types.TTypes-LongInt})} {-} 0(zero) se sucesso ou o código do erro se fracasso.
\end{itemize}
\item \textbf{REFERÊNCIA} \begin{itemize}
\item [Truncate file](https://www.freepascal.org/docs-html/rtl/sysutils/filetruncate.html)
\end{itemize}
\item \textbf{EXEMPLO}

\texttt{\\\nopagebreak[3]
\\\nopagebreak[3]
}\textbf{procedure}\texttt{~TFormTests.Button{\_}tes{\_}FileTruncateClick(Sender:~TObject);\\\nopagebreak[3]
~~\textit{//~Este~procedimento~Trunca~o~arquivos~'index.html'~para~100~bytes}\\\nopagebreak[3]
~~}\textbf{Var}\texttt{\\\nopagebreak[3]
~~~~aHandle:~TFiles.THandle;\\\nopagebreak[3]
~~~~NewSize:~TFiles.word;\\\nopagebreak[3]
~~~~err~~~~:~TFiles.integer;\\\nopagebreak[3]
}\textbf{Begin}\texttt{\\\nopagebreak[3]
~~NewSize~:=~100;\\\nopagebreak[3]
\\\nopagebreak[3]
~~}\textbf{with}\texttt{~TFiles~}\textbf{do}\texttt{\\\nopagebreak[3]
~~~~err~:=~FileOpen('index.html',fileMode,aHandle);\\\nopagebreak[3]
\\\nopagebreak[3]
~~}\textbf{with}\texttt{~TFiles~}\textbf{do}\texttt{\\\nopagebreak[3]
~~}\textbf{if}\texttt{~~err~=~0\\\nopagebreak[3]
~~}\textbf{then}\texttt{~}\textbf{begin}\texttt{\\\nopagebreak[3]
~~~~~~~~~\textit{//Executa~a~função~SysFileTruncate}\\\nopagebreak[3]
~~~~~~~~~err~:=~FileTruncate(aHandle,NewSize);\\\nopagebreak[3]
\\\nopagebreak[3]
~~~~~~~~~}\textbf{if}\texttt{~err~=~0\\\nopagebreak[3]
~~~~~~~~~}\textbf{then}\texttt{~showMessage('O~arquivo~foi~truncado~para~100~bytes')\\\nopagebreak[3]
~~~~~~~~~}\textbf{else}\texttt{~ShowMessage('Error:~'+ErrorMessage(err));\\\nopagebreak[3]
~~~~~~~}\textbf{end}\texttt{\\\nopagebreak[3]
~~}\textbf{else}\texttt{~ShowMessage('Error:~'+ErrorMessage(err));\\\nopagebreak[3]
}\textbf{end}\texttt{;\\
}
\end{itemize}
\end{itemize}

\end{list}
\paragraph*{FileCreate}\hspace*{\fill}

\begin{list}{}{
\settowidth{\tmplength}{\textbf{Declaração}}
\setlength{\itemindent}{0cm}
\setlength{\listparindent}{0cm}
\setlength{\leftmargin}{\evensidemargin}
\addtolength{\leftmargin}{\tmplength}
\settowidth{\labelsep}{X}
\addtolength{\leftmargin}{\labelsep}
\setlength{\labelwidth}{\tmplength}
}
\begin{flushleft}
\item[\textbf{Declaração}\hfill]
\begin{ttfamily}
public class function FileCreate(FileName: AnsiString; Mode: Longint; ShareMode : Cardinal; out Handle: THandle): Longint; overload;\end{ttfamily}


\end{flushleft}
\par
\item[\textbf{Descrição}]
\begin{itemize}
\item A função \textbf{\begin{ttfamily}FileCreate\end{ttfamily}} cria um novo arquivo e retorna um identificador para ele ou código do erro houver fracasso.

\begin{itemize}
\item \textbf{PARÂMETROS} \begin{itemize}
\item \textbf{FileName}: AnsiString {-} Nome do arquivo a ser criado;
\item \textbf{Mode}: \begin{ttfamily}Longint\end{ttfamily}(\ref{mi.rtl.Types.TTypes-LongInt}) {-} Modo de criação do arquivo. Veja \begin{ttfamily}FileMode\end{ttfamily}(\ref{mi.rtl.Consts.TConsts-FileMode}) para mais informações;
\end{itemize}
\item \textbf{RETORNO} \begin{itemize}
\item \textbf{\begin{ttfamily}THandle\end{ttfamily}(\ref{mi.rtl.Types.TTypes-THandle})} {-} Handle do arquivo criado;
\item \textbf{\begin{ttfamily}LongInt\end{ttfamily}(\ref{mi.rtl.Types.TTypes-LongInt})} {-} 0 (zero se sucesso ou o código do erro se fracasso.
\end{itemize}
\item \textbf{EXEMPLO}

\texttt{\\\nopagebreak[3]
\\\nopagebreak[3]
}\textbf{procedure}\texttt{~TFormTests.Button{\_}test{\_}FileCreateClick(Sender:~TObject);\\\nopagebreak[3]
\\\nopagebreak[3]
~~}\textbf{function}\texttt{~tes{\_}FileCreate(FileName:AnsiString;}\textbf{out}\texttt{~aHandle:THandle):~LongInt;\\\nopagebreak[3]
~~}\textbf{Begin}\texttt{\\\nopagebreak[3]
~~~~}\textbf{with}\texttt{~TFiles~}\textbf{do}\texttt{\\\nopagebreak[3]
~~~~~~result~:=~FileCreate(FileName,fmOpenReadWrite~}\textbf{or}\texttt{~fmShareExclusive,aHandle);\\\nopagebreak[3]
~~}\textbf{end}\texttt{;\\\nopagebreak[3]
\\\nopagebreak[3]
}\textbf{var}\texttt{\\\nopagebreak[3]
~~aHandle~:~~TFiles.THandle;\\\nopagebreak[3]
}\textbf{begin}\texttt{\\\nopagebreak[3]
\\\nopagebreak[3]
~~}\textbf{with}\texttt{~TFiles~}\textbf{do}\texttt{\\\nopagebreak[3]
~~}\textbf{if}\texttt{~tes{\_}FileCreate('text.txt',aHandle)~=~0\\\nopagebreak[3]
~~}\textbf{then}\texttt{~}\textbf{begin}\texttt{\\\nopagebreak[3]
~~~~~~~~~showMessage('Arquivo~text.txt~criado~na~pasta~corrente.');\\\nopagebreak[3]
~~~~~~~~~FileClose(aHandle);\\\nopagebreak[3]
~~~~~~~}\textbf{end}\texttt{;\\\nopagebreak[3]
}\textbf{end}\texttt{;\\
}
\end{itemize}
\end{itemize}

\end{list}
\paragraph*{DeleteFile}\hspace*{\fill}

\begin{list}{}{
\settowidth{\tmplength}{\textbf{Declaração}}
\setlength{\itemindent}{0cm}
\setlength{\listparindent}{0cm}
\setlength{\leftmargin}{\evensidemargin}
\addtolength{\leftmargin}{\tmplength}
\settowidth{\labelsep}{X}
\addtolength{\leftmargin}{\labelsep}
\setlength{\labelwidth}{\tmplength}
}
\begin{flushleft}
\item[\textbf{Declaração}\hfill]
\begin{ttfamily}
public class function DeleteFile(const FileName : AnsiString): SmallInt ;\end{ttfamily}


\end{flushleft}
\end{list}
\paragraph*{FileSize}\hspace*{\fill}

\begin{list}{}{
\settowidth{\tmplength}{\textbf{Declaração}}
\setlength{\itemindent}{0cm}
\setlength{\listparindent}{0cm}
\setlength{\leftmargin}{\evensidemargin}
\addtolength{\leftmargin}{\tmplength}
\settowidth{\labelsep}{X}
\addtolength{\leftmargin}{\labelsep}
\setlength{\labelwidth}{\tmplength}
}
\begin{flushleft}
\item[\textbf{Declaração}\hfill]
\begin{ttfamily}
public class function FileSize( FileName: string; out Count : int64):longint; overload;\end{ttfamily}


\end{flushleft}
\par
\item[\textbf{Descrição}]
\begin{itemize}
\item A função \textbf{\begin{ttfamily}FileSize\end{ttfamily}} retorna o tamanho do arquivo em bytes.

\begin{itemize}
\item \textbf{PARÂMETROS} \begin{itemize}
\item FileName: AnsiString {-} Nome do arquivo;
\item Count: \begin{ttfamily}Int64\end{ttfamily}(\ref{mi.rtl.Types.TTypes-int64}) {-} Número de bytes do arquivo passo do \textbf{FileName}.
\end{itemize}
\item \textbf{RETORNO} \begin{itemize}
\item \textbf{\begin{ttfamily}longint\end{ttfamily}(\ref{mi.rtl.Types.TTypes-LongInt})} {-} 0 (zero) se sucesso ou o código do erro se houver fracasso;
\item \textbf{Count} {-} Número de bytes do arquivo.
\end{itemize}
\item \textbf{EXEMPLO}

\texttt{\\\nopagebreak[3]
\\\nopagebreak[3]
~}\textbf{procedure}\texttt{~TMi{\_}Rtl{\_}Tests.Action{\_}Test{\_}FileSizeExecute(Sender:~TObject);\\\nopagebreak[3]
~~~~~\textit{//~Este~procedimento~obtem~o~tamanho~do~arquivo~em~bytes~do~arquivo~index.html;}\\\nopagebreak[3]
~~~}\textbf{var}\texttt{\\\nopagebreak[3]
~~~~~FileName:AnsiString;\\\nopagebreak[3]
~~~~~Count:Int64;\\\nopagebreak[3]
~~~~~err:Longint;\\\nopagebreak[3]
~}\textbf{Begin}\texttt{\\\nopagebreak[3]
~~~}\textbf{with}\texttt{~TFiles~}\textbf{do}\texttt{\\\nopagebreak[3]
~~~}\textbf{begin}\texttt{\\\nopagebreak[3]
~~~~~FileName~:=~'index.html';\\\nopagebreak[3]
~~~~~err~:=~FileSize(FileName,Count);\\\nopagebreak[3]
~~~~~}\textbf{if}\texttt{~err~{$<$}{$>$}~0\\\nopagebreak[3]
~~~~~}\textbf{Then}\texttt{~showMessage(ErrorMessage(err))\\\nopagebreak[3]
~~~~~}\textbf{else}\texttt{~showMessage('Tamanho~do~arquivo~é:~'+intToStr(Count));\\\nopagebreak[3]
~~~}\textbf{end}\texttt{;\\\nopagebreak[3]
}\textbf{end}\texttt{;\\
}
\end{itemize}
\end{itemize}

\end{list}
\paragraph*{FileSize}\hspace*{\fill}

\begin{list}{}{
\settowidth{\tmplength}{\textbf{Declaração}}
\setlength{\itemindent}{0cm}
\setlength{\listparindent}{0cm}
\setlength{\leftmargin}{\evensidemargin}
\addtolength{\leftmargin}{\tmplength}
\settowidth{\labelsep}{X}
\addtolength{\leftmargin}{\labelsep}
\setlength{\labelwidth}{\tmplength}
}
\begin{flushleft}
\item[\textbf{Declaração}\hfill]
\begin{ttfamily}
public class function FileSize( FileName: string):int64; overload;\end{ttfamily}


\end{flushleft}
\end{list}
\paragraph*{FileSizes}\hspace*{\fill}

\begin{list}{}{
\settowidth{\tmplength}{\textbf{Declaração}}
\setlength{\itemindent}{0cm}
\setlength{\listparindent}{0cm}
\setlength{\leftmargin}{\evensidemargin}
\addtolength{\leftmargin}{\tmplength}
\settowidth{\labelsep}{X}
\addtolength{\leftmargin}{\labelsep}
\setlength{\labelwidth}{\tmplength}
}
\begin{flushleft}
\item[\textbf{Declaração}\hfill]
\begin{ttfamily}
public class function FileSizes(Mask: AnsiString;out aFileSize:Int64): Longint; overload;\end{ttfamily}


\end{flushleft}
\par
\item[\textbf{Descrição}]
O função \textbf{\begin{ttfamily}FileSizes\end{ttfamily}} retorna em aFileSize a soma de todos os arquivos que satisfaça a mascara em path;

\begin{itemize}
\item \textbf{NOTA} \begin{itemize}
\item Se houver error retorna o código do error em \begin{ttfamily}FileSize\end{ttfamily}(\ref{mi.rtl.files.TFiles-FileSize})
\end{itemize}
\end{itemize}

\end{list}
\paragraph*{FileSeek}\hspace*{\fill}

\begin{list}{}{
\settowidth{\tmplength}{\textbf{Declaração}}
\setlength{\itemindent}{0cm}
\setlength{\listparindent}{0cm}
\setlength{\leftmargin}{\evensidemargin}
\addtolength{\leftmargin}{\tmplength}
\settowidth{\labelsep}{X}
\addtolength{\leftmargin}{\labelsep}
\setlength{\labelwidth}{\tmplength}
}
\begin{flushleft}
\item[\textbf{Declaração}\hfill]
\begin{ttfamily}
public class function FileSeek(const Handle:THandle; Const FOffset : Int64; Origin: LongInt; out NewPos: Int64): LongInt;\end{ttfamily}


\end{flushleft}
\par
\item[\textbf{Descrição}]
\begin{itemize}
\item A função \textbf{\begin{ttfamily}FileSeek\end{ttfamily}} posiciona o ponteiro do arquivo na posição FOffSet começando da origem.

\begin{itemize}
\item \textbf{PARÂMETROS} \begin{itemize}
\item \textbf{Handle: \begin{ttfamily}THandle\end{ttfamily}(\ref{mi.rtl.Types.TTypes-THandle})} {-} Handle do arquivo;
\item \textbf{FOffset: \begin{ttfamily}Int64\end{ttfamily}(\ref{mi.rtl.Types.TTypes-int64})} {-} Ponteiro do arquivo a ser posicionado;
\item \textbf{Origin: \begin{ttfamily}LongInt\end{ttfamily}(\ref{mi.rtl.Types.TTypes-LongInt})} {-} Origem do calculo da posição do arquivos pode ser: \begin{itemize}
\item \textbf{\begin{ttfamily}TConsts.fsFromBeginning\end{ttfamily}(\ref{mi.rtl.Consts.TConsts-fsFromBeginning})};
\item \textbf{\begin{ttfamily}TConsts.fsFromCurrent\end{ttfamily}(\ref{mi.rtl.Consts.TConsts-fsFromCurrent})};
\item \textbf{\begin{ttfamily}TConsts.fsFromEnd\end{ttfamily}(\ref{mi.rtl.Consts.TConsts-fsFromEnd})} ;
\end{itemize}
\item \textbf{NewPos: \begin{ttfamily}Int64\end{ttfamily}(\ref{mi.rtl.Types.TTypes-int64})} {-} Se tiver sucesso a função retorna neste parametro a nova posição do arquivo.
\end{itemize}
\item \textbf{RETORNO} \begin{itemize}
\item \textbf{\begin{ttfamily}LongInt\end{ttfamily}(\ref{mi.rtl.Types.TTypes-LongInt})} {-} 0 (zero se sucesso ou código do erro se fracasso. \begin{itemize}
\item Em \textbf{NewPos} retorna o número da posição atual do arquivo.
\end{itemize}
\item \textbf{EXEMPLO}

\texttt{\\\nopagebreak[3]
\\\nopagebreak[3]
}\textbf{procedure}\texttt{~TMi{\_}Rtl{\_}Tests.Action{\_}Test{\_}FileSeekExecute(Sender:~TObject);\\\nopagebreak[3]
~~\textit{//~Este~procedimento~posiciona~o~cursor~no~final~do~arquivo.}\\\nopagebreak[3]
~~}\textbf{Var}\texttt{\\\nopagebreak[3]
~~~~err~:~TFiles.integer;\\\nopagebreak[3]
~~~~h~~~:~TFiles.THandle;\\\nopagebreak[3]
~~~~NRec,Count~:~TFiles.Int64;\\\nopagebreak[3]
}\textbf{Begin}\texttt{\\\nopagebreak[3]
~~}\textbf{with}\texttt{~TFiles~}\textbf{do}\texttt{\\\nopagebreak[3]
~~}\textbf{begin}\texttt{\\\nopagebreak[3]
~~~~err~:=~FileOpen('index.html',h);\\\nopagebreak[3]
~~~~}\textbf{if}\texttt{~(err~=~0)~}\textbf{and}\texttt{~(fileSize('index.html',Count)~=~0)\\\nopagebreak[3]
~~~~}\textbf{Then}\texttt{~}\textbf{Begin}\texttt{\\\nopagebreak[3]
~~~~~~~~~~~\textit{//Posiciona~no~final~do~arquivo}\\\nopagebreak[3]
~~~~~~~~~~~err~:=~FileSeek(h,Count,fsFromBeginning,NRec);\\\nopagebreak[3]
~~~~~~~~~~~}\textbf{if}\texttt{~err~{$<$}{$>$}~0\\\nopagebreak[3]
~~~~~~~~~~~}\textbf{Then}\texttt{~ShowMessage(ErrorMessage(err));\\\nopagebreak[3]
~~~~~~~~~~~FileClose(h);\\\nopagebreak[3]
~~~~~~~~~}\textbf{end}\texttt{\\\nopagebreak[3]
~~~~}\textbf{else}\texttt{~ShowMessage(ErrorMessage(err));\\\nopagebreak[3]
~~}\textbf{end}\texttt{;\\\nopagebreak[3]
}\textbf{end}\texttt{;\\
}
\end{itemize}
\item \textbf{REFERÊNCIA}: \begin{itemize}
\item [\begin{ttfamily}FileSeek\end{ttfamily}](https://www.freepascal.org/docs-html/rtl/sysutils/fileseek.html)
\end{itemize}
\end{itemize}
\end{itemize}

\end{list}
\paragraph*{FileRead}\hspace*{\fill}

\begin{list}{}{
\settowidth{\tmplength}{\textbf{Declaração}}
\setlength{\itemindent}{0cm}
\setlength{\listparindent}{0cm}
\setlength{\leftmargin}{\evensidemargin}
\addtolength{\leftmargin}{\tmplength}
\settowidth{\labelsep}{X}
\addtolength{\leftmargin}{\labelsep}
\setlength{\labelwidth}{\tmplength}
}
\begin{flushleft}
\item[\textbf{Declaração}\hfill]
\begin{ttfamily}
public class function FileRead(const Handle: THandle; out Buffer; const Count: Int64; out BytesRead: int64): LongInt;\end{ttfamily}


\end{flushleft}
\par
\item[\textbf{Descrição}]
\begin{itemize}
\item O método \textbf{\begin{ttfamily}FileRead\end{ttfamily}} ler \textbf{Count} bytes do arquivo passado pelo \textbf{Handle} e retorna o número de bytes lidos em \textbf{BytesRead}

\begin{itemize}
\item \textbf{PARÂMETROS} \begin{itemize}
\item \textbf{Handle: \begin{ttfamily}THandle\end{ttfamily}(\ref{mi.rtl.Types.TTypes-THandle})} {-} Handle do arquivo
\item \textbf{Out Buffer} {-} Buffer para onde os dados devem ser salvos;
\item \textbf{Count: \begin{ttfamily}Int64\end{ttfamily}(\ref{mi.rtl.Types.TTypes-int64})} {-} Número de bytes a ser lido para o buffer;
\item \textbf{Out BytesRead}: \begin{ttfamily}Int64\end{ttfamily}(\ref{mi.rtl.Types.TTypes-int64}) {-} Número de Bytes lidos efetivamente.
\end{itemize}
\item \textbf{RETORNO} \begin{itemize}
\item \textbf{\begin{ttfamily}Longint\end{ttfamily}(\ref{mi.rtl.Types.TTypes-LongInt})} {-} 0 (zero se sucesso ou o código do erro se fracasso;
\item Em \textbf{Buffer} os dados lidos do arquivo;
\item Em \textbf{BytesRead} Retorna o número de bytes lidos efetivamente.
\item \textbf{EXEMPLO}

\texttt{\\\nopagebreak[3]
\\\nopagebreak[3]
}\textbf{procedure}\texttt{~TMi{\_}Rtl{\_}Tests.Action{\_}Test{\_}FileReadExecute(Sender:~TObject);\\\nopagebreak[3]
~~\textit{//~Este~procedimento~ler~os~últimos~10~bytes~do~arquivo~index.html}\\\nopagebreak[3]
~~}\textbf{Const}\texttt{~Size~=~10;\\\nopagebreak[3]
~~}\textbf{Var}\texttt{\\\nopagebreak[3]
~~~~err~~:~TFiles.integer;\\\nopagebreak[3]
~~~~h~~~~:~TFiles.THandle;\\\nopagebreak[3]
~~~~NRec,\\\nopagebreak[3]
~~~~Count~:~TFiles.Int64;\\\nopagebreak[3]
~~~~s~~~~~:~}\textbf{String}\texttt{[255];\\\nopagebreak[3]
~~~~BytesLidos:~Int64;\\\nopagebreak[3]
}\textbf{Begin}\texttt{\\\nopagebreak[3]
~~}\textbf{with}\texttt{~TFiles~}\textbf{do}\texttt{\\\nopagebreak[3]
~~}\textbf{begin}\texttt{\\\nopagebreak[3]
\\\nopagebreak[3]
~~~~err~:=~FileOpen('index.html',h);\\\nopagebreak[3]
~~~~}\textbf{if}\texttt{~(err~=~0)\\\nopagebreak[3]
~~~~}\textbf{Then}\texttt{~}\textbf{Begin}\texttt{\\\nopagebreak[3]
~~~~~~~~~~~err~:=~fileSize('index.html',Count);\\\nopagebreak[3]
~~~~~~~~~~~}\textbf{if}\texttt{~(~err~=~0)\\\nopagebreak[3]
~~~~~~~~~~~}\textbf{Then}\texttt{~}\textbf{begin}\texttt{\\\nopagebreak[3]
~~~~~~~~~~~~~~~~~\textit{//Posiciona~no~final~do~arquivo}\\\nopagebreak[3]
~~~~~~~~~~~~~~~~~err~:=~FileSeek(h,Count-Size-length(LF),fsFromBeginning,NRec);\\\nopagebreak[3]
~~~~~~~~~~~~~~~~~}\textbf{if}\texttt{~err~{$<$}{$>$}~0\\\nopagebreak[3]
~~~~~~~~~~~~~~~~~}\textbf{Then}\texttt{~ShowMessage(ErrorMessage(err))\\\nopagebreak[3]
~~~~~~~~~~~~~~~~~}\textbf{Else}\texttt{~}\textbf{Begin}\texttt{\\\nopagebreak[3]
~~~~~~~~~~~~~~~~~~~~~~~~err~:=~FileRead(h,s[1],Size+length(LF),BytesLidos);\\\nopagebreak[3]
~~~~~~~~~~~~~~~~~~~~~~~~}\textbf{if}\texttt{~(err~=~0)~}\textbf{and}\texttt{~(BytesLidos~=~Size+length(LF))\\\nopagebreak[3]
~~~~~~~~~~~~~~~~~~~~~~~~}\textbf{Then}\texttt{~}\textbf{Begin}\texttt{\\\nopagebreak[3]
~~~~~~~~~~~~~~~~~~~~~~~~~~~~~~~s[0]~:=~chr(Size);\\\nopagebreak[3]
~~~~~~~~~~~~~~~~~~~~~~~~~~~~~~~ShowMessage('Bytes~Lidos:~'+s);\\\nopagebreak[3]
~~~~~~~~~~~~~~~~~~~~~~~~~~~~~}\textbf{end}\texttt{\\\nopagebreak[3]
~~~~~~~~~~~~~~~~~~~~~~~~}\textbf{else}\texttt{~ShowMessage('Ponteiro~do~arquivo~é:~'+IntToStr(NRec));\\\nopagebreak[3]
~~~~~~~~~~~~~~~~~~~~~~}\textbf{end}\texttt{;\\\nopagebreak[3]
\\\nopagebreak[3]
~~~~~~~~~~~}\textbf{end}\texttt{\\\nopagebreak[3]
~~~~~~~~~~~}\textbf{else}\texttt{~ShowMessage(ErrorMessage(err));\\\nopagebreak[3]
~~~~~~~~~~~FileClose(h);\\\nopagebreak[3]
~~~~~~~~~}\textbf{end}\texttt{\\\nopagebreak[3]
~~~~}\textbf{else}\texttt{~ShowMessage(ErrorMessage(err));\\\nopagebreak[3]
~~}\textbf{end}\texttt{;\\\nopagebreak[3]
}\textbf{end}\texttt{;\\
}
\end{itemize}
\item \textbf{REFERÊNCIA} \begin{itemize}
\item [\begin{ttfamily}FileRead\end{ttfamily}](https://www.freepascal.org/docs-html/rtl/sysutils/fileread.html)
\end{itemize}
\end{itemize}
\end{itemize}

\end{list}
\paragraph*{FileWrite}\hspace*{\fill}

\begin{list}{}{
\settowidth{\tmplength}{\textbf{Declaração}}
\setlength{\itemindent}{0cm}
\setlength{\listparindent}{0cm}
\setlength{\leftmargin}{\evensidemargin}
\addtolength{\leftmargin}{\tmplength}
\settowidth{\labelsep}{X}
\addtolength{\leftmargin}{\labelsep}
\setlength{\labelwidth}{\tmplength}
}
\begin{flushleft}
\item[\textbf{Declaração}\hfill]
\begin{ttfamily}
public class function FileWrite(const Handle: THandle; const Buffer; const Count: Int64; out BytesWrites: int64): LongInt;\end{ttfamily}


\end{flushleft}
\par
\item[\textbf{Descrição}]
\begin{itemize}
\item O método \textbf{\begin{ttfamily}FileWrite\end{ttfamily}} grava \textbf{Count} bytes do arquivo passado pelo \textbf{Handle} e retorna o número de bytes escritos em \textbf{BytesWrite}

\begin{itemize}
\item \textbf{PARÂMETROS} \begin{itemize}
\item \textbf{Handle: \begin{ttfamily}THandle\end{ttfamily}(\ref{mi.rtl.Types.TTypes-THandle})} {-} Handle do arquivo
\item \textbf{Out Buffer} {-} Buffer de onde os dados devem ser escritos para o arquivo;
\item \textbf{Count: \begin{ttfamily}Int64\end{ttfamily}(\ref{mi.rtl.Types.TTypes-int64})} {-} Número de bytes a ser escritos do Buffer para o arquivo;
\item \textbf{Out BytesRead}: \begin{ttfamily}Int64\end{ttfamily}(\ref{mi.rtl.Types.TTypes-int64}) {-} Número de Bytes efetivamente escritos.
\end{itemize}
\item \textbf{RETORNO} \begin{itemize}
\item \textbf{\begin{ttfamily}Longint\end{ttfamily}(\ref{mi.rtl.Types.TTypes-LongInt})} {-} 0 (zero se sucesso ou o código do erro se fracasso;
\item Em \textbf{Buffer} os dados as ser escrito no arquivo;
\item Em \textbf{BytesRead} Retorna o número de bytes escritos efetivamente.
\item \textbf{EXEMPLO}

\texttt{\\\nopagebreak[3]
\\\nopagebreak[3]
\\\nopagebreak[3]
}\textbf{procedure}\texttt{~TMi{\_}Rtl{\_}Tests.Action{\_}Test{\_}FileWriteExecute(Sender:~TObject);\\\nopagebreak[3]
\\\nopagebreak[3]
~~\textit{//~Este~procedimento~adiciona~a~sequência~'-0123456789-0123456789'+LF~no~fim~do~arquivo~index.html}\\\nopagebreak[3]
\\\nopagebreak[3]
~~}\textbf{Var}\texttt{\\\nopagebreak[3]
~~~~Size~:~byte~=~255;\\\nopagebreak[3]
~~~~err~~:~TFiles.integer;\\\nopagebreak[3]
~~~~h~~~~:~TFiles.THandle;\\\nopagebreak[3]
\\\nopagebreak[3]
~~~~NRec,~Count~:~TFiles.Int64;\\\nopagebreak[3]
\\\nopagebreak[3]
~~~~s~~~~~:~}\textbf{String}\texttt{[255];\\\nopagebreak[3]
\\\nopagebreak[3]
~~~~BytesLidos,BytesWrites:~Int64;\\\nopagebreak[3]
}\textbf{Begin}\texttt{\\\nopagebreak[3]
~~}\textbf{with}\texttt{~TFiles~}\textbf{do}\texttt{\\\nopagebreak[3]
~~}\textbf{begin}\texttt{\\\nopagebreak[3]
~~~~s~:=~~'-0123456789-0123456789'+LF;\\\nopagebreak[3]
~~~~Size~:=~length(s);\\\nopagebreak[3]
\\\nopagebreak[3]
~~~~}\textbf{if}\texttt{~}\textbf{not}\texttt{~FileExists('index.html')\\\nopagebreak[3]
~~~~}\textbf{Then}\texttt{~Err~:=~FileCreate('index.html',fmOpenReadWrite,~fmShareCompat~}\textbf{or}\texttt{~fmShareDenyNone~,h)\\\nopagebreak[3]
~~~~}\textbf{else}\texttt{~Err~:=~FileOpen('index.html',h);\\\nopagebreak[3]
\\\nopagebreak[3]
~~~~}\textbf{if}\texttt{~(err~=~0)\\\nopagebreak[3]
~~~~}\textbf{Then}\texttt{~}\textbf{Begin}\texttt{\\\nopagebreak[3]
~~~~~~~~~~~err~:=~fileSize('index.html',Count);\\\nopagebreak[3]
~~~~~~~~~~~}\textbf{if}\texttt{~(~err~=~0)\\\nopagebreak[3]
~~~~~~~~~~~}\textbf{Then}\texttt{~}\textbf{begin}\texttt{\\\nopagebreak[3]
~~~~~~~~~~~~~~~~~~\textit{//Posiciona~no~final~do~arquivo~-~length(LF)}\\\nopagebreak[3]
~~~~~~~~~~~~~~~~~~}\textbf{if}\texttt{~Count~{$>$}=~size\\\nopagebreak[3]
~~~~~~~~~~~~~~~~~~}\textbf{then}\texttt{~err~:=~FileSeek(h,Count-length(LF),fsFromBeginning,NRec)\\\nopagebreak[3]
~~~~~~~~~~~~~~~~~~}\textbf{else}\texttt{~err~:=~FileSeek(h,0,fsFromBeginning,NRec);\\\nopagebreak[3]
\\\nopagebreak[3]
~~~~~~~~~~~~~~~~~~}\textbf{if}\texttt{~err~=~0\\\nopagebreak[3]
~~~~~~~~~~~~~~~~~~}\textbf{Then}\texttt{~}\textbf{Begin}\texttt{\\\nopagebreak[3]
~~~~~~~~~~~~~~~~~~~~~~~~~\textit{//Acressenta~string~'-0123456789-0123456789'+LF~no~arquivo~'index.html'}\\\nopagebreak[3]
~~~~~~~~~~~~~~~~~~~~~~~~~err:=~FileWrite(h,s[1],length(s),BytesWrites);\\\nopagebreak[3]
~~~~~~~~~~~~~~~~~~~~~~~~~}\textbf{if}\texttt{~(err~=~0)~}\textbf{and}\texttt{~(BytesWrites~=~(length(s)))\\\nopagebreak[3]
~~~~~~~~~~~~~~~~~~~~~~~~~}\textbf{then}\texttt{~}\textbf{Begin}\texttt{\\\nopagebreak[3]
~~~~~~~~~~~~~~~~~~~~~~~~~~~~~~~~ShowMessage('A~sequência~'+S+'~foi~adicionada~no~fim~do~arquivo.')\\\nopagebreak[3]
~~~~~~~~~~~~~~~~~~~~~~~~~~~~~~}\textbf{end}\texttt{\\\nopagebreak[3]
~~~~~~~~~~~~~~~~~~~~~~~~~}\textbf{else}\texttt{~}\textbf{begin}\texttt{\\\nopagebreak[3]
~~~~~~~~~~~~~~~~~~~~~~~~~~~~~~~~}\textbf{if}\texttt{~(err~{$<$}{$>$}~0)\\\nopagebreak[3]
~~~~~~~~~~~~~~~~~~~~~~~~~~~~~~~~}\textbf{Then}\texttt{~ShowMessage(ErrorMessage(err))\\\nopagebreak[3]
~~~~~~~~~~~~~~~~~~~~~~~~~~~~~~~~}\textbf{else}\texttt{~ShowMessage('Número~de~bytes~escritos~diferente~de:~'+IntToStr(length(s)));\\\nopagebreak[3]
~~~~~~~~~~~~~~~~~~~~~~~~~~~~~~}\textbf{end}\texttt{;\\\nopagebreak[3]
~~~~~~~~~~~~~~~~~~~~~~~}\textbf{end}\texttt{\\\nopagebreak[3]
~~~~~~~~~~~~~~~~~~}\textbf{else}\texttt{~}\textbf{Begin}\texttt{\\\nopagebreak[3]
~~~~~~~~~~~~~~~~~~~~~~~~~ShowMessage(ErrorMessage(err))~~~;\\\nopagebreak[3]
~~~~~~~~~~~~~~~~~~~~~~~}\textbf{end}\texttt{;\\\nopagebreak[3]
~~~~~~~~~~~}\textbf{end}\texttt{\\\nopagebreak[3]
~~~~~~~~~~~}\textbf{else}\texttt{~ShowMessage(ErrorMessage(err));\\\nopagebreak[3]
~~~~~~~~~~~FileClose(h);\\\nopagebreak[3]
~~~~~~~~~}\textbf{end}\texttt{\\\nopagebreak[3]
~~~~}\textbf{else}\texttt{~ShowMessage(ErrorMessage(err));\\\nopagebreak[3]
~~}\textbf{end}\texttt{;\\\nopagebreak[3]
}\textbf{end}\texttt{;\\
}
\end{itemize}
\item \textbf{REFERÊNCIA} \begin{itemize}
\item [FileWrites](https://www.freepascal.org/docs-html/rtl/sysutils/filewrite.html)
\end{itemize}
\end{itemize}
\end{itemize}

\end{list}
\paragraph*{FileExists}\hspace*{\fill}

\begin{list}{}{
\settowidth{\tmplength}{\textbf{Declaração}}
\setlength{\itemindent}{0cm}
\setlength{\listparindent}{0cm}
\setlength{\leftmargin}{\evensidemargin}
\addtolength{\leftmargin}{\tmplength}
\settowidth{\labelsep}{X}
\addtolength{\leftmargin}{\labelsep}
\setlength{\labelwidth}{\tmplength}
}
\begin{flushleft}
\item[\textbf{Declaração}\hfill]
\begin{ttfamily}
public class Function FileExists(Const FileName : AnsiString) : Boolean;\end{ttfamily}


\end{flushleft}
\par
\item[\textbf{Descrição}]
\begin{itemize}
\item A classe método \begin{ttfamily}FileExists\end{ttfamily} checa se o arquivo passado no parâmetro existe.
\end{itemize}

\end{list}
\paragraph*{DirectoryExists}\hspace*{\fill}

\begin{list}{}{
\settowidth{\tmplength}{\textbf{Declaração}}
\setlength{\itemindent}{0cm}
\setlength{\listparindent}{0cm}
\setlength{\leftmargin}{\evensidemargin}
\addtolength{\leftmargin}{\tmplength}
\settowidth{\labelsep}{X}
\addtolength{\leftmargin}{\labelsep}
\setlength{\labelwidth}{\tmplength}
}
\begin{flushleft}
\item[\textbf{Declaração}\hfill]
\begin{ttfamily}
public class Function DirectoryExists(Const Directory : AnsiString) : Boolean;\end{ttfamily}


\end{flushleft}
\par
\item[\textbf{Descrição}]
\begin{itemize}
\item A classe método \begin{ttfamily}DirectoryExists\end{ttfamily} checa se o diretório passado no parâmetro existe
\end{itemize}

\end{list}
\paragraph*{CreateDir}\hspace*{\fill}

\begin{list}{}{
\settowidth{\tmplength}{\textbf{Declaração}}
\setlength{\itemindent}{0cm}
\setlength{\listparindent}{0cm}
\setlength{\leftmargin}{\evensidemargin}
\addtolength{\leftmargin}{\tmplength}
\settowidth{\labelsep}{X}
\addtolength{\leftmargin}{\labelsep}
\setlength{\labelwidth}{\tmplength}
}
\begin{flushleft}
\item[\textbf{Declaração}\hfill]
\begin{ttfamily}
public class Function CreateDir(Const NewDir : AnsiString) : Boolean;\end{ttfamily}


\end{flushleft}
\par
\item[\textbf{Descrição}]
\begin{itemize}
\item A classe método \begin{ttfamily}CreateDir\end{ttfamily} cria diretório passado no parâmetro
\end{itemize}

\end{list}
\paragraph*{GetTempFileName}\hspace*{\fill}

\begin{list}{}{
\settowidth{\tmplength}{\textbf{Declaração}}
\setlength{\itemindent}{0cm}
\setlength{\listparindent}{0cm}
\setlength{\leftmargin}{\evensidemargin}
\addtolength{\leftmargin}{\tmplength}
\settowidth{\labelsep}{X}
\addtolength{\leftmargin}{\labelsep}
\setlength{\labelwidth}{\tmplength}
}
\begin{flushleft}
\item[\textbf{Declaração}\hfill]
\begin{ttfamily}
public class function GetTempFileName(const Dir : string): string ;\end{ttfamily}


\end{flushleft}
\par
\item[\textbf{Descrição}]
A classe método \textbf{\begin{ttfamily}GetTempFileName\end{ttfamily}} retorna o nome de um arquivo temporário no diretório Dir.

\begin{itemize}
\item \textbf{NOTA} \begin{itemize}
\item Se Dir estiver vazio, o valor retornado por \begin{ttfamily}GetTempDir\end{ttfamily}(\ref{mi.rtl.files.TFiles-GetTempDir}) será usado.
\item O Prefix será 'TMP'.
\item Em caso de erro, uma string vazia é retornada.
\end{itemize}
\end{itemize}

\end{list}
\paragraph*{GetTempDir}\hspace*{\fill}

\begin{list}{}{
\settowidth{\tmplength}{\textbf{Declaração}}
\setlength{\itemindent}{0cm}
\setlength{\listparindent}{0cm}
\setlength{\leftmargin}{\evensidemargin}
\addtolength{\leftmargin}{\tmplength}
\settowidth{\labelsep}{X}
\addtolength{\leftmargin}{\labelsep}
\setlength{\labelwidth}{\tmplength}
}
\begin{flushleft}
\item[\textbf{Declaração}\hfill]
\begin{ttfamily}
public class function GetTempDir(): string; overload;\end{ttfamily}


\end{flushleft}
\par
\item[\textbf{Descrição}]
A classe function \textbf{\begin{ttfamily}GetTempDir\end{ttfamily}} retorna o diretório temporário do sistema.

\begin{itemize}
\item \textbf{NOTA} \begin{itemize}
\item O nome retornado terminará com um caractere delimitador de diretório.
\item Não há garantia de que esse diretório exista ou seja gravável pelo usuário.
\end{itemize}
\end{itemize}

\end{list}
\paragraph*{GetTempDir}\hspace*{\fill}

\begin{list}{}{
\settowidth{\tmplength}{\textbf{Declaração}}
\setlength{\itemindent}{0cm}
\setlength{\listparindent}{0cm}
\setlength{\leftmargin}{\evensidemargin}
\addtolength{\leftmargin}{\tmplength}
\settowidth{\labelsep}{X}
\addtolength{\leftmargin}{\labelsep}
\setlength{\labelwidth}{\tmplength}
}
\begin{flushleft}
\item[\textbf{Declaração}\hfill]
\begin{ttfamily}
public class function GetTempDir(Const env:String;out path:PathStr):SmallInt; overload;\end{ttfamily}


\end{flushleft}
\par
\item[\textbf{Descrição}]
A classe function \textbf{\begin{ttfamily}GetTempDir\end{ttfamily}} retorna o diretório temporário do sistema.

\begin{itemize}
\item \textbf{PARÂMETROS} \begin{itemize}
\item \textbf{Const env} : Variável de ambiente que tenha que contenha a pasta de arquivos temporários.
\item **out path:\begin{ttfamily}PathStr\end{ttfamily}(\ref{mi.rtl.Types.TTypes-PathStr}) : Retorna a pasta dos arquivos temporários.
\end{itemize}
\item \textbf{RETORNA} \begin{itemize}
\item SmallInt : Código do erro se houver ou zero (0) se conseguiu gerar o nome da pasta.
\end{itemize}
\end{itemize}

\end{list}
\paragraph*{GetDriveType}\hspace*{\fill}

\begin{list}{}{
\settowidth{\tmplength}{\textbf{Declaração}}
\setlength{\itemindent}{0cm}
\setlength{\listparindent}{0cm}
\setlength{\leftmargin}{\evensidemargin}
\addtolength{\leftmargin}{\tmplength}
\settowidth{\labelsep}{X}
\addtolength{\leftmargin}{\labelsep}
\setlength{\labelwidth}{\tmplength}
}
\begin{flushleft}
\item[\textbf{Declaração}\hfill]
\begin{ttfamily}
public class function GetDriveType(aPath : AnsiString): TDriveType; Overload;\end{ttfamily}


\end{flushleft}
\par
\item[\textbf{Descrição}]
\begin{itemize}
\item A função \textbf{\begin{ttfamily}GetDriveType\end{ttfamily}} é usada para saber o tipo de dispositivo associado a pasta.

\begin{itemize}
\item \textbf{PARÂMETRO} \begin{itemize}
\item \textbf{aPath} {-} A pasta dona do arquivo.
\end{itemize}
\item \textbf{RETORNA} \begin{itemize}
\item O tipo de dispositivo do tipo \begin{ttfamily}TDriveType\end{ttfamily}(\ref{mi.rtl.Types.TTypes-TDriveType}).
\end{itemize}
\end{itemize}
\end{itemize}

\end{list}
\paragraph*{DuplicateHandle}\hspace*{\fill}

\begin{list}{}{
\settowidth{\tmplength}{\textbf{Declaração}}
\setlength{\itemindent}{0cm}
\setlength{\listparindent}{0cm}
\setlength{\leftmargin}{\evensidemargin}
\addtolength{\leftmargin}{\tmplength}
\settowidth{\labelsep}{X}
\addtolength{\leftmargin}{\labelsep}
\setlength{\labelwidth}{\tmplength}
}
\begin{flushleft}
\item[\textbf{Declaração}\hfill]
\begin{ttfamily}
public Class function DuplicateHandle(hSourceHandle: THandle;Var lpTargeTHandle: THandle) : Longint;\end{ttfamily}


\end{flushleft}
\par
\item[\textbf{Descrição}]
\begin{itemize}
\item A classe método \begin{ttfamily}DuplicateHandle\end{ttfamily} duplica o handle do arquivo no windows e no linux essa função não funciona.
\end{itemize}

\end{list}
\paragraph*{FileFlushBuffers}\hspace*{\fill}

\begin{list}{}{
\settowidth{\tmplength}{\textbf{Declaração}}
\setlength{\itemindent}{0cm}
\setlength{\listparindent}{0cm}
\setlength{\leftmargin}{\evensidemargin}
\addtolength{\leftmargin}{\tmplength}
\settowidth{\labelsep}{X}
\addtolength{\leftmargin}{\labelsep}
\setlength{\labelwidth}{\tmplength}
}
\begin{flushleft}
\item[\textbf{Declaração}\hfill]
\begin{ttfamily}
public Class function FileFlushBuffers(aHandle: THandle): longint; overload;\end{ttfamily}


\end{flushleft}
\par
\item[\textbf{Descrição}]
A classe function \textbf{\begin{ttfamily}FileFlushBuffers\end{ttfamily}} descarrega o buffer do arquivo passado por aHandle

\begin{itemize}
\item \textbf{REFERÊNCIAS} \begin{itemize}
\item [Linux](https://www.freepascal.org/docs-html/rtl/unix/fpfsync.html)
\item [windows](https://docs.microsoft.com/en-us/windows/win32/api/fileapi/nf-fileapi-flushfilebuffers)
\end{itemize}
\end{itemize}

\end{list}
\paragraph*{LockFile}\hspace*{\fill}

\begin{list}{}{
\settowidth{\tmplength}{\textbf{Declaração}}
\setlength{\itemindent}{0cm}
\setlength{\listparindent}{0cm}
\setlength{\leftmargin}{\evensidemargin}
\addtolength{\leftmargin}{\tmplength}
\settowidth{\labelsep}{X}
\addtolength{\leftmargin}{\labelsep}
\setlength{\labelwidth}{\tmplength}
}
\begin{flushleft}
\item[\textbf{Declaração}\hfill]
\begin{ttfamily}
public class function LockFile({\_}Handle:THandle; {\_}LockStart, {\_}LockLength: Int64): LongInt;\end{ttfamily}


\end{flushleft}
\par
\item[\textbf{Descrição}]
\begin{itemize}
\item A função \begin{ttfamily}LockFile\end{ttfamily} trava uma região do arquivo. {-}\textbf{NOTA} \begin{itemize}
\item O \begin{ttfamily}lockfile\end{ttfamily} do linux não bloqueia região do arquivo e sim o arquivo todo.
\end{itemize}\begin{itemize}
\item \textbf{REFERÊNCIAS} \begin{itemize}
\item [Linux](https://www.freepascal.org/docs-html/rtl/unix/fpflock.html)
\item [windows](https://docs.microsoft.com/en-us/windows/win32/api/fileapi/nf-fileapi-lockfile)
\end{itemize}
\end{itemize}
\end{itemize}

\end{list}
\paragraph*{UnLockFile}\hspace*{\fill}

\begin{list}{}{
\settowidth{\tmplength}{\textbf{Declaração}}
\setlength{\itemindent}{0cm}
\setlength{\listparindent}{0cm}
\setlength{\leftmargin}{\evensidemargin}
\addtolength{\leftmargin}{\tmplength}
\settowidth{\labelsep}{X}
\addtolength{\leftmargin}{\labelsep}
\setlength{\labelwidth}{\tmplength}
}
\begin{flushleft}
\item[\textbf{Declaração}\hfill]
\begin{ttfamily}
public class function UnLockFile({\_}Handle:THandle; {\_}LockStart, {\_}LockLength: Int64): LongInt;\end{ttfamily}


\end{flushleft}
\par
\item[\textbf{Descrição}]
\begin{itemize}
\item A classe método \begin{ttfamily}UnLockFile\end{ttfamily} destrava a região travada por \textbf{\begin{ttfamily}LockFile\end{ttfamily}(\ref{mi.rtl.files.TFiles-LockFile})}.

\begin{itemize}
\item \textbf{NOTA} \begin{itemize}
\item Funciona no linux mais não funciona do linux.
\end{itemize}
\end{itemize}
\end{itemize}

\end{list}
\paragraph*{FindFiles}\hspace*{\fill}

\begin{list}{}{
\settowidth{\tmplength}{\textbf{Declaração}}
\setlength{\itemindent}{0cm}
\setlength{\listparindent}{0cm}
\setlength{\leftmargin}{\evensidemargin}
\addtolength{\leftmargin}{\tmplength}
\settowidth{\labelsep}{X}
\addtolength{\leftmargin}{\labelsep}
\setlength{\labelwidth}{\tmplength}
}
\begin{flushleft}
\item[\textbf{Declaração}\hfill]
\begin{ttfamily}
public class procedure FindFiles(Mask : AnsiString; FileAttrs : Cardinal; var List : TStringList);\end{ttfamily}


\end{flushleft}
\par
\item[\textbf{Descrição}]
\begin{itemize}
\item A classe método \textbf{\begin{ttfamily}FindFiles\end{ttfamily}} retorna uma lista de nomes de arquivos e diretórios que satisfazem os parâmetros: \textbf{Mask} e \textbf{FileAttrs}

\begin{itemize}
\item \textbf{EXEMPLO DE USO}

\texttt{\\\nopagebreak[3]
\\\nopagebreak[3]
}\textbf{procedure}\texttt{~TMi{\_}Rtl{\_}Tests.FormCreate(Sender:~TObject);\\\nopagebreak[3]
}\textbf{begin}\texttt{\\\nopagebreak[3]
~~ListFiles~:=~TMiStringList.Create;\\\nopagebreak[3]
~~Action{\_}test{\_}FindFirstExecute(Self);\\\nopagebreak[3]
}\textbf{end}\texttt{;\\\nopagebreak[3]
\\\nopagebreak[3]
}\textbf{procedure}\texttt{~TMi{\_}Rtl{\_}Tests.Action{\_}test{\_}FindFirstExecute(Sender:~TObject);\\\nopagebreak[3]
~~\textit{//Este~procedimento~ler~os~atributos~da~pasta:~'.'}\\\nopagebreak[3]
\\\nopagebreak[3]
~~}\textbf{function}\texttt{~GetInfoFile(FileName:}\textbf{string}\texttt{;attribute~:~Cardinal;~}\textbf{out}\texttt{~info~:~TSearchRec):~Integer;\\\nopagebreak[3]
\\\nopagebreak[3]
~~}\textbf{begin}\texttt{\\\nopagebreak[3]
~~~~~Result~:=~FindFirst(ExpandFileName(FileName),attribute,Info);\\\nopagebreak[3]
~~~~~}\textbf{if}\texttt{~Result~=~0\\\nopagebreak[3]
~~~~~}\textbf{then}\texttt{~}\textbf{Begin}\texttt{\\\nopagebreak[3]
~~~~~~~~~~~~ShowMessage('O~arquivo~'+fileName+'~contém~o~atributo:~'+intToStr(attribute));\\\nopagebreak[3]
~~~~~~~~~~}\textbf{end}\texttt{\\\nopagebreak[3]
~~~~~}\textbf{else}\texttt{~}\textbf{begin}\texttt{\\\nopagebreak[3]
~~~~~~~~~~~~ShowMessage('O~arquivo~'+fileName+'~não~contém~o~atributo:~'+intToStr(attribute));\\\nopagebreak[3]
~~~~~~~~~~}\textbf{end}\texttt{;\\\nopagebreak[3]
~~}\textbf{end}\texttt{;\\\nopagebreak[3]
\\\nopagebreak[3]
~~}\textbf{var}\texttt{\\\nopagebreak[3]
~~~Info:~TSearchRec;\\\nopagebreak[3]
~~~err,i~:~integer;\\\nopagebreak[3]
\\\nopagebreak[3]
~~~}\textbf{const}\texttt{~FileAttrs~:~Cardinal~=~faHidden~}\textbf{or}\texttt{\\\nopagebreak[3]
~~~~~~~~~~~~~~~~~~~~~~~~~~~~~~~~faReadOnly~}\textbf{or}\texttt{\\\nopagebreak[3]
~~~~~~~~~~~~~~~~~~~~~~~~~~~~~~~~faSysFile~}\textbf{or}\texttt{\\\nopagebreak[3]
~~~~~~~~~~~~~~~~~~~~~~~~~~~~~~~~faArchive~}\textbf{or}\texttt{\\\nopagebreak[3]
~~~~~~~~~~~~~~~~~~~~~~~~~~~~~~~~faAnyFile~}\textbf{or}\texttt{\\\nopagebreak[3]
~~~~~~~~~~~~~~~~~~~~~~~~~~~~~~~~faSymLink~}\textbf{or}\texttt{\\\nopagebreak[3]
~~~~~~~~~~~~~~~~~~~~~~~~~~~~~~~~faDirectory~;\\\nopagebreak[3]
\\\nopagebreak[3]
\\\nopagebreak[3]
}\textbf{begin}\texttt{\\\nopagebreak[3]
~~ListFiles.Clear;\\\nopagebreak[3]
~~ListBox1.Clear;\\\nopagebreak[3]
\\\nopagebreak[3]
~~FileAttrs~:=~0;\\\nopagebreak[3]
\\\nopagebreak[3]
~~}\textbf{if}\texttt{~CheckBox{\_}faHidden.Checked\\\nopagebreak[3]
~~}\textbf{then}\texttt{~FileAttrs~:=~faHidden;\\\nopagebreak[3]
\\\nopagebreak[3]
~~}\textbf{if}\texttt{~CheckBox{\_}faReadOnly.checked\\\nopagebreak[3]
~~}\textbf{then}\texttt{~FileAttrs~:=~FileAttrs~}\textbf{or}\texttt{~faReadOnly;\\\nopagebreak[3]
\\\nopagebreak[3]
~~}\textbf{if}\texttt{~CheckBox{\_}faSysFile.checked\\\nopagebreak[3]
~~}\textbf{then}\texttt{~FileAttrs~:=~FileAttrs~}\textbf{or}\texttt{~faSysFile;\\\nopagebreak[3]
\\\nopagebreak[3]
~~}\textbf{if}\texttt{~CheckBox{\_}faArchive.checked\\\nopagebreak[3]
~~}\textbf{then}\texttt{~FileAttrs~:=~FileAttrs~}\textbf{or}\texttt{~faArchive;\\\nopagebreak[3]
\\\nopagebreak[3]
~~}\textbf{if}\texttt{~CheckBox{\_}faAnyFile.checked\\\nopagebreak[3]
~~}\textbf{then}\texttt{~FileAttrs~:=~FileAttrs~}\textbf{or}\texttt{~faAnyFile;\\\nopagebreak[3]
\\\nopagebreak[3]
~~}\textbf{if}\texttt{~CheckBox{\_}faSymLink.checked\\\nopagebreak[3]
~~}\textbf{then}\texttt{~FileAttrs~:=~FileAttrs~}\textbf{or}\texttt{~faSymLink;\\\nopagebreak[3]
\\\nopagebreak[3]
~~}\textbf{if}\texttt{~CheckBox{\_}faDirectory.checked\\\nopagebreak[3]
~~}\textbf{then}\texttt{~FileAttrs~:=~FileAttrs~}\textbf{or}\texttt{~faDirectory;\\\nopagebreak[3]
\\\nopagebreak[3]
~~}\textbf{with}\texttt{~TMI{\_}ui{\_}types~}\textbf{do}\texttt{\\\nopagebreak[3]
~~~~FindFiles(Edit1.Text,FileAttrs~,ListFiles~);\\\nopagebreak[3]
\\\nopagebreak[3]
~~LabelCount.Caption~:=~Format('ListFiles.Count~{\%}d',[ListFiles.Count]);\\\nopagebreak[3]
~~LabelCount.Show;\\\nopagebreak[3]
~~}\textbf{if}\texttt{~ListFiles.Count~{$>$}~0\\\nopagebreak[3]
~~}\textbf{then}\texttt{~}\textbf{begin}\texttt{\\\nopagebreak[3]
~~~~~~~~~~}\textbf{for}\texttt{~i~:=~0~}\textbf{to}\texttt{~ListFiles.Count-1~}\textbf{do}\texttt{\\\nopagebreak[3]
~~~~~~~~~~}\textbf{begin}\texttt{\\\nopagebreak[3]
~~~~~~~~~~~~ListBox1.Items.Add(ListFiles[i]);\\\nopagebreak[3]
~~~~~~~~~~}\textbf{end}\texttt{;\\\nopagebreak[3]
~~~~~~~}\textbf{end}\texttt{;\\\nopagebreak[3]
}\textbf{end}\texttt{;\\\nopagebreak[3]
\\\nopagebreak[3]
\\\nopagebreak[3]
}\textbf{procedure}\texttt{~TMi{\_}Rtl{\_}Tests.Edit1Change(Sender:~TObject);\\\nopagebreak[3]
}\textbf{begin}\texttt{\\\nopagebreak[3]
~~Action{\_}test{\_}FindFirstExecute(Self);\\\nopagebreak[3]
}\textbf{end}\texttt{;\\\nopagebreak[3]
\\\nopagebreak[3]
}\textbf{procedure}\texttt{~TMi{\_}Rtl{\_}Tests.CheckBox{\_}faAnyFileChange(Sender:~TObject);\\\nopagebreak[3]
}\textbf{begin}\texttt{\\\nopagebreak[3]
~~Action{\_}test{\_}FindFirstExecute(Self);\\\nopagebreak[3]
}\textbf{end}\texttt{;\\\nopagebreak[3]
\\\nopagebreak[3]
}\textbf{procedure}\texttt{~TMi{\_}Rtl{\_}Tests.CheckBox{\_}faArchiveChange(Sender:~TObject);\\\nopagebreak[3]
}\textbf{begin}\texttt{\\\nopagebreak[3]
~~Action{\_}test{\_}FindFirstExecute(Self);\\\nopagebreak[3]
}\textbf{end}\texttt{;\\\nopagebreak[3]
\\\nopagebreak[3]
}\textbf{procedure}\texttt{~TMi{\_}Rtl{\_}Tests.CheckBox{\_}faDirectoryChange(Sender:~TObject);\\\nopagebreak[3]
}\textbf{begin}\texttt{\\\nopagebreak[3]
~~Action{\_}test{\_}FindFirstExecute(Self);\\\nopagebreak[3]
}\textbf{end}\texttt{;\\\nopagebreak[3]
\\\nopagebreak[3]
}\textbf{procedure}\texttt{~TMi{\_}Rtl{\_}Tests.CheckBox{\_}faHiddenChange(Sender:~TObject);\\\nopagebreak[3]
}\textbf{begin}\texttt{\\\nopagebreak[3]
~~Action{\_}test{\_}FindFirstExecute(Self);\\\nopagebreak[3]
}\textbf{end}\texttt{;\\\nopagebreak[3]
\\\nopagebreak[3]
}\textbf{procedure}\texttt{~TMi{\_}Rtl{\_}Tests.CheckBox{\_}faReadOnlyChange(Sender:~TObject);\\\nopagebreak[3]
}\textbf{begin}\texttt{\\\nopagebreak[3]
~~Action{\_}test{\_}FindFirstExecute(Self);\\\nopagebreak[3]
}\textbf{end}\texttt{;\\\nopagebreak[3]
\\\nopagebreak[3]
}\textbf{procedure}\texttt{~TMi{\_}Rtl{\_}Tests.CheckBox{\_}faSymLinkChange(Sender:~TObject);\\\nopagebreak[3]
}\textbf{begin}\texttt{\\\nopagebreak[3]
~~Action{\_}test{\_}FindFirstExecute(Self);\\\nopagebreak[3]
}\textbf{end}\texttt{;\\\nopagebreak[3]
\\\nopagebreak[3]
}\textbf{procedure}\texttt{~TMi{\_}Rtl{\_}Tests.CheckBox{\_}faSysFileChange(Sender:~TObject);\\\nopagebreak[3]
}\textbf{begin}\texttt{\\\nopagebreak[3]
~~Action{\_}test{\_}FindFirstExecute(Self);\\\nopagebreak[3]
}\textbf{end}\texttt{;\\
}
\end{itemize}
\end{itemize}

\end{list}
\paragraph*{GetCurrentDir}\hspace*{\fill}

\begin{list}{}{
\settowidth{\tmplength}{\textbf{Declaração}}
\setlength{\itemindent}{0cm}
\setlength{\listparindent}{0cm}
\setlength{\leftmargin}{\evensidemargin}
\addtolength{\leftmargin}{\tmplength}
\settowidth{\labelsep}{X}
\addtolength{\leftmargin}{\labelsep}
\setlength{\labelwidth}{\tmplength}
}
\begin{flushleft}
\item[\textbf{Declaração}\hfill]
\begin{ttfamily}
public class function GetCurrentDir: AnsiString;\end{ttfamily}


\end{flushleft}
\par
\item[\textbf{Descrição}]
A classe método \textbf{\begin{ttfamily}GetCurrentDir\end{ttfamily}} retorna o corrente pasta.

\end{list}
\paragraph*{SetCurrentDir}\hspace*{\fill}

\begin{list}{}{
\settowidth{\tmplength}{\textbf{Declaração}}
\setlength{\itemindent}{0cm}
\setlength{\listparindent}{0cm}
\setlength{\leftmargin}{\evensidemargin}
\addtolength{\leftmargin}{\tmplength}
\settowidth{\labelsep}{X}
\addtolength{\leftmargin}{\labelsep}
\setlength{\labelwidth}{\tmplength}
}
\begin{flushleft}
\item[\textbf{Declaração}\hfill]
\begin{ttfamily}
public class function SetCurrentDir( const NewDir : AnsiString):Boolean;\end{ttfamily}


\end{flushleft}
\par
\item[\textbf{Descrição}]
A classe método \textbf{\begin{ttfamily}SetCurrentDir\end{ttfamily}} define a pasta passado por \textbf{NewDir} como pasta corrente.

\textbf{PARÂMETRO} \begin{itemize}
\item \textbf{NewDir} {-} Nome da pasta a ser definida.
\end{itemize} \textbf{RETORNA} \begin{itemize}
\item \textbf{TRUE} {-} Se sucesso
\item \textbf{FALSE} {-} Se fracasso;
\end{itemize}

\end{list}
\paragraph*{IsDirectory}\hspace*{\fill}

\begin{list}{}{
\settowidth{\tmplength}{\textbf{Declaração}}
\setlength{\itemindent}{0cm}
\setlength{\listparindent}{0cm}
\setlength{\leftmargin}{\evensidemargin}
\addtolength{\leftmargin}{\tmplength}
\settowidth{\labelsep}{X}
\addtolength{\leftmargin}{\labelsep}
\setlength{\labelwidth}{\tmplength}
}
\begin{flushleft}
\item[\textbf{Declaração}\hfill]
\begin{ttfamily}
public class function IsDirectory( const Directory : AnsiString):Boolean;\end{ttfamily}


\end{flushleft}
\par
\item[\textbf{Descrição}]
A classe método \textbf{\begin{ttfamily}IsDirectory\end{ttfamily}} checa se a pasta passado por \textbf{Directory} é uma pasta válida.

\textbf{PARÂMETRO} \begin{itemize}
\item \textbf{Directory} {-} Nome da pasta
\end{itemize} \textbf{RETORNA} \begin{itemize}
\item \textbf{TRUE} {-} Se a pasta existe
\item \textbf{FALSE} {-} Se a pasta não existe;
\end{itemize}

\end{list}
\paragraph*{FPrimeiroHandleLivre}\hspace*{\fill}

\begin{list}{}{
\settowidth{\tmplength}{\textbf{Declaração}}
\setlength{\itemindent}{0cm}
\setlength{\listparindent}{0cm}
\setlength{\leftmargin}{\evensidemargin}
\addtolength{\leftmargin}{\tmplength}
\settowidth{\labelsep}{X}
\addtolength{\leftmargin}{\labelsep}
\setlength{\labelwidth}{\tmplength}
}
\begin{flushleft}
\item[\textbf{Declaração}\hfill]
\begin{ttfamily}
public class Function FPrimeiroHandleLivre: SmallInt;\end{ttfamily}


\end{flushleft}
\par
\item[\textbf{Descrição}]
Retorna o numero de arquivos abertos no sistema operacional

\begin{itemize}
\item \textbf{NOTA} \begin{itemize}
\item TaStaus : Retorna o número do error se ouver
\end{itemize}
\end{itemize}

\end{list}
\paragraph*{FlockFile}\hspace*{\fill}

\begin{list}{}{
\settowidth{\tmplength}{\textbf{Declaração}}
\setlength{\itemindent}{0cm}
\setlength{\listparindent}{0cm}
\setlength{\leftmargin}{\evensidemargin}
\addtolength{\leftmargin}{\tmplength}
\settowidth{\labelsep}{X}
\addtolength{\leftmargin}{\labelsep}
\setlength{\labelwidth}{\tmplength}
}
\begin{flushleft}
\item[\textbf{Declaração}\hfill]
\begin{ttfamily}
public class function FlockFile(Handle : Thandle; modo : LongInt): LongInt ; overload;\end{ttfamily}


\end{flushleft}
\par
\item[\textbf{Descrição}]
A class function \textbf{\begin{ttfamily}FlockFile\end{ttfamily}} define ou remove um bloqueio no arquivo passado por Handle.

\begin{itemize}
\item \textbf{PARÂMETROS}

\begin{itemize}
\item \textbf{MODE} \begin{itemize}
\item O modo pode ser uma das seguintes constantes: \begin{itemize}
\item LOCK{\_}SH : define um bloqueio compartilhado.
\item LOCK{\_}EX : define um bloqueio exclusivo.
\item LOCK{\_}UN : desbloqueia o arquivo.
\item LOCK{\_}NB : Isso pode ser OR junto com o outro. Se isso for feito, o aplicativo não bloqueia ao bloquear.
\end{itemize}
\end{itemize}
\end{itemize}
\item \textbf{RETORNO} \begin{itemize}
\item \begin{ttfamily}LONGINT\end{ttfamily}(\ref{mi.rtl.Types.TTypes-LongInt}) : A função retorna zero se for bem{-}sucedida, um valor de retorno diferente de zero indica um erro.
\end{itemize}
\item \textbf{REFERÊNCIA} \begin{itemize}
\item https://www.freepascal.org/docs-html/rtl/unix/fpflock.html
\end{itemize}
\end{itemize}

\end{list}
\paragraph*{DiskFree}\hspace*{\fill}

\begin{list}{}{
\settowidth{\tmplength}{\textbf{Declaração}}
\setlength{\itemindent}{0cm}
\setlength{\listparindent}{0cm}
\setlength{\leftmargin}{\evensidemargin}
\addtolength{\leftmargin}{\tmplength}
\settowidth{\labelsep}{X}
\addtolength{\leftmargin}{\labelsep}
\setlength{\labelwidth}{\tmplength}
}
\begin{flushleft}
\item[\textbf{Declaração}\hfill]
\begin{ttfamily}
public class function DiskFree(Partition:byte; out VrDiskFree :Int64):integer;\end{ttfamily}


\end{flushleft}
\par
\item[\textbf{Descrição}]
A class function \textbf{\begin{ttfamily}DiskFree\end{ttfamily}} retorna o espaço livre (EM BYTES) da partição passada por Partition

\begin{itemize}
\item PARÂMETRO: \begin{itemize}
\item 0 para a partição atual.
\item 1 para a primeira unidade de disquete.
\item 2 Para a segunda unidade de disquete.
\item 3 Para a primeira partição do disco rígido.
\item 4{-}26 Para todas as outras unidades e partições.
\end{itemize}
\item REFERÊNCIA: \begin{itemize}
\item https://www.freepascal.org/docs-html/rtl/sysutils/diskfree.html
\end{itemize}
\item OBSERVAÇÃO: \begin{itemize}
\item No Linux, e no Unix em geral, o conceito de disco é diferente do dos um, uma vez que o sistema de arquivos é visto como uma grande árvore de diretórios. Por esta razão, os \begin{ttfamily}DiskFree\end{ttfamily} e DiskSize funções devem ser mimetizado utilizando nomes de arquivos que residem nas partições. Para obter mais informações, consulte AddDisk.
\end{itemize}
\item EXEMPLO:

\texttt{\\\nopagebreak[3]
\\\nopagebreak[3]
}\textbf{procedure}\texttt{~testDiskFree;\\\nopagebreak[3]
~~~}\textbf{var}\texttt{\\\nopagebreak[3]
~~~~~VrDiskFree~:~int64;\\\nopagebreak[3]
}\textbf{begin}\texttt{\\\nopagebreak[3]
~~WriteLn('TestDiskFree;');\\\nopagebreak[3]
\\\nopagebreak[3]
~~VrDiskFree~:=~Diskfree(0);\\\nopagebreak[3]
~~Writeln~('Free~space~of~current~disk:~',VrDiskFree,~'~Bytes');\\\nopagebreak[3]
\\\nopagebreak[3]
~~VrDiskFree~:=~(VrDiskFree~}\textbf{div}\texttt{~1024);\\\nopagebreak[3]
~~Writeln~('Free~space~of~current~disk:~',VrDiskFree,'~KB)');\\\nopagebreak[3]
\\\nopagebreak[3]
~~VrDiskFree~:=~(VrDiskFree~}\textbf{div}\texttt{~1024);\\\nopagebreak[3]
~~Writeln~('Free~space~of~current~disk:~',VrDiskFree,'~MB)');\\\nopagebreak[3]
\\\nopagebreak[3]
~~VrDiskFree~:=~(VrDiskFree~}\textbf{div}\texttt{~1024);\\\nopagebreak[3]
~~Writeln~('Free~space~of~current~disk:~',VrDiskFree,'~GB)');\\\nopagebreak[3]
\\\nopagebreak[3]
~~VrDiskFree~:=~(VrDiskFree~}\textbf{div}\texttt{~1024);\\\nopagebreak[3]
~~Writeln~('Free~space~of~current~disk:~',VrDiskFree,'~TB)');\\\nopagebreak[3]
}\textbf{end}\texttt{;\\
}
\end{itemize}

\end{list}
\paragraph*{ByteDrive}\hspace*{\fill}

\begin{list}{}{
\settowidth{\tmplength}{\textbf{Declaração}}
\setlength{\itemindent}{0cm}
\setlength{\listparindent}{0cm}
\setlength{\leftmargin}{\evensidemargin}
\addtolength{\leftmargin}{\tmplength}
\settowidth{\labelsep}{X}
\addtolength{\leftmargin}{\labelsep}
\setlength{\labelwidth}{\tmplength}
}
\begin{flushleft}
\item[\textbf{Declaração}\hfill]
\begin{ttfamily}
public class Function ByteDrive(Const NomeArquivo:AnsiString) : Byte;\end{ttfamily}


\end{flushleft}
\end{list}
\chapter{Unit mi.rtl.Objects.Consts}
\section{Descrição}
\begin{itemize}
\item A Unit \textbf{\begin{ttfamily}mi.rtl.Objects.Consts\end{ttfamily}} reune todos as contantes da unit \textbf{TObjects} globais usados pela class TObjects e suas descendências do pacote \textbf{\begin{ttfamily}mi.rtl\end{ttfamily}(\ref{mi.rtl})}.

\begin{itemize}
\item \textbf{NOTAS} \begin{itemize}
\item Esta unit foi testada nas plataformas: win32, win64 e linux.
\end{itemize}
\item \textbf{VERSÃO} \begin{itemize}
\item Alpha {-} 0.5.0.687
\end{itemize}
\item \textbf{HISTÓRICO} \begin{itemize}
\item Criado por: Paulo Sérgio da Silva Pacheco e{-}mail: paulosspacheco@yahoo.com.br \begin{itemize}
\item \textbf{18/11/2021} 10:56 a ?? {-} Criar a unit mi.rtl.objects.Consts.pas
\item \textbf{19/11/2021} 20:35 a 21:22 {-} Conclusão da classe \textbf{\begin{ttfamily}TObjectsConsts\end{ttfamily}(\ref{mi.rtl.Objects.Consts.TObjectsConsts})}
\item \textbf{13/12/2021} 21:00 a 22:10 {-} Documentar unidade.
\end{itemize}
\end{itemize}
\item \textbf{CÓDIGO FONTE}: \begin{itemize}
\item 
\end{itemize}
\end{itemize}
\end{itemize}
\section{Uses}
\begin{itemize}
\item \begin{ttfamily}Classes\end{ttfamily}\item \begin{ttfamily}SysUtils\end{ttfamily}\item \begin{ttfamily}mi.rtl.objects.types\end{ttfamily}(\ref{mi.rtl.objects.types})\item \begin{ttfamily}mi.rtl.consts.StringListBase\end{ttfamily}(\ref{mi.rtl.Consts.StringListBase})\end{itemize}
\section{Visão Geral}
\begin{description}
\item[\texttt{\begin{ttfamily}TObjectsConsts\end{ttfamily} Classe}]
\end{description}
\section{Classes, Interfaces, Objetos e Registros}
\subsection*{TObjectsConsts Classe}
\subsubsection*{\large{\textbf{Hierarquia}}\normalsize\hspace{1ex}\hfill}
TObjectsConsts {$>$} TObjectsTypes
\subsubsection*{\large{\textbf{Descrição}}\normalsize\hspace{1ex}\hfill}
\begin{itemize}
\item A class \textbf{\begin{ttfamily}TObjectsConsts\end{ttfamily}} usada para separar as contantes da unit \textbf{TObjects} do pacote \textbf{\begin{ttfamily}mi.rtl\end{ttfamily}(\ref{mi.rtl})}.
\end{itemize}\subsubsection*{\large{\textbf{Campos}}\normalsize\hspace{1ex}\hfill}
\paragraph*{OkZeraFGetMem}\hspace*{\fill}

\begin{list}{}{
\settowidth{\tmplength}{\textbf{Declaração}}
\setlength{\itemindent}{0cm}
\setlength{\listparindent}{0cm}
\setlength{\leftmargin}{\evensidemargin}
\addtolength{\leftmargin}{\tmplength}
\settowidth{\labelsep}{X}
\addtolength{\leftmargin}{\labelsep}
\setlength{\labelwidth}{\tmplength}
}
\begin{flushleft}
\item[\textbf{Declaração}\hfill]
\begin{ttfamily}
public const OkZeraFGetMem    : Boolean = True;\end{ttfamily}


\end{flushleft}
\par
\item[\textbf{Descrição}]
True zera a memória alocada por FGetMem

\end{list}
\paragraph*{ErrorInfo}\hspace*{\fill}

\begin{list}{}{
\settowidth{\tmplength}{\textbf{Declaração}}
\setlength{\itemindent}{0cm}
\setlength{\listparindent}{0cm}
\setlength{\leftmargin}{\evensidemargin}
\addtolength{\leftmargin}{\tmplength}
\settowidth{\labelsep}{X}
\addtolength{\leftmargin}{\labelsep}
\setlength{\labelwidth}{\tmplength}
}
\begin{flushleft}
\item[\textbf{Declaração}\hfill]
\begin{ttfamily}
public const ErrorInfo : Integer = 0;\end{ttfamily}


\end{flushleft}
\par
\item[\textbf{Descrição}]
Stream error info

\end{list}
\paragraph*{stOk}\hspace*{\fill}

\begin{list}{}{
\settowidth{\tmplength}{\textbf{Declaração}}
\setlength{\itemindent}{0cm}
\setlength{\listparindent}{0cm}
\setlength{\leftmargin}{\evensidemargin}
\addtolength{\leftmargin}{\tmplength}
\settowidth{\labelsep}{X}
\addtolength{\leftmargin}{\labelsep}
\setlength{\labelwidth}{\tmplength}
}
\begin{flushleft}
\item[\textbf{Declaração}\hfill]
\begin{ttfamily}
public const stOk         =  0;\end{ttfamily}


\end{flushleft}
\par
\item[\textbf{Descrição}]
\begin{itemize}
\item No stream error
\end{itemize}

\end{list}
\paragraph*{stError}\hspace*{\fill}

\begin{list}{}{
\settowidth{\tmplength}{\textbf{Declaração}}
\setlength{\itemindent}{0cm}
\setlength{\listparindent}{0cm}
\setlength{\leftmargin}{\evensidemargin}
\addtolength{\leftmargin}{\tmplength}
\settowidth{\labelsep}{X}
\addtolength{\leftmargin}{\labelsep}
\setlength{\labelwidth}{\tmplength}
}
\begin{flushleft}
\item[\textbf{Declaração}\hfill]
\begin{ttfamily}
public const stError      = -1;\end{ttfamily}


\end{flushleft}
\par
\item[\textbf{Descrição}]
\begin{itemize}
\item Access error
\end{itemize}

\end{list}
\paragraph*{stCreateError}\hspace*{\fill}

\begin{list}{}{
\settowidth{\tmplength}{\textbf{Declaração}}
\setlength{\itemindent}{0cm}
\setlength{\listparindent}{0cm}
\setlength{\leftmargin}{\evensidemargin}
\addtolength{\leftmargin}{\tmplength}
\settowidth{\labelsep}{X}
\addtolength{\leftmargin}{\labelsep}
\setlength{\labelwidth}{\tmplength}
}
\begin{flushleft}
\item[\textbf{Declaração}\hfill]
\begin{ttfamily}
public const stCreateError= -2;\end{ttfamily}


\end{flushleft}
\par
\item[\textbf{Descrição}]
\begin{itemize}
\item Initialize error
\end{itemize}

\end{list}
\paragraph*{stReadError}\hspace*{\fill}

\begin{list}{}{
\settowidth{\tmplength}{\textbf{Declaração}}
\setlength{\itemindent}{0cm}
\setlength{\listparindent}{0cm}
\setlength{\leftmargin}{\evensidemargin}
\addtolength{\leftmargin}{\tmplength}
\settowidth{\labelsep}{X}
\addtolength{\leftmargin}{\labelsep}
\setlength{\labelwidth}{\tmplength}
}
\begin{flushleft}
\item[\textbf{Declaração}\hfill]
\begin{ttfamily}
public const stReadError  = -3;\end{ttfamily}


\end{flushleft}
\par
\item[\textbf{Descrição}]
\begin{itemize}
\item Stream read error
\end{itemize}

\end{list}
\paragraph*{stWriteError}\hspace*{\fill}

\begin{list}{}{
\settowidth{\tmplength}{\textbf{Declaração}}
\setlength{\itemindent}{0cm}
\setlength{\listparindent}{0cm}
\setlength{\leftmargin}{\evensidemargin}
\addtolength{\leftmargin}{\tmplength}
\settowidth{\labelsep}{X}
\addtolength{\leftmargin}{\labelsep}
\setlength{\labelwidth}{\tmplength}
}
\begin{flushleft}
\item[\textbf{Declaração}\hfill]
\begin{ttfamily}
public const stWriteError = -4;\end{ttfamily}


\end{flushleft}
\par
\item[\textbf{Descrição}]
\begin{itemize}
\item Stream write error
\end{itemize}

\end{list}
\paragraph*{stGetError}\hspace*{\fill}

\begin{list}{}{
\settowidth{\tmplength}{\textbf{Declaração}}
\setlength{\itemindent}{0cm}
\setlength{\listparindent}{0cm}
\setlength{\leftmargin}{\evensidemargin}
\addtolength{\leftmargin}{\tmplength}
\settowidth{\labelsep}{X}
\addtolength{\leftmargin}{\labelsep}
\setlength{\labelwidth}{\tmplength}
}
\begin{flushleft}
\item[\textbf{Declaração}\hfill]
\begin{ttfamily}
public const stGetError   = -5;\end{ttfamily}


\end{flushleft}
\par
\item[\textbf{Descrição}]
\begin{itemize}
\item Get Class error
\end{itemize}

\end{list}
\paragraph*{stPutError}\hspace*{\fill}

\begin{list}{}{
\settowidth{\tmplength}{\textbf{Declaração}}
\setlength{\itemindent}{0cm}
\setlength{\listparindent}{0cm}
\setlength{\leftmargin}{\evensidemargin}
\addtolength{\leftmargin}{\tmplength}
\settowidth{\labelsep}{X}
\addtolength{\leftmargin}{\labelsep}
\setlength{\labelwidth}{\tmplength}
}
\begin{flushleft}
\item[\textbf{Declaração}\hfill]
\begin{ttfamily}
public const stPutError   = -6;\end{ttfamily}


\end{flushleft}
\par
\item[\textbf{Descrição}]
\begin{itemize}
\item Put Class error
\end{itemize}

\end{list}
\paragraph*{stSeekError}\hspace*{\fill}

\begin{list}{}{
\settowidth{\tmplength}{\textbf{Declaração}}
\setlength{\itemindent}{0cm}
\setlength{\listparindent}{0cm}
\setlength{\leftmargin}{\evensidemargin}
\addtolength{\leftmargin}{\tmplength}
\settowidth{\labelsep}{X}
\addtolength{\leftmargin}{\labelsep}
\setlength{\labelwidth}{\tmplength}
}
\begin{flushleft}
\item[\textbf{Declaração}\hfill]
\begin{ttfamily}
public const stSeekError  = -7;\end{ttfamily}


\end{flushleft}
\par
\item[\textbf{Descrição}]
\begin{itemize}
\item Seek error in stream
\end{itemize}

\end{list}
\paragraph*{stOpenError}\hspace*{\fill}

\begin{list}{}{
\settowidth{\tmplength}{\textbf{Declaração}}
\setlength{\itemindent}{0cm}
\setlength{\listparindent}{0cm}
\setlength{\leftmargin}{\evensidemargin}
\addtolength{\leftmargin}{\tmplength}
\settowidth{\labelsep}{X}
\addtolength{\leftmargin}{\labelsep}
\setlength{\labelwidth}{\tmplength}
}
\begin{flushleft}
\item[\textbf{Declaração}\hfill]
\begin{ttfamily}
public const stOpenError  = -8;\end{ttfamily}


\end{flushleft}
\par
\item[\textbf{Descrição}]
\begin{itemize}
\item Error opening stream
\end{itemize}

\end{list}
\paragraph*{StShareError}\hspace*{\fill}

\begin{list}{}{
\settowidth{\tmplength}{\textbf{Declaração}}
\setlength{\itemindent}{0cm}
\setlength{\listparindent}{0cm}
\setlength{\leftmargin}{\evensidemargin}
\addtolength{\leftmargin}{\tmplength}
\settowidth{\labelsep}{X}
\addtolength{\leftmargin}{\labelsep}
\setlength{\labelwidth}{\tmplength}
}
\begin{flushleft}
\item[\textbf{Declaração}\hfill]
\begin{ttfamily}
public const StShareError = -9;\end{ttfamily}


\end{flushleft}
\par
\item[\textbf{Descrição}]
\begin{itemize}
\item Erro de compartilhamento
\end{itemize}

\end{list}
\paragraph*{FmReadOnly}\hspace*{\fill}

\begin{list}{}{
\settowidth{\tmplength}{\textbf{Declaração}}
\setlength{\itemindent}{0cm}
\setlength{\listparindent}{0cm}
\setlength{\leftmargin}{\evensidemargin}
\addtolength{\leftmargin}{\tmplength}
\settowidth{\labelsep}{X}
\addtolength{\leftmargin}{\labelsep}
\setlength{\labelwidth}{\tmplength}
}
\begin{flushleft}
\item[\textbf{Declaração}\hfill]
\begin{ttfamily}
public const FmReadOnly       = fmOpenRead;\end{ttfamily}


\end{flushleft}
\par
\item[\textbf{Descrição}]
000

\end{list}
\paragraph*{FmWriteOnly}\hspace*{\fill}

\begin{list}{}{
\settowidth{\tmplength}{\textbf{Declaração}}
\setlength{\itemindent}{0cm}
\setlength{\listparindent}{0cm}
\setlength{\leftmargin}{\evensidemargin}
\addtolength{\leftmargin}{\tmplength}
\settowidth{\labelsep}{X}
\addtolength{\leftmargin}{\labelsep}
\setlength{\labelwidth}{\tmplength}
}
\begin{flushleft}
\item[\textbf{Declaração}\hfill]
\begin{ttfamily}
public const FmWriteOnly      = fmOpenWrite;\end{ttfamily}


\end{flushleft}
\par
\item[\textbf{Descrição}]
001

\end{list}
\paragraph*{FmReadWrite}\hspace*{\fill}

\begin{list}{}{
\settowidth{\tmplength}{\textbf{Declaração}}
\setlength{\itemindent}{0cm}
\setlength{\listparindent}{0cm}
\setlength{\leftmargin}{\evensidemargin}
\addtolength{\leftmargin}{\tmplength}
\settowidth{\labelsep}{X}
\addtolength{\leftmargin}{\labelsep}
\setlength{\labelwidth}{\tmplength}
}
\begin{flushleft}
\item[\textbf{Declaração}\hfill]
\begin{ttfamily}
public const FmReadWrite      = fmOpenReadWrite;\end{ttfamily}


\end{flushleft}
\par
\item[\textbf{Descrição}]
010

\end{list}
\paragraph*{FmDenyALL}\hspace*{\fill}

\begin{list}{}{
\settowidth{\tmplength}{\textbf{Declaração}}
\setlength{\itemindent}{0cm}
\setlength{\listparindent}{0cm}
\setlength{\leftmargin}{\evensidemargin}
\addtolength{\leftmargin}{\tmplength}
\settowidth{\labelsep}{X}
\addtolength{\leftmargin}{\labelsep}
\setlength{\labelwidth}{\tmplength}
}
\begin{flushleft}
\item[\textbf{Declaração}\hfill]
\begin{ttfamily}
public const FmDenyALL        = fmShareCompat or fmShareExclusive;\end{ttfamily}


\end{flushleft}
\par
\item[\textbf{Descrição}]
0010000

\end{list}
\paragraph*{FmDenyWrite}\hspace*{\fill}

\begin{list}{}{
\settowidth{\tmplength}{\textbf{Declaração}}
\setlength{\itemindent}{0cm}
\setlength{\listparindent}{0cm}
\setlength{\leftmargin}{\evensidemargin}
\addtolength{\leftmargin}{\tmplength}
\settowidth{\labelsep}{X}
\addtolength{\leftmargin}{\labelsep}
\setlength{\labelwidth}{\tmplength}
}
\begin{flushleft}
\item[\textbf{Declaração}\hfill]
\begin{ttfamily}
public const FmDenyWrite      = fmShareCompat or fmShareDenyWrite;\end{ttfamily}


\end{flushleft}
\par
\item[\textbf{Descrição}]
0100000

\end{list}
\paragraph*{FmDenyRead}\hspace*{\fill}

\begin{list}{}{
\settowidth{\tmplength}{\textbf{Declaração}}
\setlength{\itemindent}{0cm}
\setlength{\listparindent}{0cm}
\setlength{\leftmargin}{\evensidemargin}
\addtolength{\leftmargin}{\tmplength}
\settowidth{\labelsep}{X}
\addtolength{\leftmargin}{\labelsep}
\setlength{\labelwidth}{\tmplength}
}
\begin{flushleft}
\item[\textbf{Declaração}\hfill]
\begin{ttfamily}
public const FmDenyRead       = fmShareCompat or fmShareDenyRead;\end{ttfamily}


\end{flushleft}
\par
\item[\textbf{Descrição}]
0110000

\end{list}
\paragraph*{FmDenyNone}\hspace*{\fill}

\begin{list}{}{
\settowidth{\tmplength}{\textbf{Declaração}}
\setlength{\itemindent}{0cm}
\setlength{\listparindent}{0cm}
\setlength{\leftmargin}{\evensidemargin}
\addtolength{\leftmargin}{\tmplength}
\settowidth{\labelsep}{X}
\addtolength{\leftmargin}{\labelsep}
\setlength{\labelwidth}{\tmplength}
}
\begin{flushleft}
\item[\textbf{Declaração}\hfill]
\begin{ttfamily}
public const FmDenyNone       = fmShareCompat or fmShareDenyNone;\end{ttfamily}


\end{flushleft}
\par
\item[\textbf{Descrição}]
1000000

\end{list}
\paragraph*{FmChildProcesses}\hspace*{\fill}

\begin{list}{}{
\settowidth{\tmplength}{\textbf{Declaração}}
\setlength{\itemindent}{0cm}
\setlength{\listparindent}{0cm}
\setlength{\leftmargin}{\evensidemargin}
\addtolength{\leftmargin}{\tmplength}
\settowidth{\labelsep}{X}
\addtolength{\leftmargin}{\labelsep}
\setlength{\labelwidth}{\tmplength}
}
\begin{flushleft}
\item[\textbf{Declaração}\hfill]
\begin{ttfamily}
public const FmChildProcesses = {\$}0080;\end{ttfamily}


\end{flushleft}
\par
\item[\textbf{Descrição}]
10000000

\end{list}
\paragraph*{FmCreate}\hspace*{\fill}

\begin{list}{}{
\settowidth{\tmplength}{\textbf{Declaração}}
\setlength{\itemindent}{0cm}
\setlength{\listparindent}{0cm}
\setlength{\leftmargin}{\evensidemargin}
\addtolength{\leftmargin}{\tmplength}
\settowidth{\labelsep}{X}
\addtolength{\leftmargin}{\labelsep}
\setlength{\labelwidth}{\tmplength}
}
\begin{flushleft}
\item[\textbf{Declaração}\hfill]
\begin{ttfamily}
public const FmCreate         = {\$}0100;\end{ttfamily}


\end{flushleft}
\par
\item[\textbf{Descrição}]
100000000

\end{list}
\paragraph*{FmWait}\hspace*{\fill}

\begin{list}{}{
\settowidth{\tmplength}{\textbf{Declaração}}
\setlength{\itemindent}{0cm}
\setlength{\listparindent}{0cm}
\setlength{\leftmargin}{\evensidemargin}
\addtolength{\leftmargin}{\tmplength}
\settowidth{\labelsep}{X}
\addtolength{\leftmargin}{\labelsep}
\setlength{\labelwidth}{\tmplength}
}
\begin{flushleft}
\item[\textbf{Declaração}\hfill]
\begin{ttfamily}
public const FmWait           = {\$}0200 ;\end{ttfamily}


\end{flushleft}
\par
\item[\textbf{Descrição}]
1000000000

\end{list}
\paragraph*{FmMemory}\hspace*{\fill}

\begin{list}{}{
\settowidth{\tmplength}{\textbf{Declaração}}
\setlength{\itemindent}{0cm}
\setlength{\listparindent}{0cm}
\setlength{\leftmargin}{\evensidemargin}
\addtolength{\leftmargin}{\tmplength}
\settowidth{\labelsep}{X}
\addtolength{\leftmargin}{\labelsep}
\setlength{\labelwidth}{\tmplength}
}
\begin{flushleft}
\item[\textbf{Declaração}\hfill]
\begin{ttfamily}
public const FmMemory         = {\$}0400 ;\end{ttfamily}


\end{flushleft}
\par
\item[\textbf{Descrição}]
10000000000 {-} Indica que o arquivo esta em tStreamemoria

\end{list}
\paragraph*{FmMemory{\_}Temp}\hspace*{\fill}

\begin{list}{}{
\settowidth{\tmplength}{\textbf{Declaração}}
\setlength{\itemindent}{0cm}
\setlength{\listparindent}{0cm}
\setlength{\leftmargin}{\evensidemargin}
\addtolength{\leftmargin}{\tmplength}
\settowidth{\labelsep}{X}
\addtolength{\leftmargin}{\labelsep}
\setlength{\labelwidth}{\tmplength}
}
\begin{flushleft}
\item[\textbf{Declaração}\hfill]
\begin{ttfamily}
public const FmMemory{\_}Temp    = {\$}0800 ;\end{ttfamily}


\end{flushleft}
\par
\item[\textbf{Descrição}]
100000000000 {-} Indica que o arquivo e temporario e esta em tStreamemoria

\end{list}
\paragraph*{stOpenRead}\hspace*{\fill}

\begin{list}{}{
\settowidth{\tmplength}{\textbf{Declaração}}
\setlength{\itemindent}{0cm}
\setlength{\listparindent}{0cm}
\setlength{\leftmargin}{\evensidemargin}
\addtolength{\leftmargin}{\tmplength}
\settowidth{\labelsep}{X}
\addtolength{\leftmargin}{\labelsep}
\setlength{\labelwidth}{\tmplength}
}
\begin{flushleft}
\item[\textbf{Declaração}\hfill]
\begin{ttfamily}
public const stOpenRead  = FmReadOnly ;\end{ttfamily}


\end{flushleft}
\par
\item[\textbf{Descrição}]
000 {-} Read access only

\end{list}
\paragraph*{stOpenWrite}\hspace*{\fill}

\begin{list}{}{
\settowidth{\tmplength}{\textbf{Declaração}}
\setlength{\itemindent}{0cm}
\setlength{\listparindent}{0cm}
\setlength{\leftmargin}{\evensidemargin}
\addtolength{\leftmargin}{\tmplength}
\settowidth{\labelsep}{X}
\addtolength{\leftmargin}{\labelsep}
\setlength{\labelwidth}{\tmplength}
}
\begin{flushleft}
\item[\textbf{Declaração}\hfill]
\begin{ttfamily}
public const stOpenWrite = FmWriteOnly;\end{ttfamily}


\end{flushleft}
\par
\item[\textbf{Descrição}]
001 {-} Write access only

\end{list}
\paragraph*{stOpen}\hspace*{\fill}

\begin{list}{}{
\settowidth{\tmplength}{\textbf{Declaração}}
\setlength{\itemindent}{0cm}
\setlength{\listparindent}{0cm}
\setlength{\leftmargin}{\evensidemargin}
\addtolength{\leftmargin}{\tmplength}
\settowidth{\labelsep}{X}
\addtolength{\leftmargin}{\labelsep}
\setlength{\labelwidth}{\tmplength}
}
\begin{flushleft}
\item[\textbf{Declaração}\hfill]
\begin{ttfamily}
public const stOpen      = fmOpenReadWrite;\end{ttfamily}


\end{flushleft}
\par
\item[\textbf{Descrição}]
010 {-} Read/write access

\end{list}
\paragraph*{StCreate}\hspace*{\fill}

\begin{list}{}{
\settowidth{\tmplength}{\textbf{Declaração}}
\setlength{\itemindent}{0cm}
\setlength{\listparindent}{0cm}
\setlength{\leftmargin}{\evensidemargin}
\addtolength{\leftmargin}{\tmplength}
\settowidth{\labelsep}{X}
\addtolength{\leftmargin}{\labelsep}
\setlength{\labelwidth}{\tmplength}
}
\begin{flushleft}
\item[\textbf{Declaração}\hfill]
\begin{ttfamily}
public const StCreate    = FmCreate   ;\end{ttfamily}


\end{flushleft}
\par
\item[\textbf{Descrição}]
100000000 {-} Create new file

\end{list}
\paragraph*{IsApp{\_}TV}\hspace*{\fill}

\begin{list}{}{
\settowidth{\tmplength}{\textbf{Declaração}}
\setlength{\itemindent}{0cm}
\setlength{\listparindent}{0cm}
\setlength{\leftmargin}{\evensidemargin}
\addtolength{\leftmargin}{\tmplength}
\settowidth{\labelsep}{X}
\addtolength{\leftmargin}{\labelsep}
\setlength{\labelwidth}{\tmplength}
}
\begin{flushleft}
\item[\textbf{Declaração}\hfill]
\begin{ttfamily}
public const IsApp{\_}TV  : Boolean = False;\end{ttfamily}


\end{flushleft}
\par
\item[\textbf{Descrição}]
\begin{itemize}
\item A constante \textbf{\begin{ttfamily}IsApp{\_}TV\end{ttfamily}} indica se a aplicação gráfica é turbo vision.
\end{itemize}

\end{list}
\paragraph*{IsApp{\_}Console}\hspace*{\fill}

\begin{list}{}{
\settowidth{\tmplength}{\textbf{Declaração}}
\setlength{\itemindent}{0cm}
\setlength{\listparindent}{0cm}
\setlength{\leftmargin}{\evensidemargin}
\addtolength{\leftmargin}{\tmplength}
\settowidth{\labelsep}{X}
\addtolength{\leftmargin}{\labelsep}
\setlength{\labelwidth}{\tmplength}
}
\begin{flushleft}
\item[\textbf{Declaração}\hfill]
\begin{ttfamily}
public const IsApp{\_}Console      : Boolean = false;\end{ttfamily}


\end{flushleft}
\par
\item[\textbf{Descrição}]
\begin{itemize}
\item A constante \textbf{\begin{ttfamily}IsApp{\_}Console\end{ttfamily}} indica se a aplicação CGI deve ser compilado no modo console.
\end{itemize}

\end{list}
\paragraph*{IsApp{\_}Gui}\hspace*{\fill}

\begin{list}{}{
\settowidth{\tmplength}{\textbf{Declaração}}
\setlength{\itemindent}{0cm}
\setlength{\listparindent}{0cm}
\setlength{\leftmargin}{\evensidemargin}
\addtolength{\leftmargin}{\tmplength}
\settowidth{\labelsep}{X}
\addtolength{\leftmargin}{\labelsep}
\setlength{\labelwidth}{\tmplength}
}
\begin{flushleft}
\item[\textbf{Declaração}\hfill]
\begin{ttfamily}
public const IsApp{\_}Gui          : Boolean = false;\end{ttfamily}


\end{flushleft}
\par
\item[\textbf{Descrição}]
\begin{itemize}
\item A constante \textbf{\begin{ttfamily}IsApp{\_}Gui\end{ttfamily}} indica se a aplicação é gráfica independente de usar vcl ou não.
\end{itemize}

\end{list}
\paragraph*{IsApp{\_}ISAPI}\hspace*{\fill}

\begin{list}{}{
\settowidth{\tmplength}{\textbf{Declaração}}
\setlength{\itemindent}{0cm}
\setlength{\listparindent}{0cm}
\setlength{\leftmargin}{\evensidemargin}
\addtolength{\leftmargin}{\tmplength}
\settowidth{\labelsep}{X}
\addtolength{\leftmargin}{\labelsep}
\setlength{\labelwidth}{\tmplength}
}
\begin{flushleft}
\item[\textbf{Declaração}\hfill]
\begin{ttfamily}
public const IsApp{\_}ISAPI  : Boolean = False;\end{ttfamily}


\end{flushleft}
\par
\item[\textbf{Descrição}]
\begin{itemize}
\item A constante \textbf{\begin{ttfamily}IsApp{\_}ISAPI\end{ttfamily}} indica se a aplicação web é compilada como dll deve ser executada em conjunto com browser.
\end{itemize}

\end{list}
\paragraph*{IsApp{\_}App{\_}WS{\_}Soap}\hspace*{\fill}

\begin{list}{}{
\settowidth{\tmplength}{\textbf{Declaração}}
\setlength{\itemindent}{0cm}
\setlength{\listparindent}{0cm}
\setlength{\leftmargin}{\evensidemargin}
\addtolength{\leftmargin}{\tmplength}
\settowidth{\labelsep}{X}
\addtolength{\leftmargin}{\labelsep}
\setlength{\labelwidth}{\tmplength}
}
\begin{flushleft}
\item[\textbf{Declaração}\hfill]
\begin{ttfamily}
public const IsApp{\_}App{\_}WS{\_}Soap  : Boolean = False;\end{ttfamily}


\end{flushleft}
\par
\item[\textbf{Descrição}]
\begin{itemize}
\item A constante \textbf{\begin{ttfamily}IsApp{\_}App{\_}WS{\_}Soap\end{ttfamily}} indica se a aplicação web é publicado com serviço XML com Protocolo Soap
\end{itemize}

\end{list}
\paragraph*{IsApp{\_}VCL}\hspace*{\fill}

\begin{list}{}{
\settowidth{\tmplength}{\textbf{Declaração}}
\setlength{\itemindent}{0cm}
\setlength{\listparindent}{0cm}
\setlength{\leftmargin}{\evensidemargin}
\addtolength{\leftmargin}{\tmplength}
\settowidth{\labelsep}{X}
\addtolength{\leftmargin}{\labelsep}
\setlength{\labelwidth}{\tmplength}
}
\begin{flushleft}
\item[\textbf{Declaração}\hfill]
\begin{ttfamily}
public const IsApp{\_}VCL          : Boolean = false;\end{ttfamily}


\end{flushleft}
\par
\item[\textbf{Descrição}]
\begin{itemize}
\item A constante \textbf{\begin{ttfamily}IsApp{\_}VCL\end{ttfamily}} indica se a aplicação VCL pode ser mista console e gráfica.
\end{itemize}

\end{list}
\paragraph*{IsApp{\_}VCL{\_}IE}\hspace*{\fill}

\begin{list}{}{
\settowidth{\tmplength}{\textbf{Declaração}}
\setlength{\itemindent}{0cm}
\setlength{\listparindent}{0cm}
\setlength{\leftmargin}{\evensidemargin}
\addtolength{\leftmargin}{\tmplength}
\settowidth{\labelsep}{X}
\addtolength{\leftmargin}{\labelsep}
\setlength{\labelwidth}{\tmplength}
}
\begin{flushleft}
\item[\textbf{Declaração}\hfill]
\begin{ttfamily}
public const IsApp{\_}VCL{\_}IE  : Boolean = False;\end{ttfamily}


\end{flushleft}
\par
\item[\textbf{Descrição}]
\begin{itemize}
\item A constante \textbf{\begin{ttfamily}IsApp{\_}VCL{\_}IE\end{ttfamily}} indica se a aplicação gráfica usa os componentes VCL e webBrowser como entrada de dados.
\end{itemize}

\end{list}
\paragraph*{IsApp{\_}Cgi}\hspace*{\fill}

\begin{list}{}{
\settowidth{\tmplength}{\textbf{Declaração}}
\setlength{\itemindent}{0cm}
\setlength{\listparindent}{0cm}
\setlength{\leftmargin}{\evensidemargin}
\addtolength{\leftmargin}{\tmplength}
\settowidth{\labelsep}{X}
\addtolength{\leftmargin}{\labelsep}
\setlength{\labelwidth}{\tmplength}
}
\begin{flushleft}
\item[\textbf{Declaração}\hfill]
\begin{ttfamily}
public const IsApp{\_}Cgi          : Boolean = false;\end{ttfamily}


\end{flushleft}
\par
\item[\textbf{Descrição}]
\begin{itemize}
\item A constante \textbf{\begin{ttfamily}IsApp{\_}Cgi\end{ttfamily}} indica se a Aplicação é CGI. \begin{itemize}
\item \textbf{NOTA} {-}Ignora todo acesso do teclado e video do console usada como aplicações web, console ou GUI quando usado como pacote em tempo de designer.
\end{itemize}
\end{itemize}

\end{list}
\paragraph*{IsApp{\_}DSServerModule}\hspace*{\fill}

\begin{list}{}{
\settowidth{\tmplength}{\textbf{Declaração}}
\setlength{\itemindent}{0cm}
\setlength{\listparindent}{0cm}
\setlength{\leftmargin}{\evensidemargin}
\addtolength{\leftmargin}{\tmplength}
\settowidth{\labelsep}{X}
\addtolength{\leftmargin}{\labelsep}
\setlength{\labelwidth}{\tmplength}
}
\begin{flushleft}
\item[\textbf{Declaração}\hfill]
\begin{ttfamily}
public const IsApp{\_}DSServerModule : Boolean = false;\end{ttfamily}


\end{flushleft}
\par
\item[\textbf{Descrição}]
\begin{itemize}
\item A constante \textbf{\begin{ttfamily}IsApp{\_}DSServerModule\end{ttfamily}} indica se a Aplicação App{\_}DSServerModule. \begin{itemize}
\item \textbf{NOTA} \begin{itemize}
\item Ignora todo acesso ao teclado e video do console usada como aplicações servidor de dados.
\end{itemize}
\end{itemize}
\end{itemize}

\end{list}
\paragraph*{coIndexError}\hspace*{\fill}

\begin{list}{}{
\settowidth{\tmplength}{\textbf{Declaração}}
\setlength{\itemindent}{0cm}
\setlength{\listparindent}{0cm}
\setlength{\leftmargin}{\evensidemargin}
\addtolength{\leftmargin}{\tmplength}
\settowidth{\labelsep}{X}
\addtolength{\leftmargin}{\labelsep}
\setlength{\labelwidth}{\tmplength}
}
\begin{flushleft}
\item[\textbf{Declaração}\hfill]
\begin{ttfamily}
public const coIndexError = -1;\end{ttfamily}


\end{flushleft}
\par
\item[\textbf{Descrição}]
Index out of range

\end{list}
\paragraph*{coOverflow}\hspace*{\fill}

\begin{list}{}{
\settowidth{\tmplength}{\textbf{Declaração}}
\setlength{\itemindent}{0cm}
\setlength{\listparindent}{0cm}
\setlength{\leftmargin}{\evensidemargin}
\addtolength{\leftmargin}{\tmplength}
\settowidth{\labelsep}{X}
\addtolength{\leftmargin}{\labelsep}
\setlength{\labelwidth}{\tmplength}
}
\begin{flushleft}
\item[\textbf{Declaração}\hfill]
\begin{ttfamily}
public const coOverflow   = -2;\end{ttfamily}


\end{flushleft}
\par
\item[\textbf{Descrição}]
Overflow

\end{list}
\paragraph*{vmtHeaderSize}\hspace*{\fill}

\begin{list}{}{
\settowidth{\tmplength}{\textbf{Declaração}}
\setlength{\itemindent}{0cm}
\setlength{\listparindent}{0cm}
\setlength{\leftmargin}{\evensidemargin}
\addtolength{\leftmargin}{\tmplength}
\settowidth{\labelsep}{X}
\addtolength{\leftmargin}{\labelsep}
\setlength{\labelwidth}{\tmplength}
}
\begin{flushleft}
\item[\textbf{Declaração}\hfill]
\begin{ttfamily}
public const vmtHeaderSize = 8;\end{ttfamily}


\end{flushleft}
\par
\item[\textbf{Descrição}]
VMT header size

\end{list}
\paragraph*{MaxBytes}\hspace*{\fill}

\begin{list}{}{
\settowidth{\tmplength}{\textbf{Declaração}}
\setlength{\itemindent}{0cm}
\setlength{\listparindent}{0cm}
\setlength{\leftmargin}{\evensidemargin}
\addtolength{\leftmargin}{\tmplength}
\settowidth{\labelsep}{X}
\addtolength{\leftmargin}{\labelsep}
\setlength{\labelwidth}{\tmplength}
}
\begin{flushleft}
\item[\textbf{Declaração}\hfill]
\begin{ttfamily}
public const MaxBytes = MAX{\_}BYTE;\end{ttfamily}


\end{flushleft}
\par
\item[\textbf{Descrição}]
Maximum data size

\end{list}
\paragraph*{MaxWords}\hspace*{\fill}

\begin{list}{}{
\settowidth{\tmplength}{\textbf{Declaração}}
\setlength{\itemindent}{0cm}
\setlength{\listparindent}{0cm}
\setlength{\leftmargin}{\evensidemargin}
\addtolength{\leftmargin}{\tmplength}
\settowidth{\labelsep}{X}
\addtolength{\leftmargin}{\labelsep}
\setlength{\labelwidth}{\tmplength}
}
\begin{flushleft}
\item[\textbf{Declaração}\hfill]
\begin{ttfamily}
public const MaxWords = MAX{\_}WORD;\end{ttfamily}


\end{flushleft}
\par
\item[\textbf{Descrição}]
Max word data size

\end{list}
\paragraph*{MaxSmallWords}\hspace*{\fill}

\begin{list}{}{
\settowidth{\tmplength}{\textbf{Declaração}}
\setlength{\itemindent}{0cm}
\setlength{\listparindent}{0cm}
\setlength{\leftmargin}{\evensidemargin}
\addtolength{\leftmargin}{\tmplength}
\settowidth{\labelsep}{X}
\addtolength{\leftmargin}{\labelsep}
\setlength{\labelwidth}{\tmplength}
}
\begin{flushleft}
\item[\textbf{Declaração}\hfill]
\begin{ttfamily}
public const MaxSmallWords = MAX{\_}SMALL{\_}WORD;\end{ttfamily}


\end{flushleft}
\par
\item[\textbf{Descrição}]
Max word data size

\end{list}
\paragraph*{MaxPtrs}\hspace*{\fill}

\begin{list}{}{
\settowidth{\tmplength}{\textbf{Declaração}}
\setlength{\itemindent}{0cm}
\setlength{\listparindent}{0cm}
\setlength{\leftmargin}{\evensidemargin}
\addtolength{\leftmargin}{\tmplength}
\settowidth{\labelsep}{X}
\addtolength{\leftmargin}{\labelsep}
\setlength{\labelwidth}{\tmplength}
}
\begin{flushleft}
\item[\textbf{Declaração}\hfill]
\begin{ttfamily}
public const MaxPtrs = MAX{\_}ARRAY{\_}PTR;\end{ttfamily}


\end{flushleft}
\par
\item[\textbf{Descrição}]
Max ptr data size

\end{list}
\paragraph*{MaxCollectionSize}\hspace*{\fill}

\begin{list}{}{
\settowidth{\tmplength}{\textbf{Declaração}}
\setlength{\itemindent}{0cm}
\setlength{\listparindent}{0cm}
\setlength{\leftmargin}{\evensidemargin}
\addtolength{\leftmargin}{\tmplength}
\settowidth{\labelsep}{X}
\addtolength{\leftmargin}{\labelsep}
\setlength{\labelwidth}{\tmplength}
}
\begin{flushleft}
\item[\textbf{Declaração}\hfill]
\begin{ttfamily}
public const MaxCollectionSize = MaxPtrs;\end{ttfamily}


\end{flushleft}
\par
\item[\textbf{Descrição}]
Max collection size

\end{list}
\paragraph*{StreamTypes}\hspace*{\fill}

\begin{list}{}{
\settowidth{\tmplength}{\textbf{Declaração}}
\setlength{\itemindent}{0cm}
\setlength{\listparindent}{0cm}
\setlength{\leftmargin}{\evensidemargin}
\addtolength{\leftmargin}{\tmplength}
\settowidth{\labelsep}{X}
\addtolength{\leftmargin}{\labelsep}
\setlength{\labelwidth}{\tmplength}
}
\begin{flushleft}
\item[\textbf{Declaração}\hfill]
\begin{ttfamily}
public const StreamTypes: PStreamRec = Nil;\end{ttfamily}


\end{flushleft}
\par
\item[\textbf{Descrição}]
Stream types reg

\end{list}
\paragraph*{AnsiChar{\_}Control{\_}Template}\hspace*{\fill}

\begin{list}{}{
\settowidth{\tmplength}{\textbf{Declaração}}
\setlength{\itemindent}{0cm}
\setlength{\listparindent}{0cm}
\setlength{\leftmargin}{\evensidemargin}
\addtolength{\leftmargin}{\tmplength}
\settowidth{\labelsep}{X}
\addtolength{\leftmargin}{\labelsep}
\setlength{\labelwidth}{\tmplength}
}
\begin{flushleft}
\item[\textbf{Declaração}\hfill]
\begin{ttfamily}
public const AnsiChar{\_}Control{\_}Template : AnsiCharSet = [{\#}0..{\#}31,'`'];\end{ttfamily}


\end{flushleft}
\par
\item[\textbf{Descrição}]
\begin{ttfamily}AnsiChar{\_}Control{\_}Template\end{ttfamily} : AnsiCharSet = [{\#}0..{\#}31,'`',{\^{}}a..{\^{}}z,{\^{}}A..{\^{}}Z];

\end{list}
\paragraph*{Okprocessing}\hspace*{\fill}

\begin{list}{}{
\settowidth{\tmplength}{\textbf{Declaração}}
\setlength{\itemindent}{0cm}
\setlength{\listparindent}{0cm}
\setlength{\leftmargin}{\evensidemargin}
\addtolength{\leftmargin}{\tmplength}
\settowidth{\labelsep}{X}
\addtolength{\leftmargin}{\labelsep}
\setlength{\labelwidth}{\tmplength}
}
\begin{flushleft}
\item[\textbf{Declaração}\hfill]
\begin{ttfamily}
public const Okprocessing              : Boolean = false;\end{ttfamily}


\end{flushleft}
\end{list}
\paragraph*{OkprocessingControlTime}\hspace*{\fill}

\begin{list}{}{
\settowidth{\tmplength}{\textbf{Declaração}}
\setlength{\itemindent}{0cm}
\setlength{\listparindent}{0cm}
\setlength{\leftmargin}{\evensidemargin}
\addtolength{\leftmargin}{\tmplength}
\settowidth{\labelsep}{X}
\addtolength{\leftmargin}{\labelsep}
\setlength{\labelwidth}{\tmplength}
}
\begin{flushleft}
\item[\textbf{Declaração}\hfill]
\begin{ttfamily}
public const OkprocessingControlTime : Boolean = true;\end{ttfamily}


\end{flushleft}
\end{list}
\paragraph*{OkOkprocessingTime}\hspace*{\fill}

\begin{list}{}{
\settowidth{\tmplength}{\textbf{Declaração}}
\setlength{\itemindent}{0cm}
\setlength{\listparindent}{0cm}
\setlength{\leftmargin}{\evensidemargin}
\addtolength{\leftmargin}{\tmplength}
\settowidth{\labelsep}{X}
\addtolength{\leftmargin}{\labelsep}
\setlength{\labelwidth}{\tmplength}
}
\begin{flushleft}
\item[\textbf{Declaração}\hfill]
\begin{ttfamily}
public const OkOkprocessingTime        : Longint = 5 * 60;\end{ttfamily}


\end{flushleft}
\end{list}
\paragraph*{OkProcessingTime{\_}Action}\hspace*{\fill}

\begin{list}{}{
\settowidth{\tmplength}{\textbf{Declaração}}
\setlength{\itemindent}{0cm}
\setlength{\listparindent}{0cm}
\setlength{\leftmargin}{\evensidemargin}
\addtolength{\leftmargin}{\tmplength}
\settowidth{\labelsep}{X}
\addtolength{\leftmargin}{\labelsep}
\setlength{\labelwidth}{\tmplength}
}
\begin{flushleft}
\item[\textbf{Declaração}\hfill]
\begin{ttfamily}
public const OkProcessingTime{\_}Action    : TOkProcessingTime{\_}Action = OkProcessingTime{\_}Action{\_}Password;\end{ttfamily}


\end{flushleft}
\end{list}
\paragraph*{OkOkprocessingClockBegin}\hspace*{\fill}

\begin{list}{}{
\settowidth{\tmplength}{\textbf{Declaração}}
\setlength{\itemindent}{0cm}
\setlength{\listparindent}{0cm}
\setlength{\leftmargin}{\evensidemargin}
\addtolength{\leftmargin}{\tmplength}
\settowidth{\labelsep}{X}
\addtolength{\leftmargin}{\labelsep}
\setlength{\labelwidth}{\tmplength}
}
\begin{flushleft}
\item[\textbf{Declaração}\hfill]
\begin{ttfamily}
public const OkOkprocessingClockBegin  : DWord = 0;\end{ttfamily}


\end{flushleft}
\end{list}
\paragraph*{OkTempoDeTentativas}\hspace*{\fill}

\begin{list}{}{
\settowidth{\tmplength}{\textbf{Declaração}}
\setlength{\itemindent}{0cm}
\setlength{\listparindent}{0cm}
\setlength{\leftmargin}{\evensidemargin}
\addtolength{\leftmargin}{\tmplength}
\settowidth{\labelsep}{X}
\addtolength{\leftmargin}{\labelsep}
\setlength{\labelwidth}{\tmplength}
}
\begin{flushleft}
\item[\textbf{Declaração}\hfill]
\begin{ttfamily}
public const OkTempoDeTentativas : Boolean = true;\end{ttfamily}


\end{flushleft}
\par
\item[\textbf{Descrição}]
Habilita \begin{ttfamily}TempoDeTentativas\end{ttfamily}(\ref{mi.rtl.Objects.Consts.TObjectsConsts-TempoDeTentativas}) nas leitura e escritas ao arquivo

\end{list}
\paragraph*{TempoDeTentativas}\hspace*{\fill}

\begin{list}{}{
\settowidth{\tmplength}{\textbf{Declaração}}
\setlength{\itemindent}{0cm}
\setlength{\listparindent}{0cm}
\setlength{\leftmargin}{\evensidemargin}
\addtolength{\leftmargin}{\tmplength}
\settowidth{\labelsep}{X}
\addtolength{\leftmargin}{\labelsep}
\setlength{\labelwidth}{\tmplength}
}
\begin{flushleft}
\item[\textbf{Declaração}\hfill]
\begin{ttfamily}
public const TempoDeTentativas   : Longint = 10;\end{ttfamily}


\end{flushleft}
\par
\item[\textbf{Descrição}]
TimeOut = Tempo em segundos de tentativos nos processos de abertura, leitura e gravacao de arquivos

\end{list}
\subsubsection*{\large{\textbf{Métodos}}\normalsize\hspace{1ex}\hfill}
\paragraph*{SetOkprocessing}\hspace*{\fill}

\begin{list}{}{
\settowidth{\tmplength}{\textbf{Declaração}}
\setlength{\itemindent}{0cm}
\setlength{\listparindent}{0cm}
\setlength{\leftmargin}{\evensidemargin}
\addtolength{\leftmargin}{\tmplength}
\settowidth{\labelsep}{X}
\addtolength{\leftmargin}{\labelsep}
\setlength{\labelwidth}{\tmplength}
}
\begin{flushleft}
\item[\textbf{Declaração}\hfill]
\begin{ttfamily}
public Class Function SetOkprocessing(aOkprocessing : Boolean) : Boolean;\end{ttfamily}


\end{flushleft}
\end{list}
\paragraph*{FreeAndNil}\hspace*{\fill}

\begin{list}{}{
\settowidth{\tmplength}{\textbf{Declaração}}
\setlength{\itemindent}{0cm}
\setlength{\listparindent}{0cm}
\setlength{\leftmargin}{\evensidemargin}
\addtolength{\leftmargin}{\tmplength}
\settowidth{\labelsep}{X}
\addtolength{\leftmargin}{\labelsep}
\setlength{\labelwidth}{\tmplength}
}
\begin{flushleft}
\item[\textbf{Declaração}\hfill]
\begin{ttfamily}
public class procedure FreeAndNil(var Obj);\end{ttfamily}


\end{flushleft}
\end{list}
\paragraph*{NewStr}\hspace*{\fill}

\begin{list}{}{
\settowidth{\tmplength}{\textbf{Declaração}}
\setlength{\itemindent}{0cm}
\setlength{\listparindent}{0cm}
\setlength{\leftmargin}{\evensidemargin}
\addtolength{\leftmargin}{\tmplength}
\settowidth{\labelsep}{X}
\addtolength{\leftmargin}{\labelsep}
\setlength{\labelwidth}{\tmplength}
}
\begin{flushleft}
\item[\textbf{Declaração}\hfill]
\begin{ttfamily}
public class function NewStr(Const S: AnsiString): ptstring;\end{ttfamily}


\end{flushleft}
\par
\item[\textbf{Descrição}]
{-}

\end{list}
\paragraph*{NewSItem}\hspace*{\fill}

\begin{list}{}{
\settowidth{\tmplength}{\textbf{Declaração}}
\setlength{\itemindent}{0cm}
\setlength{\listparindent}{0cm}
\setlength{\leftmargin}{\evensidemargin}
\addtolength{\leftmargin}{\tmplength}
\settowidth{\labelsep}{X}
\addtolength{\leftmargin}{\labelsep}
\setlength{\labelwidth}{\tmplength}
}
\begin{flushleft}
\item[\textbf{Declaração}\hfill]
\begin{ttfamily}
public class function NewSItem(const Str: tString; ANext: PSItem): PSItem;\end{ttfamily}


\end{flushleft}
\end{list}
\paragraph*{CloneSItems}\hspace*{\fill}

\begin{list}{}{
\settowidth{\tmplength}{\textbf{Declaração}}
\setlength{\itemindent}{0cm}
\setlength{\listparindent}{0cm}
\setlength{\leftmargin}{\evensidemargin}
\addtolength{\leftmargin}{\tmplength}
\settowidth{\labelsep}{X}
\addtolength{\leftmargin}{\labelsep}
\setlength{\labelwidth}{\tmplength}
}
\begin{flushleft}
\item[\textbf{Declaração}\hfill]
\begin{ttfamily}
public class Function CloneSItems(Const Items: PSItem):PSItem;\end{ttfamily}


\end{flushleft}
\end{list}
\paragraph*{CopyTemplateFrom}\hspace*{\fill}

\begin{list}{}{
\settowidth{\tmplength}{\textbf{Declaração}}
\setlength{\itemindent}{0cm}
\setlength{\listparindent}{0cm}
\setlength{\leftmargin}{\evensidemargin}
\addtolength{\leftmargin}{\tmplength}
\settowidth{\labelsep}{X}
\addtolength{\leftmargin}{\labelsep}
\setlength{\labelwidth}{\tmplength}
}
\begin{flushleft}
\item[\textbf{Declaração}\hfill]
\begin{ttfamily}
public class Function CopyTemplateFrom(Const aTemplate:tString): tString;\end{ttfamily}


\end{flushleft}
\par
\item[\textbf{Descrição}]
A class function \textbf{\begin{ttfamily}CopyTemplateFrom\end{ttfamily}} é necessario porque um Template pode ser uma lista de Strings, onde está lista pode ser inserida em um objeto e discartada ao destruir o objeto.

\end{list}
\paragraph*{PushSItem}\hspace*{\fill}

\begin{list}{}{
\settowidth{\tmplength}{\textbf{Declaração}}
\setlength{\itemindent}{0cm}
\setlength{\listparindent}{0cm}
\setlength{\leftmargin}{\evensidemargin}
\addtolength{\leftmargin}{\tmplength}
\settowidth{\labelsep}{X}
\addtolength{\leftmargin}{\labelsep}
\setlength{\labelwidth}{\tmplength}
}
\begin{flushleft}
\item[\textbf{Declaração}\hfill]
\begin{ttfamily}
public class Procedure PushSItem( Str: AnsiString; Var ANext: PSItem);\end{ttfamily}


\end{flushleft}
\end{list}
\paragraph*{Push{\_}MsgErro}\hspace*{\fill}

\begin{list}{}{
\settowidth{\tmplength}{\textbf{Declaração}}
\setlength{\itemindent}{0cm}
\setlength{\listparindent}{0cm}
\setlength{\leftmargin}{\evensidemargin}
\addtolength{\leftmargin}{\tmplength}
\settowidth{\labelsep}{X}
\addtolength{\leftmargin}{\labelsep}
\setlength{\labelwidth}{\tmplength}
}
\begin{flushleft}
\item[\textbf{Declaração}\hfill]
\begin{ttfamily}
public class Procedure Push{\_}MsgErro( Str: AnsiString);\end{ttfamily}


\end{flushleft}
\par
\item[\textbf{Descrição}]
Coloca uma string na pilha

\end{list}
\chapter{Unit mi.rtl.Objects.Consts.Logs}
\section{Descrição}
\begin{itemize}
\item A Unit \textbf{\begin{ttfamily}mi.rtl.Objects.Consts.Logs\end{ttfamily}} implementa a classe \textbf{\begin{ttfamily}TFilesLogs\end{ttfamily}(\ref{mi.rtl.Objects.Consts.Logs.TFilesLogs})} baseado na classe \textbf{TEventLog} da \textbf{FCL}

\begin{itemize}
\item \textbf{VERSÃO} \begin{itemize}
\item Alpha {-} 0.5.0.687
\end{itemize}
\item \textbf{REFERÊNCIA} \begin{itemize}
\item [TEventLog](https://www.freepascal.org/docs-html/current/fcl/eventlog/teventlog.html)
\end{itemize}
\item \textbf{CÓDIGO FONTE}: \begin{itemize}
\item 
\end{itemize}
\item \textbf{HISTÓRICO} \begin{itemize}
\item Criado por: Paulo Sérgio da Silva Pacheco e{-}mail: paulosspacheco@yahoo.com.br \begin{itemize}
\item \textbf{08/12/2021} \begin{itemize}
\item 13:55 a 18:30 : Análise do projeto \textbf{\begin{ttfamily}mi.rtl.Objects.Consts.Logs\end{ttfamily}};
\item 20:30 a 22:40 : Criar a unit \textbf{\begin{ttfamily}mi.rtl.Objects.Consts.Logs\end{ttfamily}} e a classe \textbf{\begin{ttfamily}TFilesLogs\end{ttfamily}(\ref{mi.rtl.Objects.Consts.Logs.TFilesLogs})};
\end{itemize}
\item \textbf{09/12/2021} \begin{itemize}
\item 06:20 a 07:06 : Documentar a unit \textbf{\begin{ttfamily}mi.rtl.Objects.Consts.Logs\end{ttfamily}};
\item 09:02 a 12:15 : Documentar a classe \textbf{\begin{ttfamily}TFilesLogs\end{ttfamily}(\ref{mi.rtl.Objects.Consts.Logs.TFilesLogs})}.
\item 14:45 a 15:15 : Criar os métodos \textbf{\begin{ttfamily}TFilesLogs.Warning\end{ttfamily}(\ref{mi.rtl.Objects.Consts.Logs.TFilesLogs-Warning})} e documentar.
\end{itemize}
\item \textbf{13/12/2021} \begin{itemize}
\item 20:40 a ?? {-}Criar os métodos \textbf{\begin{ttfamily}TFilesLogs.info\end{ttfamily}(\ref{mi.rtl.Objects.Consts.Logs.TFilesLogs-Info})} e documentar.
\end{itemize}
\end{itemize}
\end{itemize}
\end{itemize}
\end{itemize}
\section{Uses}
\begin{itemize}
\item \begin{ttfamily}Classes\end{ttfamily}\item \begin{ttfamily}SysUtils\end{ttfamily}\item \begin{ttfamily}dos\end{ttfamily}\item \begin{ttfamily}mi.rtl.objects.consts\end{ttfamily}(\ref{mi.rtl.Objects.Consts})\item \begin{ttfamily}eventlog\end{ttfamily}\end{itemize}
\section{Visão Geral}
\begin{description}
\item[\texttt{\begin{ttfamily}TFilesLogs\end{ttfamily} Classe}]
\end{description}
\section{Classes, Interfaces, Objetos e Registros}
\subsection*{TFilesLogs Classe}
\subsubsection*{\large{\textbf{Hierarquia}}\normalsize\hspace{1ex}\hfill}
TFilesLogs {$>$} \begin{ttfamily}TObjectsConsts\end{ttfamily}(\ref{mi.rtl.Objects.Consts.TObjectsConsts}) {$>$} 
TObjectsTypes
\subsubsection*{\large{\textbf{Descrição}}\normalsize\hspace{1ex}\hfill}
\begin{itemize}
\item A classe \textbf{\begin{ttfamily}TFilesLogs\end{ttfamily}} é usada para registrar no arquivo \textbf{ParamStr(0)}+'.log' as mensagens de \textbf{Erros}, \textbf{Atenções} e \textbf{avisos} do sistema.

\begin{itemize}
\item \textbf{NOTA} \begin{itemize}
\item A classe \begin{ttfamily}TObjectss\end{ttfamily}(\ref{mi.rtl.Objectss.TObjectss}) implementa a constante global \textbf{const TObjectss.Logs : \begin{ttfamily}TFilesLogs\end{ttfamily} = nil;} que é inicializado em \textbf{mi.rtl.objectss.Initialization} e destruído em \textbf{mi.rtl.objectss.finalization}
\end{itemize}
\end{itemize}
\end{itemize}\subsubsection*{\large{\textbf{Propriedades}}\normalsize\hspace{1ex}\hfill}
\paragraph*{fileLog}\hspace*{\fill}

\begin{list}{}{
\settowidth{\tmplength}{\textbf{Declaração}}
\setlength{\itemindent}{0cm}
\setlength{\listparindent}{0cm}
\setlength{\leftmargin}{\evensidemargin}
\addtolength{\leftmargin}{\tmplength}
\settowidth{\labelsep}{X}
\addtolength{\leftmargin}{\labelsep}
\setlength{\labelwidth}{\tmplength}
}
\begin{flushleft}
\item[\textbf{Declaração}\hfill]
\begin{ttfamily}
published property fileLog : TEventLog Read {\_}fileLog;\end{ttfamily}


\end{flushleft}
\end{list}
\paragraph*{FileName}\hspace*{\fill}

\begin{list}{}{
\settowidth{\tmplength}{\textbf{Declaração}}
\setlength{\itemindent}{0cm}
\setlength{\listparindent}{0cm}
\setlength{\leftmargin}{\evensidemargin}
\addtolength{\leftmargin}{\tmplength}
\settowidth{\labelsep}{X}
\addtolength{\leftmargin}{\labelsep}
\setlength{\labelwidth}{\tmplength}
}
\begin{flushleft}
\item[\textbf{Declaração}\hfill]
\begin{ttfamily}
public property FileName : string read GetFileName write SetFileName;\end{ttfamily}


\end{flushleft}
\par
\item[\textbf{Descrição}]
A propriedade \textbf{\begin{ttfamily}FileName\end{ttfamily}} armazena o nome do arquivo de log necessário quando a propriedade \textbf{\begin{ttfamily}LogType\end{ttfamily}(\ref{mi.rtl.Objects.Consts.Logs.TFilesLogs-LogType})=ltFile}

\end{list}
\paragraph*{Active}\hspace*{\fill}

\begin{list}{}{
\settowidth{\tmplength}{\textbf{Declaração}}
\setlength{\itemindent}{0cm}
\setlength{\listparindent}{0cm}
\setlength{\leftmargin}{\evensidemargin}
\addtolength{\leftmargin}{\tmplength}
\settowidth{\labelsep}{X}
\addtolength{\leftmargin}{\labelsep}
\setlength{\labelwidth}{\tmplength}
}
\begin{flushleft}
\item[\textbf{Declaração}\hfill]
\begin{ttfamily}
public property Active : Boolean Read GetActive write SetActive;\end{ttfamily}


\end{flushleft}
\par
\item[\textbf{Descrição}]
A propriedade \textbf{\begin{ttfamily}Active\end{ttfamily}} ativa e desativa a gravação de mensagens no arquivo de log.

\end{list}
\paragraph*{LogType}\hspace*{\fill}

\begin{list}{}{
\settowidth{\tmplength}{\textbf{Declaração}}
\setlength{\itemindent}{0cm}
\setlength{\listparindent}{0cm}
\setlength{\leftmargin}{\evensidemargin}
\addtolength{\leftmargin}{\tmplength}
\settowidth{\labelsep}{X}
\addtolength{\leftmargin}{\labelsep}
\setlength{\labelwidth}{\tmplength}
}
\begin{flushleft}
\item[\textbf{Declaração}\hfill]
\begin{ttfamily}
public property LogType : TLogType Read GetLogtype Write SetlogType;\end{ttfamily}


\end{flushleft}
\par
\item[\textbf{Descrição}]
\begin{itemize}
\item A propriedade \textbf{\begin{ttfamily}LogType\end{ttfamily}} redireciona o destino das mensagens de log.

\begin{itemize}
\item Veja o tipo \textbf{\begin{ttfamily}TLogType\end{ttfamily}(\ref{mi.rtl.Objects.Consts.Logs.TFilesLogs-TLogType})} para mais informações.
\end{itemize}
\end{itemize}

\end{list}
\paragraph*{AppendContent}\hspace*{\fill}

\begin{list}{}{
\settowidth{\tmplength}{\textbf{Declaração}}
\setlength{\itemindent}{0cm}
\setlength{\listparindent}{0cm}
\setlength{\leftmargin}{\evensidemargin}
\addtolength{\leftmargin}{\tmplength}
\settowidth{\labelsep}{X}
\addtolength{\leftmargin}{\labelsep}
\setlength{\labelwidth}{\tmplength}
}
\begin{flushleft}
\item[\textbf{Declaração}\hfill]
\begin{ttfamily}
public property AppendContent : Boolean Read GetAppendContent Write SetAppendContent;\end{ttfamily}


\end{flushleft}
\par
\item[\textbf{Descrição}]
\begin{itemize}
\item A propriedade \textbf{\begin{ttfamily}AppendContent\end{ttfamily}} é usada para indicar ao sistema que o arquivo de log deve ser único ou acumulativo

\begin{itemize}
\item \textbf{PARÂMETROS} \begin{itemize}
\item \textbf{True} : A abertura do arquivo de log é aberto com o comando \textbf{append}
\item \textbf{False} : A abertura do arquivo de log é aberto com o comando \textbf{rewrite}
\item \textbf{Código usado na abertura do arquivo}

\texttt{\\\nopagebreak[3]
\\\nopagebreak[3]
}\textbf{if}\texttt{~AppendContent~}\textbf{and}\texttt{~FileExists(FileName)~}\textbf{then}\texttt{\\\nopagebreak[3]
~~~FileFlags~:=~fmOpenWrite\\\nopagebreak[3]
}\textbf{else}\texttt{\\\nopagebreak[3]
~~~FileFlags~:=~fmCreate;\\
}
\end{itemize}
\end{itemize}
\end{itemize}

\end{list}
\paragraph*{RaiseExceptionOnError}\hspace*{\fill}

\begin{list}{}{
\settowidth{\tmplength}{\textbf{Declaração}}
\setlength{\itemindent}{0cm}
\setlength{\listparindent}{0cm}
\setlength{\leftmargin}{\evensidemargin}
\addtolength{\leftmargin}{\tmplength}
\settowidth{\labelsep}{X}
\addtolength{\leftmargin}{\labelsep}
\setlength{\labelwidth}{\tmplength}
}
\begin{flushleft}
\item[\textbf{Declaração}\hfill]
\begin{ttfamily}
public property RaiseExceptionOnError : Boolean Read GetRaiseExceptionOnError Write SetRaiseExceptionOnError;\end{ttfamily}


\end{flushleft}
\par
\item[\textbf{Descrição}]
\begin{itemize}
\item A propriedade \textbf{\begin{ttfamily}RaiseExceptionOnError\end{ttfamily}} informa ao sistema se deve gerar exceções quando houver erro ao gravar o log.

\begin{itemize}
\item \textbf{Exemplo de uso de \begin{ttfamily}RaiseExceptionOnError\end{ttfamily}}

\texttt{\\\nopagebreak[3]
\\\nopagebreak[3]
}\textbf{procedure}\texttt{~TEventLog.WriteFileLog(EventType~:~TEventType;~}\textbf{const}\texttt{~Msg~:~}\textbf{String}\texttt{);\\\nopagebreak[3]
~~}\textbf{Var}\texttt{\\\nopagebreak[3]
~~~~S~:~}\textbf{String}\texttt{;\\\nopagebreak[3]
}\textbf{begin}\texttt{\\\nopagebreak[3]
~~S:=FormatLogMessage(EventType,~Msg)+LineEnding;\\\nopagebreak[3]
~~}\textbf{try}\texttt{\\\nopagebreak[3]
~~~~FStream.WriteBuffer(S[1],Length(S));\\\nopagebreak[3]
~~~~S:='';\\\nopagebreak[3]
~~}\textbf{except}\texttt{\\\nopagebreak[3]
~~~~}\textbf{On}\texttt{~E~:~Exception~}\textbf{do}\texttt{\\\nopagebreak[3]
~~~~~~~S:=E.}\textbf{Message}\texttt{;\\\nopagebreak[3]
~~~~}\textbf{end}\texttt{;\\\nopagebreak[3]
\\\nopagebreak[3]
~~}\textbf{If}\texttt{~(S{$<$}{$>$}'')~}\textbf{and}\texttt{~RaiseExceptionOnError~\\\nopagebreak[3]
~~}\textbf{then}\texttt{~~}\textbf{Raise}\texttt{~ELogError.CreateFmt(SErrLogFailedMsg,[S]);\\\nopagebreak[3]
\\\nopagebreak[3]
}\textbf{end}\texttt{;\\
}
\end{itemize}
\end{itemize}

\end{list}
\subsubsection*{\large{\textbf{Campos}}\normalsize\hspace{1ex}\hfill}
\paragraph*{EnableWriteIdentificao}\hspace*{\fill}

\begin{list}{}{
\settowidth{\tmplength}{\textbf{Declaração}}
\setlength{\itemindent}{0cm}
\setlength{\listparindent}{0cm}
\setlength{\leftmargin}{\evensidemargin}
\addtolength{\leftmargin}{\tmplength}
\settowidth{\labelsep}{X}
\addtolength{\leftmargin}{\labelsep}
\setlength{\labelwidth}{\tmplength}
}
\begin{flushleft}
\item[\textbf{Declaração}\hfill]
\begin{ttfamily}
public const EnableWriteIdentificao : Boolean = true;\end{ttfamily}


\end{flushleft}
\end{list}
\subsubsection*{\large{\textbf{Métodos}}\normalsize\hspace{1ex}\hfill}
\paragraph*{Create}\hspace*{\fill}

\begin{list}{}{
\settowidth{\tmplength}{\textbf{Declaração}}
\setlength{\itemindent}{0cm}
\setlength{\listparindent}{0cm}
\setlength{\leftmargin}{\evensidemargin}
\addtolength{\leftmargin}{\tmplength}
\settowidth{\labelsep}{X}
\addtolength{\leftmargin}{\labelsep}
\setlength{\labelwidth}{\tmplength}
}
\begin{flushleft}
\item[\textbf{Declaração}\hfill]
\begin{ttfamily}
public constructor Create(aowner:TComponent); Override; Overload;\end{ttfamily}


\end{flushleft}
\par
\item[\textbf{Descrição}]
O constructor \textbf{\begin{ttfamily}Create\end{ttfamily}} cria uma instância da classe \begin{ttfamily}TFilesLogs\end{ttfamily}(\ref{mi.rtl.Objects.Consts.Logs.TFilesLogs}) e inicializa as propriedade \textbf{\begin{ttfamily}FileName\end{ttfamily}(\ref{mi.rtl.Objects.Consts.Logs.TFilesLogs-FileName}) := ParamStr(0)+'.log'}, \textbf{\begin{ttfamily}LogType\end{ttfamily}(\ref{mi.rtl.Objects.Consts.Logs.TFilesLogs-LogType}) := ltFile}, \textit{\begin{ttfamily}RaiseExceptionOnError\end{ttfamily}(\ref{mi.rtl.Objects.Consts.Logs.TFilesLogs-RaiseExceptionOnError}) := true}, \textbf{\begin{ttfamily}AppendContent\end{ttfamily}(\ref{mi.rtl.Objects.Consts.Logs.TFilesLogs-AppendContent}) := true} e \textbf{ \begin{ttfamily}active\end{ttfamily}(\ref{mi.rtl.Objects.Consts.Logs.TFilesLogs-Active}) := true}.

\end{list}
\paragraph*{destroy}\hspace*{\fill}

\begin{list}{}{
\settowidth{\tmplength}{\textbf{Declaração}}
\setlength{\itemindent}{0cm}
\setlength{\listparindent}{0cm}
\setlength{\leftmargin}{\evensidemargin}
\addtolength{\leftmargin}{\tmplength}
\settowidth{\labelsep}{X}
\addtolength{\leftmargin}{\labelsep}
\setlength{\labelwidth}{\tmplength}
}
\begin{flushleft}
\item[\textbf{Declaração}\hfill]
\begin{ttfamily}
public destructor destroy; override;\end{ttfamily}


\end{flushleft}
\end{list}
\paragraph*{Warning}\hspace*{\fill}

\begin{list}{}{
\settowidth{\tmplength}{\textbf{Declaração}}
\setlength{\itemindent}{0cm}
\setlength{\listparindent}{0cm}
\setlength{\leftmargin}{\evensidemargin}
\addtolength{\leftmargin}{\tmplength}
\settowidth{\labelsep}{X}
\addtolength{\leftmargin}{\labelsep}
\setlength{\labelwidth}{\tmplength}
}
\begin{flushleft}
\item[\textbf{Declaração}\hfill]
\begin{ttfamily}
public procedure Warning( const Fmt: string;Args: array of Const); overload;\end{ttfamily}


\end{flushleft}
\par
\item[\textbf{Descrição}]
\begin{itemize}
\item A procedure \textbf{\begin{ttfamily}Warning\end{ttfamily}} é usada registrar mensagens do tipo \textbf{etWarning}.

\begin{itemize}
\item \textbf{PARÂMETROS} \begin{itemize}
\item \textbf{Fmt} : String que será formatada com a função [\textbf{Format}](https://www.freepascal.org/docs-html/rtl/sysutils/format.html)
\item \textbf{Args} : Valores a serem usadas no função format(FMT,Args)
\end{itemize}
\item \textbf{REFERÊNCIA} \begin{itemize}
\item [TEventLog.warning](https://www.freepascal.org/docs-html/fcl/eventlog/teventlog.warning.html)
\end{itemize}
\item \textbf{EXEMPLO}:

\texttt{\\\nopagebreak[3]
\\\nopagebreak[3]
}\textbf{procedure}\texttt{~TMi{\_}Rtl{\_}Tests.Action{\_}test{\_}Logs{\_}WarningExecute(Sender:~TObject);\\\nopagebreak[3]
}\textbf{begin}\texttt{\\\nopagebreak[3]
~~}\textbf{with}\texttt{~Tobjectss~}\textbf{do}\texttt{\\\nopagebreak[3]
~~~~}\textbf{if}\texttt{~Logs.Active~}\textbf{then}\texttt{\\\nopagebreak[3]
~~~~}\textbf{with}\texttt{~Logs~}\textbf{do}\texttt{\\\nopagebreak[3]
~~~~}\textbf{begin}\texttt{\\\nopagebreak[3]
~~~~~~Warning('Pouco~espaço~em~disco.~Tamanho~livre~=~{\%}s.',~['1GB']);\\\nopagebreak[3]
~~~~}\textbf{end}\texttt{;\\\nopagebreak[3]
}\textbf{end}\texttt{;\\
}

\begin{itemize}
\item Arquivo de logs: \textbf{rtl.log}: \begin{itemize}
\item [2021{-}12{-}09 14:58:00.588 \begin{ttfamily}Warning\end{ttfamily}(\ref{mi.rtl.Objects.Consts.Logs.TFilesLogs-Warning})] Pouco espaço em disco. Tamanho livre = 1GB.
\end{itemize}
\end{itemize}
\end{itemize}
\end{itemize}

\end{list}
\paragraph*{Warning}\hspace*{\fill}

\begin{list}{}{
\settowidth{\tmplength}{\textbf{Declaração}}
\setlength{\itemindent}{0cm}
\setlength{\listparindent}{0cm}
\setlength{\leftmargin}{\evensidemargin}
\addtolength{\leftmargin}{\tmplength}
\settowidth{\labelsep}{X}
\addtolength{\leftmargin}{\labelsep}
\setlength{\labelwidth}{\tmplength}
}
\begin{flushleft}
\item[\textbf{Declaração}\hfill]
\begin{ttfamily}
public procedure Warning( const Msg: string ); overload;\end{ttfamily}


\end{flushleft}
\par
\item[\textbf{Descrição}]
\begin{itemize}
\item A procedure \textbf{\begin{ttfamily}Warning\end{ttfamily}} é usada registrar mensagens do tipo \textbf{etWarning}.

\begin{itemize}
\item \textbf{PARÂMETROS} \begin{itemize}
\item \textbf{Msg} : String com a mensagem de atenção.
\end{itemize}
\item \textbf{REFERÊNCIA} \begin{itemize}
\item [TEventLog.warning](https://www.freepascal.org/docs-html/fcl/eventlog/teventlog.warning.html)
\end{itemize}
\item \textbf{EXEMPLO}:

\texttt{\\\nopagebreak[3]
\\\nopagebreak[3]
}\textbf{procedure}\texttt{~TMi{\_}Rtl{\_}Tests.Button2Click(Sender:~TObject);\\\nopagebreak[3]
}\textbf{begin}\texttt{\\\nopagebreak[3]
~~}\textbf{with}\texttt{~Tobjectss~}\textbf{do}\texttt{\\\nopagebreak[3]
~~~~}\textbf{if}\texttt{~Logs.Active~}\textbf{then}\texttt{\\\nopagebreak[3]
~~~~}\textbf{with}\texttt{~Logs~}\textbf{do}\texttt{\\\nopagebreak[3]
~~~~}\textbf{begin}\texttt{\\\nopagebreak[3]
~~~~~~Warning('Senha~inválida!');\\\nopagebreak[3]
~~~~}\textbf{end}\texttt{;\\\nopagebreak[3]
}\textbf{end}\texttt{;\\
}

\begin{itemize}
\item Arquivo de logs: \textbf{rtl.log}: \begin{itemize}
\item [2021{-}12{-}09 14:58:00.588 \begin{ttfamily}Warning\end{ttfamily}] Senha inválida
\end{itemize}
\end{itemize}
\end{itemize}
\end{itemize}

\end{list}
\paragraph*{Error}\hspace*{\fill}

\begin{list}{}{
\settowidth{\tmplength}{\textbf{Declaração}}
\setlength{\itemindent}{0cm}
\setlength{\listparindent}{0cm}
\setlength{\leftmargin}{\evensidemargin}
\addtolength{\leftmargin}{\tmplength}
\settowidth{\labelsep}{X}
\addtolength{\leftmargin}{\labelsep}
\setlength{\labelwidth}{\tmplength}
}
\begin{flushleft}
\item[\textbf{Declaração}\hfill]
\begin{ttfamily}
public procedure Error(const Fmt: String; Args: array of const); overload;\end{ttfamily}


\end{flushleft}
\par
\item[\textbf{Descrição}]
\begin{itemize}
\item A procedure \textbf{\begin{ttfamily}Error\end{ttfamily}} é usada registrar mensagens do tipo \textbf{etError}.

\begin{itemize}
\item \textbf{PARÂMETROS} \begin{itemize}
\item \textbf{Fmt} : String que será formatada com a função [\textbf{Format}](https://www.freepascal.org/docs-html/rtl/sysutils/format.html)
\item \textbf{Args} : Valores a serem usadas no função format(FMT,Args)
\end{itemize}
\item \textbf{REFERÊNCIA} \begin{itemize}
\item [TEventLog.error](https://www.freepascal.org/docs-html/fcl/eventlog/teventlog.error.html)
\end{itemize}
\item \textbf{EXEMPLO}:

\texttt{\\\nopagebreak[3]
\\\nopagebreak[3]
}\textbf{procedure}\texttt{~TMi{\_}Rtl{\_}Tests.Action{\_}test{\_}Logs{\_}ErrorExecute(Sender:~TObject);\\\nopagebreak[3]
}\textbf{begin}\texttt{\\\nopagebreak[3]
~~}\textbf{with}\texttt{~Tobjectss~}\textbf{do}\texttt{\\\nopagebreak[3]
~~~~}\textbf{if}\texttt{~Logs.Active~}\textbf{then}\texttt{\\\nopagebreak[3]
~~~~}\textbf{with}\texttt{~Logs~}\textbf{do}\texttt{\\\nopagebreak[3]
~~~~}\textbf{begin}\texttt{\\\nopagebreak[3]
~~~~~~Error('Exemplo~de~mensagem~de~erro.~Número~do~erro~=~{\%}d.',~[5]);\\\nopagebreak[3]
~~~~}\textbf{end}\texttt{;\\\nopagebreak[3]
}\textbf{end}\texttt{;\\
}

\begin{itemize}
\item Arquivo de logs: \textbf{rtl.log}: \begin{itemize}
\item [2021{-}12{-}09 10:25:41.425 \begin{ttfamily}Error\end{ttfamily}(\ref{mi.rtl.Objects.Consts.Logs.TFilesLogs-Error})] Exemplo de mensagem de erro. Número do erro = 5.
\end{itemize}
\end{itemize}
\end{itemize}
\end{itemize}

\end{list}
\paragraph*{Error}\hspace*{\fill}

\begin{list}{}{
\settowidth{\tmplength}{\textbf{Declaração}}
\setlength{\itemindent}{0cm}
\setlength{\listparindent}{0cm}
\setlength{\leftmargin}{\evensidemargin}
\addtolength{\leftmargin}{\tmplength}
\settowidth{\labelsep}{X}
\addtolength{\leftmargin}{\labelsep}
\setlength{\labelwidth}{\tmplength}
}
\begin{flushleft}
\item[\textbf{Declaração}\hfill]
\begin{ttfamily}
public procedure Error(const Msg: String); overload;\end{ttfamily}


\end{flushleft}
\par
\item[\textbf{Descrição}]
\begin{itemize}
\item A procedure \textbf{\begin{ttfamily}Error\end{ttfamily}} é usada registrar mensagens do tipo \textbf{etError}.

\begin{itemize}
\item \textbf{PARÂMETROS} \begin{itemize}
\item \textbf{Msg} : String com o nome do erro.
\end{itemize}
\item \textbf{REFERÊNCIA} \begin{itemize}
\item [TEventLog.error](https://www.freepascal.org/docs-html/fcl/eventlog/teventlog.error.html)
\end{itemize}
\item \textbf{EXEMPLO}:

\texttt{\\\nopagebreak[3]
\\\nopagebreak[3]
}\textbf{procedure}\texttt{~TMi{\_}Rtl{\_}Tests.Button2Click(Sender:~TObject);\\\nopagebreak[3]
}\textbf{begin}\texttt{\\\nopagebreak[3]
~~}\textbf{with}\texttt{~TObjectss~}\textbf{do}\texttt{\\\nopagebreak[3]
~~~~}\textbf{if}\texttt{~Logs.Active~}\textbf{then}\texttt{\\\nopagebreak[3]
~~~~}\textbf{with}\texttt{~Logs~}\textbf{do}\texttt{\\\nopagebreak[3]
~~~~}\textbf{begin}\texttt{\\\nopagebreak[3]
~~~~~~Error('Acesso~ao~arquivo~negado.'~);\\\nopagebreak[3]
~~~~}\textbf{end}\texttt{;\\\nopagebreak[3]
}\textbf{end}\texttt{;\\
}

\begin{itemize}
\item Arquivo de logs: \textbf{rtl.log}: \begin{itemize}
\item [2021{-}12{-}09 10:25:41.425 \begin{ttfamily}Error\end{ttfamily}] Acesso ao arquivo negado.
\end{itemize}
\end{itemize}
\end{itemize}
\end{itemize}

\end{list}
\paragraph*{Info}\hspace*{\fill}

\begin{list}{}{
\settowidth{\tmplength}{\textbf{Declaração}}
\setlength{\itemindent}{0cm}
\setlength{\listparindent}{0cm}
\setlength{\leftmargin}{\evensidemargin}
\addtolength{\leftmargin}{\tmplength}
\settowidth{\labelsep}{X}
\addtolength{\leftmargin}{\labelsep}
\setlength{\labelwidth}{\tmplength}
}
\begin{flushleft}
\item[\textbf{Declaração}\hfill]
\begin{ttfamily}
public procedure Info(const Fmt: String; Args: array of const); overload;\end{ttfamily}


\end{flushleft}
\par
\item[\textbf{Descrição}]
\begin{itemize}
\item A procedure \textbf{\begin{ttfamily}Info\end{ttfamily}} é usada registrar mensagens do tipo \textbf{etInfo}.

\begin{itemize}
\item \textbf{PARÂMETROS} \begin{itemize}
\item \textbf{Fmt} : String que será formatada com a função [\textbf{Format}](https://www.freepascal.org/docs-html/rtl/sysutils/format.html)
\item \textbf{Args} : Valores a serem usadas no função format(FMT,Args)
\end{itemize}
\item \textbf{REFERÊNCIA} \begin{itemize}
\item [TEventLog.Info](https://www.freepascal.org/docs-html/current/fcl/eventlog/teventlog.info.html)
\end{itemize}
\item \textbf{EXEMPLO}:

\texttt{\\\nopagebreak[3]
\\\nopagebreak[3]
}\textbf{procedure}\texttt{~TMi{\_}Rtl{\_}Tests.Action{\_}test{\_}Logs{\_}InfoExecute(Sender:~TObject);\\\nopagebreak[3]
}\textbf{begin}\texttt{\\\nopagebreak[3]
~~}\textbf{with}\texttt{~Tobjectss~}\textbf{do}\texttt{\\\nopagebreak[3]
~~~~}\textbf{if}\texttt{~Logs.Active~}\textbf{then}\texttt{\\\nopagebreak[3]
~~~~}\textbf{with}\texttt{~Logs~}\textbf{do}\texttt{\\\nopagebreak[3]
~~~~}\textbf{begin}\texttt{\\\nopagebreak[3]
~~~~~~Info('A~versão~{\%}s~será~lançada~em~breve.',~['Beta~3.5']);\\\nopagebreak[3]
~~~~}\textbf{end}\texttt{;\\\nopagebreak[3]
}\textbf{end}\texttt{;\\
}

\begin{itemize}
\item Arquivo de logs: \textbf{rtl.log}: \begin{itemize}
\item [2021{-}12{-}13 20:46:32.745 \begin{ttfamily}Info\end{ttfamily}(\ref{mi.rtl.Objects.Consts.Logs.TFilesLogs-Info})] A versão Beta 3.5 será lançada em breve.
\end{itemize}
\end{itemize}
\end{itemize}
\end{itemize}

\end{list}
\paragraph*{Info}\hspace*{\fill}

\begin{list}{}{
\settowidth{\tmplength}{\textbf{Declaração}}
\setlength{\itemindent}{0cm}
\setlength{\listparindent}{0cm}
\setlength{\leftmargin}{\evensidemargin}
\addtolength{\leftmargin}{\tmplength}
\settowidth{\labelsep}{X}
\addtolength{\leftmargin}{\labelsep}
\setlength{\labelwidth}{\tmplength}
}
\begin{flushleft}
\item[\textbf{Declaração}\hfill]
\begin{ttfamily}
public procedure Info(const Msg: String); overload;\end{ttfamily}


\end{flushleft}
\par
\item[\textbf{Descrição}]
\begin{itemize}
\item A procedure \textbf{\begin{ttfamily}Info\end{ttfamily}} é usada registrar mensagens do tipo \textbf{etInfo}.

\begin{itemize}
\item \textbf{PARÂMETROS} \begin{itemize}
\item \textbf{Msg} : String com mensagem informativa.
\end{itemize}
\item \textbf{REFERÊNCIA} \begin{itemize}
\item [TEventLog.Info](https://www.freepascal.org/docs-html/current/fcl/eventlog/teventlog.info.html)
\end{itemize}
\item \textbf{EXEMPLO}:

\texttt{\\\nopagebreak[3]
\\\nopagebreak[3]
}\textbf{procedure}\texttt{~TMi{\_}Rtl{\_}Tests.Action{\_}Test{\_}Logs{\_}InfoExecute(Sender:~TObject);\\\nopagebreak[3]
}\textbf{begin}\texttt{\\\nopagebreak[3]
~~}\textbf{with}\texttt{~Tobjectss~}\textbf{do}\texttt{\\\nopagebreak[3]
~~~~}\textbf{if}\texttt{~Logs.Active~}\textbf{then}\texttt{\\\nopagebreak[3]
~~~~}\textbf{with}\texttt{~Logs~}\textbf{do}\texttt{\\\nopagebreak[3]
~~~~}\textbf{begin}\texttt{\\\nopagebreak[3]
~~~~~~info('Sistema~abortou~de~forma~inesperada.'~);\\\nopagebreak[3]
~~~~}\textbf{end}\texttt{;\\\nopagebreak[3]
}\textbf{end}\texttt{;\\
}

\begin{itemize}
\item Arquivo de logs: \textbf{rtl.log}: \begin{itemize}
\item [2021{-}12{-}13 20:46:32.745 \begin{ttfamily}Info\end{ttfamily}] Sistema abortou de forma inesperada.
\end{itemize}
\end{itemize}
\end{itemize}
\end{itemize}

\end{list}
\paragraph*{{\_}Write}\hspace*{\fill}

\begin{list}{}{
\settowidth{\tmplength}{\textbf{Declaração}}
\setlength{\itemindent}{0cm}
\setlength{\listparindent}{0cm}
\setlength{\leftmargin}{\evensidemargin}
\addtolength{\leftmargin}{\tmplength}
\settowidth{\labelsep}{X}
\addtolength{\leftmargin}{\labelsep}
\setlength{\labelwidth}{\tmplength}
}
\begin{flushleft}
\item[\textbf{Declaração}\hfill]
\begin{ttfamily}
public Function {\_}Write(Const S : AnsiString;Ln:Boolean):SmallInt;\end{ttfamily}


\end{flushleft}
\end{list}
\paragraph*{Flush{\_}WiteBuffer}\hspace*{\fill}

\begin{list}{}{
\settowidth{\tmplength}{\textbf{Declaração}}
\setlength{\itemindent}{0cm}
\setlength{\listparindent}{0cm}
\setlength{\leftmargin}{\evensidemargin}
\addtolength{\leftmargin}{\tmplength}
\settowidth{\labelsep}{X}
\addtolength{\leftmargin}{\labelsep}
\setlength{\labelwidth}{\tmplength}
}
\begin{flushleft}
\item[\textbf{Declaração}\hfill]
\begin{ttfamily}
public Function Flush{\_}WiteBuffer:Boolean;\end{ttfamily}


\end{flushleft}
\end{list}
\paragraph*{WriteFErr{\_}Ln{\_}On{\_}Off}\hspace*{\fill}

\begin{list}{}{
\settowidth{\tmplength}{\textbf{Declaração}}
\setlength{\itemindent}{0cm}
\setlength{\listparindent}{0cm}
\setlength{\leftmargin}{\evensidemargin}
\addtolength{\leftmargin}{\tmplength}
\settowidth{\labelsep}{X}
\addtolength{\leftmargin}{\labelsep}
\setlength{\labelwidth}{\tmplength}
}
\begin{flushleft}
\item[\textbf{Declaração}\hfill]
\begin{ttfamily}
public Function WriteFErr{\_}Ln{\_}On{\_}Off(Const S : AnsiString;aLn{\_}On{\_}Off:Boolean):SmallInt;\end{ttfamily}


\end{flushleft}
\end{list}
\paragraph*{WriteFErr}\hspace*{\fill}

\begin{list}{}{
\settowidth{\tmplength}{\textbf{Declaração}}
\setlength{\itemindent}{0cm}
\setlength{\listparindent}{0cm}
\setlength{\leftmargin}{\evensidemargin}
\addtolength{\leftmargin}{\tmplength}
\settowidth{\labelsep}{X}
\addtolength{\leftmargin}{\labelsep}
\setlength{\labelwidth}{\tmplength}
}
\begin{flushleft}
\item[\textbf{Declaração}\hfill]
\begin{ttfamily}
public Function WriteFErr(Const S : AnsiString):SmallInt;\end{ttfamily}


\end{flushleft}
\end{list}
\paragraph*{WriteLnFErr}\hspace*{\fill}

\begin{list}{}{
\settowidth{\tmplength}{\textbf{Declaração}}
\setlength{\itemindent}{0cm}
\setlength{\listparindent}{0cm}
\setlength{\leftmargin}{\evensidemargin}
\addtolength{\leftmargin}{\tmplength}
\settowidth{\labelsep}{X}
\addtolength{\leftmargin}{\labelsep}
\setlength{\labelwidth}{\tmplength}
}
\begin{flushleft}
\item[\textbf{Declaração}\hfill]
\begin{ttfamily}
public Function WriteLnFErr(Const S : AnsiString):SmallInt;\end{ttfamily}


\end{flushleft}
\end{list}
\paragraph*{WriteIdentificao}\hspace*{\fill}

\begin{list}{}{
\settowidth{\tmplength}{\textbf{Declaração}}
\setlength{\itemindent}{0cm}
\setlength{\listparindent}{0cm}
\setlength{\leftmargin}{\evensidemargin}
\addtolength{\leftmargin}{\tmplength}
\settowidth{\labelsep}{X}
\addtolength{\leftmargin}{\labelsep}
\setlength{\labelwidth}{\tmplength}
}
\begin{flushleft}
\item[\textbf{Declaração}\hfill]
\begin{ttfamily}
public Procedure WriteIdentificao;\end{ttfamily}


\end{flushleft}
\end{list}
\paragraph*{LogError}\hspace*{\fill}

\begin{list}{}{
\settowidth{\tmplength}{\textbf{Declaração}}
\setlength{\itemindent}{0cm}
\setlength{\listparindent}{0cm}
\setlength{\leftmargin}{\evensidemargin}
\addtolength{\leftmargin}{\tmplength}
\settowidth{\labelsep}{X}
\addtolength{\leftmargin}{\labelsep}
\setlength{\labelwidth}{\tmplength}
}
\begin{flushleft}
\item[\textbf{Declaração}\hfill]
\begin{ttfamily}
public PROCEDURE LogError(const Fmt: String; Args: array of const); overload;\end{ttfamily}


\end{flushleft}
\end{list}
\paragraph*{LogError}\hspace*{\fill}

\begin{list}{}{
\settowidth{\tmplength}{\textbf{Declaração}}
\setlength{\itemindent}{0cm}
\setlength{\listparindent}{0cm}
\setlength{\leftmargin}{\evensidemargin}
\addtolength{\leftmargin}{\tmplength}
\settowidth{\labelsep}{X}
\addtolength{\leftmargin}{\labelsep}
\setlength{\labelwidth}{\tmplength}
}
\begin{flushleft}
\item[\textbf{Declaração}\hfill]
\begin{ttfamily}
public PROCEDURE LogError(CONST Msg:AnsiString ); overload;\end{ttfamily}


\end{flushleft}
\end{list}
\chapter{Unit mi.rtl.Objects.Consts.ProgressDlg{\_}If}
\section{Descrição}
\begin{itemize}
\item A unit \textbf{\begin{ttfamily}mi.rtl.Objects.Consts.ProgressDlg{\_}If\end{ttfamily}} implementa a classe \begin{ttfamily}TProgressDlg{\_}If\end{ttfamily}(\ref{mi.rtl.Objects.Consts.ProgressDlg_If.TProgressDlg_If}) do pacote \begin{ttfamily}mi.rtl\end{ttfamily}(\ref{mi.rtl}).

\begin{itemize}
\item \textbf{VERSÃO}: \begin{itemize}
\item Alpha {-} 0.5.0.687
\end{itemize}
\item \textbf{CÓDIGO FONTE}: \begin{itemize}
\item 
\end{itemize}
\item \textbf{HISTÓRICO} \begin{itemize}
\item Criado por: Paulo Sérgio da Silva Pacheco e{-}mail: paulosspacheco@yahoo.com.br \begin{itemize}
\item 2021{-}12{-}18 \begin{itemize}
\item 14:42 15:30 {-} T12 Criar a unit \textbf{\begin{ttfamily}mi.rtl.Objects.Consts.ProgressDlg{\_}If\end{ttfamily}} e a classe \textbf{\begin{ttfamily}Tprogressdlg{\_}if\end{ttfamily}(\ref{mi.rtl.Objects.Consts.ProgressDlg_If.TProgressDlg_If})}
\end{itemize}
\item 2021{-}12{-}21 \begin{itemize}
\item 8:27 a xx {-} T21 documentar a classe \textbf{\begin{ttfamily}TProgressDlg{\_}If\end{ttfamily}(\ref{mi.rtl.Objects.Consts.ProgressDlg_If.TProgressDlg_If})}
\end{itemize}
\end{itemize}
\end{itemize}
\end{itemize}
\end{itemize}
\section{Uses}
\begin{itemize}
\item \begin{ttfamily}Classes\end{ttfamily}\item \begin{ttfamily}drivers\end{ttfamily}\item \begin{ttfamily}DOS\end{ttfamily}\item \begin{ttfamily}LazarusPackageIntf\end{ttfamily}\item \begin{ttfamily}mi.rtl.objects.consts\end{ttfamily}(\ref{mi.rtl.Objects.Consts})\end{itemize}
\section{Visão Geral}
\begin{description}
\item[\texttt{\begin{ttfamily}TProgressDlg{\_}If\end{ttfamily} Classe}]
\end{description}
\begin{description}
\item[\texttt{Register}]
\end{description}
\section{Classes, Interfaces, Objetos e Registros}
\subsection*{TProgressDlg{\_}If Classe}
\subsubsection*{\large{\textbf{Hierarquia}}\normalsize\hspace{1ex}\hfill}
TProgressDlg{\_}If {$>$} \begin{ttfamily}TObjectsConsts\end{ttfamily}(\ref{mi.rtl.Objects.Consts.TObjectsConsts}) {$>$} 
TObjectsTypes
\subsubsection*{\large{\textbf{Descrição}}\normalsize\hspace{1ex}\hfill}
\begin{itemize}
\item A classe \textbf{\begin{ttfamily}TProgressDlg{\_}If\end{ttfamily}} é uma classe abstrata para comunicação com o usuário cujo a implementação deve ser feita nas plataformas: LCL, HTML e JavaScript.

\begin{itemize}
\item \textbf{NOTA} \begin{itemize}
\item Só cria o dialogo se a posição chegar no \begin{ttfamily}delta\end{ttfamily}(\ref{mi.rtl.Objects.Consts.ProgressDlg_If.TProgressDlg_If-Delta}).
\end{itemize}
\item Exemplo do uso de \begin{ttfamily}TProgressDlg{\_}If\end{ttfamily}

\texttt{\\\nopagebreak[3]
\\\nopagebreak[3]
}\textbf{Var}\texttt{\\\nopagebreak[3]
~~ProgressDlg{\_}If~:~TProgressDlg{\_}If;\\\nopagebreak[3]
}\textbf{Begin}\texttt{\\\nopagebreak[3]
~~ProgressDlg{\_}If~:=~TProgressDlg{\_}If.Create('Pesquisando~registro',Alias,20);\\\nopagebreak[3]
\\\nopagebreak[3]
~~}\textbf{Repeat}\texttt{\\\nopagebreak[3]
~~~~ProgressDlg{\_}If.IncPosition;\\\nopagebreak[3]
~~}\textbf{Until}\texttt{~}\textbf{Not}\texttt{~next;\\\nopagebreak[3]
\\\nopagebreak[3]
~~ProgressDlg{\_}If.Free;\\\nopagebreak[3]
}\textbf{end}\texttt{.\\
}
\end{itemize}
\end{itemize}\subsubsection*{\large{\textbf{Propriedades}}\normalsize\hspace{1ex}\hfill}
\paragraph*{Total}\hspace*{\fill}

\begin{list}{}{
\settowidth{\tmplength}{\textbf{Declaração}}
\setlength{\itemindent}{0cm}
\setlength{\listparindent}{0cm}
\setlength{\leftmargin}{\evensidemargin}
\addtolength{\leftmargin}{\tmplength}
\settowidth{\labelsep}{X}
\addtolength{\leftmargin}{\labelsep}
\setlength{\labelwidth}{\tmplength}
}
\begin{flushleft}
\item[\textbf{Declaração}\hfill]
\begin{ttfamily}
published property Total: Longint Read {\_}Total   Write  Set{\_}Total;\end{ttfamily}


\end{flushleft}
\par
\item[\textbf{Descrição}]
A propriedade \textbf{\begin{ttfamily}Total\end{ttfamily}} é o \begin{ttfamily}total\end{ttfamily} de elementos previstos na lista ao inicial o dialogo

\end{list}
\paragraph*{Delta}\hspace*{\fill}

\begin{list}{}{
\settowidth{\tmplength}{\textbf{Declaração}}
\setlength{\itemindent}{0cm}
\setlength{\listparindent}{0cm}
\setlength{\leftmargin}{\evensidemargin}
\addtolength{\leftmargin}{\tmplength}
\settowidth{\labelsep}{X}
\addtolength{\leftmargin}{\labelsep}
\setlength{\labelwidth}{\tmplength}
}
\begin{flushleft}
\item[\textbf{Declaração}\hfill]
\begin{ttfamily}
published property Delta: Longint Read {\_}Delta   Write  Set{\_}Delta;\end{ttfamily}


\end{flushleft}
\par
\item[\textbf{Descrição}]
A propriedade \textbf{\begin{ttfamily}Delta\end{ttfamily}} informado ao dialogo o intervalo no qual o dialogo precisa ser criado para que o usuário veja a previsão de termino.

\end{list}
\paragraph*{Limit}\hspace*{\fill}

\begin{list}{}{
\settowidth{\tmplength}{\textbf{Declaração}}
\setlength{\itemindent}{0cm}
\setlength{\listparindent}{0cm}
\setlength{\leftmargin}{\evensidemargin}
\addtolength{\leftmargin}{\tmplength}
\settowidth{\labelsep}{X}
\addtolength{\leftmargin}{\labelsep}
\setlength{\labelwidth}{\tmplength}
}
\begin{flushleft}
\item[\textbf{Declaração}\hfill]
\begin{ttfamily}
published property Limit: Longint Read {\_}Limit   Write  Set{\_}Limit;\end{ttfamily}


\end{flushleft}
\par
\item[\textbf{Descrição}]
A propriedade \textbf{\begin{ttfamily}Limit\end{ttfamily}} é o numero de linhas do controle que está sendo visualizado.

\end{list}
\paragraph*{Title}\hspace*{\fill}

\begin{list}{}{
\settowidth{\tmplength}{\textbf{Declaração}}
\setlength{\itemindent}{0cm}
\setlength{\listparindent}{0cm}
\setlength{\leftmargin}{\evensidemargin}
\addtolength{\leftmargin}{\tmplength}
\settowidth{\labelsep}{X}
\addtolength{\leftmargin}{\labelsep}
\setlength{\labelwidth}{\tmplength}
}
\begin{flushleft}
\item[\textbf{Declaração}\hfill]
\begin{ttfamily}
published property Title: AnsiString Read {\_}Title   Write  Set{\_}Title;\end{ttfamily}


\end{flushleft}
\par
\item[\textbf{Descrição}]
A propriedade \textbf{\begin{ttfamily}Title\end{ttfamily}} é usado no título do dialogo indicando a tarefa que está sendo executada.

\end{list}
\paragraph*{observation}\hspace*{\fill}

\begin{list}{}{
\settowidth{\tmplength}{\textbf{Declaração}}
\setlength{\itemindent}{0cm}
\setlength{\listparindent}{0cm}
\setlength{\leftmargin}{\evensidemargin}
\addtolength{\leftmargin}{\tmplength}
\settowidth{\labelsep}{X}
\addtolength{\leftmargin}{\labelsep}
\setlength{\labelwidth}{\tmplength}
}
\begin{flushleft}
\item[\textbf{Declaração}\hfill]
\begin{ttfamily}
published property observation: AnsiString Read {\_}observation   Write  Set{\_}observation;\end{ttfamily}


\end{flushleft}
\par
\item[\textbf{Descrição}]
A propriedade \textbf{\begin{ttfamily}observation\end{ttfamily}} é usado na barra de status do dialogo indicando qual o atalho aborta a operação

\end{list}
\paragraph*{onCreate{\_}ProgressDlg}\hspace*{\fill}

\begin{list}{}{
\settowidth{\tmplength}{\textbf{Declaração}}
\setlength{\itemindent}{0cm}
\setlength{\listparindent}{0cm}
\setlength{\leftmargin}{\evensidemargin}
\addtolength{\leftmargin}{\tmplength}
\settowidth{\labelsep}{X}
\addtolength{\leftmargin}{\labelsep}
\setlength{\labelwidth}{\tmplength}
}
\begin{flushleft}
\item[\textbf{Declaração}\hfill]
\begin{ttfamily}
published property onCreate{\_}ProgressDlg: TCreate{\_}ProgressDlg Read {\_}onCreate{\_}ProgressDlg   Write  {\_}onCreate{\_}ProgressDlg;\end{ttfamily}


\end{flushleft}
\par
\item[\textbf{Descrição}]
A propriedade \textbf{\begin{ttfamily}onCreate{\_}ProgressDlg\end{ttfamily}} deve ser implementado no pacote visual ou seja: Na interface do usuário que pode ser \textbf{LCL}, Javascript, tv32 etc..

\end{list}
\paragraph*{onIncPosition{\_}01}\hspace*{\fill}

\begin{list}{}{
\settowidth{\tmplength}{\textbf{Declaração}}
\setlength{\itemindent}{0cm}
\setlength{\listparindent}{0cm}
\setlength{\leftmargin}{\evensidemargin}
\addtolength{\leftmargin}{\tmplength}
\settowidth{\labelsep}{X}
\addtolength{\leftmargin}{\labelsep}
\setlength{\labelwidth}{\tmplength}
}
\begin{flushleft}
\item[\textbf{Declaração}\hfill]
\begin{ttfamily}
published property onIncPosition{\_}01: TIncPosition{\_}01 Read {\_}onIncPosition{\_}01   Write  {\_}onIncPosition{\_}01;\end{ttfamily}


\end{flushleft}
\par
\item[\textbf{Descrição}]
A propriedade \textbf{\begin{ttfamily}onIncPosition{\_}01\end{ttfamily}} deve ser implementada na classe visual para incrementar aDelta na posição atual do processamento.

\end{list}
\paragraph*{onIncPosition}\hspace*{\fill}

\begin{list}{}{
\settowidth{\tmplength}{\textbf{Declaração}}
\setlength{\itemindent}{0cm}
\setlength{\listparindent}{0cm}
\setlength{\leftmargin}{\evensidemargin}
\addtolength{\leftmargin}{\tmplength}
\settowidth{\labelsep}{X}
\addtolength{\leftmargin}{\labelsep}
\setlength{\labelwidth}{\tmplength}
}
\begin{flushleft}
\item[\textbf{Declaração}\hfill]
\begin{ttfamily}
published property onIncPosition: TIncPosition Read {\_}onIncPosition   Write  {\_}onIncPosition;\end{ttfamily}


\end{flushleft}
\par
\item[\textbf{Descrição}]
A propriedade \textbf{\begin{ttfamily}onIncPosition\end{ttfamily}} deve ser implementada na classe visual para incrementar 1 na posição atual do processamento.

\end{list}
\paragraph*{OnRedraw}\hspace*{\fill}

\begin{list}{}{
\settowidth{\tmplength}{\textbf{Declaração}}
\setlength{\itemindent}{0cm}
\setlength{\listparindent}{0cm}
\setlength{\leftmargin}{\evensidemargin}
\addtolength{\leftmargin}{\tmplength}
\settowidth{\labelsep}{X}
\addtolength{\leftmargin}{\labelsep}
\setlength{\labelwidth}{\tmplength}
}
\begin{flushleft}
\item[\textbf{Declaração}\hfill]
\begin{ttfamily}
published property OnRedraw: TRedraw Read {\_}OnRedraw   Write  {\_}OnRedraw;\end{ttfamily}


\end{flushleft}
\par
\item[\textbf{Descrição}]
A propriedade \textbf{\begin{ttfamily}OnRedraw\end{ttfamily}} deve ser implementada na classe visual para atualizar a tela com a posição atual do processamento.

\end{list}
\subsubsection*{\large{\textbf{Campos}}\normalsize\hspace{1ex}\hfill}
\paragraph*{{\_}TimeCurrent}\hspace*{\fill}

\begin{list}{}{
\settowidth{\tmplength}{\textbf{Declaração}}
\setlength{\itemindent}{0cm}
\setlength{\listparindent}{0cm}
\setlength{\leftmargin}{\evensidemargin}
\addtolength{\leftmargin}{\tmplength}
\settowidth{\labelsep}{X}
\addtolength{\leftmargin}{\labelsep}
\setlength{\labelwidth}{\tmplength}
}
\begin{flushleft}
\item[\textbf{Declaração}\hfill]
\begin{ttfamily}
protected Var {\_}TimeCurrent: Longint;\end{ttfamily}


\end{flushleft}
\end{list}
\paragraph*{{\_}TimeBegin}\hspace*{\fill}

\begin{list}{}{
\settowidth{\tmplength}{\textbf{Declaração}}
\setlength{\itemindent}{0cm}
\setlength{\listparindent}{0cm}
\setlength{\leftmargin}{\evensidemargin}
\addtolength{\leftmargin}{\tmplength}
\settowidth{\labelsep}{X}
\addtolength{\leftmargin}{\labelsep}
\setlength{\labelwidth}{\tmplength}
}
\begin{flushleft}
\item[\textbf{Declaração}\hfill]
\begin{ttfamily}
protected Var {\_}TimeBegin: Longint;\end{ttfamily}


\end{flushleft}
\end{list}
\paragraph*{{\_}TimeForeseen}\hspace*{\fill}

\begin{list}{}{
\settowidth{\tmplength}{\textbf{Declaração}}
\setlength{\itemindent}{0cm}
\setlength{\listparindent}{0cm}
\setlength{\leftmargin}{\evensidemargin}
\addtolength{\leftmargin}{\tmplength}
\settowidth{\labelsep}{X}
\addtolength{\leftmargin}{\labelsep}
\setlength{\labelwidth}{\tmplength}
}
\begin{flushleft}
\item[\textbf{Declaração}\hfill]
\begin{ttfamily}
protected Var {\_}TimeForeseen: Longint;\end{ttfamily}


\end{flushleft}
\end{list}
\paragraph*{{\_}Percent}\hspace*{\fill}

\begin{list}{}{
\settowidth{\tmplength}{\textbf{Declaração}}
\setlength{\itemindent}{0cm}
\setlength{\listparindent}{0cm}
\setlength{\leftmargin}{\evensidemargin}
\addtolength{\leftmargin}{\tmplength}
\settowidth{\labelsep}{X}
\addtolength{\leftmargin}{\labelsep}
\setlength{\labelwidth}{\tmplength}
}
\begin{flushleft}
\item[\textbf{Declaração}\hfill]
\begin{ttfamily}
protected Var {\_}Percent: SmallInt;\end{ttfamily}


\end{flushleft}
\end{list}
\subsubsection*{\large{\textbf{Métodos}}\normalsize\hspace{1ex}\hfill}
\paragraph*{Set{\_}Total}\hspace*{\fill}

\begin{list}{}{
\settowidth{\tmplength}{\textbf{Declaração}}
\setlength{\itemindent}{0cm}
\setlength{\listparindent}{0cm}
\setlength{\leftmargin}{\evensidemargin}
\addtolength{\leftmargin}{\tmplength}
\settowidth{\labelsep}{X}
\addtolength{\leftmargin}{\labelsep}
\setlength{\labelwidth}{\tmplength}
}
\begin{flushleft}
\item[\textbf{Declaração}\hfill]
\begin{ttfamily}
protected procedure Set{\_}Total(aTotal: Longint); Virtual;\end{ttfamily}


\end{flushleft}
\end{list}
\paragraph*{Set{\_}Delta}\hspace*{\fill}

\begin{list}{}{
\settowidth{\tmplength}{\textbf{Declaração}}
\setlength{\itemindent}{0cm}
\setlength{\listparindent}{0cm}
\setlength{\leftmargin}{\evensidemargin}
\addtolength{\leftmargin}{\tmplength}
\settowidth{\labelsep}{X}
\addtolength{\leftmargin}{\labelsep}
\setlength{\labelwidth}{\tmplength}
}
\begin{flushleft}
\item[\textbf{Declaração}\hfill]
\begin{ttfamily}
protected procedure Set{\_}Delta(aDelta: Longint); Virtual;\end{ttfamily}


\end{flushleft}
\end{list}
\paragraph*{Set{\_}Limit}\hspace*{\fill}

\begin{list}{}{
\settowidth{\tmplength}{\textbf{Declaração}}
\setlength{\itemindent}{0cm}
\setlength{\listparindent}{0cm}
\setlength{\leftmargin}{\evensidemargin}
\addtolength{\leftmargin}{\tmplength}
\settowidth{\labelsep}{X}
\addtolength{\leftmargin}{\labelsep}
\setlength{\labelwidth}{\tmplength}
}
\begin{flushleft}
\item[\textbf{Declaração}\hfill]
\begin{ttfamily}
protected procedure Set{\_}Limit(aLimit: Longint); Virtual;\end{ttfamily}


\end{flushleft}
\end{list}
\paragraph*{Set{\_}Title}\hspace*{\fill}

\begin{list}{}{
\settowidth{\tmplength}{\textbf{Declaração}}
\setlength{\itemindent}{0cm}
\setlength{\listparindent}{0cm}
\setlength{\leftmargin}{\evensidemargin}
\addtolength{\leftmargin}{\tmplength}
\settowidth{\labelsep}{X}
\addtolength{\leftmargin}{\labelsep}
\setlength{\labelwidth}{\tmplength}
}
\begin{flushleft}
\item[\textbf{Declaração}\hfill]
\begin{ttfamily}
protected procedure Set{\_}Title(aTitle: AnsiString); Virtual;\end{ttfamily}


\end{flushleft}
\end{list}
\paragraph*{Set{\_}observation}\hspace*{\fill}

\begin{list}{}{
\settowidth{\tmplength}{\textbf{Declaração}}
\setlength{\itemindent}{0cm}
\setlength{\listparindent}{0cm}
\setlength{\leftmargin}{\evensidemargin}
\addtolength{\leftmargin}{\tmplength}
\settowidth{\labelsep}{X}
\addtolength{\leftmargin}{\labelsep}
\setlength{\labelwidth}{\tmplength}
}
\begin{flushleft}
\item[\textbf{Declaração}\hfill]
\begin{ttfamily}
protected procedure Set{\_}observation(aobservation: AnsiString); Virtual;\end{ttfamily}


\end{flushleft}
\end{list}
\paragraph*{Create{\_}ProgressDlg}\hspace*{\fill}

\begin{list}{}{
\settowidth{\tmplength}{\textbf{Declaração}}
\setlength{\itemindent}{0cm}
\setlength{\listparindent}{0cm}
\setlength{\leftmargin}{\evensidemargin}
\addtolength{\leftmargin}{\tmplength}
\settowidth{\labelsep}{X}
\addtolength{\leftmargin}{\labelsep}
\setlength{\labelwidth}{\tmplength}
}
\begin{flushleft}
\item[\textbf{Declaração}\hfill]
\begin{ttfamily}
public Procedure Create{\_}ProgressDlg; overload; Virtual;\end{ttfamily}


\end{flushleft}
\par
\item[\textbf{Descrição}]
A procedure \textbf{\begin{ttfamily}Create{\_}ProgressDlg\end{ttfamily}} deve ser anulada para implementar a criação do diálogo no pacote visual ou seja: Na interface do usuário que pode ser \textbf{LCL}, Javascript, tv32 etc..

\end{list}
\paragraph*{RegisterOnEvents}\hspace*{\fill}

\begin{list}{}{
\settowidth{\tmplength}{\textbf{Declaração}}
\setlength{\itemindent}{0cm}
\setlength{\listparindent}{0cm}
\setlength{\leftmargin}{\evensidemargin}
\addtolength{\leftmargin}{\tmplength}
\settowidth{\labelsep}{X}
\addtolength{\leftmargin}{\labelsep}
\setlength{\labelwidth}{\tmplength}
}
\begin{flushleft}
\item[\textbf{Declaração}\hfill]
\begin{ttfamily}
protected Procedure RegisterOnEvents; Virtual;\end{ttfamily}


\end{flushleft}
\par
\item[\textbf{Descrição}]
A procedure \textbf{\begin{ttfamily}RegisterOnEvents\end{ttfamily}} deve ser anulada para implementar os eventos desta classe caso a mesma não esteja registrada na IDE

\end{list}
\paragraph*{Create}\hspace*{\fill}

\begin{list}{}{
\settowidth{\tmplength}{\textbf{Declaração}}
\setlength{\itemindent}{0cm}
\setlength{\listparindent}{0cm}
\setlength{\leftmargin}{\evensidemargin}
\addtolength{\leftmargin}{\tmplength}
\settowidth{\labelsep}{X}
\addtolength{\leftmargin}{\labelsep}
\setlength{\labelwidth}{\tmplength}
}
\begin{flushleft}
\item[\textbf{Declaração}\hfill]
\begin{ttfamily}
public constructor Create(AOwner: TComponent); Overload; override;\end{ttfamily}


\end{flushleft}
\par
\item[\textbf{Descrição}]
O constructor \textbf{\begin{ttfamily}Create\end{ttfamily}} é necessário porque essa classe pode ser registrada da IDE

\end{list}
\paragraph*{Create}\hspace*{\fill}

\begin{list}{}{
\settowidth{\tmplength}{\textbf{Declaração}}
\setlength{\itemindent}{0cm}
\setlength{\listparindent}{0cm}
\setlength{\leftmargin}{\evensidemargin}
\addtolength{\leftmargin}{\tmplength}
\settowidth{\labelsep}{X}
\addtolength{\leftmargin}{\labelsep}
\setlength{\labelwidth}{\tmplength}
}
\begin{flushleft}
\item[\textbf{Declaração}\hfill]
\begin{ttfamily}
public constructor Create(aTitle : AnsiString; aobservation : AnsiString; aDelta, aTotal : longint ); Overload; Virtual;\end{ttfamily}


\end{flushleft}
\par
\item[\textbf{Descrição}]
O constructor \textbf{\begin{ttfamily}Create\end{ttfamily}} é usado para criar a classe sem a IDE

\end{list}
\paragraph*{Destroy}\hspace*{\fill}

\begin{list}{}{
\settowidth{\tmplength}{\textbf{Declaração}}
\setlength{\itemindent}{0cm}
\setlength{\listparindent}{0cm}
\setlength{\leftmargin}{\evensidemargin}
\addtolength{\leftmargin}{\tmplength}
\settowidth{\labelsep}{X}
\addtolength{\leftmargin}{\labelsep}
\setlength{\labelwidth}{\tmplength}
}
\begin{flushleft}
\item[\textbf{Declaração}\hfill]
\begin{ttfamily}
public Destructor Destroy; Override;\end{ttfamily}


\end{flushleft}
\par
\item[\textbf{Descrição}]
O destructor \textbf{\begin{ttfamily}Destroy\end{ttfamily}} é usado para destruir a classe.

\end{list}
\paragraph*{IncPosition}\hspace*{\fill}

\begin{list}{}{
\settowidth{\tmplength}{\textbf{Declaração}}
\setlength{\itemindent}{0cm}
\setlength{\listparindent}{0cm}
\setlength{\leftmargin}{\evensidemargin}
\addtolength{\leftmargin}{\tmplength}
\settowidth{\labelsep}{X}
\addtolength{\leftmargin}{\labelsep}
\setlength{\labelwidth}{\tmplength}
}
\begin{flushleft}
\item[\textbf{Declaração}\hfill]
\begin{ttfamily}
public procedure IncPosition(Const aDelta :longint); Overload; Virtual;\end{ttfamily}


\end{flushleft}
\par
\item[\textbf{Descrição}]
A propriedade \textbf{\begin{ttfamily}IncPosition\end{ttfamily}} deve ser anulada na classe visual para incrementar \textbf{aDelta} na posição atual do processamento.

\end{list}
\paragraph*{IncPosition}\hspace*{\fill}

\begin{list}{}{
\settowidth{\tmplength}{\textbf{Declaração}}
\setlength{\itemindent}{0cm}
\setlength{\listparindent}{0cm}
\setlength{\leftmargin}{\evensidemargin}
\addtolength{\leftmargin}{\tmplength}
\settowidth{\labelsep}{X}
\addtolength{\leftmargin}{\labelsep}
\setlength{\labelwidth}{\tmplength}
}
\begin{flushleft}
\item[\textbf{Declaração}\hfill]
\begin{ttfamily}
public procedure IncPosition; Overload; Virtual;\end{ttfamily}


\end{flushleft}
\par
\item[\textbf{Descrição}]
A propriedade \textbf{\begin{ttfamily}IncPosition\end{ttfamily}} deve ser anulada na classe visual para incrementar 1 na posição atual do processamento.

\end{list}
\paragraph*{Redraw}\hspace*{\fill}

\begin{list}{}{
\settowidth{\tmplength}{\textbf{Declaração}}
\setlength{\itemindent}{0cm}
\setlength{\listparindent}{0cm}
\setlength{\leftmargin}{\evensidemargin}
\addtolength{\leftmargin}{\tmplength}
\settowidth{\labelsep}{X}
\addtolength{\leftmargin}{\labelsep}
\setlength{\labelwidth}{\tmplength}
}
\begin{flushleft}
\item[\textbf{Declaração}\hfill]
\begin{ttfamily}
protected Procedure Redraw; Virtual;\end{ttfamily}


\end{flushleft}
\par
\item[\textbf{Descrição}]
A propriedade \textbf{\begin{ttfamily}Redraw\end{ttfamily}} deve ser anulada para implementar na classe visual para atualizar a tela com a posição atual do processamento.

\end{list}
\paragraph*{SetPerc}\hspace*{\fill}

\begin{list}{}{
\settowidth{\tmplength}{\textbf{Declaração}}
\setlength{\itemindent}{0cm}
\setlength{\listparindent}{0cm}
\setlength{\leftmargin}{\evensidemargin}
\addtolength{\leftmargin}{\tmplength}
\settowidth{\labelsep}{X}
\addtolength{\leftmargin}{\labelsep}
\setlength{\labelwidth}{\tmplength}
}
\begin{flushleft}
\item[\textbf{Declaração}\hfill]
\begin{ttfamily}
public procedure SetPerc(const aPosition : longint);\end{ttfamily}


\end{flushleft}
\par
\item[\textbf{Descrição}]
A procedure \textbf{\begin{ttfamily}SetPerc\end{ttfamily}} é usado para informar ao dialogo a posição atual da contagem.

\begin{itemize}
\item \textbf{NOTA} \begin{itemize}
\item Calcula o percentual atual do processamento.
\end{itemize}
\end{itemize}

\end{list}
\section{Funções e Procedimentos}
\subsection*{Register}
\begin{list}{}{
\settowidth{\tmplength}{\textbf{Declaração}}
\setlength{\itemindent}{0cm}
\setlength{\listparindent}{0cm}
\setlength{\leftmargin}{\evensidemargin}
\addtolength{\leftmargin}{\tmplength}
\settowidth{\labelsep}{X}
\addtolength{\leftmargin}{\labelsep}
\setlength{\labelwidth}{\tmplength}
}
\begin{flushleft}
\item[\textbf{Declaração}\hfill]
\begin{ttfamily}
procedure Register;\end{ttfamily}


\end{flushleft}
\end{list}
\section{Tipos}
\subsection*{TProgressDlg{\_}If{\_}Class}
\begin{list}{}{
\settowidth{\tmplength}{\textbf{Declaração}}
\setlength{\itemindent}{0cm}
\setlength{\listparindent}{0cm}
\setlength{\leftmargin}{\evensidemargin}
\addtolength{\leftmargin}{\tmplength}
\settowidth{\labelsep}{X}
\addtolength{\leftmargin}{\labelsep}
\setlength{\labelwidth}{\tmplength}
}
\begin{flushleft}
\item[\textbf{Declaração}\hfill]
\begin{ttfamily}
TProgressDlg{\_}If{\_}Class = Class of TProgressDlg{\_}If;\end{ttfamily}


\end{flushleft}
\end{list}
\chapter{Unit mi.rtl.objects.consts.strings}
\section{Uses}
\begin{itemize}
\item \begin{ttfamily}Classes\end{ttfamily}\item \begin{ttfamily}SysUtils\end{ttfamily}\end{itemize}
\chapter{Unit mi.rtl.Objects.Methods}
\section{Descrição}
\begin{itemize}
\item A Unit \textbf{\begin{ttfamily}mi.rtl.Objects.Methods\end{ttfamily}} implementa a classe \textbf{\begin{ttfamily}TObjectsMethods\end{ttfamily}(\ref{mi.rtl.Objects.Methods.TObjectsMethods})} do pacote \textbf{\begin{ttfamily}mi.rtl\end{ttfamily}(\ref{mi.rtl})}.

\begin{itemize}
\item \textbf{NOTAS} \begin{itemize}
\item Esta unit foi testada nas plataformas: win32, win64 e linux.
\end{itemize}
\item \textbf{VERSÃO} \begin{itemize}
\item Alpha {-} 0.5.0.687
\end{itemize}
\item \textbf{HISTÓRICO} \begin{itemize}
\item Criado por: Paulo Sérgio da Silva Pacheco e{-}mail: paulosspacheco@yahoo.com.br \begin{itemize}
\item \textbf{19/11/2021} 21:25 a 23:15 Criar a unit mi.rtl.objects.TObjectsMethods.pas
\item \textbf{20/11/2021} 18:00 a 19:15 Criar a unit mi.rtl.objects.tStream.pas
\end{itemize}
\end{itemize}
\item \textbf{CÓDIGO FONTE}: \begin{itemize}
\item 
\end{itemize}
\end{itemize}
\end{itemize}
\section{Uses}
\begin{itemize}
\item \begin{ttfamily}Classes\end{ttfamily}\item \begin{ttfamily}SysUtils\end{ttfamily}\item \begin{ttfamily}System.UITypes\end{ttfamily}\item \begin{ttfamily}Process\end{ttfamily}\item \begin{ttfamily}dos\end{ttfamily}\item \begin{ttfamily}sqlDB\end{ttfamily}\item \begin{ttfamily}SQLite3Conn\end{ttfamily}\item \begin{ttfamily}PQConnection\end{ttfamily}\item \begin{ttfamily}FpHttpClient\end{ttfamily}\item \begin{ttfamily}mi.rtl.Class{\_}Of{\_}Char\end{ttfamily}(\ref{mi.rtl.Class_Of_Char})\item \begin{ttfamily}mi.rtl.types\end{ttfamily}(\ref{mi.rtl.Types})\item \begin{ttfamily}mi.rtl.Consts\end{ttfamily}(\ref{mi.rtl.Consts})\item \begin{ttfamily}mi.rtl.objects.consts\end{ttfamily}(\ref{mi.rtl.Objects.Consts})\item \begin{ttfamily}mi.rtl.objects.consts.MI{\_}MsgBox\end{ttfamily}\item \begin{ttfamily}mi.rtl.objects.consts.progressdlg{\_}if\end{ttfamily}(\ref{mi.rtl.Objects.Consts.ProgressDlg_If})\item \begin{ttfamily}mi.rtl.objects.Consts.logs\end{ttfamily}(\ref{mi.rtl.Objects.Consts.Logs})\item \begin{ttfamily}mi.rtl.consts.StringList\end{ttfamily}(\ref{mi.rtl.Consts.StringList})\end{itemize}
\section{Visão Geral}
\begin{description}
\item[\texttt{\begin{ttfamily}TObjectsMethods\end{ttfamily} Classe}]
\end{description}
\section{Classes, Interfaces, Objetos e Registros}
\subsection*{TObjectsMethods Classe}
\subsubsection*{\large{\textbf{Hierarquia}}\normalsize\hspace{1ex}\hfill}
TObjectsMethods {$>$} \begin{ttfamily}TObjectsConsts\end{ttfamily}(\ref{mi.rtl.Objects.Consts.TObjectsConsts}) {$>$} 
TObjectsTypes
\subsubsection*{\large{\textbf{Descrição}}\normalsize\hspace{1ex}\hfill}
\begin{itemize}
\item A classe \textbf{\begin{ttfamily}TObjectsMethods\end{ttfamily}} implementa os método de classe comum a todas as classes de TObjects do pacote \textbf{\begin{ttfamily}mi.rtl\end{ttfamily}(\ref{mi.rtl})}.
\end{itemize}\subsubsection*{\large{\textbf{Propriedades}}\normalsize\hspace{1ex}\hfill}
\paragraph*{Alias}\hspace*{\fill}

\begin{list}{}{
\settowidth{\tmplength}{\textbf{Declaração}}
\setlength{\itemindent}{0cm}
\setlength{\listparindent}{0cm}
\setlength{\leftmargin}{\evensidemargin}
\addtolength{\leftmargin}{\tmplength}
\settowidth{\labelsep}{X}
\addtolength{\leftmargin}{\labelsep}
\setlength{\labelwidth}{\tmplength}
}
\begin{flushleft}
\item[\textbf{Declaração}\hfill]
\begin{ttfamily}
public property Alias: AnsiString Read {\_}Alias Write {\_}Alias;\end{ttfamily}


\end{flushleft}
\end{list}
\subsubsection*{\large{\textbf{Campos}}\normalsize\hspace{1ex}\hfill}
\paragraph*{MI{\_}MsgBox}\hspace*{\fill}

\begin{list}{}{
\settowidth{\tmplength}{\textbf{Declaração}}
\setlength{\itemindent}{0cm}
\setlength{\listparindent}{0cm}
\setlength{\leftmargin}{\evensidemargin}
\addtolength{\leftmargin}{\tmplength}
\settowidth{\labelsep}{X}
\addtolength{\leftmargin}{\labelsep}
\setlength{\labelwidth}{\tmplength}
}
\begin{flushleft}
\item[\textbf{Declaração}\hfill]
\begin{ttfamily}
public const MI{\_}MsgBox: TMI{\_}MsgBox = nil;\end{ttfamily}


\end{flushleft}
\end{list}
\paragraph*{{\_}Logs}\hspace*{\fill}

\begin{list}{}{
\settowidth{\tmplength}{\textbf{Declaração}}
\setlength{\itemindent}{0cm}
\setlength{\listparindent}{0cm}
\setlength{\leftmargin}{\evensidemargin}
\addtolength{\leftmargin}{\tmplength}
\settowidth{\labelsep}{X}
\addtolength{\leftmargin}{\labelsep}
\setlength{\labelwidth}{\tmplength}
}
\begin{flushleft}
\item[\textbf{Declaração}\hfill]
\begin{ttfamily}
public const {\_}Logs : TFilesLogs = nil;\end{ttfamily}


\end{flushleft}
\par
\item[\textbf{Descrição}]
\begin{itemize}
\item \begin{ttfamily}Logs\end{ttfamily}(\ref{mi.rtl.Objects.Methods.TObjectsMethods-Logs}) é inicializado em Initialization e destruído em finalization
\end{itemize}

\end{list}
\subsubsection*{\large{\textbf{Métodos}}\normalsize\hspace{1ex}\hfill}
\paragraph*{Logs}\hspace*{\fill}

\begin{list}{}{
\settowidth{\tmplength}{\textbf{Declaração}}
\setlength{\itemindent}{0cm}
\setlength{\listparindent}{0cm}
\setlength{\leftmargin}{\evensidemargin}
\addtolength{\leftmargin}{\tmplength}
\settowidth{\labelsep}{X}
\addtolength{\leftmargin}{\labelsep}
\setlength{\labelwidth}{\tmplength}
}
\begin{flushleft}
\item[\textbf{Declaração}\hfill]
\begin{ttfamily}
public class function Logs: TFilesLogs;\end{ttfamily}


\end{flushleft}
\end{list}
\paragraph*{SysMessageBox}\hspace*{\fill}

\begin{list}{}{
\settowidth{\tmplength}{\textbf{Declaração}}
\setlength{\itemindent}{0cm}
\setlength{\listparindent}{0cm}
\setlength{\leftmargin}{\evensidemargin}
\addtolength{\leftmargin}{\tmplength}
\settowidth{\labelsep}{X}
\addtolength{\leftmargin}{\labelsep}
\setlength{\labelwidth}{\tmplength}
}
\begin{flushleft}
\item[\textbf{Declaração}\hfill]
\begin{ttfamily}
public class procedure SysMessageBox(Msg, Title: AnsiString; Error: Boolean);\end{ttfamily}


\end{flushleft}
\par
\item[\textbf{Descrição}]
\begin{itemize}
\item O método de classe \begin{ttfamily}SysMessageBox\end{ttfamily} executa o diálogo \begin{ttfamily}MessageBox\end{ttfamily}(\ref{mi.rtl.Objects.Methods.TObjectsMethods-MessageBox}) do Lazarus;

\begin{itemize}
\item \textbf{REFERÊNCIA} \begin{itemize}
\item https://wiki.lazarus.freepascal.org/Dialog{\_}Examples{\#}MessageBox
\end{itemize}
\end{itemize}
\end{itemize}

\end{list}
\paragraph*{MessageBox}\hspace*{\fill}

\begin{list}{}{
\settowidth{\tmplength}{\textbf{Declaração}}
\setlength{\itemindent}{0cm}
\setlength{\listparindent}{0cm}
\setlength{\leftmargin}{\evensidemargin}
\addtolength{\leftmargin}{\tmplength}
\settowidth{\labelsep}{X}
\addtolength{\leftmargin}{\labelsep}
\setlength{\labelwidth}{\tmplength}
}
\begin{flushleft}
\item[\textbf{Declaração}\hfill]
\begin{ttfamily}
public class function MessageBox(const Msg: AnsiString): Word; Virtual; overload;\end{ttfamily}


\end{flushleft}
\par
\item[\textbf{Descrição}]
A classe \textbf{\begin{ttfamily}MessageBox\end{ttfamily}} deve ser implementada no pacote mi.ui.

\end{list}
\paragraph*{MessageBox}\hspace*{\fill}

\begin{list}{}{
\settowidth{\tmplength}{\textbf{Declaração}}
\setlength{\itemindent}{0cm}
\setlength{\listparindent}{0cm}
\setlength{\leftmargin}{\evensidemargin}
\addtolength{\leftmargin}{\tmplength}
\settowidth{\labelsep}{X}
\addtolength{\leftmargin}{\labelsep}
\setlength{\labelwidth}{\tmplength}
}
\begin{flushleft}
\item[\textbf{Declaração}\hfill]
\begin{ttfamily}
public class function MessageBox(const aMsg: AnsiString; DlgType: TMsgDlgType; Buttons: TMsgDlgButtons): TModalResult; Virtual; overload;\end{ttfamily}


\end{flushleft}
\par
\item[\textbf{Descrição}]
O método \textbf{\begin{ttfamily}MessageBox\end{ttfamily}} recebe 3 parâmetros. Criar um dialogo e retrona as opções escolhidas.

\begin{itemize}
\item \textbf{Exemplo de uso}

\texttt{\\\nopagebreak[3]
\\\nopagebreak[3]
}\textbf{If}\texttt{~MessageBox('O~arquivo~'+TMI{\_}DataFile(DatF).nomeArq+'~não~existe.'+{\^{}}M+\\\nopagebreak[3]
~~~~~~~~~~~~~~{\^{}}M+\\\nopagebreak[3]
~~~~~~~~~~~~~~'Cria~o~arquivo~agora?'\\\nopagebreak[3]
~~~~~~~~~~~~~~,MtConfirmation,mbYesNoCancel,mbYes)=~MrYes\\\nopagebreak[3]
}\textbf{Then}\texttt{~}\textbf{begin}\texttt{\\\nopagebreak[3]
~~~~~}\textbf{end}\texttt{;\\
}
\end{itemize}

\end{list}
\paragraph*{Abstracts}\hspace*{\fill}

\begin{list}{}{
\settowidth{\tmplength}{\textbf{Declaração}}
\setlength{\itemindent}{0cm}
\setlength{\listparindent}{0cm}
\setlength{\leftmargin}{\evensidemargin}
\addtolength{\leftmargin}{\tmplength}
\settowidth{\labelsep}{X}
\addtolength{\leftmargin}{\labelsep}
\setlength{\labelwidth}{\tmplength}
}
\begin{flushleft}
\item[\textbf{Declaração}\hfill]
\begin{ttfamily}
public class procedure Abstracts;\end{ttfamily}


\end{flushleft}
\par
\item[\textbf{Descrição}]
\begin{itemize}
\item A classe método \textbf{\begin{ttfamily}Abstracts\end{ttfamily}} encerra o programa com um erro de tempo de execução 211. \begin{itemize}
\item \textbf{NOTA} \begin{itemize}
\item Ao implementar um tipo de classe abstrato, chame Abstract nesses métodos Override que deve ser substituído em tipos descendentes. Isso garante que qualquer tentativa de usar instâncias do tipo abstrato de classe falhará.
\end{itemize}
\end{itemize}
\end{itemize}

\end{list}
\paragraph*{RegisterError}\hspace*{\fill}

\begin{list}{}{
\settowidth{\tmplength}{\textbf{Declaração}}
\setlength{\itemindent}{0cm}
\setlength{\listparindent}{0cm}
\setlength{\leftmargin}{\evensidemargin}
\addtolength{\leftmargin}{\tmplength}
\settowidth{\labelsep}{X}
\addtolength{\leftmargin}{\labelsep}
\setlength{\labelwidth}{\tmplength}
}
\begin{flushleft}
\item[\textbf{Declaração}\hfill]
\begin{ttfamily}
public Class procedure RegisterError;\end{ttfamily}


\end{flushleft}
\end{list}
\paragraph*{RegisterType}\hspace*{\fill}

\begin{list}{}{
\settowidth{\tmplength}{\textbf{Declaração}}
\setlength{\itemindent}{0cm}
\setlength{\listparindent}{0cm}
\setlength{\leftmargin}{\evensidemargin}
\addtolength{\leftmargin}{\tmplength}
\settowidth{\labelsep}{X}
\addtolength{\leftmargin}{\labelsep}
\setlength{\labelwidth}{\tmplength}
}
\begin{flushleft}
\item[\textbf{Declaração}\hfill]
\begin{ttfamily}
public class procedure RegisterType(Var S: TStreamRec);\end{ttfamily}


\end{flushleft}
\par
\item[\textbf{Descrição}]
\begin{itemize}
\item A classe método \textbf{\begin{ttfamily}RegisterType\end{ttfamily}} registra o tipo de classe fornecido com os fluxos do Free Vision, criando uma lista de objetos conhecidos. Streams só podem armazenar e retornar esses Tipos de classe. \begin{itemize}
\item Cada classe registrada precisa de um registro de stream único registro, do tipo TStreamRec.
\end{itemize}
\end{itemize}

\end{list}
\paragraph*{LongMul}\hspace*{\fill}

\begin{list}{}{
\settowidth{\tmplength}{\textbf{Declaração}}
\setlength{\itemindent}{0cm}
\setlength{\listparindent}{0cm}
\setlength{\leftmargin}{\evensidemargin}
\addtolength{\leftmargin}{\tmplength}
\settowidth{\labelsep}{X}
\addtolength{\leftmargin}{\labelsep}
\setlength{\labelwidth}{\tmplength}
}
\begin{flushleft}
\item[\textbf{Declaração}\hfill]
\begin{ttfamily}
public class function LongMul(X, Y: Integer): LongInt;\end{ttfamily}


\end{flushleft}
\par
\item[\textbf{Descrição}]
\begin{itemize}
\item A class function \begin{ttfamily}LongMul\end{ttfamily} retorna o valor inteiro longo de valores inteiros X * Y.
\end{itemize}

\end{list}
\paragraph*{LongDiv}\hspace*{\fill}

\begin{list}{}{
\settowidth{\tmplength}{\textbf{Declaração}}
\setlength{\itemindent}{0cm}
\setlength{\listparindent}{0cm}
\setlength{\leftmargin}{\evensidemargin}
\addtolength{\leftmargin}{\tmplength}
\settowidth{\labelsep}{X}
\addtolength{\leftmargin}{\labelsep}
\setlength{\labelwidth}{\tmplength}
}
\begin{flushleft}
\item[\textbf{Declaração}\hfill]
\begin{ttfamily}
public class function LongDiv(X: LongInt; Y: Integer): Integer;\end{ttfamily}


\end{flushleft}
\par
\item[\textbf{Descrição}]
A classe function \textbf{\begin{ttfamily}LongDiv\end{ttfamily}} retorna o valor inteiro do inteiro longo X dividido pelo inteiro Y.

\end{list}
\paragraph*{NNewStr}\hspace*{\fill}

\begin{list}{}{
\settowidth{\tmplength}{\textbf{Declaração}}
\setlength{\itemindent}{0cm}
\setlength{\listparindent}{0cm}
\setlength{\leftmargin}{\evensidemargin}
\addtolength{\leftmargin}{\tmplength}
\settowidth{\labelsep}{X}
\addtolength{\leftmargin}{\labelsep}
\setlength{\labelwidth}{\tmplength}
}
\begin{flushleft}
\item[\textbf{Declaração}\hfill]
\begin{ttfamily}
public class procedure NNewStr(Var PS : ptstring;Const S : AnsiString);\end{ttfamily}


\end{flushleft}
\par
\item[\textbf{Descrição}]
{-}

\end{list}
\paragraph*{CallPointerLocal}\hspace*{\fill}

\begin{list}{}{
\settowidth{\tmplength}{\textbf{Declaração}}
\setlength{\itemindent}{0cm}
\setlength{\listparindent}{0cm}
\setlength{\leftmargin}{\evensidemargin}
\addtolength{\leftmargin}{\tmplength}
\settowidth{\labelsep}{X}
\addtolength{\leftmargin}{\labelsep}
\setlength{\labelwidth}{\tmplength}
}
\begin{flushleft}
\item[\textbf{Declaração}\hfill]
\begin{ttfamily}
public class function CallPointerLocal(Func: codepointer; Frame: Pointer; Param1: pointer): pointer; inline;\end{ttfamily}


\end{flushleft}
\end{list}
\paragraph*{DisposeStr}\hspace*{\fill}

\begin{list}{}{
\settowidth{\tmplength}{\textbf{Declaração}}
\setlength{\itemindent}{0cm}
\setlength{\listparindent}{0cm}
\setlength{\leftmargin}{\evensidemargin}
\addtolength{\leftmargin}{\tmplength}
\settowidth{\labelsep}{X}
\addtolength{\leftmargin}{\labelsep}
\setlength{\labelwidth}{\tmplength}
}
\begin{flushleft}
\item[\textbf{Declaração}\hfill]
\begin{ttfamily}
public class PROCEDURE DisposeStr( Var P: ptstring);\end{ttfamily}


\end{flushleft}
\end{list}
\paragraph*{IsValidPtr}\hspace*{\fill}

\begin{list}{}{
\settowidth{\tmplength}{\textbf{Declaração}}
\setlength{\itemindent}{0cm}
\setlength{\listparindent}{0cm}
\setlength{\leftmargin}{\evensidemargin}
\addtolength{\leftmargin}{\tmplength}
\settowidth{\labelsep}{X}
\addtolength{\leftmargin}{\labelsep}
\setlength{\labelwidth}{\tmplength}
}
\begin{flushleft}
\item[\textbf{Declaração}\hfill]
\begin{ttfamily}
public class FUNCTION IsValidPtr( ADDR:POINTER):BOOLEAN ; overload;\end{ttfamily}


\end{flushleft}
\end{list}
\paragraph*{IsValidPtr}\hspace*{\fill}

\begin{list}{}{
\settowidth{\tmplength}{\textbf{Declaração}}
\setlength{\itemindent}{0cm}
\setlength{\listparindent}{0cm}
\setlength{\leftmargin}{\evensidemargin}
\addtolength{\leftmargin}{\tmplength}
\settowidth{\labelsep}{X}
\addtolength{\leftmargin}{\labelsep}
\setlength{\labelwidth}{\tmplength}
}
\begin{flushleft}
\item[\textbf{Declaração}\hfill]
\begin{ttfamily}
public class FUNCTION IsValidPtr( const aClass: Tobject):BOOLEAN ; overload;\end{ttfamily}


\end{flushleft}
\end{list}
\paragraph*{Name{\_}Type{\_}App{\_}MarIcaraiV1}\hspace*{\fill}

\begin{list}{}{
\settowidth{\tmplength}{\textbf{Declaração}}
\setlength{\itemindent}{0cm}
\setlength{\listparindent}{0cm}
\setlength{\leftmargin}{\evensidemargin}
\addtolength{\leftmargin}{\tmplength}
\settowidth{\labelsep}{X}
\addtolength{\leftmargin}{\labelsep}
\setlength{\labelwidth}{\tmplength}
}
\begin{flushleft}
\item[\textbf{Declaração}\hfill]
\begin{ttfamily}
public class Function Name{\_}Type{\_}App{\_}MarIcaraiV1:AnsiString;\end{ttfamily}


\end{flushleft}
\end{list}
\paragraph*{Set{\_}IsApp{\_}VCL}\hspace*{\fill}

\begin{list}{}{
\settowidth{\tmplength}{\textbf{Declaração}}
\setlength{\itemindent}{0cm}
\setlength{\listparindent}{0cm}
\setlength{\leftmargin}{\evensidemargin}
\addtolength{\leftmargin}{\tmplength}
\settowidth{\labelsep}{X}
\addtolength{\leftmargin}{\labelsep}
\setlength{\labelwidth}{\tmplength}
}
\begin{flushleft}
\item[\textbf{Declaração}\hfill]
\begin{ttfamily}
public class Function Set{\_}IsApp{\_}VCL(aIsApp{\_}VCL:Boolean):Boolean;\end{ttfamily}


\end{flushleft}
\end{list}
\paragraph*{PopSItem}\hspace*{\fill}

\begin{list}{}{
\settowidth{\tmplength}{\textbf{Declaração}}
\setlength{\itemindent}{0cm}
\setlength{\listparindent}{0cm}
\setlength{\leftmargin}{\evensidemargin}
\addtolength{\leftmargin}{\tmplength}
\settowidth{\labelsep}{X}
\addtolength{\leftmargin}{\labelsep}
\setlength{\labelwidth}{\tmplength}
}
\begin{flushleft}
\item[\textbf{Declaração}\hfill]
\begin{ttfamily}
public class Procedure PopSItem(Var Items: PSItem);\end{ttfamily}


\end{flushleft}
\end{list}
\paragraph*{DISCARD}\hspace*{\fill}

\begin{list}{}{
\settowidth{\tmplength}{\textbf{Declaração}}
\setlength{\itemindent}{0cm}
\setlength{\listparindent}{0cm}
\setlength{\leftmargin}{\evensidemargin}
\addtolength{\leftmargin}{\tmplength}
\settowidth{\labelsep}{X}
\addtolength{\leftmargin}{\labelsep}
\setlength{\labelwidth}{\tmplength}
}
\begin{flushleft}
\item[\textbf{Declaração}\hfill]
\begin{ttfamily}
public class PROCEDURE DISCARD(Var AClass); overload;\end{ttfamily}


\end{flushleft}
\end{list}
\paragraph*{DISCARD}\hspace*{\fill}

\begin{list}{}{
\settowidth{\tmplength}{\textbf{Declaração}}
\setlength{\itemindent}{0cm}
\setlength{\listparindent}{0cm}
\setlength{\leftmargin}{\evensidemargin}
\addtolength{\leftmargin}{\tmplength}
\settowidth{\labelsep}{X}
\addtolength{\leftmargin}{\labelsep}
\setlength{\labelwidth}{\tmplength}
}
\begin{flushleft}
\item[\textbf{Declaração}\hfill]
\begin{ttfamily}
public class PROCEDURE DISCARD(Var AClass:TObject); overload;\end{ttfamily}


\end{flushleft}
\end{list}
\paragraph*{SetFlushBuffer{\_}Disk}\hspace*{\fill}

\begin{list}{}{
\settowidth{\tmplength}{\textbf{Declaração}}
\setlength{\itemindent}{0cm}
\setlength{\listparindent}{0cm}
\setlength{\leftmargin}{\evensidemargin}
\addtolength{\leftmargin}{\tmplength}
\settowidth{\labelsep}{X}
\addtolength{\leftmargin}{\labelsep}
\setlength{\labelwidth}{\tmplength}
}
\begin{flushleft}
\item[\textbf{Declaração}\hfill]
\begin{ttfamily}
public class Function SetFlushBuffer{\_}Disk(Const aFlushBuffer{\_}Disk : Boolean): Boolean;\end{ttfamily}


\end{flushleft}
\end{list}
\paragraph*{SetFlushBuffer}\hspace*{\fill}

\begin{list}{}{
\settowidth{\tmplength}{\textbf{Declaração}}
\setlength{\itemindent}{0cm}
\setlength{\listparindent}{0cm}
\setlength{\leftmargin}{\evensidemargin}
\addtolength{\leftmargin}{\tmplength}
\settowidth{\labelsep}{X}
\addtolength{\leftmargin}{\labelsep}
\setlength{\labelwidth}{\tmplength}
}
\begin{flushleft}
\item[\textbf{Declaração}\hfill]
\begin{ttfamily}
public class Function SetFlushBuffer(Const aFlushBuffer : Boolean): Boolean;\end{ttfamily}


\end{flushleft}
\end{list}
\paragraph*{GetDosTicks}\hspace*{\fill}

\begin{list}{}{
\settowidth{\tmplength}{\textbf{Declaração}}
\setlength{\itemindent}{0cm}
\setlength{\listparindent}{0cm}
\setlength{\leftmargin}{\evensidemargin}
\addtolength{\leftmargin}{\tmplength}
\settowidth{\labelsep}{X}
\addtolength{\leftmargin}{\labelsep}
\setlength{\labelwidth}{\tmplength}
}
\begin{flushleft}
\item[\textbf{Declaração}\hfill]
\begin{ttfamily}
public class Function GetDosTicks:DWord;\end{ttfamily}


\end{flushleft}
\par
\item[\textbf{Descrição}]
\begin{itemize}
\item returns ticks at 18.2 Hz, just like DOS
\end{itemize}

\end{list}
\paragraph*{Seg{\_}to{\_}MillSeg}\hspace*{\fill}

\begin{list}{}{
\settowidth{\tmplength}{\textbf{Declaração}}
\setlength{\itemindent}{0cm}
\setlength{\listparindent}{0cm}
\setlength{\leftmargin}{\evensidemargin}
\addtolength{\leftmargin}{\tmplength}
\settowidth{\labelsep}{X}
\addtolength{\leftmargin}{\labelsep}
\setlength{\labelwidth}{\tmplength}
}
\begin{flushleft}
\item[\textbf{Declaração}\hfill]
\begin{ttfamily}
public class Function Seg{\_}to{\_}MillSeg(aSegundos:Longint):DWord;\end{ttfamily}


\end{flushleft}
\par
\item[\textbf{Descrição}]
A função \textbf{\begin{ttfamily}Seg{\_}to{\_}MillSeg\end{ttfamily}} converte segundos para milisegundos.

\begin{itemize}
\item \textbf{NOTA} \begin{itemize}
\item 1 Milliseconds = 1/1000 segundos {-}{$>$} 1 segundo = 1000 Milliseconds
\end{itemize}
\end{itemize}

\end{list}
\paragraph*{RunError}\hspace*{\fill}

\begin{list}{}{
\settowidth{\tmplength}{\textbf{Declaração}}
\setlength{\itemindent}{0cm}
\setlength{\listparindent}{0cm}
\setlength{\leftmargin}{\evensidemargin}
\addtolength{\leftmargin}{\tmplength}
\settowidth{\labelsep}{X}
\addtolength{\leftmargin}{\labelsep}
\setlength{\labelwidth}{\tmplength}
}
\begin{flushleft}
\item[\textbf{Declaração}\hfill]
\begin{ttfamily}
public class Procedure RunError(Error:Word);\end{ttfamily}


\end{flushleft}
\end{list}
\paragraph*{Run{\_}Error}\hspace*{\fill}

\begin{list}{}{
\settowidth{\tmplength}{\textbf{Declaração}}
\setlength{\itemindent}{0cm}
\setlength{\listparindent}{0cm}
\setlength{\leftmargin}{\evensidemargin}
\addtolength{\leftmargin}{\tmplength}
\settowidth{\labelsep}{X}
\addtolength{\leftmargin}{\labelsep}
\setlength{\labelwidth}{\tmplength}
}
\begin{flushleft}
\item[\textbf{Declaração}\hfill]
\begin{ttfamily}
public class Procedure Run{\_}Error(Error:Word;Procedimento{\_}que{\_}Executou:AnsiString);\end{ttfamily}


\end{flushleft}
\end{list}
\paragraph*{Alert}\hspace*{\fill}

\begin{list}{}{
\settowidth{\tmplength}{\textbf{Declaração}}
\setlength{\itemindent}{0cm}
\setlength{\listparindent}{0cm}
\setlength{\leftmargin}{\evensidemargin}
\addtolength{\leftmargin}{\tmplength}
\settowidth{\labelsep}{X}
\addtolength{\leftmargin}{\labelsep}
\setlength{\labelwidth}{\tmplength}
}
\begin{flushleft}
\item[\textbf{Declaração}\hfill]
\begin{ttfamily}
public Class Procedure Alert(aTitle: AnsiString;aMsg:AnsiString);\end{ttfamily}


\end{flushleft}
\par
\item[\textbf{Descrição}]
\begin{itemize}
\item A procedure \textbf{\begin{ttfamily}Alert\end{ttfamily}} executa um dialogo com botão \textbf{OK}
\end{itemize}

\end{list}
\paragraph*{ShowMessage}\hspace*{\fill}

\begin{list}{}{
\settowidth{\tmplength}{\textbf{Declaração}}
\setlength{\itemindent}{0cm}
\setlength{\listparindent}{0cm}
\setlength{\leftmargin}{\evensidemargin}
\addtolength{\leftmargin}{\tmplength}
\settowidth{\labelsep}{X}
\addtolength{\leftmargin}{\labelsep}
\setlength{\labelwidth}{\tmplength}
}
\begin{flushleft}
\item[\textbf{Declaração}\hfill]
\begin{ttfamily}
public Class procedure ShowMessage(const aMsg: string);\end{ttfamily}


\end{flushleft}
\end{list}
\paragraph*{Confirm}\hspace*{\fill}

\begin{list}{}{
\settowidth{\tmplength}{\textbf{Declaração}}
\setlength{\itemindent}{0cm}
\setlength{\listparindent}{0cm}
\setlength{\leftmargin}{\evensidemargin}
\addtolength{\leftmargin}{\tmplength}
\settowidth{\labelsep}{X}
\addtolength{\leftmargin}{\labelsep}
\setlength{\labelwidth}{\tmplength}
}
\begin{flushleft}
\item[\textbf{Declaração}\hfill]
\begin{ttfamily}
public Class Function Confirm(aTitle: AnsiString;aPergunta:AnsiString):Boolean;\end{ttfamily}


\end{flushleft}
\par
\item[\textbf{Descrição}]
\begin{itemize}
\item A função \textbf{\begin{ttfamily}Confirm\end{ttfamily}} executa um dialogo com os botões \textbf{OK} e \textbf{Cancel} fazendo uma pergunta.

\begin{itemize}
\item \textbf{RETORNA:} \begin{itemize}
\item \textbf{True} : Se o botão \textbf{OK} foi pŕessionando;
\item \textbf{False} : Se o botão \textbf{Cancel} foi pŕessionando.
\end{itemize}
\end{itemize}
\end{itemize}

\end{list}
\paragraph*{InputValue}\hspace*{\fill}

\begin{list}{}{
\settowidth{\tmplength}{\textbf{Declaração}}
\setlength{\itemindent}{0cm}
\setlength{\listparindent}{0cm}
\setlength{\leftmargin}{\evensidemargin}
\addtolength{\leftmargin}{\tmplength}
\settowidth{\labelsep}{X}
\addtolength{\leftmargin}{\labelsep}
\setlength{\labelwidth}{\tmplength}
}
\begin{flushleft}
\item[\textbf{Declaração}\hfill]
\begin{ttfamily}
public Class function InputValue(const aTitle,aLabel: AnsiString;var aValue : Variant): TModalResult;\end{ttfamily}


\end{flushleft}
\par
\item[\textbf{Descrição}]
O método \textbf{\begin{ttfamily}InputValue\end{ttfamily}} ler um valor na tela e retorna em \textbf{aValue} o valor e em result retorna \textbf{MrOk} ou \textbf{MrCancel}

\end{list}
\paragraph*{Prompt}\hspace*{\fill}

\begin{list}{}{
\settowidth{\tmplength}{\textbf{Declaração}}
\setlength{\itemindent}{0cm}
\setlength{\listparindent}{0cm}
\setlength{\leftmargin}{\evensidemargin}
\addtolength{\leftmargin}{\tmplength}
\settowidth{\labelsep}{X}
\addtolength{\leftmargin}{\labelsep}
\setlength{\labelwidth}{\tmplength}
}
\begin{flushleft}
\item[\textbf{Declaração}\hfill]
\begin{ttfamily}
public Class Function Prompt(aTitle: AnsiString;aPergunta:AnsiString;Var aResult: Variant):Boolean;\end{ttfamily}


\end{flushleft}
\par
\item[\textbf{Descrição}]
\begin{itemize}
\item A função \textbf{\begin{ttfamily}Prompt\end{ttfamily}} mostra um dialogo com dois botões \textbf{OK} e \textbf{Cancel} e um campo input solicitando que o usuário digite um valor.

\begin{itemize}
\item \textbf{RETORNA:} \begin{itemize}
\item \textbf{True} : Se o botão \textbf{ok} foi pŕessionando;
\item \textbf{False} : Se o botão \textbf{cancel} foi pŕessionando.
\item \textbf{aResult} : Retorna a string digitada no formulário;
\end{itemize}
\end{itemize}
\end{itemize}

\end{list}
\paragraph*{InputPassword}\hspace*{\fill}

\begin{list}{}{
\settowidth{\tmplength}{\textbf{Declaração}}
\setlength{\itemindent}{0cm}
\setlength{\listparindent}{0cm}
\setlength{\leftmargin}{\evensidemargin}
\addtolength{\leftmargin}{\tmplength}
\settowidth{\labelsep}{X}
\addtolength{\leftmargin}{\labelsep}
\setlength{\labelwidth}{\tmplength}
}
\begin{flushleft}
\item[\textbf{Declaração}\hfill]
\begin{ttfamily}
public Class Function InputPassword(aTitle: AnsiString;out aUsername:AnsiString;out apassword:AnsiString):Boolean; Overload;\end{ttfamily}


\end{flushleft}
\par
\item[\textbf{Descrição}]
\begin{itemize}
\item A função \textbf{\begin{ttfamily}InputPassword\end{ttfamily}} mostra um dialogo solicitando o login do usuário e a senha e dois botões \textbf{OK} e \textbf{Cancel}

\begin{itemize}
\item \textbf{RETORNA:} \begin{itemize}
\item \textbf{True} : Se o botão \textbf{ok} foi pŕessionando;
\item \textbf{False} : Se o botão \textbf{cancel} foi pŕessionando.
\item \textbf{aUsername} : Retorna a string com nome do usuário.
\item \textbf{apassword} : Retorna a string com a senha do usuário.
\end{itemize}
\end{itemize}
\end{itemize}

\end{list}
\paragraph*{InputPassword}\hspace*{\fill}

\begin{list}{}{
\settowidth{\tmplength}{\textbf{Declaração}}
\setlength{\itemindent}{0cm}
\setlength{\listparindent}{0cm}
\setlength{\leftmargin}{\evensidemargin}
\addtolength{\leftmargin}{\tmplength}
\settowidth{\labelsep}{X}
\addtolength{\leftmargin}{\labelsep}
\setlength{\labelwidth}{\tmplength}
}
\begin{flushleft}
\item[\textbf{Declaração}\hfill]
\begin{ttfamily}
public Class Function InputPassword(aTitle: AnsiString;out apassword:AnsiString):Boolean; Overload;\end{ttfamily}


\end{flushleft}
\end{list}
\paragraph*{DisposeSItems}\hspace*{\fill}

\begin{list}{}{
\settowidth{\tmplength}{\textbf{Declaração}}
\setlength{\itemindent}{0cm}
\setlength{\listparindent}{0cm}
\setlength{\leftmargin}{\evensidemargin}
\addtolength{\leftmargin}{\tmplength}
\settowidth{\labelsep}{X}
\addtolength{\leftmargin}{\labelsep}
\setlength{\labelwidth}{\tmplength}
}
\begin{flushleft}
\item[\textbf{Declaração}\hfill]
\begin{ttfamily}
public class procedure DisposeSItems(VAR AItems: PSItem); overload;\end{ttfamily}


\end{flushleft}
\end{list}
\paragraph*{DisposeSItems}\hspace*{\fill}

\begin{list}{}{
\settowidth{\tmplength}{\textbf{Declaração}}
\setlength{\itemindent}{0cm}
\setlength{\listparindent}{0cm}
\setlength{\leftmargin}{\evensidemargin}
\addtolength{\leftmargin}{\tmplength}
\settowidth{\labelsep}{X}
\addtolength{\leftmargin}{\labelsep}
\setlength{\labelwidth}{\tmplength}
}
\begin{flushleft}
\item[\textbf{Declaração}\hfill]
\begin{ttfamily}
public class procedure DisposeSItems(Var AStrItems: PtString); overload;\end{ttfamily}


\end{flushleft}
\end{list}
\paragraph*{MaxItemStrLen}\hspace*{\fill}

\begin{list}{}{
\settowidth{\tmplength}{\textbf{Declaração}}
\setlength{\itemindent}{0cm}
\setlength{\listparindent}{0cm}
\setlength{\leftmargin}{\evensidemargin}
\addtolength{\leftmargin}{\tmplength}
\settowidth{\labelsep}{X}
\addtolength{\leftmargin}{\labelsep}
\setlength{\labelwidth}{\tmplength}
}
\begin{flushleft}
\item[\textbf{Declaração}\hfill]
\begin{ttfamily}
public class function MaxItemStrLen(AItems: PSItem) : integer; Overload;\end{ttfamily}


\end{flushleft}
\end{list}
\paragraph*{MaxItemStrLen}\hspace*{\fill}

\begin{list}{}{
\settowidth{\tmplength}{\textbf{Declaração}}
\setlength{\itemindent}{0cm}
\setlength{\listparindent}{0cm}
\setlength{\leftmargin}{\evensidemargin}
\addtolength{\leftmargin}{\tmplength}
\settowidth{\labelsep}{X}
\addtolength{\leftmargin}{\labelsep}
\setlength{\labelwidth}{\tmplength}
}
\begin{flushleft}
\item[\textbf{Declaração}\hfill]
\begin{ttfamily}
public class function MaxItemStrLen(PSItems: tString) : integer; Overload;\end{ttfamily}


\end{flushleft}
\end{list}
\paragraph*{conststr}\hspace*{\fill}

\begin{list}{}{
\settowidth{\tmplength}{\textbf{Declaração}}
\setlength{\itemindent}{0cm}
\setlength{\listparindent}{0cm}
\setlength{\leftmargin}{\evensidemargin}
\addtolength{\leftmargin}{\tmplength}
\settowidth{\labelsep}{X}
\addtolength{\leftmargin}{\labelsep}
\setlength{\labelwidth}{\tmplength}
}
\begin{flushleft}
\item[\textbf{Declaração}\hfill]
\begin{ttfamily}
public class function conststr( i : Longint; Const a : AnsiChar) : AnsiString;\end{ttfamily}


\end{flushleft}
\end{list}
\paragraph*{centralizesStr}\hspace*{\fill}

\begin{list}{}{
\settowidth{\tmplength}{\textbf{Declaração}}
\setlength{\itemindent}{0cm}
\setlength{\listparindent}{0cm}
\setlength{\leftmargin}{\evensidemargin}
\addtolength{\leftmargin}{\tmplength}
\settowidth{\labelsep}{X}
\addtolength{\leftmargin}{\labelsep}
\setlength{\labelwidth}{\tmplength}
}
\begin{flushleft}
\item[\textbf{Declaração}\hfill]
\begin{ttfamily}
public class function centralizesStr(Const campo : AnsiString; Const tamanho : Integer) : AnsiString;\end{ttfamily}


\end{flushleft}
\end{list}
\paragraph*{TypeFld}\hspace*{\fill}

\begin{list}{}{
\settowidth{\tmplength}{\textbf{Declaração}}
\setlength{\itemindent}{0cm}
\setlength{\listparindent}{0cm}
\setlength{\leftmargin}{\evensidemargin}
\addtolength{\leftmargin}{\tmplength}
\settowidth{\labelsep}{X}
\addtolength{\leftmargin}{\labelsep}
\setlength{\labelwidth}{\tmplength}
}
\begin{flushleft}
\item[\textbf{Declaração}\hfill]
\begin{ttfamily}
public class Function TypeFld(Const aTemplate : tString;Var aSize : SmallWord):AnsiChar; overload;\end{ttfamily}


\end{flushleft}
\end{list}
\paragraph*{TypeFld}\hspace*{\fill}

\begin{list}{}{
\settowidth{\tmplength}{\textbf{Declaração}}
\setlength{\itemindent}{0cm}
\setlength{\listparindent}{0cm}
\setlength{\leftmargin}{\evensidemargin}
\addtolength{\leftmargin}{\tmplength}
\settowidth{\labelsep}{X}
\addtolength{\leftmargin}{\labelsep}
\setlength{\labelwidth}{\tmplength}
}
\begin{flushleft}
\item[\textbf{Declaração}\hfill]
\begin{ttfamily}
public class Function TypeFld(Const aTemplate : ShortString):AnsiChar; overload;\end{ttfamily}


\end{flushleft}
\end{list}
\paragraph*{IStr}\hspace*{\fill}

\begin{list}{}{
\settowidth{\tmplength}{\textbf{Declaração}}
\setlength{\itemindent}{0cm}
\setlength{\listparindent}{0cm}
\setlength{\leftmargin}{\evensidemargin}
\addtolength{\leftmargin}{\tmplength}
\settowidth{\labelsep}{X}
\addtolength{\leftmargin}{\labelsep}
\setlength{\labelwidth}{\tmplength}
}
\begin{flushleft}
\item[\textbf{Declaração}\hfill]
\begin{ttfamily}
public class Function IStr( Const I : Longint; Const Formato : tString) : tString; Overload;\end{ttfamily}


\end{flushleft}
\end{list}
\paragraph*{IStr}\hspace*{\fill}

\begin{list}{}{
\settowidth{\tmplength}{\textbf{Declaração}}
\setlength{\itemindent}{0cm}
\setlength{\listparindent}{0cm}
\setlength{\leftmargin}{\evensidemargin}
\addtolength{\leftmargin}{\tmplength}
\settowidth{\labelsep}{X}
\addtolength{\leftmargin}{\labelsep}
\setlength{\labelwidth}{\tmplength}
}
\begin{flushleft}
\item[\textbf{Declaração}\hfill]
\begin{ttfamily}
public class Function IStr( Const I : Longint) : tString; Overload;\end{ttfamily}


\end{flushleft}
\end{list}
\paragraph*{IStr}\hspace*{\fill}

\begin{list}{}{
\settowidth{\tmplength}{\textbf{Declaração}}
\setlength{\itemindent}{0cm}
\setlength{\listparindent}{0cm}
\setlength{\leftmargin}{\evensidemargin}
\addtolength{\leftmargin}{\tmplength}
\settowidth{\labelsep}{X}
\addtolength{\leftmargin}{\labelsep}
\setlength{\labelwidth}{\tmplength}
}
\begin{flushleft}
\item[\textbf{Declaração}\hfill]
\begin{ttfamily}
public class Function IStr( Const I : tString; Const Formato : tString) : tString; Overload;\end{ttfamily}


\end{flushleft}
\end{list}
\paragraph*{StrNum}\hspace*{\fill}

\begin{list}{}{
\settowidth{\tmplength}{\textbf{Declaração}}
\setlength{\itemindent}{0cm}
\setlength{\listparindent}{0cm}
\setlength{\leftmargin}{\evensidemargin}
\addtolength{\leftmargin}{\tmplength}
\settowidth{\labelsep}{X}
\addtolength{\leftmargin}{\labelsep}
\setlength{\labelwidth}{\tmplength}
}
\begin{flushleft}
\item[\textbf{Declaração}\hfill]
\begin{ttfamily}
public class function StrNum(formato : AnsiString; buffer :Variant; Const Tipo : AnsiChar;Const OkSpc:Boolean) : AnsiString; overload;\end{ttfamily}


\end{flushleft}
\end{list}
\paragraph*{StrNum}\hspace*{\fill}

\begin{list}{}{
\settowidth{\tmplength}{\textbf{Declaração}}
\setlength{\itemindent}{0cm}
\setlength{\listparindent}{0cm}
\setlength{\leftmargin}{\evensidemargin}
\addtolength{\leftmargin}{\tmplength}
\settowidth{\labelsep}{X}
\addtolength{\leftmargin}{\labelsep}
\setlength{\labelwidth}{\tmplength}
}
\begin{flushleft}
\item[\textbf{Declaração}\hfill]
\begin{ttfamily}
public class function StrNum(formato : AnsiString; buffer:Variant; Const Tipo : AnsiChar) : AnsiString; overload;\end{ttfamily}


\end{flushleft}
\end{list}
\paragraph*{IIF}\hspace*{\fill}

\begin{list}{}{
\settowidth{\tmplength}{\textbf{Declaração}}
\setlength{\itemindent}{0cm}
\setlength{\listparindent}{0cm}
\setlength{\leftmargin}{\evensidemargin}
\addtolength{\leftmargin}{\tmplength}
\settowidth{\labelsep}{X}
\addtolength{\leftmargin}{\labelsep}
\setlength{\labelwidth}{\tmplength}
}
\begin{flushleft}
\item[\textbf{Declaração}\hfill]
\begin{ttfamily}
public class function IIF(Const Logica : Boolean; Const E1 , E2 : Boolean) : Boolean; overload;\end{ttfamily}


\end{flushleft}
\end{list}
\paragraph*{IIF}\hspace*{\fill}

\begin{list}{}{
\settowidth{\tmplength}{\textbf{Declaração}}
\setlength{\itemindent}{0cm}
\setlength{\listparindent}{0cm}
\setlength{\leftmargin}{\evensidemargin}
\addtolength{\leftmargin}{\tmplength}
\settowidth{\labelsep}{X}
\addtolength{\leftmargin}{\labelsep}
\setlength{\labelwidth}{\tmplength}
}
\begin{flushleft}
\item[\textbf{Declaração}\hfill]
\begin{ttfamily}
public class function IIF(Const Logica : Boolean; Const E1 , E2 : AnsiChar) : AnsiChar; overload;\end{ttfamily}


\end{flushleft}
\end{list}
\paragraph*{IIF}\hspace*{\fill}

\begin{list}{}{
\settowidth{\tmplength}{\textbf{Declaração}}
\setlength{\itemindent}{0cm}
\setlength{\listparindent}{0cm}
\setlength{\leftmargin}{\evensidemargin}
\addtolength{\leftmargin}{\tmplength}
\settowidth{\labelsep}{X}
\addtolength{\leftmargin}{\labelsep}
\setlength{\labelwidth}{\tmplength}
}
\begin{flushleft}
\item[\textbf{Declaração}\hfill]
\begin{ttfamily}
public class function IIF(Const Logica : Boolean; Const E1 , E2 : Longint ) : Longint; overload;\end{ttfamily}


\end{flushleft}
\end{list}
\paragraph*{IIF}\hspace*{\fill}

\begin{list}{}{
\settowidth{\tmplength}{\textbf{Declaração}}
\setlength{\itemindent}{0cm}
\setlength{\listparindent}{0cm}
\setlength{\leftmargin}{\evensidemargin}
\addtolength{\leftmargin}{\tmplength}
\settowidth{\labelsep}{X}
\addtolength{\leftmargin}{\labelsep}
\setlength{\labelwidth}{\tmplength}
}
\begin{flushleft}
\item[\textbf{Declaração}\hfill]
\begin{ttfamily}
public class function IIF(Const Logica : Boolean; Const E1 , E2 : Extended ) : Extended; overload;\end{ttfamily}


\end{flushleft}
\end{list}
\paragraph*{IIF}\hspace*{\fill}

\begin{list}{}{
\settowidth{\tmplength}{\textbf{Declaração}}
\setlength{\itemindent}{0cm}
\setlength{\listparindent}{0cm}
\setlength{\leftmargin}{\evensidemargin}
\addtolength{\leftmargin}{\tmplength}
\settowidth{\labelsep}{X}
\addtolength{\leftmargin}{\labelsep}
\setlength{\labelwidth}{\tmplength}
}
\begin{flushleft}
\item[\textbf{Declaração}\hfill]
\begin{ttfamily}
public class function IIF(Const Logica : Boolean; Const E1 , E2 : AnsiString ) : AnsiString ; overload;\end{ttfamily}


\end{flushleft}
\end{list}
\paragraph*{SIF}\hspace*{\fill}

\begin{list}{}{
\settowidth{\tmplength}{\textbf{Declaração}}
\setlength{\itemindent}{0cm}
\setlength{\listparindent}{0cm}
\setlength{\leftmargin}{\evensidemargin}
\addtolength{\leftmargin}{\tmplength}
\settowidth{\labelsep}{X}
\addtolength{\leftmargin}{\labelsep}
\setlength{\labelwidth}{\tmplength}
}
\begin{flushleft}
\item[\textbf{Declaração}\hfill]
\begin{ttfamily}
public class Function SIF(Const Logica : Boolean; Const E1 , E2 : AnsiString ) : AnsiString ;\end{ttfamily}


\end{flushleft}
\end{list}
\paragraph*{MinL}\hspace*{\fill}

\begin{list}{}{
\settowidth{\tmplength}{\textbf{Declaração}}
\setlength{\itemindent}{0cm}
\setlength{\listparindent}{0cm}
\setlength{\leftmargin}{\evensidemargin}
\addtolength{\leftmargin}{\tmplength}
\settowidth{\labelsep}{X}
\addtolength{\leftmargin}{\labelsep}
\setlength{\labelwidth}{\tmplength}
}
\begin{flushleft}
\item[\textbf{Declaração}\hfill]
\begin{ttfamily}
public class function MinL(Const a,b:Longint):Longint;\end{ttfamily}


\end{flushleft}
\end{list}
\paragraph*{MaxL}\hspace*{\fill}

\begin{list}{}{
\settowidth{\tmplength}{\textbf{Declaração}}
\setlength{\itemindent}{0cm}
\setlength{\listparindent}{0cm}
\setlength{\leftmargin}{\evensidemargin}
\addtolength{\leftmargin}{\tmplength}
\settowidth{\labelsep}{X}
\addtolength{\leftmargin}{\labelsep}
\setlength{\labelwidth}{\tmplength}
}
\begin{flushleft}
\item[\textbf{Declaração}\hfill]
\begin{ttfamily}
public class function MaxL(Const a,b:Longint):Longint;\end{ttfamily}


\end{flushleft}
\end{list}
\paragraph*{NumToStr}\hspace*{\fill}

\begin{list}{}{
\settowidth{\tmplength}{\textbf{Declaração}}
\setlength{\itemindent}{0cm}
\setlength{\listparindent}{0cm}
\setlength{\leftmargin}{\evensidemargin}
\addtolength{\leftmargin}{\tmplength}
\settowidth{\labelsep}{X}
\addtolength{\leftmargin}{\labelsep}
\setlength{\labelwidth}{\tmplength}
}
\begin{flushleft}
\item[\textbf{Declaração}\hfill]
\begin{ttfamily}
public class function NumToStr(Const formato : AnsiString;buffer:Variant; Const Tipo : AnsiChar;Const OkSpc:Boolean):AnsiString;\end{ttfamily}


\end{flushleft}
\end{list}
\paragraph*{InsertCrtlJ}\hspace*{\fill}

\begin{list}{}{
\settowidth{\tmplength}{\textbf{Declaração}}
\setlength{\itemindent}{0cm}
\setlength{\listparindent}{0cm}
\setlength{\leftmargin}{\evensidemargin}
\addtolength{\leftmargin}{\tmplength}
\settowidth{\labelsep}{X}
\addtolength{\leftmargin}{\labelsep}
\setlength{\labelwidth}{\tmplength}
}
\begin{flushleft}
\item[\textbf{Declaração}\hfill]
\begin{ttfamily}
public class Function InsertCrtlJ(Const StrMsg:tString):tString;\end{ttfamily}


\end{flushleft}
\end{list}
\paragraph*{Create{\_}Progress1Passo}\hspace*{\fill}

\begin{list}{}{
\settowidth{\tmplength}{\textbf{Declaração}}
\setlength{\itemindent}{0cm}
\setlength{\listparindent}{0cm}
\setlength{\leftmargin}{\evensidemargin}
\addtolength{\leftmargin}{\tmplength}
\settowidth{\labelsep}{X}
\addtolength{\leftmargin}{\labelsep}
\setlength{\labelwidth}{\tmplength}
}
\begin{flushleft}
\item[\textbf{Declaração}\hfill]
\begin{ttfamily}
public class procedure Create{\_}Progress1Passo(ATitle : tstring;Obs:tstring ; ATotal : Longint); Virtual;\end{ttfamily}


\end{flushleft}
\end{list}
\paragraph*{Set{\_}Progress1Passo}\hspace*{\fill}

\begin{list}{}{
\settowidth{\tmplength}{\textbf{Declaração}}
\setlength{\itemindent}{0cm}
\setlength{\listparindent}{0cm}
\setlength{\leftmargin}{\evensidemargin}
\addtolength{\leftmargin}{\tmplength}
\settowidth{\labelsep}{X}
\addtolength{\leftmargin}{\labelsep}
\setlength{\labelwidth}{\tmplength}
}
\begin{flushleft}
\item[\textbf{Declaração}\hfill]
\begin{ttfamily}
public class procedure Set{\_}Progress1Passo(aNumber : Longint); Virtual;\end{ttfamily}


\end{flushleft}
\end{list}
\paragraph*{Destroy{\_}Progress1Passo}\hspace*{\fill}

\begin{list}{}{
\settowidth{\tmplength}{\textbf{Declaração}}
\setlength{\itemindent}{0cm}
\setlength{\listparindent}{0cm}
\setlength{\leftmargin}{\evensidemargin}
\addtolength{\leftmargin}{\tmplength}
\settowidth{\labelsep}{X}
\addtolength{\leftmargin}{\labelsep}
\setlength{\labelwidth}{\tmplength}
}
\begin{flushleft}
\item[\textbf{Declaração}\hfill]
\begin{ttfamily}
public class procedure Destroy{\_}Progress1Passo; Virtual;\end{ttfamily}


\end{flushleft}
\end{list}
\paragraph*{LogError}\hspace*{\fill}

\begin{list}{}{
\settowidth{\tmplength}{\textbf{Declaração}}
\setlength{\itemindent}{0cm}
\setlength{\listparindent}{0cm}
\setlength{\leftmargin}{\evensidemargin}
\addtolength{\leftmargin}{\tmplength}
\settowidth{\labelsep}{X}
\addtolength{\leftmargin}{\labelsep}
\setlength{\labelwidth}{\tmplength}
}
\begin{flushleft}
\item[\textbf{Declaração}\hfill]
\begin{ttfamily}
public class procedure LogError(const Fmt: String; Args: array of const); overload;\end{ttfamily}


\end{flushleft}
\end{list}
\paragraph*{LogError}\hspace*{\fill}

\begin{list}{}{
\settowidth{\tmplength}{\textbf{Declaração}}
\setlength{\itemindent}{0cm}
\setlength{\listparindent}{0cm}
\setlength{\leftmargin}{\evensidemargin}
\addtolength{\leftmargin}{\tmplength}
\settowidth{\labelsep}{X}
\addtolength{\leftmargin}{\labelsep}
\setlength{\labelwidth}{\tmplength}
}
\begin{flushleft}
\item[\textbf{Declaração}\hfill]
\begin{ttfamily}
public class procedure LogError(const Msg:AnsiString ); overload;\end{ttfamily}


\end{flushleft}
\end{list}
\paragraph*{WideStringToString}\hspace*{\fill}

\begin{list}{}{
\settowidth{\tmplength}{\textbf{Declaração}}
\setlength{\itemindent}{0cm}
\setlength{\listparindent}{0cm}
\setlength{\leftmargin}{\evensidemargin}
\addtolength{\leftmargin}{\tmplength}
\settowidth{\labelsep}{X}
\addtolength{\leftmargin}{\labelsep}
\setlength{\labelwidth}{\tmplength}
}
\begin{flushleft}
\item[\textbf{Declaração}\hfill]
\begin{ttfamily}
public class function WideStringToString(const ws: WideString): AnsiString;\end{ttfamily}


\end{flushleft}
\end{list}
\paragraph*{Set{\_}FileModeDenyALLSalvaAnt}\hspace*{\fill}

\begin{list}{}{
\settowidth{\tmplength}{\textbf{Declaração}}
\setlength{\itemindent}{0cm}
\setlength{\listparindent}{0cm}
\setlength{\leftmargin}{\evensidemargin}
\addtolength{\leftmargin}{\tmplength}
\settowidth{\labelsep}{X}
\addtolength{\leftmargin}{\labelsep}
\setlength{\labelwidth}{\tmplength}
}
\begin{flushleft}
\item[\textbf{Declaração}\hfill]
\begin{ttfamily}
public class Function Set{\_}FileModeDenyALLSalvaAnt(Const ModoDoArquivo : Boolean;Var {\_}FileModeDenyALLAnt:Boolean):Boolean;\end{ttfamily}


\end{flushleft}
\end{list}
\paragraph*{Set{\_}FileModeDenyALL}\hspace*{\fill}

\begin{list}{}{
\settowidth{\tmplength}{\textbf{Declaração}}
\setlength{\itemindent}{0cm}
\setlength{\listparindent}{0cm}
\setlength{\leftmargin}{\evensidemargin}
\addtolength{\leftmargin}{\tmplength}
\settowidth{\labelsep}{X}
\addtolength{\leftmargin}{\labelsep}
\setlength{\labelwidth}{\tmplength}
}
\begin{flushleft}
\item[\textbf{Declaração}\hfill]
\begin{ttfamily}
public Class Function Set{\_}FileModeDenyALL(Const ModoDoArquivo : Boolean):Boolean;\end{ttfamily}


\end{flushleft}
\end{list}
\paragraph*{Sitems{\_}MsgErro}\hspace*{\fill}

\begin{list}{}{
\settowidth{\tmplength}{\textbf{Declaração}}
\setlength{\itemindent}{0cm}
\setlength{\listparindent}{0cm}
\setlength{\leftmargin}{\evensidemargin}
\addtolength{\leftmargin}{\tmplength}
\settowidth{\labelsep}{X}
\addtolength{\leftmargin}{\labelsep}
\setlength{\labelwidth}{\tmplength}
}
\begin{flushleft}
\item[\textbf{Declaração}\hfill]
\begin{ttfamily}
public class Function Sitems{\_}MsgErro() : PSItem;\end{ttfamily}


\end{flushleft}
\end{list}
\paragraph*{Pop{\_}MsgErro}\hspace*{\fill}

\begin{list}{}{
\settowidth{\tmplength}{\textbf{Declaração}}
\setlength{\itemindent}{0cm}
\setlength{\listparindent}{0cm}
\setlength{\leftmargin}{\evensidemargin}
\addtolength{\leftmargin}{\tmplength}
\settowidth{\labelsep}{X}
\addtolength{\leftmargin}{\labelsep}
\setlength{\labelwidth}{\tmplength}
}
\begin{flushleft}
\item[\textbf{Declaração}\hfill]
\begin{ttfamily}
public class Function Pop{\_}MsgErro:PSItem;\end{ttfamily}


\end{flushleft}
\par
\item[\textbf{Descrição}]
Retire o ultimo string na pilha

\end{list}
\paragraph*{SpcStrD}\hspace*{\fill}

\begin{list}{}{
\settowidth{\tmplength}{\textbf{Declaração}}
\setlength{\itemindent}{0cm}
\setlength{\listparindent}{0cm}
\setlength{\leftmargin}{\evensidemargin}
\addtolength{\leftmargin}{\tmplength}
\settowidth{\labelsep}{X}
\addtolength{\leftmargin}{\labelsep}
\setlength{\labelwidth}{\tmplength}
}
\begin{flushleft}
\item[\textbf{Declaração}\hfill]
\begin{ttfamily}
public class function SpcStrD(Const campo : tString; Const Tam : Byte): tString;\end{ttfamily}


\end{flushleft}
\end{list}
\paragraph*{CentralizaStr}\hspace*{\fill}

\begin{list}{}{
\settowidth{\tmplength}{\textbf{Declaração}}
\setlength{\itemindent}{0cm}
\setlength{\listparindent}{0cm}
\setlength{\leftmargin}{\evensidemargin}
\addtolength{\leftmargin}{\tmplength}
\settowidth{\labelsep}{X}
\addtolength{\leftmargin}{\labelsep}
\setlength{\labelwidth}{\tmplength}
}
\begin{flushleft}
\item[\textbf{Declaração}\hfill]
\begin{ttfamily}
public class function CentralizaStr(Const campo : AnsiString; Const tamanho : Integer) : AnsiString;\end{ttfamily}


\end{flushleft}
\end{list}
\paragraph*{spc}\hspace*{\fill}

\begin{list}{}{
\settowidth{\tmplength}{\textbf{Declaração}}
\setlength{\itemindent}{0cm}
\setlength{\listparindent}{0cm}
\setlength{\leftmargin}{\evensidemargin}
\addtolength{\leftmargin}{\tmplength}
\settowidth{\labelsep}{X}
\addtolength{\leftmargin}{\labelsep}
\setlength{\labelwidth}{\tmplength}
}
\begin{flushleft}
\item[\textbf{Declaração}\hfill]
\begin{ttfamily}
public class function spc(Const campo:AnsiString;Const tam :Longint):AnsiString;\end{ttfamily}


\end{flushleft}
\end{list}
\paragraph*{NumberCharControl}\hspace*{\fill}

\begin{list}{}{
\settowidth{\tmplength}{\textbf{Declaração}}
\setlength{\itemindent}{0cm}
\setlength{\listparindent}{0cm}
\setlength{\leftmargin}{\evensidemargin}
\addtolength{\leftmargin}{\tmplength}
\settowidth{\labelsep}{X}
\addtolength{\leftmargin}{\labelsep}
\setlength{\labelwidth}{\tmplength}
}
\begin{flushleft}
\item[\textbf{Declaração}\hfill]
\begin{ttfamily}
public class function NumberCharControl(s:AnsiString):Integer;\end{ttfamily}


\end{flushleft}
\par
\item[\textbf{Descrição}]
O método \textbf{\begin{ttfamily}NumberCharControl\end{ttfamily}} retorna o número de caracteres de controle da string s

\end{list}
\paragraph*{StrAlinhado}\hspace*{\fill}

\begin{list}{}{
\settowidth{\tmplength}{\textbf{Declaração}}
\setlength{\itemindent}{0cm}
\setlength{\listparindent}{0cm}
\setlength{\leftmargin}{\evensidemargin}
\addtolength{\leftmargin}{\tmplength}
\settowidth{\labelsep}{X}
\addtolength{\leftmargin}{\labelsep}
\setlength{\labelwidth}{\tmplength}
}
\begin{flushleft}
\item[\textbf{Declaração}\hfill]
\begin{ttfamily}
public class Function StrAlinhado(aStrMsg:tString;Colunas : byte;Const Alinhamento:TAlinhamento):tString;\end{ttfamily}


\end{flushleft}
\end{list}
\paragraph*{StringToSItem}\hspace*{\fill}

\begin{list}{}{
\settowidth{\tmplength}{\textbf{Declaração}}
\setlength{\itemindent}{0cm}
\setlength{\listparindent}{0cm}
\setlength{\leftmargin}{\evensidemargin}
\addtolength{\leftmargin}{\tmplength}
\settowidth{\labelsep}{X}
\addtolength{\leftmargin}{\labelsep}
\setlength{\labelwidth}{\tmplength}
}
\begin{flushleft}
\item[\textbf{Declaração}\hfill]
\begin{ttfamily}
public class Function StringToSItem( StrMsg:AnsiString; Colunas : byte;Alinhamento:TAlinhamento):PSItem; virtual; overload;\end{ttfamily}


\end{flushleft}
\end{list}
\paragraph*{StringToSItem}\hspace*{\fill}

\begin{list}{}{
\settowidth{\tmplength}{\textbf{Declaração}}
\setlength{\itemindent}{0cm}
\setlength{\listparindent}{0cm}
\setlength{\leftmargin}{\evensidemargin}
\addtolength{\leftmargin}{\tmplength}
\settowidth{\labelsep}{X}
\addtolength{\leftmargin}{\labelsep}
\setlength{\labelwidth}{\tmplength}
}
\begin{flushleft}
\item[\textbf{Declaração}\hfill]
\begin{ttfamily}
public class Function StringToSItem( StrMsg:AnsiString; Colunas : byte):PSItem; virtual; overload;\end{ttfamily}


\end{flushleft}
\end{list}
\paragraph*{SItemsLen}\hspace*{\fill}

\begin{list}{}{
\settowidth{\tmplength}{\textbf{Declaração}}
\setlength{\itemindent}{0cm}
\setlength{\listparindent}{0cm}
\setlength{\leftmargin}{\evensidemargin}
\addtolength{\leftmargin}{\tmplength}
\settowidth{\labelsep}{X}
\addtolength{\leftmargin}{\labelsep}
\setlength{\labelwidth}{\tmplength}
}
\begin{flushleft}
\item[\textbf{Declaração}\hfill]
\begin{ttfamily}
public class function SItemsLen(S: PSItem) : SmallInt;\end{ttfamily}


\end{flushleft}
\end{list}
\paragraph*{SItemToString}\hspace*{\fill}

\begin{list}{}{
\settowidth{\tmplength}{\textbf{Declaração}}
\setlength{\itemindent}{0cm}
\setlength{\listparindent}{0cm}
\setlength{\leftmargin}{\evensidemargin}
\addtolength{\leftmargin}{\tmplength}
\settowidth{\labelsep}{X}
\addtolength{\leftmargin}{\labelsep}
\setlength{\labelwidth}{\tmplength}
}
\begin{flushleft}
\item[\textbf{Declaração}\hfill]
\begin{ttfamily}
public class Function SItemToString(Items: PSItem):AnsiString;\end{ttfamily}


\end{flushleft}
\end{list}
\paragraph*{WriteSItems}\hspace*{\fill}

\begin{list}{}{
\settowidth{\tmplength}{\textbf{Declaração}}
\setlength{\itemindent}{0cm}
\setlength{\listparindent}{0cm}
\setlength{\leftmargin}{\evensidemargin}
\addtolength{\leftmargin}{\tmplength}
\settowidth{\labelsep}{X}
\addtolength{\leftmargin}{\labelsep}
\setlength{\labelwidth}{\tmplength}
}
\begin{flushleft}
\item[\textbf{Declaração}\hfill]
\begin{ttfamily}
public class procedure WriteSItems(var S: TMiStringList;const Items: PSItem);\end{ttfamily}


\end{flushleft}
\par
\item[\textbf{Descrição}]
A classe procedure \textbf{\begin{ttfamily}WriteSItems\end{ttfamily}} retorna um TStringList com a lista passado por items

\begin{itemize}
\item \textbf{NOTA} \begin{itemize}
\item S : Deve ser passado não inicializado, ouseja deve ser NIL.
\end{itemize}
\end{itemize}

\end{list}
\paragraph*{PSItem{\_}ListaDeMsgErro}\hspace*{\fill}

\begin{list}{}{
\settowidth{\tmplength}{\textbf{Declaração}}
\setlength{\itemindent}{0cm}
\setlength{\listparindent}{0cm}
\setlength{\leftmargin}{\evensidemargin}
\addtolength{\leftmargin}{\tmplength}
\settowidth{\labelsep}{X}
\addtolength{\leftmargin}{\labelsep}
\setlength{\labelwidth}{\tmplength}
}
\begin{flushleft}
\item[\textbf{Declaração}\hfill]
\begin{ttfamily}
public class Function PSItem{\_}ListaDeMsgErro:PSItem; virtual;\end{ttfamily}


\end{flushleft}
\end{list}
\paragraph*{MessageError}\hspace*{\fill}

\begin{list}{}{
\settowidth{\tmplength}{\textbf{Declaração}}
\setlength{\itemindent}{0cm}
\setlength{\listparindent}{0cm}
\setlength{\leftmargin}{\evensidemargin}
\addtolength{\leftmargin}{\tmplength}
\settowidth{\labelsep}{X}
\addtolength{\leftmargin}{\labelsep}
\setlength{\labelwidth}{\tmplength}
}
\begin{flushleft}
\item[\textbf{Declaração}\hfill]
\begin{ttfamily}
public class Procedure MessageError; virtual;\end{ttfamily}


\end{flushleft}
\end{list}
\paragraph*{String{\_}ListaDeMsgErro}\hspace*{\fill}

\begin{list}{}{
\settowidth{\tmplength}{\textbf{Declaração}}
\setlength{\itemindent}{0cm}
\setlength{\listparindent}{0cm}
\setlength{\leftmargin}{\evensidemargin}
\addtolength{\leftmargin}{\tmplength}
\settowidth{\labelsep}{X}
\addtolength{\leftmargin}{\labelsep}
\setlength{\labelwidth}{\tmplength}
}
\begin{flushleft}
\item[\textbf{Declaração}\hfill]
\begin{ttfamily}
public class Function String{\_}ListaDeMsgErro(Separador:String):AnsiString; Overload;\end{ttfamily}


\end{flushleft}
\end{list}
\paragraph*{Dispose{\_}ListaDeMsgErro}\hspace*{\fill}

\begin{list}{}{
\settowidth{\tmplength}{\textbf{Declaração}}
\setlength{\itemindent}{0cm}
\setlength{\listparindent}{0cm}
\setlength{\leftmargin}{\evensidemargin}
\addtolength{\leftmargin}{\tmplength}
\settowidth{\labelsep}{X}
\addtolength{\leftmargin}{\labelsep}
\setlength{\labelwidth}{\tmplength}
}
\begin{flushleft}
\item[\textbf{Declaração}\hfill]
\begin{ttfamily}
public class Procedure Dispose{\_}ListaDeMsgErro; virtual;\end{ttfamily}


\end{flushleft}
\par
\item[\textbf{Descrição}]
A procedure \textbf{\begin{ttfamily}Dispose{\_}ListaDeMsgErro\end{ttfamily}} esvazia a pilha de mensagens de error caso as mensagen não tenhão sido tratadas antes de encerrar TMI{\_}Application.

\end{list}
\paragraph*{FMaiuscula}\hspace*{\fill}

\begin{list}{}{
\settowidth{\tmplength}{\textbf{Declaração}}
\setlength{\itemindent}{0cm}
\setlength{\listparindent}{0cm}
\setlength{\leftmargin}{\evensidemargin}
\addtolength{\leftmargin}{\tmplength}
\settowidth{\labelsep}{X}
\addtolength{\leftmargin}{\labelsep}
\setlength{\labelwidth}{\tmplength}
}
\begin{flushleft}
\item[\textbf{Declaração}\hfill]
\begin{ttfamily}
public class Function FMaiuscula(str:AnsiString):AnsiString;\end{ttfamily}


\end{flushleft}
\end{list}
\paragraph*{AnsiString{\_}to{\_}USASCII}\hspace*{\fill}

\begin{list}{}{
\settowidth{\tmplength}{\textbf{Declaração}}
\setlength{\itemindent}{0cm}
\setlength{\listparindent}{0cm}
\setlength{\leftmargin}{\evensidemargin}
\addtolength{\leftmargin}{\tmplength}
\settowidth{\labelsep}{X}
\addtolength{\leftmargin}{\labelsep}
\setlength{\labelwidth}{\tmplength}
}
\begin{flushleft}
\item[\textbf{Declaração}\hfill]
\begin{ttfamily}
public class function AnsiString{\_}to{\_}USASCII(const pText: string): string;\end{ttfamily}


\end{flushleft}
\par
\item[\textbf{Descrição}]
A função \textbf{\begin{ttfamily}AnsiString{\_}to{\_}USASCII\end{ttfamily}} remove os acentos do texto pText

\begin{itemize}
\item \textbf{REFERÊNCIA} \begin{itemize}
\item Exemplo completo: https://showdelphi.com.br/dica-funcao-para-remover-acentos-de-uma-string-delphi/
\end{itemize}
\end{itemize}

\end{list}
\paragraph*{RemoveAccents}\hspace*{\fill}

\begin{list}{}{
\settowidth{\tmplength}{\textbf{Declaração}}
\setlength{\itemindent}{0cm}
\setlength{\listparindent}{0cm}
\setlength{\leftmargin}{\evensidemargin}
\addtolength{\leftmargin}{\tmplength}
\settowidth{\labelsep}{X}
\addtolength{\leftmargin}{\labelsep}
\setlength{\labelwidth}{\tmplength}
}
\begin{flushleft}
\item[\textbf{Declaração}\hfill]
\begin{ttfamily}
public class function RemoveAccents(const str: String ): String;\end{ttfamily}


\end{flushleft}
\par
\item[\textbf{Descrição}]
A class function \textbf{\begin{ttfamily}RemoveAccents\end{ttfamily}} converte caracteres acentuados para caracteres não acentuados

\begin{itemize}
\item \textbf{POR QUE?} \begin{itemize}
\item Preciso que as chaves dos índices não tenha acentos para evitar confusão nas pesquisas.
\end{itemize}
\end{itemize}

\end{list}
\paragraph*{String{\_}Asc{\_}GUI{\_}to{\_}Asc{\_}Ingles}\hspace*{\fill}

\begin{list}{}{
\settowidth{\tmplength}{\textbf{Declaração}}
\setlength{\itemindent}{0cm}
\setlength{\listparindent}{0cm}
\setlength{\leftmargin}{\evensidemargin}
\addtolength{\leftmargin}{\tmplength}
\settowidth{\labelsep}{X}
\addtolength{\leftmargin}{\labelsep}
\setlength{\labelwidth}{\tmplength}
}
\begin{flushleft}
\item[\textbf{Declaração}\hfill]
\begin{ttfamily}
public class Function String{\_}Asc{\_}GUI{\_}to{\_}Asc{\_}Ingles(Const S: String): String;\end{ttfamily}


\end{flushleft}
\end{list}
\paragraph*{SGI}\hspace*{\fill}

\begin{list}{}{
\settowidth{\tmplength}{\textbf{Declaração}}
\setlength{\itemindent}{0cm}
\setlength{\listparindent}{0cm}
\setlength{\leftmargin}{\evensidemargin}
\addtolength{\leftmargin}{\tmplength}
\settowidth{\labelsep}{X}
\addtolength{\leftmargin}{\labelsep}
\setlength{\labelwidth}{\tmplength}
}
\begin{flushleft}
\item[\textbf{Declaração}\hfill]
\begin{ttfamily}
public class Function SGI(Const S: String): String;\end{ttfamily}


\end{flushleft}
\end{list}
\paragraph*{String{\_}Asc{\_}GUI{\_}to{\_}Asc{\_}HTML}\hspace*{\fill}

\begin{list}{}{
\settowidth{\tmplength}{\textbf{Declaração}}
\setlength{\itemindent}{0cm}
\setlength{\listparindent}{0cm}
\setlength{\leftmargin}{\evensidemargin}
\addtolength{\leftmargin}{\tmplength}
\settowidth{\labelsep}{X}
\addtolength{\leftmargin}{\labelsep}
\setlength{\labelwidth}{\tmplength}
}
\begin{flushleft}
\item[\textbf{Declaração}\hfill]
\begin{ttfamily}
public class Function String{\_}Asc{\_}GUI{\_}to{\_}Asc{\_}HTML(Const S: String): String;\end{ttfamily}


\end{flushleft}
\end{list}
\paragraph*{SGH}\hspace*{\fill}

\begin{list}{}{
\settowidth{\tmplength}{\textbf{Declaração}}
\setlength{\itemindent}{0cm}
\setlength{\listparindent}{0cm}
\setlength{\leftmargin}{\evensidemargin}
\addtolength{\leftmargin}{\tmplength}
\settowidth{\labelsep}{X}
\addtolength{\leftmargin}{\labelsep}
\setlength{\labelwidth}{\tmplength}
}
\begin{flushleft}
\item[\textbf{Declaração}\hfill]
\begin{ttfamily}
public class Function SGH(Const S: String): String;\end{ttfamily}


\end{flushleft}
\end{list}
\paragraph*{Show{\_}GetEnv{\_}System}\hspace*{\fill}

\begin{list}{}{
\settowidth{\tmplength}{\textbf{Declaração}}
\setlength{\itemindent}{0cm}
\setlength{\listparindent}{0cm}
\setlength{\leftmargin}{\evensidemargin}
\addtolength{\leftmargin}{\tmplength}
\settowidth{\labelsep}{X}
\addtolength{\leftmargin}{\labelsep}
\setlength{\labelwidth}{\tmplength}
}
\begin{flushleft}
\item[\textbf{Declaração}\hfill]
\begin{ttfamily}
public class Procedure Show{\_}GetEnv{\_}System;\end{ttfamily}


\end{flushleft}
\end{list}
\paragraph*{FGetMem}\hspace*{\fill}

\begin{list}{}{
\settowidth{\tmplength}{\textbf{Declaração}}
\setlength{\itemindent}{0cm}
\setlength{\listparindent}{0cm}
\setlength{\leftmargin}{\evensidemargin}
\addtolength{\leftmargin}{\tmplength}
\settowidth{\labelsep}{X}
\addtolength{\leftmargin}{\labelsep}
\setlength{\labelwidth}{\tmplength}
}
\begin{flushleft}
\item[\textbf{Declaração}\hfill]
\begin{ttfamily}
public class Function FGetMem(Var Buff;Const TamBuff: Word) : Boolean;\end{ttfamily}


\end{flushleft}
\end{list}
\paragraph*{FFreeMem}\hspace*{\fill}

\begin{list}{}{
\settowidth{\tmplength}{\textbf{Declaração}}
\setlength{\itemindent}{0cm}
\setlength{\listparindent}{0cm}
\setlength{\leftmargin}{\evensidemargin}
\addtolength{\leftmargin}{\tmplength}
\settowidth{\labelsep}{X}
\addtolength{\leftmargin}{\labelsep}
\setlength{\labelwidth}{\tmplength}
}
\begin{flushleft}
\item[\textbf{Declaração}\hfill]
\begin{ttfamily}
public class Procedure FFreeMem(Var Buff;Const TamBuff: Word) ;\end{ttfamily}


\end{flushleft}
\end{list}
\paragraph*{CGetMem}\hspace*{\fill}

\begin{list}{}{
\settowidth{\tmplength}{\textbf{Declaração}}
\setlength{\itemindent}{0cm}
\setlength{\listparindent}{0cm}
\setlength{\leftmargin}{\evensidemargin}
\addtolength{\leftmargin}{\tmplength}
\settowidth{\labelsep}{X}
\addtolength{\leftmargin}{\labelsep}
\setlength{\labelwidth}{\tmplength}
}
\begin{flushleft}
\item[\textbf{Declaração}\hfill]
\begin{ttfamily}
public class Function CGetMem(Const BuffOriginal:Pointer ;Const TamBuff: Word):Pointer;\end{ttfamily}


\end{flushleft}
\par
\item[\textbf{Descrição}]
Retorna um ponteiro para a memória alocada e este ponteiro aponta para uma copia dos dados passado por BuffOriginal

\end{list}
\paragraph*{isfileopen}\hspace*{\fill}

\begin{list}{}{
\settowidth{\tmplength}{\textbf{Declaração}}
\setlength{\itemindent}{0cm}
\setlength{\listparindent}{0cm}
\setlength{\leftmargin}{\evensidemargin}
\addtolength{\leftmargin}{\tmplength}
\settowidth{\labelsep}{X}
\addtolength{\leftmargin}{\labelsep}
\setlength{\labelwidth}{\tmplength}
}
\begin{flushleft}
\item[\textbf{Declaração}\hfill]
\begin{ttfamily}
public class function isfileopen(var f:file):boolean ; overload;\end{ttfamily}


\end{flushleft}
\end{list}
\paragraph*{isfileopen}\hspace*{\fill}

\begin{list}{}{
\settowidth{\tmplength}{\textbf{Declaração}}
\setlength{\itemindent}{0cm}
\setlength{\listparindent}{0cm}
\setlength{\leftmargin}{\evensidemargin}
\addtolength{\leftmargin}{\tmplength}
\settowidth{\labelsep}{X}
\addtolength{\leftmargin}{\labelsep}
\setlength{\labelwidth}{\tmplength}
}
\begin{flushleft}
\item[\textbf{Declaração}\hfill]
\begin{ttfamily}
public class function isfileopen(var f:text):boolean ; overload;\end{ttfamily}


\end{flushleft}
\end{list}
\paragraph*{CloseLst}\hspace*{\fill}

\begin{list}{}{
\settowidth{\tmplength}{\textbf{Declaração}}
\setlength{\itemindent}{0cm}
\setlength{\listparindent}{0cm}
\setlength{\leftmargin}{\evensidemargin}
\addtolength{\leftmargin}{\tmplength}
\settowidth{\labelsep}{X}
\addtolength{\leftmargin}{\labelsep}
\setlength{\labelwidth}{\tmplength}
}
\begin{flushleft}
\item[\textbf{Declaração}\hfill]
\begin{ttfamily}
public class Function CloseLst:SmallInt;\end{ttfamily}


\end{flushleft}
\end{list}
\paragraph*{RedirecionaParaImpressora}\hspace*{\fill}

\begin{list}{}{
\settowidth{\tmplength}{\textbf{Declaração}}
\setlength{\itemindent}{0cm}
\setlength{\listparindent}{0cm}
\setlength{\leftmargin}{\evensidemargin}
\addtolength{\leftmargin}{\tmplength}
\settowidth{\labelsep}{X}
\addtolength{\leftmargin}{\labelsep}
\setlength{\labelwidth}{\tmplength}
}
\begin{flushleft}
\item[\textbf{Declaração}\hfill]
\begin{ttfamily}
public class Procedure RedirecionaParaImpressora;\end{ttfamily}


\end{flushleft}
\end{list}
\paragraph*{RedirecionaRelatorio}\hspace*{\fill}

\begin{list}{}{
\settowidth{\tmplength}{\textbf{Declaração}}
\setlength{\itemindent}{0cm}
\setlength{\listparindent}{0cm}
\setlength{\leftmargin}{\evensidemargin}
\addtolength{\leftmargin}{\tmplength}
\settowidth{\labelsep}{X}
\addtolength{\leftmargin}{\labelsep}
\setlength{\labelwidth}{\tmplength}
}
\begin{flushleft}
\item[\textbf{Declaração}\hfill]
\begin{ttfamily}
public class Procedure RedirecionaRelatorio;\end{ttfamily}


\end{flushleft}
\end{list}
\paragraph*{ChangeSubStr}\hspace*{\fill}

\begin{list}{}{
\settowidth{\tmplength}{\textbf{Declaração}}
\setlength{\itemindent}{0cm}
\setlength{\listparindent}{0cm}
\setlength{\leftmargin}{\evensidemargin}
\addtolength{\leftmargin}{\tmplength}
\settowidth{\labelsep}{X}
\addtolength{\leftmargin}{\labelsep}
\setlength{\labelwidth}{\tmplength}
}
\begin{flushleft}
\item[\textbf{Declaração}\hfill]
\begin{ttfamily}
public class Function ChangeSubStr(aSubStrOld : AnsiString; aSubStrNew : AnsiString; S: AnsiString ):AnsiString;\end{ttfamily}


\end{flushleft}
\par
\item[\textbf{Descrição}]
Retorna S com o \begin{ttfamily}tString\end{ttfamily}(\ref{mi_rtl_ui_Dmxscroller-tString}) Trocado

\end{list}
\paragraph*{Alias{\_}To{\_}Name}\hspace*{\fill}

\begin{list}{}{
\settowidth{\tmplength}{\textbf{Declaração}}
\setlength{\itemindent}{0cm}
\setlength{\listparindent}{0cm}
\setlength{\leftmargin}{\evensidemargin}
\addtolength{\leftmargin}{\tmplength}
\settowidth{\labelsep}{X}
\addtolength{\leftmargin}{\labelsep}
\setlength{\labelwidth}{\tmplength}
}
\begin{flushleft}
\item[\textbf{Declaração}\hfill]
\begin{ttfamily}
public class Function Alias{\_}To{\_}Name(AAlias : AnsiString):AnsiString;\end{ttfamily}


\end{flushleft}
\end{list}
\paragraph*{CreateGUID}\hspace*{\fill}

\begin{list}{}{
\settowidth{\tmplength}{\textbf{Declaração}}
\setlength{\itemindent}{0cm}
\setlength{\listparindent}{0cm}
\setlength{\leftmargin}{\evensidemargin}
\addtolength{\leftmargin}{\tmplength}
\settowidth{\labelsep}{X}
\addtolength{\leftmargin}{\labelsep}
\setlength{\labelwidth}{\tmplength}
}
\begin{flushleft}
\item[\textbf{Declaração}\hfill]
\begin{ttfamily}
public class function CreateGUID():TString;\end{ttfamily}


\end{flushleft}
\par
\item[\textbf{Descrição}]
A class método \textbf{\begin{ttfamily}CreateGUID\end{ttfamily}} cria um novo valor de GUID (Globally Unique Identifier).

\begin{itemize}
\item \textbf{RETORNA} \begin{itemize}
\item \textbf{GUID} : Novo GUID se sucesso ou string vazia se fracasso.
\end{itemize}
\end{itemize}

\end{list}
\paragraph*{SetExecAsync}\hspace*{\fill}

\begin{list}{}{
\settowidth{\tmplength}{\textbf{Declaração}}
\setlength{\itemindent}{0cm}
\setlength{\listparindent}{0cm}
\setlength{\leftmargin}{\evensidemargin}
\addtolength{\leftmargin}{\tmplength}
\settowidth{\labelsep}{X}
\addtolength{\leftmargin}{\labelsep}
\setlength{\labelwidth}{\tmplength}
}
\begin{flushleft}
\item[\textbf{Declaração}\hfill]
\begin{ttfamily}
public class Function SetExecAsync(aExecAsync:Byte):Byte;\end{ttfamily}


\end{flushleft}
\end{list}
\paragraph*{GetExecAsync}\hspace*{\fill}

\begin{list}{}{
\settowidth{\tmplength}{\textbf{Declaração}}
\setlength{\itemindent}{0cm}
\setlength{\listparindent}{0cm}
\setlength{\leftmargin}{\evensidemargin}
\addtolength{\leftmargin}{\tmplength}
\settowidth{\labelsep}{X}
\addtolength{\leftmargin}{\labelsep}
\setlength{\labelwidth}{\tmplength}
}
\begin{flushleft}
\item[\textbf{Declaração}\hfill]
\begin{ttfamily}
public class Function GetExecAsync():Byte;\end{ttfamily}


\end{flushleft}
\end{list}
\paragraph*{ShellScript}\hspace*{\fill}

\begin{list}{}{
\settowidth{\tmplength}{\textbf{Declaração}}
\setlength{\itemindent}{0cm}
\setlength{\listparindent}{0cm}
\setlength{\leftmargin}{\evensidemargin}
\addtolength{\leftmargin}{\tmplength}
\settowidth{\labelsep}{X}
\addtolength{\leftmargin}{\labelsep}
\setlength{\labelwidth}{\tmplength}
}
\begin{flushleft}
\item[\textbf{Declaração}\hfill]
\begin{ttfamily}
public class function ShellScript(aCommand:String): String;\end{ttfamily}


\end{flushleft}
\par
\item[\textbf{Descrição}]
O método \begin{ttfamily}ShellScript\end{ttfamily} executa o shell do sistema operacional e retorna o Buffer da Tela

\begin{verbatim}Pascal

           program Project1;
             uses
               mi.rtl.objectss;

             var
               s : string;
           begin
             s := TObjectss.ShellScript('ls *.res');
             if s <>''
             then WriteLn(s);
           end.\end{verbatim}

\end{list}
\paragraph*{ShellExecute}\hspace*{\fill}

\begin{list}{}{
\settowidth{\tmplength}{\textbf{Declaração}}
\setlength{\itemindent}{0cm}
\setlength{\listparindent}{0cm}
\setlength{\leftmargin}{\evensidemargin}
\addtolength{\leftmargin}{\tmplength}
\settowidth{\labelsep}{X}
\addtolength{\leftmargin}{\labelsep}
\setlength{\labelwidth}{\tmplength}
}
\begin{flushleft}
\item[\textbf{Declaração}\hfill]
\begin{ttfamily}
public class function ShellExecute(Const lpOperation, FileName, Params, DefaultDir: AnsiString; ShowCmd: Integer): THandle; Overload;\end{ttfamily}


\end{flushleft}
\end{list}
\paragraph*{ShellExecute}\hspace*{\fill}

\begin{list}{}{
\settowidth{\tmplength}{\textbf{Declaração}}
\setlength{\itemindent}{0cm}
\setlength{\listparindent}{0cm}
\setlength{\leftmargin}{\evensidemargin}
\addtolength{\leftmargin}{\tmplength}
\settowidth{\labelsep}{X}
\addtolength{\leftmargin}{\labelsep}
\setlength{\labelwidth}{\tmplength}
}
\begin{flushleft}
\item[\textbf{Declaração}\hfill]
\begin{ttfamily}
public class function ShellExecute(const FileName, Params, DefaultDir: AnsiString;ShowCmd: Integer): THandle; Overload;\end{ttfamily}


\end{flushleft}
\end{list}
\paragraph*{ShellExecute}\hspace*{\fill}

\begin{list}{}{
\settowidth{\tmplength}{\textbf{Declaração}}
\setlength{\itemindent}{0cm}
\setlength{\listparindent}{0cm}
\setlength{\leftmargin}{\evensidemargin}
\addtolength{\leftmargin}{\tmplength}
\settowidth{\labelsep}{X}
\addtolength{\leftmargin}{\labelsep}
\setlength{\labelwidth}{\tmplength}
}
\begin{flushleft}
\item[\textbf{Declaração}\hfill]
\begin{ttfamily}
public class function ShellExecute(const FileName, Params: AnsiString): THandle; Overload;\end{ttfamily}


\end{flushleft}
\end{list}
\paragraph*{GetIpPub}\hspace*{\fill}

\begin{list}{}{
\settowidth{\tmplength}{\textbf{Declaração}}
\setlength{\itemindent}{0cm}
\setlength{\listparindent}{0cm}
\setlength{\leftmargin}{\evensidemargin}
\addtolength{\leftmargin}{\tmplength}
\settowidth{\labelsep}{X}
\addtolength{\leftmargin}{\labelsep}
\setlength{\labelwidth}{\tmplength}
}
\begin{flushleft}
\item[\textbf{Declaração}\hfill]
\begin{ttfamily}
public class Function GetIpPub:String;\end{ttfamily}


\end{flushleft}
\par
\item[\textbf{Descrição}]
A classe function \textbf{\begin{ttfamily}GetIpPub\end{ttfamily}} retorna o ip publico da máquina local

\end{list}
\paragraph*{StrToInt}\hspace*{\fill}

\begin{list}{}{
\settowidth{\tmplength}{\textbf{Declaração}}
\setlength{\itemindent}{0cm}
\setlength{\listparindent}{0cm}
\setlength{\leftmargin}{\evensidemargin}
\addtolength{\leftmargin}{\tmplength}
\settowidth{\labelsep}{X}
\addtolength{\leftmargin}{\labelsep}
\setlength{\labelwidth}{\tmplength}
}
\begin{flushleft}
\item[\textbf{Declaração}\hfill]
\begin{ttfamily}
public class function StrToInt(aStr:String):Int64;\end{ttfamily}


\end{flushleft}
\end{list}
\paragraph*{BooleanToStr}\hspace*{\fill}

\begin{list}{}{
\settowidth{\tmplength}{\textbf{Declaração}}
\setlength{\itemindent}{0cm}
\setlength{\listparindent}{0cm}
\setlength{\leftmargin}{\evensidemargin}
\addtolength{\leftmargin}{\tmplength}
\settowidth{\labelsep}{X}
\addtolength{\leftmargin}{\labelsep}
\setlength{\labelwidth}{\tmplength}
}
\begin{flushleft}
\item[\textbf{Declaração}\hfill]
\begin{ttfamily}
public class Function BooleanToStr(Const FieldData:Boolean):AnsiString;\end{ttfamily}


\end{flushleft}
\end{list}
\paragraph*{DelSpcED}\hspace*{\fill}

\begin{list}{}{
\settowidth{\tmplength}{\textbf{Declaração}}
\setlength{\itemindent}{0cm}
\setlength{\listparindent}{0cm}
\setlength{\leftmargin}{\evensidemargin}
\addtolength{\leftmargin}{\tmplength}
\settowidth{\labelsep}{X}
\addtolength{\leftmargin}{\labelsep}
\setlength{\labelwidth}{\tmplength}
}
\begin{flushleft}
\item[\textbf{Declaração}\hfill]
\begin{ttfamily}
public class function DelSpcED(campo : Ansistring): AnsiString;\end{ttfamily}


\end{flushleft}
\end{list}
\paragraph*{Delspace}\hspace*{\fill}

\begin{list}{}{
\settowidth{\tmplength}{\textbf{Declaração}}
\setlength{\itemindent}{0cm}
\setlength{\listparindent}{0cm}
\setlength{\leftmargin}{\evensidemargin}
\addtolength{\leftmargin}{\tmplength}
\settowidth{\labelsep}{X}
\addtolength{\leftmargin}{\labelsep}
\setlength{\labelwidth}{\tmplength}
}
\begin{flushleft}
\item[\textbf{Declaração}\hfill]
\begin{ttfamily}
public class function Delspace(campo : Ansistring):AnsiString;\end{ttfamily}


\end{flushleft}
\end{list}
\paragraph*{GetNameValid}\hspace*{\fill}

\begin{list}{}{
\settowidth{\tmplength}{\textbf{Declaração}}
\setlength{\itemindent}{0cm}
\setlength{\listparindent}{0cm}
\setlength{\leftmargin}{\evensidemargin}
\addtolength{\leftmargin}{\tmplength}
\settowidth{\labelsep}{X}
\addtolength{\leftmargin}{\labelsep}
\setlength{\labelwidth}{\tmplength}
}
\begin{flushleft}
\item[\textbf{Declaração}\hfill]
\begin{ttfamily}
public class function GetNameValid(aName:AnsiString):AnsiString;\end{ttfamily}


\end{flushleft}
\end{list}
\paragraph*{IsNumber{\_}Real}\hspace*{\fill}

\begin{list}{}{
\settowidth{\tmplength}{\textbf{Declaração}}
\setlength{\itemindent}{0cm}
\setlength{\listparindent}{0cm}
\setlength{\leftmargin}{\evensidemargin}
\addtolength{\leftmargin}{\tmplength}
\settowidth{\labelsep}{X}
\addtolength{\leftmargin}{\labelsep}
\setlength{\labelwidth}{\tmplength}
}
\begin{flushleft}
\item[\textbf{Declaração}\hfill]
\begin{ttfamily}
public class Function IsNumber{\_}Real(Const aTemplate : ShortString):Boolean;\end{ttfamily}


\end{flushleft}
\end{list}
\paragraph*{IsNumber}\hspace*{\fill}

\begin{list}{}{
\settowidth{\tmplength}{\textbf{Declaração}}
\setlength{\itemindent}{0cm}
\setlength{\listparindent}{0cm}
\setlength{\leftmargin}{\evensidemargin}
\addtolength{\leftmargin}{\tmplength}
\settowidth{\labelsep}{X}
\addtolength{\leftmargin}{\labelsep}
\setlength{\labelwidth}{\tmplength}
}
\begin{flushleft}
\item[\textbf{Declaração}\hfill]
\begin{ttfamily}
public class Function IsNumber(Const aTemplate : ShortString):Boolean;\end{ttfamily}


\end{flushleft}
\end{list}
\paragraph*{IsData}\hspace*{\fill}

\begin{list}{}{
\settowidth{\tmplength}{\textbf{Declaração}}
\setlength{\itemindent}{0cm}
\setlength{\listparindent}{0cm}
\setlength{\leftmargin}{\evensidemargin}
\addtolength{\leftmargin}{\tmplength}
\settowidth{\labelsep}{X}
\addtolength{\leftmargin}{\labelsep}
\setlength{\labelwidth}{\tmplength}
}
\begin{flushleft}
\item[\textbf{Declaração}\hfill]
\begin{ttfamily}
public class Function IsData(Const aTemplate : ShortString):Boolean;\end{ttfamily}


\end{flushleft}
\end{list}
\paragraph*{IsHora}\hspace*{\fill}

\begin{list}{}{
\settowidth{\tmplength}{\textbf{Declaração}}
\setlength{\itemindent}{0cm}
\setlength{\listparindent}{0cm}
\setlength{\leftmargin}{\evensidemargin}
\addtolength{\leftmargin}{\tmplength}
\settowidth{\labelsep}{X}
\addtolength{\leftmargin}{\labelsep}
\setlength{\labelwidth}{\tmplength}
}
\begin{flushleft}
\item[\textbf{Declaração}\hfill]
\begin{ttfamily}
public class Function IsHora(Const aTemplate : ShortString):Boolean;\end{ttfamily}


\end{flushleft}
\end{list}
\paragraph*{HandleEvent}\hspace*{\fill}

\begin{list}{}{
\settowidth{\tmplength}{\textbf{Declaração}}
\setlength{\itemindent}{0cm}
\setlength{\listparindent}{0cm}
\setlength{\leftmargin}{\evensidemargin}
\addtolength{\leftmargin}{\tmplength}
\settowidth{\labelsep}{X}
\addtolength{\leftmargin}{\labelsep}
\setlength{\labelwidth}{\tmplength}
}
\begin{flushleft}
\item[\textbf{Declaração}\hfill]
\begin{ttfamily}
public procedure HandleEvent(var Event: TEvent); Virtual;\end{ttfamily}


\end{flushleft}
\end{list}
\paragraph*{ClearEvent}\hspace*{\fill}

\begin{list}{}{
\settowidth{\tmplength}{\textbf{Declaração}}
\setlength{\itemindent}{0cm}
\setlength{\listparindent}{0cm}
\setlength{\leftmargin}{\evensidemargin}
\addtolength{\leftmargin}{\tmplength}
\settowidth{\labelsep}{X}
\addtolength{\leftmargin}{\labelsep}
\setlength{\labelwidth}{\tmplength}
}
\begin{flushleft}
\item[\textbf{Declaração}\hfill]
\begin{ttfamily}
public procedure ClearEvent(var Event: TEvent); Virtual;\end{ttfamily}


\end{flushleft}
\end{list}
\paragraph*{Change{\_}AnsiChar}\hspace*{\fill}

\begin{list}{}{
\settowidth{\tmplength}{\textbf{Declaração}}
\setlength{\itemindent}{0cm}
\setlength{\listparindent}{0cm}
\setlength{\leftmargin}{\evensidemargin}
\addtolength{\leftmargin}{\tmplength}
\settowidth{\labelsep}{X}
\addtolength{\leftmargin}{\labelsep}
\setlength{\labelwidth}{\tmplength}
}
\begin{flushleft}
\item[\textbf{Declaração}\hfill]
\begin{ttfamily}
public class Function Change{\_}AnsiChar(campo : AnsiString; Const AnsiChar{\_}Font,AnsiChar{\_}Dest : AnsiChar):AnsiString;\end{ttfamily}


\end{flushleft}
\end{list}
\paragraph*{DeleteMask}\hspace*{\fill}

\begin{list}{}{
\settowidth{\tmplength}{\textbf{Declaração}}
\setlength{\itemindent}{0cm}
\setlength{\listparindent}{0cm}
\setlength{\leftmargin}{\evensidemargin}
\addtolength{\leftmargin}{\tmplength}
\settowidth{\labelsep}{X}
\addtolength{\leftmargin}{\labelsep}
\setlength{\labelwidth}{\tmplength}
}
\begin{flushleft}
\item[\textbf{Declaração}\hfill]
\begin{ttfamily}
public class Function DeleteMask(S : tString;ValidSet: AnsiCharSet):AnsiString; overload;\end{ttfamily}


\end{flushleft}
\end{list}
\paragraph*{DeleteMask}\hspace*{\fill}

\begin{list}{}{
\settowidth{\tmplength}{\textbf{Declaração}}
\setlength{\itemindent}{0cm}
\setlength{\listparindent}{0cm}
\setlength{\leftmargin}{\evensidemargin}
\addtolength{\leftmargin}{\tmplength}
\settowidth{\labelsep}{X}
\addtolength{\leftmargin}{\labelsep}
\setlength{\labelwidth}{\tmplength}
}
\begin{flushleft}
\item[\textbf{Declaração}\hfill]
\begin{ttfamily}
public class function DeleteMask(S: ShortString;aMask:ShortString): AnsiString; overload;\end{ttfamily}


\end{flushleft}
\end{list}
\paragraph*{AddMask}\hspace*{\fill}

\begin{list}{}{
\settowidth{\tmplength}{\textbf{Declaração}}
\setlength{\itemindent}{0cm}
\setlength{\listparindent}{0cm}
\setlength{\leftmargin}{\evensidemargin}
\addtolength{\leftmargin}{\tmplength}
\settowidth{\labelsep}{X}
\addtolength{\leftmargin}{\labelsep}
\setlength{\labelwidth}{\tmplength}
}
\begin{flushleft}
\item[\textbf{Declaração}\hfill]
\begin{ttfamily}
public class function AddMask(S: ShortString;aMask:ShortString): AnsiString;\end{ttfamily}


\end{flushleft}
\end{list}
\paragraph*{CreateDB{\_}or{\_}DropDB}\hspace*{\fill}

\begin{list}{}{
\settowidth{\tmplength}{\textbf{Declaração}}
\setlength{\itemindent}{0cm}
\setlength{\listparindent}{0cm}
\setlength{\leftmargin}{\evensidemargin}
\addtolength{\leftmargin}{\tmplength}
\settowidth{\labelsep}{X}
\addtolength{\leftmargin}{\labelsep}
\setlength{\labelwidth}{\tmplength}
}
\begin{flushleft}
\item[\textbf{Declaração}\hfill]
\begin{ttfamily}
public class function CreateDB{\_}or{\_}DropDB(aConnectorType : TConnectorType; aHostname, aUserName, aPassword, aDataBaseName : String; okCreateDB:Boolean ):string;\end{ttfamily}


\end{flushleft}
\par
\item[\textbf{Descrição}]
A função \textbf{\begin{ttfamily}CreateDB{\_}or{\_}DropDB\end{ttfamily}} é usada para criar ou apagar um banco de dados

\begin{itemize}
\item \textbf{Banco de dados possíveis:} \begin{itemize}
\item PostreSQL;
\item SqlLite;
\end{itemize}
\end{itemize}\begin{itemize}
\item \textbf{Retorna} \begin{itemize}
\item True : Conseguiu criar o banco de dados;
\item False : Error na ação \begin{itemize}
\item Error possiveis: \begin{itemize}
\item Banco de dados já existe quando se quer criar;
\item Banco de dados não existe quando se quer apagar;
\item Banco de dados usado por outro usuário.
\end{itemize}
\end{itemize}
\end{itemize}
\item \textbf{EXEMPLO}

\texttt{\\\nopagebreak[3]
\\\nopagebreak[3]
\textit{//~Cria~banco~de~dados~maricarai~no~postgresSql}\\\nopagebreak[3]
}\textbf{Procedure}\texttt{~TForm1.Button2sqlPQConectionClick~(~Sender~:~TObject~)~;\\\nopagebreak[3]
~~}\textbf{var}\texttt{\\\nopagebreak[3]
~~~~s~:~}\textbf{String}\texttt{;\\\nopagebreak[3]
}\textbf{begin}\texttt{\\\nopagebreak[3]
~~s~:=~CreateDB{\_}or{\_}DropDB(PostgresSQL,'127.0.0.1',\\\nopagebreak[3]
~~~~~~~~~~~~~~~~~~~~~~~~~~~~~~~~~~~~'postgres',\\\nopagebreak[3]
~~~~~~~~~~~~~~~~~~~~~~~~~~~~~~~~~~~~'masterkey',\\\nopagebreak[3]
~~~~~~~~~~~~~~~~~~~~~~~~~~~~~~~~~~~~'maricarai',\\\nopagebreak[3]
~~~~~~~~~~~~~~~~~~~~~~~~~~~~~~~~~~~~true);\\\nopagebreak[3]
~~}\textbf{if}\texttt{~s~=~''\\\nopagebreak[3]
~~}\textbf{then}\texttt{~ShowMessage('Banco~de~dados~maricarai~foi~criado~no~postgresSql')\\\nopagebreak[3]
~~}\textbf{else}\texttt{~ShowMessage(s);\\\nopagebreak[3]
}\textbf{End}\texttt{;\\\nopagebreak[3]
\\\nopagebreak[3]
\textit{//~Apaga~banco~de~dados~maricarai~no~SqLite3}\\\nopagebreak[3]
}\textbf{Procedure}\texttt{~TForm1.Button2sqlPQConectionClick~(~Sender~:~TObject~)~;\\\nopagebreak[3]
~~}\textbf{var}\texttt{\\\nopagebreak[3]
~~~~s~:~}\textbf{String}\texttt{;\\\nopagebreak[3]
}\textbf{begin}\texttt{\\\nopagebreak[3]
~~s:=~CreateDB{\_}or{\_}DropDB(PostgresSQL,'127.0.0.1',\\\nopagebreak[3]
~~~~~~~~~~~~~~~~~~~~~~~~~~~~~~~~~~~~'SqLite3',\\\nopagebreak[3]
~~~~~~~~~~~~~~~~~~~~~~~~~~~~~~~~~~~~'masterkey',\\\nopagebreak[3]
~~~~~~~~~~~~~~~~~~~~~~~~~~~~~~~~~~~~'maricarai',\\\nopagebreak[3]
~~~~~~~~~~~~~~~~~~~~~~~~~~~~~~~~~~~~false);\\\nopagebreak[3]
~~}\textbf{if}\texttt{~s~=~''\\\nopagebreak[3]
~~}\textbf{then}\texttt{~ShowMessage('Banco~de~dados~maricarai~foi~apagado~SqLite3')\\\nopagebreak[3]
~~}\textbf{else}\texttt{~ShowMessage(s);\\\nopagebreak[3]
}\textbf{End}\texttt{;\\
}
\end{itemize}

\end{list}
\paragraph*{StrNumberValid}\hspace*{\fill}

\begin{list}{}{
\settowidth{\tmplength}{\textbf{Declaração}}
\setlength{\itemindent}{0cm}
\setlength{\listparindent}{0cm}
\setlength{\leftmargin}{\evensidemargin}
\addtolength{\leftmargin}{\tmplength}
\settowidth{\labelsep}{X}
\addtolength{\leftmargin}{\labelsep}
\setlength{\labelwidth}{\tmplength}
}
\begin{flushleft}
\item[\textbf{Declaração}\hfill]
\begin{ttfamily}
public class function StrNumberValid(S: AnsiString): AnsiString;\end{ttfamily}


\end{flushleft}
\par
\item[\textbf{Descrição}]
O método \begin{ttfamily}StrNumberValid\end{ttfamily} remove as mascaras do número e retorna somente números

\end{list}
\paragraph*{CheckRanger}\hspace*{\fill}

\begin{list}{}{
\settowidth{\tmplength}{\textbf{Declaração}}
\setlength{\itemindent}{0cm}
\setlength{\listparindent}{0cm}
\setlength{\leftmargin}{\evensidemargin}
\addtolength{\leftmargin}{\tmplength}
\settowidth{\labelsep}{X}
\addtolength{\leftmargin}{\labelsep}
\setlength{\labelwidth}{\tmplength}
}
\begin{flushleft}
\item[\textbf{Declaração}\hfill]
\begin{ttfamily}
public class function CheckRanger(S : AnsiString; aHigh, aLow: Int64; out aErr:Integer): Int64;\end{ttfamily}


\end{flushleft}
\par
\item[\textbf{Descrição}]
o classe método \textbf{\begin{ttfamily}CheckRanger\end{ttfamily}} checa se s está entre aHigh e aLow retorna zero se houver erro e em aErr o código do erro.

\end{list}
\paragraph*{IntValid}\hspace*{\fill}

\begin{list}{}{
\settowidth{\tmplength}{\textbf{Declaração}}
\setlength{\itemindent}{0cm}
\setlength{\listparindent}{0cm}
\setlength{\leftmargin}{\evensidemargin}
\addtolength{\leftmargin}{\tmplength}
\settowidth{\labelsep}{X}
\addtolength{\leftmargin}{\labelsep}
\setlength{\labelwidth}{\tmplength}
}
\begin{flushleft}
\item[\textbf{Declaração}\hfill]
\begin{ttfamily}
public class function IntValid(S : AnsiString; TypeCode:AnsiChar):Boolean;\end{ttfamily}


\end{flushleft}
\par
\item[\textbf{Descrição}]
O método \begin{ttfamily}IntValid\end{ttfamily} retorna \textbf{TRUE} se o parâmetro S for número inteiro ou \textbf{FALSE} caso contrário.

\end{list}
\paragraph*{ShowHtml}\hspace*{\fill}

\begin{list}{}{
\settowidth{\tmplength}{\textbf{Declaração}}
\setlength{\itemindent}{0cm}
\setlength{\listparindent}{0cm}
\setlength{\leftmargin}{\evensidemargin}
\addtolength{\leftmargin}{\tmplength}
\settowidth{\labelsep}{X}
\addtolength{\leftmargin}{\labelsep}
\setlength{\labelwidth}{\tmplength}
}
\begin{flushleft}
\item[\textbf{Declaração}\hfill]
\begin{ttfamily}
public class Procedure ShowHtml(URL:string); virtual;\end{ttfamily}


\end{flushleft}
\par
\item[\textbf{Descrição}]
O método \begin{ttfamily}ShowHtml\end{ttfamily} Executa o browser padrão do sistema operacional.

\end{list}
\chapter{Unit mi.rtl.Objects.Methods.Collection}
\section{Descrição}
\begin{itemize}
\item A Unit \textbf{\begin{ttfamily}mi.rtl.Objects.Methods.Collection\end{ttfamily}} implementa a classe \textbf{\begin{ttfamily}TCollection\end{ttfamily}(\ref{mi.rtl.Objects.Methods.Collection.TCollection})} do pacote \textbf{\begin{ttfamily}mi.rtl\end{ttfamily}(\ref{mi.rtl})}.

\begin{itemize}
\item \textbf{NOTAS} \begin{itemize}
\item Esta unit foi testada nas plataformas: linux.
\end{itemize}
\item \textbf{VERSÃO} \begin{itemize}
\item Alpha {-} 0.5.0.687
\end{itemize}
\item \textbf{CÓDIGO FONTE}: \begin{itemize}
\item 
\end{itemize}
\item \textbf{HISTÓRICO} \begin{itemize}
\item Criado por: Paulo Sérgio da Silva Pacheco e{-}mail: paulosspacheco@yahoo.com.br \begin{itemize}
\item \textbf{20/11/2021} 10:12 a 22:49 : Criada a classe \begin{ttfamily}mi.rtl.Objects.Methods.Collection\end{ttfamily}
\end{itemize}
\end{itemize}
\end{itemize}
\end{itemize}
\section{Uses}
\begin{itemize}
\item \begin{ttfamily}Classes\end{ttfamily}\item \begin{ttfamily}SysUtils\end{ttfamily}\item \begin{ttfamily}mi.rtl.objects.Methods\end{ttfamily}(\ref{mi.rtl.Objects.Methods})\item \begin{ttfamily}mi.rtl.objects.methods.StreamBase.Stream\end{ttfamily}(\ref{mi.rtl.Objects.Methods.StreamBase.Stream})\end{itemize}
\section{Visão Geral}
\begin{description}
\item[\texttt{\begin{ttfamily}TCollection\end{ttfamily} Classe}]
\end{description}
\section{Classes, Interfaces, Objetos e Registros}
\subsection*{TCollection Classe}
\subsubsection*{\large{\textbf{Hierarquia}}\normalsize\hspace{1ex}\hfill}
TCollection {$>$} \begin{ttfamily}TObjectsMethods\end{ttfamily}(\ref{mi.rtl.Objects.Methods.TObjectsMethods}) {$>$} \begin{ttfamily}TObjectsConsts\end{ttfamily}(\ref{mi.rtl.Objects.Consts.TObjectsConsts}) {$>$} 
TObjectsTypes
\subsubsection*{\large{\textbf{Descrição}}\normalsize\hspace{1ex}\hfill}
\begin{itemize}
\item A class \textbf{\begin{ttfamily}TCollection\end{ttfamily}} implementa coleções no pacote \textbf{\begin{ttfamily}mi.rtl\end{ttfamily}(\ref{mi.rtl})}.
\end{itemize}\subsubsection*{\large{\textbf{Propriedades}}\normalsize\hspace{1ex}\hfill}
\paragraph*{Count}\hspace*{\fill}

\begin{list}{}{
\settowidth{\tmplength}{\textbf{Declaração}}
\setlength{\itemindent}{0cm}
\setlength{\listparindent}{0cm}
\setlength{\leftmargin}{\evensidemargin}
\addtolength{\leftmargin}{\tmplength}
\settowidth{\labelsep}{X}
\addtolength{\leftmargin}{\labelsep}
\setlength{\labelwidth}{\tmplength}
}
\begin{flushleft}
\item[\textbf{Declaração}\hfill]
\begin{ttfamily}
published property Count: Sw{\_}Integer 
                Read {\_}Count write {\_}Count;\end{ttfamily}


\end{flushleft}
\par
\item[\textbf{Descrição}]
Item \begin{ttfamily}count\end{ttfamily}

\end{list}
\subsubsection*{\large{\textbf{Campos}}\normalsize\hspace{1ex}\hfill}
\paragraph*{Items}\hspace*{\fill}

\begin{list}{}{
\settowidth{\tmplength}{\textbf{Declaração}}
\setlength{\itemindent}{0cm}
\setlength{\listparindent}{0cm}
\setlength{\leftmargin}{\evensidemargin}
\addtolength{\leftmargin}{\tmplength}
\settowidth{\labelsep}{X}
\addtolength{\leftmargin}{\labelsep}
\setlength{\labelwidth}{\tmplength}
}
\begin{flushleft}
\item[\textbf{Declaração}\hfill]
\begin{ttfamily}
public var Items: PItemList;\end{ttfamily}


\end{flushleft}
\par
\item[\textbf{Descrição}]
Item list pointer

\end{list}
\paragraph*{State}\hspace*{\fill}

\begin{list}{}{
\settowidth{\tmplength}{\textbf{Declaração}}
\setlength{\itemindent}{0cm}
\setlength{\listparindent}{0cm}
\setlength{\leftmargin}{\evensidemargin}
\addtolength{\leftmargin}{\tmplength}
\settowidth{\labelsep}{X}
\addtolength{\leftmargin}{\labelsep}
\setlength{\labelwidth}{\tmplength}
}
\begin{flushleft}
\item[\textbf{Declaração}\hfill]
\begin{ttfamily}
public var State: Longint;\end{ttfamily}


\end{flushleft}
\end{list}
\paragraph*{Limit}\hspace*{\fill}

\begin{list}{}{
\settowidth{\tmplength}{\textbf{Declaração}}
\setlength{\itemindent}{0cm}
\setlength{\listparindent}{0cm}
\setlength{\leftmargin}{\evensidemargin}
\addtolength{\leftmargin}{\tmplength}
\settowidth{\labelsep}{X}
\addtolength{\leftmargin}{\labelsep}
\setlength{\labelwidth}{\tmplength}
}
\begin{flushleft}
\item[\textbf{Declaração}\hfill]
\begin{ttfamily}
public Limit: Sw{\_}Integer;\end{ttfamily}


\end{flushleft}
\par
\item[\textbf{Descrição}]
Item \begin{ttfamily}limit\end{ttfamily} \begin{ttfamily}count\end{ttfamily}(\ref{mi.rtl.Objects.Methods.Collection.TCollection-Count})

\end{list}
\paragraph*{Delta}\hspace*{\fill}

\begin{list}{}{
\settowidth{\tmplength}{\textbf{Declaração}}
\setlength{\itemindent}{0cm}
\setlength{\listparindent}{0cm}
\setlength{\leftmargin}{\evensidemargin}
\addtolength{\leftmargin}{\tmplength}
\settowidth{\labelsep}{X}
\addtolength{\leftmargin}{\labelsep}
\setlength{\labelwidth}{\tmplength}
}
\begin{flushleft}
\item[\textbf{Declaração}\hfill]
\begin{ttfamily}
public Delta: Sw{\_}Integer;\end{ttfamily}


\end{flushleft}
\par
\item[\textbf{Descrição}]
Inc \begin{ttfamily}delta\end{ttfamily} size

\end{list}
\paragraph*{Status}\hspace*{\fill}

\begin{list}{}{
\settowidth{\tmplength}{\textbf{Declaração}}
\setlength{\itemindent}{0cm}
\setlength{\listparindent}{0cm}
\setlength{\leftmargin}{\evensidemargin}
\addtolength{\leftmargin}{\tmplength}
\settowidth{\labelsep}{X}
\addtolength{\leftmargin}{\labelsep}
\setlength{\labelwidth}{\tmplength}
}
\begin{flushleft}
\item[\textbf{Declaração}\hfill]
\begin{ttfamily}
public Status: Integer;\end{ttfamily}


\end{flushleft}
\par
\item[\textbf{Descrição}]
\begin{ttfamily}TCollection\end{ttfamily}(\ref{mi.rtl.Objects.Methods.Collection.TCollection}) \begin{ttfamily}status\end{ttfamily}

\end{list}
\paragraph*{ErrorInfo}\hspace*{\fill}

\begin{list}{}{
\settowidth{\tmplength}{\textbf{Declaração}}
\setlength{\itemindent}{0cm}
\setlength{\listparindent}{0cm}
\setlength{\leftmargin}{\evensidemargin}
\addtolength{\leftmargin}{\tmplength}
\settowidth{\labelsep}{X}
\addtolength{\leftmargin}{\labelsep}
\setlength{\labelwidth}{\tmplength}
}
\begin{flushleft}
\item[\textbf{Declaração}\hfill]
\begin{ttfamily}
public ErrorInfo: Integer;\end{ttfamily}


\end{flushleft}
\par
\item[\textbf{Descrição}]
\begin{ttfamily}TCollection\end{ttfamily}(\ref{mi.rtl.Objects.Methods.Collection.TCollection}) \begin{ttfamily}error\end{ttfamily}(\ref{mi.rtl.Objects.Methods.Collection.TCollection-Error}) info

\end{list}
\subsubsection*{\large{\textbf{Métodos}}\normalsize\hspace{1ex}\hfill}
\paragraph*{Create}\hspace*{\fill}

\begin{list}{}{
\settowidth{\tmplength}{\textbf{Declaração}}
\setlength{\itemindent}{0cm}
\setlength{\listparindent}{0cm}
\setlength{\leftmargin}{\evensidemargin}
\addtolength{\leftmargin}{\tmplength}
\settowidth{\labelsep}{X}
\addtolength{\leftmargin}{\labelsep}
\setlength{\labelwidth}{\tmplength}
}
\begin{flushleft}
\item[\textbf{Declaração}\hfill]
\begin{ttfamily}
public constructor Create(ALimit, ADelta: Sw{\_}Integer); overload; virtual;\end{ttfamily}


\end{flushleft}
\end{list}
\paragraph*{Destroy}\hspace*{\fill}

\begin{list}{}{
\settowidth{\tmplength}{\textbf{Declaração}}
\setlength{\itemindent}{0cm}
\setlength{\listparindent}{0cm}
\setlength{\leftmargin}{\evensidemargin}
\addtolength{\leftmargin}{\tmplength}
\settowidth{\labelsep}{X}
\addtolength{\leftmargin}{\labelsep}
\setlength{\labelwidth}{\tmplength}
}
\begin{flushleft}
\item[\textbf{Declaração}\hfill]
\begin{ttfamily}
public destructor Destroy; Override;\end{ttfamily}


\end{flushleft}
\end{list}
\paragraph*{IndexOf}\hspace*{\fill}

\begin{list}{}{
\settowidth{\tmplength}{\textbf{Declaração}}
\setlength{\itemindent}{0cm}
\setlength{\listparindent}{0cm}
\setlength{\leftmargin}{\evensidemargin}
\addtolength{\leftmargin}{\tmplength}
\settowidth{\labelsep}{X}
\addtolength{\leftmargin}{\labelsep}
\setlength{\labelwidth}{\tmplength}
}
\begin{flushleft}
\item[\textbf{Declaração}\hfill]
\begin{ttfamily}
protected function IndexOf(Item: Pointer): Sw{\_}Integer; Virtual;\end{ttfamily}


\end{flushleft}
\end{list}
\paragraph*{GetItem}\hspace*{\fill}

\begin{list}{}{
\settowidth{\tmplength}{\textbf{Declaração}}
\setlength{\itemindent}{0cm}
\setlength{\listparindent}{0cm}
\setlength{\leftmargin}{\evensidemargin}
\addtolength{\leftmargin}{\tmplength}
\settowidth{\labelsep}{X}
\addtolength{\leftmargin}{\labelsep}
\setlength{\labelwidth}{\tmplength}
}
\begin{flushleft}
\item[\textbf{Declaração}\hfill]
\begin{ttfamily}
protected function GetItem(Var S: tStream ): Pointer; Virtual;\end{ttfamily}


\end{flushleft}
\end{list}
\paragraph*{Insert}\hspace*{\fill}

\begin{list}{}{
\settowidth{\tmplength}{\textbf{Declaração}}
\setlength{\itemindent}{0cm}
\setlength{\listparindent}{0cm}
\setlength{\leftmargin}{\evensidemargin}
\addtolength{\leftmargin}{\tmplength}
\settowidth{\labelsep}{X}
\addtolength{\leftmargin}{\labelsep}
\setlength{\labelwidth}{\tmplength}
}
\begin{flushleft}
\item[\textbf{Declaração}\hfill]
\begin{ttfamily}
protected procedure Insert(Item: Pointer); Virtual;\end{ttfamily}


\end{flushleft}
\end{list}
\paragraph*{FreeItem}\hspace*{\fill}

\begin{list}{}{
\settowidth{\tmplength}{\textbf{Declaração}}
\setlength{\itemindent}{0cm}
\setlength{\listparindent}{0cm}
\setlength{\leftmargin}{\evensidemargin}
\addtolength{\leftmargin}{\tmplength}
\settowidth{\labelsep}{X}
\addtolength{\leftmargin}{\labelsep}
\setlength{\labelwidth}{\tmplength}
}
\begin{flushleft}
\item[\textbf{Declaração}\hfill]
\begin{ttfamily}
protected procedure FreeItem(Item: Pointer); Virtual;\end{ttfamily}


\end{flushleft}
\end{list}
\paragraph*{SetLimit}\hspace*{\fill}

\begin{list}{}{
\settowidth{\tmplength}{\textbf{Declaração}}
\setlength{\itemindent}{0cm}
\setlength{\listparindent}{0cm}
\setlength{\leftmargin}{\evensidemargin}
\addtolength{\leftmargin}{\tmplength}
\settowidth{\labelsep}{X}
\addtolength{\leftmargin}{\labelsep}
\setlength{\labelwidth}{\tmplength}
}
\begin{flushleft}
\item[\textbf{Declaração}\hfill]
\begin{ttfamily}
protected procedure SetLimit(ALimit: Sw{\_}Integer); Virtual;\end{ttfamily}


\end{flushleft}
\end{list}
\paragraph*{Error}\hspace*{\fill}

\begin{list}{}{
\settowidth{\tmplength}{\textbf{Declaração}}
\setlength{\itemindent}{0cm}
\setlength{\listparindent}{0cm}
\setlength{\leftmargin}{\evensidemargin}
\addtolength{\leftmargin}{\tmplength}
\settowidth{\labelsep}{X}
\addtolength{\leftmargin}{\labelsep}
\setlength{\labelwidth}{\tmplength}
}
\begin{flushleft}
\item[\textbf{Declaração}\hfill]
\begin{ttfamily}
protected procedure Error(Code, Info: Integer); Virtual;\end{ttfamily}


\end{flushleft}
\end{list}
\paragraph*{PutItem}\hspace*{\fill}

\begin{list}{}{
\settowidth{\tmplength}{\textbf{Declaração}}
\setlength{\itemindent}{0cm}
\setlength{\listparindent}{0cm}
\setlength{\leftmargin}{\evensidemargin}
\addtolength{\leftmargin}{\tmplength}
\settowidth{\labelsep}{X}
\addtolength{\leftmargin}{\labelsep}
\setlength{\labelwidth}{\tmplength}
}
\begin{flushleft}
\item[\textbf{Declaração}\hfill]
\begin{ttfamily}
protected procedure PutItem(Var S: tStream ; Item: Pointer); Virtual;\end{ttfamily}


\end{flushleft}
\end{list}
\paragraph*{Create{\_}Progress1Passo}\hspace*{\fill}

\begin{list}{}{
\settowidth{\tmplength}{\textbf{Declaração}}
\setlength{\itemindent}{0cm}
\setlength{\listparindent}{0cm}
\setlength{\leftmargin}{\evensidemargin}
\addtolength{\leftmargin}{\tmplength}
\settowidth{\labelsep}{X}
\addtolength{\leftmargin}{\labelsep}
\setlength{\labelwidth}{\tmplength}
}
\begin{flushleft}
\item[\textbf{Declaração}\hfill]
\begin{ttfamily}
public procedure Create{\_}Progress1Passo(ATitle : tstring;Obs:tstring ; ATotal : Longint); Virtual;\end{ttfamily}


\end{flushleft}
\end{list}
\paragraph*{Set{\_}Progress1Passo}\hspace*{\fill}

\begin{list}{}{
\settowidth{\tmplength}{\textbf{Declaração}}
\setlength{\itemindent}{0cm}
\setlength{\listparindent}{0cm}
\setlength{\leftmargin}{\evensidemargin}
\addtolength{\leftmargin}{\tmplength}
\settowidth{\labelsep}{X}
\addtolength{\leftmargin}{\labelsep}
\setlength{\labelwidth}{\tmplength}
}
\begin{flushleft}
\item[\textbf{Declaração}\hfill]
\begin{ttfamily}
public procedure Set{\_}Progress1Passo(aNumber : Longint); Virtual;\end{ttfamily}


\end{flushleft}
\end{list}
\paragraph*{Destroy{\_}Progress1Passo}\hspace*{\fill}

\begin{list}{}{
\settowidth{\tmplength}{\textbf{Declaração}}
\setlength{\itemindent}{0cm}
\setlength{\listparindent}{0cm}
\setlength{\leftmargin}{\evensidemargin}
\addtolength{\leftmargin}{\tmplength}
\settowidth{\labelsep}{X}
\addtolength{\leftmargin}{\labelsep}
\setlength{\labelwidth}{\tmplength}
}
\begin{flushleft}
\item[\textbf{Declaração}\hfill]
\begin{ttfamily}
public procedure Destroy{\_}Progress1Passo; Virtual;\end{ttfamily}


\end{flushleft}
\end{list}
\paragraph*{MessageBox}\hspace*{\fill}

\begin{list}{}{
\settowidth{\tmplength}{\textbf{Declaração}}
\setlength{\itemindent}{0cm}
\setlength{\listparindent}{0cm}
\setlength{\leftmargin}{\evensidemargin}
\addtolength{\leftmargin}{\tmplength}
\settowidth{\labelsep}{X}
\addtolength{\leftmargin}{\labelsep}
\setlength{\labelwidth}{\tmplength}
}
\begin{flushleft}
\item[\textbf{Declaração}\hfill]
\begin{ttfamily}
public function MessageBox(const Msg: AnsiString): Word; Virtual;\end{ttfamily}


\end{flushleft}
\end{list}
\paragraph*{At}\hspace*{\fill}

\begin{list}{}{
\settowidth{\tmplength}{\textbf{Declaração}}
\setlength{\itemindent}{0cm}
\setlength{\listparindent}{0cm}
\setlength{\leftmargin}{\evensidemargin}
\addtolength{\leftmargin}{\tmplength}
\settowidth{\labelsep}{X}
\addtolength{\leftmargin}{\labelsep}
\setlength{\labelwidth}{\tmplength}
}
\begin{flushleft}
\item[\textbf{Declaração}\hfill]
\begin{ttfamily}
public function At(Index: Sw{\_}Integer): Pointer;\end{ttfamily}


\end{flushleft}
\end{list}
\paragraph*{LastThat}\hspace*{\fill}

\begin{list}{}{
\settowidth{\tmplength}{\textbf{Declaração}}
\setlength{\itemindent}{0cm}
\setlength{\listparindent}{0cm}
\setlength{\leftmargin}{\evensidemargin}
\addtolength{\leftmargin}{\tmplength}
\settowidth{\labelsep}{X}
\addtolength{\leftmargin}{\labelsep}
\setlength{\labelwidth}{\tmplength}
}
\begin{flushleft}
\item[\textbf{Declaração}\hfill]
\begin{ttfamily}
public function LastThat(Test: TCallbackFunBoolParam): Pointer;\end{ttfamily}


\end{flushleft}
\end{list}
\paragraph*{FirstThat}\hspace*{\fill}

\begin{list}{}{
\settowidth{\tmplength}{\textbf{Declaração}}
\setlength{\itemindent}{0cm}
\setlength{\listparindent}{0cm}
\setlength{\leftmargin}{\evensidemargin}
\addtolength{\leftmargin}{\tmplength}
\settowidth{\labelsep}{X}
\addtolength{\leftmargin}{\labelsep}
\setlength{\labelwidth}{\tmplength}
}
\begin{flushleft}
\item[\textbf{Declaração}\hfill]
\begin{ttfamily}
public function FirstThat(Test: Pointer): Pointer;\end{ttfamily}


\end{flushleft}
\end{list}
\paragraph*{Pack}\hspace*{\fill}

\begin{list}{}{
\settowidth{\tmplength}{\textbf{Declaração}}
\setlength{\itemindent}{0cm}
\setlength{\listparindent}{0cm}
\setlength{\leftmargin}{\evensidemargin}
\addtolength{\leftmargin}{\tmplength}
\settowidth{\labelsep}{X}
\addtolength{\leftmargin}{\labelsep}
\setlength{\labelwidth}{\tmplength}
}
\begin{flushleft}
\item[\textbf{Declaração}\hfill]
\begin{ttfamily}
public procedure Pack;\end{ttfamily}


\end{flushleft}
\end{list}
\paragraph*{FreeAll}\hspace*{\fill}

\begin{list}{}{
\settowidth{\tmplength}{\textbf{Declaração}}
\setlength{\itemindent}{0cm}
\setlength{\listparindent}{0cm}
\setlength{\leftmargin}{\evensidemargin}
\addtolength{\leftmargin}{\tmplength}
\settowidth{\labelsep}{X}
\addtolength{\leftmargin}{\labelsep}
\setlength{\labelwidth}{\tmplength}
}
\begin{flushleft}
\item[\textbf{Declaração}\hfill]
\begin{ttfamily}
public procedure FreeAll; Virtual;\end{ttfamily}


\end{flushleft}
\end{list}
\paragraph*{DeleteAll}\hspace*{\fill}

\begin{list}{}{
\settowidth{\tmplength}{\textbf{Declaração}}
\setlength{\itemindent}{0cm}
\setlength{\listparindent}{0cm}
\setlength{\leftmargin}{\evensidemargin}
\addtolength{\leftmargin}{\tmplength}
\settowidth{\labelsep}{X}
\addtolength{\leftmargin}{\labelsep}
\setlength{\labelwidth}{\tmplength}
}
\begin{flushleft}
\item[\textbf{Declaração}\hfill]
\begin{ttfamily}
public procedure DeleteAll;\end{ttfamily}


\end{flushleft}
\end{list}
\paragraph*{Free}\hspace*{\fill}

\begin{list}{}{
\settowidth{\tmplength}{\textbf{Declaração}}
\setlength{\itemindent}{0cm}
\setlength{\listparindent}{0cm}
\setlength{\leftmargin}{\evensidemargin}
\addtolength{\leftmargin}{\tmplength}
\settowidth{\labelsep}{X}
\addtolength{\leftmargin}{\labelsep}
\setlength{\labelwidth}{\tmplength}
}
\begin{flushleft}
\item[\textbf{Declaração}\hfill]
\begin{ttfamily}
public procedure Free(Item: Pointer);\end{ttfamily}


\end{flushleft}
\end{list}
\paragraph*{Delete}\hspace*{\fill}

\begin{list}{}{
\settowidth{\tmplength}{\textbf{Declaração}}
\setlength{\itemindent}{0cm}
\setlength{\listparindent}{0cm}
\setlength{\leftmargin}{\evensidemargin}
\addtolength{\leftmargin}{\tmplength}
\settowidth{\labelsep}{X}
\addtolength{\leftmargin}{\labelsep}
\setlength{\labelwidth}{\tmplength}
}
\begin{flushleft}
\item[\textbf{Declaração}\hfill]
\begin{ttfamily}
public procedure Delete(Item: Pointer);\end{ttfamily}


\end{flushleft}
\end{list}
\paragraph*{AtFree}\hspace*{\fill}

\begin{list}{}{
\settowidth{\tmplength}{\textbf{Declaração}}
\setlength{\itemindent}{0cm}
\setlength{\listparindent}{0cm}
\setlength{\leftmargin}{\evensidemargin}
\addtolength{\leftmargin}{\tmplength}
\settowidth{\labelsep}{X}
\addtolength{\leftmargin}{\labelsep}
\setlength{\labelwidth}{\tmplength}
}
\begin{flushleft}
\item[\textbf{Declaração}\hfill]
\begin{ttfamily}
public procedure AtFree(Index: Sw{\_}Integer);\end{ttfamily}


\end{flushleft}
\end{list}
\paragraph*{AtDelete}\hspace*{\fill}

\begin{list}{}{
\settowidth{\tmplength}{\textbf{Declaração}}
\setlength{\itemindent}{0cm}
\setlength{\listparindent}{0cm}
\setlength{\leftmargin}{\evensidemargin}
\addtolength{\leftmargin}{\tmplength}
\settowidth{\labelsep}{X}
\addtolength{\leftmargin}{\labelsep}
\setlength{\labelwidth}{\tmplength}
}
\begin{flushleft}
\item[\textbf{Declaração}\hfill]
\begin{ttfamily}
public procedure AtDelete(Index: Sw{\_}Integer);\end{ttfamily}


\end{flushleft}
\end{list}
\paragraph*{ForEach}\hspace*{\fill}

\begin{list}{}{
\settowidth{\tmplength}{\textbf{Declaração}}
\setlength{\itemindent}{0cm}
\setlength{\listparindent}{0cm}
\setlength{\leftmargin}{\evensidemargin}
\addtolength{\leftmargin}{\tmplength}
\settowidth{\labelsep}{X}
\addtolength{\leftmargin}{\labelsep}
\setlength{\labelwidth}{\tmplength}
}
\begin{flushleft}
\item[\textbf{Declaração}\hfill]
\begin{ttfamily}
public procedure ForEach(Action: Pointer);\end{ttfamily}


\end{flushleft}
\end{list}
\paragraph*{AtPut}\hspace*{\fill}

\begin{list}{}{
\settowidth{\tmplength}{\textbf{Declaração}}
\setlength{\itemindent}{0cm}
\setlength{\listparindent}{0cm}
\setlength{\leftmargin}{\evensidemargin}
\addtolength{\leftmargin}{\tmplength}
\settowidth{\labelsep}{X}
\addtolength{\leftmargin}{\labelsep}
\setlength{\labelwidth}{\tmplength}
}
\begin{flushleft}
\item[\textbf{Declaração}\hfill]
\begin{ttfamily}
public procedure AtPut(Index: Sw{\_}Integer; Item: Pointer);\end{ttfamily}


\end{flushleft}
\end{list}
\paragraph*{AtInsert}\hspace*{\fill}

\begin{list}{}{
\settowidth{\tmplength}{\textbf{Declaração}}
\setlength{\itemindent}{0cm}
\setlength{\listparindent}{0cm}
\setlength{\leftmargin}{\evensidemargin}
\addtolength{\leftmargin}{\tmplength}
\settowidth{\labelsep}{X}
\addtolength{\leftmargin}{\labelsep}
\setlength{\labelwidth}{\tmplength}
}
\begin{flushleft}
\item[\textbf{Declaração}\hfill]
\begin{ttfamily}
public procedure AtInsert(Index: Sw{\_}Integer; Item: Pointer);\end{ttfamily}


\end{flushleft}
\end{list}
\chapter{Unit mi.rtl.Objects.Methods.Collection.FilesStreams}
\section{Descrição}
\begin{itemize}
\item A unit \textbf{\begin{ttfamily}mi.rtl.Objects.Methods.Collection.FilesStreams\end{ttfamily}} implementa a classe \begin{ttfamily}TFilesStreams\end{ttfamily}(\ref{mi.rtl.Objects.Methods.Collection.FilesStreams.TFilesStreams}) do pacote \begin{ttfamily}mi.rtl\end{ttfamily}(\ref{mi.rtl}).

\begin{itemize}
\item \textbf{VERSÃO}: \begin{itemize}
\item Alpha {-} 0.5.0.687
\end{itemize}
\item \textbf{CÓDIGO FONTE}: \begin{itemize}
\item 
\end{itemize}
\item \textbf{HISTÓRICO} \begin{itemize}
\item Criado por: Paulo Sérgio da Silva Pacheco e{-}mail: paulosspacheco@yahoo.com.br \begin{itemize}
\item 2021{-}12{-}18 \begin{itemize}
\item 14:42 a .. {-} T12 Criar a unit \textbf{\begin{ttfamily}mi.rtl.Objects.Methods.Collection.FilesStreams\end{ttfamily}} e a classe \begin{ttfamily}TFilesStreams\end{ttfamily}(\ref{mi.rtl.Objects.Methods.Collection.FilesStreams.TFilesStreams})
\end{itemize}
\item 2021{-}12{-}20 \begin{itemize}
\item 17:11 a 18:30 {-} T12 Criar a classe \textbf{\begin{ttfamily}TFilesStreams\end{ttfamily}(\ref{mi.rtl.Objects.Methods.Collection.FilesStreams.TFilesStreams})}
\item 20:20 a 22:54 {-} T21 Criar exemplo de uso da classe \begin{ttfamily}TFilesStreams\end{ttfamily}(\ref{mi.rtl.Objects.Methods.Collection.FilesStreams.TFilesStreams})
\end{itemize}
\end{itemize}
\end{itemize}
\end{itemize}
\end{itemize}
\section{Uses}
\begin{itemize}
\item \begin{ttfamily}Classes\end{ttfamily}\item \begin{ttfamily}SysUtils\end{ttfamily}\item \begin{ttfamily}mi.rtl.objects.consts.MI{\_}MsgBox\end{ttfamily}\item \begin{ttfamily}mi.rtl.objects.consts.progressdlg{\_}if\end{ttfamily}(\ref{mi.rtl.Objects.Consts.ProgressDlg_If})\item \begin{ttfamily}mi.rtl.objects.methods.StreamBase.Stream\end{ttfamily}(\ref{mi.rtl.Objects.Methods.StreamBase.Stream})\item \begin{ttfamily}mi.rtl.objects.methods.StreamBase.Stream.FileStream\end{ttfamily}(\ref{mi.rtl.Objects.Methods.StreamBase.Stream.FileStream})\item \begin{ttfamily}mi.rtl.objects.methods.Collection\end{ttfamily}(\ref{mi.rtl.Objects.Methods.Collection})\end{itemize}
\section{Visão Geral}
\begin{description}
\item[\texttt{\begin{ttfamily}TFilesStreams\end{ttfamily} Classe}]
\end{description}
\section{Classes, Interfaces, Objetos e Registros}
\subsection*{TFilesStreams Classe}
\subsubsection*{\large{\textbf{Hierarquia}}\normalsize\hspace{1ex}\hfill}
TFilesStreams {$>$} \begin{ttfamily}TCollection\end{ttfamily}(\ref{mi.rtl.Objects.Methods.Collection.TCollection}) {$>$} \begin{ttfamily}TObjectsMethods\end{ttfamily}(\ref{mi.rtl.Objects.Methods.TObjectsMethods}) {$>$} \begin{ttfamily}TObjectsConsts\end{ttfamily}(\ref{mi.rtl.Objects.Consts.TObjectsConsts}) {$>$} 
TObjectsTypes
\subsubsection*{\large{\textbf{Descrição}}\normalsize\hspace{1ex}\hfill}
\begin{itemize}
\item A classe \textbf{\begin{ttfamily}TFilesStreams\end{ttfamily}} é usada para armazenar todos os arquivos abertos pelo sistema para poder fecha{-}los caso o programa aborte inesperadamente.

\begin{itemize}
\item EXEMPLO DE USO

\texttt{\\\nopagebreak[3]
\\\nopagebreak[3]
}\textbf{procedure}\texttt{~TMi{\_}Rtl{\_}Tests.TabSheet{\_}TFilesStreamsEnter(Sender:~TObject);\\\nopagebreak[3]
~~}\textbf{var}\texttt{\\\nopagebreak[3]
~~~i,L~:~integer;\\\nopagebreak[3]
~~~s:AnsiString;\\\nopagebreak[3]
\\\nopagebreak[3]
}\textbf{begin}\texttt{\\\nopagebreak[3]
~~filesStreams.DeleteAll;\\\nopagebreak[3]
~~StringGrid1.Clear;\\\nopagebreak[3]
\\\nopagebreak[3]
~~filesStreams.Mask~:=~edit2.Text;\\\nopagebreak[3]
~~StringGrid1.RowCount~:=~filesStreams.Count+1;\\\nopagebreak[3]
\\\nopagebreak[3]
~~LabelCount2.Caption~:=~Format('FilesStreams.Count~{\%}d',[filesStreams.Count]);\\\nopagebreak[3]
~~LabelCount2.Show;\\\nopagebreak[3]
~~L~:=~0;\\\nopagebreak[3]
~~StringGrid1.Cells[0,l]~:=~'Seq';\\\nopagebreak[3]
~~StringGrid1.Cells[1,l]~:=~'FileName';\\\nopagebreak[3]
~~StringGrid1.Cells[2,l]~:=~'FileSize';\\\nopagebreak[3]
~~inc(l);\\\nopagebreak[3]
~~}\textbf{if}\texttt{~filesStreams.Count~{$>$}~0\\\nopagebreak[3]
~~}\textbf{then}\texttt{~}\textbf{begin}\texttt{\\\nopagebreak[3]
~~~~~~~~~~}\textbf{for}\texttt{~i~:=~0~}\textbf{to}\texttt{~filesStreams.Count-1~}\textbf{do}\texttt{\\\nopagebreak[3]
~~~~~~~~~~}\textbf{with}\texttt{~filesStreams.FileByNum(i)~}\textbf{do}\texttt{\\\nopagebreak[3]
~~~~~~~~~~}\textbf{begin}\texttt{\\\nopagebreak[3]
~~~~~~~~~~~~StringGrid1.Cells[0,l]~:=~Format('{\%}d',[l]);\\\nopagebreak[3]
~~~~~~~~~~~~StringGrid1.Cells[1,l]~:=~FileName;\\\nopagebreak[3]
~~~~~~~~~~~~s~:=~Format('{\%}d',[FileSize(FileName)]);\\\nopagebreak[3]
~~~~~~~~~~~~StringGrid1.Cells[2,l]~:=~s~;\\\nopagebreak[3]
~~~~~~~~~~~~inc(L);\\\nopagebreak[3]
~~~~~~~~~~}\textbf{end}\texttt{;\\\nopagebreak[3]
~~~~~~~}\textbf{end}\texttt{;\\\nopagebreak[3]
\\\nopagebreak[3]
}\textbf{end}\texttt{;\\\nopagebreak[3]
\\\nopagebreak[3]
}\textbf{procedure}\texttt{~TMi{\_}Rtl{\_}Tests.Edit2Change(Sender:~TObject);\\\nopagebreak[3]
}\textbf{begin}\texttt{\\\nopagebreak[3]
~~TabSheet{\_}TFilesStreamsEnter(Self);\\\nopagebreak[3]
}\textbf{end}\texttt{;\\
}
\end{itemize}
\end{itemize}\subsubsection*{\large{\textbf{Propriedades}}\normalsize\hspace{1ex}\hfill}
\paragraph*{Mask}\hspace*{\fill}

\begin{list}{}{
\settowidth{\tmplength}{\textbf{Declaração}}
\setlength{\itemindent}{0cm}
\setlength{\listparindent}{0cm}
\setlength{\leftmargin}{\evensidemargin}
\addtolength{\leftmargin}{\tmplength}
\settowidth{\labelsep}{X}
\addtolength{\leftmargin}{\labelsep}
\setlength{\labelwidth}{\tmplength}
}
\begin{flushleft}
\item[\textbf{Declaração}\hfill]
\begin{ttfamily}
published property Mask  : AnsiString Read {\_}mask write SetMask;\end{ttfamily}


\end{flushleft}
\par
\item[\textbf{Descrição}]
\begin{itemize}
\item A propriedade \textbf{\begin{ttfamily}Mask\end{ttfamily}} é usada como filtro na função SysUtils.FindFirst
\end{itemize}

\end{list}
\subsubsection*{\large{\textbf{Métodos}}\normalsize\hspace{1ex}\hfill}
\paragraph*{SetMask}\hspace*{\fill}

\begin{list}{}{
\settowidth{\tmplength}{\textbf{Declaração}}
\setlength{\itemindent}{0cm}
\setlength{\listparindent}{0cm}
\setlength{\leftmargin}{\evensidemargin}
\addtolength{\leftmargin}{\tmplength}
\settowidth{\labelsep}{X}
\addtolength{\leftmargin}{\labelsep}
\setlength{\labelwidth}{\tmplength}
}
\begin{flushleft}
\item[\textbf{Declaração}\hfill]
\begin{ttfamily}
protected Procedure SetMask(a{\_}Mask : AnsiString);\end{ttfamily}


\end{flushleft}
\par
\item[\textbf{Descrição}]
\begin{itemize}
\item O método \textbf{\begin{ttfamily}SetMask\end{ttfamily}} é usado para filtrar os arquivo da pasta corrente do banco de dados.

\begin{itemize}
\item \textbf{Nota} \begin{itemize}
\item Para compreender essa função é bom ler o exemplo: \begin{ttfamily}TFiles.FindFiles\end{ttfamily}(\ref{mi.rtl.files.TFiles-FindFiles})
\end{itemize}
\end{itemize}
\end{itemize}

\end{list}
\paragraph*{Create}\hspace*{\fill}

\begin{list}{}{
\settowidth{\tmplength}{\textbf{Declaração}}
\setlength{\itemindent}{0cm}
\setlength{\listparindent}{0cm}
\setlength{\leftmargin}{\evensidemargin}
\addtolength{\leftmargin}{\tmplength}
\settowidth{\labelsep}{X}
\addtolength{\leftmargin}{\labelsep}
\setlength{\labelwidth}{\tmplength}
}
\begin{flushleft}
\item[\textbf{Declaração}\hfill]
\begin{ttfamily}
public CONSTRUCTOR Create;\end{ttfamily}


\end{flushleft}
\end{list}
\paragraph*{FileByNum}\hspace*{\fill}

\begin{list}{}{
\settowidth{\tmplength}{\textbf{Declaração}}
\setlength{\itemindent}{0cm}
\setlength{\listparindent}{0cm}
\setlength{\leftmargin}{\evensidemargin}
\addtolength{\leftmargin}{\tmplength}
\settowidth{\labelsep}{X}
\addtolength{\leftmargin}{\labelsep}
\setlength{\labelwidth}{\tmplength}
}
\begin{flushleft}
\item[\textbf{Declaração}\hfill]
\begin{ttfamily}
public Function FileByNum(const Index:Longint):TFileStream;\end{ttfamily}


\end{flushleft}
\end{list}
\paragraph*{FileByName}\hspace*{\fill}

\begin{list}{}{
\settowidth{\tmplength}{\textbf{Declaração}}
\setlength{\itemindent}{0cm}
\setlength{\listparindent}{0cm}
\setlength{\leftmargin}{\evensidemargin}
\addtolength{\leftmargin}{\tmplength}
\settowidth{\labelsep}{X}
\addtolength{\leftmargin}{\labelsep}
\setlength{\labelwidth}{\tmplength}
}
\begin{flushleft}
\item[\textbf{Declaração}\hfill]
\begin{ttfamily}
public Function FileByName(const aFileName:AnsiString):TFileStream;\end{ttfamily}


\end{flushleft}
\end{list}
\paragraph*{CopyFiles}\hspace*{\fill}

\begin{list}{}{
\settowidth{\tmplength}{\textbf{Declaração}}
\setlength{\itemindent}{0cm}
\setlength{\listparindent}{0cm}
\setlength{\leftmargin}{\evensidemargin}
\addtolength{\leftmargin}{\tmplength}
\settowidth{\labelsep}{X}
\addtolength{\leftmargin}{\labelsep}
\setlength{\labelwidth}{\tmplength}
}
\begin{flushleft}
\item[\textbf{Declaração}\hfill]
\begin{ttfamily}
public Function CopyFiles(aPathDest:PathStr):Integer;\end{ttfamily}


\end{flushleft}
\end{list}
\paragraph*{DeleteFiles}\hspace*{\fill}

\begin{list}{}{
\settowidth{\tmplength}{\textbf{Declaração}}
\setlength{\itemindent}{0cm}
\setlength{\listparindent}{0cm}
\setlength{\leftmargin}{\evensidemargin}
\addtolength{\leftmargin}{\tmplength}
\settowidth{\labelsep}{X}
\addtolength{\leftmargin}{\labelsep}
\setlength{\labelwidth}{\tmplength}
}
\begin{flushleft}
\item[\textbf{Declaração}\hfill]
\begin{ttfamily}
public Function DeleteFiles():Integer;\end{ttfamily}


\end{flushleft}
\end{list}
\paragraph*{Error}\hspace*{\fill}

\begin{list}{}{
\settowidth{\tmplength}{\textbf{Declaração}}
\setlength{\itemindent}{0cm}
\setlength{\listparindent}{0cm}
\setlength{\leftmargin}{\evensidemargin}
\addtolength{\leftmargin}{\tmplength}
\settowidth{\labelsep}{X}
\addtolength{\leftmargin}{\labelsep}
\setlength{\labelwidth}{\tmplength}
}
\begin{flushleft}
\item[\textbf{Declaração}\hfill]
\begin{ttfamily}
public PROCEDURE Error(Code, Info: Integer); Override;\end{ttfamily}


\end{flushleft}
\end{list}
\paragraph*{Create{\_}Progress1Passo}\hspace*{\fill}

\begin{list}{}{
\settowidth{\tmplength}{\textbf{Declaração}}
\setlength{\itemindent}{0cm}
\setlength{\listparindent}{0cm}
\setlength{\leftmargin}{\evensidemargin}
\addtolength{\leftmargin}{\tmplength}
\settowidth{\labelsep}{X}
\addtolength{\leftmargin}{\labelsep}
\setlength{\labelwidth}{\tmplength}
}
\begin{flushleft}
\item[\textbf{Declaração}\hfill]
\begin{ttfamily}
public Procedure Create{\_}Progress1Passo(ATitle : tString;Obs:tString ; ATotal : Longint); Override;\end{ttfamily}


\end{flushleft}
\end{list}
\paragraph*{Set{\_}Progress1Passo}\hspace*{\fill}

\begin{list}{}{
\settowidth{\tmplength}{\textbf{Declaração}}
\setlength{\itemindent}{0cm}
\setlength{\listparindent}{0cm}
\setlength{\leftmargin}{\evensidemargin}
\addtolength{\leftmargin}{\tmplength}
\settowidth{\labelsep}{X}
\addtolength{\leftmargin}{\labelsep}
\setlength{\labelwidth}{\tmplength}
}
\begin{flushleft}
\item[\textbf{Declaração}\hfill]
\begin{ttfamily}
public Procedure Set{\_}Progress1Passo(aNumero{\_}Segundos{\_}que{\_}deve{\_}esperar : Longint); Override;\end{ttfamily}


\end{flushleft}
\end{list}
\paragraph*{Destroy{\_}Progress1Passo}\hspace*{\fill}

\begin{list}{}{
\settowidth{\tmplength}{\textbf{Declaração}}
\setlength{\itemindent}{0cm}
\setlength{\listparindent}{0cm}
\setlength{\leftmargin}{\evensidemargin}
\addtolength{\leftmargin}{\tmplength}
\settowidth{\labelsep}{X}
\addtolength{\leftmargin}{\labelsep}
\setlength{\labelwidth}{\tmplength}
}
\begin{flushleft}
\item[\textbf{Declaração}\hfill]
\begin{ttfamily}
public Procedure Destroy{\_}Progress1Passo; Override;\end{ttfamily}


\end{flushleft}
\end{list}
\chapter{Unit mi.rtl.Objects.Methods.Collection.SortedCollection}
\section{Descrição}
\begin{itemize}
\item A Unit \textbf{\begin{ttfamily}mi.rtl.Objects.Methods.Collection.SortedCollection\end{ttfamily}} implementa a classe \textbf{\begin{ttfamily}TSortedCollection\end{ttfamily}(\ref{mi.rtl.Objects.Methods.Collection.SortedCollection.TSortedCollection})} do pacote \textbf{\begin{ttfamily}mi.rtl\end{ttfamily}(\ref{mi.rtl})}.

\begin{itemize}
\item \textbf{VERSÃO} \begin{itemize}
\item Alpha {-} 0.5.0.687
\end{itemize}
\item \textbf{CÓDIGO FONTE}: \begin{itemize}
\item 
\end{itemize}
\item \textbf{HISTÓRICO} \begin{itemize}
\item Criado por: Paulo Sérgio da Silva Pacheco e{-}mail: paulosspacheco@yahoo.com.br \begin{itemize}
\item \textbf{30/11/2021} \begin{itemize}
\item 9:45 a 11:47 : Criada a unit \begin{ttfamily}mi.rtl.Objects.Methods.Collection.SortedCollection\end{ttfamily} e a classe \textbf{\begin{ttfamily}TSortedCollection\end{ttfamily}(\ref{mi.rtl.Objects.Methods.Collection.SortedCollection.TSortedCollection})}
\end{itemize}
\end{itemize}
\end{itemize}
\end{itemize}
\end{itemize}
\section{Uses}
\begin{itemize}
\item \begin{ttfamily}Classes\end{ttfamily}\item \begin{ttfamily}SysUtils\end{ttfamily}\item \begin{ttfamily}mi.rtl.objects.Methods\end{ttfamily}(\ref{mi.rtl.Objects.Methods})\item \begin{ttfamily}mi.rtl.objects.methods.StreamBase\end{ttfamily}(\ref{mi.rtl.Objects.Methods.StreamBase})\item \begin{ttfamily}mi.rtl.objects.methods.Collection\end{ttfamily}(\ref{mi.rtl.Objects.Methods.Collection})\end{itemize}
\section{Visão Geral}
\begin{description}
\item[\texttt{\begin{ttfamily}TSortedCollection\end{ttfamily} Classe}]
\end{description}
\section{Classes, Interfaces, Objetos e Registros}
\subsection*{TSortedCollection Classe}
\subsubsection*{\large{\textbf{Hierarquia}}\normalsize\hspace{1ex}\hfill}
TSortedCollection {$>$} \begin{ttfamily}TCollection\end{ttfamily}(\ref{mi.rtl.Objects.Methods.Collection.TCollection}) {$>$} \begin{ttfamily}TObjectsMethods\end{ttfamily}(\ref{mi.rtl.Objects.Methods.TObjectsMethods}) {$>$} \begin{ttfamily}TObjectsConsts\end{ttfamily}(\ref{mi.rtl.Objects.Consts.TObjectsConsts}) {$>$} 
TObjectsTypes
\subsubsection*{\large{\textbf{Descrição}}\normalsize\hspace{1ex}\hfill}
\begin{itemize}
\item A class \textbf{\begin{ttfamily}TSortedCollection\end{ttfamily}} implementa coleções ordenadas de objetos.

\begin{itemize}
\item EXEMPLO DE USO

\texttt{\\\nopagebreak[3]
\\\nopagebreak[3]
???\\
}
\end{itemize}
\end{itemize}\subsubsection*{\large{\textbf{Propriedades}}\normalsize\hspace{1ex}\hfill}
\paragraph*{tstrings}\hspace*{\fill}

\begin{list}{}{
\settowidth{\tmplength}{\textbf{Declaração}}
\setlength{\itemindent}{0cm}
\setlength{\listparindent}{0cm}
\setlength{\leftmargin}{\evensidemargin}
\addtolength{\leftmargin}{\tmplength}
\settowidth{\labelsep}{X}
\addtolength{\leftmargin}{\labelsep}
\setlength{\labelwidth}{\tmplength}
}
\begin{flushleft}
\item[\textbf{Declaração}\hfill]
\begin{ttfamily}
protected property tstrings[Index: Sw{\_}Integer]: tstring Read Getstrings write Setstrings;\end{ttfamily}


\end{flushleft}
\par
\item[\textbf{Descrição}]
=n specifies the maximum number of AnsiCharacters

\end{list}
\subsubsection*{\large{\textbf{Campos}}\normalsize\hspace{1ex}\hfill}
\paragraph*{Duplicates}\hspace*{\fill}

\begin{list}{}{
\settowidth{\tmplength}{\textbf{Declaração}}
\setlength{\itemindent}{0cm}
\setlength{\listparindent}{0cm}
\setlength{\leftmargin}{\evensidemargin}
\addtolength{\leftmargin}{\tmplength}
\settowidth{\labelsep}{X}
\addtolength{\leftmargin}{\labelsep}
\setlength{\labelwidth}{\tmplength}
}
\begin{flushleft}
\item[\textbf{Declaração}\hfill]
\begin{ttfamily}
public Duplicates: Boolean;\end{ttfamily}


\end{flushleft}
\end{list}
\subsubsection*{\large{\textbf{Métodos}}\normalsize\hspace{1ex}\hfill}
\paragraph*{Create}\hspace*{\fill}

\begin{list}{}{
\settowidth{\tmplength}{\textbf{Declaração}}
\setlength{\itemindent}{0cm}
\setlength{\listparindent}{0cm}
\setlength{\leftmargin}{\evensidemargin}
\addtolength{\leftmargin}{\tmplength}
\settowidth{\labelsep}{X}
\addtolength{\leftmargin}{\labelsep}
\setlength{\labelwidth}{\tmplength}
}
\begin{flushleft}
\item[\textbf{Declaração}\hfill]
\begin{ttfamily}
public CONSTRUCTOR Create(ALimit, ADelta: Sw{\_}Integer); overload; override;\end{ttfamily}


\end{flushleft}
\par
\item[\textbf{Descrição}]
\begin{ttfamily}Duplicates\end{ttfamily}(\ref{mi.rtl.Objects.Methods.Collection.SortedCollection.TSortedCollection-Duplicates}) flag

\end{list}
\paragraph*{KeyOf}\hspace*{\fill}

\begin{list}{}{
\settowidth{\tmplength}{\textbf{Declaração}}
\setlength{\itemindent}{0cm}
\setlength{\listparindent}{0cm}
\setlength{\leftmargin}{\evensidemargin}
\addtolength{\leftmargin}{\tmplength}
\settowidth{\labelsep}{X}
\addtolength{\leftmargin}{\labelsep}
\setlength{\labelwidth}{\tmplength}
}
\begin{flushleft}
\item[\textbf{Declaração}\hfill]
\begin{ttfamily}
protected FUNCTION KeyOf(Item: Pointer): Pointer; Virtual;\end{ttfamily}


\end{flushleft}
\end{list}
\paragraph*{IndexOf}\hspace*{\fill}

\begin{list}{}{
\settowidth{\tmplength}{\textbf{Declaração}}
\setlength{\itemindent}{0cm}
\setlength{\listparindent}{0cm}
\setlength{\leftmargin}{\evensidemargin}
\addtolength{\leftmargin}{\tmplength}
\settowidth{\labelsep}{X}
\addtolength{\leftmargin}{\labelsep}
\setlength{\labelwidth}{\tmplength}
}
\begin{flushleft}
\item[\textbf{Declaração}\hfill]
\begin{ttfamily}
protected FUNCTION IndexOf(Item: Pointer): Sw{\_}Integer; Override;\end{ttfamily}


\end{flushleft}
\end{list}
\paragraph*{Compare}\hspace*{\fill}

\begin{list}{}{
\settowidth{\tmplength}{\textbf{Declaração}}
\setlength{\itemindent}{0cm}
\setlength{\listparindent}{0cm}
\setlength{\leftmargin}{\evensidemargin}
\addtolength{\leftmargin}{\tmplength}
\settowidth{\labelsep}{X}
\addtolength{\leftmargin}{\labelsep}
\setlength{\labelwidth}{\tmplength}
}
\begin{flushleft}
\item[\textbf{Declaração}\hfill]
\begin{ttfamily}
protected FUNCTION Compare(Key1, Key2: Pointer): Sw{\_}Integer; Virtual;\end{ttfamily}


\end{flushleft}
\end{list}
\paragraph*{Search}\hspace*{\fill}

\begin{list}{}{
\settowidth{\tmplength}{\textbf{Declaração}}
\setlength{\itemindent}{0cm}
\setlength{\listparindent}{0cm}
\setlength{\leftmargin}{\evensidemargin}
\addtolength{\leftmargin}{\tmplength}
\settowidth{\labelsep}{X}
\addtolength{\leftmargin}{\labelsep}
\setlength{\labelwidth}{\tmplength}
}
\begin{flushleft}
\item[\textbf{Declaração}\hfill]
\begin{ttfamily}
protected FUNCTION Search(Key: Pointer; Var Index: Sw{\_}Integer): Boolean; Virtual;\end{ttfamily}


\end{flushleft}
\end{list}
\paragraph*{Insert}\hspace*{\fill}

\begin{list}{}{
\settowidth{\tmplength}{\textbf{Declaração}}
\setlength{\itemindent}{0cm}
\setlength{\listparindent}{0cm}
\setlength{\leftmargin}{\evensidemargin}
\addtolength{\leftmargin}{\tmplength}
\settowidth{\labelsep}{X}
\addtolength{\leftmargin}{\labelsep}
\setlength{\labelwidth}{\tmplength}
}
\begin{flushleft}
\item[\textbf{Declaração}\hfill]
\begin{ttfamily}
protected PROCEDURE Insert(Item: Pointer); Override;\end{ttfamily}


\end{flushleft}
\end{list}
\paragraph*{GetMaxLength}\hspace*{\fill}

\begin{list}{}{
\settowidth{\tmplength}{\textbf{Declaração}}
\setlength{\itemindent}{0cm}
\setlength{\listparindent}{0cm}
\setlength{\leftmargin}{\evensidemargin}
\addtolength{\leftmargin}{\tmplength}
\settowidth{\labelsep}{X}
\addtolength{\leftmargin}{\labelsep}
\setlength{\labelwidth}{\tmplength}
}
\begin{flushleft}
\item[\textbf{Declaração}\hfill]
\begin{ttfamily}
protected Function GetMaxLength():Integer;\end{ttfamily}


\end{flushleft}
\end{list}
\chapter{Unit mi.rtl.Objects.Methods.Collection.SortedCollection.StrCollection}
\section{Descrição}
\begin{itemize}
\item A Unit \textbf{\begin{ttfamily}mi.rtl.Objects.Methods.Collection.SortedCollection.StrCollection\end{ttfamily}} implementa a classe \textbf{\begin{ttfamily}TStrCollection\end{ttfamily}(\ref{mi.rtl.Objects.Methods.Collection.SortedCollection.StrCollection.TStrCollection})} do pacote \textbf{\begin{ttfamily}mi.rtl\end{ttfamily}(\ref{mi.rtl})}.

\begin{itemize}
\item \textbf{NOTA} \begin{itemize}
\item A diferença de \begin{ttfamily}tstringCollection\end{ttfamily}(\ref{mi.rtl.Objects.Methods.Collection.SortedCollection.StringCollection.TStringCollection}) para \begin{ttfamily}TStrCollection\end{ttfamily}(\ref{mi.rtl.Objects.Methods.Collection.SortedCollection.StrCollection.TStrCollection}) é que a primeira é uma coleção de shortstring e a segunda é uma coleção de PByteArray usada para trabalhar com AnsiString;
\end{itemize}
\item \textbf{VERSÃO} \begin{itemize}
\item Alpha {-} 0.5.0.687
\end{itemize}
\item \textbf{CÓDIGO FONTE}: \begin{itemize}
\item 
\end{itemize}
\item \textbf{HISTÓRICO} \begin{itemize}
\item Criado por: Paulo Sérgio da Silva Pacheco e{-}mail: paulosspacheco@yahoo.com.br \begin{itemize}
\item \textbf{30/11/2021} \begin{itemize}
\item 11:47 a 11:47 : Criada a unit \begin{ttfamily}mi.rtl.Objects.Methods.Collection.SortedCollection.StrCollection\end{ttfamily} e a classe \textbf{TStCollection}
\item 14:00 a 14:20 : Documentar a unit
\end{itemize}
\end{itemize}
\end{itemize}
\end{itemize}
\end{itemize}
\section{Uses}
\begin{itemize}
\item \begin{ttfamily}Classes\end{ttfamily}\item \begin{ttfamily}SysUtils\end{ttfamily}\item \begin{ttfamily}mi.rtl.objects.Methods.Collection.SortedCollection\end{ttfamily}(\ref{mi.rtl.Objects.Methods.Collection.SortedCollection})\item \begin{ttfamily}mi.rtl.objects.methods.Streambase.Stream\end{ttfamily}(\ref{mi.rtl.Objects.Methods.StreamBase.Stream})\end{itemize}
\section{Visão Geral}
\begin{description}
\item[\texttt{\begin{ttfamily}TStrCollection\end{ttfamily} Classe}]
\item[\texttt{\begin{ttfamily}TUnSortedStrCollection\end{ttfamily} Classe}]
\end{description}
\section{Classes, Interfaces, Objetos e Registros}
\subsection*{TStrCollection Classe}
\subsubsection*{\large{\textbf{Hierarquia}}\normalsize\hspace{1ex}\hfill}
TStrCollection {$>$} \begin{ttfamily}TSortedCollection\end{ttfamily}(\ref{mi.rtl.Objects.Methods.Collection.SortedCollection.TSortedCollection}) {$>$} \begin{ttfamily}TCollection\end{ttfamily}(\ref{mi.rtl.Objects.Methods.Collection.TCollection}) {$>$} \begin{ttfamily}TObjectsMethods\end{ttfamily}(\ref{mi.rtl.Objects.Methods.TObjectsMethods}) {$>$} \begin{ttfamily}TObjectsConsts\end{ttfamily}(\ref{mi.rtl.Objects.Consts.TObjectsConsts}) {$>$} 
TObjectsTypes
\subsubsection*{\large{\textbf{Descrição}}\normalsize\hspace{1ex}\hfill}
A classe \textbf{\begin{ttfamily}TStrCollection\end{ttfamily}} implementa uma coleção de \textbf{AnsiString}\subsubsection*{\large{\textbf{Métodos}}\normalsize\hspace{1ex}\hfill}
\paragraph*{Compare}\hspace*{\fill}

\begin{list}{}{
\settowidth{\tmplength}{\textbf{Declaração}}
\setlength{\itemindent}{0cm}
\setlength{\listparindent}{0cm}
\setlength{\leftmargin}{\evensidemargin}
\addtolength{\leftmargin}{\tmplength}
\settowidth{\labelsep}{X}
\addtolength{\leftmargin}{\labelsep}
\setlength{\labelwidth}{\tmplength}
}
\begin{flushleft}
\item[\textbf{Declaração}\hfill]
\begin{ttfamily}
protected FUNCTION Compare(Key1, Key2: Pointer): Sw{\_}Integer; Override;\end{ttfamily}


\end{flushleft}
\end{list}
\paragraph*{GetItem}\hspace*{\fill}

\begin{list}{}{
\settowidth{\tmplength}{\textbf{Declaração}}
\setlength{\itemindent}{0cm}
\setlength{\listparindent}{0cm}
\setlength{\leftmargin}{\evensidemargin}
\addtolength{\leftmargin}{\tmplength}
\settowidth{\labelsep}{X}
\addtolength{\leftmargin}{\labelsep}
\setlength{\labelwidth}{\tmplength}
}
\begin{flushleft}
\item[\textbf{Declaração}\hfill]
\begin{ttfamily}
public FUNCTION GetItem(Var S: TStream): Pointer; Override;\end{ttfamily}


\end{flushleft}
\end{list}
\paragraph*{FreeItem}\hspace*{\fill}

\begin{list}{}{
\settowidth{\tmplength}{\textbf{Declaração}}
\setlength{\itemindent}{0cm}
\setlength{\listparindent}{0cm}
\setlength{\leftmargin}{\evensidemargin}
\addtolength{\leftmargin}{\tmplength}
\settowidth{\labelsep}{X}
\addtolength{\leftmargin}{\labelsep}
\setlength{\labelwidth}{\tmplength}
}
\begin{flushleft}
\item[\textbf{Declaração}\hfill]
\begin{ttfamily}
protected PROCEDURE FreeItem(Item: Pointer); Override;\end{ttfamily}


\end{flushleft}
\end{list}
\paragraph*{PutItem}\hspace*{\fill}

\begin{list}{}{
\settowidth{\tmplength}{\textbf{Declaração}}
\setlength{\itemindent}{0cm}
\setlength{\listparindent}{0cm}
\setlength{\leftmargin}{\evensidemargin}
\addtolength{\leftmargin}{\tmplength}
\settowidth{\labelsep}{X}
\addtolength{\leftmargin}{\labelsep}
\setlength{\labelwidth}{\tmplength}
}
\begin{flushleft}
\item[\textbf{Declaração}\hfill]
\begin{ttfamily}
protected PROCEDURE PutItem(Var S: TStream; Item: Pointer); Override;\end{ttfamily}


\end{flushleft}
\end{list}
\subsection*{TUnSortedStrCollection Classe}
\subsubsection*{\large{\textbf{Hierarquia}}\normalsize\hspace{1ex}\hfill}
TUnSortedStrCollection {$>$} \begin{ttfamily}TStrCollection\end{ttfamily}(\ref{mi.rtl.Objects.Methods.Collection.SortedCollection.StrCollection.TStrCollection}) {$>$} \begin{ttfamily}TSortedCollection\end{ttfamily}(\ref{mi.rtl.Objects.Methods.Collection.SortedCollection.TSortedCollection}) {$>$} \begin{ttfamily}TCollection\end{ttfamily}(\ref{mi.rtl.Objects.Methods.Collection.TCollection}) {$>$} \begin{ttfamily}TObjectsMethods\end{ttfamily}(\ref{mi.rtl.Objects.Methods.TObjectsMethods}) {$>$} \begin{ttfamily}TObjectsConsts\end{ttfamily}(\ref{mi.rtl.Objects.Consts.TObjectsConsts}) {$>$} 
TObjectsTypes
\subsubsection*{\large{\textbf{Descrição}}\normalsize\hspace{1ex}\hfill}
A classe \textbf{\begin{ttfamily}TUnSortedStrCollection\end{ttfamily}} implementa uma coleção de \textbf{AnsiString} na ordem original de inserção das AnsiStrings\subsubsection*{\large{\textbf{Métodos}}\normalsize\hspace{1ex}\hfill}
\paragraph*{Insert}\hspace*{\fill}

\begin{list}{}{
\settowidth{\tmplength}{\textbf{Declaração}}
\setlength{\itemindent}{0cm}
\setlength{\listparindent}{0cm}
\setlength{\leftmargin}{\evensidemargin}
\addtolength{\leftmargin}{\tmplength}
\settowidth{\labelsep}{X}
\addtolength{\leftmargin}{\labelsep}
\setlength{\labelwidth}{\tmplength}
}
\begin{flushleft}
\item[\textbf{Declaração}\hfill]
\begin{ttfamily}
public PROCEDURE Insert(Item: Pointer); Override;\end{ttfamily}


\end{flushleft}
\end{list}
\chapter{Unit mi.rtl.Objects.Methods.Collection.SortedCollection.StringCollection}
\section{Descrição}
\begin{itemize}
\item A Unit \textbf{\begin{ttfamily}mi.rtl.Objects.Methods.Collection.SortedCollection.StringCollection\end{ttfamily}} implementa a classe \textbf{\begin{ttfamily}tstringCollection\end{ttfamily}(\ref{mi.rtl.Objects.Methods.Collection.SortedCollection.StringCollection.TStringCollection})} do pacote \textbf{\begin{ttfamily}mi.rtl\end{ttfamily}(\ref{mi.rtl})}.

\begin{itemize}
\item \textbf{NOTA} \begin{itemize}
\item A diferença de \begin{ttfamily}tstringCollection\end{ttfamily}(\ref{mi.rtl.Objects.Methods.Collection.SortedCollection.StringCollection.TStringCollection}) para \begin{ttfamily}TStrCollection\end{ttfamily}(\ref{mi.rtl.Objects.Methods.Collection.SortedCollection.StrCollection.TStrCollection}) é que a primeira é uma coleção de shortstring e a segunda é uma coleção de PByteArray usada para trabalhar com AnsiString;
\end{itemize}
\item \textbf{VERSÃO} \begin{itemize}
\item Alpha {-} 0.5.0.687
\end{itemize}
\item \textbf{CÓDIGO FONTE}: \begin{itemize}
\item 
\end{itemize}
\item \textbf{HISTÓRICO} \begin{itemize}
\item Criado por: Paulo Sérgio da Silva Pacheco e{-}mail: paulosspacheco@yahoo.com.br \begin{itemize}
\item \textbf{30/11/2021} \begin{itemize}
\item 14:20 a 15:15 : Criada a unit \begin{ttfamily}mi.rtl.Objects.Methods.Collection.SortedCollection.StringCollection\end{ttfamily} e a classe \textbf{\begin{ttfamily}tstringCollection\end{ttfamily}(\ref{mi.rtl.Objects.Methods.Collection.SortedCollection.StringCollection.TStringCollection})}
\end{itemize}
\end{itemize}
\end{itemize}
\end{itemize}
\end{itemize}
\section{Uses}
\begin{itemize}
\item \begin{ttfamily}Classes\end{ttfamily}\item \begin{ttfamily}SysUtils\end{ttfamily}\item \begin{ttfamily}mi.rtl.objects.methods.Collection.SortedCollection\end{ttfamily}(\ref{mi.rtl.Objects.Methods.Collection.SortedCollection})\item \begin{ttfamily}mi.rtl.objects.methods.StreamBase.Stream\end{ttfamily}(\ref{mi.rtl.Objects.Methods.StreamBase.Stream})\end{itemize}
\section{Visão Geral}
\begin{description}
\item[\texttt{\begin{ttfamily}TStringCollection\end{ttfamily} Classe}]
\item[\texttt{\begin{ttfamily}TUnSortedStringCollection\end{ttfamily} Classe}]
\end{description}
\section{Classes, Interfaces, Objetos e Registros}
\subsection*{TStringCollection Classe}
\subsubsection*{\large{\textbf{Hierarquia}}\normalsize\hspace{1ex}\hfill}
TStringCollection {$>$} \begin{ttfamily}TSortedCollection\end{ttfamily}(\ref{mi.rtl.Objects.Methods.Collection.SortedCollection.TSortedCollection}) {$>$} \begin{ttfamily}TCollection\end{ttfamily}(\ref{mi.rtl.Objects.Methods.Collection.TCollection}) {$>$} \begin{ttfamily}TObjectsMethods\end{ttfamily}(\ref{mi.rtl.Objects.Methods.TObjectsMethods}) {$>$} \begin{ttfamily}TObjectsConsts\end{ttfamily}(\ref{mi.rtl.Objects.Consts.TObjectsConsts}) {$>$} 
TObjectsTypes
\subsubsection*{\large{\textbf{Descrição}}\normalsize\hspace{1ex}\hfill}
A classe \textbf{\begin{ttfamily}TStringCollection\end{ttfamily}} implementa uma coleção de \textbf{AnsiString}\subsubsection*{\large{\textbf{Métodos}}\normalsize\hspace{1ex}\hfill}
\paragraph*{GetItem}\hspace*{\fill}

\begin{list}{}{
\settowidth{\tmplength}{\textbf{Declaração}}
\setlength{\itemindent}{0cm}
\setlength{\listparindent}{0cm}
\setlength{\leftmargin}{\evensidemargin}
\addtolength{\leftmargin}{\tmplength}
\settowidth{\labelsep}{X}
\addtolength{\leftmargin}{\labelsep}
\setlength{\labelwidth}{\tmplength}
}
\begin{flushleft}
\item[\textbf{Declaração}\hfill]
\begin{ttfamily}
public FUNCTION GetItem(Var S: TStream): Pointer; Override;\end{ttfamily}


\end{flushleft}
\end{list}
\paragraph*{Compare}\hspace*{\fill}

\begin{list}{}{
\settowidth{\tmplength}{\textbf{Declaração}}
\setlength{\itemindent}{0cm}
\setlength{\listparindent}{0cm}
\setlength{\leftmargin}{\evensidemargin}
\addtolength{\leftmargin}{\tmplength}
\settowidth{\labelsep}{X}
\addtolength{\leftmargin}{\labelsep}
\setlength{\labelwidth}{\tmplength}
}
\begin{flushleft}
\item[\textbf{Declaração}\hfill]
\begin{ttfamily}
public FUNCTION Compare(Key1, Key2: Pointer): Sw{\_}Integer; Override;\end{ttfamily}


\end{flushleft}
\end{list}
\paragraph*{FreeItem}\hspace*{\fill}

\begin{list}{}{
\settowidth{\tmplength}{\textbf{Declaração}}
\setlength{\itemindent}{0cm}
\setlength{\listparindent}{0cm}
\setlength{\leftmargin}{\evensidemargin}
\addtolength{\leftmargin}{\tmplength}
\settowidth{\labelsep}{X}
\addtolength{\leftmargin}{\labelsep}
\setlength{\labelwidth}{\tmplength}
}
\begin{flushleft}
\item[\textbf{Declaração}\hfill]
\begin{ttfamily}
public PROCEDURE FreeItem(Item: Pointer); Override;\end{ttfamily}


\end{flushleft}
\end{list}
\paragraph*{PutItem}\hspace*{\fill}

\begin{list}{}{
\settowidth{\tmplength}{\textbf{Declaração}}
\setlength{\itemindent}{0cm}
\setlength{\listparindent}{0cm}
\setlength{\leftmargin}{\evensidemargin}
\addtolength{\leftmargin}{\tmplength}
\settowidth{\labelsep}{X}
\addtolength{\leftmargin}{\labelsep}
\setlength{\labelwidth}{\tmplength}
}
\begin{flushleft}
\item[\textbf{Declaração}\hfill]
\begin{ttfamily}
public PROCEDURE PutItem(Var S: TStream; Item: Pointer); Override;\end{ttfamily}


\end{flushleft}
\end{list}
\subsection*{TUnSortedStringCollection Classe}
\subsubsection*{\large{\textbf{Hierarquia}}\normalsize\hspace{1ex}\hfill}
TUnSortedStringCollection {$>$} \begin{ttfamily}TStringCollection\end{ttfamily}(\ref{mi.rtl.Objects.Methods.Collection.SortedCollection.StringCollection.TStringCollection}) {$>$} \begin{ttfamily}TSortedCollection\end{ttfamily}(\ref{mi.rtl.Objects.Methods.Collection.SortedCollection.TSortedCollection}) {$>$} \begin{ttfamily}TCollection\end{ttfamily}(\ref{mi.rtl.Objects.Methods.Collection.TCollection}) {$>$} \begin{ttfamily}TObjectsMethods\end{ttfamily}(\ref{mi.rtl.Objects.Methods.TObjectsMethods}) {$>$} \begin{ttfamily}TObjectsConsts\end{ttfamily}(\ref{mi.rtl.Objects.Consts.TObjectsConsts}) {$>$} 
TObjectsTypes
\subsubsection*{\large{\textbf{Descrição}}\normalsize\hspace{1ex}\hfill}
A classe \textbf{\begin{ttfamily}TUnSortedStringCollection\end{ttfamily}} implementa uma coleção de \textbf{Shortstring} na ordem original de inserção dos ShortStrings\subsubsection*{\large{\textbf{Métodos}}\normalsize\hspace{1ex}\hfill}
\paragraph*{Insert}\hspace*{\fill}

\begin{list}{}{
\settowidth{\tmplength}{\textbf{Declaração}}
\setlength{\itemindent}{0cm}
\setlength{\listparindent}{0cm}
\setlength{\leftmargin}{\evensidemargin}
\addtolength{\leftmargin}{\tmplength}
\settowidth{\labelsep}{X}
\addtolength{\leftmargin}{\labelsep}
\setlength{\labelwidth}{\tmplength}
}
\begin{flushleft}
\item[\textbf{Declaração}\hfill]
\begin{ttfamily}
public PROCEDURE Insert(Item: Pointer); Override;\end{ttfamily}


\end{flushleft}
\end{list}
\chapter{Unit mi.rtl.Objects.Methods.Collection.Sortedcollection.Stringcollection.Collectionstring}
\section{Descrição}
\begin{itemize}
\item A Unit \textbf{\begin{ttfamily}mi.rtl.Objects.Methods.Collection.Sortedcollection.Stringcollection.Collectionstring\end{ttfamily}} implementa a classe \textbf{\begin{ttfamily}TCollectionString\end{ttfamily}(\ref{mi.rtl.Objects.Methods.Collection.Sortedcollection.Stringcollection.Collectionstring.TCollectionString})} do pacote \textbf{\begin{ttfamily}mi.rtl\end{ttfamily}(\ref{mi.rtl})}.

\begin{itemize}
\item \textbf{NOTA} \begin{itemize}
\item Essa classe foi criada para transformar Lista \begin{ttfamily}PSItem\end{ttfamily}(\ref{mi_rtl_ui_Dmxscroller-PSItem}) eem \begin{ttfamily}TCollection\end{ttfamily}(\ref{mi.rtl.Objects.Methods.Collection.TCollection}) de strings;
\end{itemize}
\item \textbf{VERSÃO} \begin{itemize}
\item Alpha {-} 0.5.0.687
\end{itemize}
\item \textbf{CÓDIGO FONTE}: \begin{itemize}
\item 
\end{itemize}
\item \textbf{HISTÓRICO} \begin{itemize}
\item Criado por: Paulo Sérgio da Silva Pacheco e{-}mail: paulosspacheco@yahoo.com.br \begin{itemize}
\item \textbf{07/12/2021} \begin{itemize}
\item 08:00 a 10:07 : Criada a unit \begin{ttfamily}mi.rtl.Objects.Methods.Collection.Sortedcollection.Stringcollection.Collectionstring\end{ttfamily} e a classe \textbf{\begin{ttfamily}TCollectionString\end{ttfamily}(\ref{mi.rtl.Objects.Methods.Collection.Sortedcollection.Stringcollection.Collectionstring.TCollectionString})}
\end{itemize}
\end{itemize}
\end{itemize}
\end{itemize}
\end{itemize}
\section{Uses}
\begin{itemize}
\item \begin{ttfamily}Classes\end{ttfamily}\item \begin{ttfamily}SysUtils\end{ttfamily}\item \begin{ttfamily}mi.rtl.objects.Methods\end{ttfamily}(\ref{mi.rtl.Objects.Methods})\item \begin{ttfamily}mi.rtl.objects.methods.Collection.Sortedcollection.StringCollection\end{ttfamily}(\ref{mi.rtl.Objects.Methods.Collection.SortedCollection.StringCollection})\end{itemize}
\section{Visão Geral}
\begin{description}
\item[\texttt{\begin{ttfamily}TCollectionString\end{ttfamily} Classe}]
\end{description}
\section{Classes, Interfaces, Objetos e Registros}
\subsection*{TCollectionString Classe}
\subsubsection*{\large{\textbf{Hierarquia}}\normalsize\hspace{1ex}\hfill}
TCollectionString {$>$} \subsubsection*{\large{\textbf{Descrição}}\normalsize\hspace{1ex}\hfill}
omit if \begin{ttfamily}TListBoxRec\end{ttfamily}(\ref{mi.rtl.Objects.Methods.Collection.Sortedcollection.Stringcollection.Collectionstring.TCollectionString-TListBoxRec}) is defined else where\subsubsection*{\large{\textbf{Propriedades}}\normalsize\hspace{1ex}\hfill}
\paragraph*{AnsiStrings}\hspace*{\fill}

\begin{list}{}{
\settowidth{\tmplength}{\textbf{Declaração}}
\setlength{\itemindent}{0cm}
\setlength{\listparindent}{0cm}
\setlength{\leftmargin}{\evensidemargin}
\addtolength{\leftmargin}{\tmplength}
\settowidth{\labelsep}{X}
\addtolength{\leftmargin}{\labelsep}
\setlength{\labelwidth}{\tmplength}
}
\begin{flushleft}
\item[\textbf{Declaração}\hfill]
\begin{ttfamily}
public property AnsiStrings[Index: Sw{\_}Integer]: AnsiString Read GetAnsiStrings;\end{ttfamily}


\end{flushleft}
\par
\item[\textbf{Descrição}]
Ler a string sem os caracteres de controle

\end{list}
\subsubsection*{\large{\textbf{Campos}}\normalsize\hspace{1ex}\hfill}
\paragraph*{Ordem}\hspace*{\fill}

\begin{list}{}{
\settowidth{\tmplength}{\textbf{Declaração}}
\setlength{\itemindent}{0cm}
\setlength{\listparindent}{0cm}
\setlength{\leftmargin}{\evensidemargin}
\addtolength{\leftmargin}{\tmplength}
\settowidth{\labelsep}{X}
\addtolength{\leftmargin}{\labelsep}
\setlength{\labelwidth}{\tmplength}
}
\begin{flushleft}
\item[\textbf{Declaração}\hfill]
\begin{ttfamily}
public Ordem:Boolean;\end{ttfamily}


\end{flushleft}
\par
\item[\textbf{Descrição}]
\begin{itemize}
\item Se True insere em \begin{ttfamily}ordem\end{ttfamily} alfabética
\end{itemize}

\end{list}
\paragraph*{FoundTesteCompleto}\hspace*{\fill}

\begin{list}{}{
\settowidth{\tmplength}{\textbf{Declaração}}
\setlength{\itemindent}{0cm}
\setlength{\listparindent}{0cm}
\setlength{\leftmargin}{\evensidemargin}
\addtolength{\leftmargin}{\tmplength}
\settowidth{\labelsep}{X}
\addtolength{\leftmargin}{\labelsep}
\setlength{\labelwidth}{\tmplength}
}
\begin{flushleft}
\item[\textbf{Declaração}\hfill]
\begin{ttfamily}
public FoundTesteCompleto:Boolean;\end{ttfamily}


\end{flushleft}
\end{list}
\subsubsection*{\large{\textbf{Métodos}}\normalsize\hspace{1ex}\hfill}
\paragraph*{Create}\hspace*{\fill}

\begin{list}{}{
\settowidth{\tmplength}{\textbf{Declaração}}
\setlength{\itemindent}{0cm}
\setlength{\listparindent}{0cm}
\setlength{\leftmargin}{\evensidemargin}
\addtolength{\leftmargin}{\tmplength}
\settowidth{\labelsep}{X}
\addtolength{\leftmargin}{\labelsep}
\setlength{\labelwidth}{\tmplength}
}
\begin{flushleft}
\item[\textbf{Declaração}\hfill]
\begin{ttfamily}
public constructor Create(ALimit, ADelta: Sw{\_}integer;AOrdem:Boolean); overload; virtual;\end{ttfamily}


\end{flushleft}
\end{list}
\paragraph*{CreateLista}\hspace*{\fill}

\begin{list}{}{
\settowidth{\tmplength}{\textbf{Declaração}}
\setlength{\itemindent}{0cm}
\setlength{\listparindent}{0cm}
\setlength{\leftmargin}{\evensidemargin}
\addtolength{\leftmargin}{\tmplength}
\settowidth{\labelsep}{X}
\addtolength{\leftmargin}{\labelsep}
\setlength{\labelwidth}{\tmplength}
}
\begin{flushleft}
\item[\textbf{Declaração}\hfill]
\begin{ttfamily}
public constructor CreateLista(AOrdem:Boolean;aLista:tString;const aFoundTesteCompleto:Boolean);\end{ttfamily}


\end{flushleft}
\end{list}
\paragraph*{NewStr}\hspace*{\fill}

\begin{list}{}{
\settowidth{\tmplength}{\textbf{Declaração}}
\setlength{\itemindent}{0cm}
\setlength{\listparindent}{0cm}
\setlength{\leftmargin}{\evensidemargin}
\addtolength{\leftmargin}{\tmplength}
\settowidth{\labelsep}{X}
\addtolength{\leftmargin}{\labelsep}
\setlength{\labelwidth}{\tmplength}
}
\begin{flushleft}
\item[\textbf{Declaração}\hfill]
\begin{ttfamily}
public function NewStr(S : AnsiString):Boolean;\end{ttfamily}


\end{flushleft}
\end{list}
\paragraph*{Append}\hspace*{\fill}

\begin{list}{}{
\settowidth{\tmplength}{\textbf{Declaração}}
\setlength{\itemindent}{0cm}
\setlength{\listparindent}{0cm}
\setlength{\leftmargin}{\evensidemargin}
\addtolength{\leftmargin}{\tmplength}
\settowidth{\labelsep}{X}
\addtolength{\leftmargin}{\labelsep}
\setlength{\labelwidth}{\tmplength}
}
\begin{flushleft}
\item[\textbf{Declaração}\hfill]
\begin{ttfamily}
public function Append(S : AnsiString):Boolean;\end{ttfamily}


\end{flushleft}
\end{list}
\paragraph*{AddSItem}\hspace*{\fill}

\begin{list}{}{
\settowidth{\tmplength}{\textbf{Declaração}}
\setlength{\itemindent}{0cm}
\setlength{\listparindent}{0cm}
\setlength{\leftmargin}{\evensidemargin}
\addtolength{\leftmargin}{\tmplength}
\settowidth{\labelsep}{X}
\addtolength{\leftmargin}{\labelsep}
\setlength{\labelwidth}{\tmplength}
}
\begin{flushleft}
\item[\textbf{Declaração}\hfill]
\begin{ttfamily}
public procedure AddSItem(P : PSItem;OkDisposeSItems:Boolean); Overload;\end{ttfamily}


\end{flushleft}
\end{list}
\paragraph*{AddSItem}\hspace*{\fill}

\begin{list}{}{
\settowidth{\tmplength}{\textbf{Declaração}}
\setlength{\itemindent}{0cm}
\setlength{\listparindent}{0cm}
\setlength{\leftmargin}{\evensidemargin}
\addtolength{\leftmargin}{\tmplength}
\settowidth{\labelsep}{X}
\addtolength{\leftmargin}{\labelsep}
\setlength{\labelwidth}{\tmplength}
}
\begin{flushleft}
\item[\textbf{Declaração}\hfill]
\begin{ttfamily}
public procedure AddSItem(P : PSItem); Overload;\end{ttfamily}


\end{flushleft}
\end{list}
\paragraph*{PListSItem}\hspace*{\fill}

\begin{list}{}{
\settowidth{\tmplength}{\textbf{Declaração}}
\setlength{\itemindent}{0cm}
\setlength{\listparindent}{0cm}
\setlength{\leftmargin}{\evensidemargin}
\addtolength{\leftmargin}{\tmplength}
\settowidth{\labelsep}{X}
\addtolength{\leftmargin}{\labelsep}
\setlength{\labelwidth}{\tmplength}
}
\begin{flushleft}
\item[\textbf{Declaração}\hfill]
\begin{ttfamily}
public function PListSItem: PSItem;\end{ttfamily}


\end{flushleft}
\end{list}
\paragraph*{Get{\_}Html{\_}List}\hspace*{\fill}

\begin{list}{}{
\settowidth{\tmplength}{\textbf{Declaração}}
\setlength{\itemindent}{0cm}
\setlength{\listparindent}{0cm}
\setlength{\leftmargin}{\evensidemargin}
\addtolength{\leftmargin}{\tmplength}
\settowidth{\labelsep}{X}
\addtolength{\leftmargin}{\labelsep}
\setlength{\labelwidth}{\tmplength}
}
\begin{flushleft}
\item[\textbf{Declaração}\hfill]
\begin{ttfamily}
public function Get{\_}Html{\_}List:AnsiString;\end{ttfamily}


\end{flushleft}
\par
\item[\textbf{Descrição}]
Retorna Uma sequencia de {$<$}li{$>$} {$<$}/li{$>$}

\end{list}
\paragraph*{Found}\hspace*{\fill}

\begin{list}{}{
\settowidth{\tmplength}{\textbf{Declaração}}
\setlength{\itemindent}{0cm}
\setlength{\listparindent}{0cm}
\setlength{\leftmargin}{\evensidemargin}
\addtolength{\leftmargin}{\tmplength}
\settowidth{\labelsep}{X}
\addtolength{\leftmargin}{\labelsep}
\setlength{\labelwidth}{\tmplength}
}
\begin{flushleft}
\item[\textbf{Declaração}\hfill]
\begin{ttfamily}
public function Found(const akey:tString):Boolean;\end{ttfamily}


\end{flushleft}
\end{list}
\paragraph*{GetMaiorString}\hspace*{\fill}

\begin{list}{}{
\settowidth{\tmplength}{\textbf{Declaração}}
\setlength{\itemindent}{0cm}
\setlength{\listparindent}{0cm}
\setlength{\leftmargin}{\evensidemargin}
\addtolength{\leftmargin}{\tmplength}
\settowidth{\labelsep}{X}
\addtolength{\leftmargin}{\labelsep}
\setlength{\labelwidth}{\tmplength}
}
\begin{flushleft}
\item[\textbf{Declaração}\hfill]
\begin{ttfamily}
public function GetMaiorString(Const aConjDespreze:AnsiCharSet;aIgnore{\_}ShowWid:Boolean) : Byte; Overload;\end{ttfamily}


\end{flushleft}
\end{list}
\paragraph*{GetMaiorString}\hspace*{\fill}

\begin{list}{}{
\settowidth{\tmplength}{\textbf{Declaração}}
\setlength{\itemindent}{0cm}
\setlength{\listparindent}{0cm}
\setlength{\leftmargin}{\evensidemargin}
\addtolength{\leftmargin}{\tmplength}
\settowidth{\labelsep}{X}
\addtolength{\leftmargin}{\labelsep}
\setlength{\labelwidth}{\tmplength}
}
\begin{flushleft}
\item[\textbf{Declaração}\hfill]
\begin{ttfamily}
public function GetMaiorString(Const aConjDespreze:AnsiCharSet) : Byte; Overload;\end{ttfamily}


\end{flushleft}
\end{list}
\paragraph*{GetMaiorAnsiString}\hspace*{\fill}

\begin{list}{}{
\settowidth{\tmplength}{\textbf{Declaração}}
\setlength{\itemindent}{0cm}
\setlength{\listparindent}{0cm}
\setlength{\leftmargin}{\evensidemargin}
\addtolength{\leftmargin}{\tmplength}
\settowidth{\labelsep}{X}
\addtolength{\leftmargin}{\labelsep}
\setlength{\labelwidth}{\tmplength}
}
\begin{flushleft}
\item[\textbf{Declaração}\hfill]
\begin{ttfamily}
public function GetMaiorAnsiString() : Integer; Overload;\end{ttfamily}


\end{flushleft}
\end{list}
\paragraph*{Clone}\hspace*{\fill}

\begin{list}{}{
\settowidth{\tmplength}{\textbf{Declaração}}
\setlength{\itemindent}{0cm}
\setlength{\listparindent}{0cm}
\setlength{\leftmargin}{\evensidemargin}
\addtolength{\leftmargin}{\tmplength}
\settowidth{\labelsep}{X}
\addtolength{\leftmargin}{\labelsep}
\setlength{\labelwidth}{\tmplength}
}
\begin{flushleft}
\item[\textbf{Declaração}\hfill]
\begin{ttfamily}
public function Clone:TCollectionString;\end{ttfamily}


\end{flushleft}
\end{list}
\paragraph*{Search}\hspace*{\fill}

\begin{list}{}{
\settowidth{\tmplength}{\textbf{Declaração}}
\setlength{\itemindent}{0cm}
\setlength{\listparindent}{0cm}
\setlength{\leftmargin}{\evensidemargin}
\addtolength{\leftmargin}{\tmplength}
\settowidth{\labelsep}{X}
\addtolength{\leftmargin}{\labelsep}
\setlength{\labelwidth}{\tmplength}
}
\begin{flushleft}
\item[\textbf{Declaração}\hfill]
\begin{ttfamily}
public function Search(Key: Pointer; Var Index: Sw{\_}Integer): Boolean; Override;\end{ttfamily}


\end{flushleft}
\end{list}
\paragraph*{FormatStr}\hspace*{\fill}

\begin{list}{}{
\settowidth{\tmplength}{\textbf{Declaração}}
\setlength{\itemindent}{0cm}
\setlength{\listparindent}{0cm}
\setlength{\leftmargin}{\evensidemargin}
\addtolength{\leftmargin}{\tmplength}
\settowidth{\labelsep}{X}
\addtolength{\leftmargin}{\labelsep}
\setlength{\labelwidth}{\tmplength}
}
\begin{flushleft}
\item[\textbf{Declaração}\hfill]
\begin{ttfamily}
public procedure FormatStr(LengthMaxCol: Integer);\end{ttfamily}


\end{flushleft}
\end{list}
\paragraph*{FreeItem}\hspace*{\fill}

\begin{list}{}{
\settowidth{\tmplength}{\textbf{Declaração}}
\setlength{\itemindent}{0cm}
\setlength{\listparindent}{0cm}
\setlength{\leftmargin}{\evensidemargin}
\addtolength{\leftmargin}{\tmplength}
\settowidth{\labelsep}{X}
\addtolength{\leftmargin}{\labelsep}
\setlength{\labelwidth}{\tmplength}
}
\begin{flushleft}
\item[\textbf{Declaração}\hfill]
\begin{ttfamily}
public procedure FreeItem(Item: Pointer); Override;\end{ttfamily}


\end{flushleft}
\end{list}
\paragraph*{WriteSItems}\hspace*{\fill}

\begin{list}{}{
\settowidth{\tmplength}{\textbf{Declaração}}
\setlength{\itemindent}{0cm}
\setlength{\listparindent}{0cm}
\setlength{\leftmargin}{\evensidemargin}
\addtolength{\leftmargin}{\tmplength}
\settowidth{\labelsep}{X}
\addtolength{\leftmargin}{\labelsep}
\setlength{\labelwidth}{\tmplength}
}
\begin{flushleft}
\item[\textbf{Declaração}\hfill]
\begin{ttfamily}
public class procedure WriteSItems(var S: TCollectionString; Const Items: PSItem);\end{ttfamily}


\end{flushleft}
\end{list}
\paragraph*{CloneSItems}\hspace*{\fill}

\begin{list}{}{
\settowidth{\tmplength}{\textbf{Declaração}}
\setlength{\itemindent}{0cm}
\setlength{\listparindent}{0cm}
\setlength{\leftmargin}{\evensidemargin}
\addtolength{\leftmargin}{\tmplength}
\settowidth{\labelsep}{X}
\addtolength{\leftmargin}{\labelsep}
\setlength{\labelwidth}{\tmplength}
}
\begin{flushleft}
\item[\textbf{Declaração}\hfill]
\begin{ttfamily}
public class function CloneSItems(Const Items: PSItem):PSItem;\end{ttfamily}


\end{flushleft}
\end{list}
\paragraph*{CopyTemplateFrom}\hspace*{\fill}

\begin{list}{}{
\settowidth{\tmplength}{\textbf{Declaração}}
\setlength{\itemindent}{0cm}
\setlength{\listparindent}{0cm}
\setlength{\leftmargin}{\evensidemargin}
\addtolength{\leftmargin}{\tmplength}
\settowidth{\labelsep}{X}
\addtolength{\leftmargin}{\labelsep}
\setlength{\labelwidth}{\tmplength}
}
\begin{flushleft}
\item[\textbf{Declaração}\hfill]
\begin{ttfamily}
public class Function CopyTemplateFrom(Const aTemplate:tString): tString;\end{ttfamily}


\end{flushleft}
\end{list}
\section{Tipos}
\subsection*{TStringCollection}
\begin{list}{}{
\settowidth{\tmplength}{\textbf{Declaração}}
\setlength{\itemindent}{0cm}
\setlength{\listparindent}{0cm}
\setlength{\leftmargin}{\evensidemargin}
\addtolength{\leftmargin}{\tmplength}
\settowidth{\labelsep}{X}
\addtolength{\leftmargin}{\labelsep}
\setlength{\labelwidth}{\tmplength}
}
\begin{flushleft}
\item[\textbf{Declaração}\hfill]
\begin{ttfamily}
TStringCollection = mi.rtl.objects.methods.Collection.Sortedcollection.StringCollection.TStringCollection;\end{ttfamily}


\end{flushleft}
\end{list}
\chapter{Unit mi.rtl.objects.Methods.dates}
\section{Descrição}
{-}A Unit \textbf{\begin{ttfamily}mi.rtl.objects.Methods.dates\end{ttfamily}} implementa a classe \textbf{\begin{ttfamily}TDates\end{ttfamily}(\ref{mi.rtl.objects.Methods.dates.TDates})}.

\begin{itemize}
\item \textbf{VERSÃO} \begin{itemize}
\item Alpha {-} 0.5.0.687
\end{itemize}
\item \textbf{CÓDIGO FONTE}: \begin{itemize}
\item 
\end{itemize}
\item \textbf{HISTÓRICO} \begin{itemize}
\item Criado por: Paulo Sérgio da Silva Pacheco e{-}mail: paulosspacheco@yahoo.com.br \begin{itemize}
\item \textbf{27/01/99 05:12 a 08:00} {-} Retirei do Db{\_}Generic e Db{\_}Global todas as rotinas com referência a datas.
\item \textbf{27/01/99 09:05 a 09:30} {-} Debugar as rotinas de datas utilizando o ano 2000.
\item \textbf{27/01/99 09:30 a } {-} Todos os relatórios que necessitam de datas de início e fim do período devem ser inicializadas com as datas mínimas e máximas.
\item \textbf{15/12/21 13:50 a 14:30} {-} Criar classe mi.rtl.objects.Methods.dates.Tdates e adicionar as rotinas de datas do passado.
\end{itemize}
\end{itemize}
\end{itemize}
\section{Uses}
\begin{itemize}
\item \begin{ttfamily}Classes\end{ttfamily}\item \begin{ttfamily}SysUtils\end{ttfamily}\item \begin{ttfamily}DateUtils\end{ttfamily}\item \begin{ttfamily}dos\end{ttfamily}\item \begin{ttfamily}mi.rtl.objects.Methods\end{ttfamily}(\ref{mi.rtl.Objects.Methods})\item \begin{ttfamily}mi.rtl.objects.consts.MI{\_}MsgBox\end{ttfamily}\end{itemize}
\section{Visão Geral}
\begin{description}
\item[\texttt{\begin{ttfamily}TDates\end{ttfamily} Classe}]
\end{description}
\section{Classes, Interfaces, Objetos e Registros}
\subsection*{TDates Classe}
\subsubsection*{\large{\textbf{Hierarquia}}\normalsize\hspace{1ex}\hfill}
TDates {$>$} \begin{ttfamily}TObjectsMethods\end{ttfamily}(\ref{mi.rtl.Objects.Methods.TObjectsMethods}) {$>$} \begin{ttfamily}TObjectsConsts\end{ttfamily}(\ref{mi.rtl.Objects.Consts.TObjectsConsts}) {$>$} 
TObjectsTypes
\subsubsection*{\large{\textbf{Descrição}}\normalsize\hspace{1ex}\hfill}
\begin{itemize}
\item A classe \textbf{\begin{ttfamily}TDates\end{ttfamily}} contém todas os métodos necessários para acessar data no formato de 3 bytes.

\begin{itemize}
\item NOTA** \begin{itemize}
\item Formato de data e hora tratado por essa classe: \begin{itemize}
\item type \begin{ttfamily}TypeData\end{ttfamily}(\ref{mi.rtl.objects.Methods.dates.TDates-TypeData}) = Record dia:byte;mes:Byte;ano : byte; End;
\item type \begin{ttfamily}TipoHora\end{ttfamily}(\ref{mi.rtl.objects.Methods.dates.TDates-TipoHora}) = record H,M,S,S100 : Word; end;
\item Type TData{\_}e{\_}Hora{\_}Compactada = Longint;
\end{itemize}
\end{itemize}
\end{itemize}
\end{itemize}\subsubsection*{\large{\textbf{Campos}}\normalsize\hspace{1ex}\hfill}
\paragraph*{AnoLimit}\hspace*{\fill}

\begin{list}{}{
\settowidth{\tmplength}{\textbf{Declaração}}
\setlength{\itemindent}{0cm}
\setlength{\listparindent}{0cm}
\setlength{\leftmargin}{\evensidemargin}
\addtolength{\leftmargin}{\tmplength}
\settowidth{\labelsep}{X}
\addtolength{\leftmargin}{\labelsep}
\setlength{\labelwidth}{\tmplength}
}
\begin{flushleft}
\item[\textbf{Declaração}\hfill]
\begin{ttfamily}
public const AnoLimit : Byte = 30;\end{ttfamily}


\end{flushleft}
\end{list}
\paragraph*{DataMinima}\hspace*{\fill}

\begin{list}{}{
\settowidth{\tmplength}{\textbf{Declaração}}
\setlength{\itemindent}{0cm}
\setlength{\listparindent}{0cm}
\setlength{\leftmargin}{\evensidemargin}
\addtolength{\leftmargin}{\tmplength}
\settowidth{\labelsep}{X}
\addtolength{\leftmargin}{\labelsep}
\setlength{\labelwidth}{\tmplength}
}
\begin{flushleft}
\item[\textbf{Declaração}\hfill]
\begin{ttfamily}
public const DataMinima : TypeData = (dia:1;mes:1  ;ano:{\_}AnoLimit);\end{ttfamily}


\end{flushleft}
\end{list}
\paragraph*{DataMaxima}\hspace*{\fill}

\begin{list}{}{
\settowidth{\tmplength}{\textbf{Declaração}}
\setlength{\itemindent}{0cm}
\setlength{\listparindent}{0cm}
\setlength{\leftmargin}{\evensidemargin}
\addtolength{\leftmargin}{\tmplength}
\settowidth{\labelsep}{X}
\addtolength{\leftmargin}{\labelsep}
\setlength{\labelwidth}{\tmplength}
}
\begin{flushleft}
\item[\textbf{Declaração}\hfill]
\begin{ttfamily}
public const DataMaxima : TypeData = (dia:31;mes:12;ano:{\_}AnoLimit-1);\end{ttfamily}


\end{flushleft}
\end{list}
\paragraph*{ArrayStrMeses}\hspace*{\fill}

\begin{list}{}{
\settowidth{\tmplength}{\textbf{Declaração}}
\setlength{\itemindent}{0cm}
\setlength{\listparindent}{0cm}
\setlength{\leftmargin}{\evensidemargin}
\addtolength{\leftmargin}{\tmplength}
\settowidth{\labelsep}{X}
\addtolength{\leftmargin}{\labelsep}
\setlength{\labelwidth}{\tmplength}
}
\begin{flushleft}
\item[\textbf{Declaração}\hfill]
\begin{ttfamily}
public const ArrayStrMeses : Array[TMeses] of string = ('',
                                                            'Janeiro',
                                                            'Fevereiro',
                                                            'Marco',
                                                            'Abril',
                                                            'Maio',
                                                            'Junho',
                                                            'Julho',
                                                            'Agosto',
                                                            'Setembro',
                                                            'Outubro',
                                                            'Novembro',
                                                            'Dezembro');\end{ttfamily}


\end{flushleft}
\end{list}
\paragraph*{StrDiaSemana}\hspace*{\fill}

\begin{list}{}{
\settowidth{\tmplength}{\textbf{Declaração}}
\setlength{\itemindent}{0cm}
\setlength{\listparindent}{0cm}
\setlength{\leftmargin}{\evensidemargin}
\addtolength{\leftmargin}{\tmplength}
\settowidth{\labelsep}{X}
\addtolength{\leftmargin}{\labelsep}
\setlength{\labelwidth}{\tmplength}
}
\begin{flushleft}
\item[\textbf{Declaração}\hfill]
\begin{ttfamily}
public const StrDiaSemana  : Array[0..6] of String[7] = ('Sabado ',
                                                             'Domingo',
                                                             'Segunda',
                                                             'Terca  ',
                                                             'Quarta ',
                                                             'Quinta ',
                                                             'Sexta  ');\end{ttfamily}


\end{flushleft}
\end{list}
\paragraph*{HoraInicial}\hspace*{\fill}

\begin{list}{}{
\settowidth{\tmplength}{\textbf{Declaração}}
\setlength{\itemindent}{0cm}
\setlength{\listparindent}{0cm}
\setlength{\leftmargin}{\evensidemargin}
\addtolength{\leftmargin}{\tmplength}
\settowidth{\labelsep}{X}
\addtolength{\leftmargin}{\labelsep}
\setlength{\labelwidth}{\tmplength}
}
\begin{flushleft}
\item[\textbf{Declaração}\hfill]
\begin{ttfamily}
public const HoraInicial : TipoHora = (H    : 0; 
                                           M    : 0; 
                                           S    : 0; 
                                           S100 : 0 );\end{ttfamily}


\end{flushleft}
\end{list}
\paragraph*{DataSistOp}\hspace*{\fill}

\begin{list}{}{
\settowidth{\tmplength}{\textbf{Declaração}}
\setlength{\itemindent}{0cm}
\setlength{\listparindent}{0cm}
\setlength{\leftmargin}{\evensidemargin}
\addtolength{\leftmargin}{\tmplength}
\settowidth{\labelsep}{X}
\addtolength{\leftmargin}{\labelsep}
\setlength{\labelwidth}{\tmplength}
}
\begin{flushleft}
\item[\textbf{Declaração}\hfill]
\begin{ttfamily}
public DataSistOp: TypeData; static;\end{ttfamily}


\end{flushleft}
\end{list}
\paragraph*{TempData}\hspace*{\fill}

\begin{list}{}{
\settowidth{\tmplength}{\textbf{Declaração}}
\setlength{\itemindent}{0cm}
\setlength{\listparindent}{0cm}
\setlength{\leftmargin}{\evensidemargin}
\addtolength{\leftmargin}{\tmplength}
\settowidth{\labelsep}{X}
\addtolength{\leftmargin}{\labelsep}
\setlength{\labelwidth}{\tmplength}
}
\begin{flushleft}
\item[\textbf{Declaração}\hfill]
\begin{ttfamily}
public TempData: TypeData; static;\end{ttfamily}


\end{flushleft}
\end{list}
\subsubsection*{\large{\textbf{Métodos}}\normalsize\hspace{1ex}\hfill}
\paragraph*{juliano}\hspace*{\fill}

\begin{list}{}{
\settowidth{\tmplength}{\textbf{Declaração}}
\setlength{\itemindent}{0cm}
\setlength{\listparindent}{0cm}
\setlength{\leftmargin}{\evensidemargin}
\addtolength{\leftmargin}{\tmplength}
\settowidth{\labelsep}{X}
\addtolength{\leftmargin}{\labelsep}
\setlength{\labelwidth}{\tmplength}
}
\begin{flushleft}
\item[\textbf{Declaração}\hfill]
\begin{ttfamily}
public class function juliano(d,m,a : SmallInt) : TRealNum;\end{ttfamily}


\end{flushleft}
\end{list}
\paragraph*{DifeData}\hspace*{\fill}

\begin{list}{}{
\settowidth{\tmplength}{\textbf{Declaração}}
\setlength{\itemindent}{0cm}
\setlength{\listparindent}{0cm}
\setlength{\leftmargin}{\evensidemargin}
\addtolength{\leftmargin}{\tmplength}
\settowidth{\labelsep}{X}
\addtolength{\leftmargin}{\labelsep}
\setlength{\labelwidth}{\tmplength}
}
\begin{flushleft}
\item[\textbf{Declaração}\hfill]
\begin{ttfamily}
public class function DifeData(diaAtual, mesAtual, anoAtual,diaAnterior , mesAnterior , anoAnterior : byte ) :Longint; Overload;\end{ttfamily}


\end{flushleft}
\end{list}
\paragraph*{DifeData}\hspace*{\fill}

\begin{list}{}{
\settowidth{\tmplength}{\textbf{Declaração}}
\setlength{\itemindent}{0cm}
\setlength{\listparindent}{0cm}
\setlength{\leftmargin}{\evensidemargin}
\addtolength{\leftmargin}{\tmplength}
\settowidth{\labelsep}{X}
\addtolength{\leftmargin}{\labelsep}
\setlength{\labelwidth}{\tmplength}
}
\begin{flushleft}
\item[\textbf{Declaração}\hfill]
\begin{ttfamily}
public class function DifeData(Const DAntBuff,DAtuBuff:TypeData) : Longint; Overload;\end{ttfamily}


\end{flushleft}
\end{list}
\paragraph*{DifeData}\hspace*{\fill}

\begin{list}{}{
\settowidth{\tmplength}{\textbf{Declaração}}
\setlength{\itemindent}{0cm}
\setlength{\listparindent}{0cm}
\setlength{\leftmargin}{\evensidemargin}
\addtolength{\leftmargin}{\tmplength}
\settowidth{\labelsep}{X}
\addtolength{\leftmargin}{\labelsep}
\setlength{\labelwidth}{\tmplength}
}
\begin{flushleft}
\item[\textbf{Declaração}\hfill]
\begin{ttfamily}
public class function DifeData(Const DatAnterior:TypeData;Const DatAtual : TypeData; Const Operador:AnsiChar; Const operando : Longint) : Boolean; Overload;\end{ttfamily}


\end{flushleft}
\end{list}
\paragraph*{somaData}\hspace*{\fill}

\begin{list}{}{
\settowidth{\tmplength}{\textbf{Declaração}}
\setlength{\itemindent}{0cm}
\setlength{\listparindent}{0cm}
\setlength{\leftmargin}{\evensidemargin}
\addtolength{\leftmargin}{\tmplength}
\settowidth{\labelsep}{X}
\addtolength{\leftmargin}{\labelsep}
\setlength{\labelwidth}{\tmplength}
}
\begin{flushleft}
\item[\textbf{Declaração}\hfill]
\begin{ttfamily}
public class procedure somaData(var dia,mes:Byte; Var ano : SmallInt ; diasAsomar : Integer); Overload;\end{ttfamily}


\end{flushleft}
\end{list}
\paragraph*{somaData}\hspace*{\fill}

\begin{list}{}{
\settowidth{\tmplength}{\textbf{Declaração}}
\setlength{\itemindent}{0cm}
\setlength{\listparindent}{0cm}
\setlength{\leftmargin}{\evensidemargin}
\addtolength{\leftmargin}{\tmplength}
\settowidth{\labelsep}{X}
\addtolength{\leftmargin}{\labelsep}
\setlength{\labelwidth}{\tmplength}
}
\begin{flushleft}
\item[\textbf{Declaração}\hfill]
\begin{ttfamily}
public class procedure somaData(var dia,mes,ano : byte ; diasAsomar : Integer); Overload;\end{ttfamily}


\end{flushleft}
\end{list}
\paragraph*{SomaData}\hspace*{\fill}

\begin{list}{}{
\settowidth{\tmplength}{\textbf{Declaração}}
\setlength{\itemindent}{0cm}
\setlength{\listparindent}{0cm}
\setlength{\leftmargin}{\evensidemargin}
\addtolength{\leftmargin}{\tmplength}
\settowidth{\labelsep}{X}
\addtolength{\leftmargin}{\labelsep}
\setlength{\labelwidth}{\tmplength}
}
\begin{flushleft}
\item[\textbf{Declaração}\hfill]
\begin{ttfamily}
public class procedure SomaData(Var Buff:TypeData;Prazo: Integer) ; Overload;\end{ttfamily}


\end{flushleft}
\end{list}
\paragraph*{FSomaData}\hspace*{\fill}

\begin{list}{}{
\settowidth{\tmplength}{\textbf{Declaração}}
\setlength{\itemindent}{0cm}
\setlength{\listparindent}{0cm}
\setlength{\leftmargin}{\evensidemargin}
\addtolength{\leftmargin}{\tmplength}
\settowidth{\labelsep}{X}
\addtolength{\leftmargin}{\labelsep}
\setlength{\labelwidth}{\tmplength}
}
\begin{flushleft}
\item[\textbf{Declaração}\hfill]
\begin{ttfamily}
public class function FSomaData(Buff:TypeData;Prazo: Integer):TString ; Overload;\end{ttfamily}


\end{flushleft}
\end{list}
\paragraph*{SomaDataEmMeses}\hspace*{\fill}

\begin{list}{}{
\settowidth{\tmplength}{\textbf{Declaração}}
\setlength{\itemindent}{0cm}
\setlength{\listparindent}{0cm}
\setlength{\leftmargin}{\evensidemargin}
\addtolength{\leftmargin}{\tmplength}
\settowidth{\labelsep}{X}
\addtolength{\leftmargin}{\labelsep}
\setlength{\labelwidth}{\tmplength}
}
\begin{flushleft}
\item[\textbf{Declaração}\hfill]
\begin{ttfamily}
public class procedure SomaDataEmMeses(Const DataFont : TypeData; Const Meses : SmallInt; Var DataDest : TypeData);\end{ttfamily}


\end{flushleft}
\end{list}
\paragraph*{SomeDataPara}\hspace*{\fill}

\begin{list}{}{
\settowidth{\tmplength}{\textbf{Declaração}}
\setlength{\itemindent}{0cm}
\setlength{\listparindent}{0cm}
\setlength{\leftmargin}{\evensidemargin}
\addtolength{\leftmargin}{\tmplength}
\settowidth{\labelsep}{X}
\addtolength{\leftmargin}{\labelsep}
\setlength{\labelwidth}{\tmplength}
}
\begin{flushleft}
\item[\textbf{Declaração}\hfill]
\begin{ttfamily}
public class procedure SomeDataPara(Const Buff1:TypeData;Var Buff2:TypeData ;Const Prazo: Integer) ;\end{ttfamily}


\end{flushleft}
\end{list}
\paragraph*{Dtjuliana}\hspace*{\fill}

\begin{list}{}{
\settowidth{\tmplength}{\textbf{Declaração}}
\setlength{\itemindent}{0cm}
\setlength{\listparindent}{0cm}
\setlength{\leftmargin}{\evensidemargin}
\addtolength{\leftmargin}{\tmplength}
\settowidth{\labelsep}{X}
\addtolength{\leftmargin}{\labelsep}
\setlength{\labelwidth}{\tmplength}
}
\begin{flushleft}
\item[\textbf{Declaração}\hfill]
\begin{ttfamily}
public class function Dtjuliana(Var Buff) : TRealNum;\end{ttfamily}


\end{flushleft}
\end{list}
\paragraph*{moveData}\hspace*{\fill}

\begin{list}{}{
\settowidth{\tmplength}{\textbf{Declaração}}
\setlength{\itemindent}{0cm}
\setlength{\listparindent}{0cm}
\setlength{\leftmargin}{\evensidemargin}
\addtolength{\leftmargin}{\tmplength}
\settowidth{\labelsep}{X}
\addtolength{\leftmargin}{\labelsep}
\setlength{\labelwidth}{\tmplength}
}
\begin{flushleft}
\item[\textbf{Declaração}\hfill]
\begin{ttfamily}
public class procedure moveData(var dataFonte, dataDestino);\end{ttfamily}


\end{flushleft}
\end{list}
\paragraph*{ConvNomeData}\hspace*{\fill}

\begin{list}{}{
\settowidth{\tmplength}{\textbf{Declaração}}
\setlength{\itemindent}{0cm}
\setlength{\listparindent}{0cm}
\setlength{\leftmargin}{\evensidemargin}
\addtolength{\leftmargin}{\tmplength}
\settowidth{\labelsep}{X}
\addtolength{\leftmargin}{\labelsep}
\setlength{\labelwidth}{\tmplength}
}
\begin{flushleft}
\item[\textbf{Declaração}\hfill]
\begin{ttfamily}
public class function ConvNomeData(Const NomeArqFonte :String; NomeArqDestino : String; Const Mes,Ano : Byte; var Mensagem : String ) : String;\end{ttfamily}


\end{flushleft}
\end{list}
\paragraph*{StrDataMesAno}\hspace*{\fill}

\begin{list}{}{
\settowidth{\tmplength}{\textbf{Declaração}}
\setlength{\itemindent}{0cm}
\setlength{\listparindent}{0cm}
\setlength{\leftmargin}{\evensidemargin}
\addtolength{\leftmargin}{\tmplength}
\settowidth{\labelsep}{X}
\addtolength{\leftmargin}{\labelsep}
\setlength{\labelwidth}{\tmplength}
}
\begin{flushleft}
\item[\textbf{Declaração}\hfill]
\begin{ttfamily}
public class function StrDataMesAno(Const mes,Ano:byte) : String;\end{ttfamily}


\end{flushleft}
\end{list}
\paragraph*{DiaMaxDoMes}\hspace*{\fill}

\begin{list}{}{
\settowidth{\tmplength}{\textbf{Declaração}}
\setlength{\itemindent}{0cm}
\setlength{\listparindent}{0cm}
\setlength{\leftmargin}{\evensidemargin}
\addtolength{\leftmargin}{\tmplength}
\settowidth{\labelsep}{X}
\addtolength{\leftmargin}{\labelsep}
\setlength{\labelwidth}{\tmplength}
}
\begin{flushleft}
\item[\textbf{Declaração}\hfill]
\begin{ttfamily}
public class function DiaMaxDoMes(Const Mes : Byte;Ano : Integer) : Byte;\end{ttfamily}


\end{flushleft}
\end{list}
\paragraph*{FDiaSemana}\hspace*{\fill}

\begin{list}{}{
\settowidth{\tmplength}{\textbf{Declaração}}
\setlength{\itemindent}{0cm}
\setlength{\listparindent}{0cm}
\setlength{\leftmargin}{\evensidemargin}
\addtolength{\leftmargin}{\tmplength}
\settowidth{\labelsep}{X}
\addtolength{\leftmargin}{\labelsep}
\setlength{\labelwidth}{\tmplength}
}
\begin{flushleft}
\item[\textbf{Declaração}\hfill]
\begin{ttfamily}
public class function FDiaSemana(Var BuffData) : Byte;\end{ttfamily}


\end{flushleft}
\end{list}
\paragraph*{FStrDiaSemana}\hspace*{\fill}

\begin{list}{}{
\settowidth{\tmplength}{\textbf{Declaração}}
\setlength{\itemindent}{0cm}
\setlength{\listparindent}{0cm}
\setlength{\leftmargin}{\evensidemargin}
\addtolength{\leftmargin}{\tmplength}
\settowidth{\labelsep}{X}
\addtolength{\leftmargin}{\labelsep}
\setlength{\labelwidth}{\tmplength}
}
\begin{flushleft}
\item[\textbf{Declaração}\hfill]
\begin{ttfamily}
public class function FStrDiaSemana(Var data ) : String;\end{ttfamily}


\end{flushleft}
\end{list}
\paragraph*{StrMes}\hspace*{\fill}

\begin{list}{}{
\settowidth{\tmplength}{\textbf{Declaração}}
\setlength{\itemindent}{0cm}
\setlength{\listparindent}{0cm}
\setlength{\leftmargin}{\evensidemargin}
\addtolength{\leftmargin}{\tmplength}
\settowidth{\labelsep}{X}
\addtolength{\leftmargin}{\labelsep}
\setlength{\labelwidth}{\tmplength}
}
\begin{flushleft}
\item[\textbf{Declaração}\hfill]
\begin{ttfamily}
public class function StrMes(Const mes : Word ) : String;\end{ttfamily}


\end{flushleft}
\end{list}
\paragraph*{StrData}\hspace*{\fill}

\begin{list}{}{
\settowidth{\tmplength}{\textbf{Declaração}}
\setlength{\itemindent}{0cm}
\setlength{\listparindent}{0cm}
\setlength{\leftmargin}{\evensidemargin}
\addtolength{\leftmargin}{\tmplength}
\settowidth{\labelsep}{X}
\addtolength{\leftmargin}{\labelsep}
\setlength{\labelwidth}{\tmplength}
}
\begin{flushleft}
\item[\textbf{Declaração}\hfill]
\begin{ttfamily}
public class function StrData(Const Dia,mes,ano: Word ; Const ch :AnsiChar) : string;\end{ttfamily}


\end{flushleft}
\end{list}
\paragraph*{StringData}\hspace*{\fill}

\begin{list}{}{
\settowidth{\tmplength}{\textbf{Declaração}}
\setlength{\itemindent}{0cm}
\setlength{\listparindent}{0cm}
\setlength{\leftmargin}{\evensidemargin}
\addtolength{\leftmargin}{\tmplength}
\settowidth{\labelsep}{X}
\addtolength{\leftmargin}{\labelsep}
\setlength{\labelwidth}{\tmplength}
}
\begin{flushleft}
\item[\textbf{Declaração}\hfill]
\begin{ttfamily}
public class function StringData(Const Buff :TypeData;Const Ch : AnsiChar) : String;\end{ttfamily}


\end{flushleft}
\end{list}
\paragraph*{GetDataSistOp}\hspace*{\fill}

\begin{list}{}{
\settowidth{\tmplength}{\textbf{Declaração}}
\setlength{\itemindent}{0cm}
\setlength{\listparindent}{0cm}
\setlength{\leftmargin}{\evensidemargin}
\addtolength{\leftmargin}{\tmplength}
\settowidth{\labelsep}{X}
\addtolength{\leftmargin}{\labelsep}
\setlength{\labelwidth}{\tmplength}
}
\begin{flushleft}
\item[\textbf{Declaração}\hfill]
\begin{ttfamily}
public class function GetDataSistOp(Var Buff;const Separador:AnsiChar):String;\end{ttfamily}


\end{flushleft}
\end{list}
\paragraph*{FGetDataSistOp}\hspace*{\fill}

\begin{list}{}{
\settowidth{\tmplength}{\textbf{Declaração}}
\setlength{\itemindent}{0cm}
\setlength{\listparindent}{0cm}
\setlength{\leftmargin}{\evensidemargin}
\addtolength{\leftmargin}{\tmplength}
\settowidth{\labelsep}{X}
\addtolength{\leftmargin}{\labelsep}
\setlength{\labelwidth}{\tmplength}
}
\begin{flushleft}
\item[\textbf{Declaração}\hfill]
\begin{ttfamily}
public class function FGetDataSistOp(const Separador:AnsiChar):String;\end{ttfamily}


\end{flushleft}
\end{list}
\paragraph*{GetDateSystem}\hspace*{\fill}

\begin{list}{}{
\settowidth{\tmplength}{\textbf{Declaração}}
\setlength{\itemindent}{0cm}
\setlength{\listparindent}{0cm}
\setlength{\leftmargin}{\evensidemargin}
\addtolength{\leftmargin}{\tmplength}
\settowidth{\labelsep}{X}
\addtolength{\leftmargin}{\labelsep}
\setlength{\labelwidth}{\tmplength}
}
\begin{flushleft}
\item[\textbf{Declaração}\hfill]
\begin{ttfamily}
public class function GetDateSystem(const DateMask:TDateMask):String;\end{ttfamily}


\end{flushleft}
\end{list}
\paragraph*{GetHourSystem}\hspace*{\fill}

\begin{list}{}{
\settowidth{\tmplength}{\textbf{Declaração}}
\setlength{\itemindent}{0cm}
\setlength{\listparindent}{0cm}
\setlength{\leftmargin}{\evensidemargin}
\addtolength{\leftmargin}{\tmplength}
\settowidth{\labelsep}{X}
\addtolength{\leftmargin}{\labelsep}
\setlength{\labelwidth}{\tmplength}
}
\begin{flushleft}
\item[\textbf{Declaração}\hfill]
\begin{ttfamily}
public class function GetHourSystem(const HourMask :THourMask):String;\end{ttfamily}


\end{flushleft}
\end{list}
\paragraph*{GetFTimeDos}\hspace*{\fill}

\begin{list}{}{
\settowidth{\tmplength}{\textbf{Declaração}}
\setlength{\itemindent}{0cm}
\setlength{\listparindent}{0cm}
\setlength{\leftmargin}{\evensidemargin}
\addtolength{\leftmargin}{\tmplength}
\settowidth{\labelsep}{X}
\addtolength{\leftmargin}{\labelsep}
\setlength{\labelwidth}{\tmplength}
}
\begin{flushleft}
\item[\textbf{Declaração}\hfill]
\begin{ttfamily}
public class function GetFTimeDos:Longint; Overload;\end{ttfamily}


\end{flushleft}
\end{list}
\paragraph*{GetFTimeDos}\hspace*{\fill}

\begin{list}{}{
\settowidth{\tmplength}{\textbf{Declaração}}
\setlength{\itemindent}{0cm}
\setlength{\listparindent}{0cm}
\setlength{\leftmargin}{\evensidemargin}
\addtolength{\leftmargin}{\tmplength}
\settowidth{\labelsep}{X}
\addtolength{\leftmargin}{\labelsep}
\setlength{\labelwidth}{\tmplength}
}
\begin{flushleft}
\item[\textbf{Declaração}\hfill]
\begin{ttfamily}
public class function GetFTimeDos(Var wSec1000:SmallInt):Longint; Overload;\end{ttfamily}


\end{flushleft}
\end{list}
\paragraph*{GetFTimeDos}\hspace*{\fill}

\begin{list}{}{
\settowidth{\tmplength}{\textbf{Declaração}}
\setlength{\itemindent}{0cm}
\setlength{\listparindent}{0cm}
\setlength{\leftmargin}{\evensidemargin}
\addtolength{\leftmargin}{\tmplength}
\settowidth{\labelsep}{X}
\addtolength{\leftmargin}{\labelsep}
\setlength{\labelwidth}{\tmplength}
}
\begin{flushleft}
\item[\textbf{Declaração}\hfill]
\begin{ttfamily}
public class function GetFTimeDos(Var wSec100:Byte;Var wSec1000:SmallInt):Longint; Overload;\end{ttfamily}


\end{flushleft}
\end{list}
\paragraph*{GetFTimeDos{\_}Valid}\hspace*{\fill}

\begin{list}{}{
\settowidth{\tmplength}{\textbf{Declaração}}
\setlength{\itemindent}{0cm}
\setlength{\listparindent}{0cm}
\setlength{\leftmargin}{\evensidemargin}
\addtolength{\leftmargin}{\tmplength}
\settowidth{\labelsep}{X}
\addtolength{\leftmargin}{\labelsep}
\setlength{\labelwidth}{\tmplength}
}
\begin{flushleft}
\item[\textbf{Declaração}\hfill]
\begin{ttfamily}
public class function GetFTimeDos{\_}Valid(aTime{\_}UltimoAcesso:Longint;aMinutos{\_}de{\_}tolerancia{\_}do{\_}Ultimo{\_}Acesso:byte):Boolean;\end{ttfamily}


\end{flushleft}
\end{list}
\paragraph*{PackDate}\hspace*{\fill}

\begin{list}{}{
\settowidth{\tmplength}{\textbf{Declaração}}
\setlength{\itemindent}{0cm}
\setlength{\listparindent}{0cm}
\setlength{\leftmargin}{\evensidemargin}
\addtolength{\leftmargin}{\tmplength}
\settowidth{\labelsep}{X}
\addtolength{\leftmargin}{\labelsep}
\setlength{\labelwidth}{\tmplength}
}
\begin{flushleft}
\item[\textbf{Declaração}\hfill]
\begin{ttfamily}
public class function PackDate(Const Data:TypeData):Longint; Overload;\end{ttfamily}


\end{flushleft}
\end{list}
\paragraph*{PackDate}\hspace*{\fill}

\begin{list}{}{
\settowidth{\tmplength}{\textbf{Declaração}}
\setlength{\itemindent}{0cm}
\setlength{\listparindent}{0cm}
\setlength{\leftmargin}{\evensidemargin}
\addtolength{\leftmargin}{\tmplength}
\settowidth{\labelsep}{X}
\addtolength{\leftmargin}{\labelsep}
\setlength{\labelwidth}{\tmplength}
}
\begin{flushleft}
\item[\textbf{Declaração}\hfill]
\begin{ttfamily}
public class function PackDate(Const Data:String; Const Mask: TDateMask):Longint; Overload;\end{ttfamily}


\end{flushleft}
\end{list}
\paragraph*{PackDate}\hspace*{\fill}

\begin{list}{}{
\settowidth{\tmplength}{\textbf{Declaração}}
\setlength{\itemindent}{0cm}
\setlength{\listparindent}{0cm}
\setlength{\leftmargin}{\evensidemargin}
\addtolength{\leftmargin}{\tmplength}
\settowidth{\labelsep}{X}
\addtolength{\leftmargin}{\labelsep}
\setlength{\labelwidth}{\tmplength}
}
\begin{flushleft}
\item[\textbf{Declaração}\hfill]
\begin{ttfamily}
public class function PackDate(Const Data:String; Const Mask: TDateMask; TimePack:Longint):Longint; Overload;\end{ttfamily}


\end{flushleft}
\end{list}
\paragraph*{PackHour}\hspace*{\fill}

\begin{list}{}{
\settowidth{\tmplength}{\textbf{Declaração}}
\setlength{\itemindent}{0cm}
\setlength{\listparindent}{0cm}
\setlength{\leftmargin}{\evensidemargin}
\addtolength{\leftmargin}{\tmplength}
\settowidth{\labelsep}{X}
\addtolength{\leftmargin}{\labelsep}
\setlength{\labelwidth}{\tmplength}
}
\begin{flushleft}
\item[\textbf{Declaração}\hfill]
\begin{ttfamily}
public class function PackHour(Const Hora:String; Const Mask: THourMask; TimePack:Longint):Longint; Overload;\end{ttfamily}


\end{flushleft}
\end{list}
\paragraph*{UnPackDate}\hspace*{\fill}

\begin{list}{}{
\settowidth{\tmplength}{\textbf{Declaração}}
\setlength{\itemindent}{0cm}
\setlength{\listparindent}{0cm}
\setlength{\leftmargin}{\evensidemargin}
\addtolength{\leftmargin}{\tmplength}
\settowidth{\labelsep}{X}
\addtolength{\leftmargin}{\labelsep}
\setlength{\labelwidth}{\tmplength}
}
\begin{flushleft}
\item[\textbf{Declaração}\hfill]
\begin{ttfamily}
public class function UnPackDate(Const TimePack:Longint):TypeData;\end{ttfamily}


\end{flushleft}
\end{list}
\paragraph*{UnPackHora}\hspace*{\fill}

\begin{list}{}{
\settowidth{\tmplength}{\textbf{Declaração}}
\setlength{\itemindent}{0cm}
\setlength{\listparindent}{0cm}
\setlength{\leftmargin}{\evensidemargin}
\addtolength{\leftmargin}{\tmplength}
\settowidth{\labelsep}{X}
\addtolength{\leftmargin}{\labelsep}
\setlength{\labelwidth}{\tmplength}
}
\begin{flushleft}
\item[\textbf{Declaração}\hfill]
\begin{ttfamily}
public class procedure UnPackHora(Const TimePack:Longint;Var Hora : TipoHora);\end{ttfamily}


\end{flushleft}
\end{list}
\paragraph*{PackHora}\hspace*{\fill}

\begin{list}{}{
\settowidth{\tmplength}{\textbf{Declaração}}
\setlength{\itemindent}{0cm}
\setlength{\listparindent}{0cm}
\setlength{\leftmargin}{\evensidemargin}
\addtolength{\leftmargin}{\tmplength}
\settowidth{\labelsep}{X}
\addtolength{\leftmargin}{\labelsep}
\setlength{\labelwidth}{\tmplength}
}
\begin{flushleft}
\item[\textbf{Declaração}\hfill]
\begin{ttfamily}
public class procedure PackHora(Const Hora : TipoHora; Var TimePack:Longint);\end{ttfamily}


\end{flushleft}
\end{list}
\paragraph*{PackDateHora}\hspace*{\fill}

\begin{list}{}{
\settowidth{\tmplength}{\textbf{Declaração}}
\setlength{\itemindent}{0cm}
\setlength{\listparindent}{0cm}
\setlength{\leftmargin}{\evensidemargin}
\addtolength{\leftmargin}{\tmplength}
\settowidth{\labelsep}{X}
\addtolength{\leftmargin}{\labelsep}
\setlength{\labelwidth}{\tmplength}
}
\begin{flushleft}
\item[\textbf{Declaração}\hfill]
\begin{ttfamily}
public class procedure PackDateHora(Data:TypeData; Hora :TipoHora; Var TimePack:Longint);\end{ttfamily}


\end{flushleft}
\end{list}
\paragraph*{UnPackDateHora}\hspace*{\fill}

\begin{list}{}{
\settowidth{\tmplength}{\textbf{Declaração}}
\setlength{\itemindent}{0cm}
\setlength{\listparindent}{0cm}
\setlength{\leftmargin}{\evensidemargin}
\addtolength{\leftmargin}{\tmplength}
\settowidth{\labelsep}{X}
\addtolength{\leftmargin}{\labelsep}
\setlength{\labelwidth}{\tmplength}
}
\begin{flushleft}
\item[\textbf{Declaração}\hfill]
\begin{ttfamily}
public class procedure UnPackDateHora(Const TimePack:Longint; Var Data:TypeData; Var Hora :TipoHora);\end{ttfamily}


\end{flushleft}
\end{list}
\paragraph*{StringTimeD}\hspace*{\fill}

\begin{list}{}{
\settowidth{\tmplength}{\textbf{Declaração}}
\setlength{\itemindent}{0cm}
\setlength{\listparindent}{0cm}
\setlength{\leftmargin}{\evensidemargin}
\addtolength{\leftmargin}{\tmplength}
\settowidth{\labelsep}{X}
\addtolength{\leftmargin}{\labelsep}
\setlength{\labelwidth}{\tmplength}
}
\begin{flushleft}
\item[\textbf{Declaração}\hfill]
\begin{ttfamily}
public class function StringTimeD(Const TimePack:Longint;Const Ch : AnsiChar) : String;\end{ttfamily}


\end{flushleft}
\end{list}
\paragraph*{StringTimeH}\hspace*{\fill}

\begin{list}{}{
\settowidth{\tmplength}{\textbf{Declaração}}
\setlength{\itemindent}{0cm}
\setlength{\listparindent}{0cm}
\setlength{\leftmargin}{\evensidemargin}
\addtolength{\leftmargin}{\tmplength}
\settowidth{\labelsep}{X}
\addtolength{\leftmargin}{\labelsep}
\setlength{\labelwidth}{\tmplength}
}
\begin{flushleft}
\item[\textbf{Declaração}\hfill]
\begin{ttfamily}
public class function StringTimeH(Const TimePack:Longint) : String;\end{ttfamily}


\end{flushleft}
\end{list}
\paragraph*{StringTimeHSemPonto}\hspace*{\fill}

\begin{list}{}{
\settowidth{\tmplength}{\textbf{Declaração}}
\setlength{\itemindent}{0cm}
\setlength{\listparindent}{0cm}
\setlength{\leftmargin}{\evensidemargin}
\addtolength{\leftmargin}{\tmplength}
\settowidth{\labelsep}{X}
\addtolength{\leftmargin}{\labelsep}
\setlength{\labelwidth}{\tmplength}
}
\begin{flushleft}
\item[\textbf{Declaração}\hfill]
\begin{ttfamily}
public class function StringTimeHSemPonto(Const TimePack:Longint) : String;\end{ttfamily}


\end{flushleft}
\end{list}
\paragraph*{FIncAno}\hspace*{\fill}

\begin{list}{}{
\settowidth{\tmplength}{\textbf{Declaração}}
\setlength{\itemindent}{0cm}
\setlength{\listparindent}{0cm}
\setlength{\leftmargin}{\evensidemargin}
\addtolength{\leftmargin}{\tmplength}
\settowidth{\labelsep}{X}
\addtolength{\leftmargin}{\labelsep}
\setlength{\labelwidth}{\tmplength}
}
\begin{flushleft}
\item[\textbf{Declaração}\hfill]
\begin{ttfamily}
public class function FIncAno(Ano:SmallInt) : Byte;\end{ttfamily}


\end{flushleft}
\end{list}
\paragraph*{FDecAno}\hspace*{\fill}

\begin{list}{}{
\settowidth{\tmplength}{\textbf{Declaração}}
\setlength{\itemindent}{0cm}
\setlength{\listparindent}{0cm}
\setlength{\leftmargin}{\evensidemargin}
\addtolength{\leftmargin}{\tmplength}
\settowidth{\labelsep}{X}
\addtolength{\leftmargin}{\labelsep}
\setlength{\labelwidth}{\tmplength}
}
\begin{flushleft}
\item[\textbf{Declaração}\hfill]
\begin{ttfamily}
public class function FDecAno(Ano:SmallInt) : Byte;\end{ttfamily}


\end{flushleft}
\end{list}
\paragraph*{FAno}\hspace*{\fill}

\begin{list}{}{
\settowidth{\tmplength}{\textbf{Declaração}}
\setlength{\itemindent}{0cm}
\setlength{\listparindent}{0cm}
\setlength{\leftmargin}{\evensidemargin}
\addtolength{\leftmargin}{\tmplength}
\settowidth{\labelsep}{X}
\addtolength{\leftmargin}{\labelsep}
\setlength{\labelwidth}{\tmplength}
}
\begin{flushleft}
\item[\textbf{Declaração}\hfill]
\begin{ttfamily}
public class function FAno(Ano:SmallWord) : SmallWord;\end{ttfamily}


\end{flushleft}
\end{list}
\paragraph*{FAno2Digito}\hspace*{\fill}

\begin{list}{}{
\settowidth{\tmplength}{\textbf{Declaração}}
\setlength{\itemindent}{0cm}
\setlength{\listparindent}{0cm}
\setlength{\leftmargin}{\evensidemargin}
\addtolength{\leftmargin}{\tmplength}
\settowidth{\labelsep}{X}
\addtolength{\leftmargin}{\labelsep}
\setlength{\labelwidth}{\tmplength}
}
\begin{flushleft}
\item[\textbf{Declaração}\hfill]
\begin{ttfamily}
public class function FAno2Digito(Ano : SmallWord):Byte;\end{ttfamily}


\end{flushleft}
\end{list}
\paragraph*{FAnoDoIndex}\hspace*{\fill}

\begin{list}{}{
\settowidth{\tmplength}{\textbf{Declaração}}
\setlength{\itemindent}{0cm}
\setlength{\listparindent}{0cm}
\setlength{\leftmargin}{\evensidemargin}
\addtolength{\leftmargin}{\tmplength}
\settowidth{\labelsep}{X}
\addtolength{\leftmargin}{\labelsep}
\setlength{\labelwidth}{\tmplength}
}
\begin{flushleft}
\item[\textbf{Declaração}\hfill]
\begin{ttfamily}
public class function FAnoDoIndex(Const Dia,Ano : byte):String;\end{ttfamily}


\end{flushleft}
\end{list}
\paragraph*{StrAno}\hspace*{\fill}

\begin{list}{}{
\settowidth{\tmplength}{\textbf{Declaração}}
\setlength{\itemindent}{0cm}
\setlength{\listparindent}{0cm}
\setlength{\leftmargin}{\evensidemargin}
\addtolength{\leftmargin}{\tmplength}
\settowidth{\labelsep}{X}
\addtolength{\leftmargin}{\labelsep}
\setlength{\labelwidth}{\tmplength}
}
\begin{flushleft}
\item[\textbf{Declaração}\hfill]
\begin{ttfamily}
public class function StrAno(ano : SmallInt ) : String;\end{ttfamily}


\end{flushleft}
\end{list}
\paragraph*{StrToDate}\hspace*{\fill}

\begin{list}{}{
\settowidth{\tmplength}{\textbf{Declaração}}
\setlength{\itemindent}{0cm}
\setlength{\listparindent}{0cm}
\setlength{\leftmargin}{\evensidemargin}
\addtolength{\leftmargin}{\tmplength}
\settowidth{\labelsep}{X}
\addtolength{\leftmargin}{\labelsep}
\setlength{\labelwidth}{\tmplength}
}
\begin{flushleft}
\item[\textbf{Declaração}\hfill]
\begin{ttfamily}
public class function StrToDate(aStrDate:String; Const Mask: TDateMask):PTypeData;\end{ttfamily}


\end{flushleft}
\end{list}
\paragraph*{DateToStr}\hspace*{\fill}

\begin{list}{}{
\settowidth{\tmplength}{\textbf{Declaração}}
\setlength{\itemindent}{0cm}
\setlength{\listparindent}{0cm}
\setlength{\leftmargin}{\evensidemargin}
\addtolength{\leftmargin}{\tmplength}
\settowidth{\labelsep}{X}
\addtolength{\leftmargin}{\labelsep}
\setlength{\labelwidth}{\tmplength}
}
\begin{flushleft}
\item[\textbf{Declaração}\hfill]
\begin{ttfamily}
public class function DateToStr(Const aDate:TypeData;Const Mask: TDateMask) : String; Overload;\end{ttfamily}


\end{flushleft}
\end{list}
\paragraph*{DateToStr}\hspace*{\fill}

\begin{list}{}{
\settowidth{\tmplength}{\textbf{Declaração}}
\setlength{\itemindent}{0cm}
\setlength{\listparindent}{0cm}
\setlength{\leftmargin}{\evensidemargin}
\addtolength{\leftmargin}{\tmplength}
\settowidth{\labelsep}{X}
\addtolength{\leftmargin}{\labelsep}
\setlength{\labelwidth}{\tmplength}
}
\begin{flushleft}
\item[\textbf{Declaração}\hfill]
\begin{ttfamily}
public class function DateToStr(Const aDate:Longint;Const Mask: TDateMask) : String; Overload;\end{ttfamily}


\end{flushleft}
\end{list}
\paragraph*{DateToDateTime}\hspace*{\fill}

\begin{list}{}{
\settowidth{\tmplength}{\textbf{Declaração}}
\setlength{\itemindent}{0cm}
\setlength{\listparindent}{0cm}
\setlength{\leftmargin}{\evensidemargin}
\addtolength{\leftmargin}{\tmplength}
\settowidth{\labelsep}{X}
\addtolength{\leftmargin}{\labelsep}
\setlength{\labelwidth}{\tmplength}
}
\begin{flushleft}
\item[\textbf{Declaração}\hfill]
\begin{ttfamily}
public class function DateToDateTime(aDate:TypeData): System.TDateTime; Overload;\end{ttfamily}


\end{flushleft}
\end{list}
\paragraph*{DateToDateTime}\hspace*{\fill}

\begin{list}{}{
\settowidth{\tmplength}{\textbf{Declaração}}
\setlength{\itemindent}{0cm}
\setlength{\listparindent}{0cm}
\setlength{\leftmargin}{\evensidemargin}
\addtolength{\leftmargin}{\tmplength}
\settowidth{\labelsep}{X}
\addtolength{\leftmargin}{\labelsep}
\setlength{\labelwidth}{\tmplength}
}
\begin{flushleft}
\item[\textbf{Declaração}\hfill]
\begin{ttfamily}
public class function DateToDateTime(aTimePack:Longint):System.TDateTime; Overload;\end{ttfamily}


\end{flushleft}
\end{list}
\paragraph*{DateTimeToDate}\hspace*{\fill}

\begin{list}{}{
\settowidth{\tmplength}{\textbf{Declaração}}
\setlength{\itemindent}{0cm}
\setlength{\listparindent}{0cm}
\setlength{\leftmargin}{\evensidemargin}
\addtolength{\leftmargin}{\tmplength}
\settowidth{\labelsep}{X}
\addtolength{\leftmargin}{\labelsep}
\setlength{\labelwidth}{\tmplength}
}
\begin{flushleft}
\item[\textbf{Declaração}\hfill]
\begin{ttfamily}
public class function DateTimeToDate(aDateTime:TDateTime):TypeData; Overload;\end{ttfamily}


\end{flushleft}
\end{list}
\paragraph*{DateTimeToDateStr}\hspace*{\fill}

\begin{list}{}{
\settowidth{\tmplength}{\textbf{Declaração}}
\setlength{\itemindent}{0cm}
\setlength{\listparindent}{0cm}
\setlength{\leftmargin}{\evensidemargin}
\addtolength{\leftmargin}{\tmplength}
\settowidth{\labelsep}{X}
\addtolength{\leftmargin}{\labelsep}
\setlength{\labelwidth}{\tmplength}
}
\begin{flushleft}
\item[\textbf{Declaração}\hfill]
\begin{ttfamily}
public class function DateTimeToDateStr(aDateTime:TDateTime):String; Overload;\end{ttfamily}


\end{flushleft}
\end{list}
\paragraph*{DateTimeToTimeStr}\hspace*{\fill}

\begin{list}{}{
\settowidth{\tmplength}{\textbf{Declaração}}
\setlength{\itemindent}{0cm}
\setlength{\listparindent}{0cm}
\setlength{\leftmargin}{\evensidemargin}
\addtolength{\leftmargin}{\tmplength}
\settowidth{\labelsep}{X}
\addtolength{\leftmargin}{\labelsep}
\setlength{\labelwidth}{\tmplength}
}
\begin{flushleft}
\item[\textbf{Declaração}\hfill]
\begin{ttfamily}
public class function DateTimeToTimeStr(aDateTime:TDateTime):String;\end{ttfamily}


\end{flushleft}
\end{list}
\paragraph*{DateTimeToDateTimeDos}\hspace*{\fill}

\begin{list}{}{
\settowidth{\tmplength}{\textbf{Declaração}}
\setlength{\itemindent}{0cm}
\setlength{\listparindent}{0cm}
\setlength{\leftmargin}{\evensidemargin}
\addtolength{\leftmargin}{\tmplength}
\settowidth{\labelsep}{X}
\addtolength{\leftmargin}{\labelsep}
\setlength{\labelwidth}{\tmplength}
}
\begin{flushleft}
\item[\textbf{Declaração}\hfill]
\begin{ttfamily}
public class function DateTimeToDateTimeDos(aDateTime:TDateTime):Longint; Overload;\end{ttfamily}


\end{flushleft}
\end{list}
\paragraph*{DateTimeDosToStr}\hspace*{\fill}

\begin{list}{}{
\settowidth{\tmplength}{\textbf{Declaração}}
\setlength{\itemindent}{0cm}
\setlength{\listparindent}{0cm}
\setlength{\leftmargin}{\evensidemargin}
\addtolength{\leftmargin}{\tmplength}
\settowidth{\labelsep}{X}
\addtolength{\leftmargin}{\labelsep}
\setlength{\labelwidth}{\tmplength}
}
\begin{flushleft}
\item[\textbf{Declaração}\hfill]
\begin{ttfamily}
public class function DateTimeDosToStr(aTimePack:Longint;Mask:TDateMask):String;\end{ttfamily}


\end{flushleft}
\end{list}
\paragraph*{StrToDateTime}\hspace*{\fill}

\begin{list}{}{
\settowidth{\tmplength}{\textbf{Declaração}}
\setlength{\itemindent}{0cm}
\setlength{\listparindent}{0cm}
\setlength{\leftmargin}{\evensidemargin}
\addtolength{\leftmargin}{\tmplength}
\settowidth{\labelsep}{X}
\addtolength{\leftmargin}{\labelsep}
\setlength{\labelwidth}{\tmplength}
}
\begin{flushleft}
\item[\textbf{Declaração}\hfill]
\begin{ttfamily}
public class function StrToDateTime(aDataTime:String;Mask:TDateMask):TDateTime; Overload;\end{ttfamily}


\end{flushleft}
\end{list}
\paragraph*{StrToDateTimeDos}\hspace*{\fill}

\begin{list}{}{
\settowidth{\tmplength}{\textbf{Declaração}}
\setlength{\itemindent}{0cm}
\setlength{\listparindent}{0cm}
\setlength{\leftmargin}{\evensidemargin}
\addtolength{\leftmargin}{\tmplength}
\settowidth{\labelsep}{X}
\addtolength{\leftmargin}{\labelsep}
\setlength{\labelwidth}{\tmplength}
}
\begin{flushleft}
\item[\textbf{Declaração}\hfill]
\begin{ttfamily}
public class function StrToDateTimeDos(aDataTime:String;Mask:TDateMask):Longint; Overload;\end{ttfamily}


\end{flushleft}
\end{list}
\paragraph*{StrToHora}\hspace*{\fill}

\begin{list}{}{
\settowidth{\tmplength}{\textbf{Declaração}}
\setlength{\itemindent}{0cm}
\setlength{\listparindent}{0cm}
\setlength{\leftmargin}{\evensidemargin}
\addtolength{\leftmargin}{\tmplength}
\settowidth{\labelsep}{X}
\addtolength{\leftmargin}{\labelsep}
\setlength{\labelwidth}{\tmplength}
}
\begin{flushleft}
\item[\textbf{Declaração}\hfill]
\begin{ttfamily}
public class function StrToHora( aStrHora:String; Const Mask: THourMask):TipoHora;\end{ttfamily}


\end{flushleft}
\end{list}
\paragraph*{StrToHour}\hspace*{\fill}

\begin{list}{}{
\settowidth{\tmplength}{\textbf{Declaração}}
\setlength{\itemindent}{0cm}
\setlength{\listparindent}{0cm}
\setlength{\leftmargin}{\evensidemargin}
\addtolength{\leftmargin}{\tmplength}
\settowidth{\labelsep}{X}
\addtolength{\leftmargin}{\labelsep}
\setlength{\labelwidth}{\tmplength}
}
\begin{flushleft}
\item[\textbf{Declaração}\hfill]
\begin{ttfamily}
public class function StrToHour(Const aStrHora:String; Const Mask: THourMask):Longint; Overload;\end{ttfamily}


\end{flushleft}
\end{list}
\paragraph*{StrToHour}\hspace*{\fill}

\begin{list}{}{
\settowidth{\tmplength}{\textbf{Declaração}}
\setlength{\itemindent}{0cm}
\setlength{\listparindent}{0cm}
\setlength{\leftmargin}{\evensidemargin}
\addtolength{\leftmargin}{\tmplength}
\settowidth{\labelsep}{X}
\addtolength{\leftmargin}{\labelsep}
\setlength{\labelwidth}{\tmplength}
}
\begin{flushleft}
\item[\textbf{Declaração}\hfill]
\begin{ttfamily}
public class function StrToHour(Const aStrHora:String; Const Mask: THourMask;TimePack:Longint):Longint; Overload;\end{ttfamily}


\end{flushleft}
\end{list}
\paragraph*{HourToStr}\hspace*{\fill}

\begin{list}{}{
\settowidth{\tmplength}{\textbf{Declaração}}
\setlength{\itemindent}{0cm}
\setlength{\listparindent}{0cm}
\setlength{\leftmargin}{\evensidemargin}
\addtolength{\leftmargin}{\tmplength}
\settowidth{\labelsep}{X}
\addtolength{\leftmargin}{\labelsep}
\setlength{\labelwidth}{\tmplength}
}
\begin{flushleft}
\item[\textbf{Declaração}\hfill]
\begin{ttfamily}
public class function HourToStr(Const aStrHora:Longint; Const Mask: THourMask;Const OkSpc : Boolean):String; Overload;\end{ttfamily}


\end{flushleft}
\end{list}
\paragraph*{HourToStr}\hspace*{\fill}

\begin{list}{}{
\settowidth{\tmplength}{\textbf{Declaração}}
\setlength{\itemindent}{0cm}
\setlength{\listparindent}{0cm}
\setlength{\leftmargin}{\evensidemargin}
\addtolength{\leftmargin}{\tmplength}
\settowidth{\labelsep}{X}
\addtolength{\leftmargin}{\labelsep}
\setlength{\labelwidth}{\tmplength}
}
\begin{flushleft}
\item[\textbf{Declaração}\hfill]
\begin{ttfamily}
public class function HourToStr(Const Hora:TipoHora; Const Mask: THourMask;Const OkSpc : Boolean):String; Overload;\end{ttfamily}


\end{flushleft}
\end{list}
\paragraph*{HourToDateTime}\hspace*{\fill}

\begin{list}{}{
\settowidth{\tmplength}{\textbf{Declaração}}
\setlength{\itemindent}{0cm}
\setlength{\listparindent}{0cm}
\setlength{\leftmargin}{\evensidemargin}
\addtolength{\leftmargin}{\tmplength}
\settowidth{\labelsep}{X}
\addtolength{\leftmargin}{\labelsep}
\setlength{\labelwidth}{\tmplength}
}
\begin{flushleft}
\item[\textbf{Declaração}\hfill]
\begin{ttfamily}
public class function HourToDateTime(Const aTimePack:Longint): TDateTime; Overload;\end{ttfamily}


\end{flushleft}
\end{list}
\paragraph*{str2jul}\hspace*{\fill}

\begin{list}{}{
\settowidth{\tmplength}{\textbf{Declaração}}
\setlength{\itemindent}{0cm}
\setlength{\listparindent}{0cm}
\setlength{\leftmargin}{\evensidemargin}
\addtolength{\leftmargin}{\tmplength}
\settowidth{\labelsep}{X}
\addtolength{\leftmargin}{\labelsep}
\setlength{\labelwidth}{\tmplength}
}
\begin{flushleft}
\item[\textbf{Declaração}\hfill]
\begin{ttfamily}
public class function str2jul(DateStr:string): longint;\end{ttfamily}


\end{flushleft}
\end{list}
\paragraph*{jul2str}\hspace*{\fill}

\begin{list}{}{
\settowidth{\tmplength}{\textbf{Declaração}}
\setlength{\itemindent}{0cm}
\setlength{\listparindent}{0cm}
\setlength{\leftmargin}{\evensidemargin}
\addtolength{\leftmargin}{\tmplength}
\settowidth{\labelsep}{X}
\addtolength{\leftmargin}{\labelsep}
\setlength{\labelwidth}{\tmplength}
}
\begin{flushleft}
\item[\textbf{Declaração}\hfill]
\begin{ttfamily}
public class function jul2str(JulDate:longint) :string;\end{ttfamily}


\end{flushleft}
\end{list}
\paragraph*{Julian}\hspace*{\fill}

\begin{list}{}{
\settowidth{\tmplength}{\textbf{Declaração}}
\setlength{\itemindent}{0cm}
\setlength{\listparindent}{0cm}
\setlength{\leftmargin}{\evensidemargin}
\addtolength{\leftmargin}{\tmplength}
\settowidth{\labelsep}{X}
\addtolength{\leftmargin}{\labelsep}
\setlength{\labelwidth}{\tmplength}
}
\begin{flushleft}
\item[\textbf{Declaração}\hfill]
\begin{ttfamily}
public class function Julian( Year, Month, Day : Word ) : LongInt;\end{ttfamily}


\end{flushleft}
\end{list}
\paragraph*{LeapYear}\hspace*{\fill}

\begin{list}{}{
\settowidth{\tmplength}{\textbf{Declaração}}
\setlength{\itemindent}{0cm}
\setlength{\listparindent}{0cm}
\setlength{\leftmargin}{\evensidemargin}
\addtolength{\leftmargin}{\tmplength}
\settowidth{\labelsep}{X}
\addtolength{\leftmargin}{\labelsep}
\setlength{\labelwidth}{\tmplength}
}
\begin{flushleft}
\item[\textbf{Declaração}\hfill]
\begin{ttfamily}
public class function LeapYear( Year : Word ) : Boolean ;\end{ttfamily}


\end{flushleft}
\end{list}
\paragraph*{DiaMaxMes}\hspace*{\fill}

\begin{list}{}{
\settowidth{\tmplength}{\textbf{Declaração}}
\setlength{\itemindent}{0cm}
\setlength{\listparindent}{0cm}
\setlength{\leftmargin}{\evensidemargin}
\addtolength{\leftmargin}{\tmplength}
\settowidth{\labelsep}{X}
\addtolength{\leftmargin}{\labelsep}
\setlength{\labelwidth}{\tmplength}
}
\begin{flushleft}
\item[\textbf{Declaração}\hfill]
\begin{ttfamily}
public class function DiaMaxMes(Const DataAtual :TypeData) : byte;\end{ttfamily}


\end{flushleft}
\end{list}
\paragraph*{DateMask{\_}to{\_}Str}\hspace*{\fill}

\begin{list}{}{
\settowidth{\tmplength}{\textbf{Declaração}}
\setlength{\itemindent}{0cm}
\setlength{\listparindent}{0cm}
\setlength{\leftmargin}{\evensidemargin}
\addtolength{\leftmargin}{\tmplength}
\settowidth{\labelsep}{X}
\addtolength{\leftmargin}{\labelsep}
\setlength{\labelwidth}{\tmplength}
}
\begin{flushleft}
\item[\textbf{Declaração}\hfill]
\begin{ttfamily}
public class function DateMask{\_}to{\_}Str(Const aDateMask : TDateMask ):String;\end{ttfamily}


\end{flushleft}
\end{list}
\paragraph*{Str{\_}to{\_}DateMask}\hspace*{\fill}

\begin{list}{}{
\settowidth{\tmplength}{\textbf{Declaração}}
\setlength{\itemindent}{0cm}
\setlength{\listparindent}{0cm}
\setlength{\leftmargin}{\evensidemargin}
\addtolength{\leftmargin}{\tmplength}
\settowidth{\labelsep}{X}
\addtolength{\leftmargin}{\labelsep}
\setlength{\labelwidth}{\tmplength}
}
\begin{flushleft}
\item[\textbf{Declaração}\hfill]
\begin{ttfamily}
public class function Str{\_}to{\_}DateMask(aStrDate:String):TDateMask;\end{ttfamily}


\end{flushleft}
\end{list}
\paragraph*{HourMask{\_}to{\_}Str}\hspace*{\fill}

\begin{list}{}{
\settowidth{\tmplength}{\textbf{Declaração}}
\setlength{\itemindent}{0cm}
\setlength{\listparindent}{0cm}
\setlength{\leftmargin}{\evensidemargin}
\addtolength{\leftmargin}{\tmplength}
\settowidth{\labelsep}{X}
\addtolength{\leftmargin}{\labelsep}
\setlength{\labelwidth}{\tmplength}
}
\begin{flushleft}
\item[\textbf{Declaração}\hfill]
\begin{ttfamily}
public class function HourMask{\_}to{\_}Str(Const aHourMask : THourMask ):String;\end{ttfamily}


\end{flushleft}
\end{list}
\paragraph*{ValidDate}\hspace*{\fill}

\begin{list}{}{
\settowidth{\tmplength}{\textbf{Declaração}}
\setlength{\itemindent}{0cm}
\setlength{\listparindent}{0cm}
\setlength{\leftmargin}{\evensidemargin}
\addtolength{\leftmargin}{\tmplength}
\settowidth{\labelsep}{X}
\addtolength{\leftmargin}{\labelsep}
\setlength{\labelwidth}{\tmplength}
}
\begin{flushleft}
\item[\textbf{Declaração}\hfill]
\begin{ttfamily}
public class function ValidDate( aData : TypeData):Byte;\end{ttfamily}


\end{flushleft}
\end{list}
\paragraph*{ValidHour}\hspace*{\fill}

\begin{list}{}{
\settowidth{\tmplength}{\textbf{Declaração}}
\setlength{\itemindent}{0cm}
\setlength{\listparindent}{0cm}
\setlength{\leftmargin}{\evensidemargin}
\addtolength{\leftmargin}{\tmplength}
\settowidth{\labelsep}{X}
\addtolength{\leftmargin}{\labelsep}
\setlength{\labelwidth}{\tmplength}
}
\begin{flushleft}
\item[\textbf{Declaração}\hfill]
\begin{ttfamily}
public class function ValidHour( H,M,S,S100 : Word):Byte;\end{ttfamily}


\end{flushleft}
\end{list}
\paragraph*{DifHoraEmSegundos}\hspace*{\fill}

\begin{list}{}{
\settowidth{\tmplength}{\textbf{Declaração}}
\setlength{\itemindent}{0cm}
\setlength{\listparindent}{0cm}
\setlength{\leftmargin}{\evensidemargin}
\addtolength{\leftmargin}{\tmplength}
\settowidth{\labelsep}{X}
\addtolength{\leftmargin}{\labelsep}
\setlength{\labelwidth}{\tmplength}
}
\begin{flushleft}
\item[\textbf{Declaração}\hfill]
\begin{ttfamily}
public class function DifHoraEmSegundos(Const HAtu,HAnt : TipoHora ):Longint;\end{ttfamily}


\end{flushleft}
\end{list}
\paragraph*{DifHora{\_}Retorne{\_}TipoHora}\hspace*{\fill}

\begin{list}{}{
\settowidth{\tmplength}{\textbf{Declaração}}
\setlength{\itemindent}{0cm}
\setlength{\listparindent}{0cm}
\setlength{\leftmargin}{\evensidemargin}
\addtolength{\leftmargin}{\tmplength}
\settowidth{\labelsep}{X}
\addtolength{\leftmargin}{\labelsep}
\setlength{\labelwidth}{\tmplength}
}
\begin{flushleft}
\item[\textbf{Declaração}\hfill]
\begin{ttfamily}
public class function DifHora{\_}Retorne{\_}TipoHora(Const HAtu,HAnt : TipoHora ):TipoHora; Overload;\end{ttfamily}


\end{flushleft}
\end{list}
\paragraph*{DifHora{\_}Retorne{\_}TipoHora}\hspace*{\fill}

\begin{list}{}{
\settowidth{\tmplength}{\textbf{Declaração}}
\setlength{\itemindent}{0cm}
\setlength{\listparindent}{0cm}
\setlength{\leftmargin}{\evensidemargin}
\addtolength{\leftmargin}{\tmplength}
\settowidth{\labelsep}{X}
\addtolength{\leftmargin}{\labelsep}
\setlength{\labelwidth}{\tmplength}
}
\begin{flushleft}
\item[\textbf{Declaração}\hfill]
\begin{ttfamily}
public class function DifHora{\_}Retorne{\_}TipoHora(Const HAtu,HAnt : Longint ):TipoHora; Overload;\end{ttfamily}


\end{flushleft}
\end{list}
\paragraph*{DifHora{\_}Retorne{\_}Horas{\_}Fracao}\hspace*{\fill}

\begin{list}{}{
\settowidth{\tmplength}{\textbf{Declaração}}
\setlength{\itemindent}{0cm}
\setlength{\listparindent}{0cm}
\setlength{\leftmargin}{\evensidemargin}
\addtolength{\leftmargin}{\tmplength}
\settowidth{\labelsep}{X}
\addtolength{\leftmargin}{\labelsep}
\setlength{\labelwidth}{\tmplength}
}
\begin{flushleft}
\item[\textbf{Declaração}\hfill]
\begin{ttfamily}
public class function DifHora{\_}Retorne{\_}Horas{\_}Fracao(const HAtu,HAnt : TipoHora ):Double; Overload;\end{ttfamily}


\end{flushleft}
\end{list}
\paragraph*{DifHora{\_}Retorne{\_}Horas{\_}Fracao}\hspace*{\fill}

\begin{list}{}{
\settowidth{\tmplength}{\textbf{Declaração}}
\setlength{\itemindent}{0cm}
\setlength{\listparindent}{0cm}
\setlength{\leftmargin}{\evensidemargin}
\addtolength{\leftmargin}{\tmplength}
\settowidth{\labelsep}{X}
\addtolength{\leftmargin}{\labelsep}
\setlength{\labelwidth}{\tmplength}
}
\begin{flushleft}
\item[\textbf{Declaração}\hfill]
\begin{ttfamily}
public class function DifHora{\_}Retorne{\_}Horas{\_}Fracao(const HAtu,HAnt : Longint ):Double; Overload;\end{ttfamily}


\end{flushleft}
\end{list}
\paragraph*{DifHora{\_}Retorne{\_}Minutos}\hspace*{\fill}

\begin{list}{}{
\settowidth{\tmplength}{\textbf{Declaração}}
\setlength{\itemindent}{0cm}
\setlength{\listparindent}{0cm}
\setlength{\leftmargin}{\evensidemargin}
\addtolength{\leftmargin}{\tmplength}
\settowidth{\labelsep}{X}
\addtolength{\leftmargin}{\labelsep}
\setlength{\labelwidth}{\tmplength}
}
\begin{flushleft}
\item[\textbf{Declaração}\hfill]
\begin{ttfamily}
public class function DifHora{\_}Retorne{\_}Minutos(Const HAtu,HAnt : TipoHora ):Longint; Overload;\end{ttfamily}


\end{flushleft}
\end{list}
\paragraph*{DifHora{\_}Retorne{\_}Minutos}\hspace*{\fill}

\begin{list}{}{
\settowidth{\tmplength}{\textbf{Declaração}}
\setlength{\itemindent}{0cm}
\setlength{\listparindent}{0cm}
\setlength{\leftmargin}{\evensidemargin}
\addtolength{\leftmargin}{\tmplength}
\settowidth{\labelsep}{X}
\addtolength{\leftmargin}{\labelsep}
\setlength{\labelwidth}{\tmplength}
}
\begin{flushleft}
\item[\textbf{Declaração}\hfill]
\begin{ttfamily}
public class function DifHora{\_}Retorne{\_}Minutos(Const HAtu,HAnt : Longint ):Longint; Overload;\end{ttfamily}


\end{flushleft}
\end{list}
\paragraph*{DifHora{\_}Retorne{\_}Time}\hspace*{\fill}

\begin{list}{}{
\settowidth{\tmplength}{\textbf{Declaração}}
\setlength{\itemindent}{0cm}
\setlength{\listparindent}{0cm}
\setlength{\leftmargin}{\evensidemargin}
\addtolength{\leftmargin}{\tmplength}
\settowidth{\labelsep}{X}
\addtolength{\leftmargin}{\labelsep}
\setlength{\labelwidth}{\tmplength}
}
\begin{flushleft}
\item[\textbf{Declaração}\hfill]
\begin{ttfamily}
public class function DifHora{\_}Retorne{\_}Time(Const HAtu,HAnt : TipoHora ):Longint; Overload;\end{ttfamily}


\end{flushleft}
\end{list}
\paragraph*{DifHora{\_}Retorne{\_}Time}\hspace*{\fill}

\begin{list}{}{
\settowidth{\tmplength}{\textbf{Declaração}}
\setlength{\itemindent}{0cm}
\setlength{\listparindent}{0cm}
\setlength{\leftmargin}{\evensidemargin}
\addtolength{\leftmargin}{\tmplength}
\settowidth{\labelsep}{X}
\addtolength{\leftmargin}{\labelsep}
\setlength{\labelwidth}{\tmplength}
}
\begin{flushleft}
\item[\textbf{Declaração}\hfill]
\begin{ttfamily}
public class function DifHora{\_}Retorne{\_}Time(Const HAtu,HAnt : Longint ):Longint; Overload;\end{ttfamily}


\end{flushleft}
\end{list}
\paragraph*{SegundosEmHora}\hspace*{\fill}

\begin{list}{}{
\settowidth{\tmplength}{\textbf{Declaração}}
\setlength{\itemindent}{0cm}
\setlength{\listparindent}{0cm}
\setlength{\leftmargin}{\evensidemargin}
\addtolength{\leftmargin}{\tmplength}
\settowidth{\labelsep}{X}
\addtolength{\leftmargin}{\labelsep}
\setlength{\labelwidth}{\tmplength}
}
\begin{flushleft}
\item[\textbf{Declaração}\hfill]
\begin{ttfamily}
public class function SegundosEmHora(Const Segundos:Longint):String;\end{ttfamily}


\end{flushleft}
\end{list}
\paragraph*{New{\_}Lista{\_}Str{\_}Meses}\hspace*{\fill}

\begin{list}{}{
\settowidth{\tmplength}{\textbf{Declaração}}
\setlength{\itemindent}{0cm}
\setlength{\listparindent}{0cm}
\setlength{\leftmargin}{\evensidemargin}
\addtolength{\leftmargin}{\tmplength}
\settowidth{\labelsep}{X}
\addtolength{\leftmargin}{\labelsep}
\setlength{\labelwidth}{\tmplength}
}
\begin{flushleft}
\item[\textbf{Declaração}\hfill]
\begin{ttfamily}
public class function New{\_}Lista{\_}Str{\_}Meses: PSitem;\end{ttfamily}


\end{flushleft}
\end{list}
\paragraph*{getDateStr}\hspace*{\fill}

\begin{list}{}{
\settowidth{\tmplength}{\textbf{Declaração}}
\setlength{\itemindent}{0cm}
\setlength{\listparindent}{0cm}
\setlength{\leftmargin}{\evensidemargin}
\addtolength{\leftmargin}{\tmplength}
\settowidth{\labelsep}{X}
\addtolength{\leftmargin}{\labelsep}
\setlength{\labelwidth}{\tmplength}
}
\begin{flushleft}
\item[\textbf{Declaração}\hfill]
\begin{ttfamily}
public class function getDateStr:tstring ;\end{ttfamily}


\end{flushleft}
\end{list}
\paragraph*{getTimeStr}\hspace*{\fill}

\begin{list}{}{
\settowidth{\tmplength}{\textbf{Declaração}}
\setlength{\itemindent}{0cm}
\setlength{\listparindent}{0cm}
\setlength{\leftmargin}{\evensidemargin}
\addtolength{\leftmargin}{\tmplength}
\settowidth{\labelsep}{X}
\addtolength{\leftmargin}{\labelsep}
\setlength{\labelwidth}{\tmplength}
}
\begin{flushleft}
\item[\textbf{Declaração}\hfill]
\begin{ttfamily}
public class function getTimeStr:tstring ;\end{ttfamily}


\end{flushleft}
\end{list}
\chapter{Unit mi.rtl.Objects.Methods.Db.Tb{\_}Access}
\section{Descrição}
\textbf{}\textbf{}\textbf{}\textbf{}\textbf{}\textbf{}\textbf{}\textbf{}\textbf{}\textbf{}\textbf{}\textbf{}\textbf{}\textbf{}\textbf{}
\section{Uses}
\begin{itemize}
\item \begin{ttfamily}classes\end{ttfamily}\item \begin{ttfamily}SysUtils\end{ttfamily}\item \begin{ttfamily}Dos\end{ttfamily}\item \begin{ttfamily}crt\end{ttfamily}\item \begin{ttfamily}strings\end{ttfamily}\item \begin{ttfamily}Memory\end{ttfamily}\item \begin{ttfamily}mi.rtl.types\end{ttfamily}(\ref{mi.rtl.Types})\item \begin{ttfamily}mi.rtl.objects.consts.MI{\_}MsgBox\end{ttfamily}\item \begin{ttfamily}mi.rtl.Consts.StrError\end{ttfamily}(\ref{mi.rtl.Consts.StrError})\item \begin{ttfamily}mi.rtl.files\end{ttfamily}(\ref{mi.rtl.files})\item \begin{ttfamily}mi.rtl.objects.Methods.Exception\end{ttfamily}(\ref{mi.rtl.Objects.Methods.Exception})\item \begin{ttfamily}mi.rtl.objects.methods.StreamBase.Stream\end{ttfamily}(\ref{mi.rtl.Objects.Methods.StreamBase.Stream})\item \begin{ttfamily}mi.rtl.objects.methods.StreamBase.Stream.MemoryStream.BufferMemory\end{ttfamily}(\ref{mi.rtl.objects.methods.StreamBase.Stream.MemoryStream.BufferMemory})\item \begin{ttfamily}mi.rtl.objects.methods.Collection.FilesStreams\end{ttfamily}(\ref{mi.rtl.Objects.Methods.Collection.FilesStreams})\item \begin{ttfamily}mi.rtl.objects.methods.Collection\end{ttfamily}(\ref{mi.rtl.Objects.Methods.Collection})\item \begin{ttfamily}mi.rtl.objects.methods.Collection.SortedCollection\end{ttfamily}(\ref{mi.rtl.Objects.Methods.Collection.SortedCollection})\item \begin{ttfamily}mi.rtl.objects.methods.Collection.SortedCollection.StringCollection.CollectionString\end{ttfamily}(\ref{mi.rtl.Objects.Methods.Collection.Sortedcollection.Stringcollection.Collectionstring})\item \begin{ttfamily}mi.rtl.objects.methods.StreamBase.Stream.FileStream\end{ttfamily}(\ref{mi.rtl.Objects.Methods.StreamBase.Stream.FileStream})\item \begin{ttfamily}mi.rtl.objects.Methods.dates\end{ttfamily}(\ref{mi.rtl.objects.Methods.dates})\item \begin{ttfamily}mi.rtl.objects.Methods.System\end{ttfamily}(\ref{mi.rtl.Objects.Methods.System})\end{itemize}
\section{Visão Geral}
\begin{description}
\item[\texttt{\begin{ttfamily}TTb{\_}Access{\_}types\end{ttfamily} Classe}]
\item[\texttt{\begin{ttfamily}TTb{\_}Access{\_}consts\end{ttfamily} Classe}]
\item[\texttt{\begin{ttfamily}TDataFile\end{ttfamily} Classe}]
\item[\texttt{\begin{ttfamily}TTb{\_}Access\end{ttfamily} Classe}]
\item[\texttt{\begin{ttfamily}TFilesOpens\end{ttfamily} Classe}]
\end{description}
\section{Classes, Interfaces, Objetos e Registros}
\subsection*{TTb{\_}Access{\_}types Classe}
\subsubsection*{\large{\textbf{Hierarquia}}\normalsize\hspace{1ex}\hfill}
TTb{\_}Access{\_}types {$>$} \begin{ttfamily}TObjectsSystem\end{ttfamily}(\ref{mi.rtl.Objects.Methods.System.TObjectsSystem}) {$>$} \begin{ttfamily}TObjectsMethods\end{ttfamily}(\ref{mi.rtl.Objects.Methods.TObjectsMethods}) {$>$} \begin{ttfamily}TObjectsConsts\end{ttfamily}(\ref{mi.rtl.Objects.Consts.TObjectsConsts}) {$>$} 
TObjectsTypes
\subsubsection*{\large{\textbf{Descrição}}\normalsize\hspace{1ex}\hfill}
A classe \textbf{\begin{ttfamily}TTb{\_}Access{\_}types\end{ttfamily}} é usada para declarar todos os types da classe \textbf{\begin{ttfamily}TTb{\_}Access\end{ttfamily}(\ref{mi.rtl.Objects.Methods.Db.Tb_Access.TTb_Access})}\subsubsection*{\large{\textbf{Campos}}\normalsize\hspace{1ex}\hfill}
\paragraph*{maxKeyLen}\hspace*{\fill}

\begin{list}{}{
\settowidth{\tmplength}{\textbf{Declaração}}
\setlength{\itemindent}{0cm}
\setlength{\listparindent}{0cm}
\setlength{\leftmargin}{\evensidemargin}
\addtolength{\leftmargin}{\tmplength}
\settowidth{\labelsep}{X}
\addtolength{\leftmargin}{\labelsep}
\setlength{\labelwidth}{\tmplength}
}
\begin{flushleft}
\item[\textbf{Declaração}\hfill]
\begin{ttfamily}
public const maxKeyLen       =  254  ;\end{ttfamily}


\end{flushleft}
\par
\item[\textbf{Descrição}]
Tamanho máximo da Chave

\end{list}
\paragraph*{pageSize}\hspace*{\fill}

\begin{list}{}{
\settowidth{\tmplength}{\textbf{Declaração}}
\setlength{\itemindent}{0cm}
\setlength{\listparindent}{0cm}
\setlength{\leftmargin}{\evensidemargin}
\addtolength{\leftmargin}{\tmplength}
\settowidth{\labelsep}{X}
\addtolength{\leftmargin}{\labelsep}
\setlength{\labelwidth}{\tmplength}
}
\begin{flushleft}
\item[\textbf{Declaração}\hfill]
\begin{ttfamily}
public const pageSize        =  510  ;\end{ttfamily}


\end{flushleft}
\par
\item[\textbf{Descrição}]
Número máximo de chaves permitido em uma página

\end{list}
\paragraph*{order}\hspace*{\fill}

\begin{list}{}{
\settowidth{\tmplength}{\textbf{Declaração}}
\setlength{\itemindent}{0cm}
\setlength{\listparindent}{0cm}
\setlength{\leftmargin}{\evensidemargin}
\addtolength{\leftmargin}{\tmplength}
\settowidth{\labelsep}{X}
\addtolength{\leftmargin}{\labelsep}
\setlength{\labelwidth}{\tmplength}
}
\begin{flushleft}
\item[\textbf{Declaração}\hfill]
\begin{ttfamily}
public const order           =  255  ;\end{ttfamily}


\end{flushleft}
\par
\item[\textbf{Descrição}]
Número mínimo de chaves permitido em uma página

\end{list}
\paragraph*{maxHeight}\hspace*{\fill}

\begin{list}{}{
\settowidth{\tmplength}{\textbf{Declaração}}
\setlength{\itemindent}{0cm}
\setlength{\listparindent}{0cm}
\setlength{\leftmargin}{\evensidemargin}
\addtolength{\leftmargin}{\tmplength}
\settowidth{\labelsep}{X}
\addtolength{\leftmargin}{\labelsep}
\setlength{\labelwidth}{\tmplength}
}
\begin{flushleft}
\item[\textbf{Declaração}\hfill]
\begin{ttfamily}
public const maxHeight       =  4    ;\end{ttfamily}


\end{flushleft}
\par
\item[\textbf{Descrição}]
Número máximo de níveis na árvore B+

\end{list}
\paragraph*{maxDataRecSize}\hspace*{\fill}

\begin{list}{}{
\settowidth{\tmplength}{\textbf{Declaração}}
\setlength{\itemindent}{0cm}
\setlength{\listparindent}{0cm}
\setlength{\leftmargin}{\evensidemargin}
\addtolength{\leftmargin}{\tmplength}
\settowidth{\labelsep}{X}
\addtolength{\leftmargin}{\labelsep}
\setlength{\labelwidth}{\tmplength}
}
\begin{flushleft}
\item[\textbf{Declaração}\hfill]
\begin{ttfamily}
public const maxDataRecSize  = High (SmallWord)-1  ;\end{ttfamily}


\end{flushleft}
\end{list}
\subsection*{TTb{\_}Access{\_}consts Classe}
\subsubsection*{\large{\textbf{Hierarquia}}\normalsize\hspace{1ex}\hfill}
TTb{\_}Access{\_}consts {$>$} \begin{ttfamily}TTb{\_}Access{\_}types\end{ttfamily}(\ref{mi.rtl.Objects.Methods.Db.Tb_Access.TTb_Access_types}) {$>$} \begin{ttfamily}TObjectsSystem\end{ttfamily}(\ref{mi.rtl.Objects.Methods.System.TObjectsSystem}) {$>$} \begin{ttfamily}TObjectsMethods\end{ttfamily}(\ref{mi.rtl.Objects.Methods.TObjectsMethods}) {$>$} \begin{ttfamily}TObjectsConsts\end{ttfamily}(\ref{mi.rtl.Objects.Consts.TObjectsConsts}) {$>$} 
TObjectsTypes
\subsubsection*{\large{\textbf{Descrição}}\normalsize\hspace{1ex}\hfill}
A classe \textbf{\begin{ttfamily}TTb{\_}Access{\_}consts\end{ttfamily}} é usada para declarar todas as constantes da classe \textbf{\begin{ttfamily}TTb{\_}Access\end{ttfamily}(\ref{mi.rtl.Objects.Methods.Db.Tb_Access.TTb_Access})}\subsubsection*{\large{\textbf{Campos}}\normalsize\hspace{1ex}\hfill}
\paragraph*{ErroDOS}\hspace*{\fill}

\begin{list}{}{
\settowidth{\tmplength}{\textbf{Declaração}}
\setlength{\itemindent}{0cm}
\setlength{\listparindent}{0cm}
\setlength{\leftmargin}{\evensidemargin}
\addtolength{\leftmargin}{\tmplength}
\settowidth{\labelsep}{X}
\addtolength{\leftmargin}{\labelsep}
\setlength{\labelwidth}{\tmplength}
}
\begin{flushleft}
\item[\textbf{Declaração}\hfill]
\begin{ttfamily}
public const ErroDOS : TErroDOS = (Status: 0);\end{ttfamily}


\end{flushleft}
\end{list}
\paragraph*{FileName{\_}Transaction}\hspace*{\fill}

\begin{list}{}{
\settowidth{\tmplength}{\textbf{Declaração}}
\setlength{\itemindent}{0cm}
\setlength{\listparindent}{0cm}
\setlength{\leftmargin}{\evensidemargin}
\addtolength{\leftmargin}{\tmplength}
\settowidth{\labelsep}{X}
\addtolength{\leftmargin}{\labelsep}
\setlength{\labelwidth}{\tmplength}
}
\begin{flushleft}
\item[\textbf{Declaração}\hfill]
\begin{ttfamily}
public const FileName{\_}Transaction : PathStr = '';\end{ttfamily}


\end{flushleft}
\end{list}
\paragraph*{ok{\_}Debug{\_}Transaction}\hspace*{\fill}

\begin{list}{}{
\settowidth{\tmplength}{\textbf{Declaração}}
\setlength{\itemindent}{0cm}
\setlength{\listparindent}{0cm}
\setlength{\leftmargin}{\evensidemargin}
\addtolength{\leftmargin}{\tmplength}
\settowidth{\labelsep}{X}
\addtolength{\leftmargin}{\labelsep}
\setlength{\labelwidth}{\tmplength}
}
\begin{flushleft}
\item[\textbf{Declaração}\hfill]
\begin{ttfamily}
public const ok{\_}Debug{\_}Transaction : Boolean = False ;\end{ttfamily}


\end{flushleft}
\end{list}
\paragraph*{OkTransaction}\hspace*{\fill}

\begin{list}{}{
\settowidth{\tmplength}{\textbf{Declaração}}
\setlength{\itemindent}{0cm}
\setlength{\listparindent}{0cm}
\setlength{\leftmargin}{\evensidemargin}
\addtolength{\leftmargin}{\tmplength}
\settowidth{\labelsep}{X}
\addtolength{\leftmargin}{\labelsep}
\setlength{\labelwidth}{\tmplength}
}
\begin{flushleft}
\item[\textbf{Declaração}\hfill]
\begin{ttfamily}
public const OkTransaction        : Boolean = True;\end{ttfamily}


\end{flushleft}
\par
\item[\textbf{Descrição}]
False = desabilita transação

\end{list}
\paragraph*{OkAddRecFirstFree}\hspace*{\fill}

\begin{list}{}{
\settowidth{\tmplength}{\textbf{Declaração}}
\setlength{\itemindent}{0cm}
\setlength{\listparindent}{0cm}
\setlength{\leftmargin}{\evensidemargin}
\addtolength{\leftmargin}{\tmplength}
\settowidth{\labelsep}{X}
\addtolength{\leftmargin}{\labelsep}
\setlength{\labelwidth}{\tmplength}
}
\begin{flushleft}
\item[\textbf{Declaração}\hfill]
\begin{ttfamily}
public const OkAddRecFirstFree   : Boolean = True;\end{ttfamily}


\end{flushleft}
\par
\item[\textbf{Descrição}]
A Constante \textbf{\begin{ttfamily}OkAddRecFirstFree\end{ttfamily}} é usado para habilitar aproveitamento do espaço deletado.

\begin{itemize}
\item \textbf{NOTA} \begin{itemize}
\item True = O procedimento AddRec aproveita o espaço dos registros deletados.
\item False = O procedimento AddRec Não aproveita o espaço dos registros deletados. ou melhor o novo registro e adicionado no final do arquivo.
\end{itemize}
\end{itemize}

\end{list}
\paragraph*{MaxFilesOpens}\hspace*{\fill}

\begin{list}{}{
\settowidth{\tmplength}{\textbf{Declaração}}
\setlength{\itemindent}{0cm}
\setlength{\listparindent}{0cm}
\setlength{\leftmargin}{\evensidemargin}
\addtolength{\leftmargin}{\tmplength}
\settowidth{\labelsep}{X}
\addtolength{\leftmargin}{\labelsep}
\setlength{\labelwidth}{\tmplength}
}
\begin{flushleft}
\item[\textbf{Declaração}\hfill]
\begin{ttfamily}
public const MaxFilesOpens : Byte = 20;\end{ttfamily}


\end{flushleft}
\par
\item[\textbf{Descrição}]
Máximo de arquivo que o DOS pode abrir sem a chamada da Interrupção No. {\$}67

\end{list}
\paragraph*{MemoriaLivreEmTaPageStack}\hspace*{\fill}

\begin{list}{}{
\settowidth{\tmplength}{\textbf{Declaração}}
\setlength{\itemindent}{0cm}
\setlength{\listparindent}{0cm}
\setlength{\leftmargin}{\evensidemargin}
\addtolength{\leftmargin}{\tmplength}
\settowidth{\labelsep}{X}
\addtolength{\leftmargin}{\labelsep}
\setlength{\labelwidth}{\tmplength}
}
\begin{flushleft}
\item[\textbf{Declaração}\hfill]
\begin{ttfamily}
public const MemoriaLivreEmTaPageStack = 64 *1024;\end{ttfamily}


\end{flushleft}
\par
\item[\textbf{Descrição}]
16K de Memória fica livre em setBufIndex

\end{list}
\paragraph*{MsgOkDuplicidade}\hspace*{\fill}

\begin{list}{}{
\settowidth{\tmplength}{\textbf{Declaração}}
\setlength{\itemindent}{0cm}
\setlength{\listparindent}{0cm}
\setlength{\leftmargin}{\evensidemargin}
\addtolength{\leftmargin}{\tmplength}
\settowidth{\labelsep}{X}
\addtolength{\leftmargin}{\labelsep}
\setlength{\labelwidth}{\tmplength}
}
\begin{flushleft}
\item[\textbf{Declaração}\hfill]
\begin{ttfamily}
public const MsgOkDuplicidade : Boolean = True;\end{ttfamily}


\end{flushleft}
\par
\item[\textbf{Descrição}]
Indica se deve dar mensagem de chave em duplicidade

\end{list}
\paragraph*{Neterr}\hspace*{\fill}

\begin{list}{}{
\settowidth{\tmplength}{\textbf{Declaração}}
\setlength{\itemindent}{0cm}
\setlength{\listparindent}{0cm}
\setlength{\leftmargin}{\evensidemargin}
\addtolength{\leftmargin}{\tmplength}
\settowidth{\labelsep}{X}
\addtolength{\leftmargin}{\labelsep}
\setlength{\labelwidth}{\tmplength}
}
\begin{flushleft}
\item[\textbf{Declaração}\hfill]
\begin{ttfamily}
public const Neterr           : SmallInt =  0 ;\end{ttfamily}


\end{flushleft}
\par
\item[\textbf{Descrição}]
Indica se houve erro na rede

\end{list}
\paragraph*{MaxFiles}\hspace*{\fill}

\begin{list}{}{
\settowidth{\tmplength}{\textbf{Declaração}}
\setlength{\itemindent}{0cm}
\setlength{\listparindent}{0cm}
\setlength{\leftmargin}{\evensidemargin}
\addtolength{\leftmargin}{\tmplength}
\settowidth{\labelsep}{X}
\addtolength{\leftmargin}{\labelsep}
\setlength{\labelwidth}{\tmplength}
}
\begin{flushleft}
\item[\textbf{Declaração}\hfill]
\begin{ttfamily}
public const MaxFiles         : Byte = 254;\end{ttfamily}


\end{flushleft}
\end{list}
\paragraph*{OkTestaAberturaDeArquivo}\hspace*{\fill}

\begin{list}{}{
\settowidth{\tmplength}{\textbf{Declaração}}
\setlength{\itemindent}{0cm}
\setlength{\listparindent}{0cm}
\setlength{\leftmargin}{\evensidemargin}
\addtolength{\leftmargin}{\tmplength}
\settowidth{\labelsep}{X}
\addtolength{\leftmargin}{\labelsep}
\setlength{\labelwidth}{\tmplength}
}
\begin{flushleft}
\item[\textbf{Declaração}\hfill]
\begin{ttfamily}
public const OkTestaAberturaDeArquivo : Boolean = true;\end{ttfamily}


\end{flushleft}
\end{list}
\paragraph*{MaxPageEmMemoria}\hspace*{\fill}

\begin{list}{}{
\settowidth{\tmplength}{\textbf{Declaração}}
\setlength{\itemindent}{0cm}
\setlength{\listparindent}{0cm}
\setlength{\leftmargin}{\evensidemargin}
\addtolength{\leftmargin}{\tmplength}
\settowidth{\labelsep}{X}
\addtolength{\leftmargin}{\labelsep}
\setlength{\labelwidth}{\tmplength}
}
\begin{flushleft}
\item[\textbf{Declaração}\hfill]
\begin{ttfamily}
public const MaxPageEmMemoria = 20;\end{ttfamily}


\end{flushleft}
\par
\item[\textbf{Descrição}]
10 ficou mais lento do que 20; 100 ficou mais lento do que 20

\end{list}
\paragraph*{MinPageEmMemoria}\hspace*{\fill}

\begin{list}{}{
\settowidth{\tmplength}{\textbf{Declaração}}
\setlength{\itemindent}{0cm}
\setlength{\listparindent}{0cm}
\setlength{\leftmargin}{\evensidemargin}
\addtolength{\leftmargin}{\tmplength}
\settowidth{\labelsep}{X}
\addtolength{\leftmargin}{\labelsep}
\setlength{\labelwidth}{\tmplength}
}
\begin{flushleft}
\item[\textbf{Declaração}\hfill]
\begin{ttfamily}
public const MinPageEmMemoria =   3;\end{ttfamily}


\end{flushleft}
\par
\item[\textbf{Descrição}]
Usado no Buffer dos índice

\end{list}
\paragraph*{NoDuplicates}\hspace*{\fill}

\begin{list}{}{
\settowidth{\tmplength}{\textbf{Declaração}}
\setlength{\itemindent}{0cm}
\setlength{\listparindent}{0cm}
\setlength{\leftmargin}{\evensidemargin}
\addtolength{\leftmargin}{\tmplength}
\settowidth{\labelsep}{X}
\addtolength{\leftmargin}{\labelsep}
\setlength{\labelwidth}{\tmplength}
}
\begin{flushleft}
\item[\textbf{Declaração}\hfill]
\begin{ttfamily}
public const NoDuplicates = 0;\end{ttfamily}


\end{flushleft}
\end{list}
\paragraph*{Duplicates}\hspace*{\fill}

\begin{list}{}{
\settowidth{\tmplength}{\textbf{Declaração}}
\setlength{\itemindent}{0cm}
\setlength{\listparindent}{0cm}
\setlength{\leftmargin}{\evensidemargin}
\addtolength{\leftmargin}{\tmplength}
\settowidth{\labelsep}{X}
\addtolength{\leftmargin}{\labelsep}
\setlength{\labelwidth}{\tmplength}
}
\begin{flushleft}
\item[\textbf{Declaração}\hfill]
\begin{ttfamily}
public const Duplicates = 1;\end{ttfamily}


\end{flushleft}
\end{list}
\paragraph*{FileHeaderSize}\hspace*{\fill}

\begin{list}{}{
\settowidth{\tmplength}{\textbf{Declaração}}
\setlength{\itemindent}{0cm}
\setlength{\listparindent}{0cm}
\setlength{\leftmargin}{\evensidemargin}
\addtolength{\leftmargin}{\tmplength}
\settowidth{\labelsep}{X}
\addtolength{\leftmargin}{\labelsep}
\setlength{\labelwidth}{\tmplength}
}
\begin{flushleft}
\item[\textbf{Declaração}\hfill]
\begin{ttfamily}
public const FileHeaderSize   = sizeof(TsImagemHeader);\end{ttfamily}


\end{flushleft}
\par
\item[\textbf{Descrição}]
A constante \textbf{{\#}name} é usado para guardar o número de chaves do Indice

\end{list}
\paragraph*{MinDataRecSize}\hspace*{\fill}

\begin{list}{}{
\settowidth{\tmplength}{\textbf{Declaração}}
\setlength{\itemindent}{0cm}
\setlength{\listparindent}{0cm}
\setlength{\leftmargin}{\evensidemargin}
\addtolength{\leftmargin}{\tmplength}
\settowidth{\labelsep}{X}
\addtolength{\leftmargin}{\labelsep}
\setlength{\labelwidth}{\tmplength}
}
\begin{flushleft}
\item[\textbf{Declaração}\hfill]
\begin{ttfamily}
public const MinDataRecSize   = FileHeaderSize;\end{ttfamily}


\end{flushleft}
\end{list}
\paragraph*{ItemOverhead}\hspace*{\fill}

\begin{list}{}{
\settowidth{\tmplength}{\textbf{Declaração}}
\setlength{\itemindent}{0cm}
\setlength{\listparindent}{0cm}
\setlength{\leftmargin}{\evensidemargin}
\addtolength{\leftmargin}{\tmplength}
\settowidth{\labelsep}{X}
\addtolength{\leftmargin}{\labelsep}
\setlength{\labelwidth}{\tmplength}
}
\begin{flushleft}
\item[\textbf{Declaração}\hfill]
\begin{ttfamily}
public const ItemOverhead =   SizeOf(TaItem) - Sizeof(TaKeyStr  ) + 1;\end{ttfamily}


\end{flushleft}
\end{list}
\paragraph*{PageOverhead}\hspace*{\fill}

\begin{list}{}{
\settowidth{\tmplength}{\textbf{Declaração}}
\setlength{\itemindent}{0cm}
\setlength{\listparindent}{0cm}
\setlength{\leftmargin}{\evensidemargin}
\addtolength{\leftmargin}{\tmplength}
\settowidth{\labelsep}{X}
\addtolength{\leftmargin}{\labelsep}
\setlength{\labelwidth}{\tmplength}
}
\begin{flushleft}
\item[\textbf{Declaração}\hfill]
\begin{ttfamily}
public const PageOverhead =   SizeOf(TaPage) - SizeOf(TItemArray );\end{ttfamily}


\end{flushleft}
\end{list}
\paragraph*{TaRecBuf}\hspace*{\fill}

\begin{list}{}{
\settowidth{\tmplength}{\textbf{Declaração}}
\setlength{\itemindent}{0cm}
\setlength{\listparindent}{0cm}
\setlength{\leftmargin}{\evensidemargin}
\addtolength{\leftmargin}{\tmplength}
\settowidth{\labelsep}{X}
\addtolength{\leftmargin}{\labelsep}
\setlength{\labelwidth}{\tmplength}
}
\begin{flushleft}
\item[\textbf{Declaração}\hfill]
\begin{ttfamily}
public const TaRecBuf  : TaRecordBufPtr     = nil;\end{ttfamily}


\end{flushleft}
\end{list}
\paragraph*{Const{\_}Ext{\_}Tabela}\hspace*{\fill}

\begin{list}{}{
\settowidth{\tmplength}{\textbf{Declaração}}
\setlength{\itemindent}{0cm}
\setlength{\listparindent}{0cm}
\setlength{\leftmargin}{\evensidemargin}
\addtolength{\leftmargin}{\tmplength}
\settowidth{\labelsep}{X}
\addtolength{\leftmargin}{\labelsep}
\setlength{\labelwidth}{\tmplength}
}
\begin{flushleft}
\item[\textbf{Declaração}\hfill]
\begin{ttfamily}
public const Const{\_}Ext{\_}Tabela           = '.Tb';\end{ttfamily}


\end{flushleft}
\end{list}
\paragraph*{Const{\_}Ext{\_}Indice{\_}da{\_}tebela}\hspace*{\fill}

\begin{list}{}{
\settowidth{\tmplength}{\textbf{Declaração}}
\setlength{\itemindent}{0cm}
\setlength{\listparindent}{0cm}
\setlength{\leftmargin}{\evensidemargin}
\addtolength{\leftmargin}{\tmplength}
\settowidth{\labelsep}{X}
\addtolength{\leftmargin}{\labelsep}
\setlength{\labelwidth}{\tmplength}
}
\begin{flushleft}
\item[\textbf{Declaração}\hfill]
\begin{ttfamily}
public const Const{\_}Ext{\_}Indice{\_}da{\_}tebela = '.Ix';\end{ttfamily}


\end{flushleft}
\end{list}
\paragraph*{Const{\_}Ext{\_}Tabela{\_}com{\_}a{\_}copia{\_}da{\_}versao{\_}anterior{\_}da{\_}tabela}\hspace*{\fill}

\begin{list}{}{
\settowidth{\tmplength}{\textbf{Declaração}}
\setlength{\itemindent}{0cm}
\setlength{\listparindent}{0cm}
\setlength{\leftmargin}{\evensidemargin}
\addtolength{\leftmargin}{\tmplength}
\settowidth{\labelsep}{X}
\addtolength{\leftmargin}{\labelsep}
\setlength{\labelwidth}{\tmplength}
}
\begin{flushleft}
\item[\textbf{Declaração}\hfill]
\begin{ttfamily}
public const Const{\_}Ext{\_}Tabela{\_}com{\_}a{\_}copia{\_}da{\_}versao{\_}anterior{\_}da{\_}tabela = '.Tb{\_}';\end{ttfamily}


\end{flushleft}
\end{list}
\paragraph*{Const{\_}Ext{\_}Tabela{\_}de{\_}objetos{\_}vinculados{\_}a{\_}tabela}\hspace*{\fill}

\begin{list}{}{
\settowidth{\tmplength}{\textbf{Declaração}}
\setlength{\itemindent}{0cm}
\setlength{\listparindent}{0cm}
\setlength{\leftmargin}{\evensidemargin}
\addtolength{\leftmargin}{\tmplength}
\settowidth{\labelsep}{X}
\addtolength{\leftmargin}{\labelsep}
\setlength{\labelwidth}{\tmplength}
}
\begin{flushleft}
\item[\textbf{Declaração}\hfill]
\begin{ttfamily}
public const Const{\_}Ext{\_}Tabela{\_}de{\_}objetos{\_}vinculados{\_}a{\_}tabela = '.TbO';\end{ttfamily}


\end{flushleft}
\end{list}
\paragraph*{Const{\_}Ext{\_}Tabela{\_}com{\_}os{\_}registro{\_}duplicados}\hspace*{\fill}

\begin{list}{}{
\settowidth{\tmplength}{\textbf{Declaração}}
\setlength{\itemindent}{0cm}
\setlength{\listparindent}{0cm}
\setlength{\leftmargin}{\evensidemargin}
\addtolength{\leftmargin}{\tmplength}
\settowidth{\labelsep}{X}
\addtolength{\leftmargin}{\labelsep}
\setlength{\labelwidth}{\tmplength}
}
\begin{flushleft}
\item[\textbf{Declaração}\hfill]
\begin{ttfamily}
public const Const{\_}Ext{\_}Tabela{\_}com{\_}os{\_}registro{\_}duplicados     = '.Tb1';\end{ttfamily}


\end{flushleft}
\end{list}
\paragraph*{Const{\_}Ext{\_}Tebela{\_}com{\_}as{\_}Tabelas}\hspace*{\fill}

\begin{list}{}{
\settowidth{\tmplength}{\textbf{Declaração}}
\setlength{\itemindent}{0cm}
\setlength{\listparindent}{0cm}
\setlength{\leftmargin}{\evensidemargin}
\addtolength{\leftmargin}{\tmplength}
\settowidth{\labelsep}{X}
\addtolength{\leftmargin}{\labelsep}
\setlength{\labelwidth}{\tmplength}
}
\begin{flushleft}
\item[\textbf{Declaração}\hfill]
\begin{ttfamily}
public const Const{\_}Ext{\_}Tebela{\_}com{\_}as{\_}Tabelas          = '.TbT';\end{ttfamily}


\end{flushleft}
\end{list}
\paragraph*{Const{\_}Ext{\_}Indice{\_}da{\_}Tabela{\_}das{\_}tabelas}\hspace*{\fill}

\begin{list}{}{
\settowidth{\tmplength}{\textbf{Declaração}}
\setlength{\itemindent}{0cm}
\setlength{\listparindent}{0cm}
\setlength{\leftmargin}{\evensidemargin}
\addtolength{\leftmargin}{\tmplength}
\settowidth{\labelsep}{X}
\addtolength{\leftmargin}{\labelsep}
\setlength{\labelwidth}{\tmplength}
}
\begin{flushleft}
\item[\textbf{Declaração}\hfill]
\begin{ttfamily}
public const Const{\_}Ext{\_}Indice{\_}da{\_}Tabela{\_}das{\_}tabelas   = '.IxT';\end{ttfamily}


\end{flushleft}
\end{list}
\paragraph*{Const{\_}Ext{\_}Tabela{\_}com{\_}os{\_}Indices}\hspace*{\fill}

\begin{list}{}{
\settowidth{\tmplength}{\textbf{Declaração}}
\setlength{\itemindent}{0cm}
\setlength{\listparindent}{0cm}
\setlength{\leftmargin}{\evensidemargin}
\addtolength{\leftmargin}{\tmplength}
\settowidth{\labelsep}{X}
\addtolength{\leftmargin}{\labelsep}
\setlength{\labelwidth}{\tmplength}
}
\begin{flushleft}
\item[\textbf{Declaração}\hfill]
\begin{ttfamily}
public const Const{\_}Ext{\_}Tabela{\_}com{\_}os{\_}Indices       = '.TbI';\end{ttfamily}


\end{flushleft}
\end{list}
\paragraph*{Const{\_}Ext{\_}Indice{\_}da{\_}tebala{\_}de{\_}Indices}\hspace*{\fill}

\begin{list}{}{
\settowidth{\tmplength}{\textbf{Declaração}}
\setlength{\itemindent}{0cm}
\setlength{\listparindent}{0cm}
\setlength{\leftmargin}{\evensidemargin}
\addtolength{\leftmargin}{\tmplength}
\settowidth{\labelsep}{X}
\addtolength{\leftmargin}{\labelsep}
\setlength{\labelwidth}{\tmplength}
}
\begin{flushleft}
\item[\textbf{Declaração}\hfill]
\begin{ttfamily}
public const Const{\_}Ext{\_}Indice{\_}da{\_}tebala{\_}de{\_}Indices = '.IxI';\end{ttfamily}


\end{flushleft}
\end{list}
\paragraph*{Const{\_}Ext{\_}Tabela{\_}com{\_}os{\_}Relationships}\hspace*{\fill}

\begin{list}{}{
\settowidth{\tmplength}{\textbf{Declaração}}
\setlength{\itemindent}{0cm}
\setlength{\listparindent}{0cm}
\setlength{\leftmargin}{\evensidemargin}
\addtolength{\leftmargin}{\tmplength}
\settowidth{\labelsep}{X}
\addtolength{\leftmargin}{\labelsep}
\setlength{\labelwidth}{\tmplength}
}
\begin{flushleft}
\item[\textbf{Declaração}\hfill]
\begin{ttfamily}
public const Const{\_}Ext{\_}Tabela{\_}com{\_}os{\_}Relationships        = '.TbR';\end{ttfamily}


\end{flushleft}
\end{list}
\paragraph*{Const{\_}Ext{\_}Indice{\_}da{\_}tebala{\_}dos{\_}Relationships}\hspace*{\fill}

\begin{list}{}{
\settowidth{\tmplength}{\textbf{Declaração}}
\setlength{\itemindent}{0cm}
\setlength{\listparindent}{0cm}
\setlength{\leftmargin}{\evensidemargin}
\addtolength{\leftmargin}{\tmplength}
\settowidth{\labelsep}{X}
\addtolength{\leftmargin}{\labelsep}
\setlength{\labelwidth}{\tmplength}
}
\begin{flushleft}
\item[\textbf{Declaração}\hfill]
\begin{ttfamily}
public const Const{\_}Ext{\_}Indice{\_}da{\_}tebala{\_}dos{\_}Relationships = '.IxR';\end{ttfamily}


\end{flushleft}
\end{list}
\paragraph*{Const{\_}Ext{\_}Tabela{\_}com{\_}todos{\_}os{\_}campos{\_}de{\_}todas{\_}as{\_}tabelas}\hspace*{\fill}

\begin{list}{}{
\settowidth{\tmplength}{\textbf{Declaração}}
\setlength{\itemindent}{0cm}
\setlength{\listparindent}{0cm}
\setlength{\leftmargin}{\evensidemargin}
\addtolength{\leftmargin}{\tmplength}
\settowidth{\labelsep}{X}
\addtolength{\leftmargin}{\labelsep}
\setlength{\labelwidth}{\tmplength}
}
\begin{flushleft}
\item[\textbf{Declaração}\hfill]
\begin{ttfamily}
public const Const{\_}Ext{\_}Tabela{\_}com{\_}todos{\_}os{\_}campos{\_}de{\_}todas{\_}as{\_}tabelas = '.TbC';\end{ttfamily}


\end{flushleft}
\end{list}
\paragraph*{Const{\_}Ext{\_}Indice{\_}da{\_}tabela{\_}com{\_}todos{\_}os{\_}campos}\hspace*{\fill}

\begin{list}{}{
\settowidth{\tmplength}{\textbf{Declaração}}
\setlength{\itemindent}{0cm}
\setlength{\listparindent}{0cm}
\setlength{\leftmargin}{\evensidemargin}
\addtolength{\leftmargin}{\tmplength}
\settowidth{\labelsep}{X}
\addtolength{\leftmargin}{\labelsep}
\setlength{\labelwidth}{\tmplength}
}
\begin{flushleft}
\item[\textbf{Declaração}\hfill]
\begin{ttfamily}
public const Const{\_}Ext{\_}Indice{\_}da{\_}tabela{\_}com{\_}todos{\_}os{\_}campos           = '.IxC';\end{ttfamily}


\end{flushleft}
\end{list}
\paragraph*{Const{\_}Ext{\_}Tabela{\_}de{\_}Parametros}\hspace*{\fill}

\begin{list}{}{
\settowidth{\tmplength}{\textbf{Declaração}}
\setlength{\itemindent}{0cm}
\setlength{\listparindent}{0cm}
\setlength{\leftmargin}{\evensidemargin}
\addtolength{\leftmargin}{\tmplength}
\settowidth{\labelsep}{X}
\addtolength{\leftmargin}{\labelsep}
\setlength{\labelwidth}{\tmplength}
}
\begin{flushleft}
\item[\textbf{Declaração}\hfill]
\begin{ttfamily}
public const Const{\_}Ext{\_}Tabela{\_}de{\_}Parametros           =  '.TbP';\end{ttfamily}


\end{flushleft}
\end{list}
\paragraph*{Const{\_}Ext{\_}Tabela{\_}de{\_}Usuarios}\hspace*{\fill}

\begin{list}{}{
\settowidth{\tmplength}{\textbf{Declaração}}
\setlength{\itemindent}{0cm}
\setlength{\listparindent}{0cm}
\setlength{\leftmargin}{\evensidemargin}
\addtolength{\leftmargin}{\tmplength}
\settowidth{\labelsep}{X}
\addtolength{\leftmargin}{\labelsep}
\setlength{\labelwidth}{\tmplength}
}
\begin{flushleft}
\item[\textbf{Declaração}\hfill]
\begin{ttfamily}
public const Const{\_}Ext{\_}Tabela{\_}de{\_}Usuarios            = '.TbU';\end{ttfamily}


\end{flushleft}
\end{list}
\paragraph*{Const{\_}Ext{\_}Indice{\_}da{\_}Tabela{\_}de{\_}Usuarios}\hspace*{\fill}

\begin{list}{}{
\settowidth{\tmplength}{\textbf{Declaração}}
\setlength{\itemindent}{0cm}
\setlength{\listparindent}{0cm}
\setlength{\leftmargin}{\evensidemargin}
\addtolength{\leftmargin}{\tmplength}
\settowidth{\labelsep}{X}
\addtolength{\leftmargin}{\labelsep}
\setlength{\labelwidth}{\tmplength}
}
\begin{flushleft}
\item[\textbf{Declaração}\hfill]
\begin{ttfamily}
public const Const{\_}Ext{\_}Indice{\_}da{\_}Tabela{\_}de{\_}Usuarios  ='.IxU';\end{ttfamily}


\end{flushleft}
\end{list}
\paragraph*{Const{\_}Ext{\_}Backup{\_}da{\_}Tabela}\hspace*{\fill}

\begin{list}{}{
\settowidth{\tmplength}{\textbf{Declaração}}
\setlength{\itemindent}{0cm}
\setlength{\listparindent}{0cm}
\setlength{\leftmargin}{\evensidemargin}
\addtolength{\leftmargin}{\tmplength}
\settowidth{\labelsep}{X}
\addtolength{\leftmargin}{\labelsep}
\setlength{\labelwidth}{\tmplength}
}
\begin{flushleft}
\item[\textbf{Declaração}\hfill]
\begin{ttfamily}
public const Const{\_}Ext{\_}Backup{\_}da{\_}Tabela              = '.TbK';\end{ttfamily}


\end{flushleft}
\end{list}
\paragraph*{Const{\_}Ext{\_}Banco{\_}de{\_}dados{\_}Access}\hspace*{\fill}

\begin{list}{}{
\settowidth{\tmplength}{\textbf{Declaração}}
\setlength{\itemindent}{0cm}
\setlength{\listparindent}{0cm}
\setlength{\leftmargin}{\evensidemargin}
\addtolength{\leftmargin}{\tmplength}
\settowidth{\labelsep}{X}
\addtolength{\leftmargin}{\labelsep}
\setlength{\labelwidth}{\tmplength}
}
\begin{flushleft}
\item[\textbf{Declaração}\hfill]
\begin{ttfamily}
public const Const{\_}Ext{\_}Banco{\_}de{\_}dados{\_}Access             = '.Mdb';\end{ttfamily}


\end{flushleft}
\end{list}
\paragraph*{Const{\_}Ext{\_}Banco{\_}de{\_}dados{\_}Access{\_}Secundario}\hspace*{\fill}

\begin{list}{}{
\settowidth{\tmplength}{\textbf{Declaração}}
\setlength{\itemindent}{0cm}
\setlength{\listparindent}{0cm}
\setlength{\leftmargin}{\evensidemargin}
\addtolength{\leftmargin}{\tmplength}
\settowidth{\labelsep}{X}
\addtolength{\leftmargin}{\labelsep}
\setlength{\labelwidth}{\tmplength}
}
\begin{flushleft}
\item[\textbf{Declaração}\hfill]
\begin{ttfamily}
public const Const{\_}Ext{\_}Banco{\_}de{\_}dados{\_}Access{\_}Secundario  = '.ldb';\end{ttfamily}


\end{flushleft}
\end{list}
\paragraph*{Const{\_}Ext{\_}Banco{\_}de{\_}dados{\_}Interbase}\hspace*{\fill}

\begin{list}{}{
\settowidth{\tmplength}{\textbf{Declaração}}
\setlength{\itemindent}{0cm}
\setlength{\listparindent}{0cm}
\setlength{\leftmargin}{\evensidemargin}
\addtolength{\leftmargin}{\tmplength}
\settowidth{\labelsep}{X}
\addtolength{\leftmargin}{\labelsep}
\setlength{\labelwidth}{\tmplength}
}
\begin{flushleft}
\item[\textbf{Declaração}\hfill]
\begin{ttfamily}
public const Const{\_}Ext{\_}Banco{\_}de{\_}dados{\_}Interbase             = '.GDB';\end{ttfamily}


\end{flushleft}
\end{list}
\paragraph*{Const{\_}Ext{\_}Tabela{\_}Paradox}\hspace*{\fill}

\begin{list}{}{
\settowidth{\tmplength}{\textbf{Declaração}}
\setlength{\itemindent}{0cm}
\setlength{\listparindent}{0cm}
\setlength{\leftmargin}{\evensidemargin}
\addtolength{\leftmargin}{\tmplength}
\settowidth{\labelsep}{X}
\addtolength{\leftmargin}{\labelsep}
\setlength{\labelwidth}{\tmplength}
}
\begin{flushleft}
\item[\textbf{Declaração}\hfill]
\begin{ttfamily}
public const Const{\_}Ext{\_}Tabela{\_}Paradox                       = '.Db';\end{ttfamily}


\end{flushleft}
\end{list}
\paragraph*{Const{\_}Ext{\_}Tabela{\_}Paradox{\_}Px}\hspace*{\fill}

\begin{list}{}{
\settowidth{\tmplength}{\textbf{Declaração}}
\setlength{\itemindent}{0cm}
\setlength{\listparindent}{0cm}
\setlength{\leftmargin}{\evensidemargin}
\addtolength{\leftmargin}{\tmplength}
\settowidth{\labelsep}{X}
\addtolength{\leftmargin}{\labelsep}
\setlength{\labelwidth}{\tmplength}
}
\begin{flushleft}
\item[\textbf{Declaração}\hfill]
\begin{ttfamily}
public const Const{\_}Ext{\_}Tabela{\_}Paradox{\_}Px                    = '.Px';\end{ttfamily}


\end{flushleft}
\end{list}
\paragraph*{Const{\_}Ext{\_}Tabela{\_}Paradox{\_}Yx}\hspace*{\fill}

\begin{list}{}{
\settowidth{\tmplength}{\textbf{Declaração}}
\setlength{\itemindent}{0cm}
\setlength{\listparindent}{0cm}
\setlength{\leftmargin}{\evensidemargin}
\addtolength{\leftmargin}{\tmplength}
\settowidth{\labelsep}{X}
\addtolength{\leftmargin}{\labelsep}
\setlength{\labelwidth}{\tmplength}
}
\begin{flushleft}
\item[\textbf{Declaração}\hfill]
\begin{ttfamily}
public const Const{\_}Ext{\_}Tabela{\_}Paradox{\_}Yx                    = '.Yx';\end{ttfamily}


\end{flushleft}
\end{list}
\paragraph*{Const{\_}Ext{\_}Tabela{\_}DBF}\hspace*{\fill}

\begin{list}{}{
\settowidth{\tmplength}{\textbf{Declaração}}
\setlength{\itemindent}{0cm}
\setlength{\listparindent}{0cm}
\setlength{\leftmargin}{\evensidemargin}
\addtolength{\leftmargin}{\tmplength}
\settowidth{\labelsep}{X}
\addtolength{\leftmargin}{\labelsep}
\setlength{\labelwidth}{\tmplength}
}
\begin{flushleft}
\item[\textbf{Declaração}\hfill]
\begin{ttfamily}
public const Const{\_}Ext{\_}Tabela{\_}DBF                           = '.DBF';\end{ttfamily}


\end{flushleft}
\end{list}
\paragraph*{Const{\_}Ext{\_}Tabela{\_}DBF{\_}Ndx}\hspace*{\fill}

\begin{list}{}{
\settowidth{\tmplength}{\textbf{Declaração}}
\setlength{\itemindent}{0cm}
\setlength{\listparindent}{0cm}
\setlength{\leftmargin}{\evensidemargin}
\addtolength{\leftmargin}{\tmplength}
\settowidth{\labelsep}{X}
\addtolength{\leftmargin}{\labelsep}
\setlength{\labelwidth}{\tmplength}
}
\begin{flushleft}
\item[\textbf{Declaração}\hfill]
\begin{ttfamily}
public const Const{\_}Ext{\_}Tabela{\_}DBF{\_}Ndx                       = '.Ndx';\end{ttfamily}


\end{flushleft}
\end{list}
\paragraph*{Const{\_}Ext{\_}Tabela{\_}DBF{\_}Idx}\hspace*{\fill}

\begin{list}{}{
\settowidth{\tmplength}{\textbf{Declaração}}
\setlength{\itemindent}{0cm}
\setlength{\listparindent}{0cm}
\setlength{\leftmargin}{\evensidemargin}
\addtolength{\leftmargin}{\tmplength}
\settowidth{\labelsep}{X}
\addtolength{\leftmargin}{\labelsep}
\setlength{\labelwidth}{\tmplength}
}
\begin{flushleft}
\item[\textbf{Declaração}\hfill]
\begin{ttfamily}
public const Const{\_}Ext{\_}Tabela{\_}DBF{\_}Idx                       = '.Idx';\end{ttfamily}


\end{flushleft}
\end{list}
\paragraph*{Const{\_}Ext{\_}Tabela{\_}Word}\hspace*{\fill}

\begin{list}{}{
\settowidth{\tmplength}{\textbf{Declaração}}
\setlength{\itemindent}{0cm}
\setlength{\listparindent}{0cm}
\setlength{\leftmargin}{\evensidemargin}
\addtolength{\leftmargin}{\tmplength}
\settowidth{\labelsep}{X}
\addtolength{\leftmargin}{\labelsep}
\setlength{\labelwidth}{\tmplength}
}
\begin{flushleft}
\item[\textbf{Declaração}\hfill]
\begin{ttfamily}
public const Const{\_}Ext{\_}Tabela{\_}Word                          = '.Doc';\end{ttfamily}


\end{flushleft}
\end{list}
\paragraph*{Const{\_}Ext{\_}Tabela{\_}Excel}\hspace*{\fill}

\begin{list}{}{
\settowidth{\tmplength}{\textbf{Declaração}}
\setlength{\itemindent}{0cm}
\setlength{\listparindent}{0cm}
\setlength{\leftmargin}{\evensidemargin}
\addtolength{\leftmargin}{\tmplength}
\settowidth{\labelsep}{X}
\addtolength{\leftmargin}{\labelsep}
\setlength{\labelwidth}{\tmplength}
}
\begin{flushleft}
\item[\textbf{Declaração}\hfill]
\begin{ttfamily}
public const Const{\_}Ext{\_}Tabela{\_}Excel                         = '.Xls';\end{ttfamily}


\end{flushleft}
\end{list}
\paragraph*{Const{\_}Ext{\_}Array}\hspace*{\fill}

\begin{list}{}{
\settowidth{\tmplength}{\textbf{Declaração}}
\setlength{\itemindent}{0cm}
\setlength{\listparindent}{0cm}
\setlength{\leftmargin}{\evensidemargin}
\addtolength{\leftmargin}{\tmplength}
\settowidth{\labelsep}{X}
\addtolength{\leftmargin}{\labelsep}
\setlength{\labelwidth}{\tmplength}
}
\begin{flushleft}
\item[\textbf{Declaração}\hfill]
\begin{ttfamily}
public const Const{\_}Ext{\_}Array : Array[1..24] of string[4] = (
                                Const{\_}Ext{\_}Tabela,
                                Const{\_}Ext{\_}Indice{\_}da{\_}tebela,

                                Const{\_}Ext{\_}Tabela{\_}com{\_}a{\_}copia{\_}da{\_}versao{\_}anterior{\_}da{\_}tabela,

                                Const{\_}Ext{\_}Tabela{\_}de{\_}objetos{\_}vinculados{\_}a{\_}tabela,
                                Const{\_}Ext{\_}Tabela{\_}com{\_}os{\_}registro{\_}duplicados,

                                Const{\_}Ext{\_}Tebela{\_}com{\_}as{\_}Tabelas,
                                Const{\_}Ext{\_}Indice{\_}da{\_}Tabela{\_}das{\_}tabelas,

                                Const{\_}Ext{\_}Tabela{\_}com{\_}os{\_}Indices,
                                Const{\_}Ext{\_}Indice{\_}da{\_}tebala{\_}de{\_}Indices,

                                Const{\_}Ext{\_}Tabela{\_}com{\_}os{\_}Relationships,
                                Const{\_}Ext{\_}Indice{\_}da{\_}tebala{\_}dos{\_}Relationships,

                                Const{\_}Ext{\_}Tabela{\_}com{\_}todos{\_}os{\_}campos{\_}de{\_}todas{\_}as{\_}tabelas,
                                Const{\_}Ext{\_}Indice{\_}da{\_}tabela{\_}com{\_}todos{\_}os{\_}campos,

                                Const{\_}Ext{\_}Tabela{\_}de{\_}Parametros,

                                Const{\_}Ext{\_}Tabela{\_}de{\_}Usuarios,
                                Const{\_}Ext{\_}Indice{\_}da{\_}Tabela{\_}de{\_}Usuarios,

                                Const{\_}Ext{\_}Backup{\_}da{\_}Tabela,

                                Const{\_}Ext{\_}Banco{\_}de{\_}dados{\_}Access,
                                Const{\_}Ext{\_}Banco{\_}de{\_}dados{\_}Access{\_}Secundario,

                                Const{\_}Ext{\_}Tabela{\_}DBF,
                                Const{\_}Ext{\_}Tabela{\_}DBF{\_}Ndx,
                                Const{\_}Ext{\_}Tabela{\_}DBF{\_}Idx,

                                Const{\_}Ext{\_}Tabela{\_}Word,
                                Const{\_}Ext{\_}Tabela{\_}Excel
                                );\end{ttfamily}


\end{flushleft}
\end{list}
\subsection*{TDataFile Classe}
\subsubsection*{\large{\textbf{Hierarquia}}\normalsize\hspace{1ex}\hfill}
TDataFile {$>$} \begin{ttfamily}TTb{\_}Access{\_}consts\end{ttfamily}(\ref{mi.rtl.Objects.Methods.Db.Tb_Access.TTb_Access_consts}) {$>$} \begin{ttfamily}TTb{\_}Access{\_}types\end{ttfamily}(\ref{mi.rtl.Objects.Methods.Db.Tb_Access.TTb_Access_types}) {$>$} \begin{ttfamily}TObjectsSystem\end{ttfamily}(\ref{mi.rtl.Objects.Methods.System.TObjectsSystem}) {$>$} \begin{ttfamily}TObjectsMethods\end{ttfamily}(\ref{mi.rtl.Objects.Methods.TObjectsMethods}) {$>$} \begin{ttfamily}TObjectsConsts\end{ttfamily}(\ref{mi.rtl.Objects.Consts.TObjectsConsts}) {$>$} 
TObjectsTypes
\subsubsection*{\large{\textbf{Descrição}}\normalsize\hspace{1ex}\hfill}
A classe \textbf{\begin{ttfamily}TDataFile\end{ttfamily}} é usada para acessar arquivos em disco \textbf{\begin{ttfamily}TTb{\_}Access\end{ttfamily}(\ref{mi.rtl.Objects.Methods.Db.Tb_Access.TTb_Access})}\subsubsection*{\large{\textbf{Campos}}\normalsize\hspace{1ex}\hfill}
\paragraph*{DataFile}\hspace*{\fill}

\begin{list}{}{
\settowidth{\tmplength}{\textbf{Declaração}}
\setlength{\itemindent}{0cm}
\setlength{\listparindent}{0cm}
\setlength{\leftmargin}{\evensidemargin}
\addtolength{\leftmargin}{\tmplength}
\settowidth{\labelsep}{X}
\addtolength{\leftmargin}{\labelsep}
\setlength{\labelwidth}{\tmplength}
}
\begin{flushleft}
\item[\textbf{Declaração}\hfill]
\begin{ttfamily}
public DataFile: {\^{}}DataFile;\end{ttfamily}


\end{flushleft}
\end{list}
\paragraph*{Ok{\_}CloseDataFile}\hspace*{\fill}

\begin{list}{}{
\settowidth{\tmplength}{\textbf{Declaração}}
\setlength{\itemindent}{0cm}
\setlength{\listparindent}{0cm}
\setlength{\leftmargin}{\evensidemargin}
\addtolength{\leftmargin}{\tmplength}
\settowidth{\labelsep}{X}
\addtolength{\leftmargin}{\labelsep}
\setlength{\labelwidth}{\tmplength}
}
\begin{flushleft}
\item[\textbf{Declaração}\hfill]
\begin{ttfamily}
public Ok{\_}CloseDataFile: Boolean;\end{ttfamily}


\end{flushleft}
\end{list}
\subsubsection*{\large{\textbf{Métodos}}\normalsize\hspace{1ex}\hfill}
\paragraph*{Create}\hspace*{\fill}

\begin{list}{}{
\settowidth{\tmplength}{\textbf{Declaração}}
\setlength{\itemindent}{0cm}
\setlength{\listparindent}{0cm}
\setlength{\leftmargin}{\evensidemargin}
\addtolength{\leftmargin}{\tmplength}
\settowidth{\labelsep}{X}
\addtolength{\leftmargin}{\labelsep}
\setlength{\labelwidth}{\tmplength}
}
\begin{flushleft}
\item[\textbf{Declaração}\hfill]
\begin{ttfamily}
public Constructor Create(const aDataFile : DataFile;Const aOk{\_}CloseDataFile:Boolean); virtual;\end{ttfamily}


\end{flushleft}
\end{list}
\paragraph*{Destroy}\hspace*{\fill}

\begin{list}{}{
\settowidth{\tmplength}{\textbf{Declaração}}
\setlength{\itemindent}{0cm}
\setlength{\listparindent}{0cm}
\setlength{\leftmargin}{\evensidemargin}
\addtolength{\leftmargin}{\tmplength}
\settowidth{\labelsep}{X}
\addtolength{\leftmargin}{\labelsep}
\setlength{\labelwidth}{\tmplength}
}
\begin{flushleft}
\item[\textbf{Declaração}\hfill]
\begin{ttfamily}
public Destructor Destroy; Override;\end{ttfamily}


\end{flushleft}
\end{list}
\paragraph*{SetDataFile}\hspace*{\fill}

\begin{list}{}{
\settowidth{\tmplength}{\textbf{Declaração}}
\setlength{\itemindent}{0cm}
\setlength{\listparindent}{0cm}
\setlength{\leftmargin}{\evensidemargin}
\addtolength{\leftmargin}{\tmplength}
\settowidth{\labelsep}{X}
\addtolength{\leftmargin}{\labelsep}
\setlength{\labelwidth}{\tmplength}
}
\begin{flushleft}
\item[\textbf{Declaração}\hfill]
\begin{ttfamily}
public procedure SetDataFile(const aDataFile : DataFile;Const aOk{\_}CloseDataFile : Boolean);\end{ttfamily}


\end{flushleft}
\end{list}
\subsection*{TTb{\_}Access Classe}
\subsubsection*{\large{\textbf{Hierarquia}}\normalsize\hspace{1ex}\hfill}
TTb{\_}Access {$>$} \begin{ttfamily}TTb{\_}Access{\_}consts\end{ttfamily}(\ref{mi.rtl.Objects.Methods.Db.Tb_Access.TTb_Access_consts}) {$>$} \begin{ttfamily}TTb{\_}Access{\_}types\end{ttfamily}(\ref{mi.rtl.Objects.Methods.Db.Tb_Access.TTb_Access_types}) {$>$} \begin{ttfamily}TObjectsSystem\end{ttfamily}(\ref{mi.rtl.Objects.Methods.System.TObjectsSystem}) {$>$} \begin{ttfamily}TObjectsMethods\end{ttfamily}(\ref{mi.rtl.Objects.Methods.TObjectsMethods}) {$>$} \begin{ttfamily}TObjectsConsts\end{ttfamily}(\ref{mi.rtl.Objects.Consts.TObjectsConsts}) {$>$} 
TObjectsTypes
\subsubsection*{\large{\textbf{Descrição}}\normalsize\hspace{1ex}\hfill}
no description available, TTb{\_}Access{\_}consts description followsA classe \textbf{\begin{ttfamily}TTb{\_}Access{\_}consts\end{ttfamily}} é usada para declarar todas as constantes da classe \textbf{\begin{ttfamily}TTb{\_}Access\end{ttfamily}(\ref{mi.rtl.Objects.Methods.Db.Tb_Access.TTb_Access})}\subsubsection*{\large{\textbf{Métodos}}\normalsize\hspace{1ex}\hfill}
\paragraph*{Create}\hspace*{\fill}

\begin{list}{}{
\settowidth{\tmplength}{\textbf{Declaração}}
\setlength{\itemindent}{0cm}
\setlength{\listparindent}{0cm}
\setlength{\leftmargin}{\evensidemargin}
\addtolength{\leftmargin}{\tmplength}
\settowidth{\labelsep}{X}
\addtolength{\leftmargin}{\labelsep}
\setlength{\labelwidth}{\tmplength}
}
\begin{flushleft}
\item[\textbf{Declaração}\hfill]
\begin{ttfamily}
public class Procedure Create;\end{ttfamily}


\end{flushleft}
\end{list}
\paragraph*{Destroy}\hspace*{\fill}

\begin{list}{}{
\settowidth{\tmplength}{\textbf{Declaração}}
\setlength{\itemindent}{0cm}
\setlength{\listparindent}{0cm}
\setlength{\leftmargin}{\evensidemargin}
\addtolength{\leftmargin}{\tmplength}
\settowidth{\labelsep}{X}
\addtolength{\leftmargin}{\labelsep}
\setlength{\labelwidth}{\tmplength}
}
\begin{flushleft}
\item[\textbf{Declaração}\hfill]
\begin{ttfamily}
public class Procedure Destroy;\end{ttfamily}


\end{flushleft}
\end{list}
\paragraph*{FExisteCodigo}\hspace*{\fill}

\begin{list}{}{
\settowidth{\tmplength}{\textbf{Declaração}}
\setlength{\itemindent}{0cm}
\setlength{\listparindent}{0cm}
\setlength{\leftmargin}{\evensidemargin}
\addtolength{\leftmargin}{\tmplength}
\settowidth{\labelsep}{X}
\addtolength{\leftmargin}{\labelsep}
\setlength{\labelwidth}{\tmplength}
}
\begin{flushleft}
\item[\textbf{Declaração}\hfill]
\begin{ttfamily}
public class function FExisteCodigo(Var IxF:IndexFile; Const Codigo:tString):Boolean;\end{ttfamily}


\end{flushleft}
\end{list}
\paragraph*{CreateTAccess}\hspace*{\fill}

\begin{list}{}{
\settowidth{\tmplength}{\textbf{Declaração}}
\setlength{\itemindent}{0cm}
\setlength{\listparindent}{0cm}
\setlength{\leftmargin}{\evensidemargin}
\addtolength{\leftmargin}{\tmplength}
\settowidth{\labelsep}{X}
\addtolength{\leftmargin}{\labelsep}
\setlength{\labelwidth}{\tmplength}
}
\begin{flushleft}
\item[\textbf{Declaração}\hfill]
\begin{ttfamily}
public class Procedure CreateTAccess;\end{ttfamily}


\end{flushleft}
\end{list}
\paragraph*{DestroyTAccess}\hspace*{\fill}

\begin{list}{}{
\settowidth{\tmplength}{\textbf{Declaração}}
\setlength{\itemindent}{0cm}
\setlength{\listparindent}{0cm}
\setlength{\leftmargin}{\evensidemargin}
\addtolength{\leftmargin}{\tmplength}
\settowidth{\labelsep}{X}
\addtolength{\leftmargin}{\labelsep}
\setlength{\labelwidth}{\tmplength}
}
\begin{flushleft}
\item[\textbf{Declaração}\hfill]
\begin{ttfamily}
public class Procedure DestroyTAccess;\end{ttfamily}


\end{flushleft}
\end{list}
\paragraph*{EscrevaTurboError}\hspace*{\fill}

\begin{list}{}{
\settowidth{\tmplength}{\textbf{Declaração}}
\setlength{\itemindent}{0cm}
\setlength{\listparindent}{0cm}
\setlength{\leftmargin}{\evensidemargin}
\addtolength{\leftmargin}{\tmplength}
\settowidth{\labelsep}{X}
\addtolength{\leftmargin}{\labelsep}
\setlength{\labelwidth}{\tmplength}
}
\begin{flushleft}
\item[\textbf{Declaração}\hfill]
\begin{ttfamily}
public class function EscrevaTurboError(DatF : DataFile;Const NR : Longint;Error:SmallWord):Boolean;\end{ttfamily}


\end{flushleft}
\end{list}
\paragraph*{TAIOCheck}\hspace*{\fill}

\begin{list}{}{
\settowidth{\tmplength}{\textbf{Declaração}}
\setlength{\itemindent}{0cm}
\setlength{\listparindent}{0cm}
\setlength{\leftmargin}{\evensidemargin}
\addtolength{\leftmargin}{\tmplength}
\settowidth{\labelsep}{X}
\addtolength{\leftmargin}{\labelsep}
\setlength{\labelwidth}{\tmplength}
}
\begin{flushleft}
\item[\textbf{Declaração}\hfill]
\begin{ttfamily}
public class function TAIOCheck(VAR DatF : DataFile;Const R : LONGINT):Boolean;\end{ttfamily}


\end{flushleft}
\end{list}
\paragraph*{SincronizaPosChave}\hspace*{\fill}

\begin{list}{}{
\settowidth{\tmplength}{\textbf{Declaração}}
\setlength{\itemindent}{0cm}
\setlength{\listparindent}{0cm}
\setlength{\leftmargin}{\evensidemargin}
\addtolength{\leftmargin}{\tmplength}
\settowidth{\labelsep}{X}
\addtolength{\leftmargin}{\labelsep}
\setlength{\labelwidth}{\tmplength}
}
\begin{flushleft}
\item[\textbf{Declaração}\hfill]
\begin{ttfamily}
public class function SincronizaPosChave(Var datFIx:IndexFile;Const NrCurrent:Longint; KeyCurrent : TaKeyStr):Boolean;\end{ttfamily}


\end{flushleft}
\end{list}
\paragraph*{GetRec}\hspace*{\fill}

\begin{list}{}{
\settowidth{\tmplength}{\textbf{Declaração}}
\setlength{\itemindent}{0cm}
\setlength{\listparindent}{0cm}
\setlength{\leftmargin}{\evensidemargin}
\addtolength{\leftmargin}{\tmplength}
\settowidth{\labelsep}{X}
\addtolength{\leftmargin}{\labelsep}
\setlength{\labelwidth}{\tmplength}
}
\begin{flushleft}
\item[\textbf{Declaração}\hfill]
\begin{ttfamily}
public class function GetRec(var DatF : DataFile;Const R : Longint;var Buffer ):Boolean; overload;\end{ttfamily}


\end{flushleft}
\end{list}
\paragraph*{GetRecBlock}\hspace*{\fill}

\begin{list}{}{
\settowidth{\tmplength}{\textbf{Declaração}}
\setlength{\itemindent}{0cm}
\setlength{\listparindent}{0cm}
\setlength{\leftmargin}{\evensidemargin}
\addtolength{\leftmargin}{\tmplength}
\settowidth{\labelsep}{X}
\addtolength{\leftmargin}{\labelsep}
\setlength{\labelwidth}{\tmplength}
}
\begin{flushleft}
\item[\textbf{Declaração}\hfill]
\begin{ttfamily}
public class function GetRecBlock(VAR DatF : DataFile; Const R : LONGINT; delta:Word;Var BlocksRead:Word ;VAR Buffer ):Boolean;\end{ttfamily}


\end{flushleft}
\end{list}
\paragraph*{PutRec}\hspace*{\fill}

\begin{list}{}{
\settowidth{\tmplength}{\textbf{Declaração}}
\setlength{\itemindent}{0cm}
\setlength{\listparindent}{0cm}
\setlength{\leftmargin}{\evensidemargin}
\addtolength{\leftmargin}{\tmplength}
\settowidth{\labelsep}{X}
\addtolength{\leftmargin}{\labelsep}
\setlength{\labelwidth}{\tmplength}
}
\begin{flushleft}
\item[\textbf{Declaração}\hfill]
\begin{ttfamily}
public class function PutRec(var DatF : DataFile;Const R : Longint;var Buffer;Const Transaction{\_}Current : T{\_}TTransaction):Boolean; overload;\end{ttfamily}


\end{flushleft}
\end{list}
\paragraph*{PutRec}\hspace*{\fill}

\begin{list}{}{
\settowidth{\tmplength}{\textbf{Declaração}}
\setlength{\itemindent}{0cm}
\setlength{\listparindent}{0cm}
\setlength{\leftmargin}{\evensidemargin}
\addtolength{\leftmargin}{\tmplength}
\settowidth{\labelsep}{X}
\addtolength{\leftmargin}{\labelsep}
\setlength{\labelwidth}{\tmplength}
}
\begin{flushleft}
\item[\textbf{Declaração}\hfill]
\begin{ttfamily}
public class function PutRec(var DatF : DataFile;Const R : Longint;var Buffer ):Boolean; overload;\end{ttfamily}


\end{flushleft}
\end{list}
\paragraph*{MakeFile}\hspace*{\fill}

\begin{list}{}{
\settowidth{\tmplength}{\textbf{Declaração}}
\setlength{\itemindent}{0cm}
\setlength{\listparindent}{0cm}
\setlength{\leftmargin}{\evensidemargin}
\addtolength{\leftmargin}{\tmplength}
\settowidth{\labelsep}{X}
\addtolength{\leftmargin}{\labelsep}
\setlength{\labelwidth}{\tmplength}
}
\begin{flushleft}
\item[\textbf{Declaração}\hfill]
\begin{ttfamily}
public class function MakeFile(var DatF : DataFile;Const FName : FileName;Const RecLen : SmallWord):Integer; overload;\end{ttfamily}


\end{flushleft}
\end{list}
\paragraph*{FDelStrBrancos}\hspace*{\fill}

\begin{list}{}{
\settowidth{\tmplength}{\textbf{Declaração}}
\setlength{\itemindent}{0cm}
\setlength{\listparindent}{0cm}
\setlength{\leftmargin}{\evensidemargin}
\addtolength{\leftmargin}{\tmplength}
\settowidth{\labelsep}{X}
\addtolength{\leftmargin}{\labelsep}
\setlength{\labelwidth}{\tmplength}
}
\begin{flushleft}
\item[\textbf{Declaração}\hfill]
\begin{ttfamily}
public class function FDelStrBrancos(S:tString):tString;\end{ttfamily}


\end{flushleft}
\end{list}
\paragraph*{FileNameTemp{\_}Ext}\hspace*{\fill}

\begin{list}{}{
\settowidth{\tmplength}{\textbf{Declaração}}
\setlength{\itemindent}{0cm}
\setlength{\listparindent}{0cm}
\setlength{\leftmargin}{\evensidemargin}
\addtolength{\leftmargin}{\tmplength}
\settowidth{\labelsep}{X}
\addtolength{\leftmargin}{\labelsep}
\setlength{\labelwidth}{\tmplength}
}
\begin{flushleft}
\item[\textbf{Declaração}\hfill]
\begin{ttfamily}
public class function FileNameTemp{\_}Ext(const aPath:PathStr;Var NomeArqTemp : PathStr;const Extencao : PathStr):Boolean; overload;\end{ttfamily}


\end{flushleft}
\end{list}
\paragraph*{FileNameTemp{\_}Ext}\hspace*{\fill}

\begin{list}{}{
\settowidth{\tmplength}{\textbf{Declaração}}
\setlength{\itemindent}{0cm}
\setlength{\listparindent}{0cm}
\setlength{\leftmargin}{\evensidemargin}
\addtolength{\leftmargin}{\tmplength}
\settowidth{\labelsep}{X}
\addtolength{\leftmargin}{\labelsep}
\setlength{\labelwidth}{\tmplength}
}
\begin{flushleft}
\item[\textbf{Declaração}\hfill]
\begin{ttfamily}
public class function FileNameTemp{\_}Ext(Var NomeArqTemp : PathStr;const Extencao : PathStr):Boolean; Overload;\end{ttfamily}


\end{flushleft}
\end{list}
\paragraph*{FileNameTemp}\hspace*{\fill}

\begin{list}{}{
\settowidth{\tmplength}{\textbf{Declaração}}
\setlength{\itemindent}{0cm}
\setlength{\listparindent}{0cm}
\setlength{\leftmargin}{\evensidemargin}
\addtolength{\leftmargin}{\tmplength}
\settowidth{\labelsep}{X}
\addtolength{\leftmargin}{\labelsep}
\setlength{\labelwidth}{\tmplength}
}
\begin{flushleft}
\item[\textbf{Declaração}\hfill]
\begin{ttfamily}
public class function FileNameTemp(const Extencao : PathStr):PathStr;\end{ttfamily}


\end{flushleft}
\end{list}
\paragraph*{FileName{\_}Seq}\hspace*{\fill}

\begin{list}{}{
\settowidth{\tmplength}{\textbf{Declaração}}
\setlength{\itemindent}{0cm}
\setlength{\listparindent}{0cm}
\setlength{\leftmargin}{\evensidemargin}
\addtolength{\leftmargin}{\tmplength}
\settowidth{\labelsep}{X}
\addtolength{\leftmargin}{\labelsep}
\setlength{\labelwidth}{\tmplength}
}
\begin{flushleft}
\item[\textbf{Declaração}\hfill]
\begin{ttfamily}
public class function FileName{\_}Seq(Const aName:PathStr;Const aExt : PathStr):PathStr;\end{ttfamily}


\end{flushleft}
\end{list}
\paragraph*{IsPortLocal}\hspace*{\fill}

\begin{list}{}{
\settowidth{\tmplength}{\textbf{Declaração}}
\setlength{\itemindent}{0cm}
\setlength{\listparindent}{0cm}
\setlength{\leftmargin}{\evensidemargin}
\addtolength{\leftmargin}{\tmplength}
\settowidth{\labelsep}{X}
\addtolength{\leftmargin}{\labelsep}
\setlength{\labelwidth}{\tmplength}
}
\begin{flushleft}
\item[\textbf{Declaração}\hfill]
\begin{ttfamily}
public class function IsPortLocal(WPort: TTb{\_}Access.tString):Boolean;\end{ttfamily}


\end{flushleft}
\end{list}
\paragraph*{DelFile}\hspace*{\fill}

\begin{list}{}{
\settowidth{\tmplength}{\textbf{Declaração}}
\setlength{\itemindent}{0cm}
\setlength{\listparindent}{0cm}
\setlength{\leftmargin}{\evensidemargin}
\addtolength{\leftmargin}{\tmplength}
\settowidth{\labelsep}{X}
\addtolength{\leftmargin}{\labelsep}
\setlength{\labelwidth}{\tmplength}
}
\begin{flushleft}
\item[\textbf{Declaração}\hfill]
\begin{ttfamily}
public class function DelFile( Const Nome : NameStr):Boolean;\end{ttfamily}


\end{flushleft}
\end{list}
\paragraph*{SetOkAddRecFirstFree}\hspace*{\fill}

\begin{list}{}{
\settowidth{\tmplength}{\textbf{Declaração}}
\setlength{\itemindent}{0cm}
\setlength{\listparindent}{0cm}
\setlength{\leftmargin}{\evensidemargin}
\addtolength{\leftmargin}{\tmplength}
\settowidth{\labelsep}{X}
\addtolength{\leftmargin}{\labelsep}
\setlength{\labelwidth}{\tmplength}
}
\begin{flushleft}
\item[\textbf{Declaração}\hfill]
\begin{ttfamily}
public class function SetOkAddRecFirstFree(Const aOkAddRecFirstFree: Boolean):Boolean;\end{ttfamily}


\end{flushleft}
\end{list}
\paragraph*{TestaSePodeAbrirArquivo}\hspace*{\fill}

\begin{list}{}{
\settowidth{\tmplength}{\textbf{Declaração}}
\setlength{\itemindent}{0cm}
\setlength{\listparindent}{0cm}
\setlength{\leftmargin}{\evensidemargin}
\addtolength{\leftmargin}{\tmplength}
\settowidth{\labelsep}{X}
\addtolength{\leftmargin}{\labelsep}
\setlength{\labelwidth}{\tmplength}
}
\begin{flushleft}
\item[\textbf{Declaração}\hfill]
\begin{ttfamily}
public class function TestaSePodeAbrirArquivo(Const FName : PathStr): Byte;\end{ttfamily}


\end{flushleft}
\end{list}
\paragraph*{FileShared}\hspace*{\fill}

\begin{list}{}{
\settowidth{\tmplength}{\textbf{Declaração}}
\setlength{\itemindent}{0cm}
\setlength{\listparindent}{0cm}
\setlength{\leftmargin}{\evensidemargin}
\addtolength{\leftmargin}{\tmplength}
\settowidth{\labelsep}{X}
\addtolength{\leftmargin}{\labelsep}
\setlength{\labelwidth}{\tmplength}
}
\begin{flushleft}
\item[\textbf{Declaração}\hfill]
\begin{ttfamily}
public class function FileShared(Const FName : PathStr) : Boolean;\end{ttfamily}


\end{flushleft}
\end{list}
\paragraph*{FTrocaExtencao}\hspace*{\fill}

\begin{list}{}{
\settowidth{\tmplength}{\textbf{Declaração}}
\setlength{\itemindent}{0cm}
\setlength{\listparindent}{0cm}
\setlength{\leftmargin}{\evensidemargin}
\addtolength{\leftmargin}{\tmplength}
\settowidth{\labelsep}{X}
\addtolength{\leftmargin}{\labelsep}
\setlength{\labelwidth}{\tmplength}
}
\begin{flushleft}
\item[\textbf{Declaração}\hfill]
\begin{ttfamily}
public class function FTrocaExtencao(Const NomeArq:NameStr; Extencao:PathStr) : PathStr;\end{ttfamily}


\end{flushleft}
\end{list}
\paragraph*{Ren}\hspace*{\fill}

\begin{list}{}{
\settowidth{\tmplength}{\textbf{Declaração}}
\setlength{\itemindent}{0cm}
\setlength{\listparindent}{0cm}
\setlength{\leftmargin}{\evensidemargin}
\addtolength{\leftmargin}{\tmplength}
\settowidth{\labelsep}{X}
\addtolength{\leftmargin}{\labelsep}
\setlength{\labelwidth}{\tmplength}
}
\begin{flushleft}
\item[\textbf{Declaração}\hfill]
\begin{ttfamily}
public class function Ren(NomeFonte,NomeDestino: PathStr) : Boolean;\end{ttfamily}


\end{flushleft}
\end{list}
\paragraph*{OkRecSizeMismatch}\hspace*{\fill}

\begin{list}{}{
\settowidth{\tmplength}{\textbf{Declaração}}
\setlength{\itemindent}{0cm}
\setlength{\listparindent}{0cm}
\setlength{\leftmargin}{\evensidemargin}
\addtolength{\leftmargin}{\tmplength}
\settowidth{\labelsep}{X}
\addtolength{\leftmargin}{\labelsep}
\setlength{\labelwidth}{\tmplength}
}
\begin{flushleft}
\item[\textbf{Declaração}\hfill]
\begin{ttfamily}
public class function OkRecSizeMismatch(Const FName : FileName;Const RecLenBufferRecord : SmallWord):Boolean;\end{ttfamily}


\end{flushleft}
\par
\item[\textbf{Descrição}]
A class método \textbf{\begin{ttfamily}OkRecSizeMismatch\end{ttfamily}} retorna true se o Tamanho do registro em arquivo é maior que o tamanho do buffer do registro em Memória.

\end{list}
\paragraph*{ModifyStructurFile}\hspace*{\fill}

\begin{list}{}{
\settowidth{\tmplength}{\textbf{Declaração}}
\setlength{\itemindent}{0cm}
\setlength{\listparindent}{0cm}
\setlength{\leftmargin}{\evensidemargin}
\addtolength{\leftmargin}{\tmplength}
\settowidth{\labelsep}{X}
\addtolength{\leftmargin}{\labelsep}
\setlength{\labelwidth}{\tmplength}
}
\begin{flushleft}
\item[\textbf{Declaração}\hfill]
\begin{ttfamily}
public class function ModifyStructurFile(Const FName:FileName;Const RecLenDest : SmallWord ):Boolean; virtual; abstract;\end{ttfamily}


\end{flushleft}
\end{list}
\paragraph*{OpenFile}\hspace*{\fill}

\begin{list}{}{
\settowidth{\tmplength}{\textbf{Declaração}}
\setlength{\itemindent}{0cm}
\setlength{\listparindent}{0cm}
\setlength{\leftmargin}{\evensidemargin}
\addtolength{\leftmargin}{\tmplength}
\settowidth{\labelsep}{X}
\addtolength{\leftmargin}{\labelsep}
\setlength{\labelwidth}{\tmplength}
}
\begin{flushleft}
\item[\textbf{Declaração}\hfill]
\begin{ttfamily}
public class function OpenFile(var DatF:DataFile; Const FName : FileName; Const RecLen:SmallWord):Integer;\end{ttfamily}


\end{flushleft}
\end{list}
\paragraph*{ReadHeader}\hspace*{\fill}

\begin{list}{}{
\settowidth{\tmplength}{\textbf{Declaração}}
\setlength{\itemindent}{0cm}
\setlength{\listparindent}{0cm}
\setlength{\leftmargin}{\evensidemargin}
\addtolength{\leftmargin}{\tmplength}
\settowidth{\labelsep}{X}
\addtolength{\leftmargin}{\labelsep}
\setlength{\labelwidth}{\tmplength}
}
\begin{flushleft}
\item[\textbf{Declaração}\hfill]
\begin{ttfamily}
public class function ReadHeader(VAR DatF : DataFile):Boolean;\end{ttfamily}


\end{flushleft}
\end{list}
\paragraph*{PutFileHeader}\hspace*{\fill}

\begin{list}{}{
\settowidth{\tmplength}{\textbf{Declaração}}
\setlength{\itemindent}{0cm}
\setlength{\listparindent}{0cm}
\setlength{\leftmargin}{\evensidemargin}
\addtolength{\leftmargin}{\tmplength}
\settowidth{\labelsep}{X}
\addtolength{\leftmargin}{\labelsep}
\setlength{\labelwidth}{\tmplength}
}
\begin{flushleft}
\item[\textbf{Declaração}\hfill]
\begin{ttfamily}
public class function PutFileHeader(VAR DatF : DataFile):Boolean;\end{ttfamily}


\end{flushleft}
\end{list}
\paragraph*{NaoMuDOuHeader}\hspace*{\fill}

\begin{list}{}{
\settowidth{\tmplength}{\textbf{Declaração}}
\setlength{\itemindent}{0cm}
\setlength{\listparindent}{0cm}
\setlength{\leftmargin}{\evensidemargin}
\addtolength{\leftmargin}{\tmplength}
\settowidth{\labelsep}{X}
\addtolength{\leftmargin}{\labelsep}
\setlength{\labelwidth}{\tmplength}
}
\begin{flushleft}
\item[\textbf{Declaração}\hfill]
\begin{ttfamily}
public class function NaoMuDOuHeader(VAR DatF : DataFile) : BOOLEAN;\end{ttfamily}


\end{flushleft}
\end{list}
\paragraph*{MudouHeaderEmMemoria}\hspace*{\fill}

\begin{list}{}{
\settowidth{\tmplength}{\textbf{Declaração}}
\setlength{\itemindent}{0cm}
\setlength{\listparindent}{0cm}
\setlength{\leftmargin}{\evensidemargin}
\addtolength{\leftmargin}{\tmplength}
\settowidth{\labelsep}{X}
\addtolength{\leftmargin}{\labelsep}
\setlength{\labelwidth}{\tmplength}
}
\begin{flushleft}
\item[\textbf{Declaração}\hfill]
\begin{ttfamily}
public class function MudouHeaderEmMemoria(VAR DatF : DataFile) : BOOLEAN;\end{ttfamily}


\end{flushleft}
\end{list}
\paragraph*{aCloseFile}\hspace*{\fill}

\begin{list}{}{
\settowidth{\tmplength}{\textbf{Declaração}}
\setlength{\itemindent}{0cm}
\setlength{\listparindent}{0cm}
\setlength{\leftmargin}{\evensidemargin}
\addtolength{\leftmargin}{\tmplength}
\settowidth{\labelsep}{X}
\addtolength{\leftmargin}{\labelsep}
\setlength{\labelwidth}{\tmplength}
}
\begin{flushleft}
\item[\textbf{Declaração}\hfill]
\begin{ttfamily}
public class function aCloseFile(VAR DatF : DataFile):boolean;\end{ttfamily}


\end{flushleft}
\end{list}
\paragraph*{CloseFile}\hspace*{\fill}

\begin{list}{}{
\settowidth{\tmplength}{\textbf{Declaração}}
\setlength{\itemindent}{0cm}
\setlength{\listparindent}{0cm}
\setlength{\leftmargin}{\evensidemargin}
\addtolength{\leftmargin}{\tmplength}
\settowidth{\labelsep}{X}
\addtolength{\leftmargin}{\labelsep}
\setlength{\labelwidth}{\tmplength}
}
\begin{flushleft}
\item[\textbf{Declaração}\hfill]
\begin{ttfamily}
public class function CloseFile(VAR DatF : DataFile):boolean; Overload;\end{ttfamily}


\end{flushleft}
\end{list}
\paragraph*{CloseFile}\hspace*{\fill}

\begin{list}{}{
\settowidth{\tmplength}{\textbf{Declaração}}
\setlength{\itemindent}{0cm}
\setlength{\listparindent}{0cm}
\setlength{\leftmargin}{\evensidemargin}
\addtolength{\leftmargin}{\tmplength}
\settowidth{\labelsep}{X}
\addtolength{\leftmargin}{\labelsep}
\setlength{\labelwidth}{\tmplength}
}
\begin{flushleft}
\item[\textbf{Declaração}\hfill]
\begin{ttfamily}
public class function CloseFile(VAR DatF : DataFile;OkCondicional:Boolean):boolean; Overload;\end{ttfamily}


\end{flushleft}
\end{list}
\paragraph*{FlushFile}\hspace*{\fill}

\begin{list}{}{
\settowidth{\tmplength}{\textbf{Declaração}}
\setlength{\itemindent}{0cm}
\setlength{\listparindent}{0cm}
\setlength{\leftmargin}{\evensidemargin}
\addtolength{\leftmargin}{\tmplength}
\settowidth{\labelsep}{X}
\addtolength{\leftmargin}{\labelsep}
\setlength{\labelwidth}{\tmplength}
}
\begin{flushleft}
\item[\textbf{Declaração}\hfill]
\begin{ttfamily}
public class function FlushFile(VAR DatF : DataFile):Boolean; overload;\end{ttfamily}


\end{flushleft}
\end{list}
\paragraph*{TraveRegistro}\hspace*{\fill}

\begin{list}{}{
\settowidth{\tmplength}{\textbf{Declaração}}
\setlength{\itemindent}{0cm}
\setlength{\listparindent}{0cm}
\setlength{\leftmargin}{\evensidemargin}
\addtolength{\leftmargin}{\tmplength}
\settowidth{\labelsep}{X}
\addtolength{\leftmargin}{\labelsep}
\setlength{\labelwidth}{\tmplength}
}
\begin{flushleft}
\item[\textbf{Declaração}\hfill]
\begin{ttfamily}
public class function TraveRegistro(VAR DatF : DataFile; Const R : LONGINT):Boolean;\end{ttfamily}


\end{flushleft}
\end{list}
\paragraph*{DestraveRegistro}\hspace*{\fill}

\begin{list}{}{
\settowidth{\tmplength}{\textbf{Declaração}}
\setlength{\itemindent}{0cm}
\setlength{\listparindent}{0cm}
\setlength{\leftmargin}{\evensidemargin}
\addtolength{\leftmargin}{\tmplength}
\settowidth{\labelsep}{X}
\addtolength{\leftmargin}{\labelsep}
\setlength{\labelwidth}{\tmplength}
}
\begin{flushleft}
\item[\textbf{Declaração}\hfill]
\begin{ttfamily}
public class function DestraveRegistro(Var DatF : DataFile;Const R : Longint):Boolean;\end{ttfamily}


\end{flushleft}
\end{list}
\paragraph*{TraveHeader}\hspace*{\fill}

\begin{list}{}{
\settowidth{\tmplength}{\textbf{Declaração}}
\setlength{\itemindent}{0cm}
\setlength{\listparindent}{0cm}
\setlength{\leftmargin}{\evensidemargin}
\addtolength{\leftmargin}{\tmplength}
\settowidth{\labelsep}{X}
\addtolength{\leftmargin}{\labelsep}
\setlength{\labelwidth}{\tmplength}
}
\begin{flushleft}
\item[\textbf{Declaração}\hfill]
\begin{ttfamily}
public class function TraveHeader(VAR DatF : DataFile):Boolean;\end{ttfamily}


\end{flushleft}
\end{list}
\paragraph*{DestraveHeader}\hspace*{\fill}

\begin{list}{}{
\settowidth{\tmplength}{\textbf{Declaração}}
\setlength{\itemindent}{0cm}
\setlength{\listparindent}{0cm}
\setlength{\leftmargin}{\evensidemargin}
\addtolength{\leftmargin}{\tmplength}
\settowidth{\labelsep}{X}
\addtolength{\leftmargin}{\labelsep}
\setlength{\labelwidth}{\tmplength}
}
\begin{flushleft}
\item[\textbf{Declaração}\hfill]
\begin{ttfamily}
public class function DestraveHeader(VAR DatF : DataFile):Boolean;\end{ttfamily}


\end{flushleft}
\end{list}
\paragraph*{IoResult}\hspace*{\fill}

\begin{list}{}{
\settowidth{\tmplength}{\textbf{Declaração}}
\setlength{\itemindent}{0cm}
\setlength{\listparindent}{0cm}
\setlength{\leftmargin}{\evensidemargin}
\addtolength{\leftmargin}{\tmplength}
\settowidth{\labelsep}{X}
\addtolength{\leftmargin}{\labelsep}
\setlength{\labelwidth}{\tmplength}
}
\begin{flushleft}
\item[\textbf{Declaração}\hfill]
\begin{ttfamily}
public class function IoResult(Var DatF : DataFile) : Integer; overload;\end{ttfamily}


\end{flushleft}
\end{list}
\paragraph*{FileSize}\hspace*{\fill}

\begin{list}{}{
\settowidth{\tmplength}{\textbf{Declaração}}
\setlength{\itemindent}{0cm}
\setlength{\listparindent}{0cm}
\setlength{\leftmargin}{\evensidemargin}
\addtolength{\leftmargin}{\tmplength}
\settowidth{\labelsep}{X}
\addtolength{\leftmargin}{\labelsep}
\setlength{\labelwidth}{\tmplength}
}
\begin{flushleft}
\item[\textbf{Declaração}\hfill]
\begin{ttfamily}
public class function FileSize(VAR DatF : DataFile):Longint; Overload;\end{ttfamily}


\end{flushleft}
\end{list}
\paragraph*{NewRec}\hspace*{\fill}

\begin{list}{}{
\settowidth{\tmplength}{\textbf{Declaração}}
\setlength{\itemindent}{0cm}
\setlength{\listparindent}{0cm}
\setlength{\leftmargin}{\evensidemargin}
\addtolength{\leftmargin}{\tmplength}
\settowidth{\labelsep}{X}
\addtolength{\leftmargin}{\labelsep}
\setlength{\labelwidth}{\tmplength}
}
\begin{flushleft}
\item[\textbf{Declaração}\hfill]
\begin{ttfamily}
public class Procedure NewRec(VAR DatF : DataFile;VAR R : LONGINT );\end{ttfamily}


\end{flushleft}
\end{list}
\paragraph*{AddRec}\hspace*{\fill}

\begin{list}{}{
\settowidth{\tmplength}{\textbf{Declaração}}
\setlength{\itemindent}{0cm}
\setlength{\listparindent}{0cm}
\setlength{\leftmargin}{\evensidemargin}
\addtolength{\leftmargin}{\tmplength}
\settowidth{\labelsep}{X}
\addtolength{\leftmargin}{\labelsep}
\setlength{\labelwidth}{\tmplength}
}
\begin{flushleft}
\item[\textbf{Declaração}\hfill]
\begin{ttfamily}
public class function AddRec(var DatF : DataFile;var R : Longint;var Buffer ):Boolean; overload;\end{ttfamily}


\end{flushleft}
\end{list}
\paragraph*{DeleteRecord}\hspace*{\fill}

\begin{list}{}{
\settowidth{\tmplength}{\textbf{Declaração}}
\setlength{\itemindent}{0cm}
\setlength{\listparindent}{0cm}
\setlength{\leftmargin}{\evensidemargin}
\addtolength{\leftmargin}{\tmplength}
\settowidth{\labelsep}{X}
\addtolength{\leftmargin}{\labelsep}
\setlength{\labelwidth}{\tmplength}
}
\begin{flushleft}
\item[\textbf{Declaração}\hfill]
\begin{ttfamily}
public class function DeleteRecord(VAR DatF : DataFile;Const R : LONGINT; Var Buffer ):Boolean; overload;\end{ttfamily}


\end{flushleft}
\end{list}
\paragraph*{DeleteRec}\hspace*{\fill}

\begin{list}{}{
\settowidth{\tmplength}{\textbf{Declaração}}
\setlength{\itemindent}{0cm}
\setlength{\listparindent}{0cm}
\setlength{\leftmargin}{\evensidemargin}
\addtolength{\leftmargin}{\tmplength}
\settowidth{\labelsep}{X}
\addtolength{\leftmargin}{\labelsep}
\setlength{\labelwidth}{\tmplength}
}
\begin{flushleft}
\item[\textbf{Declaração}\hfill]
\begin{ttfamily}
public class function DeleteRec(var DatF : DataFile;Const R : Longint):Boolean; overload;\end{ttfamily}


\end{flushleft}
\end{list}
\paragraph*{FileLen}\hspace*{\fill}

\begin{list}{}{
\settowidth{\tmplength}{\textbf{Declaração}}
\setlength{\itemindent}{0cm}
\setlength{\listparindent}{0cm}
\setlength{\leftmargin}{\evensidemargin}
\addtolength{\leftmargin}{\tmplength}
\settowidth{\labelsep}{X}
\addtolength{\leftmargin}{\labelsep}
\setlength{\labelwidth}{\tmplength}
}
\begin{flushleft}
\item[\textbf{Declaração}\hfill]
\begin{ttfamily}
public class function FileLen(VAR DatF : DataFile) : LONGINT; overload;\end{ttfamily}


\end{flushleft}
\end{list}
\paragraph*{UsedRecs}\hspace*{\fill}

\begin{list}{}{
\settowidth{\tmplength}{\textbf{Declaração}}
\setlength{\itemindent}{0cm}
\setlength{\listparindent}{0cm}
\setlength{\leftmargin}{\evensidemargin}
\addtolength{\leftmargin}{\tmplength}
\settowidth{\labelsep}{X}
\addtolength{\leftmargin}{\labelsep}
\setlength{\labelwidth}{\tmplength}
}
\begin{flushleft}
\item[\textbf{Declaração}\hfill]
\begin{ttfamily}
public class function UsedRecs(VAR DatF : DataFile) : LONGINT; Overload;\end{ttfamily}


\end{flushleft}
\end{list}
\paragraph*{UsedRecs}\hspace*{\fill}

\begin{list}{}{
\settowidth{\tmplength}{\textbf{Declaração}}
\setlength{\itemindent}{0cm}
\setlength{\listparindent}{0cm}
\setlength{\leftmargin}{\evensidemargin}
\addtolength{\leftmargin}{\tmplength}
\settowidth{\labelsep}{X}
\addtolength{\leftmargin}{\labelsep}
\setlength{\labelwidth}{\tmplength}
}
\begin{flushleft}
\item[\textbf{Declaração}\hfill]
\begin{ttfamily}
public class function UsedRecs(VAR DatF : DataFile;OK{\_}GetHeader : Boolean) : LONGINT; Overload;\end{ttfamily}


\end{flushleft}
\end{list}
\paragraph*{UsedRecs}\hspace*{\fill}

\begin{list}{}{
\settowidth{\tmplength}{\textbf{Declaração}}
\setlength{\itemindent}{0cm}
\setlength{\listparindent}{0cm}
\setlength{\leftmargin}{\evensidemargin}
\addtolength{\leftmargin}{\tmplength}
\settowidth{\labelsep}{X}
\addtolength{\leftmargin}{\labelsep}
\setlength{\labelwidth}{\tmplength}
}
\begin{flushleft}
\item[\textbf{Declaração}\hfill]
\begin{ttfamily}
public class function UsedRecs(Var IxF :IndexFile;OK{\_}GetHeaderDoIndice : Boolean) : LONGINT; Overload;\end{ttfamily}


\end{flushleft}
\end{list}
\paragraph*{UsedRecs}\hspace*{\fill}

\begin{list}{}{
\settowidth{\tmplength}{\textbf{Declaração}}
\setlength{\itemindent}{0cm}
\setlength{\listparindent}{0cm}
\setlength{\leftmargin}{\evensidemargin}
\addtolength{\leftmargin}{\tmplength}
\settowidth{\labelsep}{X}
\addtolength{\leftmargin}{\labelsep}
\setlength{\labelwidth}{\tmplength}
}
\begin{flushleft}
\item[\textbf{Declaração}\hfill]
\begin{ttfamily}
public class function UsedRecs(Var IxF :IndexFile) : LONGINT; Overload;\end{ttfamily}


\end{flushleft}
\end{list}
\paragraph*{UsedRecs}\hspace*{\fill}

\begin{list}{}{
\settowidth{\tmplength}{\textbf{Declaração}}
\setlength{\itemindent}{0cm}
\setlength{\listparindent}{0cm}
\setlength{\leftmargin}{\evensidemargin}
\addtolength{\leftmargin}{\tmplength}
\settowidth{\labelsep}{X}
\addtolength{\leftmargin}{\labelsep}
\setlength{\labelwidth}{\tmplength}
}
\begin{flushleft}
\item[\textbf{Declaração}\hfill]
\begin{ttfamily}
public class function UsedRecs(Const FileName:PathStr) : Longint; Overload;\end{ttfamily}


\end{flushleft}
\end{list}
\paragraph*{TaPack}\hspace*{\fill}

\begin{list}{}{
\settowidth{\tmplength}{\textbf{Declaração}}
\setlength{\itemindent}{0cm}
\setlength{\listparindent}{0cm}
\setlength{\leftmargin}{\evensidemargin}
\addtolength{\leftmargin}{\tmplength}
\settowidth{\labelsep}{X}
\addtolength{\leftmargin}{\labelsep}
\setlength{\labelwidth}{\tmplength}
}
\begin{flushleft}
\item[\textbf{Declaração}\hfill]
\begin{ttfamily}
public class Procedure TaPack(VAR Page : TaPage;Const KeyL : BYTE);\end{ttfamily}


\end{flushleft}
\end{list}
\paragraph*{TaUnpack}\hspace*{\fill}

\begin{list}{}{
\settowidth{\tmplength}{\textbf{Declaração}}
\setlength{\itemindent}{0cm}
\setlength{\listparindent}{0cm}
\setlength{\leftmargin}{\evensidemargin}
\addtolength{\leftmargin}{\tmplength}
\settowidth{\labelsep}{X}
\addtolength{\leftmargin}{\labelsep}
\setlength{\labelwidth}{\tmplength}
}
\begin{flushleft}
\item[\textbf{Declaração}\hfill]
\begin{ttfamily}
public class Procedure TaUnpack(VAR Page : TaPage; Const KeyL : BYTE);\end{ttfamily}


\end{flushleft}
\end{list}
\paragraph*{Multiplo{\_}Mais{\_}proximo{\_}de{\_}N}\hspace*{\fill}

\begin{list}{}{
\settowidth{\tmplength}{\textbf{Declaração}}
\setlength{\itemindent}{0cm}
\setlength{\listparindent}{0cm}
\setlength{\leftmargin}{\evensidemargin}
\addtolength{\leftmargin}{\tmplength}
\settowidth{\labelsep}{X}
\addtolength{\leftmargin}{\labelsep}
\setlength{\labelwidth}{\tmplength}
}
\begin{flushleft}
\item[\textbf{Declaração}\hfill]
\begin{ttfamily}
public class function Multiplo{\_}Mais{\_}proximo{\_}de{\_}N(Const K,N:Longint): Longint;\end{ttfamily}


\end{flushleft}
\end{list}
\paragraph*{MakeIndex}\hspace*{\fill}

\begin{list}{}{
\settowidth{\tmplength}{\textbf{Declaração}}
\setlength{\itemindent}{0cm}
\setlength{\listparindent}{0cm}
\setlength{\leftmargin}{\evensidemargin}
\addtolength{\leftmargin}{\tmplength}
\settowidth{\labelsep}{X}
\addtolength{\leftmargin}{\labelsep}
\setlength{\labelwidth}{\tmplength}
}
\begin{flushleft}
\item[\textbf{Declaração}\hfill]
\begin{ttfamily}
public class function MakeIndex(var IxF : IndexFile;Const FName : FileName;Const KeyLen,S : byte):Integer; overload;\end{ttfamily}


\end{flushleft}
\end{list}
\paragraph*{OpenIndex}\hspace*{\fill}

\begin{list}{}{
\settowidth{\tmplength}{\textbf{Declaração}}
\setlength{\itemindent}{0cm}
\setlength{\listparindent}{0cm}
\setlength{\leftmargin}{\evensidemargin}
\addtolength{\leftmargin}{\tmplength}
\settowidth{\labelsep}{X}
\addtolength{\leftmargin}{\labelsep}
\setlength{\labelwidth}{\tmplength}
}
\begin{flushleft}
\item[\textbf{Declaração}\hfill]
\begin{ttfamily}
public class function OpenIndex(var IxF : IndexFile;Const FName : FileName;Const KeyLen,S : byte):Integer; overload;\end{ttfamily}


\end{flushleft}
\end{list}
\paragraph*{LeiaHeaderDoIndice}\hspace*{\fill}

\begin{list}{}{
\settowidth{\tmplength}{\textbf{Declaração}}
\setlength{\itemindent}{0cm}
\setlength{\listparindent}{0cm}
\setlength{\leftmargin}{\evensidemargin}
\addtolength{\leftmargin}{\tmplength}
\settowidth{\labelsep}{X}
\addtolength{\leftmargin}{\labelsep}
\setlength{\labelwidth}{\tmplength}
}
\begin{flushleft}
\item[\textbf{Declaração}\hfill]
\begin{ttfamily}
public class Procedure LeiaHeaderDoIndice( VAR IxF : IndexFile);\end{ttfamily}


\end{flushleft}
\end{list}
\paragraph*{aCloseIndex}\hspace*{\fill}

\begin{list}{}{
\settowidth{\tmplength}{\textbf{Declaração}}
\setlength{\itemindent}{0cm}
\setlength{\listparindent}{0cm}
\setlength{\leftmargin}{\evensidemargin}
\addtolength{\leftmargin}{\tmplength}
\settowidth{\labelsep}{X}
\addtolength{\leftmargin}{\labelsep}
\setlength{\labelwidth}{\tmplength}
}
\begin{flushleft}
\item[\textbf{Declaração}\hfill]
\begin{ttfamily}
public class function aCloseIndex(VAR IxF : IndexFile):Boolean; overload;\end{ttfamily}


\end{flushleft}
\end{list}
\paragraph*{CloseIndex}\hspace*{\fill}

\begin{list}{}{
\settowidth{\tmplength}{\textbf{Declaração}}
\setlength{\itemindent}{0cm}
\setlength{\listparindent}{0cm}
\setlength{\leftmargin}{\evensidemargin}
\addtolength{\leftmargin}{\tmplength}
\settowidth{\labelsep}{X}
\addtolength{\leftmargin}{\labelsep}
\setlength{\labelwidth}{\tmplength}
}
\begin{flushleft}
\item[\textbf{Declaração}\hfill]
\begin{ttfamily}
public class function CloseIndex(VAR IxF : IndexFile):boolean; Overload;\end{ttfamily}


\end{flushleft}
\end{list}
\paragraph*{CloseIndex}\hspace*{\fill}

\begin{list}{}{
\settowidth{\tmplength}{\textbf{Declaração}}
\setlength{\itemindent}{0cm}
\setlength{\listparindent}{0cm}
\setlength{\leftmargin}{\evensidemargin}
\addtolength{\leftmargin}{\tmplength}
\settowidth{\labelsep}{X}
\addtolength{\leftmargin}{\labelsep}
\setlength{\labelwidth}{\tmplength}
}
\begin{flushleft}
\item[\textbf{Declaração}\hfill]
\begin{ttfamily}
public class function CloseIndex(VAR IxF : IndexFile;OkCondicional:Boolean):boolean; Overload;\end{ttfamily}


\end{flushleft}
\end{list}
\paragraph*{FlushIndex}\hspace*{\fill}

\begin{list}{}{
\settowidth{\tmplength}{\textbf{Declaração}}
\setlength{\itemindent}{0cm}
\setlength{\listparindent}{0cm}
\setlength{\leftmargin}{\evensidemargin}
\addtolength{\leftmargin}{\tmplength}
\settowidth{\labelsep}{X}
\addtolength{\leftmargin}{\labelsep}
\setlength{\labelwidth}{\tmplength}
}
\begin{flushleft}
\item[\textbf{Declaração}\hfill]
\begin{ttfamily}
public class function FlushIndex(VAR IxF : IndexFile):boolean; overload;\end{ttfamily}


\end{flushleft}
\end{list}
\paragraph*{EraseFile}\hspace*{\fill}

\begin{list}{}{
\settowidth{\tmplength}{\textbf{Declaração}}
\setlength{\itemindent}{0cm}
\setlength{\listparindent}{0cm}
\setlength{\leftmargin}{\evensidemargin}
\addtolength{\leftmargin}{\tmplength}
\settowidth{\labelsep}{X}
\addtolength{\leftmargin}{\labelsep}
\setlength{\labelwidth}{\tmplength}
}
\begin{flushleft}
\item[\textbf{Declaração}\hfill]
\begin{ttfamily}
public class function EraseFile(VAR DatF : DataFile):boolean; overload;\end{ttfamily}


\end{flushleft}
\end{list}
\paragraph*{EraseIndex}\hspace*{\fill}

\begin{list}{}{
\settowidth{\tmplength}{\textbf{Declaração}}
\setlength{\itemindent}{0cm}
\setlength{\listparindent}{0cm}
\setlength{\leftmargin}{\evensidemargin}
\addtolength{\leftmargin}{\tmplength}
\settowidth{\labelsep}{X}
\addtolength{\leftmargin}{\labelsep}
\setlength{\labelwidth}{\tmplength}
}
\begin{flushleft}
\item[\textbf{Declaração}\hfill]
\begin{ttfamily}
public class function EraseIndex(VAR IxF : IndexFile):boolean; overload;\end{ttfamily}


\end{flushleft}
\end{list}
\paragraph*{TaGetPage}\hspace*{\fill}

\begin{list}{}{
\settowidth{\tmplength}{\textbf{Declaração}}
\setlength{\itemindent}{0cm}
\setlength{\listparindent}{0cm}
\setlength{\leftmargin}{\evensidemargin}
\addtolength{\leftmargin}{\tmplength}
\settowidth{\labelsep}{X}
\addtolength{\leftmargin}{\labelsep}
\setlength{\labelwidth}{\tmplength}
}
\begin{flushleft}
\item[\textbf{Declaração}\hfill]
\begin{ttfamily}
public class function TaGetPage(VAR IxF : IndexFile;Const R : LONGINT;VAR PgPtr : TaPagePtr):boolean;\end{ttfamily}


\end{flushleft}
\end{list}
\paragraph*{TaNewPage}\hspace*{\fill}

\begin{list}{}{
\settowidth{\tmplength}{\textbf{Declaração}}
\setlength{\itemindent}{0cm}
\setlength{\listparindent}{0cm}
\setlength{\leftmargin}{\evensidemargin}
\addtolength{\leftmargin}{\tmplength}
\settowidth{\labelsep}{X}
\addtolength{\leftmargin}{\labelsep}
\setlength{\labelwidth}{\tmplength}
}
\begin{flushleft}
\item[\textbf{Declaração}\hfill]
\begin{ttfamily}
public class Procedure TaNewPage(VAR IxF : IndexFile; VAR R : LONGINT; VAR PgPtr : TaPagePtr);\end{ttfamily}


\end{flushleft}
\end{list}
\paragraph*{TaDeletePage}\hspace*{\fill}

\begin{list}{}{
\settowidth{\tmplength}{\textbf{Declaração}}
\setlength{\itemindent}{0cm}
\setlength{\listparindent}{0cm}
\setlength{\leftmargin}{\evensidemargin}
\addtolength{\leftmargin}{\tmplength}
\settowidth{\labelsep}{X}
\addtolength{\leftmargin}{\labelsep}
\setlength{\labelwidth}{\tmplength}
}
\begin{flushleft}
\item[\textbf{Declaração}\hfill]
\begin{ttfamily}
public class Procedure TaDeletePage(var IxF : IndexFile; VAR R : LONGINT; VAR PgPtr : TaPagePtr);\end{ttfamily}


\end{flushleft}
\end{list}
\paragraph*{ClearKey}\hspace*{\fill}

\begin{list}{}{
\settowidth{\tmplength}{\textbf{Declaração}}
\setlength{\itemindent}{0cm}
\setlength{\listparindent}{0cm}
\setlength{\leftmargin}{\evensidemargin}
\addtolength{\leftmargin}{\tmplength}
\settowidth{\labelsep}{X}
\addtolength{\leftmargin}{\labelsep}
\setlength{\labelwidth}{\tmplength}
}
\begin{flushleft}
\item[\textbf{Declaração}\hfill]
\begin{ttfamily}
public class Procedure ClearKey(VAR IxF : IndexFile); overload;\end{ttfamily}


\end{flushleft}
\end{list}
\paragraph*{NextKey}\hspace*{\fill}

\begin{list}{}{
\settowidth{\tmplength}{\textbf{Declaração}}
\setlength{\itemindent}{0cm}
\setlength{\listparindent}{0cm}
\setlength{\leftmargin}{\evensidemargin}
\addtolength{\leftmargin}{\tmplength}
\settowidth{\labelsep}{X}
\addtolength{\leftmargin}{\labelsep}
\setlength{\labelwidth}{\tmplength}
}
\begin{flushleft}
\item[\textbf{Declaração}\hfill]
\begin{ttfamily}
public class function NextKey(VAR IxF : IndexFile; VAR DataRecNum : LONGINT; VAR ProcKey ):Boolean; overload;\end{ttfamily}


\end{flushleft}
\end{list}
\paragraph*{PrevKey}\hspace*{\fill}

\begin{list}{}{
\settowidth{\tmplength}{\textbf{Declaração}}
\setlength{\itemindent}{0cm}
\setlength{\listparindent}{0cm}
\setlength{\leftmargin}{\evensidemargin}
\addtolength{\leftmargin}{\tmplength}
\settowidth{\labelsep}{X}
\addtolength{\leftmargin}{\labelsep}
\setlength{\labelwidth}{\tmplength}
}
\begin{flushleft}
\item[\textbf{Declaração}\hfill]
\begin{ttfamily}
public class function PrevKey(var IxF : IndexFile; var DataRecNum : Longint; var ProcKey ):Boolean; overload;\end{ttfamily}


\end{flushleft}
\end{list}
\paragraph*{TaXKey}\hspace*{\fill}

\begin{list}{}{
\settowidth{\tmplength}{\textbf{Declaração}}
\setlength{\itemindent}{0cm}
\setlength{\listparindent}{0cm}
\setlength{\leftmargin}{\evensidemargin}
\addtolength{\leftmargin}{\tmplength}
\settowidth{\labelsep}{X}
\addtolength{\leftmargin}{\labelsep}
\setlength{\labelwidth}{\tmplength}
}
\begin{flushleft}
\item[\textbf{Declaração}\hfill]
\begin{ttfamily}
public class Procedure TaXKey(VAR K:TaKeyStr; Const KeyL : BYTE);\end{ttfamily}


\end{flushleft}
\end{list}
\paragraph*{TaCompKeys}\hspace*{\fill}

\begin{list}{}{
\settowidth{\tmplength}{\textbf{Declaração}}
\setlength{\itemindent}{0cm}
\setlength{\listparindent}{0cm}
\setlength{\leftmargin}{\evensidemargin}
\addtolength{\leftmargin}{\tmplength}
\settowidth{\labelsep}{X}
\addtolength{\leftmargin}{\labelsep}
\setlength{\labelwidth}{\tmplength}
}
\begin{flushleft}
\item[\textbf{Declaração}\hfill]
\begin{ttfamily}
public class function TaCompKeys(Const K1 ,K2; DR1,DR2 : LONGINT; Const Dup : BOOLEAN ) : Shortint;\end{ttfamily}


\end{flushleft}
\end{list}
\paragraph*{TaFindKey}\hspace*{\fill}

\begin{list}{}{
\settowidth{\tmplength}{\textbf{Declaração}}
\setlength{\itemindent}{0cm}
\setlength{\listparindent}{0cm}
\setlength{\leftmargin}{\evensidemargin}
\addtolength{\leftmargin}{\tmplength}
\settowidth{\labelsep}{X}
\addtolength{\leftmargin}{\labelsep}
\setlength{\labelwidth}{\tmplength}
}
\begin{flushleft}
\item[\textbf{Declaração}\hfill]
\begin{ttfamily}
public class function TaFindKey(VAR IxF : IndexFile;VAR DataRecNum : LONGINT;VAR ProcKey ):boolean;\end{ttfamily}


\end{flushleft}
\end{list}
\paragraph*{FindKey}\hspace*{\fill}

\begin{list}{}{
\settowidth{\tmplength}{\textbf{Declaração}}
\setlength{\itemindent}{0cm}
\setlength{\listparindent}{0cm}
\setlength{\leftmargin}{\evensidemargin}
\addtolength{\leftmargin}{\tmplength}
\settowidth{\labelsep}{X}
\addtolength{\leftmargin}{\labelsep}
\setlength{\labelwidth}{\tmplength}
}
\begin{flushleft}
\item[\textbf{Declaração}\hfill]
\begin{ttfamily}
public class function FindKey(var IxF : IndexFile;var DataRecNum : Longint;var ProcKey ):Boolean; overload;\end{ttfamily}


\end{flushleft}
\end{list}
\paragraph*{FindKeyTop}\hspace*{\fill}

\begin{list}{}{
\settowidth{\tmplength}{\textbf{Declaração}}
\setlength{\itemindent}{0cm}
\setlength{\listparindent}{0cm}
\setlength{\leftmargin}{\evensidemargin}
\addtolength{\leftmargin}{\tmplength}
\settowidth{\labelsep}{X}
\addtolength{\leftmargin}{\labelsep}
\setlength{\labelwidth}{\tmplength}
}
\begin{flushleft}
\item[\textbf{Declaração}\hfill]
\begin{ttfamily}
public class function FindKeyTop(var IxF : IndexFile;var DataRecNum : Longint;var ProcKey ):Boolean; overload;\end{ttfamily}


\end{flushleft}
\end{list}
\paragraph*{SearchKey}\hspace*{\fill}

\begin{list}{}{
\settowidth{\tmplength}{\textbf{Declaração}}
\setlength{\itemindent}{0cm}
\setlength{\listparindent}{0cm}
\setlength{\leftmargin}{\evensidemargin}
\addtolength{\leftmargin}{\tmplength}
\settowidth{\labelsep}{X}
\addtolength{\leftmargin}{\labelsep}
\setlength{\labelwidth}{\tmplength}
}
\begin{flushleft}
\item[\textbf{Declaração}\hfill]
\begin{ttfamily}
public class function SearchKey(var IxF : IndexFile; var DataRecNum : Longint; var ProcKey:TaKeyStr):Boolean; overload;\end{ttfamily}


\end{flushleft}
\end{list}
\paragraph*{SearchKeyTop}\hspace*{\fill}

\begin{list}{}{
\settowidth{\tmplength}{\textbf{Declaração}}
\setlength{\itemindent}{0cm}
\setlength{\listparindent}{0cm}
\setlength{\leftmargin}{\evensidemargin}
\addtolength{\leftmargin}{\tmplength}
\settowidth{\labelsep}{X}
\addtolength{\leftmargin}{\labelsep}
\setlength{\labelwidth}{\tmplength}
}
\begin{flushleft}
\item[\textbf{Declaração}\hfill]
\begin{ttfamily}
public class function SearchKeyTop(var IxF : IndexFile; var DataRecNum : Longint; var ProcKey:TaKeyStr;Const Okequal : Boolean):Boolean; overload;\end{ttfamily}


\end{flushleft}
\end{list}
\paragraph*{TaUpdatePage}\hspace*{\fill}

\begin{list}{}{
\settowidth{\tmplength}{\textbf{Declaração}}
\setlength{\itemindent}{0cm}
\setlength{\listparindent}{0cm}
\setlength{\leftmargin}{\evensidemargin}
\addtolength{\leftmargin}{\tmplength}
\settowidth{\labelsep}{X}
\addtolength{\leftmargin}{\labelsep}
\setlength{\labelwidth}{\tmplength}
}
\begin{flushleft}
\item[\textbf{Declaração}\hfill]
\begin{ttfamily}
public class Procedure TaUpdatePage(VAR IxF : IndexFile; VAR R : LONGINT; VAR PgPtr : TaPagePtr; Const Transaction{\_}Current : T{\_}TTransaction); overload;\end{ttfamily}


\end{flushleft}
\end{list}
\paragraph*{AddKey{\_}Search{\_}Insert}\hspace*{\fill}

\begin{list}{}{
\settowidth{\tmplength}{\textbf{Declaração}}
\setlength{\itemindent}{0cm}
\setlength{\listparindent}{0cm}
\setlength{\leftmargin}{\evensidemargin}
\addtolength{\leftmargin}{\tmplength}
\settowidth{\labelsep}{X}
\addtolength{\leftmargin}{\labelsep}
\setlength{\labelwidth}{\tmplength}
}
\begin{flushleft}
\item[\textbf{Declaração}\hfill]
\begin{ttfamily}
public class Procedure AddKey{\_}Search{\_}Insert( var IxF : IndexFile; Var PrPgRef1 : LONGINT; VAR PrPgRef2,c : LONGINT; VAR PagePtr1,PagePtr2 : TaPagePtr; VAR ProcItem1, ProcItem2 : TaItem; vAR PassUp, okAddKey : BOOLEAN; Const ProcKey : TaKeyStr; Const DataRecNum : Longint; VAR K,L : SmallInt; Var R : SmallInt );\end{ttfamily}


\end{flushleft}
\end{list}
\paragraph*{AddKey{\_}Search{\_}Init{\_}ProcItem1}\hspace*{\fill}

\begin{list}{}{
\settowidth{\tmplength}{\textbf{Declaração}}
\setlength{\itemindent}{0cm}
\setlength{\listparindent}{0cm}
\setlength{\leftmargin}{\evensidemargin}
\addtolength{\leftmargin}{\tmplength}
\settowidth{\labelsep}{X}
\addtolength{\leftmargin}{\labelsep}
\setlength{\labelwidth}{\tmplength}
}
\begin{flushleft}
\item[\textbf{Declaração}\hfill]
\begin{ttfamily}
public class Procedure AddKey{\_}Search{\_}Init{\_}ProcItem1(Const ProcKey : TaKeyStr; Const DataRecNum : Longint; vAR PassUp : BOOLEAN; VAR ProcItem1 : TaItem);\end{ttfamily}


\end{flushleft}
\end{list}
\paragraph*{AddKey{\_}Search}\hspace*{\fill}

\begin{list}{}{
\settowidth{\tmplength}{\textbf{Declaração}}
\setlength{\itemindent}{0cm}
\setlength{\listparindent}{0cm}
\setlength{\leftmargin}{\evensidemargin}
\addtolength{\leftmargin}{\tmplength}
\settowidth{\labelsep}{X}
\addtolength{\leftmargin}{\labelsep}
\setlength{\labelwidth}{\tmplength}
}
\begin{flushleft}
\item[\textbf{Declaração}\hfill]
\begin{ttfamily}
public class Procedure AddKey{\_}Search(var IxF : IndexFile; PrPgRef1 : LONGINT; VAR PrPgRef2,c : LONGINT; VAR PagePtr1,PagePtr2 : TaPagePtr; VAR ProcItem1, ProcItem2 : TaItem; vAR PassUp, okAddKey : BOOLEAN; Const ProcKey : TaKeyStr; Const DataRecNum : Longint; VAR K,L : SmallInt );\end{ttfamily}


\end{flushleft}
\end{list}
\paragraph*{AddKey}\hspace*{\fill}

\begin{list}{}{
\settowidth{\tmplength}{\textbf{Declaração}}
\setlength{\itemindent}{0cm}
\setlength{\listparindent}{0cm}
\setlength{\leftmargin}{\evensidemargin}
\addtolength{\leftmargin}{\tmplength}
\settowidth{\labelsep}{X}
\addtolength{\leftmargin}{\labelsep}
\setlength{\labelwidth}{\tmplength}
}
\begin{flushleft}
\item[\textbf{Declaração}\hfill]
\begin{ttfamily}
public class function AddKey(var IxF : IndexFile; Const DataRecNum : Longint; Const ProcKey : TaKeyStr):Boolean; overload;\end{ttfamily}


\end{flushleft}
\end{list}
\paragraph*{DeleteKey}\hspace*{\fill}

\begin{list}{}{
\settowidth{\tmplength}{\textbf{Declaração}}
\setlength{\itemindent}{0cm}
\setlength{\listparindent}{0cm}
\setlength{\leftmargin}{\evensidemargin}
\addtolength{\leftmargin}{\tmplength}
\settowidth{\labelsep}{X}
\addtolength{\leftmargin}{\labelsep}
\setlength{\labelwidth}{\tmplength}
}
\begin{flushleft}
\item[\textbf{Declaração}\hfill]
\begin{ttfamily}
public class function DeleteKey(var IxF : IndexFile;Const DataRecNum : Longint;var ProcKey:TaKeyStr ):Boolean; OVERLOAD;\end{ttfamily}


\end{flushleft}
\end{list}
\paragraph*{FGetHeaderDataFile}\hspace*{\fill}

\begin{list}{}{
\settowidth{\tmplength}{\textbf{Declaração}}
\setlength{\itemindent}{0cm}
\setlength{\listparindent}{0cm}
\setlength{\leftmargin}{\evensidemargin}
\addtolength{\leftmargin}{\tmplength}
\settowidth{\labelsep}{X}
\addtolength{\leftmargin}{\labelsep}
\setlength{\labelwidth}{\tmplength}
}
\begin{flushleft}
\item[\textbf{Declaração}\hfill]
\begin{ttfamily}
public class function FGetHeaderDataFile(Const FileName: PathStr;Var Header : TsImagemHeader;Var aFileSize : Longint):Boolean;\end{ttfamily}


\end{flushleft}
\end{list}
\paragraph*{FTamRegDataFile}\hspace*{\fill}

\begin{list}{}{
\settowidth{\tmplength}{\textbf{Declaração}}
\setlength{\itemindent}{0cm}
\setlength{\listparindent}{0cm}
\setlength{\leftmargin}{\evensidemargin}
\addtolength{\leftmargin}{\tmplength}
\settowidth{\labelsep}{X}
\addtolength{\leftmargin}{\labelsep}
\setlength{\labelwidth}{\tmplength}
}
\begin{flushleft}
\item[\textbf{Declaração}\hfill]
\begin{ttfamily}
public class function FTamRegDataFile(Const FileName: PathStr):SmallWord;\end{ttfamily}


\end{flushleft}
\end{list}
\paragraph*{NewFileName}\hspace*{\fill}

\begin{list}{}{
\settowidth{\tmplength}{\textbf{Declaração}}
\setlength{\itemindent}{0cm}
\setlength{\listparindent}{0cm}
\setlength{\leftmargin}{\evensidemargin}
\addtolength{\leftmargin}{\tmplength}
\settowidth{\labelsep}{X}
\addtolength{\leftmargin}{\labelsep}
\setlength{\labelwidth}{\tmplength}
}
\begin{flushleft}
\item[\textbf{Declaração}\hfill]
\begin{ttfamily}
public class function NewFileName(FileName,Extencao:PathStr):PathStr;\end{ttfamily}


\end{flushleft}
\end{list}
\paragraph*{FTb}\hspace*{\fill}

\begin{list}{}{
\settowidth{\tmplength}{\textbf{Declaração}}
\setlength{\itemindent}{0cm}
\setlength{\listparindent}{0cm}
\setlength{\leftmargin}{\evensidemargin}
\addtolength{\leftmargin}{\tmplength}
\settowidth{\labelsep}{X}
\addtolength{\leftmargin}{\labelsep}
\setlength{\labelwidth}{\tmplength}
}
\begin{flushleft}
\item[\textbf{Declaração}\hfill]
\begin{ttfamily}
public class function FTb(Const FileName:PathStr):PathStr;\end{ttfamily}


\end{flushleft}
\end{list}
\paragraph*{FObj}\hspace*{\fill}

\begin{list}{}{
\settowidth{\tmplength}{\textbf{Declaração}}
\setlength{\itemindent}{0cm}
\setlength{\listparindent}{0cm}
\setlength{\leftmargin}{\evensidemargin}
\addtolength{\leftmargin}{\tmplength}
\settowidth{\labelsep}{X}
\addtolength{\leftmargin}{\labelsep}
\setlength{\labelwidth}{\tmplength}
}
\begin{flushleft}
\item[\textbf{Declaração}\hfill]
\begin{ttfamily}
public class function FObj(Const FileName:PathStr):PathStr;\end{ttfamily}


\end{flushleft}
\end{list}
\paragraph*{FIx}\hspace*{\fill}

\begin{list}{}{
\settowidth{\tmplength}{\textbf{Declaração}}
\setlength{\itemindent}{0cm}
\setlength{\listparindent}{0cm}
\setlength{\leftmargin}{\evensidemargin}
\addtolength{\leftmargin}{\tmplength}
\settowidth{\labelsep}{X}
\addtolength{\leftmargin}{\labelsep}
\setlength{\labelwidth}{\tmplength}
}
\begin{flushleft}
\item[\textbf{Declaração}\hfill]
\begin{ttfamily}
public class function FIx(Const FileName:PathStr):PathStr;\end{ttfamily}


\end{flushleft}
\end{list}
\paragraph*{FDup}\hspace*{\fill}

\begin{list}{}{
\settowidth{\tmplength}{\textbf{Declaração}}
\setlength{\itemindent}{0cm}
\setlength{\listparindent}{0cm}
\setlength{\leftmargin}{\evensidemargin}
\addtolength{\leftmargin}{\tmplength}
\settowidth{\labelsep}{X}
\addtolength{\leftmargin}{\labelsep}
\setlength{\labelwidth}{\tmplength}
}
\begin{flushleft}
\item[\textbf{Declaração}\hfill]
\begin{ttfamily}
public class function FDup(Const FileName:PathStr):PathStr;\end{ttfamily}


\end{flushleft}
\end{list}
\paragraph*{AssignDataFile}\hspace*{\fill}

\begin{list}{}{
\settowidth{\tmplength}{\textbf{Declaração}}
\setlength{\itemindent}{0cm}
\setlength{\listparindent}{0cm}
\setlength{\leftmargin}{\evensidemargin}
\addtolength{\leftmargin}{\tmplength}
\settowidth{\labelsep}{X}
\addtolength{\leftmargin}{\labelsep}
\setlength{\labelwidth}{\tmplength}
}
\begin{flushleft}
\item[\textbf{Declaração}\hfill]
\begin{ttfamily}
public class Procedure AssignDataFile(Var DatF :DataFile; Const aFileName:PathStr; aBaseSize, aRecSize:SmallWord; aF :TStream; WTipo : AnsiChar ); Overload;\end{ttfamily}


\end{flushleft}
\end{list}
\paragraph*{AssignDataFile}\hspace*{\fill}

\begin{list}{}{
\settowidth{\tmplength}{\textbf{Declaração}}
\setlength{\itemindent}{0cm}
\setlength{\listparindent}{0cm}
\setlength{\leftmargin}{\evensidemargin}
\addtolength{\leftmargin}{\tmplength}
\settowidth{\labelsep}{X}
\addtolength{\leftmargin}{\labelsep}
\setlength{\labelwidth}{\tmplength}
}
\begin{flushleft}
\item[\textbf{Declaração}\hfill]
\begin{ttfamily}
public class Procedure AssignDataFile(Var DatF :DataFile; Const aFileName:PathStr; aBaseSize, aRecSize:SmallWord); Overload;\end{ttfamily}


\end{flushleft}
\end{list}
\paragraph*{AssignIndexFile}\hspace*{\fill}

\begin{list}{}{
\settowidth{\tmplength}{\textbf{Declaração}}
\setlength{\itemindent}{0cm}
\setlength{\listparindent}{0cm}
\setlength{\leftmargin}{\evensidemargin}
\addtolength{\leftmargin}{\tmplength}
\settowidth{\labelsep}{X}
\addtolength{\leftmargin}{\labelsep}
\setlength{\labelwidth}{\tmplength}
}
\begin{flushleft}
\item[\textbf{Declaração}\hfill]
\begin{ttfamily}
public class Procedure AssignIndexFile(Var IxF : IndexFile; Const aFileName : PathStr; aBaseSize, aRecSize : SmallWord); Overload;\end{ttfamily}


\end{flushleft}
\end{list}
\paragraph*{UpperCase}\hspace*{\fill}

\begin{list}{}{
\settowidth{\tmplength}{\textbf{Declaração}}
\setlength{\itemindent}{0cm}
\setlength{\listparindent}{0cm}
\setlength{\leftmargin}{\evensidemargin}
\addtolength{\leftmargin}{\tmplength}
\settowidth{\labelsep}{X}
\addtolength{\leftmargin}{\labelsep}
\setlength{\labelwidth}{\tmplength}
}
\begin{flushleft}
\item[\textbf{Declaração}\hfill]
\begin{ttfamily}
public class function UpperCase(str:AnsiString):AnsiString;\end{ttfamily}


\end{flushleft}
\end{list}
\paragraph*{FMinuscula}\hspace*{\fill}

\begin{list}{}{
\settowidth{\tmplength}{\textbf{Declaração}}
\setlength{\itemindent}{0cm}
\setlength{\listparindent}{0cm}
\setlength{\leftmargin}{\evensidemargin}
\addtolength{\leftmargin}{\tmplength}
\settowidth{\labelsep}{X}
\addtolength{\leftmargin}{\labelsep}
\setlength{\labelwidth}{\tmplength}
}
\begin{flushleft}
\item[\textbf{Declaração}\hfill]
\begin{ttfamily}
public class function FMinuscula(str:AnsiString):AnsiString;\end{ttfamily}


\end{flushleft}
\end{list}
\paragraph*{Int2str}\hspace*{\fill}

\begin{list}{}{
\settowidth{\tmplength}{\textbf{Declaração}}
\setlength{\itemindent}{0cm}
\setlength{\listparindent}{0cm}
\setlength{\leftmargin}{\evensidemargin}
\addtolength{\leftmargin}{\tmplength}
\settowidth{\labelsep}{X}
\addtolength{\leftmargin}{\labelsep}
\setlength{\labelwidth}{\tmplength}
}
\begin{flushleft}
\item[\textbf{Declaração}\hfill]
\begin{ttfamily}
public class function Int2str(Const L : LongInt) : TTb{\_}Access.tString;\end{ttfamily}


\end{flushleft}
\end{list}
\paragraph*{Beep}\hspace*{\fill}

\begin{list}{}{
\settowidth{\tmplength}{\textbf{Declaração}}
\setlength{\itemindent}{0cm}
\setlength{\listparindent}{0cm}
\setlength{\leftmargin}{\evensidemargin}
\addtolength{\leftmargin}{\tmplength}
\settowidth{\labelsep}{X}
\addtolength{\leftmargin}{\labelsep}
\setlength{\labelwidth}{\tmplength}
}
\begin{flushleft}
\item[\textbf{Declaração}\hfill]
\begin{ttfamily}
public class procedure Beep;\end{ttfamily}


\end{flushleft}
\end{list}
\paragraph*{spc}\hspace*{\fill}

\begin{list}{}{
\settowidth{\tmplength}{\textbf{Declaração}}
\setlength{\itemindent}{0cm}
\setlength{\listparindent}{0cm}
\setlength{\leftmargin}{\evensidemargin}
\addtolength{\leftmargin}{\tmplength}
\settowidth{\labelsep}{X}
\addtolength{\leftmargin}{\labelsep}
\setlength{\labelwidth}{\tmplength}
}
\begin{flushleft}
\item[\textbf{Declaração}\hfill]
\begin{ttfamily}
public class function spc(Const campo:AnsiString;Const tam :Longint):AnsiString;\end{ttfamily}


\end{flushleft}
\end{list}
\paragraph*{SetOkTransaction}\hspace*{\fill}

\begin{list}{}{
\settowidth{\tmplength}{\textbf{Declaração}}
\setlength{\itemindent}{0cm}
\setlength{\listparindent}{0cm}
\setlength{\leftmargin}{\evensidemargin}
\addtolength{\leftmargin}{\tmplength}
\settowidth{\labelsep}{X}
\addtolength{\leftmargin}{\labelsep}
\setlength{\labelwidth}{\tmplength}
}
\begin{flushleft}
\item[\textbf{Declaração}\hfill]
\begin{ttfamily}
public class function SetOkTransaction(Const aOkTransaction : BOOLEAN):BOOLEAN;\end{ttfamily}


\end{flushleft}
\end{list}
\paragraph*{StartTransaction}\hspace*{\fill}

\begin{list}{}{
\settowidth{\tmplength}{\textbf{Declaração}}
\setlength{\itemindent}{0cm}
\setlength{\listparindent}{0cm}
\setlength{\leftmargin}{\evensidemargin}
\addtolength{\leftmargin}{\tmplength}
\settowidth{\labelsep}{X}
\addtolength{\leftmargin}{\labelsep}
\setlength{\labelwidth}{\tmplength}
}
\begin{flushleft}
\item[\textbf{Declaração}\hfill]
\begin{ttfamily}
public class function StartTransaction(Const aDelta : SmallWord):Integer; Overload;\end{ttfamily}


\end{flushleft}
\end{list}
\paragraph*{StartTransaction}\hspace*{\fill}

\begin{list}{}{
\settowidth{\tmplength}{\textbf{Declaração}}
\setlength{\itemindent}{0cm}
\setlength{\listparindent}{0cm}
\setlength{\leftmargin}{\evensidemargin}
\addtolength{\leftmargin}{\tmplength}
\settowidth{\labelsep}{X}
\addtolength{\leftmargin}{\labelsep}
\setlength{\labelwidth}{\tmplength}
}
\begin{flushleft}
\item[\textbf{Declaração}\hfill]
\begin{ttfamily}
public class function StartTransaction(Const DatF : DataFile ; Var aok{\_}Set{\_}Transaction : Boolean): Integer; Overload;\end{ttfamily}


\end{flushleft}
\end{list}
\paragraph*{COMMIT}\hspace*{\fill}

\begin{list}{}{
\settowidth{\tmplength}{\textbf{Declaração}}
\setlength{\itemindent}{0cm}
\setlength{\listparindent}{0cm}
\setlength{\leftmargin}{\evensidemargin}
\addtolength{\leftmargin}{\tmplength}
\settowidth{\labelsep}{X}
\addtolength{\leftmargin}{\labelsep}
\setlength{\labelwidth}{\tmplength}
}
\begin{flushleft}
\item[\textbf{Declaração}\hfill]
\begin{ttfamily}
public class function COMMIT:Boolean; Overload;\end{ttfamily}


\end{flushleft}
\end{list}
\paragraph*{COMMIT}\hspace*{\fill}

\begin{list}{}{
\settowidth{\tmplength}{\textbf{Declaração}}
\setlength{\itemindent}{0cm}
\setlength{\listparindent}{0cm}
\setlength{\leftmargin}{\evensidemargin}
\addtolength{\leftmargin}{\tmplength}
\settowidth{\labelsep}{X}
\addtolength{\leftmargin}{\labelsep}
\setlength{\labelwidth}{\tmplength}
}
\begin{flushleft}
\item[\textbf{Declaração}\hfill]
\begin{ttfamily}
public class function COMMIT(Const Wok{\_}Set{\_}Transaction : Boolean):Boolean; Overload;\end{ttfamily}


\end{flushleft}
\end{list}
\paragraph*{Rollback}\hspace*{\fill}

\begin{list}{}{
\settowidth{\tmplength}{\textbf{Declaração}}
\setlength{\itemindent}{0cm}
\setlength{\listparindent}{0cm}
\setlength{\leftmargin}{\evensidemargin}
\addtolength{\leftmargin}{\tmplength}
\settowidth{\labelsep}{X}
\addtolength{\leftmargin}{\labelsep}
\setlength{\labelwidth}{\tmplength}
}
\begin{flushleft}
\item[\textbf{Declaração}\hfill]
\begin{ttfamily}
public class Procedure Rollback;\end{ttfamily}


\end{flushleft}
\end{list}
\paragraph*{SetTransaction}\hspace*{\fill}

\begin{list}{}{
\settowidth{\tmplength}{\textbf{Declaração}}
\setlength{\itemindent}{0cm}
\setlength{\listparindent}{0cm}
\setlength{\leftmargin}{\evensidemargin}
\addtolength{\leftmargin}{\tmplength}
\settowidth{\labelsep}{X}
\addtolength{\leftmargin}{\labelsep}
\setlength{\labelwidth}{\tmplength}
}
\begin{flushleft}
\item[\textbf{Declaração}\hfill]
\begin{ttfamily}
public class function SetTransaction(const OnOff:Boolean; Var WOK : Boolean ):Boolean; Overload;\end{ttfamily}


\end{flushleft}
\end{list}
\paragraph*{SetTransaction}\hspace*{\fill}

\begin{list}{}{
\settowidth{\tmplength}{\textbf{Declaração}}
\setlength{\itemindent}{0cm}
\setlength{\listparindent}{0cm}
\setlength{\leftmargin}{\evensidemargin}
\addtolength{\leftmargin}{\tmplength}
\settowidth{\labelsep}{X}
\addtolength{\leftmargin}{\labelsep}
\setlength{\labelwidth}{\tmplength}
}
\begin{flushleft}
\item[\textbf{Declaração}\hfill]
\begin{ttfamily}
public class function SetTransaction(const OnOff:Boolean; Var WOK, Wok{\_}Set{\_}Transaction:Boolean):Boolean; Overload;\end{ttfamily}


\end{flushleft}
\end{list}
\paragraph*{GetFileName{\_}Transaction}\hspace*{\fill}

\begin{list}{}{
\settowidth{\tmplength}{\textbf{Declaração}}
\setlength{\itemindent}{0cm}
\setlength{\listparindent}{0cm}
\setlength{\leftmargin}{\evensidemargin}
\addtolength{\leftmargin}{\tmplength}
\settowidth{\labelsep}{X}
\addtolength{\leftmargin}{\labelsep}
\setlength{\labelwidth}{\tmplength}
}
\begin{flushleft}
\item[\textbf{Declaração}\hfill]
\begin{ttfamily}
public class function GetFileName{\_}Transaction(): tString;\end{ttfamily}


\end{flushleft}
\end{list}
\paragraph*{Assign{\_}Transaction}\hspace*{\fill}

\begin{list}{}{
\settowidth{\tmplength}{\textbf{Declaração}}
\setlength{\itemindent}{0cm}
\setlength{\listparindent}{0cm}
\setlength{\leftmargin}{\evensidemargin}
\addtolength{\leftmargin}{\tmplength}
\settowidth{\labelsep}{X}
\addtolength{\leftmargin}{\labelsep}
\setlength{\labelwidth}{\tmplength}
}
\begin{flushleft}
\item[\textbf{Declaração}\hfill]
\begin{ttfamily}
public class function Assign{\_}Transaction(Const aFileName : PathStr):SmallWord;\end{ttfamily}


\end{flushleft}
\end{list}
\paragraph*{TransactionPendant{\_}Error}\hspace*{\fill}

\begin{list}{}{
\settowidth{\tmplength}{\textbf{Declaração}}
\setlength{\itemindent}{0cm}
\setlength{\listparindent}{0cm}
\setlength{\leftmargin}{\evensidemargin}
\addtolength{\leftmargin}{\tmplength}
\settowidth{\labelsep}{X}
\addtolength{\leftmargin}{\labelsep}
\setlength{\labelwidth}{\tmplength}
}
\begin{flushleft}
\item[\textbf{Declaração}\hfill]
\begin{ttfamily}
public class function TransactionPendant{\_}Error:Boolean;\end{ttfamily}


\end{flushleft}
\end{list}
\paragraph*{TransactionPendant}\hspace*{\fill}

\begin{list}{}{
\settowidth{\tmplength}{\textbf{Declaração}}
\setlength{\itemindent}{0cm}
\setlength{\listparindent}{0cm}
\setlength{\leftmargin}{\evensidemargin}
\addtolength{\leftmargin}{\tmplength}
\settowidth{\labelsep}{X}
\addtolength{\leftmargin}{\labelsep}
\setlength{\labelwidth}{\tmplength}
}
\begin{flushleft}
\item[\textbf{Declaração}\hfill]
\begin{ttfamily}
public class function TransactionPendant:Boolean;\end{ttfamily}


\end{flushleft}
\end{list}
\paragraph*{Truncate}\hspace*{\fill}

\begin{list}{}{
\settowidth{\tmplength}{\textbf{Declaração}}
\setlength{\itemindent}{0cm}
\setlength{\listparindent}{0cm}
\setlength{\leftmargin}{\evensidemargin}
\addtolength{\leftmargin}{\tmplength}
\settowidth{\labelsep}{X}
\addtolength{\leftmargin}{\labelsep}
\setlength{\labelwidth}{\tmplength}
}
\begin{flushleft}
\item[\textbf{Declaração}\hfill]
\begin{ttfamily}
public class Procedure Truncate(Var DatF: DataFile;NR : LongInt);\end{ttfamily}


\end{flushleft}
\end{list}
\paragraph*{CopyFrom}\hspace*{\fill}

\begin{list}{}{
\settowidth{\tmplength}{\textbf{Declaração}}
\setlength{\itemindent}{0cm}
\setlength{\listparindent}{0cm}
\setlength{\leftmargin}{\evensidemargin}
\addtolength{\leftmargin}{\tmplength}
\settowidth{\labelsep}{X}
\addtolength{\leftmargin}{\labelsep}
\setlength{\labelwidth}{\tmplength}
}
\begin{flushleft}
\item[\textbf{Declaração}\hfill]
\begin{ttfamily}
public class Procedure CopyFrom(Font{\_}DatF: DataFile ;Var Dest{\_}DatF: DataFile); Overload;\end{ttfamily}


\end{flushleft}
\end{list}
\paragraph*{CopyFrom}\hspace*{\fill}

\begin{list}{}{
\settowidth{\tmplength}{\textbf{Declaração}}
\setlength{\itemindent}{0cm}
\setlength{\listparindent}{0cm}
\setlength{\leftmargin}{\evensidemargin}
\addtolength{\leftmargin}{\tmplength}
\settowidth{\labelsep}{X}
\addtolength{\leftmargin}{\labelsep}
\setlength{\labelwidth}{\tmplength}
}
\begin{flushleft}
\item[\textbf{Declaração}\hfill]
\begin{ttfamily}
public class Procedure CopyFrom(Font{\_}IxF : IndexFile ;Var Dest{\_}IxF : IndexFile); Overload;\end{ttfamily}


\end{flushleft}
\end{list}
\subsection*{TFilesOpens Classe}
\subsubsection*{\large{\textbf{Hierarquia}}\normalsize\hspace{1ex}\hfill}
TFilesOpens {$>$} \begin{ttfamily}TSortedCollection\end{ttfamily}(\ref{mi.rtl.Objects.Methods.Collection.SortedCollection.TSortedCollection}) {$>$} \begin{ttfamily}TCollection\end{ttfamily}(\ref{mi.rtl.Objects.Methods.Collection.TCollection}) {$>$} \begin{ttfamily}TObjectsMethods\end{ttfamily}(\ref{mi.rtl.Objects.Methods.TObjectsMethods}) {$>$} \begin{ttfamily}TObjectsConsts\end{ttfamily}(\ref{mi.rtl.Objects.Consts.TObjectsConsts}) {$>$} 
TObjectsTypes
\subsubsection*{\large{\textbf{Descrição}}\normalsize\hspace{1ex}\hfill}
\begin{itemize}
\item \begin{ttfamily}FilesOpens\end{ttfamily}(\ref{mi.rtl.Objects.Methods.Db.Tb_Access-FilesOpens}) é uma coleção que mantém todos os arquivos abertos \begin{ttfamily}at\end{ttfamily}(\ref{mi.rtl.Objects.Methods.Collection.TCollection-At})é o momento com objetivo de fecha{-}los nos casos exceção.
\end{itemize}\subsubsection*{\large{\textbf{Campos}}\normalsize\hspace{1ex}\hfill}
\paragraph*{okFlushAllFiles}\hspace*{\fill}

\begin{list}{}{
\settowidth{\tmplength}{\textbf{Declaração}}
\setlength{\itemindent}{0cm}
\setlength{\listparindent}{0cm}
\setlength{\leftmargin}{\evensidemargin}
\addtolength{\leftmargin}{\tmplength}
\settowidth{\labelsep}{X}
\addtolength{\leftmargin}{\labelsep}
\setlength{\labelwidth}{\tmplength}
}
\begin{flushleft}
\item[\textbf{Declaração}\hfill]
\begin{ttfamily}
public okFlushAllFiles: Boolean;\end{ttfamily}


\end{flushleft}
\end{list}
\subsubsection*{\large{\textbf{Métodos}}\normalsize\hspace{1ex}\hfill}
\paragraph*{Create}\hspace*{\fill}

\begin{list}{}{
\settowidth{\tmplength}{\textbf{Declaração}}
\setlength{\itemindent}{0cm}
\setlength{\listparindent}{0cm}
\setlength{\leftmargin}{\evensidemargin}
\addtolength{\leftmargin}{\tmplength}
\settowidth{\labelsep}{X}
\addtolength{\leftmargin}{\labelsep}
\setlength{\labelwidth}{\tmplength}
}
\begin{flushleft}
\item[\textbf{Declaração}\hfill]
\begin{ttfamily}
public constructor Create;\end{ttfamily}


\end{flushleft}
\end{list}
\paragraph*{destroy}\hspace*{\fill}

\begin{list}{}{
\settowidth{\tmplength}{\textbf{Declaração}}
\setlength{\itemindent}{0cm}
\setlength{\listparindent}{0cm}
\setlength{\leftmargin}{\evensidemargin}
\addtolength{\leftmargin}{\tmplength}
\settowidth{\labelsep}{X}
\addtolength{\leftmargin}{\labelsep}
\setlength{\labelwidth}{\tmplength}
}
\begin{flushleft}
\item[\textbf{Declaração}\hfill]
\begin{ttfamily}
public destructor destroy; override;\end{ttfamily}


\end{flushleft}
\end{list}
\paragraph*{Compare}\hspace*{\fill}

\begin{list}{}{
\settowidth{\tmplength}{\textbf{Declaração}}
\setlength{\itemindent}{0cm}
\setlength{\listparindent}{0cm}
\setlength{\leftmargin}{\evensidemargin}
\addtolength{\leftmargin}{\tmplength}
\settowidth{\labelsep}{X}
\addtolength{\leftmargin}{\labelsep}
\setlength{\labelwidth}{\tmplength}
}
\begin{flushleft}
\item[\textbf{Declaração}\hfill]
\begin{ttfamily}
public function Compare(Key1, Key2: Pointer): Sw{\_}Integer; Override;\end{ttfamily}


\end{flushleft}
\end{list}
\paragraph*{Set{\_}OkFlushAllFiles}\hspace*{\fill}

\begin{list}{}{
\settowidth{\tmplength}{\textbf{Declaração}}
\setlength{\itemindent}{0cm}
\setlength{\listparindent}{0cm}
\setlength{\leftmargin}{\evensidemargin}
\addtolength{\leftmargin}{\tmplength}
\settowidth{\labelsep}{X}
\addtolength{\leftmargin}{\labelsep}
\setlength{\labelwidth}{\tmplength}
}
\begin{flushleft}
\item[\textbf{Declaração}\hfill]
\begin{ttfamily}
public Function Set{\_}OkFlushAllFiles(wOkFlushAllFiles:boolean):Boolean;\end{ttfamily}


\end{flushleft}
\end{list}
\paragraph*{ListaTabelas}\hspace*{\fill}

\begin{list}{}{
\settowidth{\tmplength}{\textbf{Declaração}}
\setlength{\itemindent}{0cm}
\setlength{\listparindent}{0cm}
\setlength{\leftmargin}{\evensidemargin}
\addtolength{\leftmargin}{\tmplength}
\settowidth{\labelsep}{X}
\addtolength{\leftmargin}{\labelsep}
\setlength{\labelwidth}{\tmplength}
}
\begin{flushleft}
\item[\textbf{Declaração}\hfill]
\begin{ttfamily}
public Procedure ListaTabelas;\end{ttfamily}


\end{flushleft}
\end{list}
\paragraph*{Insert}\hspace*{\fill}

\begin{list}{}{
\settowidth{\tmplength}{\textbf{Declaração}}
\setlength{\itemindent}{0cm}
\setlength{\listparindent}{0cm}
\setlength{\leftmargin}{\evensidemargin}
\addtolength{\leftmargin}{\tmplength}
\settowidth{\labelsep}{X}
\addtolength{\leftmargin}{\labelsep}
\setlength{\labelwidth}{\tmplength}
}
\begin{flushleft}
\item[\textbf{Declaração}\hfill]
\begin{ttfamily}
public procedure Insert(Item: Pointer); Override;\end{ttfamily}


\end{flushleft}
\end{list}
\paragraph*{FreeItem}\hspace*{\fill}

\begin{list}{}{
\settowidth{\tmplength}{\textbf{Declaração}}
\setlength{\itemindent}{0cm}
\setlength{\listparindent}{0cm}
\setlength{\leftmargin}{\evensidemargin}
\addtolength{\leftmargin}{\tmplength}
\settowidth{\labelsep}{X}
\addtolength{\leftmargin}{\labelsep}
\setlength{\labelwidth}{\tmplength}
}
\begin{flushleft}
\item[\textbf{Declaração}\hfill]
\begin{ttfamily}
public procedure FreeItem(Item: Pointer); Override;\end{ttfamily}


\end{flushleft}
\end{list}
\paragraph*{FOkCodigo}\hspace*{\fill}

\begin{list}{}{
\settowidth{\tmplength}{\textbf{Declaração}}
\setlength{\itemindent}{0cm}
\setlength{\listparindent}{0cm}
\setlength{\leftmargin}{\evensidemargin}
\addtolength{\leftmargin}{\tmplength}
\settowidth{\labelsep}{X}
\addtolength{\leftmargin}{\labelsep}
\setlength{\labelwidth}{\tmplength}
}
\begin{flushleft}
\item[\textbf{Declaração}\hfill]
\begin{ttfamily}
public Function FOkCodigo(NomeIxF:PathStr;Const Codigo:tString):Boolean;\end{ttfamily}


\end{flushleft}
\end{list}
\paragraph*{FlushIndexs}\hspace*{\fill}

\begin{list}{}{
\settowidth{\tmplength}{\textbf{Declaração}}
\setlength{\itemindent}{0cm}
\setlength{\listparindent}{0cm}
\setlength{\leftmargin}{\evensidemargin}
\addtolength{\leftmargin}{\tmplength}
\settowidth{\labelsep}{X}
\addtolength{\leftmargin}{\labelsep}
\setlength{\labelwidth}{\tmplength}
}
\begin{flushleft}
\item[\textbf{Declaração}\hfill]
\begin{ttfamily}
public Procedure FlushIndexs;\end{ttfamily}


\end{flushleft}
\end{list}
\paragraph*{FlushAllFiles}\hspace*{\fill}

\begin{list}{}{
\settowidth{\tmplength}{\textbf{Declaração}}
\setlength{\itemindent}{0cm}
\setlength{\listparindent}{0cm}
\setlength{\leftmargin}{\evensidemargin}
\addtolength{\leftmargin}{\tmplength}
\settowidth{\labelsep}{X}
\addtolength{\leftmargin}{\labelsep}
\setlength{\labelwidth}{\tmplength}
}
\begin{flushleft}
\item[\textbf{Declaração}\hfill]
\begin{ttfamily}
public Procedure FlushAllFiles;\end{ttfamily}


\end{flushleft}
\end{list}
\paragraph*{OpenAllFiles}\hspace*{\fill}

\begin{list}{}{
\settowidth{\tmplength}{\textbf{Declaração}}
\setlength{\itemindent}{0cm}
\setlength{\listparindent}{0cm}
\setlength{\leftmargin}{\evensidemargin}
\addtolength{\leftmargin}{\tmplength}
\settowidth{\labelsep}{X}
\addtolength{\leftmargin}{\labelsep}
\setlength{\labelwidth}{\tmplength}
}
\begin{flushleft}
\item[\textbf{Declaração}\hfill]
\begin{ttfamily}
public Function OpenAllFiles:Boolean;\end{ttfamily}


\end{flushleft}
\end{list}
\paragraph*{CloseAllFiles}\hspace*{\fill}

\begin{list}{}{
\settowidth{\tmplength}{\textbf{Declaração}}
\setlength{\itemindent}{0cm}
\setlength{\listparindent}{0cm}
\setlength{\leftmargin}{\evensidemargin}
\addtolength{\leftmargin}{\tmplength}
\settowidth{\labelsep}{X}
\addtolength{\leftmargin}{\labelsep}
\setlength{\labelwidth}{\tmplength}
}
\begin{flushleft}
\item[\textbf{Declaração}\hfill]
\begin{ttfamily}
public Function CloseAllFiles:Boolean;\end{ttfamily}


\end{flushleft}
\end{list}
\paragraph*{EstatisticasDosArquivosAbertos}\hspace*{\fill}

\begin{list}{}{
\settowidth{\tmplength}{\textbf{Declaração}}
\setlength{\itemindent}{0cm}
\setlength{\listparindent}{0cm}
\setlength{\leftmargin}{\evensidemargin}
\addtolength{\leftmargin}{\tmplength}
\settowidth{\labelsep}{X}
\addtolength{\leftmargin}{\labelsep}
\setlength{\labelwidth}{\tmplength}
}
\begin{flushleft}
\item[\textbf{Declaração}\hfill]
\begin{ttfamily}
public Procedure EstatisticasDosArquivosAbertos;\end{ttfamily}


\end{flushleft}
\end{list}
\paragraph*{SaveCurrentState}\hspace*{\fill}

\begin{list}{}{
\settowidth{\tmplength}{\textbf{Declaração}}
\setlength{\itemindent}{0cm}
\setlength{\listparindent}{0cm}
\setlength{\leftmargin}{\evensidemargin}
\addtolength{\leftmargin}{\tmplength}
\settowidth{\labelsep}{X}
\addtolength{\leftmargin}{\labelsep}
\setlength{\labelwidth}{\tmplength}
}
\begin{flushleft}
\item[\textbf{Declaração}\hfill]
\begin{ttfamily}
public Procedure SaveCurrentState;\end{ttfamily}


\end{flushleft}
\end{list}
\paragraph*{RestoreCurrentState}\hspace*{\fill}

\begin{list}{}{
\settowidth{\tmplength}{\textbf{Declaração}}
\setlength{\itemindent}{0cm}
\setlength{\listparindent}{0cm}
\setlength{\leftmargin}{\evensidemargin}
\addtolength{\leftmargin}{\tmplength}
\settowidth{\labelsep}{X}
\addtolength{\leftmargin}{\labelsep}
\setlength{\labelwidth}{\tmplength}
}
\begin{flushleft}
\item[\textbf{Declaração}\hfill]
\begin{ttfamily}
public Procedure RestoreCurrentState;\end{ttfamily}


\end{flushleft}
\end{list}
\paragraph*{MaxTamReg}\hspace*{\fill}

\begin{list}{}{
\settowidth{\tmplength}{\textbf{Declaração}}
\setlength{\itemindent}{0cm}
\setlength{\listparindent}{0cm}
\setlength{\leftmargin}{\evensidemargin}
\addtolength{\leftmargin}{\tmplength}
\settowidth{\labelsep}{X}
\addtolength{\leftmargin}{\labelsep}
\setlength{\labelwidth}{\tmplength}
}
\begin{flushleft}
\item[\textbf{Declaração}\hfill]
\begin{ttfamily}
public Function MaxTamReg:SmallWord;\end{ttfamily}


\end{flushleft}
\end{list}
\section{Variáveis}
\subsection*{FilesOpens}
\begin{list}{}{
\settowidth{\tmplength}{\textbf{Declaração}}
\setlength{\itemindent}{0cm}
\setlength{\listparindent}{0cm}
\setlength{\leftmargin}{\evensidemargin}
\addtolength{\leftmargin}{\tmplength}
\settowidth{\labelsep}{X}
\addtolength{\leftmargin}{\labelsep}
\setlength{\labelwidth}{\tmplength}
}
\begin{flushleft}
\item[\textbf{Declaração}\hfill]
\begin{ttfamily}
FilesOpens: TFilesOpens = nil;\end{ttfamily}


\end{flushleft}
\end{list}
\chapter{Unit mi.rtl.Objects.Methods.Db.Tb{\_}{\_}Access}
\section{Descrição}
Esta unit \textbf{\begin{ttfamily}mi.rtl.Objects.Methods.Db.Tb{\_}{\_}Access\end{ttfamily}} é usada para criar banco de dados local usando estrutura \textbf{Type Record End};
\section{Uses}
\begin{itemize}
\item \begin{ttfamily}Classes\end{ttfamily}\item \begin{ttfamily}SysUtils\end{ttfamily}\item \begin{ttfamily}dos\end{ttfamily}\item \begin{ttfamily}mi.rtl.Objects.Methods.Paramexecucao.Application\end{ttfamily}(\ref{mi.rtl.Objects.Methods.Paramexecucao.Application})\item \begin{ttfamily}mi.rtl.objects.Methods.Exception\end{ttfamily}(\ref{mi.rtl.Objects.Methods.Exception})\item \begin{ttfamily}mi.rtl.objects.Methods.dates\end{ttfamily}(\ref{mi.rtl.objects.Methods.dates})\item \begin{ttfamily}mi.rtl.objects.methods.ParamExecucao\end{ttfamily}(\ref{mi.rtl.Objects.Methods.Paramexecucao})\item \begin{ttfamily}mi.rtl.objects.methods.db.tb{\_}access\end{ttfamily}(\ref{mi.rtl.Objects.Methods.Db.Tb_Access})\end{itemize}
\section{Visão Geral}
\begin{description}
\item[\texttt{\begin{ttfamily}TTb{\_}{\_}Access{\_}types\end{ttfamily} Classe}]
\item[\texttt{\begin{ttfamily}TTb{\_}{\_}Access{\_}consts\end{ttfamily} Classe}]
\item[\texttt{\begin{ttfamily}TTb{\_}{\_}Access\end{ttfamily} Classe}]
\end{description}
\section{Classes, Interfaces, Objetos e Registros}
\subsection*{TTb{\_}{\_}Access{\_}types Classe}
\subsubsection*{\large{\textbf{Hierarquia}}\normalsize\hspace{1ex}\hfill}
TTb{\_}{\_}Access{\_}types {$>$} \begin{ttfamily}TTb{\_}Access\end{ttfamily}(\ref{mi.rtl.Objects.Methods.Db.Tb_Access.TTb_Access}) {$>$} \begin{ttfamily}TTb{\_}Access{\_}consts\end{ttfamily}(\ref{mi.rtl.Objects.Methods.Db.Tb_Access.TTb_Access_consts}) {$>$} \begin{ttfamily}TTb{\_}Access{\_}types\end{ttfamily}(\ref{mi.rtl.Objects.Methods.Db.Tb_Access.TTb_Access_types}) {$>$} \begin{ttfamily}TObjectsSystem\end{ttfamily}(\ref{mi.rtl.Objects.Methods.System.TObjectsSystem}) {$>$} \begin{ttfamily}TObjectsMethods\end{ttfamily}(\ref{mi.rtl.Objects.Methods.TObjectsMethods}) {$>$} \begin{ttfamily}TObjectsConsts\end{ttfamily}(\ref{mi.rtl.Objects.Consts.TObjectsConsts}) {$>$} 
TObjectsTypes
\subsubsection*{\large{\textbf{Descrição}}\normalsize\hspace{1ex}\hfill}
no description available, TTb{\_}Access description followsno description available, TTb{\_}Access{\_}consts description followsA classe \textbf{\begin{ttfamily}TTb{\_}Access{\_}consts\end{ttfamily}} é usada para declarar todas as constantes da classe \textbf{\begin{ttfamily}TTb{\_}Access\end{ttfamily}(\ref{mi.rtl.Objects.Methods.Db.Tb_Access.TTb_Access})}\subsection*{TTb{\_}{\_}Access{\_}consts Classe}
\subsubsection*{\large{\textbf{Hierarquia}}\normalsize\hspace{1ex}\hfill}
TTb{\_}{\_}Access{\_}consts {$>$} \begin{ttfamily}TTb{\_}{\_}Access{\_}types\end{ttfamily}(\ref{mi.rtl.Objects.Methods.Db.Tb__Access.TTb__Access_types}) {$>$} \begin{ttfamily}TTb{\_}Access\end{ttfamily}(\ref{mi.rtl.Objects.Methods.Db.Tb_Access.TTb_Access}) {$>$} \begin{ttfamily}TTb{\_}Access{\_}consts\end{ttfamily}(\ref{mi.rtl.Objects.Methods.Db.Tb_Access.TTb_Access_consts}) {$>$} \begin{ttfamily}TTb{\_}Access{\_}types\end{ttfamily}(\ref{mi.rtl.Objects.Methods.Db.Tb_Access.TTb_Access_types}) {$>$} \begin{ttfamily}TObjectsSystem\end{ttfamily}(\ref{mi.rtl.Objects.Methods.System.TObjectsSystem}) {$>$} \begin{ttfamily}TObjectsMethods\end{ttfamily}(\ref{mi.rtl.Objects.Methods.TObjectsMethods}) {$>$} \begin{ttfamily}TObjectsConsts\end{ttfamily}(\ref{mi.rtl.Objects.Consts.TObjectsConsts}) {$>$} 
TObjectsTypes
\subsubsection*{\large{\textbf{Descrição}}\normalsize\hspace{1ex}\hfill}
no description available, TTb{\_}{\_}Access{\_}types description followsno description available, TTb{\_}Access description followsno description available, TTb{\_}Access{\_}consts description followsA classe \textbf{\begin{ttfamily}TTb{\_}Access{\_}consts\end{ttfamily}} é usada para declarar todas as constantes da classe \textbf{\begin{ttfamily}TTb{\_}Access\end{ttfamily}(\ref{mi.rtl.Objects.Methods.Db.Tb_Access.TTb_Access})}\subsubsection*{\large{\textbf{Campos}}\normalsize\hspace{1ex}\hfill}
\paragraph*{EndClearAll}\hspace*{\fill}

\begin{list}{}{
\settowidth{\tmplength}{\textbf{Declaração}}
\setlength{\itemindent}{0cm}
\setlength{\listparindent}{0cm}
\setlength{\leftmargin}{\evensidemargin}
\addtolength{\leftmargin}{\tmplength}
\settowidth{\labelsep}{X}
\addtolength{\leftmargin}{\labelsep}
\setlength{\labelwidth}{\tmplength}
}
\begin{flushleft}
\item[\textbf{Declaração}\hfill]
\begin{ttfamily}
public const EndClearAll     : TipoProc = Nil;\end{ttfamily}


\end{flushleft}
\end{list}
\paragraph*{EndOpenFiles}\hspace*{\fill}

\begin{list}{}{
\settowidth{\tmplength}{\textbf{Declaração}}
\setlength{\itemindent}{0cm}
\setlength{\listparindent}{0cm}
\setlength{\leftmargin}{\evensidemargin}
\addtolength{\leftmargin}{\tmplength}
\settowidth{\labelsep}{X}
\addtolength{\leftmargin}{\labelsep}
\setlength{\labelwidth}{\tmplength}
}
\begin{flushleft}
\item[\textbf{Declaração}\hfill]
\begin{ttfamily}
public const EndOpenFiles    : TipoProc = Nil;\end{ttfamily}


\end{flushleft}
\end{list}
\paragraph*{EndCloseFiles}\hspace*{\fill}

\begin{list}{}{
\settowidth{\tmplength}{\textbf{Declaração}}
\setlength{\itemindent}{0cm}
\setlength{\listparindent}{0cm}
\setlength{\leftmargin}{\evensidemargin}
\addtolength{\leftmargin}{\tmplength}
\settowidth{\labelsep}{X}
\addtolength{\leftmargin}{\labelsep}
\setlength{\labelwidth}{\tmplength}
}
\begin{flushleft}
\item[\textbf{Declaração}\hfill]
\begin{ttfamily}
public const EndCloseFiles   : TipoProc = Nil;\end{ttfamily}


\end{flushleft}
\end{list}
\paragraph*{OkDeveReindexarDB}\hspace*{\fill}

\begin{list}{}{
\settowidth{\tmplength}{\textbf{Declaração}}
\setlength{\itemindent}{0cm}
\setlength{\listparindent}{0cm}
\setlength{\leftmargin}{\evensidemargin}
\addtolength{\leftmargin}{\tmplength}
\settowidth{\labelsep}{X}
\addtolength{\leftmargin}{\labelsep}
\setlength{\labelwidth}{\tmplength}
}
\begin{flushleft}
\item[\textbf{Declaração}\hfill]
\begin{ttfamily}
public const OkDeveReindexarDB : Boolean = false;\end{ttfamily}


\end{flushleft}
\par
\item[\textbf{Descrição}]
True = O sistema deve reindexar todos os arquivos

\end{list}
\paragraph*{OkDeveRepararDB}\hspace*{\fill}

\begin{list}{}{
\settowidth{\tmplength}{\textbf{Declaração}}
\setlength{\itemindent}{0cm}
\setlength{\listparindent}{0cm}
\setlength{\leftmargin}{\evensidemargin}
\addtolength{\leftmargin}{\tmplength}
\settowidth{\labelsep}{X}
\addtolength{\leftmargin}{\labelsep}
\setlength{\labelwidth}{\tmplength}
}
\begin{flushleft}
\item[\textbf{Declaração}\hfill]
\begin{ttfamily}
public const OkDeveRepararDB   : Boolean = false;\end{ttfamily}


\end{flushleft}
\par
\item[\textbf{Descrição}]
True = O Sistema deve executar a rotina para reparar as consistências entre tabelas

\end{list}
\paragraph*{NRecAux}\hspace*{\fill}

\begin{list}{}{
\settowidth{\tmplength}{\textbf{Declaração}}
\setlength{\itemindent}{0cm}
\setlength{\listparindent}{0cm}
\setlength{\leftmargin}{\evensidemargin}
\addtolength{\leftmargin}{\tmplength}
\settowidth{\labelsep}{X}
\addtolength{\leftmargin}{\labelsep}
\setlength{\labelwidth}{\tmplength}
}
\begin{flushleft}
\item[\textbf{Declaração}\hfill]
\begin{ttfamily}
public const NRecAux      : Longint = 0;\end{ttfamily}


\end{flushleft}
\par
\item[\textbf{Descrição}]
A constante \textbf{\begin{ttfamily}NRecAux\end{ttfamily}} é o número do registro corrente temporário.

\begin{itemize}
\item \textbf{NOTA} \begin{itemize}
\item É usado para manter a compatibilidade com o passado.
\end{itemize}
\end{itemize}

\end{list}
\paragraph*{NRec}\hspace*{\fill}

\begin{list}{}{
\settowidth{\tmplength}{\textbf{Declaração}}
\setlength{\itemindent}{0cm}
\setlength{\listparindent}{0cm}
\setlength{\leftmargin}{\evensidemargin}
\addtolength{\leftmargin}{\tmplength}
\settowidth{\labelsep}{X}
\addtolength{\leftmargin}{\labelsep}
\setlength{\labelwidth}{\tmplength}
}
\begin{flushleft}
\item[\textbf{Declaração}\hfill]
\begin{ttfamily}
public const NRec         : Longint = 0;\end{ttfamily}


\end{flushleft}
\par
\item[\textbf{Descrição}]
A constante \textbf{\begin{ttfamily}NRec\end{ttfamily}} é o número do registro corrente

\begin{itemize}
\item \textbf{NOTA} \begin{itemize}
\item É usado para manter a compatibilidade com o passado.
\end{itemize}
\end{itemize}

\end{list}
\paragraph*{WCursorLigado}\hspace*{\fill}

\begin{list}{}{
\settowidth{\tmplength}{\textbf{Declaração}}
\setlength{\itemindent}{0cm}
\setlength{\listparindent}{0cm}
\setlength{\leftmargin}{\evensidemargin}
\addtolength{\leftmargin}{\tmplength}
\settowidth{\labelsep}{X}
\addtolength{\leftmargin}{\labelsep}
\setlength{\labelwidth}{\tmplength}
}
\begin{flushleft}
\item[\textbf{Declaração}\hfill]
\begin{ttfamily}
public const WCursorLigado : Boolean = true;\end{ttfamily}


\end{flushleft}
\end{list}
\paragraph*{WEndCloseFiles}\hspace*{\fill}

\begin{list}{}{
\settowidth{\tmplength}{\textbf{Declaração}}
\setlength{\itemindent}{0cm}
\setlength{\listparindent}{0cm}
\setlength{\leftmargin}{\evensidemargin}
\addtolength{\leftmargin}{\tmplength}
\settowidth{\labelsep}{X}
\addtolength{\leftmargin}{\labelsep}
\setlength{\labelwidth}{\tmplength}
}
\begin{flushleft}
\item[\textbf{Declaração}\hfill]
\begin{ttfamily}
public const WEndCloseFiles: TipoProc = nil;\end{ttfamily}


\end{flushleft}
\end{list}
\paragraph*{WEndOpenFiles}\hspace*{\fill}

\begin{list}{}{
\settowidth{\tmplength}{\textbf{Declaração}}
\setlength{\itemindent}{0cm}
\setlength{\listparindent}{0cm}
\setlength{\leftmargin}{\evensidemargin}
\addtolength{\leftmargin}{\tmplength}
\settowidth{\labelsep}{X}
\addtolength{\leftmargin}{\labelsep}
\setlength{\labelwidth}{\tmplength}
}
\begin{flushleft}
\item[\textbf{Declaração}\hfill]
\begin{ttfamily}
public const WEndOpenFiles : TipoProc = nil;\end{ttfamily}


\end{flushleft}
\end{list}
\subsection*{TTb{\_}{\_}Access Classe}
\subsubsection*{\large{\textbf{Hierarquia}}\normalsize\hspace{1ex}\hfill}
TTb{\_}{\_}Access {$>$} \begin{ttfamily}TTb{\_}{\_}Access{\_}consts\end{ttfamily}(\ref{mi.rtl.Objects.Methods.Db.Tb__Access.TTb__Access_consts}) {$>$} \begin{ttfamily}TTb{\_}{\_}Access{\_}types\end{ttfamily}(\ref{mi.rtl.Objects.Methods.Db.Tb__Access.TTb__Access_types}) {$>$} \begin{ttfamily}TTb{\_}Access\end{ttfamily}(\ref{mi.rtl.Objects.Methods.Db.Tb_Access.TTb_Access}) {$>$} \begin{ttfamily}TTb{\_}Access{\_}consts\end{ttfamily}(\ref{mi.rtl.Objects.Methods.Db.Tb_Access.TTb_Access_consts}) {$>$} \begin{ttfamily}TTb{\_}Access{\_}types\end{ttfamily}(\ref{mi.rtl.Objects.Methods.Db.Tb_Access.TTb_Access_types}) {$>$} \begin{ttfamily}TObjectsSystem\end{ttfamily}(\ref{mi.rtl.Objects.Methods.System.TObjectsSystem}) {$>$} \begin{ttfamily}TObjectsMethods\end{ttfamily}(\ref{mi.rtl.Objects.Methods.TObjectsMethods}) {$>$} \begin{ttfamily}TObjectsConsts\end{ttfamily}(\ref{mi.rtl.Objects.Consts.TObjectsConsts}) {$>$} 
TObjectsTypes
\subsubsection*{\large{\textbf{Descrição}}\normalsize\hspace{1ex}\hfill}
no description available, TTb{\_}{\_}Access{\_}consts description followsno description available, TTb{\_}{\_}Access{\_}types description followsno description available, TTb{\_}Access description followsno description available, TTb{\_}Access{\_}consts description followsA classe \textbf{\begin{ttfamily}TTb{\_}Access{\_}consts\end{ttfamily}} é usada para declarar todas as constantes da classe \textbf{\begin{ttfamily}TTb{\_}Access\end{ttfamily}(\ref{mi.rtl.Objects.Methods.Db.Tb_Access.TTb_Access})}\subsubsection*{\large{\textbf{Métodos}}\normalsize\hspace{1ex}\hfill}
\paragraph*{StartTransaction}\hspace*{\fill}

\begin{list}{}{
\settowidth{\tmplength}{\textbf{Declaração}}
\setlength{\itemindent}{0cm}
\setlength{\listparindent}{0cm}
\setlength{\leftmargin}{\evensidemargin}
\addtolength{\leftmargin}{\tmplength}
\settowidth{\labelsep}{X}
\addtolength{\leftmargin}{\labelsep}
\setlength{\labelwidth}{\tmplength}
}
\begin{flushleft}
\item[\textbf{Declaração}\hfill]
\begin{ttfamily}
public class function StartTransaction(Const DatF : TMI{\_}DataFile ; Var aok{\_}Set{\_}Transaction : Boolean): Integer; Overload;\end{ttfamily}


\end{flushleft}
\end{list}
\paragraph*{FileSize}\hspace*{\fill}

\begin{list}{}{
\settowidth{\tmplength}{\textbf{Declaração}}
\setlength{\itemindent}{0cm}
\setlength{\listparindent}{0cm}
\setlength{\leftmargin}{\evensidemargin}
\addtolength{\leftmargin}{\tmplength}
\settowidth{\labelsep}{X}
\addtolength{\leftmargin}{\labelsep}
\setlength{\labelwidth}{\tmplength}
}
\begin{flushleft}
\item[\textbf{Declaração}\hfill]
\begin{ttfamily}
public class function FileSize(Var MI{\_}DataFile : TMI{\_}DataFile):Longint; Overload;\end{ttfamily}


\end{flushleft}
\end{list}
\paragraph*{Init{\_}MI{\_}DataFile}\hspace*{\fill}

\begin{list}{}{
\settowidth{\tmplength}{\textbf{Declaração}}
\setlength{\itemindent}{0cm}
\setlength{\listparindent}{0cm}
\setlength{\leftmargin}{\evensidemargin}
\addtolength{\leftmargin}{\tmplength}
\settowidth{\labelsep}{X}
\addtolength{\leftmargin}{\labelsep}
\setlength{\labelwidth}{\tmplength}
}
\begin{flushleft}
\item[\textbf{Declaração}\hfill]
\begin{ttfamily}
public class Procedure Init{\_}MI{\_}DataFile(Var MI{\_}DataFile : TMI{\_}DataFile; NomeArquivo : PathStr; tamanhoRegistro : SmallWord; NumeroDeArqIndice : byte ); Overload;\end{ttfamily}


\end{flushleft}
\end{list}
\paragraph*{Init{\_}MI{\_}DataFile}\hspace*{\fill}

\begin{list}{}{
\settowidth{\tmplength}{\textbf{Declaração}}
\setlength{\itemindent}{0cm}
\setlength{\listparindent}{0cm}
\setlength{\leftmargin}{\evensidemargin}
\addtolength{\leftmargin}{\tmplength}
\settowidth{\labelsep}{X}
\addtolength{\leftmargin}{\labelsep}
\setlength{\labelwidth}{\tmplength}
}
\begin{flushleft}
\item[\textbf{Declaração}\hfill]
\begin{ttfamily}
public class Procedure Init{\_}MI{\_}DataFile(Var MI{\_}DataFile : TMI{\_}DataFile; NomeArquivo : PathStr; tamanhoRegistro : TTb{\_}Access.SmallWord; NumeroDeArqIndice : byte; wOkTemporario : Boolean); Overload;\end{ttfamily}


\end{flushleft}
\end{list}
\paragraph*{Init{\_}IxF}\hspace*{\fill}

\begin{list}{}{
\settowidth{\tmplength}{\textbf{Declaração}}
\setlength{\itemindent}{0cm}
\setlength{\listparindent}{0cm}
\setlength{\leftmargin}{\evensidemargin}
\addtolength{\leftmargin}{\tmplength}
\settowidth{\labelsep}{X}
\addtolength{\leftmargin}{\labelsep}
\setlength{\labelwidth}{\tmplength}
}
\begin{flushleft}
\item[\textbf{Declaração}\hfill]
\begin{ttfamily}
public class Procedure Init{\_}IxF(Const Indice : Byte; Var IxF : TMI{\_}IndexFile; Const CNomeArqIndice : PathStr; Const CRepeteChave : Byte; Const StrCondicao : tString );\end{ttfamily}


\end{flushleft}
\end{list}
\paragraph*{MakeFile}\hspace*{\fill}

\begin{list}{}{
\settowidth{\tmplength}{\textbf{Declaração}}
\setlength{\itemindent}{0cm}
\setlength{\listparindent}{0cm}
\setlength{\leftmargin}{\evensidemargin}
\addtolength{\leftmargin}{\tmplength}
\settowidth{\labelsep}{X}
\addtolength{\leftmargin}{\labelsep}
\setlength{\labelwidth}{\tmplength}
}
\begin{flushleft}
\item[\textbf{Declaração}\hfill]
\begin{ttfamily}
public class function MakeFile(Const FileName:PathStr;Const TamArq:Longint):Integer; overload;\end{ttfamily}


\end{flushleft}
\end{list}
\paragraph*{MakeFile}\hspace*{\fill}

\begin{list}{}{
\settowidth{\tmplength}{\textbf{Declaração}}
\setlength{\itemindent}{0cm}
\setlength{\listparindent}{0cm}
\setlength{\leftmargin}{\evensidemargin}
\addtolength{\leftmargin}{\tmplength}
\settowidth{\labelsep}{X}
\addtolength{\leftmargin}{\labelsep}
\setlength{\labelwidth}{\tmplength}
}
\begin{flushleft}
\item[\textbf{Declaração}\hfill]
\begin{ttfamily}
public class function MakeFile(var DatF : TMI{\_}DataFile ):Integer; overload;\end{ttfamily}


\end{flushleft}
\end{list}
\paragraph*{MakeIndex}\hspace*{\fill}

\begin{list}{}{
\settowidth{\tmplength}{\textbf{Declaração}}
\setlength{\itemindent}{0cm}
\setlength{\listparindent}{0cm}
\setlength{\leftmargin}{\evensidemargin}
\addtolength{\leftmargin}{\tmplength}
\settowidth{\labelsep}{X}
\addtolength{\leftmargin}{\labelsep}
\setlength{\labelwidth}{\tmplength}
}
\begin{flushleft}
\item[\textbf{Declaração}\hfill]
\begin{ttfamily}
public class function MakeIndex(Const FileName:PathStr;Const RepeteChave,TamChave:Byte):Integer; overload;\end{ttfamily}


\end{flushleft}
\end{list}
\paragraph*{MakeIndex}\hspace*{\fill}

\begin{list}{}{
\settowidth{\tmplength}{\textbf{Declaração}}
\setlength{\itemindent}{0cm}
\setlength{\listparindent}{0cm}
\setlength{\leftmargin}{\evensidemargin}
\addtolength{\leftmargin}{\tmplength}
\settowidth{\labelsep}{X}
\addtolength{\leftmargin}{\labelsep}
\setlength{\labelwidth}{\tmplength}
}
\begin{flushleft}
\item[\textbf{Declaração}\hfill]
\begin{ttfamily}
public class function MakeIndex(var IxF : TMI{\_}IndexFile):Integer; Overload;\end{ttfamily}


\end{flushleft}
\end{list}
\paragraph*{OpenFile}\hspace*{\fill}

\begin{list}{}{
\settowidth{\tmplength}{\textbf{Declaração}}
\setlength{\itemindent}{0cm}
\setlength{\listparindent}{0cm}
\setlength{\leftmargin}{\evensidemargin}
\addtolength{\leftmargin}{\tmplength}
\settowidth{\labelsep}{X}
\addtolength{\leftmargin}{\labelsep}
\setlength{\labelwidth}{\tmplength}
}
\begin{flushleft}
\item[\textbf{Declaração}\hfill]
\begin{ttfamily}
public class function OpenFile(var DatF : TMI{\_}DataFile;OkCreate : Boolean):Integer; Overload;\end{ttfamily}


\end{flushleft}
\end{list}
\paragraph*{OpenFile}\hspace*{\fill}

\begin{list}{}{
\settowidth{\tmplength}{\textbf{Declaração}}
\setlength{\itemindent}{0cm}
\setlength{\listparindent}{0cm}
\setlength{\leftmargin}{\evensidemargin}
\addtolength{\leftmargin}{\tmplength}
\settowidth{\labelsep}{X}
\addtolength{\leftmargin}{\labelsep}
\setlength{\labelwidth}{\tmplength}
}
\begin{flushleft}
\item[\textbf{Declaração}\hfill]
\begin{ttfamily}
public class function OpenFile(var DatF : TMI{\_}DataFile ):Integer; Overload;\end{ttfamily}


\end{flushleft}
\end{list}
\paragraph*{OpenIndex}\hspace*{\fill}

\begin{list}{}{
\settowidth{\tmplength}{\textbf{Declaração}}
\setlength{\itemindent}{0cm}
\setlength{\listparindent}{0cm}
\setlength{\leftmargin}{\evensidemargin}
\addtolength{\leftmargin}{\tmplength}
\settowidth{\labelsep}{X}
\addtolength{\leftmargin}{\labelsep}
\setlength{\labelwidth}{\tmplength}
}
\begin{flushleft}
\item[\textbf{Declaração}\hfill]
\begin{ttfamily}
public class function OpenIndex(var IxF : TMI{\_}IndexFile):Integer; Overload;\end{ttfamily}


\end{flushleft}
\end{list}
\paragraph*{CloseFile}\hspace*{\fill}

\begin{list}{}{
\settowidth{\tmplength}{\textbf{Declaração}}
\setlength{\itemindent}{0cm}
\setlength{\listparindent}{0cm}
\setlength{\leftmargin}{\evensidemargin}
\addtolength{\leftmargin}{\tmplength}
\settowidth{\labelsep}{X}
\addtolength{\leftmargin}{\labelsep}
\setlength{\labelwidth}{\tmplength}
}
\begin{flushleft}
\item[\textbf{Declaração}\hfill]
\begin{ttfamily}
public class function CloseFile(var DatF : TMI{\_}DataFile ):Integer; overload;\end{ttfamily}


\end{flushleft}
\end{list}
\paragraph*{CloseIndex}\hspace*{\fill}

\begin{list}{}{
\settowidth{\tmplength}{\textbf{Declaração}}
\setlength{\itemindent}{0cm}
\setlength{\listparindent}{0cm}
\setlength{\leftmargin}{\evensidemargin}
\addtolength{\leftmargin}{\tmplength}
\settowidth{\labelsep}{X}
\addtolength{\leftmargin}{\labelsep}
\setlength{\labelwidth}{\tmplength}
}
\begin{flushleft}
\item[\textbf{Declaração}\hfill]
\begin{ttfamily}
public class function CloseIndex(var IxF : TMI{\_}IndexFile):Boolean; overload;\end{ttfamily}


\end{flushleft}
\end{list}
\paragraph*{MakeArq}\hspace*{\fill}

\begin{list}{}{
\settowidth{\tmplength}{\textbf{Declaração}}
\setlength{\itemindent}{0cm}
\setlength{\listparindent}{0cm}
\setlength{\leftmargin}{\evensidemargin}
\addtolength{\leftmargin}{\tmplength}
\settowidth{\labelsep}{X}
\addtolength{\leftmargin}{\labelsep}
\setlength{\labelwidth}{\tmplength}
}
\begin{flushleft}
\item[\textbf{Declaração}\hfill]
\begin{ttfamily}
public class Procedure MakeArq(VAR DatF : TMI{\_}DataFile; VAR Buff );\end{ttfamily}


\end{flushleft}
\par
\item[\textbf{Descrição}]
A class procedure \textbf{\begin{ttfamily}MakeArq\end{ttfamily}} é usado criar aquivo sem o registro 0

\end{list}
\paragraph*{OpenArq}\hspace*{\fill}

\begin{list}{}{
\settowidth{\tmplength}{\textbf{Declaração}}
\setlength{\itemindent}{0cm}
\setlength{\listparindent}{0cm}
\setlength{\leftmargin}{\evensidemargin}
\addtolength{\leftmargin}{\tmplength}
\settowidth{\labelsep}{X}
\addtolength{\leftmargin}{\labelsep}
\setlength{\labelwidth}{\tmplength}
}
\begin{flushleft}
\item[\textbf{Declaração}\hfill]
\begin{ttfamily}
public class Procedure OpenArq(VAR DatF : TMI{\_}DataFile; VAR Buff );\end{ttfamily}


\end{flushleft}
\par
\item[\textbf{Descrição}]
A class procedure \textbf{\begin{ttfamily}OpenArq\end{ttfamily}} é usado abrir aquivo sem o registro 0

\end{list}
\paragraph*{AbreArqSemHeader}\hspace*{\fill}

\begin{list}{}{
\settowidth{\tmplength}{\textbf{Declaração}}
\setlength{\itemindent}{0cm}
\setlength{\listparindent}{0cm}
\setlength{\leftmargin}{\evensidemargin}
\addtolength{\leftmargin}{\tmplength}
\settowidth{\labelsep}{X}
\addtolength{\leftmargin}{\labelsep}
\setlength{\labelwidth}{\tmplength}
}
\begin{flushleft}
\item[\textbf{Declaração}\hfill]
\begin{ttfamily}
public class Procedure AbreArqSemHeader(VAR Arqdados:TMI{\_}DataFile ; VAR Buff );\end{ttfamily}


\end{flushleft}
\end{list}
\paragraph*{CloseArqSemHeader}\hspace*{\fill}

\begin{list}{}{
\settowidth{\tmplength}{\textbf{Declaração}}
\setlength{\itemindent}{0cm}
\setlength{\listparindent}{0cm}
\setlength{\leftmargin}{\evensidemargin}
\addtolength{\leftmargin}{\tmplength}
\settowidth{\labelsep}{X}
\addtolength{\leftmargin}{\labelsep}
\setlength{\labelwidth}{\tmplength}
}
\begin{flushleft}
\item[\textbf{Declaração}\hfill]
\begin{ttfamily}
public class Procedure CloseArqSemHeader(VAR DatF : TMI{\_}DataFile);\end{ttfamily}


\end{flushleft}
\end{list}
\paragraph*{GetAddRec}\hspace*{\fill}

\begin{list}{}{
\settowidth{\tmplength}{\textbf{Declaração}}
\setlength{\itemindent}{0cm}
\setlength{\listparindent}{0cm}
\setlength{\leftmargin}{\evensidemargin}
\addtolength{\leftmargin}{\tmplength}
\settowidth{\labelsep}{X}
\addtolength{\leftmargin}{\labelsep}
\setlength{\labelwidth}{\tmplength}
}
\begin{flushleft}
\item[\textbf{Declaração}\hfill]
\begin{ttfamily}
public class function GetAddRec(Const Title : tString; Const NomeFonte:PathStr; Var RegFonte; Const TamFonte : SmallWord; Const NomeDestino : PathStr; Var regDestino; Const TamDestino : SmallWord; Const AtualizaDestino : TFuncGetAddRec; Const OkMakeFile :Boolean) : Boolean;\end{ttfamily}


\end{flushleft}
\end{list}
\paragraph*{ModifyStructurFile}\hspace*{\fill}

\begin{list}{}{
\settowidth{\tmplength}{\textbf{Declaração}}
\setlength{\itemindent}{0cm}
\setlength{\listparindent}{0cm}
\setlength{\leftmargin}{\evensidemargin}
\addtolength{\leftmargin}{\tmplength}
\settowidth{\labelsep}{X}
\addtolength{\leftmargin}{\labelsep}
\setlength{\labelwidth}{\tmplength}
}
\begin{flushleft}
\item[\textbf{Declaração}\hfill]
\begin{ttfamily}
public class function ModifyStructurFile(Const FName:FileName;Const RecLenDest : SmallWord ):Boolean; override;\end{ttfamily}


\end{flushleft}
\end{list}
\paragraph*{PrimeiroLivre}\hspace*{\fill}

\begin{list}{}{
\settowidth{\tmplength}{\textbf{Declaração}}
\setlength{\itemindent}{0cm}
\setlength{\listparindent}{0cm}
\setlength{\leftmargin}{\evensidemargin}
\addtolength{\leftmargin}{\tmplength}
\settowidth{\labelsep}{X}
\addtolength{\leftmargin}{\labelsep}
\setlength{\labelwidth}{\tmplength}
}
\begin{flushleft}
\item[\textbf{Declaração}\hfill]
\begin{ttfamily}
public class function PrimeiroLivre(VAR DatF: TMI{\_}DataFile) : LONGINT;\end{ttfamily}


\end{flushleft}
\end{list}
\paragraph*{TraveArq}\hspace*{\fill}

\begin{list}{}{
\settowidth{\tmplength}{\textbf{Declaração}}
\setlength{\itemindent}{0cm}
\setlength{\listparindent}{0cm}
\setlength{\leftmargin}{\evensidemargin}
\addtolength{\leftmargin}{\tmplength}
\settowidth{\labelsep}{X}
\addtolength{\leftmargin}{\labelsep}
\setlength{\labelwidth}{\tmplength}
}
\begin{flushleft}
\item[\textbf{Declaração}\hfill]
\begin{ttfamily}
public class function TraveArq(Var DatF : TMI{\_}DataFile):Boolean;\end{ttfamily}


\end{flushleft}
\end{list}
\paragraph*{DestraveArq}\hspace*{\fill}

\begin{list}{}{
\settowidth{\tmplength}{\textbf{Declaração}}
\setlength{\itemindent}{0cm}
\setlength{\listparindent}{0cm}
\setlength{\leftmargin}{\evensidemargin}
\addtolength{\leftmargin}{\tmplength}
\settowidth{\labelsep}{X}
\addtolength{\leftmargin}{\labelsep}
\setlength{\labelwidth}{\tmplength}
}
\begin{flushleft}
\item[\textbf{Declaração}\hfill]
\begin{ttfamily}
public class function DestraveArq(Var DatF : TMI{\_}DataFile):Boolean;\end{ttfamily}


\end{flushleft}
\end{list}
\paragraph*{UsedRecs}\hspace*{\fill}

\begin{list}{}{
\settowidth{\tmplength}{\textbf{Declaração}}
\setlength{\itemindent}{0cm}
\setlength{\listparindent}{0cm}
\setlength{\leftmargin}{\evensidemargin}
\addtolength{\leftmargin}{\tmplength}
\settowidth{\labelsep}{X}
\addtolength{\leftmargin}{\labelsep}
\setlength{\labelwidth}{\tmplength}
}
\begin{flushleft}
\item[\textbf{Declaração}\hfill]
\begin{ttfamily}
public class function UsedRecs(var DatF : TMI{\_}DataFile ) : Longint; Overload;\end{ttfamily}


\end{flushleft}
\end{list}
\paragraph*{GetRec}\hspace*{\fill}

\begin{list}{}{
\settowidth{\tmplength}{\textbf{Declaração}}
\setlength{\itemindent}{0cm}
\setlength{\listparindent}{0cm}
\setlength{\leftmargin}{\evensidemargin}
\addtolength{\leftmargin}{\tmplength}
\settowidth{\labelsep}{X}
\addtolength{\leftmargin}{\labelsep}
\setlength{\labelwidth}{\tmplength}
}
\begin{flushleft}
\item[\textbf{Declaração}\hfill]
\begin{ttfamily}
public class function GetRec(var DatF : TMI{\_}DataFile ;Const R : Longint;var Buffer ):Boolean; overload;\end{ttfamily}


\end{flushleft}
\end{list}
\paragraph*{PutRec}\hspace*{\fill}

\begin{list}{}{
\settowidth{\tmplength}{\textbf{Declaração}}
\setlength{\itemindent}{0cm}
\setlength{\listparindent}{0cm}
\setlength{\leftmargin}{\evensidemargin}
\addtolength{\leftmargin}{\tmplength}
\settowidth{\labelsep}{X}
\addtolength{\leftmargin}{\labelsep}
\setlength{\labelwidth}{\tmplength}
}
\begin{flushleft}
\item[\textbf{Declaração}\hfill]
\begin{ttfamily}
public class function PutRec(var DatF : TMI{\_}DataFile ;Const R : Longint;var Buffer ):Boolean; overload;\end{ttfamily}


\end{flushleft}
\end{list}
\paragraph*{AddRec}\hspace*{\fill}

\begin{list}{}{
\settowidth{\tmplength}{\textbf{Declaração}}
\setlength{\itemindent}{0cm}
\setlength{\listparindent}{0cm}
\setlength{\leftmargin}{\evensidemargin}
\addtolength{\leftmargin}{\tmplength}
\settowidth{\labelsep}{X}
\addtolength{\leftmargin}{\labelsep}
\setlength{\labelwidth}{\tmplength}
}
\begin{flushleft}
\item[\textbf{Declaração}\hfill]
\begin{ttfamily}
public class function AddRec(var DatF : TMI{\_}DataFile ;var R : Longint;var Buffer ):Boolean; overload;\end{ttfamily}


\end{flushleft}
\end{list}
\paragraph*{DeleteRec}\hspace*{\fill}

\begin{list}{}{
\settowidth{\tmplength}{\textbf{Declaração}}
\setlength{\itemindent}{0cm}
\setlength{\listparindent}{0cm}
\setlength{\leftmargin}{\evensidemargin}
\addtolength{\leftmargin}{\tmplength}
\settowidth{\labelsep}{X}
\addtolength{\leftmargin}{\labelsep}
\setlength{\labelwidth}{\tmplength}
}
\begin{flushleft}
\item[\textbf{Declaração}\hfill]
\begin{ttfamily}
public class function DeleteRec(var DatF : TMI{\_}DataFile ;Const R : Longint):Boolean; overload;\end{ttfamily}


\end{flushleft}
\end{list}
\paragraph*{FileLen}\hspace*{\fill}

\begin{list}{}{
\settowidth{\tmplength}{\textbf{Declaração}}
\setlength{\itemindent}{0cm}
\setlength{\listparindent}{0cm}
\setlength{\leftmargin}{\evensidemargin}
\addtolength{\leftmargin}{\tmplength}
\settowidth{\labelsep}{X}
\addtolength{\leftmargin}{\labelsep}
\setlength{\labelwidth}{\tmplength}
}
\begin{flushleft}
\item[\textbf{Declaração}\hfill]
\begin{ttfamily}
public class function FileLen(var DatF : TMI{\_}DataFile ) : Longint; overload;\end{ttfamily}


\end{flushleft}
\end{list}
\paragraph*{MakeIndex}\hspace*{\fill}

\begin{list}{}{
\settowidth{\tmplength}{\textbf{Declaração}}
\setlength{\itemindent}{0cm}
\setlength{\listparindent}{0cm}
\setlength{\leftmargin}{\evensidemargin}
\addtolength{\leftmargin}{\tmplength}
\settowidth{\labelsep}{X}
\addtolength{\leftmargin}{\labelsep}
\setlength{\labelwidth}{\tmplength}
}
\begin{flushleft}
\item[\textbf{Declaração}\hfill]
\begin{ttfamily}
public class function MakeIndex(var IxF : TMI{\_}IndexFile;Exclusivo:Boolean ):Integer; Overload;\end{ttfamily}


\end{flushleft}
\end{list}
\paragraph*{OpenIndex}\hspace*{\fill}

\begin{list}{}{
\settowidth{\tmplength}{\textbf{Declaração}}
\setlength{\itemindent}{0cm}
\setlength{\listparindent}{0cm}
\setlength{\leftmargin}{\evensidemargin}
\addtolength{\leftmargin}{\tmplength}
\settowidth{\labelsep}{X}
\addtolength{\leftmargin}{\labelsep}
\setlength{\labelwidth}{\tmplength}
}
\begin{flushleft}
\item[\textbf{Declaração}\hfill]
\begin{ttfamily}
public class function OpenIndex(var IxF : TMI{\_}IndexFile;Exclusivo :Boolean ):Integer; Overload;\end{ttfamily}


\end{flushleft}
\end{list}
\paragraph*{OpenIndex}\hspace*{\fill}

\begin{list}{}{
\settowidth{\tmplength}{\textbf{Declaração}}
\setlength{\itemindent}{0cm}
\setlength{\listparindent}{0cm}
\setlength{\leftmargin}{\evensidemargin}
\addtolength{\leftmargin}{\tmplength}
\settowidth{\labelsep}{X}
\addtolength{\leftmargin}{\labelsep}
\setlength{\labelwidth}{\tmplength}
}
\begin{flushleft}
\item[\textbf{Declaração}\hfill]
\begin{ttfamily}
public class function OpenIndex(var IxF : TMI{\_}IndexFile;Exclusivo,OkCreate:Boolean ):Integer; Overload;\end{ttfamily}


\end{flushleft}
\end{list}
\paragraph*{ClearKey}\hspace*{\fill}

\begin{list}{}{
\settowidth{\tmplength}{\textbf{Declaração}}
\setlength{\itemindent}{0cm}
\setlength{\listparindent}{0cm}
\setlength{\leftmargin}{\evensidemargin}
\addtolength{\leftmargin}{\tmplength}
\settowidth{\labelsep}{X}
\addtolength{\leftmargin}{\labelsep}
\setlength{\labelwidth}{\tmplength}
}
\begin{flushleft}
\item[\textbf{Declaração}\hfill]
\begin{ttfamily}
public class function ClearKey(var IxF : TMI{\_}IndexFile) :Boolean; overload;\end{ttfamily}


\end{flushleft}
\end{list}
\paragraph*{NextKey}\hspace*{\fill}

\begin{list}{}{
\settowidth{\tmplength}{\textbf{Declaração}}
\setlength{\itemindent}{0cm}
\setlength{\listparindent}{0cm}
\setlength{\leftmargin}{\evensidemargin}
\addtolength{\leftmargin}{\tmplength}
\settowidth{\labelsep}{X}
\addtolength{\leftmargin}{\labelsep}
\setlength{\labelwidth}{\tmplength}
}
\begin{flushleft}
\item[\textbf{Declaração}\hfill]
\begin{ttfamily}
public class function NextKey(var IxF : TMI{\_}IndexFile; var ProcDatRef : Longint; var ProcKey ):Boolean; overload;\end{ttfamily}


\end{flushleft}
\end{list}
\paragraph*{PrevKey}\hspace*{\fill}

\begin{list}{}{
\settowidth{\tmplength}{\textbf{Declaração}}
\setlength{\itemindent}{0cm}
\setlength{\listparindent}{0cm}
\setlength{\leftmargin}{\evensidemargin}
\addtolength{\leftmargin}{\tmplength}
\settowidth{\labelsep}{X}
\addtolength{\leftmargin}{\labelsep}
\setlength{\labelwidth}{\tmplength}
}
\begin{flushleft}
\item[\textbf{Declaração}\hfill]
\begin{ttfamily}
public class function PrevKey(var IxF : TMI{\_}IndexFile; var ProcDatRef : Longint; var ProcKey ):Boolean; overload;\end{ttfamily}


\end{flushleft}
\end{list}
\paragraph*{FindKeyTop}\hspace*{\fill}

\begin{list}{}{
\settowidth{\tmplength}{\textbf{Declaração}}
\setlength{\itemindent}{0cm}
\setlength{\listparindent}{0cm}
\setlength{\leftmargin}{\evensidemargin}
\addtolength{\leftmargin}{\tmplength}
\settowidth{\labelsep}{X}
\addtolength{\leftmargin}{\labelsep}
\setlength{\labelwidth}{\tmplength}
}
\begin{flushleft}
\item[\textbf{Declaração}\hfill]
\begin{ttfamily}
public class function FindKeyTop(var IxF : TMI{\_}IndexFile ; var ProcDatRef : Longint; var ProcKey ):Boolean; overload;\end{ttfamily}


\end{flushleft}
\end{list}
\paragraph*{FindKey}\hspace*{\fill}

\begin{list}{}{
\settowidth{\tmplength}{\textbf{Declaração}}
\setlength{\itemindent}{0cm}
\setlength{\listparindent}{0cm}
\setlength{\leftmargin}{\evensidemargin}
\addtolength{\leftmargin}{\tmplength}
\settowidth{\labelsep}{X}
\addtolength{\leftmargin}{\labelsep}
\setlength{\labelwidth}{\tmplength}
}
\begin{flushleft}
\item[\textbf{Declaração}\hfill]
\begin{ttfamily}
public class function FindKey(var IxF : TMI{\_}IndexFile; var ProcDatRef : Longint; var ProcKey ):Boolean; overload;\end{ttfamily}


\end{flushleft}
\end{list}
\paragraph*{SearchKey}\hspace*{\fill}

\begin{list}{}{
\settowidth{\tmplength}{\textbf{Declaração}}
\setlength{\itemindent}{0cm}
\setlength{\listparindent}{0cm}
\setlength{\leftmargin}{\evensidemargin}
\addtolength{\leftmargin}{\tmplength}
\settowidth{\labelsep}{X}
\addtolength{\leftmargin}{\labelsep}
\setlength{\labelwidth}{\tmplength}
}
\begin{flushleft}
\item[\textbf{Declaração}\hfill]
\begin{ttfamily}
public class function SearchKey(var IxF : TMI{\_}IndexFile; var ProcDatRef : Longint; var ProcKey:TaKeyStr ):Boolean; overload;\end{ttfamily}


\end{flushleft}
\end{list}
\paragraph*{SearchKeyTop}\hspace*{\fill}

\begin{list}{}{
\settowidth{\tmplength}{\textbf{Declaração}}
\setlength{\itemindent}{0cm}
\setlength{\listparindent}{0cm}
\setlength{\leftmargin}{\evensidemargin}
\addtolength{\leftmargin}{\tmplength}
\settowidth{\labelsep}{X}
\addtolength{\leftmargin}{\labelsep}
\setlength{\labelwidth}{\tmplength}
}
\begin{flushleft}
\item[\textbf{Declaração}\hfill]
\begin{ttfamily}
public class function SearchKeyTop(var IxF : TMI{\_}IndexFile ; var ProcDatRef : Longint; var ProcKey :TaKeyStr; Const Okequal : Boolean ):Boolean; overload;\end{ttfamily}


\end{flushleft}
\end{list}
\paragraph*{AddKey}\hspace*{\fill}

\begin{list}{}{
\settowidth{\tmplength}{\textbf{Declaração}}
\setlength{\itemindent}{0cm}
\setlength{\listparindent}{0cm}
\setlength{\leftmargin}{\evensidemargin}
\addtolength{\leftmargin}{\tmplength}
\settowidth{\labelsep}{X}
\addtolength{\leftmargin}{\labelsep}
\setlength{\labelwidth}{\tmplength}
}
\begin{flushleft}
\item[\textbf{Declaração}\hfill]
\begin{ttfamily}
public class function AddKey(var IxF : TMI{\_}IndexFile; Const ProcDatRef : Longint; Const ProcKey : TaKeyStr ):Boolean; overload;\end{ttfamily}


\end{flushleft}
\end{list}
\paragraph*{DeleteKey}\hspace*{\fill}

\begin{list}{}{
\settowidth{\tmplength}{\textbf{Declaração}}
\setlength{\itemindent}{0cm}
\setlength{\listparindent}{0cm}
\setlength{\leftmargin}{\evensidemargin}
\addtolength{\leftmargin}{\tmplength}
\settowidth{\labelsep}{X}
\addtolength{\leftmargin}{\labelsep}
\setlength{\labelwidth}{\tmplength}
}
\begin{flushleft}
\item[\textbf{Declaração}\hfill]
\begin{ttfamily}
public class function DeleteKey(var IxF : TMI{\_}IndexFile; Const ProcDatRef : Longint; Const ProcKey : TaKeyStr):Boolean; overload;\end{ttfamily}


\end{flushleft}
\end{list}
\paragraph*{FlushFile}\hspace*{\fill}

\begin{list}{}{
\settowidth{\tmplength}{\textbf{Declaração}}
\setlength{\itemindent}{0cm}
\setlength{\listparindent}{0cm}
\setlength{\leftmargin}{\evensidemargin}
\addtolength{\leftmargin}{\tmplength}
\settowidth{\labelsep}{X}
\addtolength{\leftmargin}{\labelsep}
\setlength{\labelwidth}{\tmplength}
}
\begin{flushleft}
\item[\textbf{Declaração}\hfill]
\begin{ttfamily}
public class procedure FlushFile(var DatF :TMI{\_}DataFile ); overload;\end{ttfamily}


\end{flushleft}
\end{list}
\paragraph*{FlushIndex}\hspace*{\fill}

\begin{list}{}{
\settowidth{\tmplength}{\textbf{Declaração}}
\setlength{\itemindent}{0cm}
\setlength{\listparindent}{0cm}
\setlength{\leftmargin}{\evensidemargin}
\addtolength{\leftmargin}{\tmplength}
\settowidth{\labelsep}{X}
\addtolength{\leftmargin}{\labelsep}
\setlength{\labelwidth}{\tmplength}
}
\begin{flushleft}
\item[\textbf{Declaração}\hfill]
\begin{ttfamily}
public class procedure FlushIndex(var IxF : TMI{\_}IndexFile ); overload;\end{ttfamily}


\end{flushleft}
\end{list}
\paragraph*{Seek}\hspace*{\fill}

\begin{list}{}{
\settowidth{\tmplength}{\textbf{Declaração}}
\setlength{\itemindent}{0cm}
\setlength{\listparindent}{0cm}
\setlength{\leftmargin}{\evensidemargin}
\addtolength{\leftmargin}{\tmplength}
\settowidth{\labelsep}{X}
\addtolength{\leftmargin}{\labelsep}
\setlength{\labelwidth}{\tmplength}
}
\begin{flushleft}
\item[\textbf{Declaração}\hfill]
\begin{ttfamily}
public class function Seek(Var DatF : TMI{\_}DataFile;Const R : Longint ):SmallInt; overload;\end{ttfamily}


\end{flushleft}
\end{list}
\paragraph*{CloseFilesOpens}\hspace*{\fill}

\begin{list}{}{
\settowidth{\tmplength}{\textbf{Declaração}}
\setlength{\itemindent}{0cm}
\setlength{\listparindent}{0cm}
\setlength{\leftmargin}{\evensidemargin}
\addtolength{\leftmargin}{\tmplength}
\settowidth{\labelsep}{X}
\addtolength{\leftmargin}{\labelsep}
\setlength{\labelwidth}{\tmplength}
}
\begin{flushleft}
\item[\textbf{Declaração}\hfill]
\begin{ttfamily}
public class Procedure CloseFilesOpens; virtual;\end{ttfamily}


\end{flushleft}
\end{list}
\paragraph*{MyDestroyMemory}\hspace*{\fill}

\begin{list}{}{
\settowidth{\tmplength}{\textbf{Declaração}}
\setlength{\itemindent}{0cm}
\setlength{\listparindent}{0cm}
\setlength{\leftmargin}{\evensidemargin}
\addtolength{\leftmargin}{\tmplength}
\settowidth{\labelsep}{X}
\addtolength{\leftmargin}{\labelsep}
\setlength{\labelwidth}{\tmplength}
}
\begin{flushleft}
\item[\textbf{Declaração}\hfill]
\begin{ttfamily}
public class Procedure MyDestroyMemory;\end{ttfamily}


\end{flushleft}
\end{list}
\paragraph*{MyCreateMemory}\hspace*{\fill}

\begin{list}{}{
\settowidth{\tmplength}{\textbf{Declaração}}
\setlength{\itemindent}{0cm}
\setlength{\listparindent}{0cm}
\setlength{\leftmargin}{\evensidemargin}
\addtolength{\leftmargin}{\tmplength}
\settowidth{\labelsep}{X}
\addtolength{\leftmargin}{\labelsep}
\setlength{\labelwidth}{\tmplength}
}
\begin{flushleft}
\item[\textbf{Declaração}\hfill]
\begin{ttfamily}
public class Procedure MyCreateMemory;\end{ttfamily}


\end{flushleft}
\end{list}
\paragraph*{MyDestroyMemorySemVideo}\hspace*{\fill}

\begin{list}{}{
\settowidth{\tmplength}{\textbf{Declaração}}
\setlength{\itemindent}{0cm}
\setlength{\listparindent}{0cm}
\setlength{\leftmargin}{\evensidemargin}
\addtolength{\leftmargin}{\tmplength}
\settowidth{\labelsep}{X}
\addtolength{\leftmargin}{\labelsep}
\setlength{\labelwidth}{\tmplength}
}
\begin{flushleft}
\item[\textbf{Declaração}\hfill]
\begin{ttfamily}
public class Procedure MyDestroyMemorySemVideo;\end{ttfamily}


\end{flushleft}
\end{list}
\paragraph*{MyCreateMemorySemVideo}\hspace*{\fill}

\begin{list}{}{
\settowidth{\tmplength}{\textbf{Declaração}}
\setlength{\itemindent}{0cm}
\setlength{\listparindent}{0cm}
\setlength{\leftmargin}{\evensidemargin}
\addtolength{\leftmargin}{\tmplength}
\settowidth{\labelsep}{X}
\addtolength{\leftmargin}{\labelsep}
\setlength{\labelwidth}{\tmplength}
}
\begin{flushleft}
\item[\textbf{Declaração}\hfill]
\begin{ttfamily}
public class Procedure MyCreateMemorySemVideo;\end{ttfamily}


\end{flushleft}
\end{list}
\paragraph*{ExecCommand}\hspace*{\fill}

\begin{list}{}{
\settowidth{\tmplength}{\textbf{Declaração}}
\setlength{\itemindent}{0cm}
\setlength{\listparindent}{0cm}
\setlength{\leftmargin}{\evensidemargin}
\addtolength{\leftmargin}{\tmplength}
\settowidth{\labelsep}{X}
\addtolength{\leftmargin}{\labelsep}
\setlength{\labelwidth}{\tmplength}
}
\begin{flushleft}
\item[\textbf{Declaração}\hfill]
\begin{ttfamily}
public class function ExecCommand(FileName:PathStr;Flags: Longint;aExecAsync : Longint): Byte; Overload;\end{ttfamily}


\end{flushleft}
\end{list}
\paragraph*{ExecCommand}\hspace*{\fill}

\begin{list}{}{
\settowidth{\tmplength}{\textbf{Declaração}}
\setlength{\itemindent}{0cm}
\setlength{\listparindent}{0cm}
\setlength{\leftmargin}{\evensidemargin}
\addtolength{\leftmargin}{\tmplength}
\settowidth{\labelsep}{X}
\addtolength{\leftmargin}{\labelsep}
\setlength{\labelwidth}{\tmplength}
}
\begin{flushleft}
\item[\textbf{Declaração}\hfill]
\begin{ttfamily}
public class function ExecCommand(FileName:PathStr;Flags: Longint): Byte; Overload;\end{ttfamily}


\end{flushleft}
\end{list}
\paragraph*{ExecDos}\hspace*{\fill}

\begin{list}{}{
\settowidth{\tmplength}{\textbf{Declaração}}
\setlength{\itemindent}{0cm}
\setlength{\listparindent}{0cm}
\setlength{\leftmargin}{\evensidemargin}
\addtolength{\leftmargin}{\tmplength}
\settowidth{\labelsep}{X}
\addtolength{\leftmargin}{\labelsep}
\setlength{\labelwidth}{\tmplength}
}
\begin{flushleft}
\item[\textbf{Declaração}\hfill]
\begin{ttfamily}
public class function ExecDos(Const Path: PathStr; Const ComLine: ComStr): Byte;\end{ttfamily}


\end{flushleft}
\par
\item[\textbf{Descrição}]
A classe método \textbf{\begin{ttfamily}ExecDos\end{ttfamily}} executa um programa externo de form assíncrona.

\begin{itemize}
\item \textbf{EXEMPLO}

\texttt{\\\nopagebreak[3]
\\\nopagebreak[3]
ExecDos('/usr/bin/gnome-terminal','ls');\\
}
\end{itemize}

\end{list}
\paragraph*{FindKey}\hspace*{\fill}

\begin{list}{}{
\settowidth{\tmplength}{\textbf{Declaração}}
\setlength{\itemindent}{0cm}
\setlength{\listparindent}{0cm}
\setlength{\leftmargin}{\evensidemargin}
\addtolength{\leftmargin}{\tmplength}
\settowidth{\labelsep}{X}
\addtolength{\leftmargin}{\labelsep}
\setlength{\labelwidth}{\tmplength}
}
\begin{flushleft}
\item[\textbf{Declaração}\hfill]
\begin{ttfamily}
public class function FindKey(var IxF : TMI{\_}IndexFile; var ProcDatRef : Longint; ProcKey : TaKeyStr):Boolean ; overload;\end{ttfamily}


\end{flushleft}
\end{list}
\paragraph*{AdicioneChave}\hspace*{\fill}

\begin{list}{}{
\settowidth{\tmplength}{\textbf{Declaração}}
\setlength{\itemindent}{0cm}
\setlength{\listparindent}{0cm}
\setlength{\leftmargin}{\evensidemargin}
\addtolength{\leftmargin}{\tmplength}
\settowidth{\labelsep}{X}
\addtolength{\leftmargin}{\labelsep}
\setlength{\labelwidth}{\tmplength}
}
\begin{flushleft}
\item[\textbf{Declaração}\hfill]
\begin{ttfamily}
public class function AdicioneChave(var IxF : TMI{\_}IndexFile ; Const ProcDatRef : Longint; Const ProcKey : TaKeyStr):Boolean;\end{ttfamily}


\end{flushleft}
\end{list}
\paragraph*{EliminaChave}\hspace*{\fill}

\begin{list}{}{
\settowidth{\tmplength}{\textbf{Declaração}}
\setlength{\itemindent}{0cm}
\setlength{\listparindent}{0cm}
\setlength{\leftmargin}{\evensidemargin}
\addtolength{\leftmargin}{\tmplength}
\settowidth{\labelsep}{X}
\addtolength{\leftmargin}{\labelsep}
\setlength{\labelwidth}{\tmplength}
}
\begin{flushleft}
\item[\textbf{Declaração}\hfill]
\begin{ttfamily}
public class function EliminaChave(var IxF : TMI{\_}IndexFile ; Const ProcDatRef : Longint; Const ProcKey : TaKeyStr):Boolean;\end{ttfamily}


\end{flushleft}
\end{list}
\paragraph*{NomeDaEstacao}\hspace*{\fill}

\begin{list}{}{
\settowidth{\tmplength}{\textbf{Declaração}}
\setlength{\itemindent}{0cm}
\setlength{\listparindent}{0cm}
\setlength{\leftmargin}{\evensidemargin}
\addtolength{\leftmargin}{\tmplength}
\settowidth{\labelsep}{X}
\addtolength{\leftmargin}{\labelsep}
\setlength{\labelwidth}{\tmplength}
}
\begin{flushleft}
\item[\textbf{Declaração}\hfill]
\begin{ttfamily}
public class function NomeDaEstacao:tString;\end{ttfamily}


\end{flushleft}
\end{list}
\paragraph*{ValidFileName}\hspace*{\fill}

\begin{list}{}{
\settowidth{\tmplength}{\textbf{Declaração}}
\setlength{\itemindent}{0cm}
\setlength{\listparindent}{0cm}
\setlength{\leftmargin}{\evensidemargin}
\addtolength{\leftmargin}{\tmplength}
\settowidth{\labelsep}{X}
\addtolength{\leftmargin}{\labelsep}
\setlength{\labelwidth}{\tmplength}
}
\begin{flushleft}
\item[\textbf{Declaração}\hfill]
\begin{ttfamily}
public class function ValidFileName(Const Name : PathStr):Byte;\end{ttfamily}


\end{flushleft}
\end{list}
\paragraph*{FConcatNomeArq}\hspace*{\fill}

\begin{list}{}{
\settowidth{\tmplength}{\textbf{Declaração}}
\setlength{\itemindent}{0cm}
\setlength{\listparindent}{0cm}
\setlength{\leftmargin}{\evensidemargin}
\addtolength{\leftmargin}{\tmplength}
\settowidth{\labelsep}{X}
\addtolength{\leftmargin}{\labelsep}
\setlength{\labelwidth}{\tmplength}
}
\begin{flushleft}
\item[\textbf{Declaração}\hfill]
\begin{ttfamily}
public class function FConcatNomeArq(Nome,Extencao:PathStr) : PathStr;\end{ttfamily}


\end{flushleft}
\end{list}
\paragraph*{CriaArqTemp}\hspace*{\fill}

\begin{list}{}{
\settowidth{\tmplength}{\textbf{Declaração}}
\setlength{\itemindent}{0cm}
\setlength{\listparindent}{0cm}
\setlength{\leftmargin}{\evensidemargin}
\addtolength{\leftmargin}{\tmplength}
\settowidth{\labelsep}{X}
\addtolength{\leftmargin}{\labelsep}
\setlength{\labelwidth}{\tmplength}
}
\begin{flushleft}
\item[\textbf{Declaração}\hfill]
\begin{ttfamily}
public class function CriaArqTemp(Var ArqF : TMI{\_}DataFile; Const TamArqTemp : SmallWord; Const NumeroDeIndice : Byte ):Boolean;\end{ttfamily}


\end{flushleft}
\end{list}
\paragraph*{CriaArqTempI}\hspace*{\fill}

\begin{list}{}{
\settowidth{\tmplength}{\textbf{Declaração}}
\setlength{\itemindent}{0cm}
\setlength{\listparindent}{0cm}
\setlength{\leftmargin}{\evensidemargin}
\addtolength{\leftmargin}{\tmplength}
\settowidth{\labelsep}{X}
\addtolength{\leftmargin}{\labelsep}
\setlength{\labelwidth}{\tmplength}
}
\begin{flushleft}
\item[\textbf{Declaração}\hfill]
\begin{ttfamily}
public class function CriaArqTempI(Var IxF: TMI{\_}IndexFile; Const RepeteChave : Byte; Const EndChaveNoRegistro: tString):Boolean;\end{ttfamily}


\end{flushleft}
\end{list}
\paragraph*{EspacoEmDisco}\hspace*{\fill}

\begin{list}{}{
\settowidth{\tmplength}{\textbf{Declaração}}
\setlength{\itemindent}{0cm}
\setlength{\listparindent}{0cm}
\setlength{\leftmargin}{\evensidemargin}
\addtolength{\leftmargin}{\tmplength}
\settowidth{\labelsep}{X}
\addtolength{\leftmargin}{\labelsep}
\setlength{\labelwidth}{\tmplength}
}
\begin{flushleft}
\item[\textbf{Declaração}\hfill]
\begin{ttfamily}
public class function EspacoEmDisco(NomeFonte,DriveDestino:PathStr) :boolean;\end{ttfamily}


\end{flushleft}
\end{list}
\paragraph*{TTraveRegistro}\hspace*{\fill}

\begin{list}{}{
\settowidth{\tmplength}{\textbf{Declaração}}
\setlength{\itemindent}{0cm}
\setlength{\listparindent}{0cm}
\setlength{\leftmargin}{\evensidemargin}
\addtolength{\leftmargin}{\tmplength}
\settowidth{\labelsep}{X}
\addtolength{\leftmargin}{\labelsep}
\setlength{\labelwidth}{\tmplength}
}
\begin{flushleft}
\item[\textbf{Declaração}\hfill]
\begin{ttfamily}
public class function TTraveRegistro(Var DatF : TMI{\_}DataFile;Const R : Longint):Boolean;\end{ttfamily}


\end{flushleft}
\end{list}
\paragraph*{TDestraveRegistro}\hspace*{\fill}

\begin{list}{}{
\settowidth{\tmplength}{\textbf{Declaração}}
\setlength{\itemindent}{0cm}
\setlength{\listparindent}{0cm}
\setlength{\leftmargin}{\evensidemargin}
\addtolength{\leftmargin}{\tmplength}
\settowidth{\labelsep}{X}
\addtolength{\leftmargin}{\labelsep}
\setlength{\labelwidth}{\tmplength}
}
\begin{flushleft}
\item[\textbf{Declaração}\hfill]
\begin{ttfamily}
public class function TDestraveRegistro(Var DatF : TMI{\_}DataFile;Const R : Longint):Boolean;\end{ttfamily}


\end{flushleft}
\end{list}
\paragraph*{FPackDataFile}\hspace*{\fill}

\begin{list}{}{
\settowidth{\tmplength}{\textbf{Declaração}}
\setlength{\itemindent}{0cm}
\setlength{\listparindent}{0cm}
\setlength{\leftmargin}{\evensidemargin}
\addtolength{\leftmargin}{\tmplength}
\settowidth{\labelsep}{X}
\addtolength{\leftmargin}{\labelsep}
\setlength{\labelwidth}{\tmplength}
}
\begin{flushleft}
\item[\textbf{Declaração}\hfill]
\begin{ttfamily}
public class function FPackDataFile(NomeArq :PathStr):Boolean;\end{ttfamily}


\end{flushleft}
\end{list}
\paragraph*{FLeiaGrave}\hspace*{\fill}

\begin{list}{}{
\settowidth{\tmplength}{\textbf{Declaração}}
\setlength{\itemindent}{0cm}
\setlength{\listparindent}{0cm}
\setlength{\leftmargin}{\evensidemargin}
\addtolength{\leftmargin}{\tmplength}
\settowidth{\labelsep}{X}
\addtolength{\leftmargin}{\labelsep}
\setlength{\labelwidth}{\tmplength}
}
\begin{flushleft}
\item[\textbf{Declaração}\hfill]
\begin{ttfamily}
public class function FLeiaGrave(Const MsgStr : tString; Const NomeArqDados : PathStr; Var RegBuff ; Const TamRegBuff : SmallWord; Const OkFunc : TipoFuncao) : Boolean;\end{ttfamily}


\end{flushleft}
\end{list}
\paragraph*{LeiaGrave}\hspace*{\fill}

\begin{list}{}{
\settowidth{\tmplength}{\textbf{Declaração}}
\setlength{\itemindent}{0cm}
\setlength{\listparindent}{0cm}
\setlength{\leftmargin}{\evensidemargin}
\addtolength{\leftmargin}{\tmplength}
\settowidth{\labelsep}{X}
\addtolength{\leftmargin}{\labelsep}
\setlength{\labelwidth}{\tmplength}
}
\begin{flushleft}
\item[\textbf{Declaração}\hfill]
\begin{ttfamily}
public class function LeiaGrave(Const MsgStr : tString; Var ArqDados : TMI{\_}DataFile; Var RegBuff ; Const OkFunc : TipoFuncao) : Boolean;\end{ttfamily}


\end{flushleft}
\end{list}
\paragraph*{FLeieEGraveRegistro}\hspace*{\fill}

\begin{list}{}{
\settowidth{\tmplength}{\textbf{Declaração}}
\setlength{\itemindent}{0cm}
\setlength{\listparindent}{0cm}
\setlength{\leftmargin}{\evensidemargin}
\addtolength{\leftmargin}{\tmplength}
\settowidth{\labelsep}{X}
\addtolength{\leftmargin}{\labelsep}
\setlength{\labelwidth}{\tmplength}
}
\begin{flushleft}
\item[\textbf{Declaração}\hfill]
\begin{ttfamily}
public class function FLeieEGraveRegistro(Const NomeFonte:PathStr; Var RegFonte; Const TamFonte : SmallWord; Const NomeDestino : PathStr; Var regDestino; Const TamDestino : SmallWord; AtualizaDestino : TipoFuncao; Const OkMakeFile :Boolean) : Boolean;\end{ttfamily}


\end{flushleft}
\end{list}
\paragraph*{StrDataEmQueFoiAlterado}\hspace*{\fill}

\begin{list}{}{
\settowidth{\tmplength}{\textbf{Declaração}}
\setlength{\itemindent}{0cm}
\setlength{\listparindent}{0cm}
\setlength{\leftmargin}{\evensidemargin}
\addtolength{\leftmargin}{\tmplength}
\settowidth{\labelsep}{X}
\addtolength{\leftmargin}{\labelsep}
\setlength{\labelwidth}{\tmplength}
}
\begin{flushleft}
\item[\textbf{Declaração}\hfill]
\begin{ttfamily}
public class function StrDataEmQueFoiAlterado(Const NomeArquivo :PathStr) : tString;\end{ttfamily}


\end{flushleft}
\end{list}
\paragraph*{StrDateFile}\hspace*{\fill}

\begin{list}{}{
\settowidth{\tmplength}{\textbf{Declaração}}
\setlength{\itemindent}{0cm}
\setlength{\listparindent}{0cm}
\setlength{\leftmargin}{\evensidemargin}
\addtolength{\leftmargin}{\tmplength}
\settowidth{\labelsep}{X}
\addtolength{\leftmargin}{\labelsep}
\setlength{\labelwidth}{\tmplength}
}
\begin{flushleft}
\item[\textbf{Declaração}\hfill]
\begin{ttfamily}
public class function StrDateFile(Const NomeArquivo : PathStr;Const Ch:AnsiChar) : tString;\end{ttfamily}


\end{flushleft}
\end{list}
\paragraph*{CreateLst}\hspace*{\fill}

\begin{list}{}{
\settowidth{\tmplength}{\textbf{Declaração}}
\setlength{\itemindent}{0cm}
\setlength{\listparindent}{0cm}
\setlength{\leftmargin}{\evensidemargin}
\addtolength{\leftmargin}{\tmplength}
\settowidth{\labelsep}{X}
\addtolength{\leftmargin}{\labelsep}
\setlength{\labelwidth}{\tmplength}
}
\begin{flushleft}
\item[\textbf{Declaração}\hfill]
\begin{ttfamily}
public class function CreateLst:Boolean;\end{ttfamily}


\end{flushleft}
\end{list}
\paragraph*{DestroyLst}\hspace*{\fill}

\begin{list}{}{
\settowidth{\tmplength}{\textbf{Declaração}}
\setlength{\itemindent}{0cm}
\setlength{\listparindent}{0cm}
\setlength{\leftmargin}{\evensidemargin}
\addtolength{\leftmargin}{\tmplength}
\settowidth{\labelsep}{X}
\addtolength{\leftmargin}{\labelsep}
\setlength{\labelwidth}{\tmplength}
}
\begin{flushleft}
\item[\textbf{Declaração}\hfill]
\begin{ttfamily}
public class Procedure DestroyLst;\end{ttfamily}


\end{flushleft}
\end{list}
\paragraph*{GeraSequencia}\hspace*{\fill}

\begin{list}{}{
\settowidth{\tmplength}{\textbf{Declaração}}
\setlength{\itemindent}{0cm}
\setlength{\listparindent}{0cm}
\setlength{\leftmargin}{\evensidemargin}
\addtolength{\leftmargin}{\tmplength}
\settowidth{\labelsep}{X}
\addtolength{\leftmargin}{\labelsep}
\setlength{\labelwidth}{\tmplength}
}
\begin{flushleft}
\item[\textbf{Declaração}\hfill]
\begin{ttfamily}
public class function GeraSequencia(Var ArqI:IndexFile) :Longint;\end{ttfamily}


\end{flushleft}
\end{list}
\paragraph*{TestaAberturaDeArquivo}\hspace*{\fill}

\begin{list}{}{
\settowidth{\tmplength}{\textbf{Declaração}}
\setlength{\itemindent}{0cm}
\setlength{\listparindent}{0cm}
\setlength{\leftmargin}{\evensidemargin}
\addtolength{\leftmargin}{\tmplength}
\settowidth{\labelsep}{X}
\addtolength{\leftmargin}{\labelsep}
\setlength{\labelwidth}{\tmplength}
}
\begin{flushleft}
\item[\textbf{Declaração}\hfill]
\begin{ttfamily}
public class function TestaAberturaDeArquivo(MaxFile : Byte; Var NumMaxPossivel:Byte ): Boolean;\end{ttfamily}


\end{flushleft}
\end{list}
\paragraph*{AssingLst}\hspace*{\fill}

\begin{list}{}{
\settowidth{\tmplength}{\textbf{Declaração}}
\setlength{\itemindent}{0cm}
\setlength{\listparindent}{0cm}
\setlength{\leftmargin}{\evensidemargin}
\addtolength{\leftmargin}{\tmplength}
\settowidth{\labelsep}{X}
\addtolength{\leftmargin}{\labelsep}
\setlength{\labelwidth}{\tmplength}
}
\begin{flushleft}
\item[\textbf{Declaração}\hfill]
\begin{ttfamily}
public class function AssingLst(Const WopcaoRedireciona : AnsiChar; Const aNomeRedireciona : PathStr):Boolean;\end{ttfamily}


\end{flushleft}
\end{list}
\paragraph*{redirecionaParaNul}\hspace*{\fill}

\begin{list}{}{
\settowidth{\tmplength}{\textbf{Declaração}}
\setlength{\itemindent}{0cm}
\setlength{\listparindent}{0cm}
\setlength{\leftmargin}{\evensidemargin}
\addtolength{\leftmargin}{\tmplength}
\settowidth{\labelsep}{X}
\addtolength{\leftmargin}{\labelsep}
\setlength{\labelwidth}{\tmplength}
}
\begin{flushleft}
\item[\textbf{Declaração}\hfill]
\begin{ttfamily}
public class procedure redirecionaParaNul;\end{ttfamily}


\end{flushleft}
\end{list}
\paragraph*{Create}\hspace*{\fill}

\begin{list}{}{
\settowidth{\tmplength}{\textbf{Declaração}}
\setlength{\itemindent}{0cm}
\setlength{\listparindent}{0cm}
\setlength{\leftmargin}{\evensidemargin}
\addtolength{\leftmargin}{\tmplength}
\settowidth{\labelsep}{X}
\addtolength{\leftmargin}{\labelsep}
\setlength{\labelwidth}{\tmplength}
}
\begin{flushleft}
\item[\textbf{Declaração}\hfill]
\begin{ttfamily}
public class Procedure Create;\end{ttfamily}


\end{flushleft}
\end{list}
\paragraph*{Destroy}\hspace*{\fill}

\begin{list}{}{
\settowidth{\tmplength}{\textbf{Declaração}}
\setlength{\itemindent}{0cm}
\setlength{\listparindent}{0cm}
\setlength{\leftmargin}{\evensidemargin}
\addtolength{\leftmargin}{\tmplength}
\settowidth{\labelsep}{X}
\addtolength{\leftmargin}{\labelsep}
\setlength{\labelwidth}{\tmplength}
}
\begin{flushleft}
\item[\textbf{Declaração}\hfill]
\begin{ttfamily}
public class Procedure Destroy;\end{ttfamily}


\end{flushleft}
\end{list}
\chapter{Unit mi.rtl.objects.methods.db.tb{\_}{\_}access{\_}test}
\section{Uses}
\begin{itemize}
\item \begin{ttfamily}Classes\end{ttfamily}\item \begin{ttfamily}SysUtils\end{ttfamily}\item \begin{ttfamily}mi.rtl.Consts\end{ttfamily}(\ref{mi.rtl.Consts})\item \begin{ttfamily}mi.rtl.objectss\end{ttfamily}(\ref{mi.rtl.Objectss})\end{itemize}
\section{Visão Geral}
\begin{description}
\item[\texttt{\begin{ttfamily}TAluno\end{ttfamily} Classe}]
\end{description}
\section{Classes, Interfaces, Objetos e Registros}
\subsection*{TAluno Classe}
\subsubsection*{\large{\textbf{Hierarquia}}\normalsize\hspace{1ex}\hfill}
TAluno {$>$} TObject
\subsubsection*{\large{\textbf{Descrição}}\normalsize\hspace{1ex}\hfill}
A class \textbf{\begin{ttfamily}TAluno\end{ttfamily}} desmonstra o uso da classe \begin{ttfamily}TObjectss.Ttb{\_}{\_}access\end{ttfamily}(\ref{mi.rtl.Objectss.TObjectss-Ttb__access})\subsubsection*{\large{\textbf{Campos}}\normalsize\hspace{1ex}\hfill}
\paragraph*{NR{\_}Bof}\hspace*{\fill}

\begin{list}{}{
\settowidth{\tmplength}{\textbf{Declaração}}
\setlength{\itemindent}{0cm}
\setlength{\listparindent}{0cm}
\setlength{\leftmargin}{\evensidemargin}
\addtolength{\leftmargin}{\tmplength}
\settowidth{\labelsep}{X}
\addtolength{\leftmargin}{\labelsep}
\setlength{\labelwidth}{\tmplength}
}
\begin{flushleft}
\item[\textbf{Declaração}\hfill]
\begin{ttfamily}
public const NR{\_}Bof      : Longint = 0 ;\end{ttfamily}


\end{flushleft}
\end{list}
\paragraph*{NR{\_}Current}\hspace*{\fill}

\begin{list}{}{
\settowidth{\tmplength}{\textbf{Declaração}}
\setlength{\itemindent}{0cm}
\setlength{\listparindent}{0cm}
\setlength{\leftmargin}{\evensidemargin}
\addtolength{\leftmargin}{\tmplength}
\settowidth{\labelsep}{X}
\addtolength{\leftmargin}{\labelsep}
\setlength{\labelwidth}{\tmplength}
}
\begin{flushleft}
\item[\textbf{Declaração}\hfill]
\begin{ttfamily}
public const NR{\_}Current  : Longint = 0 ;\end{ttfamily}


\end{flushleft}
\end{list}
\paragraph*{NR{\_}Eof}\hspace*{\fill}

\begin{list}{}{
\settowidth{\tmplength}{\textbf{Declaração}}
\setlength{\itemindent}{0cm}
\setlength{\listparindent}{0cm}
\setlength{\leftmargin}{\evensidemargin}
\addtolength{\leftmargin}{\tmplength}
\settowidth{\labelsep}{X}
\addtolength{\leftmargin}{\labelsep}
\setlength{\labelwidth}{\tmplength}
}
\begin{flushleft}
\item[\textbf{Declaração}\hfill]
\begin{ttfamily}
public const NR{\_}Eof     : Longint = 0 ;\end{ttfamily}


\end{flushleft}
\end{list}
\paragraph*{OkBof}\hspace*{\fill}

\begin{list}{}{
\settowidth{\tmplength}{\textbf{Declaração}}
\setlength{\itemindent}{0cm}
\setlength{\listparindent}{0cm}
\setlength{\leftmargin}{\evensidemargin}
\addtolength{\leftmargin}{\tmplength}
\settowidth{\labelsep}{X}
\addtolength{\leftmargin}{\labelsep}
\setlength{\labelwidth}{\tmplength}
}
\begin{flushleft}
\item[\textbf{Declaração}\hfill]
\begin{ttfamily}
public const OkBof : boolean = true;\end{ttfamily}


\end{flushleft}
\end{list}
\paragraph*{OkEof}\hspace*{\fill}

\begin{list}{}{
\settowidth{\tmplength}{\textbf{Declaração}}
\setlength{\itemindent}{0cm}
\setlength{\listparindent}{0cm}
\setlength{\leftmargin}{\evensidemargin}
\addtolength{\leftmargin}{\tmplength}
\settowidth{\labelsep}{X}
\addtolength{\leftmargin}{\labelsep}
\setlength{\labelwidth}{\tmplength}
}
\begin{flushleft}
\item[\textbf{Declaração}\hfill]
\begin{ttfamily}
public const OkEof : Boolean = false;\end{ttfamily}


\end{flushleft}
\end{list}
\paragraph*{RecordSelected}\hspace*{\fill}

\begin{list}{}{
\settowidth{\tmplength}{\textbf{Declaração}}
\setlength{\itemindent}{0cm}
\setlength{\listparindent}{0cm}
\setlength{\leftmargin}{\evensidemargin}
\addtolength{\leftmargin}{\tmplength}
\settowidth{\labelsep}{X}
\addtolength{\leftmargin}{\labelsep}
\setlength{\labelwidth}{\tmplength}
}
\begin{flushleft}
\item[\textbf{Declaração}\hfill]
\begin{ttfamily}
public const RecordSelected : Boolean = false;\end{ttfamily}


\end{flushleft}
\end{list}
\paragraph*{ArqAluno}\hspace*{\fill}

\begin{list}{}{
\settowidth{\tmplength}{\textbf{Declaração}}
\setlength{\itemindent}{0cm}
\setlength{\listparindent}{0cm}
\setlength{\leftmargin}{\evensidemargin}
\addtolength{\leftmargin}{\tmplength}
\settowidth{\labelsep}{X}
\addtolength{\leftmargin}{\labelsep}
\setlength{\labelwidth}{\tmplength}
}
\begin{flushleft}
\item[\textbf{Declaração}\hfill]
\begin{ttfamily}
public var ArqAluno: TObjectss.Ttb{\_}{\_}access.TMI{\_}DataFile; static;\end{ttfamily}


\end{flushleft}
\end{list}
\paragraph*{RegAluno}\hspace*{\fill}

\begin{list}{}{
\settowidth{\tmplength}{\textbf{Declaração}}
\setlength{\itemindent}{0cm}
\setlength{\listparindent}{0cm}
\setlength{\leftmargin}{\evensidemargin}
\addtolength{\leftmargin}{\tmplength}
\settowidth{\labelsep}{X}
\addtolength{\leftmargin}{\labelsep}
\setlength{\labelwidth}{\tmplength}
}
\begin{flushleft}
\item[\textbf{Declaração}\hfill]
\begin{ttfamily}
public var RegAluno: TRegAluno; static;\end{ttfamily}


\end{flushleft}
\end{list}
\subsubsection*{\large{\textbf{Métodos}}\normalsize\hspace{1ex}\hfill}
\paragraph*{Init{\_}ArqAluno}\hspace*{\fill}

\begin{list}{}{
\settowidth{\tmplength}{\textbf{Declaração}}
\setlength{\itemindent}{0cm}
\setlength{\listparindent}{0cm}
\setlength{\leftmargin}{\evensidemargin}
\addtolength{\leftmargin}{\tmplength}
\settowidth{\labelsep}{X}
\addtolength{\leftmargin}{\labelsep}
\setlength{\labelwidth}{\tmplength}
}
\begin{flushleft}
\item[\textbf{Declaração}\hfill]
\begin{ttfamily}
public class procedure Init{\_}ArqAluno;\end{ttfamily}


\end{flushleft}
\end{list}
\paragraph*{Create{\_}ArqAluno}\hspace*{\fill}

\begin{list}{}{
\settowidth{\tmplength}{\textbf{Declaração}}
\setlength{\itemindent}{0cm}
\setlength{\listparindent}{0cm}
\setlength{\leftmargin}{\evensidemargin}
\addtolength{\leftmargin}{\tmplength}
\settowidth{\labelsep}{X}
\addtolength{\leftmargin}{\labelsep}
\setlength{\labelwidth}{\tmplength}
}
\begin{flushleft}
\item[\textbf{Declaração}\hfill]
\begin{ttfamily}
public class function Create{\_}ArqAluno:Boolean;\end{ttfamily}


\end{flushleft}
\end{list}
\paragraph*{DoOnNewRecord}\hspace*{\fill}

\begin{list}{}{
\settowidth{\tmplength}{\textbf{Declaração}}
\setlength{\itemindent}{0cm}
\setlength{\listparindent}{0cm}
\setlength{\leftmargin}{\evensidemargin}
\addtolength{\leftmargin}{\tmplength}
\settowidth{\labelsep}{X}
\addtolength{\leftmargin}{\labelsep}
\setlength{\labelwidth}{\tmplength}
}
\begin{flushleft}
\item[\textbf{Declaração}\hfill]
\begin{ttfamily}
public class procedure DoOnNewRecord(aNome:AnsiString);\end{ttfamily}


\end{flushleft}
\end{list}
\paragraph*{AddRec}\hspace*{\fill}

\begin{list}{}{
\settowidth{\tmplength}{\textbf{Declaração}}
\setlength{\itemindent}{0cm}
\setlength{\listparindent}{0cm}
\setlength{\leftmargin}{\evensidemargin}
\addtolength{\leftmargin}{\tmplength}
\settowidth{\labelsep}{X}
\addtolength{\leftmargin}{\labelsep}
\setlength{\labelwidth}{\tmplength}
}
\begin{flushleft}
\item[\textbf{Declaração}\hfill]
\begin{ttfamily}
public class function AddRec(aNome:AnsiString):Boolean;\end{ttfamily}


\end{flushleft}
\end{list}
\paragraph*{NextRec}\hspace*{\fill}

\begin{list}{}{
\settowidth{\tmplength}{\textbf{Declaração}}
\setlength{\itemindent}{0cm}
\setlength{\listparindent}{0cm}
\setlength{\leftmargin}{\evensidemargin}
\addtolength{\leftmargin}{\tmplength}
\settowidth{\labelsep}{X}
\addtolength{\leftmargin}{\labelsep}
\setlength{\labelwidth}{\tmplength}
}
\begin{flushleft}
\item[\textbf{Declaração}\hfill]
\begin{ttfamily}
public class function NextRec:Boolean;\end{ttfamily}


\end{flushleft}
\end{list}
\paragraph*{GoBof}\hspace*{\fill}

\begin{list}{}{
\settowidth{\tmplength}{\textbf{Declaração}}
\setlength{\itemindent}{0cm}
\setlength{\listparindent}{0cm}
\setlength{\leftmargin}{\evensidemargin}
\addtolength{\leftmargin}{\tmplength}
\settowidth{\labelsep}{X}
\addtolength{\leftmargin}{\labelsep}
\setlength{\labelwidth}{\tmplength}
}
\begin{flushleft}
\item[\textbf{Declaração}\hfill]
\begin{ttfamily}
public class Function GoBof: Boolean ;\end{ttfamily}


\end{flushleft}
\end{list}
\paragraph*{Cadastra}\hspace*{\fill}

\begin{list}{}{
\settowidth{\tmplength}{\textbf{Declaração}}
\setlength{\itemindent}{0cm}
\setlength{\listparindent}{0cm}
\setlength{\leftmargin}{\evensidemargin}
\addtolength{\leftmargin}{\tmplength}
\settowidth{\labelsep}{X}
\addtolength{\leftmargin}{\labelsep}
\setlength{\labelwidth}{\tmplength}
}
\begin{flushleft}
\item[\textbf{Declaração}\hfill]
\begin{ttfamily}
public class function Cadastra:Boolean;\end{ttfamily}


\end{flushleft}
\end{list}
\paragraph*{Lista{\_}ordem{\_}crescente}\hspace*{\fill}

\begin{list}{}{
\settowidth{\tmplength}{\textbf{Declaração}}
\setlength{\itemindent}{0cm}
\setlength{\listparindent}{0cm}
\setlength{\leftmargin}{\evensidemargin}
\addtolength{\leftmargin}{\tmplength}
\settowidth{\labelsep}{X}
\addtolength{\leftmargin}{\labelsep}
\setlength{\labelwidth}{\tmplength}
}
\begin{flushleft}
\item[\textbf{Declaração}\hfill]
\begin{ttfamily}
public class Procedure Lista{\_}ordem{\_}crescente;\end{ttfamily}


\end{flushleft}
\end{list}
\paragraph*{GoEof}\hspace*{\fill}

\begin{list}{}{
\settowidth{\tmplength}{\textbf{Declaração}}
\setlength{\itemindent}{0cm}
\setlength{\listparindent}{0cm}
\setlength{\leftmargin}{\evensidemargin}
\addtolength{\leftmargin}{\tmplength}
\settowidth{\labelsep}{X}
\addtolength{\leftmargin}{\labelsep}
\setlength{\labelwidth}{\tmplength}
}
\begin{flushleft}
\item[\textbf{Declaração}\hfill]
\begin{ttfamily}
public class Function GoEof: Boolean ;\end{ttfamily}


\end{flushleft}
\end{list}
\paragraph*{PrevRec}\hspace*{\fill}

\begin{list}{}{
\settowidth{\tmplength}{\textbf{Declaração}}
\setlength{\itemindent}{0cm}
\setlength{\listparindent}{0cm}
\setlength{\leftmargin}{\evensidemargin}
\addtolength{\leftmargin}{\tmplength}
\settowidth{\labelsep}{X}
\addtolength{\leftmargin}{\labelsep}
\setlength{\labelwidth}{\tmplength}
}
\begin{flushleft}
\item[\textbf{Declaração}\hfill]
\begin{ttfamily}
public class function PrevRec:Boolean;\end{ttfamily}


\end{flushleft}
\end{list}
\paragraph*{Lista{\_}ordem{\_}Decrescente}\hspace*{\fill}

\begin{list}{}{
\settowidth{\tmplength}{\textbf{Declaração}}
\setlength{\itemindent}{0cm}
\setlength{\listparindent}{0cm}
\setlength{\leftmargin}{\evensidemargin}
\addtolength{\leftmargin}{\tmplength}
\settowidth{\labelsep}{X}
\addtolength{\leftmargin}{\labelsep}
\setlength{\labelwidth}{\tmplength}
}
\begin{flushleft}
\item[\textbf{Declaração}\hfill]
\begin{ttfamily}
public class Procedure Lista{\_}ordem{\_}Decrescente;\end{ttfamily}


\end{flushleft}
\end{list}
\paragraph*{DeleteRec}\hspace*{\fill}

\begin{list}{}{
\settowidth{\tmplength}{\textbf{Declaração}}
\setlength{\itemindent}{0cm}
\setlength{\listparindent}{0cm}
\setlength{\leftmargin}{\evensidemargin}
\addtolength{\leftmargin}{\tmplength}
\settowidth{\labelsep}{X}
\addtolength{\leftmargin}{\labelsep}
\setlength{\labelwidth}{\tmplength}
}
\begin{flushleft}
\item[\textbf{Declaração}\hfill]
\begin{ttfamily}
public class function DeleteRec:Boolean;\end{ttfamily}


\end{flushleft}
\end{list}
\paragraph*{Test{\_}SetTransaction}\hspace*{\fill}

\begin{list}{}{
\settowidth{\tmplength}{\textbf{Declaração}}
\setlength{\itemindent}{0cm}
\setlength{\listparindent}{0cm}
\setlength{\leftmargin}{\evensidemargin}
\addtolength{\leftmargin}{\tmplength}
\settowidth{\labelsep}{X}
\addtolength{\leftmargin}{\labelsep}
\setlength{\labelwidth}{\tmplength}
}
\begin{flushleft}
\item[\textbf{Declaração}\hfill]
\begin{ttfamily}
public class Function Test{\_}SetTransaction:Boolean;\end{ttfamily}


\end{flushleft}
\end{list}
\chapter{Unit mi.rtl.Objects.Methods.Db.Tb{\_}{\_}{\_}Access}
\section{Descrição}
Esta unit \textbf{\begin{ttfamily}mi.rtl.Objects.Methods.Db.Tb{\_}{\_}{\_}Access\end{ttfamily}} é usada para criar banco de dados local usando estrutura \textbf{Type Record End};

\begin{itemize}
\item \textbf{NOTA} \begin{itemize}
\item Comandos parecidos com clipper
\item Está obsoleto não recomendo seu uso.
\end{itemize}
\end{itemize}
\section{Uses}
\begin{itemize}
\item \begin{ttfamily}Classes\end{ttfamily}\item \begin{ttfamily}SysUtils\end{ttfamily}\item \begin{ttfamily}dos\end{ttfamily}\item \begin{ttfamily}mi.rtl.Objects.Methods.Paramexecucao.Application\end{ttfamily}(\ref{mi.rtl.Objects.Methods.Paramexecucao.Application})\item \begin{ttfamily}mi.rtl.objects.Methods.Exception\end{ttfamily}(\ref{mi.rtl.Objects.Methods.Exception})\item \begin{ttfamily}mi.rtl.objects.Methods.dates\end{ttfamily}(\ref{mi.rtl.objects.Methods.dates})\item \begin{ttfamily}mi.rtl.objects.methods.ParamExecucao\end{ttfamily}(\ref{mi.rtl.Objects.Methods.Paramexecucao})\item \begin{ttfamily}mi.rtl.objects.methods.db.tb{\_}access\end{ttfamily}(\ref{mi.rtl.Objects.Methods.Db.Tb_Access})\item \begin{ttfamily}mi.rtl.objects.methods.db.tb{\_}{\_}access\end{ttfamily}(\ref{mi.rtl.Objects.Methods.Db.Tb__Access})\end{itemize}
\section{Visão Geral}
\begin{description}
\item[\texttt{\begin{ttfamily}TTb{\_}{\_}{\_}Access{\_}types\end{ttfamily} Classe}]
\item[\texttt{\begin{ttfamily}TTb{\_}{\_}{\_}Access{\_}consts\end{ttfamily} Classe}]
\item[\texttt{\begin{ttfamily}TTb{\_}{\_}{\_}Access\end{ttfamily} Classe}]
\end{description}
\section{Classes, Interfaces, Objetos e Registros}
\subsection*{TTb{\_}{\_}{\_}Access{\_}types Classe}
\subsubsection*{\large{\textbf{Hierarquia}}\normalsize\hspace{1ex}\hfill}
TTb{\_}{\_}{\_}Access{\_}types {$>$} \begin{ttfamily}TTb{\_}{\_}Access\end{ttfamily}(\ref{mi.rtl.Objects.Methods.Db.Tb__Access.TTb__Access}) {$>$} \begin{ttfamily}TTb{\_}{\_}Access{\_}consts\end{ttfamily}(\ref{mi.rtl.Objects.Methods.Db.Tb__Access.TTb__Access_consts}) {$>$} \begin{ttfamily}TTb{\_}{\_}Access{\_}types\end{ttfamily}(\ref{mi.rtl.Objects.Methods.Db.Tb__Access.TTb__Access_types}) {$>$} \begin{ttfamily}TTb{\_}Access\end{ttfamily}(\ref{mi.rtl.Objects.Methods.Db.Tb_Access.TTb_Access}) {$>$} \begin{ttfamily}TTb{\_}Access{\_}consts\end{ttfamily}(\ref{mi.rtl.Objects.Methods.Db.Tb_Access.TTb_Access_consts}) {$>$} \begin{ttfamily}TTb{\_}Access{\_}types\end{ttfamily}(\ref{mi.rtl.Objects.Methods.Db.Tb_Access.TTb_Access_types}) {$>$} \begin{ttfamily}TObjectsSystem\end{ttfamily}(\ref{mi.rtl.Objects.Methods.System.TObjectsSystem}) {$>$} \begin{ttfamily}TObjectsMethods\end{ttfamily}(\ref{mi.rtl.Objects.Methods.TObjectsMethods}) {$>$} \begin{ttfamily}TObjectsConsts\end{ttfamily}(\ref{mi.rtl.Objects.Consts.TObjectsConsts}) {$>$} 
TObjectsTypes
\subsubsection*{\large{\textbf{Descrição}}\normalsize\hspace{1ex}\hfill}
no description available, TTb{\_}{\_}Access description followsno description available, TTb{\_}{\_}Access{\_}consts description followsno description available, TTb{\_}{\_}Access{\_}types description followsno description available, TTb{\_}Access description followsno description available, TTb{\_}Access{\_}consts description followsA classe \textbf{\begin{ttfamily}TTb{\_}Access{\_}consts\end{ttfamily}} é usada para declarar todas as constantes da classe \textbf{\begin{ttfamily}TTb{\_}Access\end{ttfamily}(\ref{mi.rtl.Objects.Methods.Db.Tb_Access.TTb_Access})}\subsubsection*{\large{\textbf{Campos}}\normalsize\hspace{1ex}\hfill}
\paragraph*{MaxBase}\hspace*{\fill}

\begin{list}{}{
\settowidth{\tmplength}{\textbf{Declaração}}
\setlength{\itemindent}{0cm}
\setlength{\listparindent}{0cm}
\setlength{\leftmargin}{\evensidemargin}
\addtolength{\leftmargin}{\tmplength}
\settowidth{\labelsep}{X}
\addtolength{\leftmargin}{\labelsep}
\setlength{\labelwidth}{\tmplength}
}
\begin{flushleft}
\item[\textbf{Declaração}\hfill]
\begin{ttfamily}
public const MaxBase = 135;\end{ttfamily}


\end{flushleft}
\par
\item[\textbf{Descrição}]
Uma Base permite \begin{ttfamily}um\end{ttfamily}(\ref{}) arquivo de dados + \begin{ttfamily}Um\end{ttfamily}(\ref{}) arquivo de indice

\end{list}
\subsection*{TTb{\_}{\_}{\_}Access{\_}consts Classe}
\subsubsection*{\large{\textbf{Hierarquia}}\normalsize\hspace{1ex}\hfill}
TTb{\_}{\_}{\_}Access{\_}consts {$>$} \begin{ttfamily}TTb{\_}{\_}{\_}Access{\_}types\end{ttfamily}(\ref{mi.rtl.Objects.Methods.Db.Tb___Access.TTb___Access_types}) {$>$} \begin{ttfamily}TTb{\_}{\_}Access\end{ttfamily}(\ref{mi.rtl.Objects.Methods.Db.Tb__Access.TTb__Access}) {$>$} \begin{ttfamily}TTb{\_}{\_}Access{\_}consts\end{ttfamily}(\ref{mi.rtl.Objects.Methods.Db.Tb__Access.TTb__Access_consts}) {$>$} \begin{ttfamily}TTb{\_}{\_}Access{\_}types\end{ttfamily}(\ref{mi.rtl.Objects.Methods.Db.Tb__Access.TTb__Access_types}) {$>$} \begin{ttfamily}TTb{\_}Access\end{ttfamily}(\ref{mi.rtl.Objects.Methods.Db.Tb_Access.TTb_Access}) {$>$} \begin{ttfamily}TTb{\_}Access{\_}consts\end{ttfamily}(\ref{mi.rtl.Objects.Methods.Db.Tb_Access.TTb_Access_consts}) {$>$} \begin{ttfamily}TTb{\_}Access{\_}types\end{ttfamily}(\ref{mi.rtl.Objects.Methods.Db.Tb_Access.TTb_Access_types}) {$>$} \begin{ttfamily}TObjectsSystem\end{ttfamily}(\ref{mi.rtl.Objects.Methods.System.TObjectsSystem}) {$>$} \begin{ttfamily}TObjectsMethods\end{ttfamily}(\ref{mi.rtl.Objects.Methods.TObjectsMethods}) {$>$} \begin{ttfamily}TObjectsConsts\end{ttfamily}(\ref{mi.rtl.Objects.Consts.TObjectsConsts}) {$>$} 
TObjectsTypes
\subsubsection*{\large{\textbf{Descrição}}\normalsize\hspace{1ex}\hfill}
no description available, TTb{\_}{\_}{\_}Access{\_}types description followsno description available, TTb{\_}{\_}Access description followsno description available, TTb{\_}{\_}Access{\_}consts description followsno description available, TTb{\_}{\_}Access{\_}types description followsno description available, TTb{\_}Access description followsno description available, TTb{\_}Access{\_}consts description followsA classe \textbf{\begin{ttfamily}TTb{\_}Access{\_}consts\end{ttfamily}} é usada para declarar todas as constantes da classe \textbf{\begin{ttfamily}TTb{\_}Access\end{ttfamily}(\ref{mi.rtl.Objects.Methods.Db.Tb_Access.TTb_Access})}\subsubsection*{\large{\textbf{Campos}}\normalsize\hspace{1ex}\hfill}
\paragraph*{NBase}\hspace*{\fill}

\begin{list}{}{
\settowidth{\tmplength}{\textbf{Declaração}}
\setlength{\itemindent}{0cm}
\setlength{\listparindent}{0cm}
\setlength{\leftmargin}{\evensidemargin}
\addtolength{\leftmargin}{\tmplength}
\settowidth{\labelsep}{X}
\addtolength{\leftmargin}{\labelsep}
\setlength{\labelwidth}{\tmplength}
}
\begin{flushleft}
\item[\textbf{Declaração}\hfill]
\begin{ttfamily}
public const NBase : Byte = 0;\end{ttfamily}


\end{flushleft}
\end{list}
\paragraph*{MatPFile}\hspace*{\fill}

\begin{list}{}{
\settowidth{\tmplength}{\textbf{Declaração}}
\setlength{\itemindent}{0cm}
\setlength{\listparindent}{0cm}
\setlength{\leftmargin}{\evensidemargin}
\addtolength{\leftmargin}{\tmplength}
\settowidth{\labelsep}{X}
\addtolength{\leftmargin}{\labelsep}
\setlength{\labelwidth}{\tmplength}
}
\begin{flushleft}
\item[\textbf{Declaração}\hfill]
\begin{ttfamily}
public const MatPFile : TipoBaseDeDados = (Nil,Nil,Nil,Nil,Nil,Nil,Nil,Nil,Nil,Nil,Nil,Nil,Nil,Nil,Nil,Nil,Nil,Nil,Nil,Nil,
                                    Nil,Nil,Nil,Nil,Nil,Nil,Nil,Nil,Nil,Nil,Nil,Nil,Nil,Nil,Nil,Nil,Nil,Nil,Nil,Nil,
                                    Nil,Nil,Nil,Nil,Nil,Nil,Nil,Nil,Nil,Nil,Nil,Nil,Nil,Nil,Nil,Nil,Nil,Nil,Nil,Nil,
                                    Nil,Nil,Nil,Nil,Nil,Nil,Nil,Nil,Nil,Nil,Nil,Nil,Nil,Nil,Nil,Nil,Nil,Nil,Nil,Nil,
                                    Nil,Nil,Nil,Nil,Nil,Nil,Nil,Nil,Nil,Nil,Nil,Nil,Nil,Nil,Nil,Nil,Nil,Nil,Nil,Nil,
                                    Nil,Nil,Nil,Nil,Nil,Nil,Nil,Nil,Nil,Nil,Nil,Nil,Nil,Nil,Nil,Nil,Nil,Nil,Nil,Nil,
                                    Nil,Nil,Nil,Nil,Nil,Nil,Nil,Nil,Nil,Nil,Nil,Nil,Nil,Nil,Nil );\end{ttfamily}


\end{flushleft}
\end{list}
\subsection*{TTb{\_}{\_}{\_}Access Classe}
\subsubsection*{\large{\textbf{Hierarquia}}\normalsize\hspace{1ex}\hfill}
TTb{\_}{\_}{\_}Access {$>$} \begin{ttfamily}TTb{\_}{\_}{\_}Access{\_}consts\end{ttfamily}(\ref{mi.rtl.Objects.Methods.Db.Tb___Access.TTb___Access_consts}) {$>$} \begin{ttfamily}TTb{\_}{\_}{\_}Access{\_}types\end{ttfamily}(\ref{mi.rtl.Objects.Methods.Db.Tb___Access.TTb___Access_types}) {$>$} \begin{ttfamily}TTb{\_}{\_}Access\end{ttfamily}(\ref{mi.rtl.Objects.Methods.Db.Tb__Access.TTb__Access}) {$>$} \begin{ttfamily}TTb{\_}{\_}Access{\_}consts\end{ttfamily}(\ref{mi.rtl.Objects.Methods.Db.Tb__Access.TTb__Access_consts}) {$>$} \begin{ttfamily}TTb{\_}{\_}Access{\_}types\end{ttfamily}(\ref{mi.rtl.Objects.Methods.Db.Tb__Access.TTb__Access_types}) {$>$} \begin{ttfamily}TTb{\_}Access\end{ttfamily}(\ref{mi.rtl.Objects.Methods.Db.Tb_Access.TTb_Access}) {$>$} \begin{ttfamily}TTb{\_}Access{\_}consts\end{ttfamily}(\ref{mi.rtl.Objects.Methods.Db.Tb_Access.TTb_Access_consts}) {$>$} \begin{ttfamily}TTb{\_}Access{\_}types\end{ttfamily}(\ref{mi.rtl.Objects.Methods.Db.Tb_Access.TTb_Access_types}) {$>$} \begin{ttfamily}TObjectsSystem\end{ttfamily}(\ref{mi.rtl.Objects.Methods.System.TObjectsSystem}) {$>$} \begin{ttfamily}TObjectsMethods\end{ttfamily}(\ref{mi.rtl.Objects.Methods.TObjectsMethods}) {$>$} \begin{ttfamily}TObjectsConsts\end{ttfamily}(\ref{mi.rtl.Objects.Consts.TObjectsConsts}) {$>$} 
TObjectsTypes
\subsubsection*{\large{\textbf{Descrição}}\normalsize\hspace{1ex}\hfill}
no description available, TTb{\_}{\_}{\_}Access{\_}consts description followsno description available, TTb{\_}{\_}{\_}Access{\_}types description followsno description available, TTb{\_}{\_}Access description followsno description available, TTb{\_}{\_}Access{\_}consts description followsno description available, TTb{\_}{\_}Access{\_}types description followsno description available, TTb{\_}Access description followsno description available, TTb{\_}Access{\_}consts description followsA classe \textbf{\begin{ttfamily}TTb{\_}Access{\_}consts\end{ttfamily}} é usada para declarar todas as constantes da classe \textbf{\begin{ttfamily}TTb{\_}Access\end{ttfamily}(\ref{mi.rtl.Objects.Methods.Db.Tb_Access.TTb_Access})}\subsubsection*{\large{\textbf{Métodos}}\normalsize\hspace{1ex}\hfill}
\paragraph*{OpenFile}\hspace*{\fill}

\begin{list}{}{
\settowidth{\tmplength}{\textbf{Declaração}}
\setlength{\itemindent}{0cm}
\setlength{\listparindent}{0cm}
\setlength{\leftmargin}{\evensidemargin}
\addtolength{\leftmargin}{\tmplength}
\settowidth{\labelsep}{X}
\addtolength{\leftmargin}{\labelsep}
\setlength{\labelwidth}{\tmplength}
}
\begin{flushleft}
\item[\textbf{Declaração}\hfill]
\begin{ttfamily}
public class function OpenFile(var DatF : TMI{\_}DataFile ; var IxF : TMI{\_}IndexFile; Const condicao : NumeroDeArquivos ) : Integer; overload;\end{ttfamily}


\end{flushleft}
\end{list}
\paragraph*{CloseFile}\hspace*{\fill}

\begin{list}{}{
\settowidth{\tmplength}{\textbf{Declaração}}
\setlength{\itemindent}{0cm}
\setlength{\listparindent}{0cm}
\setlength{\leftmargin}{\evensidemargin}
\addtolength{\leftmargin}{\tmplength}
\settowidth{\labelsep}{X}
\addtolength{\leftmargin}{\labelsep}
\setlength{\labelwidth}{\tmplength}
}
\begin{flushleft}
\item[\textbf{Declaração}\hfill]
\begin{ttfamily}
public class procedure CloseFile(var DatF : TMI{\_}DataFile ; var IxF : TMI{\_}IndexFile ; Const condicao : NumeroDeArquivos ); overload;\end{ttfamily}


\end{flushleft}
\end{list}
\paragraph*{LocRegUse}\hspace*{\fill}

\begin{list}{}{
\settowidth{\tmplength}{\textbf{Declaração}}
\setlength{\itemindent}{0cm}
\setlength{\listparindent}{0cm}
\setlength{\leftmargin}{\evensidemargin}
\addtolength{\leftmargin}{\tmplength}
\settowidth{\labelsep}{X}
\addtolength{\leftmargin}{\labelsep}
\setlength{\labelwidth}{\tmplength}
}
\begin{flushleft}
\item[\textbf{Declaração}\hfill]
\begin{ttfamily}
public class function LocRegUse(Const P : TipoPonteiroBD): Byte;\end{ttfamily}


\end{flushleft}
\end{list}
\paragraph*{NRecSkip}\hspace*{\fill}

\begin{list}{}{
\settowidth{\tmplength}{\textbf{Declaração}}
\setlength{\itemindent}{0cm}
\setlength{\listparindent}{0cm}
\setlength{\leftmargin}{\evensidemargin}
\addtolength{\leftmargin}{\tmplength}
\settowidth{\labelsep}{X}
\addtolength{\leftmargin}{\labelsep}
\setlength{\labelwidth}{\tmplength}
}
\begin{flushleft}
\item[\textbf{Declaração}\hfill]
\begin{ttfamily}
public class function NRecSkip(Var Buff) : Longint;\end{ttfamily}


\end{flushleft}
\end{list}
\paragraph*{Replace}\hspace*{\fill}

\begin{list}{}{
\settowidth{\tmplength}{\textbf{Declaração}}
\setlength{\itemindent}{0cm}
\setlength{\listparindent}{0cm}
\setlength{\leftmargin}{\evensidemargin}
\addtolength{\leftmargin}{\tmplength}
\settowidth{\labelsep}{X}
\addtolength{\leftmargin}{\labelsep}
\setlength{\labelwidth}{\tmplength}
}
\begin{flushleft}
\item[\textbf{Declaração}\hfill]
\begin{ttfamily}
public class Procedure Replace(Var Buff);\end{ttfamily}


\end{flushleft}
\end{list}
\paragraph*{Locate}\hspace*{\fill}

\begin{list}{}{
\settowidth{\tmplength}{\textbf{Declaração}}
\setlength{\itemindent}{0cm}
\setlength{\listparindent}{0cm}
\setlength{\leftmargin}{\evensidemargin}
\addtolength{\leftmargin}{\tmplength}
\settowidth{\labelsep}{X}
\addtolength{\leftmargin}{\labelsep}
\setlength{\labelwidth}{\tmplength}
}
\begin{flushleft}
\item[\textbf{Declaração}\hfill]
\begin{ttfamily}
public class Procedure Locate(Var Buff;NRec:Longint);\end{ttfamily}


\end{flushleft}
\end{list}
\paragraph*{ReplaceUnLock}\hspace*{\fill}

\begin{list}{}{
\settowidth{\tmplength}{\textbf{Declaração}}
\setlength{\itemindent}{0cm}
\setlength{\listparindent}{0cm}
\setlength{\leftmargin}{\evensidemargin}
\addtolength{\leftmargin}{\tmplength}
\settowidth{\labelsep}{X}
\addtolength{\leftmargin}{\labelsep}
\setlength{\labelwidth}{\tmplength}
}
\begin{flushleft}
\item[\textbf{Declaração}\hfill]
\begin{ttfamily}
public class Procedure ReplaceUnLock(Var Buff);\end{ttfamily}


\end{flushleft}
\end{list}
\paragraph*{TrocaChave}\hspace*{\fill}

\begin{list}{}{
\settowidth{\tmplength}{\textbf{Declaração}}
\setlength{\itemindent}{0cm}
\setlength{\listparindent}{0cm}
\setlength{\leftmargin}{\evensidemargin}
\addtolength{\leftmargin}{\tmplength}
\settowidth{\labelsep}{X}
\addtolength{\leftmargin}{\labelsep}
\setlength{\labelwidth}{\tmplength}
}
\begin{flushleft}
\item[\textbf{Declaração}\hfill]
\begin{ttfamily}
public class Procedure TrocaChave(Var Buff; Var IxF : TMI{\_}IndexFile; ChaveAnterior, ChaveAtual : TaKeyStr);\end{ttfamily}


\end{flushleft}
\end{list}
\paragraph*{FPosicao}\hspace*{\fill}

\begin{list}{}{
\settowidth{\tmplength}{\textbf{Declaração}}
\setlength{\itemindent}{0cm}
\setlength{\listparindent}{0cm}
\setlength{\leftmargin}{\evensidemargin}
\addtolength{\leftmargin}{\tmplength}
\settowidth{\labelsep}{X}
\addtolength{\leftmargin}{\labelsep}
\setlength{\labelwidth}{\tmplength}
}
\begin{flushleft}
\item[\textbf{Declaração}\hfill]
\begin{ttfamily}
public class function FPosicao(Posicao : Byte ) : Byte;\end{ttfamily}


\end{flushleft}
\end{list}
\paragraph*{IndiceSele}\hspace*{\fill}

\begin{list}{}{
\settowidth{\tmplength}{\textbf{Declaração}}
\setlength{\itemindent}{0cm}
\setlength{\listparindent}{0cm}
\setlength{\leftmargin}{\evensidemargin}
\addtolength{\leftmargin}{\tmplength}
\settowidth{\labelsep}{X}
\addtolength{\leftmargin}{\labelsep}
\setlength{\labelwidth}{\tmplength}
}
\begin{flushleft}
\item[\textbf{Declaração}\hfill]
\begin{ttfamily}
public class function IndiceSele(Var Buff) : PathStr;\end{ttfamily}


\end{flushleft}
\end{list}
\paragraph*{ClearSkipAll}\hspace*{\fill}

\begin{list}{}{
\settowidth{\tmplength}{\textbf{Declaração}}
\setlength{\itemindent}{0cm}
\setlength{\listparindent}{0cm}
\setlength{\leftmargin}{\evensidemargin}
\addtolength{\leftmargin}{\tmplength}
\settowidth{\labelsep}{X}
\addtolength{\leftmargin}{\labelsep}
\setlength{\labelwidth}{\tmplength}
}
\begin{flushleft}
\item[\textbf{Declaração}\hfill]
\begin{ttfamily}
public class Procedure ClearSkipAll;\end{ttfamily}


\end{flushleft}
\end{list}
\paragraph*{Use}\hspace*{\fill}

\begin{list}{}{
\settowidth{\tmplength}{\textbf{Declaração}}
\setlength{\itemindent}{0cm}
\setlength{\listparindent}{0cm}
\setlength{\leftmargin}{\evensidemargin}
\addtolength{\leftmargin}{\tmplength}
\settowidth{\labelsep}{X}
\addtolength{\leftmargin}{\labelsep}
\setlength{\labelwidth}{\tmplength}
}
\begin{flushleft}
\item[\textbf{Declaração}\hfill]
\begin{ttfamily}
public class function Use(Var DatF : TMI{\_}DataFile ; Var IxF : TMI{\_}IndexFile ; Var Buff ):Boolean;\end{ttfamily}


\end{flushleft}
\end{list}
\paragraph*{UseC}\hspace*{\fill}

\begin{list}{}{
\settowidth{\tmplength}{\textbf{Declaração}}
\setlength{\itemindent}{0cm}
\setlength{\listparindent}{0cm}
\setlength{\leftmargin}{\evensidemargin}
\addtolength{\leftmargin}{\tmplength}
\settowidth{\labelsep}{X}
\addtolength{\leftmargin}{\labelsep}
\setlength{\labelwidth}{\tmplength}
}
\begin{flushleft}
\item[\textbf{Declaração}\hfill]
\begin{ttfamily}
public class Procedure UseC(Var Buff);\end{ttfamily}


\end{flushleft}
\end{list}
\paragraph*{Skip}\hspace*{\fill}

\begin{list}{}{
\settowidth{\tmplength}{\textbf{Declaração}}
\setlength{\itemindent}{0cm}
\setlength{\listparindent}{0cm}
\setlength{\leftmargin}{\evensidemargin}
\addtolength{\leftmargin}{\tmplength}
\settowidth{\labelsep}{X}
\addtolength{\leftmargin}{\labelsep}
\setlength{\labelwidth}{\tmplength}
}
\begin{flushleft}
\item[\textbf{Declaração}\hfill]
\begin{ttfamily}
public class Procedure Skip(Var Buff; Var Chave : TaKeyStr; Const ModoDePesquisa : TipoSkip );\end{ttfamily}


\end{flushleft}
\end{list}
\paragraph*{FSkip}\hspace*{\fill}

\begin{list}{}{
\settowidth{\tmplength}{\textbf{Declaração}}
\setlength{\itemindent}{0cm}
\setlength{\listparindent}{0cm}
\setlength{\leftmargin}{\evensidemargin}
\addtolength{\leftmargin}{\tmplength}
\settowidth{\labelsep}{X}
\addtolength{\leftmargin}{\labelsep}
\setlength{\labelwidth}{\tmplength}
}
\begin{flushleft}
\item[\textbf{Declaração}\hfill]
\begin{ttfamily}
public class function FSkip(Var Buff; Var Chave : tString; Const ModoDePesquisa : TipoSkip ):Boolean;\end{ttfamily}


\end{flushleft}
\end{list}
\paragraph*{FSkipSearch}\hspace*{\fill}

\begin{list}{}{
\settowidth{\tmplength}{\textbf{Declaração}}
\setlength{\itemindent}{0cm}
\setlength{\listparindent}{0cm}
\setlength{\leftmargin}{\evensidemargin}
\addtolength{\leftmargin}{\tmplength}
\settowidth{\labelsep}{X}
\addtolength{\leftmargin}{\labelsep}
\setlength{\labelwidth}{\tmplength}
}
\begin{flushleft}
\item[\textbf{Declaração}\hfill]
\begin{ttfamily}
public class function FSkipSearch(Var Buff; Chave : tString):Boolean;\end{ttfamily}


\end{flushleft}
\end{list}
\paragraph*{SeekRec}\hspace*{\fill}

\begin{list}{}{
\settowidth{\tmplength}{\textbf{Declaração}}
\setlength{\itemindent}{0cm}
\setlength{\listparindent}{0cm}
\setlength{\leftmargin}{\evensidemargin}
\addtolength{\leftmargin}{\tmplength}
\settowidth{\labelsep}{X}
\addtolength{\leftmargin}{\labelsep}
\setlength{\labelwidth}{\tmplength}
}
\begin{flushleft}
\item[\textbf{Declaração}\hfill]
\begin{ttfamily}
public class Procedure SeekRec(Var Buff; Var IxF : TMI{\_}IndexFile ; Var Chave : TaKeyStr; Const ModoDePesquisa : TipoSkip );\end{ttfamily}


\end{flushleft}
\end{list}
\paragraph*{SkipLock}\hspace*{\fill}

\begin{list}{}{
\settowidth{\tmplength}{\textbf{Declaração}}
\setlength{\itemindent}{0cm}
\setlength{\listparindent}{0cm}
\setlength{\leftmargin}{\evensidemargin}
\addtolength{\leftmargin}{\tmplength}
\settowidth{\labelsep}{X}
\addtolength{\leftmargin}{\labelsep}
\setlength{\labelwidth}{\tmplength}
}
\begin{flushleft}
\item[\textbf{Declaração}\hfill]
\begin{ttfamily}
public class Procedure SkipLock(Var Buff; Var Chave : tString; Const ModoDePesquisa : TipoSkip );\end{ttfamily}


\end{flushleft}
\end{list}
\paragraph*{LocRegSkip}\hspace*{\fill}

\begin{list}{}{
\settowidth{\tmplength}{\textbf{Declaração}}
\setlength{\itemindent}{0cm}
\setlength{\listparindent}{0cm}
\setlength{\leftmargin}{\evensidemargin}
\addtolength{\leftmargin}{\tmplength}
\settowidth{\labelsep}{X}
\addtolength{\leftmargin}{\labelsep}
\setlength{\labelwidth}{\tmplength}
}
\begin{flushleft}
\item[\textbf{Declaração}\hfill]
\begin{ttfamily}
public class function LocRegSkip(P : TipoPonteiroBD): Byte;\end{ttfamily}


\end{flushleft}
\end{list}
\paragraph*{Sele}\hspace*{\fill}

\begin{list}{}{
\settowidth{\tmplength}{\textbf{Declaração}}
\setlength{\itemindent}{0cm}
\setlength{\listparindent}{0cm}
\setlength{\leftmargin}{\evensidemargin}
\addtolength{\leftmargin}{\tmplength}
\settowidth{\labelsep}{X}
\addtolength{\leftmargin}{\labelsep}
\setlength{\labelwidth}{\tmplength}
}
\begin{flushleft}
\item[\textbf{Declaração}\hfill]
\begin{ttfamily}
public class Procedure Sele(Var Buff; Var IxF : TMI{\_}IndexFile );\end{ttfamily}


\end{flushleft}
\end{list}
\paragraph*{AAddRec}\hspace*{\fill}

\begin{list}{}{
\settowidth{\tmplength}{\textbf{Declaração}}
\setlength{\itemindent}{0cm}
\setlength{\listparindent}{0cm}
\setlength{\leftmargin}{\evensidemargin}
\addtolength{\leftmargin}{\tmplength}
\settowidth{\labelsep}{X}
\addtolength{\leftmargin}{\labelsep}
\setlength{\labelwidth}{\tmplength}
}
\begin{flushleft}
\item[\textbf{Declaração}\hfill]
\begin{ttfamily}
public class Procedure AAddRec(Var Buff;Var IxF : TMI{\_}IndexFile ; Var Chave : tString); overload;\end{ttfamily}


\end{flushleft}
\end{list}
\paragraph*{ADeleteRec}\hspace*{\fill}

\begin{list}{}{
\settowidth{\tmplength}{\textbf{Declaração}}
\setlength{\itemindent}{0cm}
\setlength{\listparindent}{0cm}
\setlength{\leftmargin}{\evensidemargin}
\addtolength{\leftmargin}{\tmplength}
\settowidth{\labelsep}{X}
\addtolength{\leftmargin}{\labelsep}
\setlength{\labelwidth}{\tmplength}
}
\begin{flushleft}
\item[\textbf{Declaração}\hfill]
\begin{ttfamily}
public class Procedure ADeleteRec(Var Buff;Var IxF : TMI{\_}IndexFile ;Var Chave : tString); overload;\end{ttfamily}


\end{flushleft}
\end{list}
\paragraph*{CloseFilesOpens}\hspace*{\fill}

\begin{list}{}{
\settowidth{\tmplength}{\textbf{Declaração}}
\setlength{\itemindent}{0cm}
\setlength{\listparindent}{0cm}
\setlength{\leftmargin}{\evensidemargin}
\addtolength{\leftmargin}{\tmplength}
\settowidth{\labelsep}{X}
\addtolength{\leftmargin}{\labelsep}
\setlength{\labelwidth}{\tmplength}
}
\begin{flushleft}
\item[\textbf{Declaração}\hfill]
\begin{ttfamily}
public class Procedure CloseFilesOpens; override;\end{ttfamily}


\end{flushleft}
\end{list}
\paragraph*{EscrevaDataFile}\hspace*{\fill}

\begin{list}{}{
\settowidth{\tmplength}{\textbf{Declaração}}
\setlength{\itemindent}{0cm}
\setlength{\listparindent}{0cm}
\setlength{\leftmargin}{\evensidemargin}
\addtolength{\leftmargin}{\tmplength}
\settowidth{\labelsep}{X}
\addtolength{\leftmargin}{\labelsep}
\setlength{\labelwidth}{\tmplength}
}
\begin{flushleft}
\item[\textbf{Declaração}\hfill]
\begin{ttfamily}
public class procedure EscrevaDataFile(Var DatF : TMI{\_}DataFile; Var IxF : TMI{\_}IndexFile);\end{ttfamily}


\end{flushleft}
\end{list}
\chapter{Unit mi.rtl.objects.Methods.db.types.consts.Methods}
\section{Descrição}
\begin{itemize}
\item A unit \textbf{\begin{ttfamily}mi.rtl.objects.Methods.db.types.consts.Methods\end{ttfamily}} implementa a classe \textbf{\begin{ttfamily}TDb{\_}Methods\end{ttfamily}(\ref{mi.rtl.objects.Methods.db.types.consts.Methods.TDb_Methods})} do pacote \textbf{mi.rtl.db}.

\begin{itemize}
\item \textbf{NOTAS} {-}
\item \textbf{VERSÃO} \begin{itemize}
\item Alpha {-} 0.5.0.687
\end{itemize}
\item \textbf{HISTÓRICO} \begin{itemize}
\item Criado por: Paulo Sérgio da Silva Pacheco e{-}mail: paulosspacheco@yahoo.com.br \begin{itemize}
\item \textbf{01/12/2021} \begin{itemize}
\item 10:15 a ?? : Criar a unit mi.rtl.db{\_}Methods.pas
\end{itemize}
\end{itemize}
\end{itemize}
\item \textbf{CÓDIGO FONTE}: \begin{itemize}
\item 
\end{itemize}
\end{itemize}
\end{itemize}
\section{Uses}
\begin{itemize}
\item \begin{ttfamily}Classes\end{ttfamily}\item \begin{ttfamily}SysUtils\end{ttfamily}\item \begin{ttfamily}mi.rtl.objects.Methods.db.types.consts\end{ttfamily}\item \begin{ttfamily}mi.rtl.objects.methods.StreamBase.Stream\end{ttfamily}(\ref{mi.rtl.Objects.Methods.StreamBase.Stream})\item \begin{ttfamily}mi.rtl.objects.TException\end{ttfamily}\end{itemize}
\section{Visão Geral}
\begin{description}
\item[\texttt{\begin{ttfamily}TDb{\_}Methods\end{ttfamily} Classe}]
\end{description}
\section{Classes, Interfaces, Objetos e Registros}
\subsection*{TDb{\_}Methods Classe}
\subsubsection*{\large{\textbf{Hierarquia}}\normalsize\hspace{1ex}\hfill}
TDb{\_}Methods {$>$} TDbConsts
\subsubsection*{\large{\textbf{Descrição}}\normalsize\hspace{1ex}\hfill}
\begin{itemize}
\item A classe \textbf{\begin{ttfamily}TDb{\_}Methods\end{ttfamily}} implementa os método de classe comum a todas as classes de TDB do pacote \textbf{mi.rtl.db}.
\end{itemize}\subsubsection*{\large{\textbf{Métodos}}\normalsize\hspace{1ex}\hfill}
\paragraph*{Create}\hspace*{\fill}

\begin{list}{}{
\settowidth{\tmplength}{\textbf{Declaração}}
\setlength{\itemindent}{0cm}
\setlength{\listparindent}{0cm}
\setlength{\leftmargin}{\evensidemargin}
\addtolength{\leftmargin}{\tmplength}
\settowidth{\labelsep}{X}
\addtolength{\leftmargin}{\labelsep}
\setlength{\labelwidth}{\tmplength}
}
\begin{flushleft}
\item[\textbf{Declaração}\hfill]
\begin{ttfamily}
public constructor Create(aowner:TComponent); Overload; Override;\end{ttfamily}


\end{flushleft}
\end{list}
\paragraph*{Destroy}\hspace*{\fill}

\begin{list}{}{
\settowidth{\tmplength}{\textbf{Declaração}}
\setlength{\itemindent}{0cm}
\setlength{\listparindent}{0cm}
\setlength{\leftmargin}{\evensidemargin}
\addtolength{\leftmargin}{\tmplength}
\settowidth{\labelsep}{X}
\addtolength{\leftmargin}{\labelsep}
\setlength{\labelwidth}{\tmplength}
}
\begin{flushleft}
\item[\textbf{Declaração}\hfill]
\begin{ttfamily}
public destructor Destroy; override;\end{ttfamily}


\end{flushleft}
\end{list}
\paragraph*{FExisteCodigo}\hspace*{\fill}

\begin{list}{}{
\settowidth{\tmplength}{\textbf{Declaração}}
\setlength{\itemindent}{0cm}
\setlength{\listparindent}{0cm}
\setlength{\leftmargin}{\evensidemargin}
\addtolength{\leftmargin}{\tmplength}
\settowidth{\labelsep}{X}
\addtolength{\leftmargin}{\labelsep}
\setlength{\labelwidth}{\tmplength}
}
\begin{flushleft}
\item[\textbf{Declaração}\hfill]
\begin{ttfamily}
public function FExisteCodigo(Var IxF:IndexFile; Const Codigo:tString):Boolean;\end{ttfamily}


\end{flushleft}
\end{list}
\paragraph*{CreateTAccess}\hspace*{\fill}

\begin{list}{}{
\settowidth{\tmplength}{\textbf{Declaração}}
\setlength{\itemindent}{0cm}
\setlength{\listparindent}{0cm}
\setlength{\leftmargin}{\evensidemargin}
\addtolength{\leftmargin}{\tmplength}
\settowidth{\labelsep}{X}
\addtolength{\leftmargin}{\labelsep}
\setlength{\labelwidth}{\tmplength}
}
\begin{flushleft}
\item[\textbf{Declaração}\hfill]
\begin{ttfamily}
public procedure CreateTAccess;\end{ttfamily}


\end{flushleft}
\end{list}
\paragraph*{DestroyTAccess}\hspace*{\fill}

\begin{list}{}{
\settowidth{\tmplength}{\textbf{Declaração}}
\setlength{\itemindent}{0cm}
\setlength{\listparindent}{0cm}
\setlength{\leftmargin}{\evensidemargin}
\addtolength{\leftmargin}{\tmplength}
\settowidth{\labelsep}{X}
\addtolength{\leftmargin}{\labelsep}
\setlength{\labelwidth}{\tmplength}
}
\begin{flushleft}
\item[\textbf{Declaração}\hfill]
\begin{ttfamily}
public procedure DestroyTAccess;\end{ttfamily}


\end{flushleft}
\end{list}
\paragraph*{EscrevaTurboError}\hspace*{\fill}

\begin{list}{}{
\settowidth{\tmplength}{\textbf{Declaração}}
\setlength{\itemindent}{0cm}
\setlength{\listparindent}{0cm}
\setlength{\leftmargin}{\evensidemargin}
\addtolength{\leftmargin}{\tmplength}
\settowidth{\labelsep}{X}
\addtolength{\leftmargin}{\labelsep}
\setlength{\labelwidth}{\tmplength}
}
\begin{flushleft}
\item[\textbf{Declaração}\hfill]
\begin{ttfamily}
public function EscrevaTurboError(DatF : DataFile;Const NR : Longint;Error:SmallWord):Boolean;\end{ttfamily}


\end{flushleft}
\end{list}
\paragraph*{TAIOCheck}\hspace*{\fill}

\begin{list}{}{
\settowidth{\tmplength}{\textbf{Declaração}}
\setlength{\itemindent}{0cm}
\setlength{\listparindent}{0cm}
\setlength{\leftmargin}{\evensidemargin}
\addtolength{\leftmargin}{\tmplength}
\settowidth{\labelsep}{X}
\addtolength{\leftmargin}{\labelsep}
\setlength{\labelwidth}{\tmplength}
}
\begin{flushleft}
\item[\textbf{Declaração}\hfill]
\begin{ttfamily}
public function TAIOCheck(VAR DatF : DataFile;Const R : LONGINT):Boolean;\end{ttfamily}


\end{flushleft}
\end{list}
\paragraph*{SincronizaPosChave}\hspace*{\fill}

\begin{list}{}{
\settowidth{\tmplength}{\textbf{Declaração}}
\setlength{\itemindent}{0cm}
\setlength{\listparindent}{0cm}
\setlength{\leftmargin}{\evensidemargin}
\addtolength{\leftmargin}{\tmplength}
\settowidth{\labelsep}{X}
\addtolength{\leftmargin}{\labelsep}
\setlength{\labelwidth}{\tmplength}
}
\begin{flushleft}
\item[\textbf{Declaração}\hfill]
\begin{ttfamily}
public function SincronizaPosChave(Var datFIx:IndexFile;Const NrCurrent:Longint; KeyCurrent : TaKeyStr):Boolean;\end{ttfamily}


\end{flushleft}
\end{list}
\paragraph*{GetRec}\hspace*{\fill}

\begin{list}{}{
\settowidth{\tmplength}{\textbf{Declaração}}
\setlength{\itemindent}{0cm}
\setlength{\listparindent}{0cm}
\setlength{\leftmargin}{\evensidemargin}
\addtolength{\leftmargin}{\tmplength}
\settowidth{\labelsep}{X}
\addtolength{\leftmargin}{\labelsep}
\setlength{\labelwidth}{\tmplength}
}
\begin{flushleft}
\item[\textbf{Declaração}\hfill]
\begin{ttfamily}
public function GetRec(var DatF : DataFile;Const R : Longint;var Buffer ):Boolean;\end{ttfamily}


\end{flushleft}
\end{list}
\paragraph*{GetRecBlock}\hspace*{\fill}

\begin{list}{}{
\settowidth{\tmplength}{\textbf{Declaração}}
\setlength{\itemindent}{0cm}
\setlength{\listparindent}{0cm}
\setlength{\leftmargin}{\evensidemargin}
\addtolength{\leftmargin}{\tmplength}
\settowidth{\labelsep}{X}
\addtolength{\leftmargin}{\labelsep}
\setlength{\labelwidth}{\tmplength}
}
\begin{flushleft}
\item[\textbf{Declaração}\hfill]
\begin{ttfamily}
public function GetRecBlock(VAR DatF : DataFile; Const R : LONGINT; delta:Word;Var BlocksRead:Word ;VAR Buffer ):Boolean;\end{ttfamily}


\end{flushleft}
\end{list}
\paragraph*{{\_}PutRec}\hspace*{\fill}

\begin{list}{}{
\settowidth{\tmplength}{\textbf{Declaração}}
\setlength{\itemindent}{0cm}
\setlength{\listparindent}{0cm}
\setlength{\leftmargin}{\evensidemargin}
\addtolength{\leftmargin}{\tmplength}
\settowidth{\labelsep}{X}
\addtolength{\leftmargin}{\labelsep}
\setlength{\labelwidth}{\tmplength}
}
\begin{flushleft}
\item[\textbf{Declaração}\hfill]
\begin{ttfamily}
public function {\_}PutRec(var DatF : DataFile;Const R : Longint;var Buffer;Const Transaction{\_}Current : T{\_}TTransaction):Boolean;\end{ttfamily}


\end{flushleft}
\end{list}
\paragraph*{PutRec}\hspace*{\fill}

\begin{list}{}{
\settowidth{\tmplength}{\textbf{Declaração}}
\setlength{\itemindent}{0cm}
\setlength{\listparindent}{0cm}
\setlength{\leftmargin}{\evensidemargin}
\addtolength{\leftmargin}{\tmplength}
\settowidth{\labelsep}{X}
\addtolength{\leftmargin}{\labelsep}
\setlength{\labelwidth}{\tmplength}
}
\begin{flushleft}
\item[\textbf{Declaração}\hfill]
\begin{ttfamily}
public function PutRec(var DatF : DataFile;Const R : Longint;var Buffer ):Boolean;\end{ttfamily}


\end{flushleft}
\end{list}
\paragraph*{MakeFile}\hspace*{\fill}

\begin{list}{}{
\settowidth{\tmplength}{\textbf{Declaração}}
\setlength{\itemindent}{0cm}
\setlength{\listparindent}{0cm}
\setlength{\leftmargin}{\evensidemargin}
\addtolength{\leftmargin}{\tmplength}
\settowidth{\labelsep}{X}
\addtolength{\leftmargin}{\labelsep}
\setlength{\labelwidth}{\tmplength}
}
\begin{flushleft}
\item[\textbf{Declaração}\hfill]
\begin{ttfamily}
public function MakeFile(var DatF : DataFile;Const FName : TFileName;Const RecLen : SmallWord):Integer;\end{ttfamily}


\end{flushleft}
\end{list}
\paragraph*{FMakeFile}\hspace*{\fill}

\begin{list}{}{
\settowidth{\tmplength}{\textbf{Declaração}}
\setlength{\itemindent}{0cm}
\setlength{\listparindent}{0cm}
\setlength{\leftmargin}{\evensidemargin}
\addtolength{\leftmargin}{\tmplength}
\settowidth{\labelsep}{X}
\addtolength{\leftmargin}{\labelsep}
\setlength{\labelwidth}{\tmplength}
}
\begin{flushleft}
\item[\textbf{Declaração}\hfill]
\begin{ttfamily}
public function FMakeFile(Const FileName:PathStr;Const TamArq:Longint):Integer;\end{ttfamily}


\end{flushleft}
\end{list}
\paragraph*{AtualizaDestino}\hspace*{\fill}

\begin{list}{}{
\settowidth{\tmplength}{\textbf{Declaração}}
\setlength{\itemindent}{0cm}
\setlength{\listparindent}{0cm}
\setlength{\leftmargin}{\evensidemargin}
\addtolength{\leftmargin}{\tmplength}
\settowidth{\labelsep}{X}
\addtolength{\leftmargin}{\labelsep}
\setlength{\labelwidth}{\tmplength}
}
\begin{flushleft}
\item[\textbf{Declaração}\hfill]
\begin{ttfamily}
public function AtualizaDestino(Var RegFonte; Const TamFonte:SmallWord; var RegDestino; Const TamDestino : SmallWord) : BOOLEAN;\end{ttfamily}


\end{flushleft}
\end{list}
\paragraph*{FDelStrBrancos}\hspace*{\fill}

\begin{list}{}{
\settowidth{\tmplength}{\textbf{Declaração}}
\setlength{\itemindent}{0cm}
\setlength{\listparindent}{0cm}
\setlength{\leftmargin}{\evensidemargin}
\addtolength{\leftmargin}{\tmplength}
\settowidth{\labelsep}{X}
\addtolength{\leftmargin}{\labelsep}
\setlength{\labelwidth}{\tmplength}
}
\begin{flushleft}
\item[\textbf{Declaração}\hfill]
\begin{ttfamily}
public function FDelStrBrancos(S:tString):tString;\end{ttfamily}


\end{flushleft}
\end{list}
\paragraph*{FileNameTemp{\_}Ext}\hspace*{\fill}

\begin{list}{}{
\settowidth{\tmplength}{\textbf{Declaração}}
\setlength{\itemindent}{0cm}
\setlength{\listparindent}{0cm}
\setlength{\leftmargin}{\evensidemargin}
\addtolength{\leftmargin}{\tmplength}
\settowidth{\labelsep}{X}
\addtolength{\leftmargin}{\labelsep}
\setlength{\labelwidth}{\tmplength}
}
\begin{flushleft}
\item[\textbf{Declaração}\hfill]
\begin{ttfamily}
public function FileNameTemp{\_}Ext(const aPath:PathStr;Var NomeArqTemp : PathStr;Extencao : PathStr):Boolean; overload;\end{ttfamily}


\end{flushleft}
\end{list}
\paragraph*{FileNameTemp{\_}Ext}\hspace*{\fill}

\begin{list}{}{
\settowidth{\tmplength}{\textbf{Declaração}}
\setlength{\itemindent}{0cm}
\setlength{\listparindent}{0cm}
\setlength{\leftmargin}{\evensidemargin}
\addtolength{\leftmargin}{\tmplength}
\settowidth{\labelsep}{X}
\addtolength{\leftmargin}{\labelsep}
\setlength{\labelwidth}{\tmplength}
}
\begin{flushleft}
\item[\textbf{Declaração}\hfill]
\begin{ttfamily}
public function FileNameTemp{\_}Ext(Var NomeArqTemp : PathStr;Extencao : PathStr):Boolean; Overload;\end{ttfamily}


\end{flushleft}
\end{list}
\paragraph*{FileNameTemp}\hspace*{\fill}

\begin{list}{}{
\settowidth{\tmplength}{\textbf{Declaração}}
\setlength{\itemindent}{0cm}
\setlength{\listparindent}{0cm}
\setlength{\leftmargin}{\evensidemargin}
\addtolength{\leftmargin}{\tmplength}
\settowidth{\labelsep}{X}
\addtolength{\leftmargin}{\labelsep}
\setlength{\labelwidth}{\tmplength}
}
\begin{flushleft}
\item[\textbf{Declaração}\hfill]
\begin{ttfamily}
public function FileNameTemp(Extencao : PathStr):PathStr;\end{ttfamily}


\end{flushleft}
\end{list}
\paragraph*{FileName{\_}Seq}\hspace*{\fill}

\begin{list}{}{
\settowidth{\tmplength}{\textbf{Declaração}}
\setlength{\itemindent}{0cm}
\setlength{\listparindent}{0cm}
\setlength{\leftmargin}{\evensidemargin}
\addtolength{\leftmargin}{\tmplength}
\settowidth{\labelsep}{X}
\addtolength{\leftmargin}{\labelsep}
\setlength{\labelwidth}{\tmplength}
}
\begin{flushleft}
\item[\textbf{Declaração}\hfill]
\begin{ttfamily}
public function FileName{\_}Seq(Const aName:PathStr;Const aExt : PathStr):PathStr;\end{ttfamily}


\end{flushleft}
\end{list}
\paragraph*{IsPortLocal}\hspace*{\fill}

\begin{list}{}{
\settowidth{\tmplength}{\textbf{Declaração}}
\setlength{\itemindent}{0cm}
\setlength{\listparindent}{0cm}
\setlength{\leftmargin}{\evensidemargin}
\addtolength{\leftmargin}{\tmplength}
\settowidth{\labelsep}{X}
\addtolength{\leftmargin}{\labelsep}
\setlength{\labelwidth}{\tmplength}
}
\begin{flushleft}
\item[\textbf{Declaração}\hfill]
\begin{ttfamily}
public function IsPortLocal(WPort: tString):Boolean;\end{ttfamily}


\end{flushleft}
\end{list}
\paragraph*{DelFile}\hspace*{\fill}

\begin{list}{}{
\settowidth{\tmplength}{\textbf{Declaração}}
\setlength{\itemindent}{0cm}
\setlength{\listparindent}{0cm}
\setlength{\leftmargin}{\evensidemargin}
\addtolength{\leftmargin}{\tmplength}
\settowidth{\labelsep}{X}
\addtolength{\leftmargin}{\labelsep}
\setlength{\labelwidth}{\tmplength}
}
\begin{flushleft}
\item[\textbf{Declaração}\hfill]
\begin{ttfamily}
public function DelFile( Const Nome : TFileName):Boolean;\end{ttfamily}


\end{flushleft}
\end{list}
\paragraph*{SetOkAddRecFirstFree}\hspace*{\fill}

\begin{list}{}{
\settowidth{\tmplength}{\textbf{Declaração}}
\setlength{\itemindent}{0cm}
\setlength{\listparindent}{0cm}
\setlength{\leftmargin}{\evensidemargin}
\addtolength{\leftmargin}{\tmplength}
\settowidth{\labelsep}{X}
\addtolength{\leftmargin}{\labelsep}
\setlength{\labelwidth}{\tmplength}
}
\begin{flushleft}
\item[\textbf{Declaração}\hfill]
\begin{ttfamily}
public function SetOkAddRecFirstFree(Const aOkAddRecFirstFree: Boolean):Boolean;\end{ttfamily}


\end{flushleft}
\end{list}
\paragraph*{TestaSePodeAbrirArquivo}\hspace*{\fill}

\begin{list}{}{
\settowidth{\tmplength}{\textbf{Declaração}}
\setlength{\itemindent}{0cm}
\setlength{\listparindent}{0cm}
\setlength{\leftmargin}{\evensidemargin}
\addtolength{\leftmargin}{\tmplength}
\settowidth{\labelsep}{X}
\addtolength{\leftmargin}{\labelsep}
\setlength{\labelwidth}{\tmplength}
}
\begin{flushleft}
\item[\textbf{Declaração}\hfill]
\begin{ttfamily}
public function TestaSePodeAbrirArquivo(Const FName : PathStr): Byte;\end{ttfamily}


\end{flushleft}
\end{list}
\paragraph*{FileShared}\hspace*{\fill}

\begin{list}{}{
\settowidth{\tmplength}{\textbf{Declaração}}
\setlength{\itemindent}{0cm}
\setlength{\listparindent}{0cm}
\setlength{\leftmargin}{\evensidemargin}
\addtolength{\leftmargin}{\tmplength}
\settowidth{\labelsep}{X}
\addtolength{\leftmargin}{\labelsep}
\setlength{\labelwidth}{\tmplength}
}
\begin{flushleft}
\item[\textbf{Declaração}\hfill]
\begin{ttfamily}
public function FileShared(Const FName : PathStr) : Boolean;\end{ttfamily}


\end{flushleft}
\end{list}
\paragraph*{FTrocaExtencao}\hspace*{\fill}

\begin{list}{}{
\settowidth{\tmplength}{\textbf{Declaração}}
\setlength{\itemindent}{0cm}
\setlength{\listparindent}{0cm}
\setlength{\leftmargin}{\evensidemargin}
\addtolength{\leftmargin}{\tmplength}
\settowidth{\labelsep}{X}
\addtolength{\leftmargin}{\labelsep}
\setlength{\labelwidth}{\tmplength}
}
\begin{flushleft}
\item[\textbf{Declaração}\hfill]
\begin{ttfamily}
public function FTrocaExtencao(Const NomeArq:TFileName; Extencao:PathStr) : PathStr;\end{ttfamily}


\end{flushleft}
\end{list}
\paragraph*{Ren}\hspace*{\fill}

\begin{list}{}{
\settowidth{\tmplength}{\textbf{Declaração}}
\setlength{\itemindent}{0cm}
\setlength{\listparindent}{0cm}
\setlength{\leftmargin}{\evensidemargin}
\addtolength{\leftmargin}{\tmplength}
\settowidth{\labelsep}{X}
\addtolength{\leftmargin}{\labelsep}
\setlength{\labelwidth}{\tmplength}
}
\begin{flushleft}
\item[\textbf{Declaração}\hfill]
\begin{ttfamily}
public function Ren(NomeFonte,NomeDestino: PathStr) : Boolean;\end{ttfamily}


\end{flushleft}
\end{list}
\paragraph*{OkRecSizeMismatch}\hspace*{\fill}

\begin{list}{}{
\settowidth{\tmplength}{\textbf{Declaração}}
\setlength{\itemindent}{0cm}
\setlength{\listparindent}{0cm}
\setlength{\leftmargin}{\evensidemargin}
\addtolength{\leftmargin}{\tmplength}
\settowidth{\labelsep}{X}
\addtolength{\leftmargin}{\labelsep}
\setlength{\labelwidth}{\tmplength}
}
\begin{flushleft}
\item[\textbf{Declaração}\hfill]
\begin{ttfamily}
public function OkRecSizeMismatch(Const FName : TFileName;Const RecLenBufferRecord : SmallWord):Boolean;\end{ttfamily}


\end{flushleft}
\end{list}
\paragraph*{ModifyStructurFile}\hspace*{\fill}

\begin{list}{}{
\settowidth{\tmplength}{\textbf{Declaração}}
\setlength{\itemindent}{0cm}
\setlength{\listparindent}{0cm}
\setlength{\leftmargin}{\evensidemargin}
\addtolength{\leftmargin}{\tmplength}
\settowidth{\labelsep}{X}
\addtolength{\leftmargin}{\labelsep}
\setlength{\labelwidth}{\tmplength}
}
\begin{flushleft}
\item[\textbf{Declaração}\hfill]
\begin{ttfamily}
public function ModifyStructurFile(Const FName:TFileName;Const RecLenDest : SmallWord ):Boolean;\end{ttfamily}


\end{flushleft}
\end{list}
\paragraph*{OpenFile}\hspace*{\fill}

\begin{list}{}{
\settowidth{\tmplength}{\textbf{Declaração}}
\setlength{\itemindent}{0cm}
\setlength{\listparindent}{0cm}
\setlength{\leftmargin}{\evensidemargin}
\addtolength{\leftmargin}{\tmplength}
\settowidth{\labelsep}{X}
\addtolength{\leftmargin}{\labelsep}
\setlength{\labelwidth}{\tmplength}
}
\begin{flushleft}
\item[\textbf{Declaração}\hfill]
\begin{ttfamily}
public function OpenFile(var DatF:DataFile; Const FName : TFileName; Const RecLen:SmallWord; AFileMode:Word ):Integer;\end{ttfamily}


\end{flushleft}
\end{list}
\paragraph*{ReadHeader}\hspace*{\fill}

\begin{list}{}{
\settowidth{\tmplength}{\textbf{Declaração}}
\setlength{\itemindent}{0cm}
\setlength{\listparindent}{0cm}
\setlength{\leftmargin}{\evensidemargin}
\addtolength{\leftmargin}{\tmplength}
\settowidth{\labelsep}{X}
\addtolength{\leftmargin}{\labelsep}
\setlength{\labelwidth}{\tmplength}
}
\begin{flushleft}
\item[\textbf{Declaração}\hfill]
\begin{ttfamily}
public function ReadHeader(VAR DatF : DataFile):Boolean;\end{ttfamily}


\end{flushleft}
\end{list}
\paragraph*{PutFileHeader}\hspace*{\fill}

\begin{list}{}{
\settowidth{\tmplength}{\textbf{Declaração}}
\setlength{\itemindent}{0cm}
\setlength{\listparindent}{0cm}
\setlength{\leftmargin}{\evensidemargin}
\addtolength{\leftmargin}{\tmplength}
\settowidth{\labelsep}{X}
\addtolength{\leftmargin}{\labelsep}
\setlength{\labelwidth}{\tmplength}
}
\begin{flushleft}
\item[\textbf{Declaração}\hfill]
\begin{ttfamily}
public function PutFileHeader(VAR DatF : DataFile):Boolean;\end{ttfamily}


\end{flushleft}
\end{list}
\paragraph*{NaoMuDOuHeader}\hspace*{\fill}

\begin{list}{}{
\settowidth{\tmplength}{\textbf{Declaração}}
\setlength{\itemindent}{0cm}
\setlength{\listparindent}{0cm}
\setlength{\leftmargin}{\evensidemargin}
\addtolength{\leftmargin}{\tmplength}
\settowidth{\labelsep}{X}
\addtolength{\leftmargin}{\labelsep}
\setlength{\labelwidth}{\tmplength}
}
\begin{flushleft}
\item[\textbf{Declaração}\hfill]
\begin{ttfamily}
public function NaoMuDOuHeader(VAR DatF : DataFile) : BOOLEAN;\end{ttfamily}


\end{flushleft}
\end{list}
\paragraph*{MudouHeaderEmMemoria}\hspace*{\fill}

\begin{list}{}{
\settowidth{\tmplength}{\textbf{Declaração}}
\setlength{\itemindent}{0cm}
\setlength{\listparindent}{0cm}
\setlength{\leftmargin}{\evensidemargin}
\addtolength{\leftmargin}{\tmplength}
\settowidth{\labelsep}{X}
\addtolength{\leftmargin}{\labelsep}
\setlength{\labelwidth}{\tmplength}
}
\begin{flushleft}
\item[\textbf{Declaração}\hfill]
\begin{ttfamily}
public function MudouHeaderEmMemoria(VAR DatF : DataFile) : BOOLEAN;\end{ttfamily}


\end{flushleft}
\end{list}
\paragraph*{aCloseFile}\hspace*{\fill}

\begin{list}{}{
\settowidth{\tmplength}{\textbf{Declaração}}
\setlength{\itemindent}{0cm}
\setlength{\listparindent}{0cm}
\setlength{\leftmargin}{\evensidemargin}
\addtolength{\leftmargin}{\tmplength}
\settowidth{\labelsep}{X}
\addtolength{\leftmargin}{\labelsep}
\setlength{\labelwidth}{\tmplength}
}
\begin{flushleft}
\item[\textbf{Declaração}\hfill]
\begin{ttfamily}
public function aCloseFile(VAR DatF : DataFile):boolean;\end{ttfamily}


\end{flushleft}
\end{list}
\paragraph*{CloseFile}\hspace*{\fill}

\begin{list}{}{
\settowidth{\tmplength}{\textbf{Declaração}}
\setlength{\itemindent}{0cm}
\setlength{\listparindent}{0cm}
\setlength{\leftmargin}{\evensidemargin}
\addtolength{\leftmargin}{\tmplength}
\settowidth{\labelsep}{X}
\addtolength{\leftmargin}{\labelsep}
\setlength{\labelwidth}{\tmplength}
}
\begin{flushleft}
\item[\textbf{Declaração}\hfill]
\begin{ttfamily}
public function CloseFile(VAR DatF : DataFile):boolean; Overload;\end{ttfamily}


\end{flushleft}
\end{list}
\paragraph*{CloseFile}\hspace*{\fill}

\begin{list}{}{
\settowidth{\tmplength}{\textbf{Declaração}}
\setlength{\itemindent}{0cm}
\setlength{\listparindent}{0cm}
\setlength{\leftmargin}{\evensidemargin}
\addtolength{\leftmargin}{\tmplength}
\settowidth{\labelsep}{X}
\addtolength{\leftmargin}{\labelsep}
\setlength{\labelwidth}{\tmplength}
}
\begin{flushleft}
\item[\textbf{Declaração}\hfill]
\begin{ttfamily}
public function CloseFile(VAR DatF : DataFile;OkCondicional:Boolean):boolean; Overload;\end{ttfamily}


\end{flushleft}
\end{list}
\paragraph*{FlushFile}\hspace*{\fill}

\begin{list}{}{
\settowidth{\tmplength}{\textbf{Declaração}}
\setlength{\itemindent}{0cm}
\setlength{\listparindent}{0cm}
\setlength{\leftmargin}{\evensidemargin}
\addtolength{\leftmargin}{\tmplength}
\settowidth{\labelsep}{X}
\addtolength{\leftmargin}{\labelsep}
\setlength{\labelwidth}{\tmplength}
}
\begin{flushleft}
\item[\textbf{Declaração}\hfill]
\begin{ttfamily}
public function FlushFile(VAR DatF : DataFile):Boolean;\end{ttfamily}


\end{flushleft}
\end{list}
\paragraph*{TraveRegistro}\hspace*{\fill}

\begin{list}{}{
\settowidth{\tmplength}{\textbf{Declaração}}
\setlength{\itemindent}{0cm}
\setlength{\listparindent}{0cm}
\setlength{\leftmargin}{\evensidemargin}
\addtolength{\leftmargin}{\tmplength}
\settowidth{\labelsep}{X}
\addtolength{\leftmargin}{\labelsep}
\setlength{\labelwidth}{\tmplength}
}
\begin{flushleft}
\item[\textbf{Declaração}\hfill]
\begin{ttfamily}
public function TraveRegistro(VAR DatF : DataFile; Const R : LONGINT):Boolean;\end{ttfamily}


\end{flushleft}
\end{list}
\paragraph*{DestraveRegistro}\hspace*{\fill}

\begin{list}{}{
\settowidth{\tmplength}{\textbf{Declaração}}
\setlength{\itemindent}{0cm}
\setlength{\listparindent}{0cm}
\setlength{\leftmargin}{\evensidemargin}
\addtolength{\leftmargin}{\tmplength}
\settowidth{\labelsep}{X}
\addtolength{\leftmargin}{\labelsep}
\setlength{\labelwidth}{\tmplength}
}
\begin{flushleft}
\item[\textbf{Declaração}\hfill]
\begin{ttfamily}
public function DestraveRegistro(Var DatF : DataFile;Const R : Longint):Boolean;\end{ttfamily}


\end{flushleft}
\end{list}
\paragraph*{TraveHeader}\hspace*{\fill}

\begin{list}{}{
\settowidth{\tmplength}{\textbf{Declaração}}
\setlength{\itemindent}{0cm}
\setlength{\listparindent}{0cm}
\setlength{\leftmargin}{\evensidemargin}
\addtolength{\leftmargin}{\tmplength}
\settowidth{\labelsep}{X}
\addtolength{\leftmargin}{\labelsep}
\setlength{\labelwidth}{\tmplength}
}
\begin{flushleft}
\item[\textbf{Declaração}\hfill]
\begin{ttfamily}
public function TraveHeader(VAR DatF : DataFile):Boolean;\end{ttfamily}


\end{flushleft}
\end{list}
\paragraph*{DestraveHeader}\hspace*{\fill}

\begin{list}{}{
\settowidth{\tmplength}{\textbf{Declaração}}
\setlength{\itemindent}{0cm}
\setlength{\listparindent}{0cm}
\setlength{\leftmargin}{\evensidemargin}
\addtolength{\leftmargin}{\tmplength}
\settowidth{\labelsep}{X}
\addtolength{\leftmargin}{\labelsep}
\setlength{\labelwidth}{\tmplength}
}
\begin{flushleft}
\item[\textbf{Declaração}\hfill]
\begin{ttfamily}
public function DestraveHeader(VAR DatF : DataFile):Boolean;\end{ttfamily}


\end{flushleft}
\end{list}
\paragraph*{FileSize}\hspace*{\fill}

\begin{list}{}{
\settowidth{\tmplength}{\textbf{Declaração}}
\setlength{\itemindent}{0cm}
\setlength{\listparindent}{0cm}
\setlength{\leftmargin}{\evensidemargin}
\addtolength{\leftmargin}{\tmplength}
\settowidth{\labelsep}{X}
\addtolength{\leftmargin}{\labelsep}
\setlength{\labelwidth}{\tmplength}
}
\begin{flushleft}
\item[\textbf{Declaração}\hfill]
\begin{ttfamily}
public function FileSize(VAR DatF : DataFile):Longint; Overload;\end{ttfamily}


\end{flushleft}
\end{list}
\paragraph*{NewRec}\hspace*{\fill}

\begin{list}{}{
\settowidth{\tmplength}{\textbf{Declaração}}
\setlength{\itemindent}{0cm}
\setlength{\listparindent}{0cm}
\setlength{\leftmargin}{\evensidemargin}
\addtolength{\leftmargin}{\tmplength}
\settowidth{\labelsep}{X}
\addtolength{\leftmargin}{\labelsep}
\setlength{\labelwidth}{\tmplength}
}
\begin{flushleft}
\item[\textbf{Declaração}\hfill]
\begin{ttfamily}
public procedure NewRec(VAR DatF : DataFile;VAR R : LONGINT );\end{ttfamily}


\end{flushleft}
\end{list}
\paragraph*{AddRec}\hspace*{\fill}

\begin{list}{}{
\settowidth{\tmplength}{\textbf{Declaração}}
\setlength{\itemindent}{0cm}
\setlength{\listparindent}{0cm}
\setlength{\leftmargin}{\evensidemargin}
\addtolength{\leftmargin}{\tmplength}
\settowidth{\labelsep}{X}
\addtolength{\leftmargin}{\labelsep}
\setlength{\labelwidth}{\tmplength}
}
\begin{flushleft}
\item[\textbf{Declaração}\hfill]
\begin{ttfamily}
public function AddRec(var DatF : DataFile;var R : Longint;var Buffer ):Boolean;\end{ttfamily}


\end{flushleft}
\end{list}
\paragraph*{DeleteRecord}\hspace*{\fill}

\begin{list}{}{
\settowidth{\tmplength}{\textbf{Declaração}}
\setlength{\itemindent}{0cm}
\setlength{\listparindent}{0cm}
\setlength{\leftmargin}{\evensidemargin}
\addtolength{\leftmargin}{\tmplength}
\settowidth{\labelsep}{X}
\addtolength{\leftmargin}{\labelsep}
\setlength{\labelwidth}{\tmplength}
}
\begin{flushleft}
\item[\textbf{Declaração}\hfill]
\begin{ttfamily}
public function DeleteRecord(VAR DatF : DataFile; Const R : LONGINT; Var Buffer ):Boolean;\end{ttfamily}


\end{flushleft}
\end{list}
\paragraph*{DeleteRec}\hspace*{\fill}

\begin{list}{}{
\settowidth{\tmplength}{\textbf{Declaração}}
\setlength{\itemindent}{0cm}
\setlength{\listparindent}{0cm}
\setlength{\leftmargin}{\evensidemargin}
\addtolength{\leftmargin}{\tmplength}
\settowidth{\labelsep}{X}
\addtolength{\leftmargin}{\labelsep}
\setlength{\labelwidth}{\tmplength}
}
\begin{flushleft}
\item[\textbf{Declaração}\hfill]
\begin{ttfamily}
public function DeleteRec(var DatF : DataFile;Const R: Longint):Boolean;\end{ttfamily}


\end{flushleft}
\end{list}
\paragraph*{FileLen}\hspace*{\fill}

\begin{list}{}{
\settowidth{\tmplength}{\textbf{Declaração}}
\setlength{\itemindent}{0cm}
\setlength{\listparindent}{0cm}
\setlength{\leftmargin}{\evensidemargin}
\addtolength{\leftmargin}{\tmplength}
\settowidth{\labelsep}{X}
\addtolength{\leftmargin}{\labelsep}
\setlength{\labelwidth}{\tmplength}
}
\begin{flushleft}
\item[\textbf{Declaração}\hfill]
\begin{ttfamily}
public function FileLen(VAR DatF : DataFile) : LONGINT;\end{ttfamily}


\end{flushleft}
\end{list}
\paragraph*{UsedRecs}\hspace*{\fill}

\begin{list}{}{
\settowidth{\tmplength}{\textbf{Declaração}}
\setlength{\itemindent}{0cm}
\setlength{\listparindent}{0cm}
\setlength{\leftmargin}{\evensidemargin}
\addtolength{\leftmargin}{\tmplength}
\settowidth{\labelsep}{X}
\addtolength{\leftmargin}{\labelsep}
\setlength{\labelwidth}{\tmplength}
}
\begin{flushleft}
\item[\textbf{Declaração}\hfill]
\begin{ttfamily}
public function UsedRecs(VAR DatF : DataFile) : LONGINT; Overload;\end{ttfamily}


\end{flushleft}
\end{list}
\paragraph*{UsedRecs}\hspace*{\fill}

\begin{list}{}{
\settowidth{\tmplength}{\textbf{Declaração}}
\setlength{\itemindent}{0cm}
\setlength{\listparindent}{0cm}
\setlength{\leftmargin}{\evensidemargin}
\addtolength{\leftmargin}{\tmplength}
\settowidth{\labelsep}{X}
\addtolength{\leftmargin}{\labelsep}
\setlength{\labelwidth}{\tmplength}
}
\begin{flushleft}
\item[\textbf{Declaração}\hfill]
\begin{ttfamily}
public function UsedRecs(VAR DatF : DataFile;OK{\_}GetHeader : Boolean) : LONGINT; Overload;\end{ttfamily}


\end{flushleft}
\end{list}
\paragraph*{UsedRecs}\hspace*{\fill}

\begin{list}{}{
\settowidth{\tmplength}{\textbf{Declaração}}
\setlength{\itemindent}{0cm}
\setlength{\listparindent}{0cm}
\setlength{\leftmargin}{\evensidemargin}
\addtolength{\leftmargin}{\tmplength}
\settowidth{\labelsep}{X}
\addtolength{\leftmargin}{\labelsep}
\setlength{\labelwidth}{\tmplength}
}
\begin{flushleft}
\item[\textbf{Declaração}\hfill]
\begin{ttfamily}
public function UsedRecs(Var IxF :IndexFile;OK{\_}GetHeaderDoIndice : Boolean) : LONGINT; Overload;\end{ttfamily}


\end{flushleft}
\end{list}
\paragraph*{UsedRecs}\hspace*{\fill}

\begin{list}{}{
\settowidth{\tmplength}{\textbf{Declaração}}
\setlength{\itemindent}{0cm}
\setlength{\listparindent}{0cm}
\setlength{\leftmargin}{\evensidemargin}
\addtolength{\leftmargin}{\tmplength}
\settowidth{\labelsep}{X}
\addtolength{\leftmargin}{\labelsep}
\setlength{\labelwidth}{\tmplength}
}
\begin{flushleft}
\item[\textbf{Declaração}\hfill]
\begin{ttfamily}
public function UsedRecs(Var IxF :IndexFile) : LONGINT; Overload;\end{ttfamily}


\end{flushleft}
\end{list}
\paragraph*{UsedRecs}\hspace*{\fill}

\begin{list}{}{
\settowidth{\tmplength}{\textbf{Declaração}}
\setlength{\itemindent}{0cm}
\setlength{\listparindent}{0cm}
\setlength{\leftmargin}{\evensidemargin}
\addtolength{\leftmargin}{\tmplength}
\settowidth{\labelsep}{X}
\addtolength{\leftmargin}{\labelsep}
\setlength{\labelwidth}{\tmplength}
}
\begin{flushleft}
\item[\textbf{Declaração}\hfill]
\begin{ttfamily}
public function UsedRecs(Const FileName:PathStr) : Longint; Overload;\end{ttfamily}


\end{flushleft}
\end{list}
\paragraph*{TaPack}\hspace*{\fill}

\begin{list}{}{
\settowidth{\tmplength}{\textbf{Declaração}}
\setlength{\itemindent}{0cm}
\setlength{\listparindent}{0cm}
\setlength{\leftmargin}{\evensidemargin}
\addtolength{\leftmargin}{\tmplength}
\settowidth{\labelsep}{X}
\addtolength{\leftmargin}{\labelsep}
\setlength{\labelwidth}{\tmplength}
}
\begin{flushleft}
\item[\textbf{Declaração}\hfill]
\begin{ttfamily}
public procedure TaPack(VAR Page : TaPage;Const KeyL : BYTE);\end{ttfamily}


\end{flushleft}
\end{list}
\paragraph*{TaUnpack}\hspace*{\fill}

\begin{list}{}{
\settowidth{\tmplength}{\textbf{Declaração}}
\setlength{\itemindent}{0cm}
\setlength{\listparindent}{0cm}
\setlength{\leftmargin}{\evensidemargin}
\addtolength{\leftmargin}{\tmplength}
\settowidth{\labelsep}{X}
\addtolength{\leftmargin}{\labelsep}
\setlength{\labelwidth}{\tmplength}
}
\begin{flushleft}
\item[\textbf{Declaração}\hfill]
\begin{ttfamily}
public procedure TaUnpack(VAR Page : TaPage; Const KeyL : BYTE);\end{ttfamily}


\end{flushleft}
\end{list}
\paragraph*{Multiplo{\_}Mais{\_}proximo{\_}de{\_}N}\hspace*{\fill}

\begin{list}{}{
\settowidth{\tmplength}{\textbf{Declaração}}
\setlength{\itemindent}{0cm}
\setlength{\listparindent}{0cm}
\setlength{\leftmargin}{\evensidemargin}
\addtolength{\leftmargin}{\tmplength}
\settowidth{\labelsep}{X}
\addtolength{\leftmargin}{\labelsep}
\setlength{\labelwidth}{\tmplength}
}
\begin{flushleft}
\item[\textbf{Declaração}\hfill]
\begin{ttfamily}
public function Multiplo{\_}Mais{\_}proximo{\_}de{\_}N(Const K,N:Longint): Longint;\end{ttfamily}


\end{flushleft}
\end{list}
\paragraph*{MakeIndex}\hspace*{\fill}

\begin{list}{}{
\settowidth{\tmplength}{\textbf{Declaração}}
\setlength{\itemindent}{0cm}
\setlength{\listparindent}{0cm}
\setlength{\leftmargin}{\evensidemargin}
\addtolength{\leftmargin}{\tmplength}
\settowidth{\labelsep}{X}
\addtolength{\leftmargin}{\labelsep}
\setlength{\labelwidth}{\tmplength}
}
\begin{flushleft}
\item[\textbf{Declaração}\hfill]
\begin{ttfamily}
public function MakeIndex(var IxF : IndexFile;Const FName : TFileName;Const KeyLen,S : byte):Integer;\end{ttfamily}


\end{flushleft}
\end{list}
\paragraph*{FMakeIndex}\hspace*{\fill}

\begin{list}{}{
\settowidth{\tmplength}{\textbf{Declaração}}
\setlength{\itemindent}{0cm}
\setlength{\listparindent}{0cm}
\setlength{\leftmargin}{\evensidemargin}
\addtolength{\leftmargin}{\tmplength}
\settowidth{\labelsep}{X}
\addtolength{\leftmargin}{\labelsep}
\setlength{\labelwidth}{\tmplength}
}
\begin{flushleft}
\item[\textbf{Declaração}\hfill]
\begin{ttfamily}
public function FMakeIndex(Const FileName:PathStr;Const RepeteChave,TamChave:Byte):Integer;\end{ttfamily}


\end{flushleft}
\end{list}
\paragraph*{LeiaHeaderDoIndice}\hspace*{\fill}

\begin{list}{}{
\settowidth{\tmplength}{\textbf{Declaração}}
\setlength{\itemindent}{0cm}
\setlength{\listparindent}{0cm}
\setlength{\leftmargin}{\evensidemargin}
\addtolength{\leftmargin}{\tmplength}
\settowidth{\labelsep}{X}
\addtolength{\leftmargin}{\labelsep}
\setlength{\labelwidth}{\tmplength}
}
\begin{flushleft}
\item[\textbf{Declaração}\hfill]
\begin{ttfamily}
public procedure LeiaHeaderDoIndice( VAR IxF : IndexFile);\end{ttfamily}


\end{flushleft}
\end{list}
\paragraph*{aCloseIndex}\hspace*{\fill}

\begin{list}{}{
\settowidth{\tmplength}{\textbf{Declaração}}
\setlength{\itemindent}{0cm}
\setlength{\listparindent}{0cm}
\setlength{\leftmargin}{\evensidemargin}
\addtolength{\leftmargin}{\tmplength}
\settowidth{\labelsep}{X}
\addtolength{\leftmargin}{\labelsep}
\setlength{\labelwidth}{\tmplength}
}
\begin{flushleft}
\item[\textbf{Declaração}\hfill]
\begin{ttfamily}
public function aCloseIndex(VAR IxF : IndexFile):Boolean;\end{ttfamily}


\end{flushleft}
\end{list}
\paragraph*{CloseIndex}\hspace*{\fill}

\begin{list}{}{
\settowidth{\tmplength}{\textbf{Declaração}}
\setlength{\itemindent}{0cm}
\setlength{\listparindent}{0cm}
\setlength{\leftmargin}{\evensidemargin}
\addtolength{\leftmargin}{\tmplength}
\settowidth{\labelsep}{X}
\addtolength{\leftmargin}{\labelsep}
\setlength{\labelwidth}{\tmplength}
}
\begin{flushleft}
\item[\textbf{Declaração}\hfill]
\begin{ttfamily}
public function CloseIndex(VAR IxF : IndexFile;OkCondicional:Boolean):boolean; Overload;\end{ttfamily}


\end{flushleft}
\end{list}
\paragraph*{CloseIndex}\hspace*{\fill}

\begin{list}{}{
\settowidth{\tmplength}{\textbf{Declaração}}
\setlength{\itemindent}{0cm}
\setlength{\listparindent}{0cm}
\setlength{\leftmargin}{\evensidemargin}
\addtolength{\leftmargin}{\tmplength}
\settowidth{\labelsep}{X}
\addtolength{\leftmargin}{\labelsep}
\setlength{\labelwidth}{\tmplength}
}
\begin{flushleft}
\item[\textbf{Declaração}\hfill]
\begin{ttfamily}
public function CloseIndex(VAR IxF : IndexFile):boolean; Overload;\end{ttfamily}


\end{flushleft}
\end{list}
\paragraph*{FlushIndex}\hspace*{\fill}

\begin{list}{}{
\settowidth{\tmplength}{\textbf{Declaração}}
\setlength{\itemindent}{0cm}
\setlength{\listparindent}{0cm}
\setlength{\leftmargin}{\evensidemargin}
\addtolength{\leftmargin}{\tmplength}
\settowidth{\labelsep}{X}
\addtolength{\leftmargin}{\labelsep}
\setlength{\labelwidth}{\tmplength}
}
\begin{flushleft}
\item[\textbf{Declaração}\hfill]
\begin{ttfamily}
public function FlushIndex(VAR IxF : IndexFile):boolean;\end{ttfamily}


\end{flushleft}
\end{list}
\paragraph*{EraseFile}\hspace*{\fill}

\begin{list}{}{
\settowidth{\tmplength}{\textbf{Declaração}}
\setlength{\itemindent}{0cm}
\setlength{\listparindent}{0cm}
\setlength{\leftmargin}{\evensidemargin}
\addtolength{\leftmargin}{\tmplength}
\settowidth{\labelsep}{X}
\addtolength{\leftmargin}{\labelsep}
\setlength{\labelwidth}{\tmplength}
}
\begin{flushleft}
\item[\textbf{Declaração}\hfill]
\begin{ttfamily}
public function EraseFile(VAR DatF : DataFile):boolean;\end{ttfamily}


\end{flushleft}
\end{list}
\paragraph*{EraseIndex}\hspace*{\fill}

\begin{list}{}{
\settowidth{\tmplength}{\textbf{Declaração}}
\setlength{\itemindent}{0cm}
\setlength{\listparindent}{0cm}
\setlength{\leftmargin}{\evensidemargin}
\addtolength{\leftmargin}{\tmplength}
\settowidth{\labelsep}{X}
\addtolength{\leftmargin}{\labelsep}
\setlength{\labelwidth}{\tmplength}
}
\begin{flushleft}
\item[\textbf{Declaração}\hfill]
\begin{ttfamily}
public function EraseIndex(VAR IxF : IndexFile):boolean;\end{ttfamily}


\end{flushleft}
\end{list}
\paragraph*{TaGetPage}\hspace*{\fill}

\begin{list}{}{
\settowidth{\tmplength}{\textbf{Declaração}}
\setlength{\itemindent}{0cm}
\setlength{\listparindent}{0cm}
\setlength{\leftmargin}{\evensidemargin}
\addtolength{\leftmargin}{\tmplength}
\settowidth{\labelsep}{X}
\addtolength{\leftmargin}{\labelsep}
\setlength{\labelwidth}{\tmplength}
}
\begin{flushleft}
\item[\textbf{Declaração}\hfill]
\begin{ttfamily}
public function TaGetPage(VAR IxF : IndexFile;Const R : LONGINT;VAR PgPtr : TaPagePtr):boolean;\end{ttfamily}


\end{flushleft}
\end{list}
\paragraph*{TaNewPage}\hspace*{\fill}

\begin{list}{}{
\settowidth{\tmplength}{\textbf{Declaração}}
\setlength{\itemindent}{0cm}
\setlength{\listparindent}{0cm}
\setlength{\leftmargin}{\evensidemargin}
\addtolength{\leftmargin}{\tmplength}
\settowidth{\labelsep}{X}
\addtolength{\leftmargin}{\labelsep}
\setlength{\labelwidth}{\tmplength}
}
\begin{flushleft}
\item[\textbf{Declaração}\hfill]
\begin{ttfamily}
public procedure TaNewPage(VAR IxF : IndexFile; VAR R : LONGINT; VAR PgPtr : TaPagePtr);\end{ttfamily}


\end{flushleft}
\end{list}
\paragraph*{TaDeletePage}\hspace*{\fill}

\begin{list}{}{
\settowidth{\tmplength}{\textbf{Declaração}}
\setlength{\itemindent}{0cm}
\setlength{\listparindent}{0cm}
\setlength{\leftmargin}{\evensidemargin}
\addtolength{\leftmargin}{\tmplength}
\settowidth{\labelsep}{X}
\addtolength{\leftmargin}{\labelsep}
\setlength{\labelwidth}{\tmplength}
}
\begin{flushleft}
\item[\textbf{Declaração}\hfill]
\begin{ttfamily}
public procedure TaDeletePage(var IxF : IndexFile; VAR R : LONGINT; VAR PgPtr : TaPagePtr);\end{ttfamily}


\end{flushleft}
\end{list}
\paragraph*{ClearKey}\hspace*{\fill}

\begin{list}{}{
\settowidth{\tmplength}{\textbf{Declaração}}
\setlength{\itemindent}{0cm}
\setlength{\listparindent}{0cm}
\setlength{\leftmargin}{\evensidemargin}
\addtolength{\leftmargin}{\tmplength}
\settowidth{\labelsep}{X}
\addtolength{\leftmargin}{\labelsep}
\setlength{\labelwidth}{\tmplength}
}
\begin{flushleft}
\item[\textbf{Declaração}\hfill]
\begin{ttfamily}
public procedure ClearKey(VAR IxF : IndexFile);\end{ttfamily}


\end{flushleft}
\end{list}
\paragraph*{NextKey}\hspace*{\fill}

\begin{list}{}{
\settowidth{\tmplength}{\textbf{Declaração}}
\setlength{\itemindent}{0cm}
\setlength{\listparindent}{0cm}
\setlength{\leftmargin}{\evensidemargin}
\addtolength{\leftmargin}{\tmplength}
\settowidth{\labelsep}{X}
\addtolength{\leftmargin}{\labelsep}
\setlength{\labelwidth}{\tmplength}
}
\begin{flushleft}
\item[\textbf{Declaração}\hfill]
\begin{ttfamily}
public function NextKey(VAR IxF : IndexFile; VAR DataRecNum : LONGINT; VAR ProcKey ):Boolean;\end{ttfamily}


\end{flushleft}
\end{list}
\paragraph*{PrevKey}\hspace*{\fill}

\begin{list}{}{
\settowidth{\tmplength}{\textbf{Declaração}}
\setlength{\itemindent}{0cm}
\setlength{\listparindent}{0cm}
\setlength{\leftmargin}{\evensidemargin}
\addtolength{\leftmargin}{\tmplength}
\settowidth{\labelsep}{X}
\addtolength{\leftmargin}{\labelsep}
\setlength{\labelwidth}{\tmplength}
}
\begin{flushleft}
\item[\textbf{Declaração}\hfill]
\begin{ttfamily}
public function PrevKey(var IxF : IndexFile; var DataRecNum : Longint; var ProcKey ):Boolean;\end{ttfamily}


\end{flushleft}
\end{list}
\paragraph*{TaXKey}\hspace*{\fill}

\begin{list}{}{
\settowidth{\tmplength}{\textbf{Declaração}}
\setlength{\itemindent}{0cm}
\setlength{\listparindent}{0cm}
\setlength{\leftmargin}{\evensidemargin}
\addtolength{\leftmargin}{\tmplength}
\settowidth{\labelsep}{X}
\addtolength{\leftmargin}{\labelsep}
\setlength{\labelwidth}{\tmplength}
}
\begin{flushleft}
\item[\textbf{Declaração}\hfill]
\begin{ttfamily}
public procedure TaXKey(VAR K:TaKeyStr; Const KeyL : BYTE);\end{ttfamily}


\end{flushleft}
\end{list}
\paragraph*{TaCompKeys}\hspace*{\fill}

\begin{list}{}{
\settowidth{\tmplength}{\textbf{Declaração}}
\setlength{\itemindent}{0cm}
\setlength{\listparindent}{0cm}
\setlength{\leftmargin}{\evensidemargin}
\addtolength{\leftmargin}{\tmplength}
\settowidth{\labelsep}{X}
\addtolength{\leftmargin}{\labelsep}
\setlength{\labelwidth}{\tmplength}
}
\begin{flushleft}
\item[\textbf{Declaração}\hfill]
\begin{ttfamily}
public function TaCompKeys(Const K1 ,K2; DR1,DR2 : LONGINT; Const Dup : BOOLEAN ) : Shortint;\end{ttfamily}


\end{flushleft}
\end{list}
\paragraph*{TaFindKey}\hspace*{\fill}

\begin{list}{}{
\settowidth{\tmplength}{\textbf{Declaração}}
\setlength{\itemindent}{0cm}
\setlength{\listparindent}{0cm}
\setlength{\leftmargin}{\evensidemargin}
\addtolength{\leftmargin}{\tmplength}
\settowidth{\labelsep}{X}
\addtolength{\leftmargin}{\labelsep}
\setlength{\labelwidth}{\tmplength}
}
\begin{flushleft}
\item[\textbf{Declaração}\hfill]
\begin{ttfamily}
public function TaFindKey(VAR IxF : IndexFile;VAR DataRecNum : LONGINT;VAR ProcKey ):boolean;\end{ttfamily}


\end{flushleft}
\end{list}
\paragraph*{FindKey}\hspace*{\fill}

\begin{list}{}{
\settowidth{\tmplength}{\textbf{Declaração}}
\setlength{\itemindent}{0cm}
\setlength{\listparindent}{0cm}
\setlength{\leftmargin}{\evensidemargin}
\addtolength{\leftmargin}{\tmplength}
\settowidth{\labelsep}{X}
\addtolength{\leftmargin}{\labelsep}
\setlength{\labelwidth}{\tmplength}
}
\begin{flushleft}
\item[\textbf{Declaração}\hfill]
\begin{ttfamily}
public function FindKey(var IxF : IndexFile;var DataRecNum : Longint;var ProcKey ):Boolean;\end{ttfamily}


\end{flushleft}
\end{list}
\paragraph*{FindKeyTop}\hspace*{\fill}

\begin{list}{}{
\settowidth{\tmplength}{\textbf{Declaração}}
\setlength{\itemindent}{0cm}
\setlength{\listparindent}{0cm}
\setlength{\leftmargin}{\evensidemargin}
\addtolength{\leftmargin}{\tmplength}
\settowidth{\labelsep}{X}
\addtolength{\leftmargin}{\labelsep}
\setlength{\labelwidth}{\tmplength}
}
\begin{flushleft}
\item[\textbf{Declaração}\hfill]
\begin{ttfamily}
public function FindKeyTop(var IxF : IndexFile;var DataRecNum : Longint;var ProcKey ):Boolean;\end{ttfamily}


\end{flushleft}
\end{list}
\paragraph*{SearchKey}\hspace*{\fill}

\begin{list}{}{
\settowidth{\tmplength}{\textbf{Declaração}}
\setlength{\itemindent}{0cm}
\setlength{\listparindent}{0cm}
\setlength{\leftmargin}{\evensidemargin}
\addtolength{\leftmargin}{\tmplength}
\settowidth{\labelsep}{X}
\addtolength{\leftmargin}{\labelsep}
\setlength{\labelwidth}{\tmplength}
}
\begin{flushleft}
\item[\textbf{Declaração}\hfill]
\begin{ttfamily}
public function SearchKey(var IxF : IndexFile; var DataRecNum : Longint; var ProcKey:TaKeyStr):Boolean;\end{ttfamily}


\end{flushleft}
\end{list}
\paragraph*{SearchKeyTop}\hspace*{\fill}

\begin{list}{}{
\settowidth{\tmplength}{\textbf{Declaração}}
\setlength{\itemindent}{0cm}
\setlength{\listparindent}{0cm}
\setlength{\leftmargin}{\evensidemargin}
\addtolength{\leftmargin}{\tmplength}
\settowidth{\labelsep}{X}
\addtolength{\leftmargin}{\labelsep}
\setlength{\labelwidth}{\tmplength}
}
\begin{flushleft}
\item[\textbf{Declaração}\hfill]
\begin{ttfamily}
public function SearchKeyTop(var IxF : IndexFile; var DataRecNum : Longint; var ProcKey:TaKeyStr;Const Okequal : Boolean):Boolean;\end{ttfamily}


\end{flushleft}
\end{list}
\paragraph*{TaUpdatePage}\hspace*{\fill}

\begin{list}{}{
\settowidth{\tmplength}{\textbf{Declaração}}
\setlength{\itemindent}{0cm}
\setlength{\listparindent}{0cm}
\setlength{\leftmargin}{\evensidemargin}
\addtolength{\leftmargin}{\tmplength}
\settowidth{\labelsep}{X}
\addtolength{\leftmargin}{\labelsep}
\setlength{\labelwidth}{\tmplength}
}
\begin{flushleft}
\item[\textbf{Declaração}\hfill]
\begin{ttfamily}
public procedure TaUpdatePage(VAR IxF : IndexFile; VAR R : LONGINT; VAR PgPtr : TaPagePtr; Const Transaction{\_}Current : T{\_}TTransaction);\end{ttfamily}


\end{flushleft}
\end{list}
\paragraph*{AddKey{\_}Search{\_}Insert}\hspace*{\fill}

\begin{list}{}{
\settowidth{\tmplength}{\textbf{Declaração}}
\setlength{\itemindent}{0cm}
\setlength{\listparindent}{0cm}
\setlength{\leftmargin}{\evensidemargin}
\addtolength{\leftmargin}{\tmplength}
\settowidth{\labelsep}{X}
\addtolength{\leftmargin}{\labelsep}
\setlength{\labelwidth}{\tmplength}
}
\begin{flushleft}
\item[\textbf{Declaração}\hfill]
\begin{ttfamily}
public procedure AddKey{\_}Search{\_}Insert( var IxF : IndexFile; Var PrPgRef1 : LONGINT; VAR PrPgRef2,c : LONGINT; VAR PagePtr1,PagePtr2 : TaPagePtr; VAR ProcItem1, ProcItem2 : TaItem; vAR PassUp, okAddKey : BOOLEAN; Const ProcKey : TaKeyStr; Const DataRecNum : Longint; VAR K,L : SmallInt; Var R : SmallInt );\end{ttfamily}


\end{flushleft}
\end{list}
\paragraph*{AddKey{\_}Search{\_}Init{\_}ProcItem1}\hspace*{\fill}

\begin{list}{}{
\settowidth{\tmplength}{\textbf{Declaração}}
\setlength{\itemindent}{0cm}
\setlength{\listparindent}{0cm}
\setlength{\leftmargin}{\evensidemargin}
\addtolength{\leftmargin}{\tmplength}
\settowidth{\labelsep}{X}
\addtolength{\leftmargin}{\labelsep}
\setlength{\labelwidth}{\tmplength}
}
\begin{flushleft}
\item[\textbf{Declaração}\hfill]
\begin{ttfamily}
public procedure AddKey{\_}Search{\_}Init{\_}ProcItem1(Const ProcKey : TaKeyStr; Const DataRecNum : Longint; vAR PassUp : BOOLEAN; VAR ProcItem1 : TaItem);\end{ttfamily}


\end{flushleft}
\end{list}
\paragraph*{AddKey{\_}Search}\hspace*{\fill}

\begin{list}{}{
\settowidth{\tmplength}{\textbf{Declaração}}
\setlength{\itemindent}{0cm}
\setlength{\listparindent}{0cm}
\setlength{\leftmargin}{\evensidemargin}
\addtolength{\leftmargin}{\tmplength}
\settowidth{\labelsep}{X}
\addtolength{\leftmargin}{\labelsep}
\setlength{\labelwidth}{\tmplength}
}
\begin{flushleft}
\item[\textbf{Declaração}\hfill]
\begin{ttfamily}
public procedure AddKey{\_}Search(var IxF : IndexFile; PrPgRef1 : LONGINT; VAR PrPgRef2,c : LONGINT; VAR PagePtr1,PagePtr2 : TaPagePtr; VAR ProcItem1, ProcItem2 : TaItem; vAR PassUp, okAddKey : BOOLEAN; Const ProcKey : TaKeyStr; Const DataRecNum : Longint; VAR K,L : SmallInt );\end{ttfamily}


\end{flushleft}
\end{list}
\paragraph*{AddKey}\hspace*{\fill}

\begin{list}{}{
\settowidth{\tmplength}{\textbf{Declaração}}
\setlength{\itemindent}{0cm}
\setlength{\listparindent}{0cm}
\setlength{\leftmargin}{\evensidemargin}
\addtolength{\leftmargin}{\tmplength}
\settowidth{\labelsep}{X}
\addtolength{\leftmargin}{\labelsep}
\setlength{\labelwidth}{\tmplength}
}
\begin{flushleft}
\item[\textbf{Declaração}\hfill]
\begin{ttfamily}
public function AddKey(var IxF : IndexFile; Const DataRecNum : Longint; Const ProcKey : TaKeyStr):Boolean;\end{ttfamily}


\end{flushleft}
\end{list}
\paragraph*{DeleteKey}\hspace*{\fill}

\begin{list}{}{
\settowidth{\tmplength}{\textbf{Declaração}}
\setlength{\itemindent}{0cm}
\setlength{\listparindent}{0cm}
\setlength{\leftmargin}{\evensidemargin}
\addtolength{\leftmargin}{\tmplength}
\settowidth{\labelsep}{X}
\addtolength{\leftmargin}{\labelsep}
\setlength{\labelwidth}{\tmplength}
}
\begin{flushleft}
\item[\textbf{Declaração}\hfill]
\begin{ttfamily}
public function DeleteKey(var IxF : IndexFile;Const DataRecNum : Longint;var ProcKey:TaKeyStr ):Boolean;\end{ttfamily}


\end{flushleft}
\end{list}
\paragraph*{FGetHeaderDataFile}\hspace*{\fill}

\begin{list}{}{
\settowidth{\tmplength}{\textbf{Declaração}}
\setlength{\itemindent}{0cm}
\setlength{\listparindent}{0cm}
\setlength{\leftmargin}{\evensidemargin}
\addtolength{\leftmargin}{\tmplength}
\settowidth{\labelsep}{X}
\addtolength{\leftmargin}{\labelsep}
\setlength{\labelwidth}{\tmplength}
}
\begin{flushleft}
\item[\textbf{Declaração}\hfill]
\begin{ttfamily}
public function FGetHeaderDataFile(Const FileName: PathStr;Var Header : TsImagemHeader;Var aFileSize : Longint):Boolean;\end{ttfamily}


\end{flushleft}
\end{list}
\paragraph*{FTamRegDataFile}\hspace*{\fill}

\begin{list}{}{
\settowidth{\tmplength}{\textbf{Declaração}}
\setlength{\itemindent}{0cm}
\setlength{\listparindent}{0cm}
\setlength{\leftmargin}{\evensidemargin}
\addtolength{\leftmargin}{\tmplength}
\settowidth{\labelsep}{X}
\addtolength{\leftmargin}{\labelsep}
\setlength{\labelwidth}{\tmplength}
}
\begin{flushleft}
\item[\textbf{Declaração}\hfill]
\begin{ttfamily}
public function FTamRegDataFile(Const FileName: PathStr):SmallWord;\end{ttfamily}


\end{flushleft}
\end{list}
\paragraph*{NewFileName}\hspace*{\fill}

\begin{list}{}{
\settowidth{\tmplength}{\textbf{Declaração}}
\setlength{\itemindent}{0cm}
\setlength{\listparindent}{0cm}
\setlength{\leftmargin}{\evensidemargin}
\addtolength{\leftmargin}{\tmplength}
\settowidth{\labelsep}{X}
\addtolength{\leftmargin}{\labelsep}
\setlength{\labelwidth}{\tmplength}
}
\begin{flushleft}
\item[\textbf{Declaração}\hfill]
\begin{ttfamily}
public function NewFileName(FileName,Extencao:PathStr):PathStr;\end{ttfamily}


\end{flushleft}
\end{list}
\paragraph*{FTb}\hspace*{\fill}

\begin{list}{}{
\settowidth{\tmplength}{\textbf{Declaração}}
\setlength{\itemindent}{0cm}
\setlength{\listparindent}{0cm}
\setlength{\leftmargin}{\evensidemargin}
\addtolength{\leftmargin}{\tmplength}
\settowidth{\labelsep}{X}
\addtolength{\leftmargin}{\labelsep}
\setlength{\labelwidth}{\tmplength}
}
\begin{flushleft}
\item[\textbf{Declaração}\hfill]
\begin{ttfamily}
public function FTb(Const FileName:PathStr):PathStr;\end{ttfamily}


\end{flushleft}
\end{list}
\paragraph*{FObj}\hspace*{\fill}

\begin{list}{}{
\settowidth{\tmplength}{\textbf{Declaração}}
\setlength{\itemindent}{0cm}
\setlength{\listparindent}{0cm}
\setlength{\leftmargin}{\evensidemargin}
\addtolength{\leftmargin}{\tmplength}
\settowidth{\labelsep}{X}
\addtolength{\leftmargin}{\labelsep}
\setlength{\labelwidth}{\tmplength}
}
\begin{flushleft}
\item[\textbf{Declaração}\hfill]
\begin{ttfamily}
public function FObj(Const FileName:PathStr):PathStr;\end{ttfamily}


\end{flushleft}
\end{list}
\paragraph*{FIx}\hspace*{\fill}

\begin{list}{}{
\settowidth{\tmplength}{\textbf{Declaração}}
\setlength{\itemindent}{0cm}
\setlength{\listparindent}{0cm}
\setlength{\leftmargin}{\evensidemargin}
\addtolength{\leftmargin}{\tmplength}
\settowidth{\labelsep}{X}
\addtolength{\leftmargin}{\labelsep}
\setlength{\labelwidth}{\tmplength}
}
\begin{flushleft}
\item[\textbf{Declaração}\hfill]
\begin{ttfamily}
public function FIx(Const FileName:PathStr):PathStr;\end{ttfamily}


\end{flushleft}
\end{list}
\paragraph*{FDup}\hspace*{\fill}

\begin{list}{}{
\settowidth{\tmplength}{\textbf{Declaração}}
\setlength{\itemindent}{0cm}
\setlength{\listparindent}{0cm}
\setlength{\leftmargin}{\evensidemargin}
\addtolength{\leftmargin}{\tmplength}
\settowidth{\labelsep}{X}
\addtolength{\leftmargin}{\labelsep}
\setlength{\labelwidth}{\tmplength}
}
\begin{flushleft}
\item[\textbf{Declaração}\hfill]
\begin{ttfamily}
public function FDup(Const FileName:PathStr):PathStr;\end{ttfamily}


\end{flushleft}
\end{list}
\paragraph*{AssignDataFile}\hspace*{\fill}

\begin{list}{}{
\settowidth{\tmplength}{\textbf{Declaração}}
\setlength{\itemindent}{0cm}
\setlength{\listparindent}{0cm}
\setlength{\leftmargin}{\evensidemargin}
\addtolength{\leftmargin}{\tmplength}
\settowidth{\labelsep}{X}
\addtolength{\leftmargin}{\labelsep}
\setlength{\labelwidth}{\tmplength}
}
\begin{flushleft}
\item[\textbf{Declaração}\hfill]
\begin{ttfamily}
public procedure AssignDataFile(Var DatF :DataFile; Const aFileName:PathStr; aBaseSize, aRecSize:SmallWord; Const AFileMode: Word; aF :TStream; WTipo : AnsiChar ); Overload;\end{ttfamily}


\end{flushleft}
\end{list}
\paragraph*{AssignDataFile}\hspace*{\fill}

\begin{list}{}{
\settowidth{\tmplength}{\textbf{Declaração}}
\setlength{\itemindent}{0cm}
\setlength{\listparindent}{0cm}
\setlength{\leftmargin}{\evensidemargin}
\addtolength{\leftmargin}{\tmplength}
\settowidth{\labelsep}{X}
\addtolength{\leftmargin}{\labelsep}
\setlength{\labelwidth}{\tmplength}
}
\begin{flushleft}
\item[\textbf{Declaração}\hfill]
\begin{ttfamily}
public procedure AssignDataFile(Var DatF :DataFile;Const aFileName:PathStr;aBaseSize,aRecSize:SmallWord); Overload;\end{ttfamily}


\end{flushleft}
\end{list}
\paragraph*{AssignIndexFile}\hspace*{\fill}

\begin{list}{}{
\settowidth{\tmplength}{\textbf{Declaração}}
\setlength{\itemindent}{0cm}
\setlength{\listparindent}{0cm}
\setlength{\leftmargin}{\evensidemargin}
\addtolength{\leftmargin}{\tmplength}
\settowidth{\labelsep}{X}
\addtolength{\leftmargin}{\labelsep}
\setlength{\labelwidth}{\tmplength}
}
\begin{flushleft}
\item[\textbf{Declaração}\hfill]
\begin{ttfamily}
public procedure AssignIndexFile(Var IxF : IndexFile; Const aFileName : PathStr; aBaseSize, aRecSize : SmallWord); Overload;\end{ttfamily}


\end{flushleft}
\end{list}
\paragraph*{UpperCase}\hspace*{\fill}

\begin{list}{}{
\settowidth{\tmplength}{\textbf{Declaração}}
\setlength{\itemindent}{0cm}
\setlength{\listparindent}{0cm}
\setlength{\leftmargin}{\evensidemargin}
\addtolength{\leftmargin}{\tmplength}
\settowidth{\labelsep}{X}
\addtolength{\leftmargin}{\labelsep}
\setlength{\labelwidth}{\tmplength}
}
\begin{flushleft}
\item[\textbf{Declaração}\hfill]
\begin{ttfamily}
public function UpperCase(str:AnsiString):AnsiString;\end{ttfamily}


\end{flushleft}
\end{list}
\paragraph*{FMinuscula}\hspace*{\fill}

\begin{list}{}{
\settowidth{\tmplength}{\textbf{Declaração}}
\setlength{\itemindent}{0cm}
\setlength{\listparindent}{0cm}
\setlength{\leftmargin}{\evensidemargin}
\addtolength{\leftmargin}{\tmplength}
\settowidth{\labelsep}{X}
\addtolength{\leftmargin}{\labelsep}
\setlength{\labelwidth}{\tmplength}
}
\begin{flushleft}
\item[\textbf{Declaração}\hfill]
\begin{ttfamily}
public function FMinuscula(str:AnsiString):AnsiString;\end{ttfamily}


\end{flushleft}
\end{list}
\paragraph*{Int2str}\hspace*{\fill}

\begin{list}{}{
\settowidth{\tmplength}{\textbf{Declaração}}
\setlength{\itemindent}{0cm}
\setlength{\listparindent}{0cm}
\setlength{\leftmargin}{\evensidemargin}
\addtolength{\leftmargin}{\tmplength}
\settowidth{\labelsep}{X}
\addtolength{\leftmargin}{\labelsep}
\setlength{\labelwidth}{\tmplength}
}
\begin{flushleft}
\item[\textbf{Declaração}\hfill]
\begin{ttfamily}
public function Int2str(Const L : LongInt) : tString;\end{ttfamily}


\end{flushleft}
\end{list}
\paragraph*{spc}\hspace*{\fill}

\begin{list}{}{
\settowidth{\tmplength}{\textbf{Declaração}}
\setlength{\itemindent}{0cm}
\setlength{\listparindent}{0cm}
\setlength{\leftmargin}{\evensidemargin}
\addtolength{\leftmargin}{\tmplength}
\settowidth{\labelsep}{X}
\addtolength{\leftmargin}{\labelsep}
\setlength{\labelwidth}{\tmplength}
}
\begin{flushleft}
\item[\textbf{Declaração}\hfill]
\begin{ttfamily}
public function spc(Const campo:AnsiString;Const tam :Longint):AnsiString;\end{ttfamily}


\end{flushleft}
\end{list}
\paragraph*{SetOkTransaction}\hspace*{\fill}

\begin{list}{}{
\settowidth{\tmplength}{\textbf{Declaração}}
\setlength{\itemindent}{0cm}
\setlength{\listparindent}{0cm}
\setlength{\leftmargin}{\evensidemargin}
\addtolength{\leftmargin}{\tmplength}
\settowidth{\labelsep}{X}
\addtolength{\leftmargin}{\labelsep}
\setlength{\labelwidth}{\tmplength}
}
\begin{flushleft}
\item[\textbf{Declaração}\hfill]
\begin{ttfamily}
public function SetOkTransaction(Const aOkTransaction : BOOLEAN):BOOLEAN;\end{ttfamily}


\end{flushleft}
\end{list}
\paragraph*{StartTransaction}\hspace*{\fill}

\begin{list}{}{
\settowidth{\tmplength}{\textbf{Declaração}}
\setlength{\itemindent}{0cm}
\setlength{\listparindent}{0cm}
\setlength{\leftmargin}{\evensidemargin}
\addtolength{\leftmargin}{\tmplength}
\settowidth{\labelsep}{X}
\addtolength{\leftmargin}{\labelsep}
\setlength{\labelwidth}{\tmplength}
}
\begin{flushleft}
\item[\textbf{Declaração}\hfill]
\begin{ttfamily}
public function StartTransaction(Const aDelta : SmallWord):Integer; Overload;\end{ttfamily}


\end{flushleft}
\end{list}
\paragraph*{StartTransaction}\hspace*{\fill}

\begin{list}{}{
\settowidth{\tmplength}{\textbf{Declaração}}
\setlength{\itemindent}{0cm}
\setlength{\listparindent}{0cm}
\setlength{\leftmargin}{\evensidemargin}
\addtolength{\leftmargin}{\tmplength}
\settowidth{\labelsep}{X}
\addtolength{\leftmargin}{\labelsep}
\setlength{\labelwidth}{\tmplength}
}
\begin{flushleft}
\item[\textbf{Declaração}\hfill]
\begin{ttfamily}
public function StartTransaction(Const DatF : DataFile ; Var aok{\_}Set{\_}Transaction : Boolean): Integer; Overload;\end{ttfamily}


\end{flushleft}
\end{list}
\paragraph*{COMMIT}\hspace*{\fill}

\begin{list}{}{
\settowidth{\tmplength}{\textbf{Declaração}}
\setlength{\itemindent}{0cm}
\setlength{\listparindent}{0cm}
\setlength{\leftmargin}{\evensidemargin}
\addtolength{\leftmargin}{\tmplength}
\settowidth{\labelsep}{X}
\addtolength{\leftmargin}{\labelsep}
\setlength{\labelwidth}{\tmplength}
}
\begin{flushleft}
\item[\textbf{Declaração}\hfill]
\begin{ttfamily}
public function COMMIT:Boolean; Overload;\end{ttfamily}


\end{flushleft}
\end{list}
\paragraph*{COMMIT}\hspace*{\fill}

\begin{list}{}{
\settowidth{\tmplength}{\textbf{Declaração}}
\setlength{\itemindent}{0cm}
\setlength{\listparindent}{0cm}
\setlength{\leftmargin}{\evensidemargin}
\addtolength{\leftmargin}{\tmplength}
\settowidth{\labelsep}{X}
\addtolength{\leftmargin}{\labelsep}
\setlength{\labelwidth}{\tmplength}
}
\begin{flushleft}
\item[\textbf{Declaração}\hfill]
\begin{ttfamily}
public function COMMIT(Const Wok{\_}Set{\_}Transaction : Boolean):Boolean; Overload;\end{ttfamily}


\end{flushleft}
\end{list}
\paragraph*{Rollback}\hspace*{\fill}

\begin{list}{}{
\settowidth{\tmplength}{\textbf{Declaração}}
\setlength{\itemindent}{0cm}
\setlength{\listparindent}{0cm}
\setlength{\leftmargin}{\evensidemargin}
\addtolength{\leftmargin}{\tmplength}
\settowidth{\labelsep}{X}
\addtolength{\leftmargin}{\labelsep}
\setlength{\labelwidth}{\tmplength}
}
\begin{flushleft}
\item[\textbf{Declaração}\hfill]
\begin{ttfamily}
public procedure Rollback;\end{ttfamily}


\end{flushleft}
\end{list}
\paragraph*{SetTransaction}\hspace*{\fill}

\begin{list}{}{
\settowidth{\tmplength}{\textbf{Declaração}}
\setlength{\itemindent}{0cm}
\setlength{\listparindent}{0cm}
\setlength{\leftmargin}{\evensidemargin}
\addtolength{\leftmargin}{\tmplength}
\settowidth{\labelsep}{X}
\addtolength{\leftmargin}{\labelsep}
\setlength{\labelwidth}{\tmplength}
}
\begin{flushleft}
\item[\textbf{Declaração}\hfill]
\begin{ttfamily}
public function SetTransaction(const OnOff:Boolean; Var WOK : Boolean ):Boolean; Overload;\end{ttfamily}


\end{flushleft}
\end{list}
\paragraph*{SetTransaction}\hspace*{\fill}

\begin{list}{}{
\settowidth{\tmplength}{\textbf{Declaração}}
\setlength{\itemindent}{0cm}
\setlength{\listparindent}{0cm}
\setlength{\leftmargin}{\evensidemargin}
\addtolength{\leftmargin}{\tmplength}
\settowidth{\labelsep}{X}
\addtolength{\leftmargin}{\labelsep}
\setlength{\labelwidth}{\tmplength}
}
\begin{flushleft}
\item[\textbf{Declaração}\hfill]
\begin{ttfamily}
public function SetTransaction(const OnOff:Boolean; Var WOK, Wok{\_}Set{\_}Transaction:Boolean):Boolean; Overload;\end{ttfamily}


\end{flushleft}
\end{list}
\paragraph*{GetFileName{\_}Transaction}\hspace*{\fill}

\begin{list}{}{
\settowidth{\tmplength}{\textbf{Declaração}}
\setlength{\itemindent}{0cm}
\setlength{\listparindent}{0cm}
\setlength{\leftmargin}{\evensidemargin}
\addtolength{\leftmargin}{\tmplength}
\settowidth{\labelsep}{X}
\addtolength{\leftmargin}{\labelsep}
\setlength{\labelwidth}{\tmplength}
}
\begin{flushleft}
\item[\textbf{Declaração}\hfill]
\begin{ttfamily}
public function GetFileName{\_}Transaction(): tString;\end{ttfamily}


\end{flushleft}
\end{list}
\paragraph*{Assign{\_}Transaction}\hspace*{\fill}

\begin{list}{}{
\settowidth{\tmplength}{\textbf{Declaração}}
\setlength{\itemindent}{0cm}
\setlength{\listparindent}{0cm}
\setlength{\leftmargin}{\evensidemargin}
\addtolength{\leftmargin}{\tmplength}
\settowidth{\labelsep}{X}
\addtolength{\leftmargin}{\labelsep}
\setlength{\labelwidth}{\tmplength}
}
\begin{flushleft}
\item[\textbf{Declaração}\hfill]
\begin{ttfamily}
public function Assign{\_}Transaction(Const aFileName : PathStr):SmallWord;\end{ttfamily}


\end{flushleft}
\end{list}
\paragraph*{TransactionPendant{\_}Error}\hspace*{\fill}

\begin{list}{}{
\settowidth{\tmplength}{\textbf{Declaração}}
\setlength{\itemindent}{0cm}
\setlength{\listparindent}{0cm}
\setlength{\leftmargin}{\evensidemargin}
\addtolength{\leftmargin}{\tmplength}
\settowidth{\labelsep}{X}
\addtolength{\leftmargin}{\labelsep}
\setlength{\labelwidth}{\tmplength}
}
\begin{flushleft}
\item[\textbf{Declaração}\hfill]
\begin{ttfamily}
public function TransactionPendant{\_}Error:Boolean;\end{ttfamily}


\end{flushleft}
\end{list}
\paragraph*{TransactionPendant}\hspace*{\fill}

\begin{list}{}{
\settowidth{\tmplength}{\textbf{Declaração}}
\setlength{\itemindent}{0cm}
\setlength{\listparindent}{0cm}
\setlength{\leftmargin}{\evensidemargin}
\addtolength{\leftmargin}{\tmplength}
\settowidth{\labelsep}{X}
\addtolength{\leftmargin}{\labelsep}
\setlength{\labelwidth}{\tmplength}
}
\begin{flushleft}
\item[\textbf{Declaração}\hfill]
\begin{ttfamily}
public function TransactionPendant:Boolean;\end{ttfamily}


\end{flushleft}
\end{list}
\paragraph*{Truncate}\hspace*{\fill}

\begin{list}{}{
\settowidth{\tmplength}{\textbf{Declaração}}
\setlength{\itemindent}{0cm}
\setlength{\listparindent}{0cm}
\setlength{\leftmargin}{\evensidemargin}
\addtolength{\leftmargin}{\tmplength}
\settowidth{\labelsep}{X}
\addtolength{\leftmargin}{\labelsep}
\setlength{\labelwidth}{\tmplength}
}
\begin{flushleft}
\item[\textbf{Declaração}\hfill]
\begin{ttfamily}
public procedure Truncate(Var DatF: DataFile;NR : LongInt);\end{ttfamily}


\end{flushleft}
\end{list}
\paragraph*{CopyFrom}\hspace*{\fill}

\begin{list}{}{
\settowidth{\tmplength}{\textbf{Declaração}}
\setlength{\itemindent}{0cm}
\setlength{\listparindent}{0cm}
\setlength{\leftmargin}{\evensidemargin}
\addtolength{\leftmargin}{\tmplength}
\settowidth{\labelsep}{X}
\addtolength{\leftmargin}{\labelsep}
\setlength{\labelwidth}{\tmplength}
}
\begin{flushleft}
\item[\textbf{Declaração}\hfill]
\begin{ttfamily}
public procedure CopyFrom(Font{\_}DatF: DataFile ;Var Dest{\_}DatF: DataFile); Overload;\end{ttfamily}


\end{flushleft}
\end{list}
\paragraph*{CopyFrom}\hspace*{\fill}

\begin{list}{}{
\settowidth{\tmplength}{\textbf{Declaração}}
\setlength{\itemindent}{0cm}
\setlength{\listparindent}{0cm}
\setlength{\leftmargin}{\evensidemargin}
\addtolength{\leftmargin}{\tmplength}
\settowidth{\labelsep}{X}
\addtolength{\leftmargin}{\labelsep}
\setlength{\labelwidth}{\tmplength}
}
\begin{flushleft}
\item[\textbf{Declaração}\hfill]
\begin{ttfamily}
public procedure CopyFrom(Font{\_}IxF : IndexFile ;Var Dest{\_}IxF : IndexFile); Overload;\end{ttfamily}


\end{flushleft}
\end{list}
\paragraph*{Is{\_}TFileOpen}\hspace*{\fill}

\begin{list}{}{
\settowidth{\tmplength}{\textbf{Declaração}}
\setlength{\itemindent}{0cm}
\setlength{\listparindent}{0cm}
\setlength{\leftmargin}{\evensidemargin}
\addtolength{\leftmargin}{\tmplength}
\settowidth{\labelsep}{X}
\addtolength{\leftmargin}{\labelsep}
\setlength{\labelwidth}{\tmplength}
}
\begin{flushleft}
\item[\textbf{Declaração}\hfill]
\begin{ttfamily}
public Function Is{\_}TFileOpen(const a{\_}TFile : TStream):Boolean;\end{ttfamily}


\end{flushleft}
\end{list}
\chapter{Unit mi.rtl.Objects.Methods.Exception}
\section{Descrição}
{-}A unit \textbf{\begin{ttfamily}mi.rtl.Objects.Methods.Exception\end{ttfamily}} implementa a classe \begin{ttfamily}TException\end{ttfamily}(\ref{mi.rtl.Objects.Methods.Exception.TException}) do pacote \begin{ttfamily}mi.rtl\end{ttfamily}(\ref{mi.rtl}).

\begin{itemize}
\item \textbf{VERSÃO}: \begin{itemize}
\item Alpha {-} 0.5.0.687
\end{itemize}
\item \textbf{CÓDIGO FONTE}: \begin{itemize}
\item 
\end{itemize}
\item \textbf{HISTÓRICO} \begin{itemize}
\item Criado por: Paulo Sérgio da Silva Pacheco e{-}mail: paulosspacheco@yahoo.com.br \begin{itemize}
\item 2021{-}12{-}14 \begin{itemize}
\item 11:00 a 11:30 {-} Criado a unit \textbf{\begin{ttfamily}mi.rtl.Objects.Methods.Exception\end{ttfamily}} e implementação da classe \textbf{\begin{ttfamily}TException\end{ttfamily}(\ref{mi.rtl.Objects.Methods.Exception.TException})}
\end{itemize}
\item 2021{-}12{-}15 \begin{itemize}
\item 15:00 a 18:42 {-} T12 Criar a classe \begin{ttfamily}TException\end{ttfamily}(\ref{mi.rtl.Objects.Methods.Exception.TException})
\item 21:25 a 22:40 {-} Troquei o nome de constructor create para que fique equivalente as mensagem de \begin{ttfamily}TStrError.ErrorMessage\end{ttfamily}(\ref{mi.rtl.Consts.StrError.TStrError-ErrorMessage}).
\end{itemize}
\end{itemize}
\end{itemize}
\end{itemize}
\section{Uses}
\begin{itemize}
\item \begin{ttfamily}Classes\end{ttfamily}\item \begin{ttfamily}SysUtils\end{ttfamily}\item \begin{ttfamily}mi.rtl.Consts.StrError\end{ttfamily}(\ref{mi.rtl.Consts.StrError})\item \begin{ttfamily}mi.rtl.objects.Methods\end{ttfamily}(\ref{mi.rtl.Objects.Methods})\end{itemize}
\section{Visão Geral}
\begin{description}
\item[\texttt{\begin{ttfamily}TException\end{ttfamily} Classe}]
\end{description}
\section{Classes, Interfaces, Objetos e Registros}
\subsection*{TException Classe}
\subsubsection*{\large{\textbf{Hierarquia}}\normalsize\hspace{1ex}\hfill}
TException {$>$} \begin{ttfamily}TObjectsMethods\end{ttfamily}(\ref{mi.rtl.Objects.Methods.TObjectsMethods}) {$>$} \begin{ttfamily}TObjectsConsts\end{ttfamily}(\ref{mi.rtl.Objects.Consts.TObjectsConsts}) {$>$} 
TObjectsTypes
\subsubsection*{\large{\textbf{Descrição}}\normalsize\hspace{1ex}\hfill}
\begin{itemize}
\item A classe \textbf{\begin{ttfamily}TException\end{ttfamily}} é usada com a palavra reservada \textbf{raise} para mostrar o erro, sua localização e em seguida salva no dispositivo definido em \textbf{TObjectss.Logs.LogType}.

\begin{itemize}
\item \textbf{NOTA} \begin{itemize}
\item LogType = TLogType = (ltSystem,ltFile,ltStdOut,ltStdErr); \begin{itemize}
\item ltSystem = Arquivo definido pelo sistema operacional;
\item ltFile = Arquivo definido pela aplicação;
\item ltStdOut,ltStdErr = Terminal do sistema operacional.
\end{itemize}
\end{itemize}
\item \textbf{EXEMPLO DE USO}:

\texttt{\\\nopagebreak[3]
\\\nopagebreak[3]
}\textbf{procedure}\texttt{~TMi{\_}Rtl{\_}Tests.Action{\_}test{\_}TExceptionExecute(Sender:~TObject);\\\nopagebreak[3]
}\textbf{begin}\texttt{\\\nopagebreak[3]
~~}\textbf{with}\texttt{~TMI{\_}ui{\_}types~}\textbf{do}\texttt{~}\textbf{begin}\texttt{\\\nopagebreak[3]
~~~~logs.EnableWriteIdentificao~:=~true;\\\nopagebreak[3]
~~~~}\textbf{try}\texttt{\\\nopagebreak[3]
~~~~~~}\textbf{raise}\texttt{~TException.Create(5);\\\nopagebreak[3]
~~~~}\textbf{except}\texttt{\\\nopagebreak[3]
~~~~}\textbf{end}\texttt{;\\\nopagebreak[3]
\\\nopagebreak[3]
~~~~}\textbf{try}\texttt{\\\nopagebreak[3]
~~~~~~}\textbf{raise}\texttt{~TException.Create('Acesso~ao~arquivo~negado');\\\nopagebreak[3]
~~~~}\textbf{except}\texttt{\\\nopagebreak[3]
~~~~}\textbf{end}\texttt{;\\\nopagebreak[3]
\\\nopagebreak[3]
~~~~}\textbf{try}\texttt{\\\nopagebreak[3]
~~~~~~}\textbf{raise}\texttt{~TException.Create(Self,~'Action{\_}test{\_}TExceptionExecute','aFileName','AFieldName',5);\\\nopagebreak[3]
~~~~}\textbf{except}\texttt{\\\nopagebreak[3]
~~~~}\textbf{end}\texttt{;\\\nopagebreak[3]
\\\nopagebreak[3]
~~~~}\textbf{try}\texttt{\\\nopagebreak[3]
~~~~~~}\textbf{raise}\texttt{~TException.Create(Self,~'Action{\_}test{\_}TExceptionExecute','aFileName','AFieldName','Acesso~ao~arquivo~negado');\\\nopagebreak[3]
~~~~}\textbf{except}\texttt{\\\nopagebreak[3]
~~~~}\textbf{end}\texttt{;\\\nopagebreak[3]
\\\nopagebreak[3]
\\\nopagebreak[3]
~~~~}\textbf{try}\texttt{\\\nopagebreak[3]
~~~~~~}\textbf{raise}\texttt{~TException.Create(Self,~'Action{\_}test{\_}TExceptionExecute',5);\\\nopagebreak[3]
~~~~}\textbf{except}\texttt{\\\nopagebreak[3]
~~~~}\textbf{end}\texttt{;\\\nopagebreak[3]
\\\nopagebreak[3]
~~~~}\textbf{try}\texttt{\\\nopagebreak[3]
~~~~~~}\textbf{raise}\texttt{~TException.Create(Self,~'Action{\_}test{\_}TExceptionExecute','Acesso~ao~arquivo~negado');\\\nopagebreak[3]
~~~~}\textbf{except}\texttt{\\\nopagebreak[3]
~~~~}\textbf{end}\texttt{;\\\nopagebreak[3]
\\\nopagebreak[3]
\\\nopagebreak[3]
\textit{//~Os~exemplos~abaixo~são~mantidos~para~manter~a~compatibilidade~com~o~passado.}\\\nopagebreak[3]
\\\nopagebreak[3]
\\\nopagebreak[3]
~~~~~}\textbf{try}\texttt{\\\nopagebreak[3]
~~~~~~~}\textbf{raise}\texttt{~TException.Create4('aModule',~'aUnit',~'Procedure{\_}or{\_}Function',~~~'ParamResult');\\\nopagebreak[3]
~~~~~}\textbf{except}\texttt{\\\nopagebreak[3]
~~~~~}\textbf{end}\texttt{;\\\nopagebreak[3]
\\\nopagebreak[3]
~~~~~}\textbf{try}\texttt{\\\nopagebreak[3]
~~~~~~~}\textbf{raise}\texttt{~TException.Create4('aModule',~'aUnit',~'Procedure{\_}or{\_}Function',~~~5);\\\nopagebreak[3]
~~~~~}\textbf{except}\texttt{\\\nopagebreak[3]
~~~~~}\textbf{end}\texttt{;\\\nopagebreak[3]
\\\nopagebreak[3]
~~~~~}\textbf{try}\texttt{\\\nopagebreak[3]
~~~~~~~}\textbf{raise}\texttt{~TException.Create5('aModule',~'aUnit','ObjectName',~'aMethodName',~~~'aMsgError');\\\nopagebreak[3]
~~~~~}\textbf{except}\texttt{\\\nopagebreak[3]
~~~~~}\textbf{end}\texttt{;\\\nopagebreak[3]
\\\nopagebreak[3]
~~~~~}\textbf{try}\texttt{\\\nopagebreak[3]
~~~~~~~}\textbf{raise}\texttt{~TException.Create5('aModule',~'aUnit','ObjectName',~'aMethodName',~~~5);\\\nopagebreak[3]
~~~~~}\textbf{except}\texttt{\\\nopagebreak[3]
~~~~~}\textbf{end}\texttt{;\\\nopagebreak[3]
\\\nopagebreak[3]
~~~~~}\textbf{try}\texttt{\\\nopagebreak[3]
~~~~~~~}\textbf{raise}\texttt{~TException.Create6('aModule',~'ObjectName',~'aMethodName','aFileName','AFieldName',~5);\\\nopagebreak[3]
~~~~~}\textbf{except}\texttt{\\\nopagebreak[3]
~~~~~}\textbf{end}\texttt{;\\\nopagebreak[3]
\\\nopagebreak[3]
~~~~~}\textbf{try}\texttt{\\\nopagebreak[3]
~~~~~~~}\textbf{raise}\texttt{~TException.Create7('aModule',~'aUnit','ObjectName',~'aMethodName','aFileName','AFieldName',~~5);\\\nopagebreak[3]
~~~~~}\textbf{except}\texttt{\\\nopagebreak[3]
~~~~~}\textbf{end}\texttt{;\\\nopagebreak[3]
\\\nopagebreak[3]
~~~~~}\textbf{try}\texttt{\\\nopagebreak[3]
~~~~~~~}\textbf{raise}\texttt{~TException.Create7('aModule',~'aUnit','ObjectName',~'aMethodName','aFileName','AFieldName',~~'ParamResult');\\\nopagebreak[3]
~~~~~}\textbf{except}\texttt{\\\nopagebreak[3]
~~~~~}\textbf{end}\texttt{;\\\nopagebreak[3]
\\\nopagebreak[3]
~~~~~}\textbf{try}\texttt{\\\nopagebreak[3]
~~~~~~~}\textbf{raise}\texttt{~TException.Create8('aModule',~'aUnit','ObjectName',~'aMethodName','aFileName','AFieldName',~~'aMessage','aProcedure{\_}or{\_}Function');\\\nopagebreak[3]
~~~~~}\textbf{except}\texttt{\\\nopagebreak[3]
~~~~~}\textbf{end}\texttt{;\\\nopagebreak[3]
~~~}\textbf{end}\texttt{;\\\nopagebreak[3]
}\textbf{end}\texttt{;\\
}
\end{itemize}
\end{itemize}\subsubsection*{\large{\textbf{Propriedades}}\normalsize\hspace{1ex}\hfill}
\paragraph*{Message}\hspace*{\fill}

\begin{list}{}{
\settowidth{\tmplength}{\textbf{Declaração}}
\setlength{\itemindent}{0cm}
\setlength{\listparindent}{0cm}
\setlength{\leftmargin}{\evensidemargin}
\addtolength{\leftmargin}{\tmplength}
\settowidth{\labelsep}{X}
\addtolength{\leftmargin}{\labelsep}
\setlength{\labelwidth}{\tmplength}
}
\begin{flushleft}
\item[\textbf{Declaração}\hfill]
\begin{ttfamily}
public property Message: Ansistring read FMessage write FMessage;\end{ttfamily}


\end{flushleft}
\end{list}
\subsubsection*{\large{\textbf{Métodos}}\normalsize\hspace{1ex}\hfill}
\paragraph*{Create}\hspace*{\fill}

\begin{list}{}{
\settowidth{\tmplength}{\textbf{Declaração}}
\setlength{\itemindent}{0cm}
\setlength{\listparindent}{0cm}
\setlength{\leftmargin}{\evensidemargin}
\addtolength{\leftmargin}{\tmplength}
\settowidth{\labelsep}{X}
\addtolength{\leftmargin}{\labelsep}
\setlength{\labelwidth}{\tmplength}
}
\begin{flushleft}
\item[\textbf{Declaração}\hfill]
\begin{ttfamily}
public constructor Create(const Msg: Ansistring); Overload;\end{ttfamily}


\end{flushleft}
\end{list}
\paragraph*{Create}\hspace*{\fill}

\begin{list}{}{
\settowidth{\tmplength}{\textbf{Declaração}}
\setlength{\itemindent}{0cm}
\setlength{\listparindent}{0cm}
\setlength{\leftmargin}{\evensidemargin}
\addtolength{\leftmargin}{\tmplength}
\settowidth{\labelsep}{X}
\addtolength{\leftmargin}{\labelsep}
\setlength{\labelwidth}{\tmplength}
}
\begin{flushleft}
\item[\textbf{Declaração}\hfill]
\begin{ttfamily}
public constructor Create(const aCodError:SmallInt); Overload;\end{ttfamily}


\end{flushleft}
\end{list}
\paragraph*{Create}\hspace*{\fill}

\begin{list}{}{
\settowidth{\tmplength}{\textbf{Declaração}}
\setlength{\itemindent}{0cm}
\setlength{\listparindent}{0cm}
\setlength{\leftmargin}{\evensidemargin}
\addtolength{\leftmargin}{\tmplength}
\settowidth{\labelsep}{X}
\addtolength{\leftmargin}{\labelsep}
\setlength{\labelwidth}{\tmplength}
}
\begin{flushleft}
\item[\textbf{Declaração}\hfill]
\begin{ttfamily}
public constructor Create(const Sender: TObject;Const aMethodName, aFileName, AFieldName:AnsiString;aCodError:integer ); Overload;\end{ttfamily}


\end{flushleft}
\end{list}
\paragraph*{Create}\hspace*{\fill}

\begin{list}{}{
\settowidth{\tmplength}{\textbf{Declaração}}
\setlength{\itemindent}{0cm}
\setlength{\listparindent}{0cm}
\setlength{\leftmargin}{\evensidemargin}
\addtolength{\leftmargin}{\tmplength}
\settowidth{\labelsep}{X}
\addtolength{\leftmargin}{\labelsep}
\setlength{\labelwidth}{\tmplength}
}
\begin{flushleft}
\item[\textbf{Declaração}\hfill]
\begin{ttfamily}
public constructor Create(const Sender: TObject;Const aMethodName, aFileName, AFieldName:AnsiString;aMsg:AnsiString); Overload;\end{ttfamily}


\end{flushleft}
\end{list}
\paragraph*{Create}\hspace*{\fill}

\begin{list}{}{
\settowidth{\tmplength}{\textbf{Declaração}}
\setlength{\itemindent}{0cm}
\setlength{\listparindent}{0cm}
\setlength{\leftmargin}{\evensidemargin}
\addtolength{\leftmargin}{\tmplength}
\settowidth{\labelsep}{X}
\addtolength{\leftmargin}{\labelsep}
\setlength{\labelwidth}{\tmplength}
}
\begin{flushleft}
\item[\textbf{Declaração}\hfill]
\begin{ttfamily}
public constructor Create(const Sender: TObject;Const aMethodName:AnsiString;aCodError:SmallInt ); Overload;\end{ttfamily}


\end{flushleft}
\end{list}
\paragraph*{Create}\hspace*{\fill}

\begin{list}{}{
\settowidth{\tmplength}{\textbf{Declaração}}
\setlength{\itemindent}{0cm}
\setlength{\listparindent}{0cm}
\setlength{\leftmargin}{\evensidemargin}
\addtolength{\leftmargin}{\tmplength}
\settowidth{\labelsep}{X}
\addtolength{\leftmargin}{\labelsep}
\setlength{\labelwidth}{\tmplength}
}
\begin{flushleft}
\item[\textbf{Declaração}\hfill]
\begin{ttfamily}
public constructor Create(const Sender: TObject;Const aMethodName:AnsiString;aMsg:AnsiString ); Overload;\end{ttfamily}


\end{flushleft}
\end{list}
\paragraph*{Create4}\hspace*{\fill}

\begin{list}{}{
\settowidth{\tmplength}{\textbf{Declaração}}
\setlength{\itemindent}{0cm}
\setlength{\listparindent}{0cm}
\setlength{\leftmargin}{\evensidemargin}
\addtolength{\leftmargin}{\tmplength}
\settowidth{\labelsep}{X}
\addtolength{\leftmargin}{\labelsep}
\setlength{\labelwidth}{\tmplength}
}
\begin{flushleft}
\item[\textbf{Declaração}\hfill]
\begin{ttfamily}
public constructor Create4(Const aModule, aUnit, Procedure{\_}or{\_}Function, aMessage:AnsiString); Overload;\end{ttfamily}


\end{flushleft}
\end{list}
\paragraph*{Create4}\hspace*{\fill}

\begin{list}{}{
\settowidth{\tmplength}{\textbf{Declaração}}
\setlength{\itemindent}{0cm}
\setlength{\listparindent}{0cm}
\setlength{\leftmargin}{\evensidemargin}
\addtolength{\leftmargin}{\tmplength}
\settowidth{\labelsep}{X}
\addtolength{\leftmargin}{\labelsep}
\setlength{\labelwidth}{\tmplength}
}
\begin{flushleft}
\item[\textbf{Declaração}\hfill]
\begin{ttfamily}
public constructor Create4(Const aModule, aUnit, Procedure{\_}or{\_}Function:AnsiString; aCodError:SmallInt); Overload;\end{ttfamily}


\end{flushleft}
\end{list}
\paragraph*{Create5}\hspace*{\fill}

\begin{list}{}{
\settowidth{\tmplength}{\textbf{Declaração}}
\setlength{\itemindent}{0cm}
\setlength{\listparindent}{0cm}
\setlength{\leftmargin}{\evensidemargin}
\addtolength{\leftmargin}{\tmplength}
\settowidth{\labelsep}{X}
\addtolength{\leftmargin}{\labelsep}
\setlength{\labelwidth}{\tmplength}
}
\begin{flushleft}
\item[\textbf{Declaração}\hfill]
\begin{ttfamily}
public constructor Create5(aModule, aUnit, aObjectName, aMethodName :AnsiString; aCodError:SmallInt); Overload;\end{ttfamily}


\end{flushleft}
\end{list}
\paragraph*{Create5}\hspace*{\fill}

\begin{list}{}{
\settowidth{\tmplength}{\textbf{Declaração}}
\setlength{\itemindent}{0cm}
\setlength{\listparindent}{0cm}
\setlength{\leftmargin}{\evensidemargin}
\addtolength{\leftmargin}{\tmplength}
\settowidth{\labelsep}{X}
\addtolength{\leftmargin}{\labelsep}
\setlength{\labelwidth}{\tmplength}
}
\begin{flushleft}
\item[\textbf{Declaração}\hfill]
\begin{ttfamily}
public constructor Create5(aModule, aUnit, aObjectName, aMethodName :AnsiString; aMsgError:AnsiString); Overload;\end{ttfamily}


\end{flushleft}
\end{list}
\paragraph*{Create6}\hspace*{\fill}

\begin{list}{}{
\settowidth{\tmplength}{\textbf{Declaração}}
\setlength{\itemindent}{0cm}
\setlength{\listparindent}{0cm}
\setlength{\leftmargin}{\evensidemargin}
\addtolength{\leftmargin}{\tmplength}
\settowidth{\labelsep}{X}
\addtolength{\leftmargin}{\labelsep}
\setlength{\labelwidth}{\tmplength}
}
\begin{flushleft}
\item[\textbf{Declaração}\hfill]
\begin{ttfamily}
public constructor Create6(aModule, aObjectName, aMethodName, aFileName, AFieldName:AnsiString; aCodError:SmallInt); Overload;\end{ttfamily}


\end{flushleft}
\end{list}
\paragraph*{Create7}\hspace*{\fill}

\begin{list}{}{
\settowidth{\tmplength}{\textbf{Declaração}}
\setlength{\itemindent}{0cm}
\setlength{\listparindent}{0cm}
\setlength{\leftmargin}{\evensidemargin}
\addtolength{\leftmargin}{\tmplength}
\settowidth{\labelsep}{X}
\addtolength{\leftmargin}{\labelsep}
\setlength{\labelwidth}{\tmplength}
}
\begin{flushleft}
\item[\textbf{Declaração}\hfill]
\begin{ttfamily}
public constructor Create7(aModule, aUnit, aObjectName, aMethodName, aFileName, AFieldName:AnsiString; aCodError:SmallInt); Overload;\end{ttfamily}


\end{flushleft}
\end{list}
\paragraph*{Create7}\hspace*{\fill}

\begin{list}{}{
\settowidth{\tmplength}{\textbf{Declaração}}
\setlength{\itemindent}{0cm}
\setlength{\listparindent}{0cm}
\setlength{\leftmargin}{\evensidemargin}
\addtolength{\leftmargin}{\tmplength}
\settowidth{\labelsep}{X}
\addtolength{\leftmargin}{\labelsep}
\setlength{\labelwidth}{\tmplength}
}
\begin{flushleft}
\item[\textbf{Declaração}\hfill]
\begin{ttfamily}
public constructor Create7(aModule, aUnit, aObjectName, aMethodName, aFileName, AFieldName:AnsiString; aMessage:AnsiString); Overload;\end{ttfamily}


\end{flushleft}
\end{list}
\paragraph*{Create8}\hspace*{\fill}

\begin{list}{}{
\settowidth{\tmplength}{\textbf{Declaração}}
\setlength{\itemindent}{0cm}
\setlength{\listparindent}{0cm}
\setlength{\leftmargin}{\evensidemargin}
\addtolength{\leftmargin}{\tmplength}
\settowidth{\labelsep}{X}
\addtolength{\leftmargin}{\labelsep}
\setlength{\labelwidth}{\tmplength}
}
\begin{flushleft}
\item[\textbf{Declaração}\hfill]
\begin{ttfamily}
public constructor Create8(aModule, aUnit, aObjectName, aMethodName, aFileName, AFieldName, aMessage, aProcedure{\_}or{\_}Function :AnsiString); Overload;\end{ttfamily}


\end{flushleft}
\end{list}
\chapter{Unit mi.rtl.Objects.Methods.Paramexecucao}
\section{Uses}
\begin{itemize}
\item \begin{ttfamily}Classes\end{ttfamily}\item \begin{ttfamily}SysUtils\end{ttfamily}\item \begin{ttfamily}dos\end{ttfamily}\item \begin{ttfamily}mi.rtl.objects.Methods\end{ttfamily}(\ref{mi.rtl.Objects.Methods})\item \begin{ttfamily}mi.rtl.objects.Methods.Exception\end{ttfamily}(\ref{mi.rtl.Objects.Methods.Exception})\item \begin{ttfamily}mi.rtl.objects.Methods.dates\end{ttfamily}(\ref{mi.rtl.objects.Methods.dates})\end{itemize}
\section{Visão Geral}
\begin{description}
\item[\texttt{\begin{ttfamily}TR{\_}ParamExecucao{\_}Local\end{ttfamily} Registro}]
\item[\texttt{\begin{ttfamily}TParamExecucao{\_}types\end{ttfamily} Classe}]
\item[\texttt{\begin{ttfamily}TParamExecucao{\_}consts\end{ttfamily} Classe}]
\item[\texttt{\begin{ttfamily}TParamExecucao\end{ttfamily} Classe}]
\end{description}
\section{Classes, Interfaces, Objetos e Registros}
\subsection*{TR{\_}ParamExecucao{\_}Local Registro}
%%%%Descrição
\subsubsection*{\large{\textbf{Campos}}\normalsize\hspace{1ex}\hfill}
\paragraph*{{\_}ParamExecucao}\hspace*{\fill}

\begin{list}{}{
\settowidth{\tmplength}{\textbf{Declaração}}
\setlength{\itemindent}{0cm}
\setlength{\listparindent}{0cm}
\setlength{\leftmargin}{\evensidemargin}
\addtolength{\leftmargin}{\tmplength}
\settowidth{\labelsep}{X}
\addtolength{\leftmargin}{\labelsep}
\setlength{\labelwidth}{\tmplength}
}
\begin{flushleft}
\item[\textbf{Declaração}\hfill]
\begin{ttfamily}
public {\_}ParamExecucao: TParamExecucao;\end{ttfamily}


\end{flushleft}
\end{list}
\paragraph*{{\_}Destoy{\_}ParamExecucao}\hspace*{\fill}

\begin{list}{}{
\settowidth{\tmplength}{\textbf{Declaração}}
\setlength{\itemindent}{0cm}
\setlength{\listparindent}{0cm}
\setlength{\leftmargin}{\evensidemargin}
\addtolength{\leftmargin}{\tmplength}
\settowidth{\labelsep}{X}
\addtolength{\leftmargin}{\labelsep}
\setlength{\labelwidth}{\tmplength}
}
\begin{flushleft}
\item[\textbf{Declaração}\hfill]
\begin{ttfamily}
public {\_}Destoy{\_}ParamExecucao: Boolean;\end{ttfamily}


\end{flushleft}
\end{list}
\subsection*{TParamExecucao{\_}types Classe}
\subsubsection*{\large{\textbf{Hierarquia}}\normalsize\hspace{1ex}\hfill}
TParamExecucao{\_}types {$>$} \begin{ttfamily}TObjectsMethods\end{ttfamily}(\ref{mi.rtl.Objects.Methods.TObjectsMethods}) {$>$} \begin{ttfamily}TObjectsConsts\end{ttfamily}(\ref{mi.rtl.Objects.Consts.TObjectsConsts}) {$>$} 
TObjectsTypes
\subsubsection*{\large{\textbf{Descrição}}\normalsize\hspace{1ex}\hfill}
no description available, TObjectsMethods description follows\begin{itemize}
\item A classe \textbf{\begin{ttfamily}TObjectsMethods\end{ttfamily}} implementa os método de classe comum a todas as classes de TObjects do pacote \textbf{\begin{ttfamily}mi.rtl\end{ttfamily}(\ref{mi.rtl})}.
\end{itemize}\subsection*{TParamExecucao{\_}consts Classe}
\subsubsection*{\large{\textbf{Hierarquia}}\normalsize\hspace{1ex}\hfill}
TParamExecucao{\_}consts {$>$} \begin{ttfamily}TParamExecucao{\_}types\end{ttfamily}(\ref{mi.rtl.Objects.Methods.Paramexecucao.TParamExecucao_types}) {$>$} \begin{ttfamily}TObjectsMethods\end{ttfamily}(\ref{mi.rtl.Objects.Methods.TObjectsMethods}) {$>$} \begin{ttfamily}TObjectsConsts\end{ttfamily}(\ref{mi.rtl.Objects.Consts.TObjectsConsts}) {$>$} 
TObjectsTypes
\subsubsection*{\large{\textbf{Descrição}}\normalsize\hspace{1ex}\hfill}
no description available, TParamExecucao{\_}types description followsno description available, TObjectsMethods description follows\begin{itemize}
\item A classe \textbf{\begin{ttfamily}TObjectsMethods\end{ttfamily}} implementa os método de classe comum a todas as classes de TObjects do pacote \textbf{\begin{ttfamily}mi.rtl\end{ttfamily}(\ref{mi.rtl})}.
\end{itemize}\subsubsection*{\large{\textbf{Campos}}\normalsize\hspace{1ex}\hfill}
\paragraph*{{\_}Set{\_}NomeDeArquivosGenericos}\hspace*{\fill}

\begin{list}{}{
\settowidth{\tmplength}{\textbf{Declaração}}
\setlength{\itemindent}{0cm}
\setlength{\listparindent}{0cm}
\setlength{\leftmargin}{\evensidemargin}
\addtolength{\leftmargin}{\tmplength}
\settowidth{\labelsep}{X}
\addtolength{\leftmargin}{\labelsep}
\setlength{\labelwidth}{\tmplength}
}
\begin{flushleft}
\item[\textbf{Declaração}\hfill]
\begin{ttfamily}
public var {\_}Set{\_}NomeDeArquivosGenericos: TSet{\_}NomeDeArquivosGenericos; static;\end{ttfamily}


\end{flushleft}
\par
\item[\textbf{Descrição}]
Usado para inicializar os paths da sessão.

\end{list}
\paragraph*{Set{\_}NomeDeArquivosGenericos{\_}Global}\hspace*{\fill}

\begin{list}{}{
\settowidth{\tmplength}{\textbf{Declaração}}
\setlength{\itemindent}{0cm}
\setlength{\listparindent}{0cm}
\setlength{\leftmargin}{\evensidemargin}
\addtolength{\leftmargin}{\tmplength}
\settowidth{\labelsep}{X}
\addtolength{\leftmargin}{\labelsep}
\setlength{\labelwidth}{\tmplength}
}
\begin{flushleft}
\item[\textbf{Declaração}\hfill]
\begin{ttfamily}
public const Set{\_}NomeDeArquivosGenericos{\_}Global : TSet{\_}NomeDeArquivosGenericos = nil;\end{ttfamily}


\end{flushleft}
\end{list}
\subsection*{TParamExecucao Classe}
\subsubsection*{\large{\textbf{Hierarquia}}\normalsize\hspace{1ex}\hfill}
TParamExecucao {$>$} \begin{ttfamily}TParamExecucao{\_}consts\end{ttfamily}(\ref{mi.rtl.Objects.Methods.Paramexecucao.TParamExecucao_consts}) {$>$} \begin{ttfamily}TParamExecucao{\_}types\end{ttfamily}(\ref{mi.rtl.Objects.Methods.Paramexecucao.TParamExecucao_types}) {$>$} \begin{ttfamily}TObjectsMethods\end{ttfamily}(\ref{mi.rtl.Objects.Methods.TObjectsMethods}) {$>$} \begin{ttfamily}TObjectsConsts\end{ttfamily}(\ref{mi.rtl.Objects.Consts.TObjectsConsts}) {$>$} 
TObjectsTypes
\subsubsection*{\large{\textbf{Descrição}}\normalsize\hspace{1ex}\hfill}
no description available, TParamExecucao{\_}consts description followsno description available, TParamExecucao{\_}types description followsno description available, TObjectsMethods description follows\begin{itemize}
\item A classe \textbf{\begin{ttfamily}TObjectsMethods\end{ttfamily}} implementa os método de classe comum a todas as classes de TObjects do pacote \textbf{\begin{ttfamily}mi.rtl\end{ttfamily}(\ref{mi.rtl})}.
\end{itemize}\subsubsection*{\large{\textbf{Propriedades}}\normalsize\hspace{1ex}\hfill}
\paragraph*{HostName}\hspace*{\fill}

\begin{list}{}{
\settowidth{\tmplength}{\textbf{Declaração}}
\setlength{\itemindent}{0cm}
\setlength{\listparindent}{0cm}
\setlength{\leftmargin}{\evensidemargin}
\addtolength{\leftmargin}{\tmplength}
\settowidth{\labelsep}{X}
\addtolength{\leftmargin}{\labelsep}
\setlength{\labelwidth}{\tmplength}
}
\begin{flushleft}
\item[\textbf{Declaração}\hfill]
\begin{ttfamily}
public property HostName : AnsiString Read Get{\_}HostName Write SetDominioHost;\end{ttfamily}


\end{flushleft}
\par
\item[\textbf{Descrição}]
O campo \textbf{\begin{ttfamily}HostName\end{ttfamily}} contem o nome ou o ip do banco de dados

\begin{itemize}
\item \textbf{REFERÊNCIA} [nomes de host da Internet](https://networkencyclopedia.com/hostname/{\#}:~:text=What{\%}20is{\%}20Hostname{\%}3F,DNS{\%}20server{\%}20or{\%}20hosts{\%}20files.)
\end{itemize}

\end{list}
\subsubsection*{\large{\textbf{Campos}}\normalsize\hspace{1ex}\hfill}
\paragraph*{okCreate}\hspace*{\fill}

\begin{list}{}{
\settowidth{\tmplength}{\textbf{Declaração}}
\setlength{\itemindent}{0cm}
\setlength{\listparindent}{0cm}
\setlength{\leftmargin}{\evensidemargin}
\addtolength{\leftmargin}{\tmplength}
\settowidth{\labelsep}{X}
\addtolength{\leftmargin}{\labelsep}
\setlength{\labelwidth}{\tmplength}
}
\begin{flushleft}
\item[\textbf{Declaração}\hfill]
\begin{ttfamily}
protected Var okCreate:Boolean;\end{ttfamily}


\end{flushleft}
\end{list}
\paragraph*{NomeDeArquivosGenericos}\hspace*{\fill}

\begin{list}{}{
\settowidth{\tmplength}{\textbf{Declaração}}
\setlength{\itemindent}{0cm}
\setlength{\listparindent}{0cm}
\setlength{\leftmargin}{\evensidemargin}
\addtolength{\leftmargin}{\tmplength}
\settowidth{\labelsep}{X}
\addtolength{\leftmargin}{\labelsep}
\setlength{\labelwidth}{\tmplength}
}
\begin{flushleft}
\item[\textbf{Declaração}\hfill]
\begin{ttfamily}
public NomeDeArquivosGenericos: TNomeDeArquivosGenericos;\end{ttfamily}


\end{flushleft}
\end{list}
\paragraph*{PathRaiz}\hspace*{\fill}

\begin{list}{}{
\settowidth{\tmplength}{\textbf{Declaração}}
\setlength{\itemindent}{0cm}
\setlength{\listparindent}{0cm}
\setlength{\leftmargin}{\evensidemargin}
\addtolength{\leftmargin}{\tmplength}
\settowidth{\labelsep}{X}
\addtolength{\leftmargin}{\labelsep}
\setlength{\labelwidth}{\tmplength}
}
\begin{flushleft}
\item[\textbf{Declaração}\hfill]
\begin{ttfamily}
public PathRaiz: AnsiString;\end{ttfamily}


\end{flushleft}
\end{list}
\paragraph*{Tipo{\_}de{\_}Execucao}\hspace*{\fill}

\begin{list}{}{
\settowidth{\tmplength}{\textbf{Declaração}}
\setlength{\itemindent}{0cm}
\setlength{\listparindent}{0cm}
\setlength{\leftmargin}{\evensidemargin}
\addtolength{\leftmargin}{\tmplength}
\settowidth{\labelsep}{X}
\addtolength{\leftmargin}{\labelsep}
\setlength{\labelwidth}{\tmplength}
}
\begin{flushleft}
\item[\textbf{Declaração}\hfill]
\begin{ttfamily}
public Tipo{\_}de{\_}Execucao: TParamExecucao{\_}Tipo{\_}de{\_}Execucao;\end{ttfamily}


\end{flushleft}
\end{list}
\paragraph*{Identificacao}\hspace*{\fill}

\begin{list}{}{
\settowidth{\tmplength}{\textbf{Declaração}}
\setlength{\itemindent}{0cm}
\setlength{\listparindent}{0cm}
\setlength{\leftmargin}{\evensidemargin}
\addtolength{\leftmargin}{\tmplength}
\settowidth{\labelsep}{X}
\addtolength{\leftmargin}{\labelsep}
\setlength{\labelwidth}{\tmplength}
}
\begin{flushleft}
\item[\textbf{Declaração}\hfill]
\begin{ttfamily}
public Identificacao: TIdentificacao;\end{ttfamily}


\end{flushleft}
\end{list}
\paragraph*{Command}\hspace*{\fill}

\begin{list}{}{
\settowidth{\tmplength}{\textbf{Declaração}}
\setlength{\itemindent}{0cm}
\setlength{\listparindent}{0cm}
\setlength{\leftmargin}{\evensidemargin}
\addtolength{\leftmargin}{\tmplength}
\settowidth{\labelsep}{X}
\addtolength{\leftmargin}{\labelsep}
\setlength{\labelwidth}{\tmplength}
}
\begin{flushleft}
\item[\textbf{Declaração}\hfill]
\begin{ttfamily}
public Command: SmallInt;\end{ttfamily}


\end{flushleft}
\end{list}
\paragraph*{Modulo}\hspace*{\fill}

\begin{list}{}{
\settowidth{\tmplength}{\textbf{Declaração}}
\setlength{\itemindent}{0cm}
\setlength{\listparindent}{0cm}
\setlength{\leftmargin}{\evensidemargin}
\addtolength{\leftmargin}{\tmplength}
\settowidth{\labelsep}{X}
\addtolength{\leftmargin}{\labelsep}
\setlength{\labelwidth}{\tmplength}
}
\begin{flushleft}
\item[\textbf{Declaração}\hfill]
\begin{ttfamily}
public Modulo: SmallInt;\end{ttfamily}


\end{flushleft}
\end{list}
\paragraph*{Acao{\_}Form}\hspace*{\fill}

\begin{list}{}{
\settowidth{\tmplength}{\textbf{Declaração}}
\setlength{\itemindent}{0cm}
\setlength{\listparindent}{0cm}
\setlength{\leftmargin}{\evensidemargin}
\addtolength{\leftmargin}{\tmplength}
\settowidth{\labelsep}{X}
\addtolength{\leftmargin}{\labelsep}
\setlength{\labelwidth}{\tmplength}
}
\begin{flushleft}
\item[\textbf{Declaração}\hfill]
\begin{ttfamily}
public Acao{\_}Form: AnsiString;\end{ttfamily}


\end{flushleft}
\end{list}
\paragraph*{DataAtual}\hspace*{\fill}

\begin{list}{}{
\settowidth{\tmplength}{\textbf{Declaração}}
\setlength{\itemindent}{0cm}
\setlength{\listparindent}{0cm}
\setlength{\leftmargin}{\evensidemargin}
\addtolength{\leftmargin}{\tmplength}
\settowidth{\labelsep}{X}
\addtolength{\leftmargin}{\labelsep}
\setlength{\labelwidth}{\tmplength}
}
\begin{flushleft}
\item[\textbf{Declaração}\hfill]
\begin{ttfamily}
public DataAtual: TDates.typeData;\end{ttfamily}


\end{flushleft}
\end{list}
\paragraph*{DatabaseNameCharSet}\hspace*{\fill}

\begin{list}{}{
\settowidth{\tmplength}{\textbf{Declaração}}
\setlength{\itemindent}{0cm}
\setlength{\listparindent}{0cm}
\setlength{\leftmargin}{\evensidemargin}
\addtolength{\leftmargin}{\tmplength}
\settowidth{\labelsep}{X}
\addtolength{\leftmargin}{\labelsep}
\setlength{\labelwidth}{\tmplength}
}
\begin{flushleft}
\item[\textbf{Declaração}\hfill]
\begin{ttfamily}
public DatabaseNameCharSet: AnsiString;\end{ttfamily}


\end{flushleft}
\end{list}
\paragraph*{List{\_}Value{\_}Default}\hspace*{\fill}

\begin{list}{}{
\settowidth{\tmplength}{\textbf{Declaração}}
\setlength{\itemindent}{0cm}
\setlength{\listparindent}{0cm}
\setlength{\leftmargin}{\evensidemargin}
\addtolength{\leftmargin}{\tmplength}
\settowidth{\labelsep}{X}
\addtolength{\leftmargin}{\labelsep}
\setlength{\labelwidth}{\tmplength}
}
\begin{flushleft}
\item[\textbf{Declaração}\hfill]
\begin{ttfamily}
public List{\_}Value{\_}Default: AnsiString;\end{ttfamily}


\end{flushleft}
\end{list}
\subsubsection*{\large{\textbf{Métodos}}\normalsize\hspace{1ex}\hfill}
\paragraph*{Create}\hspace*{\fill}

\begin{list}{}{
\settowidth{\tmplength}{\textbf{Declaração}}
\setlength{\itemindent}{0cm}
\setlength{\listparindent}{0cm}
\setlength{\leftmargin}{\evensidemargin}
\addtolength{\leftmargin}{\tmplength}
\settowidth{\labelsep}{X}
\addtolength{\leftmargin}{\labelsep}
\setlength{\labelwidth}{\tmplength}
}
\begin{flushleft}
\item[\textbf{Declaração}\hfill]
\begin{ttfamily}
public Constructor Create(aPathRaiz:Ansistring); overload; Virtual;\end{ttfamily}


\end{flushleft}
\end{list}
\paragraph*{Create}\hspace*{\fill}

\begin{list}{}{
\settowidth{\tmplength}{\textbf{Declaração}}
\setlength{\itemindent}{0cm}
\setlength{\listparindent}{0cm}
\setlength{\leftmargin}{\evensidemargin}
\addtolength{\leftmargin}{\tmplength}
\settowidth{\labelsep}{X}
\addtolength{\leftmargin}{\labelsep}
\setlength{\labelwidth}{\tmplength}
}
\begin{flushleft}
\item[\textbf{Declaração}\hfill]
\begin{ttfamily}
public Constructor Create(aOwner:TComponent); overload; override;\end{ttfamily}


\end{flushleft}
\end{list}
\paragraph*{Destroy}\hspace*{\fill}

\begin{list}{}{
\settowidth{\tmplength}{\textbf{Declaração}}
\setlength{\itemindent}{0cm}
\setlength{\listparindent}{0cm}
\setlength{\leftmargin}{\evensidemargin}
\addtolength{\leftmargin}{\tmplength}
\settowidth{\labelsep}{X}
\addtolength{\leftmargin}{\labelsep}
\setlength{\labelwidth}{\tmplength}
}
\begin{flushleft}
\item[\textbf{Declaração}\hfill]
\begin{ttfamily}
public Destructor Destroy; override;\end{ttfamily}


\end{flushleft}
\end{list}
\paragraph*{Set{\_}NomeDeArquivosGenericos}\hspace*{\fill}

\begin{list}{}{
\settowidth{\tmplength}{\textbf{Declaração}}
\setlength{\itemindent}{0cm}
\setlength{\listparindent}{0cm}
\setlength{\leftmargin}{\evensidemargin}
\addtolength{\leftmargin}{\tmplength}
\settowidth{\labelsep}{X}
\addtolength{\leftmargin}{\labelsep}
\setlength{\labelwidth}{\tmplength}
}
\begin{flushleft}
\item[\textbf{Declaração}\hfill]
\begin{ttfamily}
public Function Set{\_}NomeDeArquivosGenericos() :boolean;\end{ttfamily}


\end{flushleft}
\end{list}
\paragraph*{Acao{\_}Form{\_}Is{\_}Event}\hspace*{\fill}

\begin{list}{}{
\settowidth{\tmplength}{\textbf{Declaração}}
\setlength{\itemindent}{0cm}
\setlength{\listparindent}{0cm}
\setlength{\leftmargin}{\evensidemargin}
\addtolength{\leftmargin}{\tmplength}
\settowidth{\labelsep}{X}
\addtolength{\leftmargin}{\labelsep}
\setlength{\labelwidth}{\tmplength}
}
\begin{flushleft}
\item[\textbf{Declaração}\hfill]
\begin{ttfamily}
public Function Acao{\_}Form{\_}Is{\_}Event:Boolean;\end{ttfamily}


\end{flushleft}
\end{list}
\paragraph*{Acao{\_}Form{\_}Is{\_}Mb{\_}Bit}\hspace*{\fill}

\begin{list}{}{
\settowidth{\tmplength}{\textbf{Declaração}}
\setlength{\itemindent}{0cm}
\setlength{\listparindent}{0cm}
\setlength{\leftmargin}{\evensidemargin}
\addtolength{\leftmargin}{\tmplength}
\settowidth{\labelsep}{X}
\addtolength{\leftmargin}{\labelsep}
\setlength{\labelwidth}{\tmplength}
}
\begin{flushleft}
\item[\textbf{Declaração}\hfill]
\begin{ttfamily}
public Function Acao{\_}Form{\_}Is{\_}Mb{\_}Bit:Boolean;\end{ttfamily}


\end{flushleft}
\end{list}
\paragraph*{Set{\_}ParamStr}\hspace*{\fill}

\begin{list}{}{
\settowidth{\tmplength}{\textbf{Declaração}}
\setlength{\itemindent}{0cm}
\setlength{\listparindent}{0cm}
\setlength{\leftmargin}{\evensidemargin}
\addtolength{\leftmargin}{\tmplength}
\settowidth{\labelsep}{X}
\addtolength{\leftmargin}{\labelsep}
\setlength{\labelwidth}{\tmplength}
}
\begin{flushleft}
\item[\textbf{Declaração}\hfill]
\begin{ttfamily}
public procedure Set{\_}ParamStr(wFilial,WUsuario,wPassword: tString); Overload;\end{ttfamily}


\end{flushleft}
\end{list}
\paragraph*{Set{\_}ParamStr}\hspace*{\fill}

\begin{list}{}{
\settowidth{\tmplength}{\textbf{Declaração}}
\setlength{\itemindent}{0cm}
\setlength{\listparindent}{0cm}
\setlength{\leftmargin}{\evensidemargin}
\addtolength{\leftmargin}{\tmplength}
\settowidth{\labelsep}{X}
\addtolength{\leftmargin}{\labelsep}
\setlength{\labelwidth}{\tmplength}
}
\begin{flushleft}
\item[\textbf{Declaração}\hfill]
\begin{ttfamily}
public procedure Set{\_}ParamStr(wFilial:Byte;WUsuario:SmallInt;wPassword:tString); Overload;\end{ttfamily}


\end{flushleft}
\end{list}
\paragraph*{Set{\_}ParamStr}\hspace*{\fill}

\begin{list}{}{
\settowidth{\tmplength}{\textbf{Declaração}}
\setlength{\itemindent}{0cm}
\setlength{\listparindent}{0cm}
\setlength{\leftmargin}{\evensidemargin}
\addtolength{\leftmargin}{\tmplength}
\settowidth{\labelsep}{X}
\addtolength{\leftmargin}{\labelsep}
\setlength{\labelwidth}{\tmplength}
}
\begin{flushleft}
\item[\textbf{Declaração}\hfill]
\begin{ttfamily}
public procedure Set{\_}ParamStr(wFilial,WUsuario,wPassword,wCommand : tString); Overload;\end{ttfamily}


\end{flushleft}
\end{list}
\paragraph*{Set{\_}ParamStr}\hspace*{\fill}

\begin{list}{}{
\settowidth{\tmplength}{\textbf{Declaração}}
\setlength{\itemindent}{0cm}
\setlength{\listparindent}{0cm}
\setlength{\leftmargin}{\evensidemargin}
\addtolength{\leftmargin}{\tmplength}
\settowidth{\labelsep}{X}
\addtolength{\leftmargin}{\labelsep}
\setlength{\labelwidth}{\tmplength}
}
\begin{flushleft}
\item[\textbf{Declaração}\hfill]
\begin{ttfamily}
public procedure Set{\_}ParamStr(wFilial:Byte;WUsuario:SmallInt;wPassword:tString;wCommand : SmallInt); Overload;\end{ttfamily}


\end{flushleft}
\end{list}
\paragraph*{Set{\_}ParamStr}\hspace*{\fill}

\begin{list}{}{
\settowidth{\tmplength}{\textbf{Declaração}}
\setlength{\itemindent}{0cm}
\setlength{\listparindent}{0cm}
\setlength{\leftmargin}{\evensidemargin}
\addtolength{\leftmargin}{\tmplength}
\settowidth{\labelsep}{X}
\addtolength{\leftmargin}{\labelsep}
\setlength{\labelwidth}{\tmplength}
}
\begin{flushleft}
\item[\textbf{Declaração}\hfill]
\begin{ttfamily}
public procedure Set{\_}ParamStr(wFilial:Byte;WUsuario:SmallInt;wNome{\_}Compreto{\_}do{\_}Usuario : tString;wPassword:tString); Overload;\end{ttfamily}


\end{flushleft}
\end{list}
\paragraph*{Set{\_}ParamStr}\hspace*{\fill}

\begin{list}{}{
\settowidth{\tmplength}{\textbf{Declaração}}
\setlength{\itemindent}{0cm}
\setlength{\listparindent}{0cm}
\setlength{\leftmargin}{\evensidemargin}
\addtolength{\leftmargin}{\tmplength}
\settowidth{\labelsep}{X}
\addtolength{\leftmargin}{\labelsep}
\setlength{\labelwidth}{\tmplength}
}
\begin{flushleft}
\item[\textbf{Declaração}\hfill]
\begin{ttfamily}
public procedure Set{\_}ParamStr(wFilial:Byte;WUsuario:SmallInt;wNome{\_}Compreto{\_}do{\_}Usuario : tString;wPassword:tString;aUsername: tString); Overload;\end{ttfamily}


\end{flushleft}
\end{list}
\paragraph*{Set{\_}ParamStr}\hspace*{\fill}

\begin{list}{}{
\settowidth{\tmplength}{\textbf{Declaração}}
\setlength{\itemindent}{0cm}
\setlength{\listparindent}{0cm}
\setlength{\leftmargin}{\evensidemargin}
\addtolength{\leftmargin}{\tmplength}
\settowidth{\labelsep}{X}
\addtolength{\leftmargin}{\labelsep}
\setlength{\labelwidth}{\tmplength}
}
\begin{flushleft}
\item[\textbf{Declaração}\hfill]
\begin{ttfamily}
public procedure Set{\_}ParamStr(wFilial,WUsuario,wPassword,wModulo,wCommand : tString); Overload;\end{ttfamily}


\end{flushleft}
\end{list}
\paragraph*{Set{\_}ParamStr}\hspace*{\fill}

\begin{list}{}{
\settowidth{\tmplength}{\textbf{Declaração}}
\setlength{\itemindent}{0cm}
\setlength{\listparindent}{0cm}
\setlength{\leftmargin}{\evensidemargin}
\addtolength{\leftmargin}{\tmplength}
\settowidth{\labelsep}{X}
\addtolength{\leftmargin}{\labelsep}
\setlength{\labelwidth}{\tmplength}
}
\begin{flushleft}
\item[\textbf{Declaração}\hfill]
\begin{ttfamily}
public procedure Set{\_}ParamStr(wFilial,WUsuario,wPassword,wModulo,wCommand,a{\_}DataAtual : tString); Overload;\end{ttfamily}


\end{flushleft}
\end{list}
\paragraph*{Set{\_}ParamStr}\hspace*{\fill}

\begin{list}{}{
\settowidth{\tmplength}{\textbf{Declaração}}
\setlength{\itemindent}{0cm}
\setlength{\listparindent}{0cm}
\setlength{\leftmargin}{\evensidemargin}
\addtolength{\leftmargin}{\tmplength}
\settowidth{\labelsep}{X}
\addtolength{\leftmargin}{\labelsep}
\setlength{\labelwidth}{\tmplength}
}
\begin{flushleft}
\item[\textbf{Declaração}\hfill]
\begin{ttfamily}
public procedure Set{\_}ParamStr(wFilial,WUsuario,wPassword,wModulo,wCommand,a{\_}DataAtual : tString;wAcao{\_}Form: AnsiString); Overload;\end{ttfamily}


\end{flushleft}
\end{list}
\paragraph*{Set{\_}ParamStr}\hspace*{\fill}

\begin{list}{}{
\settowidth{\tmplength}{\textbf{Declaração}}
\setlength{\itemindent}{0cm}
\setlength{\listparindent}{0cm}
\setlength{\leftmargin}{\evensidemargin}
\addtolength{\leftmargin}{\tmplength}
\settowidth{\labelsep}{X}
\addtolength{\leftmargin}{\labelsep}
\setlength{\labelwidth}{\tmplength}
}
\begin{flushleft}
\item[\textbf{Declaração}\hfill]
\begin{ttfamily}
public procedure Set{\_}ParamStr(wFilial,WUsuario,wPassword,wModulo,wCommand,a{\_}DataAtual : tString;wAcao{\_}Form: AnsiString;wList{\_}Value{\_}Default:AnsiString); Overload;\end{ttfamily}


\end{flushleft}
\end{list}
\paragraph*{Set{\_}ParamStr}\hspace*{\fill}

\begin{list}{}{
\settowidth{\tmplength}{\textbf{Declaração}}
\setlength{\itemindent}{0cm}
\setlength{\listparindent}{0cm}
\setlength{\leftmargin}{\evensidemargin}
\addtolength{\leftmargin}{\tmplength}
\settowidth{\labelsep}{X}
\addtolength{\leftmargin}{\labelsep}
\setlength{\labelwidth}{\tmplength}
}
\begin{flushleft}
\item[\textbf{Declaração}\hfill]
\begin{ttfamily}
public procedure Set{\_}ParamStr(wFilial:Byte; WUsuario:SmallInt; wPassword:tString; wModulo:SmallInt; wCommand : SmallInt); Overload;\end{ttfamily}


\end{flushleft}
\end{list}
\paragraph*{Set{\_}ParamStr}\hspace*{\fill}

\begin{list}{}{
\settowidth{\tmplength}{\textbf{Declaração}}
\setlength{\itemindent}{0cm}
\setlength{\listparindent}{0cm}
\setlength{\leftmargin}{\evensidemargin}
\addtolength{\leftmargin}{\tmplength}
\settowidth{\labelsep}{X}
\addtolength{\leftmargin}{\labelsep}
\setlength{\labelwidth}{\tmplength}
}
\begin{flushleft}
\item[\textbf{Declaração}\hfill]
\begin{ttfamily}
public procedure Set{\_}ParamStr(wFilial:AnsiString; WUsuario:SmallInt; wPassword:tString; wModulo:SmallInt; wCommand : SmallInt); Overload;\end{ttfamily}


\end{flushleft}
\end{list}
\paragraph*{Check{\_}Se{\_}Comando{\_}autorizado}\hspace*{\fill}

\begin{list}{}{
\settowidth{\tmplength}{\textbf{Declaração}}
\setlength{\itemindent}{0cm}
\setlength{\listparindent}{0cm}
\setlength{\leftmargin}{\evensidemargin}
\addtolength{\leftmargin}{\tmplength}
\settowidth{\labelsep}{X}
\addtolength{\leftmargin}{\labelsep}
\setlength{\labelwidth}{\tmplength}
}
\begin{flushleft}
\item[\textbf{Declaração}\hfill]
\begin{ttfamily}
public procedure Check{\_}Se{\_}Comando{\_}autorizado;\end{ttfamily}


\end{flushleft}
\end{list}
\paragraph*{Get{\_}Password{\_}do{\_}Comando}\hspace*{\fill}

\begin{list}{}{
\settowidth{\tmplength}{\textbf{Declaração}}
\setlength{\itemindent}{0cm}
\setlength{\listparindent}{0cm}
\setlength{\leftmargin}{\evensidemargin}
\addtolength{\leftmargin}{\tmplength}
\settowidth{\labelsep}{X}
\addtolength{\leftmargin}{\labelsep}
\setlength{\labelwidth}{\tmplength}
}
\begin{flushleft}
\item[\textbf{Declaração}\hfill]
\begin{ttfamily}
public Function Get{\_}Password{\_}do{\_}Comando(aModulo: Byte; aComando : SmallInt):tString;\end{ttfamily}


\end{flushleft}
\end{list}
\paragraph*{Param{\_}Execucao}\hspace*{\fill}

\begin{list}{}{
\settowidth{\tmplength}{\textbf{Declaração}}
\setlength{\itemindent}{0cm}
\setlength{\listparindent}{0cm}
\setlength{\leftmargin}{\evensidemargin}
\addtolength{\leftmargin}{\tmplength}
\settowidth{\labelsep}{X}
\addtolength{\leftmargin}{\labelsep}
\setlength{\labelwidth}{\tmplength}
}
\begin{flushleft}
\item[\textbf{Declaração}\hfill]
\begin{ttfamily}
public class Function Param{\_}Execucao: TParamExecucao;\end{ttfamily}


\end{flushleft}
\end{list}
\paragraph*{Set{\_}ParamExecucao}\hspace*{\fill}

\begin{list}{}{
\settowidth{\tmplength}{\textbf{Declaração}}
\setlength{\itemindent}{0cm}
\setlength{\listparindent}{0cm}
\setlength{\leftmargin}{\evensidemargin}
\addtolength{\leftmargin}{\tmplength}
\settowidth{\labelsep}{X}
\addtolength{\leftmargin}{\labelsep}
\setlength{\labelwidth}{\tmplength}
}
\begin{flushleft}
\item[\textbf{Declaração}\hfill]
\begin{ttfamily}
public class Procedure Set{\_}ParamExecucao(aParamExecucao : TParamExecucao);\end{ttfamily}


\end{flushleft}
\end{list}
\chapter{Unit mi.rtl.Objects.Methods.Paramexecucao.Application}
\section{Uses}
\begin{itemize}
\item \begin{ttfamily}Classes\end{ttfamily}\item \begin{ttfamily}SysUtils\end{ttfamily}\item \begin{ttfamily}mi.rtl.types\end{ttfamily}(\ref{mi.rtl.Types})\item \begin{ttfamily}mi.rtl.applicationabstract\end{ttfamily}(\ref{mi.rtl.ApplicationAbstract})\item \begin{ttfamily}mi.rtl.objects.consts.MI{\_}MsgBox\end{ttfamily}\item \begin{ttfamily}mi.rtl.objects.consts.progressdlg{\_}if\end{ttfamily}(\ref{mi.rtl.Objects.Consts.ProgressDlg_If})\item \begin{ttfamily}mi.rtl.objects.consts.logs\end{ttfamily}(\ref{mi.rtl.Objects.Consts.Logs})\item \begin{ttfamily}mi.rtl.objects.methods.ParamExecucao\end{ttfamily}(\ref{mi.rtl.Objects.Methods.Paramexecucao})\end{itemize}
\section{Visão Geral}
\begin{description}
\item[\texttt{\begin{ttfamily}TApplication{\_}type\end{ttfamily} Classe}]
\item[\texttt{\begin{ttfamily}TApplicationConsts\end{ttfamily} Classe}]
\item[\texttt{\begin{ttfamily}TApplication\end{ttfamily} Classe}]
\end{description}
\begin{description}
\item[\texttt{application}]
\item[\texttt{Setapplication}]
\end{description}
\section{Classes, Interfaces, Objetos e Registros}
\subsection*{TApplication{\_}type Classe}
\subsubsection*{\large{\textbf{Hierarquia}}\normalsize\hspace{1ex}\hfill}
TApplication{\_}type {$>$} \begin{ttfamily}TApplicationAbstract\end{ttfamily}(\ref{mi.rtl.ApplicationAbstract.TApplicationAbstract}) {$>$} 
TCustomApplication
\subsubsection*{\large{\textbf{Descrição}}\normalsize\hspace{1ex}\hfill}
A class \textit{\begin{ttfamily}TApplication{\_}type\end{ttfamily}}* é usada para capsular todas as variáveis globais do projeto e gerenciar o ciclo de vida do aplicativo\subsection*{TApplicationConsts Classe}
\subsubsection*{\large{\textbf{Hierarquia}}\normalsize\hspace{1ex}\hfill}
TApplicationConsts {$>$} \begin{ttfamily}TApplication{\_}type\end{ttfamily}(\ref{mi.rtl.Objects.Methods.Paramexecucao.Application.TApplication_type}) {$>$} \begin{ttfamily}TApplicationAbstract\end{ttfamily}(\ref{mi.rtl.ApplicationAbstract.TApplicationAbstract}) {$>$} 
TCustomApplication
\subsubsection*{\large{\textbf{Descrição}}\normalsize\hspace{1ex}\hfill}
no description available, TApplication{\_}type description followsA class \textit{\begin{ttfamily}TApplication{\_}type\end{ttfamily}}* é usada para capsular todas as variáveis globais do projeto e gerenciar o ciclo de vida do aplicativo\subsubsection*{\large{\textbf{Campos}}\normalsize\hspace{1ex}\hfill}
\paragraph*{origin}\hspace*{\fill}

\begin{list}{}{
\settowidth{\tmplength}{\textbf{Declaração}}
\setlength{\itemindent}{0cm}
\setlength{\listparindent}{0cm}
\setlength{\leftmargin}{\evensidemargin}
\addtolength{\leftmargin}{\tmplength}
\settowidth{\labelsep}{X}
\addtolength{\leftmargin}{\labelsep}
\setlength{\labelwidth}{\tmplength}
}
\begin{flushleft}
\item[\textbf{Declaração}\hfill]
\begin{ttfamily}
public origin: TTypes.TPoint;\end{ttfamily}


\end{flushleft}
\par
\item[\textbf{Descrição}]
Ponto inferior a esqueda da aplicação

\end{list}
\paragraph*{Size}\hspace*{\fill}

\begin{list}{}{
\settowidth{\tmplength}{\textbf{Declaração}}
\setlength{\itemindent}{0cm}
\setlength{\listparindent}{0cm}
\setlength{\leftmargin}{\evensidemargin}
\addtolength{\leftmargin}{\tmplength}
\settowidth{\labelsep}{X}
\addtolength{\leftmargin}{\labelsep}
\setlength{\labelwidth}{\tmplength}
}
\begin{flushleft}
\item[\textbf{Declaração}\hfill]
\begin{ttfamily}
public Size: TTypes.TPoint;\end{ttfamily}


\end{flushleft}
\par
\item[\textbf{Descrição}]
Ponto superior a direta da aplicação

\end{list}
\paragraph*{MI{\_}MsgBox}\hspace*{\fill}

\begin{list}{}{
\settowidth{\tmplength}{\textbf{Declaração}}
\setlength{\itemindent}{0cm}
\setlength{\listparindent}{0cm}
\setlength{\leftmargin}{\evensidemargin}
\addtolength{\leftmargin}{\tmplength}
\settowidth{\labelsep}{X}
\addtolength{\leftmargin}{\labelsep}
\setlength{\labelwidth}{\tmplength}
}
\begin{flushleft}
\item[\textbf{Declaração}\hfill]
\begin{ttfamily}
public MI{\_}MsgBox: TMI{\_}MsgBox;\end{ttfamily}


\end{flushleft}
\end{list}
\paragraph*{Logs}\hspace*{\fill}

\begin{list}{}{
\settowidth{\tmplength}{\textbf{Declaração}}
\setlength{\itemindent}{0cm}
\setlength{\listparindent}{0cm}
\setlength{\leftmargin}{\evensidemargin}
\addtolength{\leftmargin}{\tmplength}
\settowidth{\labelsep}{X}
\addtolength{\leftmargin}{\labelsep}
\setlength{\labelwidth}{\tmplength}
}
\begin{flushleft}
\item[\textbf{Declaração}\hfill]
\begin{ttfamily}
public Logs: TFilesLogs;\end{ttfamily}


\end{flushleft}
\par
\item[\textbf{Descrição}]
\begin{itemize}
\item \begin{ttfamily}Logs\end{ttfamily} é inicializado em Initialization e destruído em finalization
\end{itemize}

\end{list}
\subsection*{TApplication Classe}
\subsubsection*{\large{\textbf{Hierarquia}}\normalsize\hspace{1ex}\hfill}
TApplication {$>$} \begin{ttfamily}TApplicationConsts\end{ttfamily}(\ref{mi.rtl.Objects.Methods.Paramexecucao.Application.TApplicationConsts}) {$>$} \begin{ttfamily}TApplication{\_}type\end{ttfamily}(\ref{mi.rtl.Objects.Methods.Paramexecucao.Application.TApplication_type}) {$>$} \begin{ttfamily}TApplicationAbstract\end{ttfamily}(\ref{mi.rtl.ApplicationAbstract.TApplicationAbstract}) {$>$} 
TCustomApplication
\subsubsection*{\large{\textbf{Descrição}}\normalsize\hspace{1ex}\hfill}
no description available, TApplicationConsts description followsno description available, TApplication{\_}type description followsA class \textit{\begin{ttfamily}TApplication{\_}type\end{ttfamily}}* é usada para capsular todas as variáveis globais do projeto e gerenciar o ciclo de vida do aplicativo\subsubsection*{\large{\textbf{Campos}}\normalsize\hspace{1ex}\hfill}
\paragraph*{ParamExecucao}\hspace*{\fill}

\begin{list}{}{
\settowidth{\tmplength}{\textbf{Declaração}}
\setlength{\itemindent}{0cm}
\setlength{\listparindent}{0cm}
\setlength{\leftmargin}{\evensidemargin}
\addtolength{\leftmargin}{\tmplength}
\settowidth{\labelsep}{X}
\addtolength{\leftmargin}{\labelsep}
\setlength{\labelwidth}{\tmplength}
}
\begin{flushleft}
\item[\textbf{Declaração}\hfill]
\begin{ttfamily}
public ParamExecucao: TParamExecucao;\end{ttfamily}


\end{flushleft}
\end{list}
\subsubsection*{\large{\textbf{Métodos}}\normalsize\hspace{1ex}\hfill}
\paragraph*{Create}\hspace*{\fill}

\begin{list}{}{
\settowidth{\tmplength}{\textbf{Declaração}}
\setlength{\itemindent}{0cm}
\setlength{\listparindent}{0cm}
\setlength{\leftmargin}{\evensidemargin}
\addtolength{\leftmargin}{\tmplength}
\settowidth{\labelsep}{X}
\addtolength{\leftmargin}{\labelsep}
\setlength{\labelwidth}{\tmplength}
}
\begin{flushleft}
\item[\textbf{Declaração}\hfill]
\begin{ttfamily}
public constructor Create(AOwner: TComponent); override;\end{ttfamily}


\end{flushleft}
\end{list}
\paragraph*{Destroy}\hspace*{\fill}

\begin{list}{}{
\settowidth{\tmplength}{\textbf{Declaração}}
\setlength{\itemindent}{0cm}
\setlength{\listparindent}{0cm}
\setlength{\leftmargin}{\evensidemargin}
\addtolength{\leftmargin}{\tmplength}
\settowidth{\labelsep}{X}
\addtolength{\leftmargin}{\labelsep}
\setlength{\labelwidth}{\tmplength}
}
\begin{flushleft}
\item[\textbf{Declaração}\hfill]
\begin{ttfamily}
public destructor Destroy; override;\end{ttfamily}


\end{flushleft}
\end{list}
\paragraph*{FileOptions{\_}CommandEnabled}\hspace*{\fill}

\begin{list}{}{
\settowidth{\tmplength}{\textbf{Declaração}}
\setlength{\itemindent}{0cm}
\setlength{\listparindent}{0cm}
\setlength{\leftmargin}{\evensidemargin}
\addtolength{\leftmargin}{\tmplength}
\settowidth{\labelsep}{X}
\addtolength{\leftmargin}{\labelsep}
\setlength{\labelwidth}{\tmplength}
}
\begin{flushleft}
\item[\textbf{Declaração}\hfill]
\begin{ttfamily}
public function FileOptions{\_}CommandEnabled(aCommand: AnsiString): Boolean; Virtual;\end{ttfamily}


\end{flushleft}
\par
\item[\textbf{Descrição}]
O método \textbf{\begin{ttfamily}FileOptions{\_}CommandEnabled\end{ttfamily}} deve ser redefinido na aplicações filhas para indicar se o comando a ser executado está habilitado no arquivo de opções.

\end{list}
\paragraph*{EnableCommands}\hspace*{\fill}

\begin{list}{}{
\settowidth{\tmplength}{\textbf{Declaração}}
\setlength{\itemindent}{0cm}
\setlength{\listparindent}{0cm}
\setlength{\leftmargin}{\evensidemargin}
\addtolength{\leftmargin}{\tmplength}
\settowidth{\labelsep}{X}
\addtolength{\leftmargin}{\labelsep}
\setlength{\labelwidth}{\tmplength}
}
\begin{flushleft}
\item[\textbf{Declaração}\hfill]
\begin{ttfamily}
public procedure EnableCommands(aCommands: TCommandSet); virtual;\end{ttfamily}


\end{flushleft}
\end{list}
\paragraph*{DisableCommands}\hspace*{\fill}

\begin{list}{}{
\settowidth{\tmplength}{\textbf{Declaração}}
\setlength{\itemindent}{0cm}
\setlength{\listparindent}{0cm}
\setlength{\leftmargin}{\evensidemargin}
\addtolength{\leftmargin}{\tmplength}
\settowidth{\labelsep}{X}
\addtolength{\leftmargin}{\labelsep}
\setlength{\labelwidth}{\tmplength}
}
\begin{flushleft}
\item[\textbf{Declaração}\hfill]
\begin{ttfamily}
public procedure DisableCommands(aCommands: TCommandSet); virtual;\end{ttfamily}


\end{flushleft}
\end{list}
\section{Funções e Procedimentos}
\subsection*{application}
\begin{list}{}{
\settowidth{\tmplength}{\textbf{Declaração}}
\setlength{\itemindent}{0cm}
\setlength{\listparindent}{0cm}
\setlength{\leftmargin}{\evensidemargin}
\addtolength{\leftmargin}{\tmplength}
\settowidth{\labelsep}{X}
\addtolength{\leftmargin}{\labelsep}
\setlength{\labelwidth}{\tmplength}
}
\begin{flushleft}
\item[\textbf{Declaração}\hfill]
\begin{ttfamily}
function application: TApplication;\end{ttfamily}


\end{flushleft}
\end{list}
\subsection*{Setapplication}
\begin{list}{}{
\settowidth{\tmplength}{\textbf{Declaração}}
\setlength{\itemindent}{0cm}
\setlength{\listparindent}{0cm}
\setlength{\leftmargin}{\evensidemargin}
\addtolength{\leftmargin}{\tmplength}
\settowidth{\labelsep}{X}
\addtolength{\leftmargin}{\labelsep}
\setlength{\labelwidth}{\tmplength}
}
\begin{flushleft}
\item[\textbf{Declaração}\hfill]
\begin{ttfamily}
Procedure Setapplication(aApplication : TApplication);\end{ttfamily}


\end{flushleft}
\end{list}
\chapter{Unit mi.rtl.Objects.Methods.StreamBase}
\section{Descrição}
\begin{itemize}
\item A unit \textbf{\begin{ttfamily}mi.rtl.Objects.Methods.StreamBase\end{ttfamily}} implementa a classe \begin{ttfamily}TStreamBase\end{ttfamily}(\ref{mi.rtl.Objects.Methods.StreamBase.TStreamBase}) do pacote \textbf{\begin{ttfamily}mi.rtl\end{ttfamily}(\ref{mi.rtl})}.

\begin{itemize}
\item \textbf{NOTAS} \begin{itemize}
\item O Use da classe \begin{ttfamily}mi.rtl.Objects.Methods.StreamBase\end{ttfamily} não deve ser instanciada antes de implementar os métodos abstratos;
\end{itemize}
\item \textbf{VERSÃO} \begin{itemize}
\item Alpha {-} 0.5.0.687
\end{itemize}
\item \textbf{HISTÓRICO} \begin{itemize}
\item Criado por: Paulo Sérgio da Silva Pacheco e{-}mail: paulosspacheco@yahoo.com.br \begin{itemize}
\item \textbf{19/11/2021} 21:25 a 23:15 Criar a unit mi.rtl.objects.methods.StreamBase.pas
\item \textbf{20/11/2021} 14:02 a 15:19 Documentação da classe e agrupar métodos virtuais, métodos não virtuais e proteger os métodos abstratos.
\end{itemize}
\end{itemize}
\item \textbf{CÓDIGO FONTE}: \begin{itemize}
\item 
\end{itemize}
\end{itemize}
\end{itemize}
\section{Uses}
\begin{itemize}
\item \begin{ttfamily}Classes\end{ttfamily}\item \begin{ttfamily}SysUtils\end{ttfamily}\item \begin{ttfamily}mi.rtl.types\end{ttfamily}(\ref{mi.rtl.Types})\item \begin{ttfamily}mi.rtl.objects.Methods\end{ttfamily}(\ref{mi.rtl.Objects.Methods})\end{itemize}
\section{Visão Geral}
\begin{description}
\item[\texttt{\begin{ttfamily}TStreamBase\end{ttfamily} Classe}]
\end{description}
\section{Classes, Interfaces, Objetos e Registros}
\subsection*{TStreamBase Classe}
\subsubsection*{\large{\textbf{Hierarquia}}\normalsize\hspace{1ex}\hfill}
TStreamBase {$>$} \begin{ttfamily}TObjectsMethods\end{ttfamily}(\ref{mi.rtl.Objects.Methods.TObjectsMethods}) {$>$} \begin{ttfamily}TObjectsConsts\end{ttfamily}(\ref{mi.rtl.Objects.Consts.TObjectsConsts}) {$>$} 
TObjectsTypes
\subsubsection*{\large{\textbf{Descrição}}\normalsize\hspace{1ex}\hfill}
\begin{itemize}
\item A class \textbf{\begin{ttfamily}TStreamBase\end{ttfamily}} é uma classe abstrata para implementação de streams.
\end{itemize}\subsubsection*{\large{\textbf{Campos}}\normalsize\hspace{1ex}\hfill}
\paragraph*{Status}\hspace*{\fill}

\begin{list}{}{
\settowidth{\tmplength}{\textbf{Declaração}}
\setlength{\itemindent}{0cm}
\setlength{\listparindent}{0cm}
\setlength{\leftmargin}{\evensidemargin}
\addtolength{\leftmargin}{\tmplength}
\settowidth{\labelsep}{X}
\addtolength{\leftmargin}{\labelsep}
\setlength{\labelwidth}{\tmplength}
}
\begin{flushleft}
\item[\textbf{Declaração}\hfill]
\begin{ttfamily}
public Status: Integer;\end{ttfamily}


\end{flushleft}
\par
\item[\textbf{Descrição}]
Stream \begin{ttfamily}status\end{ttfamily}

\end{list}
\paragraph*{StreamSize}\hspace*{\fill}

\begin{list}{}{
\settowidth{\tmplength}{\textbf{Declaração}}
\setlength{\itemindent}{0cm}
\setlength{\listparindent}{0cm}
\setlength{\leftmargin}{\evensidemargin}
\addtolength{\leftmargin}{\tmplength}
\settowidth{\labelsep}{X}
\addtolength{\leftmargin}{\labelsep}
\setlength{\labelwidth}{\tmplength}
}
\begin{flushleft}
\item[\textbf{Declaração}\hfill]
\begin{ttfamily}
public StreamSize: int64;\end{ttfamily}


\end{flushleft}
\par
\item[\textbf{Descrição}]
Stream current size

\end{list}
\paragraph*{Position}\hspace*{\fill}

\begin{list}{}{
\settowidth{\tmplength}{\textbf{Declaração}}
\setlength{\itemindent}{0cm}
\setlength{\listparindent}{0cm}
\setlength{\leftmargin}{\evensidemargin}
\addtolength{\leftmargin}{\tmplength}
\settowidth{\labelsep}{X}
\addtolength{\leftmargin}{\labelsep}
\setlength{\labelwidth}{\tmplength}
}
\begin{flushleft}
\item[\textbf{Declaração}\hfill]
\begin{ttfamily}
public Position: Int64;\end{ttfamily}


\end{flushleft}
\par
\item[\textbf{Descrição}]
Current \begin{ttfamily}position\end{ttfamily}

\end{list}
\paragraph*{Alias}\hspace*{\fill}

\begin{list}{}{
\settowidth{\tmplength}{\textbf{Declaração}}
\setlength{\itemindent}{0cm}
\setlength{\listparindent}{0cm}
\setlength{\leftmargin}{\evensidemargin}
\addtolength{\leftmargin}{\tmplength}
\settowidth{\labelsep}{X}
\addtolength{\leftmargin}{\labelsep}
\setlength{\labelwidth}{\tmplength}
}
\begin{flushleft}
\item[\textbf{Declaração}\hfill]
\begin{ttfamily}
public Alias: AnsiString;\end{ttfamily}


\end{flushleft}
\end{list}
\subsubsection*{\large{\textbf{Métodos}}\normalsize\hspace{1ex}\hfill}
\paragraph*{Create}\hspace*{\fill}

\begin{list}{}{
\settowidth{\tmplength}{\textbf{Declaração}}
\setlength{\itemindent}{0cm}
\setlength{\listparindent}{0cm}
\setlength{\leftmargin}{\evensidemargin}
\addtolength{\leftmargin}{\tmplength}
\settowidth{\labelsep}{X}
\addtolength{\leftmargin}{\labelsep}
\setlength{\labelwidth}{\tmplength}
}
\begin{flushleft}
\item[\textbf{Declaração}\hfill]
\begin{ttfamily}
public constructor Create; overload; virtual;\end{ttfamily}


\end{flushleft}
\end{list}
\paragraph*{Destroy}\hspace*{\fill}

\begin{list}{}{
\settowidth{\tmplength}{\textbf{Declaração}}
\setlength{\itemindent}{0cm}
\setlength{\listparindent}{0cm}
\setlength{\leftmargin}{\evensidemargin}
\addtolength{\leftmargin}{\tmplength}
\settowidth{\labelsep}{X}
\addtolength{\leftmargin}{\labelsep}
\setlength{\labelwidth}{\tmplength}
}
\begin{flushleft}
\item[\textbf{Declaração}\hfill]
\begin{ttfamily}
public destructor Destroy; Override;\end{ttfamily}


\end{flushleft}
\end{list}
\paragraph*{Open}\hspace*{\fill}

\begin{list}{}{
\settowidth{\tmplength}{\textbf{Declaração}}
\setlength{\itemindent}{0cm}
\setlength{\listparindent}{0cm}
\setlength{\leftmargin}{\evensidemargin}
\addtolength{\leftmargin}{\tmplength}
\settowidth{\labelsep}{X}
\addtolength{\leftmargin}{\labelsep}
\setlength{\labelwidth}{\tmplength}
}
\begin{flushleft}
\item[\textbf{Declaração}\hfill]
\begin{ttfamily}
protected procedure Open; overload; Virtual;\end{ttfamily}


\end{flushleft}
\end{list}
\paragraph*{Close}\hspace*{\fill}

\begin{list}{}{
\settowidth{\tmplength}{\textbf{Declaração}}
\setlength{\itemindent}{0cm}
\setlength{\listparindent}{0cm}
\setlength{\leftmargin}{\evensidemargin}
\addtolength{\leftmargin}{\tmplength}
\settowidth{\labelsep}{X}
\addtolength{\leftmargin}{\labelsep}
\setlength{\labelwidth}{\tmplength}
}
\begin{flushleft}
\item[\textbf{Declaração}\hfill]
\begin{ttfamily}
public procedure Close; Virtual;\end{ttfamily}


\end{flushleft}
\end{list}
\paragraph*{Rewrite}\hspace*{\fill}

\begin{list}{}{
\settowidth{\tmplength}{\textbf{Declaração}}
\setlength{\itemindent}{0cm}
\setlength{\listparindent}{0cm}
\setlength{\leftmargin}{\evensidemargin}
\addtolength{\leftmargin}{\tmplength}
\settowidth{\labelsep}{X}
\addtolength{\leftmargin}{\labelsep}
\setlength{\labelwidth}{\tmplength}
}
\begin{flushleft}
\item[\textbf{Declaração}\hfill]
\begin{ttfamily}
protected procedure Rewrite; Overload; Virtual;\end{ttfamily}


\end{flushleft}
\end{list}
\paragraph*{Flush}\hspace*{\fill}

\begin{list}{}{
\settowidth{\tmplength}{\textbf{Declaração}}
\setlength{\itemindent}{0cm}
\setlength{\listparindent}{0cm}
\setlength{\leftmargin}{\evensidemargin}
\addtolength{\leftmargin}{\tmplength}
\settowidth{\labelsep}{X}
\addtolength{\leftmargin}{\labelsep}
\setlength{\labelwidth}{\tmplength}
}
\begin{flushleft}
\item[\textbf{Declaração}\hfill]
\begin{ttfamily}
protected procedure Flush; Virtual;\end{ttfamily}


\end{flushleft}
\end{list}
\paragraph*{Truncate}\hspace*{\fill}

\begin{list}{}{
\settowidth{\tmplength}{\textbf{Declaração}}
\setlength{\itemindent}{0cm}
\setlength{\listparindent}{0cm}
\setlength{\leftmargin}{\evensidemargin}
\addtolength{\leftmargin}{\tmplength}
\settowidth{\labelsep}{X}
\addtolength{\leftmargin}{\labelsep}
\setlength{\labelwidth}{\tmplength}
}
\begin{flushleft}
\item[\textbf{Declaração}\hfill]
\begin{ttfamily}
protected procedure Truncate; Overload; Virtual;\end{ttfamily}


\end{flushleft}
\end{list}
\paragraph*{Read}\hspace*{\fill}

\begin{list}{}{
\settowidth{\tmplength}{\textbf{Declaração}}
\setlength{\itemindent}{0cm}
\setlength{\listparindent}{0cm}
\setlength{\leftmargin}{\evensidemargin}
\addtolength{\leftmargin}{\tmplength}
\settowidth{\labelsep}{X}
\addtolength{\leftmargin}{\labelsep}
\setlength{\labelwidth}{\tmplength}
}
\begin{flushleft}
\item[\textbf{Declaração}\hfill]
\begin{ttfamily}
protected procedure Read(Var Buf; Count: Sw{\_}Word); Overload; Virtual;\end{ttfamily}


\end{flushleft}
\end{list}
\paragraph*{Write}\hspace*{\fill}

\begin{list}{}{
\settowidth{\tmplength}{\textbf{Declaração}}
\setlength{\itemindent}{0cm}
\setlength{\listparindent}{0cm}
\setlength{\leftmargin}{\evensidemargin}
\addtolength{\leftmargin}{\tmplength}
\settowidth{\labelsep}{X}
\addtolength{\leftmargin}{\labelsep}
\setlength{\labelwidth}{\tmplength}
}
\begin{flushleft}
\item[\textbf{Declaração}\hfill]
\begin{ttfamily}
public procedure Write(Var Buf; Count: Sw{\_}Word); Overload; Virtual;\end{ttfamily}


\end{flushleft}
\end{list}
\paragraph*{ReadStr}\hspace*{\fill}

\begin{list}{}{
\settowidth{\tmplength}{\textbf{Declaração}}
\setlength{\itemindent}{0cm}
\setlength{\listparindent}{0cm}
\setlength{\leftmargin}{\evensidemargin}
\addtolength{\leftmargin}{\tmplength}
\settowidth{\labelsep}{X}
\addtolength{\leftmargin}{\labelsep}
\setlength{\labelwidth}{\tmplength}
}
\begin{flushleft}
\item[\textbf{Declaração}\hfill]
\begin{ttfamily}
public function ReadStr: ptstring;\end{ttfamily}


\end{flushleft}
\end{list}
\paragraph*{Get}\hspace*{\fill}

\begin{list}{}{
\settowidth{\tmplength}{\textbf{Declaração}}
\setlength{\itemindent}{0cm}
\setlength{\listparindent}{0cm}
\setlength{\leftmargin}{\evensidemargin}
\addtolength{\leftmargin}{\tmplength}
\settowidth{\labelsep}{X}
\addtolength{\leftmargin}{\labelsep}
\setlength{\labelwidth}{\tmplength}
}
\begin{flushleft}
\item[\textbf{Declaração}\hfill]
\begin{ttfamily}
public function Get: TClass;\end{ttfamily}


\end{flushleft}
\end{list}
\paragraph*{StrRead}\hspace*{\fill}

\begin{list}{}{
\settowidth{\tmplength}{\textbf{Declaração}}
\setlength{\itemindent}{0cm}
\setlength{\listparindent}{0cm}
\setlength{\leftmargin}{\evensidemargin}
\addtolength{\leftmargin}{\tmplength}
\settowidth{\labelsep}{X}
\addtolength{\leftmargin}{\labelsep}
\setlength{\labelwidth}{\tmplength}
}
\begin{flushleft}
\item[\textbf{Declaração}\hfill]
\begin{ttfamily}
public function StrRead: PAnsiChar;\end{ttfamily}


\end{flushleft}
\end{list}
\paragraph*{Put}\hspace*{\fill}

\begin{list}{}{
\settowidth{\tmplength}{\textbf{Declaração}}
\setlength{\itemindent}{0cm}
\setlength{\listparindent}{0cm}
\setlength{\leftmargin}{\evensidemargin}
\addtolength{\leftmargin}{\tmplength}
\settowidth{\labelsep}{X}
\addtolength{\leftmargin}{\labelsep}
\setlength{\labelwidth}{\tmplength}
}
\begin{flushleft}
\item[\textbf{Declaração}\hfill]
\begin{ttfamily}
public procedure Put(P: TClass);\end{ttfamily}


\end{flushleft}
\end{list}
\paragraph*{StrWrite}\hspace*{\fill}

\begin{list}{}{
\settowidth{\tmplength}{\textbf{Declaração}}
\setlength{\itemindent}{0cm}
\setlength{\listparindent}{0cm}
\setlength{\leftmargin}{\evensidemargin}
\addtolength{\leftmargin}{\tmplength}
\settowidth{\labelsep}{X}
\addtolength{\leftmargin}{\labelsep}
\setlength{\labelwidth}{\tmplength}
}
\begin{flushleft}
\item[\textbf{Declaração}\hfill]
\begin{ttfamily}
public procedure StrWrite(P: PAnsiChar);\end{ttfamily}


\end{flushleft}
\end{list}
\paragraph*{WriteStr}\hspace*{\fill}

\begin{list}{}{
\settowidth{\tmplength}{\textbf{Declaração}}
\setlength{\itemindent}{0cm}
\setlength{\listparindent}{0cm}
\setlength{\leftmargin}{\evensidemargin}
\addtolength{\leftmargin}{\tmplength}
\settowidth{\labelsep}{X}
\addtolength{\leftmargin}{\labelsep}
\setlength{\labelwidth}{\tmplength}
}
\begin{flushleft}
\item[\textbf{Declaração}\hfill]
\begin{ttfamily}
public procedure WriteStr(P: ptstring);\end{ttfamily}


\end{flushleft}
\end{list}
\paragraph*{CopyFrom}\hspace*{\fill}

\begin{list}{}{
\settowidth{\tmplength}{\textbf{Declaração}}
\setlength{\itemindent}{0cm}
\setlength{\listparindent}{0cm}
\setlength{\leftmargin}{\evensidemargin}
\addtolength{\leftmargin}{\tmplength}
\settowidth{\labelsep}{X}
\addtolength{\leftmargin}{\labelsep}
\setlength{\labelwidth}{\tmplength}
}
\begin{flushleft}
\item[\textbf{Declaração}\hfill]
\begin{ttfamily}
public procedure CopyFrom(Var S: TStreamBase; Count: LongInt);\end{ttfamily}


\end{flushleft}
\end{list}
\paragraph*{GetPos}\hspace*{\fill}

\begin{list}{}{
\settowidth{\tmplength}{\textbf{Declaração}}
\setlength{\itemindent}{0cm}
\setlength{\listparindent}{0cm}
\setlength{\leftmargin}{\evensidemargin}
\addtolength{\leftmargin}{\tmplength}
\settowidth{\labelsep}{X}
\addtolength{\leftmargin}{\labelsep}
\setlength{\labelwidth}{\tmplength}
}
\begin{flushleft}
\item[\textbf{Declaração}\hfill]
\begin{ttfamily}
public function GetPos: LongInt; Virtual;\end{ttfamily}


\end{flushleft}
\end{list}
\paragraph*{GetSize}\hspace*{\fill}

\begin{list}{}{
\settowidth{\tmplength}{\textbf{Declaração}}
\setlength{\itemindent}{0cm}
\setlength{\listparindent}{0cm}
\setlength{\leftmargin}{\evensidemargin}
\addtolength{\leftmargin}{\tmplength}
\settowidth{\labelsep}{X}
\addtolength{\leftmargin}{\labelsep}
\setlength{\labelwidth}{\tmplength}
}
\begin{flushleft}
\item[\textbf{Declaração}\hfill]
\begin{ttfamily}
public function GetSize: LongInt; Virtual;\end{ttfamily}


\end{flushleft}
\end{list}
\paragraph*{Reset}\hspace*{\fill}

\begin{list}{}{
\settowidth{\tmplength}{\textbf{Declaração}}
\setlength{\itemindent}{0cm}
\setlength{\listparindent}{0cm}
\setlength{\leftmargin}{\evensidemargin}
\addtolength{\leftmargin}{\tmplength}
\settowidth{\labelsep}{X}
\addtolength{\leftmargin}{\labelsep}
\setlength{\labelwidth}{\tmplength}
}
\begin{flushleft}
\item[\textbf{Declaração}\hfill]
\begin{ttfamily}
public procedure Reset; Overload; Virtual;\end{ttfamily}


\end{flushleft}
\end{list}
\paragraph*{Seek}\hspace*{\fill}

\begin{list}{}{
\settowidth{\tmplength}{\textbf{Declaração}}
\setlength{\itemindent}{0cm}
\setlength{\listparindent}{0cm}
\setlength{\leftmargin}{\evensidemargin}
\addtolength{\leftmargin}{\tmplength}
\settowidth{\labelsep}{X}
\addtolength{\leftmargin}{\labelsep}
\setlength{\labelwidth}{\tmplength}
}
\begin{flushleft}
\item[\textbf{Declaração}\hfill]
\begin{ttfamily}
public procedure Seek(Pos: LongInt); overload; Virtual;\end{ttfamily}


\end{flushleft}
\end{list}
\paragraph*{Seek}\hspace*{\fill}

\begin{list}{}{
\settowidth{\tmplength}{\textbf{Declaração}}
\setlength{\itemindent}{0cm}
\setlength{\listparindent}{0cm}
\setlength{\leftmargin}{\evensidemargin}
\addtolength{\leftmargin}{\tmplength}
\settowidth{\labelsep}{X}
\addtolength{\leftmargin}{\labelsep}
\setlength{\labelwidth}{\tmplength}
}
\begin{flushleft}
\item[\textbf{Declaração}\hfill]
\begin{ttfamily}
public procedure Seek(NR: LongInt;a{\_}RecSize:Longint); Overload; Virtual;\end{ttfamily}


\end{flushleft}
\end{list}
\paragraph*{Error}\hspace*{\fill}

\begin{list}{}{
\settowidth{\tmplength}{\textbf{Declaração}}
\setlength{\itemindent}{0cm}
\setlength{\listparindent}{0cm}
\setlength{\leftmargin}{\evensidemargin}
\addtolength{\leftmargin}{\tmplength}
\settowidth{\labelsep}{X}
\addtolength{\leftmargin}{\labelsep}
\setlength{\labelwidth}{\tmplength}
}
\begin{flushleft}
\item[\textbf{Declaração}\hfill]
\begin{ttfamily}
public procedure Error(Code, Info: Integer); Virtual;\end{ttfamily}


\end{flushleft}
\end{list}
\paragraph*{GetDriveType}\hspace*{\fill}

\begin{list}{}{
\settowidth{\tmplength}{\textbf{Declaração}}
\setlength{\itemindent}{0cm}
\setlength{\listparindent}{0cm}
\setlength{\leftmargin}{\evensidemargin}
\addtolength{\leftmargin}{\tmplength}
\settowidth{\labelsep}{X}
\addtolength{\leftmargin}{\labelsep}
\setlength{\labelwidth}{\tmplength}
}
\begin{flushleft}
\item[\textbf{Declaração}\hfill]
\begin{ttfamily}
public function GetDriveType:TDriveType; overload; virtual;\end{ttfamily}


\end{flushleft}
\end{list}
\chapter{Unit mi.rtl.Objects.Methods.StreamBase.Stream}
\section{Descrição}
\begin{itemize}
\item A Unit \textbf{\begin{ttfamily}mi.rtl.Objects.Methods.StreamBase.Stream\end{ttfamily}} implementa a classe \begin{ttfamily}TStream\end{ttfamily}(\ref{mi.rtl.Objects.Methods.StreamBase.Stream.TStream}) do pacote \textbf{\begin{ttfamily}mi.rtl\end{ttfamily}(\ref{mi.rtl})}.

\begin{itemize}
\item \textbf{NOTAS} \begin{itemize}
\item Está unit foi testada nas plataformas: win32, win64 e linux.
\item Como o linux não tem opção de travar a região de uma arquivo eu removi as classes \textbf{{\_}TRecLock} e \textbf{TCollRecsLocks}.
\end{itemize}
\item \textbf{VERSÃO} \begin{itemize}
\item Alpha {-} 0.5.0.687
\end{itemize}
\item \textbf{HISTÓRICO} \begin{itemize}
\item Criado por: Paulo Sérgio da Silva Pacheco e{-}mail: paulosspacheco@yahoo.com.br \begin{itemize}
\item \textbf{20/11/2021} {-} 09:10 a ??:?? Criar a unit mi.rtl.objects.methods.StreamBase.Stream.pas
\item \textbf{22/11/2021} \begin{itemize}
\item 09:44 a 12:05 Adaptar \textbf{\begin{ttfamily}TStream\end{ttfamily}(\ref{mi.rtl.Objects.Methods.StreamBase.Stream.TStream})} ao free pascal;
\item 14:10 a 19:05 Adaptar \textbf{{\_}TStream} e \textbf{\begin{ttfamily}TStream\end{ttfamily}(\ref{mi.rtl.Objects.Methods.StreamBase.Stream.TStream})} ao free pascal;
\end{itemize}
\end{itemize} {-}
\end{itemize}
\item \textbf{CÓDIGO FONTE}: \begin{itemize}
\item 
\end{itemize}
\end{itemize}
\end{itemize}
\section{Uses}
\begin{itemize}
\item \begin{ttfamily}Classes\end{ttfamily}\item \begin{ttfamily}SysUtils\end{ttfamily}\item \begin{ttfamily}mi.rtl.types\end{ttfamily}(\ref{mi.rtl.Types})\item \begin{ttfamily}mi.rtl.consts\end{ttfamily}(\ref{mi.rtl.Consts})\item \begin{ttfamily}mi.rtl.files\end{ttfamily}(\ref{mi.rtl.files})\item \begin{ttfamily}mi.rtl.objects.types\end{ttfamily}(\ref{mi.rtl.objects.types})\item \begin{ttfamily}mi.rtl.objects.Methods\end{ttfamily}(\ref{mi.rtl.Objects.Methods})\item \begin{ttfamily}mi.rtl.objects.methods.StreamBase\end{ttfamily}(\ref{mi.rtl.Objects.Methods.StreamBase})\end{itemize}
\section{Visão Geral}
\begin{description}
\item[\texttt{\begin{ttfamily}TStream\end{ttfamily} Classe}]
\end{description}
\section{Classes, Interfaces, Objetos e Registros}
\subsection*{TStream Classe}
\subsubsection*{\large{\textbf{Hierarquia}}\normalsize\hspace{1ex}\hfill}
TStream {$>$} \begin{ttfamily}TStreamBase\end{ttfamily}(\ref{mi.rtl.Objects.Methods.StreamBase.TStreamBase}) {$>$} \begin{ttfamily}TObjectsMethods\end{ttfamily}(\ref{mi.rtl.Objects.Methods.TObjectsMethods}) {$>$} \begin{ttfamily}TObjectsConsts\end{ttfamily}(\ref{mi.rtl.Objects.Consts.TObjectsConsts}) {$>$} 
TObjectsTypes
\subsubsection*{\large{\textbf{Descrição}}\normalsize\hspace{1ex}\hfill}
\begin{itemize}
\item A class \textbf{\begin{ttfamily}TStream\end{ttfamily}} é a classe base da classes \textbf{{\_}TStream} do pacote \textbf{\begin{ttfamily}mi.rtl\end{ttfamily}(\ref{mi.rtl})}.
\end{itemize}\subsubsection*{\large{\textbf{Propriedades}}\normalsize\hspace{1ex}\hfill}
\paragraph*{BaseSize}\hspace*{\fill}

\begin{list}{}{
\settowidth{\tmplength}{\textbf{Declaração}}
\setlength{\itemindent}{0cm}
\setlength{\listparindent}{0cm}
\setlength{\leftmargin}{\evensidemargin}
\addtolength{\leftmargin}{\tmplength}
\settowidth{\labelsep}{X}
\addtolength{\leftmargin}{\labelsep}
\setlength{\labelwidth}{\tmplength}
}
\begin{flushleft}
\item[\textbf{Declaração}\hfill]
\begin{ttfamily}
published property BaseSize: Longint  Read {\_}BaseSize write SetBaseSize;\end{ttfamily}


\end{flushleft}
\end{list}
\paragraph*{RecSize}\hspace*{\fill}

\begin{list}{}{
\settowidth{\tmplength}{\textbf{Declaração}}
\setlength{\itemindent}{0cm}
\setlength{\listparindent}{0cm}
\setlength{\leftmargin}{\evensidemargin}
\addtolength{\leftmargin}{\tmplength}
\settowidth{\labelsep}{X}
\addtolength{\leftmargin}{\labelsep}
\setlength{\labelwidth}{\tmplength}
}
\begin{flushleft}
\item[\textbf{Declaração}\hfill]
\begin{ttfamily}
published property RecSize: Longint  Read {\_}RecSize write SetRecSize;\end{ttfamily}


\end{flushleft}
\end{list}
\paragraph*{Ok{\_}Aguarde}\hspace*{\fill}

\begin{list}{}{
\settowidth{\tmplength}{\textbf{Declaração}}
\setlength{\itemindent}{0cm}
\setlength{\listparindent}{0cm}
\setlength{\leftmargin}{\evensidemargin}
\addtolength{\leftmargin}{\tmplength}
\settowidth{\labelsep}{X}
\addtolength{\leftmargin}{\labelsep}
\setlength{\labelwidth}{\tmplength}
}
\begin{flushleft}
\item[\textbf{Declaração}\hfill]
\begin{ttfamily}
published property Ok{\_}Aguarde: Boolean Read {\_}Ok{\_}Aguarde write Set{\_}Ok{\_}Aguarde;\end{ttfamily}


\end{flushleft}
\end{list}
\paragraph*{FileMode}\hspace*{\fill}

\begin{list}{}{
\settowidth{\tmplength}{\textbf{Declaração}}
\setlength{\itemindent}{0cm}
\setlength{\listparindent}{0cm}
\setlength{\leftmargin}{\evensidemargin}
\addtolength{\leftmargin}{\tmplength}
\settowidth{\labelsep}{X}
\addtolength{\leftmargin}{\labelsep}
\setlength{\labelwidth}{\tmplength}
}
\begin{flushleft}
\item[\textbf{Declaração}\hfill]
\begin{ttfamily}
published property FileMode : Word Read {\_}FileMode write SetFileMode;\end{ttfamily}


\end{flushleft}
\end{list}
\paragraph*{ShareMode}\hspace*{\fill}

\begin{list}{}{
\settowidth{\tmplength}{\textbf{Declaração}}
\setlength{\itemindent}{0cm}
\setlength{\listparindent}{0cm}
\setlength{\leftmargin}{\evensidemargin}
\addtolength{\leftmargin}{\tmplength}
\settowidth{\labelsep}{X}
\addtolength{\leftmargin}{\labelsep}
\setlength{\labelwidth}{\tmplength}
}
\begin{flushleft}
\item[\textbf{Declaração}\hfill]
\begin{ttfamily}
published property ShareMode : Cardinal read {\_}ShareMode write SetShareMode;\end{ttfamily}


\end{flushleft}
\end{list}
\paragraph*{FileName}\hspace*{\fill}

\begin{list}{}{
\settowidth{\tmplength}{\textbf{Declaração}}
\setlength{\itemindent}{0cm}
\setlength{\listparindent}{0cm}
\setlength{\leftmargin}{\evensidemargin}
\addtolength{\leftmargin}{\tmplength}
\settowidth{\labelsep}{X}
\addtolength{\leftmargin}{\labelsep}
\setlength{\labelwidth}{\tmplength}
}
\begin{flushleft}
\item[\textbf{Declaração}\hfill]
\begin{ttfamily}
published property FileName   : AnsiString read GetFileName write SetFileName;\end{ttfamily}


\end{flushleft}
\end{list}
\subsubsection*{\large{\textbf{Campos}}\normalsize\hspace{1ex}\hfill}
\paragraph*{{\_}Base}\hspace*{\fill}

\begin{list}{}{
\settowidth{\tmplength}{\textbf{Declaração}}
\setlength{\itemindent}{0cm}
\setlength{\listparindent}{0cm}
\setlength{\leftmargin}{\evensidemargin}
\addtolength{\leftmargin}{\tmplength}
\settowidth{\labelsep}{X}
\addtolength{\leftmargin}{\labelsep}
\setlength{\labelwidth}{\tmplength}
}
\begin{flushleft}
\item[\textbf{Declaração}\hfill]
\begin{ttfamily}
protected {\_}Base: Pointer;\end{ttfamily}


\end{flushleft}
\end{list}
\paragraph*{{\_}Rec}\hspace*{\fill}

\begin{list}{}{
\settowidth{\tmplength}{\textbf{Declaração}}
\setlength{\itemindent}{0cm}
\setlength{\listparindent}{0cm}
\setlength{\leftmargin}{\evensidemargin}
\addtolength{\leftmargin}{\tmplength}
\settowidth{\labelsep}{X}
\addtolength{\leftmargin}{\labelsep}
\setlength{\labelwidth}{\tmplength}
}
\begin{flushleft}
\item[\textbf{Declaração}\hfill]
\begin{ttfamily}
protected {\_}Rec: Pointer;\end{ttfamily}


\end{flushleft}
\end{list}
\paragraph*{Status{\_}Rewrite}\hspace*{\fill}

\begin{list}{}{
\settowidth{\tmplength}{\textbf{Declaração}}
\setlength{\itemindent}{0cm}
\setlength{\listparindent}{0cm}
\setlength{\leftmargin}{\evensidemargin}
\addtolength{\leftmargin}{\tmplength}
\settowidth{\labelsep}{X}
\addtolength{\leftmargin}{\labelsep}
\setlength{\labelwidth}{\tmplength}
}
\begin{flushleft}
\item[\textbf{Declaração}\hfill]
\begin{ttfamily}
public Status{\_}Rewrite: Byte;\end{ttfamily}


\end{flushleft}
\end{list}
\paragraph*{ClockBegin}\hspace*{\fill}

\begin{list}{}{
\settowidth{\tmplength}{\textbf{Declaração}}
\setlength{\itemindent}{0cm}
\setlength{\listparindent}{0cm}
\setlength{\leftmargin}{\evensidemargin}
\addtolength{\leftmargin}{\tmplength}
\settowidth{\labelsep}{X}
\addtolength{\leftmargin}{\labelsep}
\setlength{\labelwidth}{\tmplength}
}
\begin{flushleft}
\item[\textbf{Declaração}\hfill]
\begin{ttfamily}
public ClockBegin: DWord;\end{ttfamily}


\end{flushleft}
\end{list}
\paragraph*{Last{\_}Mode}\hspace*{\fill}

\begin{list}{}{
\settowidth{\tmplength}{\textbf{Declaração}}
\setlength{\itemindent}{0cm}
\setlength{\listparindent}{0cm}
\setlength{\leftmargin}{\evensidemargin}
\addtolength{\leftmargin}{\tmplength}
\settowidth{\labelsep}{X}
\addtolength{\leftmargin}{\labelsep}
\setlength{\labelwidth}{\tmplength}
}
\begin{flushleft}
\item[\textbf{Declaração}\hfill]
\begin{ttfamily}
public Last{\_}Mode: TLast{\_}Mode{\_}Read{\_}Write;\end{ttfamily}


\end{flushleft}
\end{list}
\paragraph*{State}\hspace*{\fill}

\begin{list}{}{
\settowidth{\tmplength}{\textbf{Declaração}}
\setlength{\itemindent}{0cm}
\setlength{\listparindent}{0cm}
\setlength{\leftmargin}{\evensidemargin}
\addtolength{\leftmargin}{\tmplength}
\settowidth{\labelsep}{X}
\addtolength{\leftmargin}{\labelsep}
\setlength{\labelwidth}{\tmplength}
}
\begin{flushleft}
\item[\textbf{Declaração}\hfill]
\begin{ttfamily}
public var State: Longint;\end{ttfamily}


\end{flushleft}
\end{list}
\paragraph*{Ok{\_}FreeMem{\_}Rec}\hspace*{\fill}

\begin{list}{}{
\settowidth{\tmplength}{\textbf{Declaração}}
\setlength{\itemindent}{0cm}
\setlength{\listparindent}{0cm}
\setlength{\leftmargin}{\evensidemargin}
\addtolength{\leftmargin}{\tmplength}
\settowidth{\labelsep}{X}
\addtolength{\leftmargin}{\labelsep}
\setlength{\labelwidth}{\tmplength}
}
\begin{flushleft}
\item[\textbf{Declaração}\hfill]
\begin{ttfamily}
protected Ok{\_}FreeMem{\_}Rec:Boolean;\end{ttfamily}


\end{flushleft}
\end{list}
\paragraph*{{\_}FileName}\hspace*{\fill}

\begin{list}{}{
\settowidth{\tmplength}{\textbf{Declaração}}
\setlength{\itemindent}{0cm}
\setlength{\listparindent}{0cm}
\setlength{\leftmargin}{\evensidemargin}
\addtolength{\leftmargin}{\tmplength}
\settowidth{\labelsep}{X}
\addtolength{\leftmargin}{\labelsep}
\setlength{\labelwidth}{\tmplength}
}
\begin{flushleft}
\item[\textbf{Declaração}\hfill]
\begin{ttfamily}
protected {\_}FileName: AnsiString;\end{ttfamily}


\end{flushleft}
\end{list}
\subsubsection*{\large{\textbf{Métodos}}\normalsize\hspace{1ex}\hfill}
\paragraph*{Set{\_}BaseSize}\hspace*{\fill}

\begin{list}{}{
\settowidth{\tmplength}{\textbf{Declaração}}
\setlength{\itemindent}{0cm}
\setlength{\listparindent}{0cm}
\setlength{\leftmargin}{\evensidemargin}
\addtolength{\leftmargin}{\tmplength}
\settowidth{\labelsep}{X}
\addtolength{\leftmargin}{\labelsep}
\setlength{\labelwidth}{\tmplength}
}
\begin{flushleft}
\item[\textbf{Declaração}\hfill]
\begin{ttfamily}
public procedure Set{\_}BaseSize(a{\_}Base : Pointer;a{\_}BaseSize:Longint); Overload; Virtual;\end{ttfamily}


\end{flushleft}
\end{list}
\paragraph*{SetBaseSize}\hspace*{\fill}

\begin{list}{}{
\settowidth{\tmplength}{\textbf{Declaração}}
\setlength{\itemindent}{0cm}
\setlength{\listparindent}{0cm}
\setlength{\leftmargin}{\evensidemargin}
\addtolength{\leftmargin}{\tmplength}
\settowidth{\labelsep}{X}
\addtolength{\leftmargin}{\labelsep}
\setlength{\labelwidth}{\tmplength}
}
\begin{flushleft}
\item[\textbf{Declaração}\hfill]
\begin{ttfamily}
protected procedure SetBaseSize(a{\_}BaseSize : Longint); Overload; Virtual;\end{ttfamily}


\end{flushleft}
\end{list}
\paragraph*{Set{\_}RecSize}\hspace*{\fill}

\begin{list}{}{
\settowidth{\tmplength}{\textbf{Declaração}}
\setlength{\itemindent}{0cm}
\setlength{\listparindent}{0cm}
\setlength{\leftmargin}{\evensidemargin}
\addtolength{\leftmargin}{\tmplength}
\settowidth{\labelsep}{X}
\addtolength{\leftmargin}{\labelsep}
\setlength{\labelwidth}{\tmplength}
}
\begin{flushleft}
\item[\textbf{Declaração}\hfill]
\begin{ttfamily}
public procedure Set{\_}RecSize(a{\_}Rec : Pointer;a{\_}RecSize:Longint); Overload; Virtual;\end{ttfamily}


\end{flushleft}
\end{list}
\paragraph*{SetRecSize}\hspace*{\fill}

\begin{list}{}{
\settowidth{\tmplength}{\textbf{Declaração}}
\setlength{\itemindent}{0cm}
\setlength{\listparindent}{0cm}
\setlength{\leftmargin}{\evensidemargin}
\addtolength{\leftmargin}{\tmplength}
\settowidth{\labelsep}{X}
\addtolength{\leftmargin}{\labelsep}
\setlength{\labelwidth}{\tmplength}
}
\begin{flushleft}
\item[\textbf{Declaração}\hfill]
\begin{ttfamily}
protected procedure SetRecSize(a{\_}RecSize : Longint); Overload; Virtual;\end{ttfamily}


\end{flushleft}
\end{list}
\paragraph*{Calc{\_}Pos}\hspace*{\fill}

\begin{list}{}{
\settowidth{\tmplength}{\textbf{Declaração}}
\setlength{\itemindent}{0cm}
\setlength{\listparindent}{0cm}
\setlength{\leftmargin}{\evensidemargin}
\addtolength{\leftmargin}{\tmplength}
\settowidth{\labelsep}{X}
\addtolength{\leftmargin}{\labelsep}
\setlength{\labelwidth}{\tmplength}
}
\begin{flushleft}
\item[\textbf{Declaração}\hfill]
\begin{ttfamily}
public Function Calc{\_}Pos(NR: LongInt;a{\_}RecSize:Longint):Longint;\end{ttfamily}


\end{flushleft}
\end{list}
\paragraph*{FileSize}\hspace*{\fill}

\begin{list}{}{
\settowidth{\tmplength}{\textbf{Declaração}}
\setlength{\itemindent}{0cm}
\setlength{\listparindent}{0cm}
\setlength{\leftmargin}{\evensidemargin}
\addtolength{\leftmargin}{\tmplength}
\settowidth{\labelsep}{X}
\addtolength{\leftmargin}{\labelsep}
\setlength{\labelwidth}{\tmplength}
}
\begin{flushleft}
\item[\textbf{Declaração}\hfill]
\begin{ttfamily}
public function FileSize: Longint; overload; Virtual;\end{ttfamily}


\end{flushleft}
\end{list}
\paragraph*{Seek}\hspace*{\fill}

\begin{list}{}{
\settowidth{\tmplength}{\textbf{Declaração}}
\setlength{\itemindent}{0cm}
\setlength{\listparindent}{0cm}
\setlength{\leftmargin}{\evensidemargin}
\addtolength{\leftmargin}{\tmplength}
\settowidth{\labelsep}{X}
\addtolength{\leftmargin}{\labelsep}
\setlength{\labelwidth}{\tmplength}
}
\begin{flushleft}
\item[\textbf{Declaração}\hfill]
\begin{ttfamily}
public procedure Seek(NR: LongInt;a{\_}RecSize:Longint); Overload; override;\end{ttfamily}


\end{flushleft}
\end{list}
\paragraph*{Create}\hspace*{\fill}

\begin{list}{}{
\settowidth{\tmplength}{\textbf{Declaração}}
\setlength{\itemindent}{0cm}
\setlength{\listparindent}{0cm}
\setlength{\leftmargin}{\evensidemargin}
\addtolength{\leftmargin}{\tmplength}
\settowidth{\labelsep}{X}
\addtolength{\leftmargin}{\labelsep}
\setlength{\labelwidth}{\tmplength}
}
\begin{flushleft}
\item[\textbf{Declaração}\hfill]
\begin{ttfamily}
public constructor Create(); overload; override;\end{ttfamily}


\end{flushleft}
\end{list}
\paragraph*{Destroy}\hspace*{\fill}

\begin{list}{}{
\settowidth{\tmplength}{\textbf{Declaração}}
\setlength{\itemindent}{0cm}
\setlength{\listparindent}{0cm}
\setlength{\leftmargin}{\evensidemargin}
\addtolength{\leftmargin}{\tmplength}
\settowidth{\labelsep}{X}
\addtolength{\leftmargin}{\labelsep}
\setlength{\labelwidth}{\tmplength}
}
\begin{flushleft}
\item[\textbf{Declaração}\hfill]
\begin{ttfamily}
public destructor Destroy; Override;\end{ttfamily}


\end{flushleft}
\end{list}
\paragraph*{Set{\_}Ok{\_}Aguarde}\hspace*{\fill}

\begin{list}{}{
\settowidth{\tmplength}{\textbf{Declaração}}
\setlength{\itemindent}{0cm}
\setlength{\listparindent}{0cm}
\setlength{\leftmargin}{\evensidemargin}
\addtolength{\leftmargin}{\tmplength}
\settowidth{\labelsep}{X}
\addtolength{\leftmargin}{\labelsep}
\setlength{\labelwidth}{\tmplength}
}
\begin{flushleft}
\item[\textbf{Declaração}\hfill]
\begin{ttfamily}
protected procedure Set{\_}Ok{\_}Aguarde(a{\_}Ok{\_}Aguarde: Boolean); Virtual;\end{ttfamily}


\end{flushleft}
\end{list}
\paragraph*{CloseOpen}\hspace*{\fill}

\begin{list}{}{
\settowidth{\tmplength}{\textbf{Declaração}}
\setlength{\itemindent}{0cm}
\setlength{\listparindent}{0cm}
\setlength{\leftmargin}{\evensidemargin}
\addtolength{\leftmargin}{\tmplength}
\settowidth{\labelsep}{X}
\addtolength{\leftmargin}{\labelsep}
\setlength{\labelwidth}{\tmplength}
}
\begin{flushleft}
\item[\textbf{Declaração}\hfill]
\begin{ttfamily}
public function CloseOpen:Integer; VIRTUAL;\end{ttfamily}


\end{flushleft}
\end{list}
\paragraph*{Flush{\_}Disk}\hspace*{\fill}

\begin{list}{}{
\settowidth{\tmplength}{\textbf{Declaração}}
\setlength{\itemindent}{0cm}
\setlength{\listparindent}{0cm}
\setlength{\leftmargin}{\evensidemargin}
\addtolength{\leftmargin}{\tmplength}
\settowidth{\labelsep}{X}
\addtolength{\leftmargin}{\labelsep}
\setlength{\labelwidth}{\tmplength}
}
\begin{flushleft}
\item[\textbf{Declaração}\hfill]
\begin{ttfamily}
public function Flush{\_}Disk:Integer; Virtual;\end{ttfamily}


\end{flushleft}
\end{list}
\paragraph*{Flush}\hspace*{\fill}

\begin{list}{}{
\settowidth{\tmplength}{\textbf{Declaração}}
\setlength{\itemindent}{0cm}
\setlength{\listparindent}{0cm}
\setlength{\leftmargin}{\evensidemargin}
\addtolength{\leftmargin}{\tmplength}
\settowidth{\labelsep}{X}
\addtolength{\leftmargin}{\labelsep}
\setlength{\labelwidth}{\tmplength}
}
\begin{flushleft}
\item[\textbf{Declaração}\hfill]
\begin{ttfamily}
public procedure Flush; Override;\end{ttfamily}


\end{flushleft}
\end{list}
\paragraph*{Read}\hspace*{\fill}

\begin{list}{}{
\settowidth{\tmplength}{\textbf{Declaração}}
\setlength{\itemindent}{0cm}
\setlength{\listparindent}{0cm}
\setlength{\leftmargin}{\evensidemargin}
\addtolength{\leftmargin}{\tmplength}
\settowidth{\labelsep}{X}
\addtolength{\leftmargin}{\labelsep}
\setlength{\labelwidth}{\tmplength}
}
\begin{flushleft}
\item[\textbf{Declaração}\hfill]
\begin{ttfamily}
public procedure Read(Var Buf; Count: Sw{\_}Word;Var BytesRead:Sw{\_}Word) ; Overload; Virtual;\end{ttfamily}


\end{flushleft}
\end{list}
\paragraph*{Write}\hspace*{\fill}

\begin{list}{}{
\settowidth{\tmplength}{\textbf{Declaração}}
\setlength{\itemindent}{0cm}
\setlength{\listparindent}{0cm}
\setlength{\leftmargin}{\evensidemargin}
\addtolength{\leftmargin}{\tmplength}
\settowidth{\labelsep}{X}
\addtolength{\leftmargin}{\labelsep}
\setlength{\labelwidth}{\tmplength}
}
\begin{flushleft}
\item[\textbf{Declaração}\hfill]
\begin{ttfamily}
public procedure Write(Var Buf; Count: Sw{\_}Word;Var BytesWrite:Sw{\_}Word); Overload; Virtual;\end{ttfamily}


\end{flushleft}
\end{list}
\paragraph*{SetFileMode}\hspace*{\fill}

\begin{list}{}{
\settowidth{\tmplength}{\textbf{Declaração}}
\setlength{\itemindent}{0cm}
\setlength{\listparindent}{0cm}
\setlength{\leftmargin}{\evensidemargin}
\addtolength{\leftmargin}{\tmplength}
\settowidth{\labelsep}{X}
\addtolength{\leftmargin}{\labelsep}
\setlength{\labelwidth}{\tmplength}
}
\begin{flushleft}
\item[\textbf{Declaração}\hfill]
\begin{ttfamily}
public procedure SetFileMode(Const aFileMode:Word); virtual;\end{ttfamily}


\end{flushleft}
\end{list}
\paragraph*{SetShareMode}\hspace*{\fill}

\begin{list}{}{
\settowidth{\tmplength}{\textbf{Declaração}}
\setlength{\itemindent}{0cm}
\setlength{\listparindent}{0cm}
\setlength{\leftmargin}{\evensidemargin}
\addtolength{\leftmargin}{\tmplength}
\settowidth{\labelsep}{X}
\addtolength{\leftmargin}{\labelsep}
\setlength{\labelwidth}{\tmplength}
}
\begin{flushleft}
\item[\textbf{Declaração}\hfill]
\begin{ttfamily}
public procedure SetShareMode(Const aShareMode:Cardinal); virtual;\end{ttfamily}


\end{flushleft}
\end{list}
\paragraph*{SetStateFileMode}\hspace*{\fill}

\begin{list}{}{
\settowidth{\tmplength}{\textbf{Declaração}}
\setlength{\itemindent}{0cm}
\setlength{\listparindent}{0cm}
\setlength{\leftmargin}{\evensidemargin}
\addtolength{\leftmargin}{\tmplength}
\settowidth{\labelsep}{X}
\addtolength{\leftmargin}{\labelsep}
\setlength{\labelwidth}{\tmplength}
}
\begin{flushleft}
\item[\textbf{Declaração}\hfill]
\begin{ttfamily}
public function SetStateFileMode(Const AState: Longint; Const Enable: boolean):Boolean;\end{ttfamily}


\end{flushleft}
\end{list}
\paragraph*{GetStateFileMode}\hspace*{\fill}

\begin{list}{}{
\settowidth{\tmplength}{\textbf{Declaração}}
\setlength{\itemindent}{0cm}
\setlength{\listparindent}{0cm}
\setlength{\leftmargin}{\evensidemargin}
\addtolength{\leftmargin}{\tmplength}
\settowidth{\labelsep}{X}
\addtolength{\leftmargin}{\labelsep}
\setlength{\labelwidth}{\tmplength}
}
\begin{flushleft}
\item[\textbf{Declaração}\hfill]
\begin{ttfamily}
public function GetStateFileMode(Const AState: Longint): Boolean;\end{ttfamily}


\end{flushleft}
\end{list}
\paragraph*{Reset}\hspace*{\fill}

\begin{list}{}{
\settowidth{\tmplength}{\textbf{Declaração}}
\setlength{\itemindent}{0cm}
\setlength{\listparindent}{0cm}
\setlength{\leftmargin}{\evensidemargin}
\addtolength{\leftmargin}{\tmplength}
\settowidth{\labelsep}{X}
\addtolength{\leftmargin}{\labelsep}
\setlength{\labelwidth}{\tmplength}
}
\begin{flushleft}
\item[\textbf{Declaração}\hfill]
\begin{ttfamily}
public procedure Reset; overload; Override;\end{ttfamily}


\end{flushleft}
\end{list}
\paragraph*{Reset}\hspace*{\fill}

\begin{list}{}{
\settowidth{\tmplength}{\textbf{Declaração}}
\setlength{\itemindent}{0cm}
\setlength{\listparindent}{0cm}
\setlength{\leftmargin}{\evensidemargin}
\addtolength{\leftmargin}{\tmplength}
\settowidth{\labelsep}{X}
\addtolength{\leftmargin}{\labelsep}
\setlength{\labelwidth}{\tmplength}
}
\begin{flushleft}
\item[\textbf{Declaração}\hfill]
\begin{ttfamily}
public procedure Reset(aFileMode: Word;ShareMode : Cardinal); overload; Virtual; abstract;\end{ttfamily}


\end{flushleft}
\end{list}
\paragraph*{Rewrite}\hspace*{\fill}

\begin{list}{}{
\settowidth{\tmplength}{\textbf{Declaração}}
\setlength{\itemindent}{0cm}
\setlength{\listparindent}{0cm}
\setlength{\leftmargin}{\evensidemargin}
\addtolength{\leftmargin}{\tmplength}
\settowidth{\labelsep}{X}
\addtolength{\leftmargin}{\labelsep}
\setlength{\labelwidth}{\tmplength}
}
\begin{flushleft}
\item[\textbf{Declaração}\hfill]
\begin{ttfamily}
public procedure Rewrite; overload; override;\end{ttfamily}


\end{flushleft}
\end{list}
\paragraph*{Rewrite}\hspace*{\fill}

\begin{list}{}{
\settowidth{\tmplength}{\textbf{Declaração}}
\setlength{\itemindent}{0cm}
\setlength{\listparindent}{0cm}
\setlength{\leftmargin}{\evensidemargin}
\addtolength{\leftmargin}{\tmplength}
\settowidth{\labelsep}{X}
\addtolength{\leftmargin}{\labelsep}
\setlength{\labelwidth}{\tmplength}
}
\begin{flushleft}
\item[\textbf{Declaração}\hfill]
\begin{ttfamily}
public procedure Rewrite(aFileMode: Word;ShareMode : Cardinal); Overload; Virtual; abstract;\end{ttfamily}


\end{flushleft}
\end{list}
\paragraph*{SetBufSize}\hspace*{\fill}

\begin{list}{}{
\settowidth{\tmplength}{\textbf{Declaração}}
\setlength{\itemindent}{0cm}
\setlength{\listparindent}{0cm}
\setlength{\leftmargin}{\evensidemargin}
\addtolength{\leftmargin}{\tmplength}
\settowidth{\labelsep}{X}
\addtolength{\leftmargin}{\labelsep}
\setlength{\labelwidth}{\tmplength}
}
\begin{flushleft}
\item[\textbf{Declaração}\hfill]
\begin{ttfamily}
public function SetBufSize(Const aBufSize : Sw{\_}Word):Sw{\_}Word; Overload; Virtual;\end{ttfamily}


\end{flushleft}
\end{list}
\paragraph*{IsFileOpen}\hspace*{\fill}

\begin{list}{}{
\settowidth{\tmplength}{\textbf{Declaração}}
\setlength{\itemindent}{0cm}
\setlength{\listparindent}{0cm}
\setlength{\leftmargin}{\evensidemargin}
\addtolength{\leftmargin}{\tmplength}
\settowidth{\labelsep}{X}
\addtolength{\leftmargin}{\labelsep}
\setlength{\labelwidth}{\tmplength}
}
\begin{flushleft}
\item[\textbf{Declaração}\hfill]
\begin{ttfamily}
public function IsFileOpen:Boolean; Virtual;\end{ttfamily}


\end{flushleft}
\end{list}
\paragraph*{GetRecBase}\hspace*{\fill}

\begin{list}{}{
\settowidth{\tmplength}{\textbf{Declaração}}
\setlength{\itemindent}{0cm}
\setlength{\listparindent}{0cm}
\setlength{\leftmargin}{\evensidemargin}
\addtolength{\leftmargin}{\tmplength}
\settowidth{\labelsep}{X}
\addtolength{\leftmargin}{\labelsep}
\setlength{\labelwidth}{\tmplength}
}
\begin{flushleft}
\item[\textbf{Declaração}\hfill]
\begin{ttfamily}
public function GetRecBase(Var RecBase):Integer; Overload; Virtual;\end{ttfamily}


\end{flushleft}
\end{list}
\paragraph*{PutRecBase}\hspace*{\fill}

\begin{list}{}{
\settowidth{\tmplength}{\textbf{Declaração}}
\setlength{\itemindent}{0cm}
\setlength{\listparindent}{0cm}
\setlength{\leftmargin}{\evensidemargin}
\addtolength{\leftmargin}{\tmplength}
\settowidth{\labelsep}{X}
\addtolength{\leftmargin}{\labelsep}
\setlength{\labelwidth}{\tmplength}
}
\begin{flushleft}
\item[\textbf{Declaração}\hfill]
\begin{ttfamily}
public function PutRecBase(Var RecBase):Integer; Overload; Virtual;\end{ttfamily}


\end{flushleft}
\end{list}
\paragraph*{GetRecBase}\hspace*{\fill}

\begin{list}{}{
\settowidth{\tmplength}{\textbf{Declaração}}
\setlength{\itemindent}{0cm}
\setlength{\listparindent}{0cm}
\setlength{\leftmargin}{\evensidemargin}
\addtolength{\leftmargin}{\tmplength}
\settowidth{\labelsep}{X}
\addtolength{\leftmargin}{\labelsep}
\setlength{\labelwidth}{\tmplength}
}
\begin{flushleft}
\item[\textbf{Declaração}\hfill]
\begin{ttfamily}
public function GetRecBase:Integer; Overload; Virtual;\end{ttfamily}


\end{flushleft}
\end{list}
\paragraph*{PutRecBase}\hspace*{\fill}

\begin{list}{}{
\settowidth{\tmplength}{\textbf{Declaração}}
\setlength{\itemindent}{0cm}
\setlength{\listparindent}{0cm}
\setlength{\leftmargin}{\evensidemargin}
\addtolength{\leftmargin}{\tmplength}
\settowidth{\labelsep}{X}
\addtolength{\leftmargin}{\labelsep}
\setlength{\labelwidth}{\tmplength}
}
\begin{flushleft}
\item[\textbf{Declaração}\hfill]
\begin{ttfamily}
public function PutRecBase:Integer; Overload; Virtual;\end{ttfamily}


\end{flushleft}
\end{list}
\paragraph*{GetRec}\hspace*{\fill}

\begin{list}{}{
\settowidth{\tmplength}{\textbf{Declaração}}
\setlength{\itemindent}{0cm}
\setlength{\listparindent}{0cm}
\setlength{\leftmargin}{\evensidemargin}
\addtolength{\leftmargin}{\tmplength}
\settowidth{\labelsep}{X}
\addtolength{\leftmargin}{\labelsep}
\setlength{\labelwidth}{\tmplength}
}
\begin{flushleft}
\item[\textbf{Declaração}\hfill]
\begin{ttfamily}
public function GetRec(Nr: Longint;Var Rec):Integer; Overload; Virtual;\end{ttfamily}


\end{flushleft}
\end{list}
\paragraph*{PutRec}\hspace*{\fill}

\begin{list}{}{
\settowidth{\tmplength}{\textbf{Declaração}}
\setlength{\itemindent}{0cm}
\setlength{\listparindent}{0cm}
\setlength{\leftmargin}{\evensidemargin}
\addtolength{\leftmargin}{\tmplength}
\settowidth{\labelsep}{X}
\addtolength{\leftmargin}{\labelsep}
\setlength{\labelwidth}{\tmplength}
}
\begin{flushleft}
\item[\textbf{Declaração}\hfill]
\begin{ttfamily}
public function PutRec(Nr: Longint;Var Rec):Integer; Overload; Virtual;\end{ttfamily}


\end{flushleft}
\end{list}
\paragraph*{GetRec}\hspace*{\fill}

\begin{list}{}{
\settowidth{\tmplength}{\textbf{Declaração}}
\setlength{\itemindent}{0cm}
\setlength{\listparindent}{0cm}
\setlength{\leftmargin}{\evensidemargin}
\addtolength{\leftmargin}{\tmplength}
\settowidth{\labelsep}{X}
\addtolength{\leftmargin}{\labelsep}
\setlength{\labelwidth}{\tmplength}
}
\begin{flushleft}
\item[\textbf{Declaração}\hfill]
\begin{ttfamily}
public function GetRec(Nr: Longint):Integer; Overload; Virtual;\end{ttfamily}


\end{flushleft}
\end{list}
\paragraph*{PutRec}\hspace*{\fill}

\begin{list}{}{
\settowidth{\tmplength}{\textbf{Declaração}}
\setlength{\itemindent}{0cm}
\setlength{\listparindent}{0cm}
\setlength{\leftmargin}{\evensidemargin}
\addtolength{\leftmargin}{\tmplength}
\settowidth{\labelsep}{X}
\addtolength{\leftmargin}{\labelsep}
\setlength{\labelwidth}{\tmplength}
}
\begin{flushleft}
\item[\textbf{Declaração}\hfill]
\begin{ttfamily}
public function PutRec(Nr: Longint):Integer; Overload; Virtual;\end{ttfamily}


\end{flushleft}
\end{list}
\paragraph*{BlockRead}\hspace*{\fill}

\begin{list}{}{
\settowidth{\tmplength}{\textbf{Declaração}}
\setlength{\itemindent}{0cm}
\setlength{\listparindent}{0cm}
\setlength{\leftmargin}{\evensidemargin}
\addtolength{\leftmargin}{\tmplength}
\settowidth{\labelsep}{X}
\addtolength{\leftmargin}{\labelsep}
\setlength{\labelwidth}{\tmplength}
}
\begin{flushleft}
\item[\textbf{Declaração}\hfill]
\begin{ttfamily}
public function BlockRead(Nr: Longint;Var Blocks ; Const Count: Longint):Longint; Virtual;\end{ttfamily}


\end{flushleft}
\end{list}
\paragraph*{BlockWrite}\hspace*{\fill}

\begin{list}{}{
\settowidth{\tmplength}{\textbf{Declaração}}
\setlength{\itemindent}{0cm}
\setlength{\listparindent}{0cm}
\setlength{\leftmargin}{\evensidemargin}
\addtolength{\leftmargin}{\tmplength}
\settowidth{\labelsep}{X}
\addtolength{\leftmargin}{\labelsep}
\setlength{\labelwidth}{\tmplength}
}
\begin{flushleft}
\item[\textbf{Declaração}\hfill]
\begin{ttfamily}
public function BlockWrite(Nr: Longint;Var Blocks ; Const Count: Longint):Longint; Virtual;\end{ttfamily}


\end{flushleft}
\end{list}
\paragraph*{Error}\hspace*{\fill}

\begin{list}{}{
\settowidth{\tmplength}{\textbf{Declaração}}
\setlength{\itemindent}{0cm}
\setlength{\listparindent}{0cm}
\setlength{\leftmargin}{\evensidemargin}
\addtolength{\leftmargin}{\tmplength}
\settowidth{\labelsep}{X}
\addtolength{\leftmargin}{\labelsep}
\setlength{\labelwidth}{\tmplength}
}
\begin{flushleft}
\item[\textbf{Declaração}\hfill]
\begin{ttfamily}
public procedure Error(Code, Info: Integer); Override;\end{ttfamily}


\end{flushleft}
\end{list}
\paragraph*{Truncate}\hspace*{\fill}

\begin{list}{}{
\settowidth{\tmplength}{\textbf{Declaração}}
\setlength{\itemindent}{0cm}
\setlength{\listparindent}{0cm}
\setlength{\leftmargin}{\evensidemargin}
\addtolength{\leftmargin}{\tmplength}
\settowidth{\labelsep}{X}
\addtolength{\leftmargin}{\labelsep}
\setlength{\labelwidth}{\tmplength}
}
\begin{flushleft}
\item[\textbf{Declaração}\hfill]
\begin{ttfamily}
public procedure Truncate(Pos: LongInt); Overload; Virtual;\end{ttfamily}


\end{flushleft}
\end{list}
\paragraph*{CopyFrom}\hspace*{\fill}

\begin{list}{}{
\settowidth{\tmplength}{\textbf{Declaração}}
\setlength{\itemindent}{0cm}
\setlength{\listparindent}{0cm}
\setlength{\leftmargin}{\evensidemargin}
\addtolength{\leftmargin}{\tmplength}
\settowidth{\labelsep}{X}
\addtolength{\leftmargin}{\labelsep}
\setlength{\labelwidth}{\tmplength}
}
\begin{flushleft}
\item[\textbf{Declaração}\hfill]
\begin{ttfamily}
public procedure CopyFrom(Var S: TStream; Count: LongInt); Overload; Virtual;\end{ttfamily}


\end{flushleft}
\end{list}
\paragraph*{CopyFrom}\hspace*{\fill}

\begin{list}{}{
\settowidth{\tmplength}{\textbf{Declaração}}
\setlength{\itemindent}{0cm}
\setlength{\listparindent}{0cm}
\setlength{\leftmargin}{\evensidemargin}
\addtolength{\leftmargin}{\tmplength}
\settowidth{\labelsep}{X}
\addtolength{\leftmargin}{\labelsep}
\setlength{\labelwidth}{\tmplength}
}
\begin{flushleft}
\item[\textbf{Declaração}\hfill]
\begin{ttfamily}
public procedure CopyFrom(Var S: TStream ); Overload; Virtual;\end{ttfamily}


\end{flushleft}
\end{list}
\paragraph*{Bof}\hspace*{\fill}

\begin{list}{}{
\settowidth{\tmplength}{\textbf{Declaração}}
\setlength{\itemindent}{0cm}
\setlength{\listparindent}{0cm}
\setlength{\leftmargin}{\evensidemargin}
\addtolength{\leftmargin}{\tmplength}
\settowidth{\labelsep}{X}
\addtolength{\leftmargin}{\labelsep}
\setlength{\labelwidth}{\tmplength}
}
\begin{flushleft}
\item[\textbf{Declaração}\hfill]
\begin{ttfamily}
public function Bof:Boolean; Virtual;\end{ttfamily}


\end{flushleft}
\end{list}
\paragraph*{Eof}\hspace*{\fill}

\begin{list}{}{
\settowidth{\tmplength}{\textbf{Declaração}}
\setlength{\itemindent}{0cm}
\setlength{\listparindent}{0cm}
\setlength{\leftmargin}{\evensidemargin}
\addtolength{\leftmargin}{\tmplength}
\settowidth{\labelsep}{X}
\addtolength{\leftmargin}{\labelsep}
\setlength{\labelwidth}{\tmplength}
}
\begin{flushleft}
\item[\textbf{Declaração}\hfill]
\begin{ttfamily}
public function Eof:Boolean; Virtual;\end{ttfamily}


\end{flushleft}
\end{list}
\paragraph*{goBof}\hspace*{\fill}

\begin{list}{}{
\settowidth{\tmplength}{\textbf{Declaração}}
\setlength{\itemindent}{0cm}
\setlength{\listparindent}{0cm}
\setlength{\leftmargin}{\evensidemargin}
\addtolength{\leftmargin}{\tmplength}
\settowidth{\labelsep}{X}
\addtolength{\leftmargin}{\labelsep}
\setlength{\labelwidth}{\tmplength}
}
\begin{flushleft}
\item[\textbf{Declaração}\hfill]
\begin{ttfamily}
public function goBof:Boolean;\end{ttfamily}


\end{flushleft}
\end{list}
\paragraph*{goEof}\hspace*{\fill}

\begin{list}{}{
\settowidth{\tmplength}{\textbf{Declaração}}
\setlength{\itemindent}{0cm}
\setlength{\listparindent}{0cm}
\setlength{\leftmargin}{\evensidemargin}
\addtolength{\leftmargin}{\tmplength}
\settowidth{\labelsep}{X}
\addtolength{\leftmargin}{\labelsep}
\setlength{\labelwidth}{\tmplength}
}
\begin{flushleft}
\item[\textbf{Declaração}\hfill]
\begin{ttfamily}
public function goEof:Boolean;\end{ttfamily}


\end{flushleft}
\end{list}
\paragraph*{SetFileName}\hspace*{\fill}

\begin{list}{}{
\settowidth{\tmplength}{\textbf{Declaração}}
\setlength{\itemindent}{0cm}
\setlength{\listparindent}{0cm}
\setlength{\leftmargin}{\evensidemargin}
\addtolength{\leftmargin}{\tmplength}
\settowidth{\labelsep}{X}
\addtolength{\leftmargin}{\labelsep}
\setlength{\labelwidth}{\tmplength}
}
\begin{flushleft}
\item[\textbf{Declaração}\hfill]
\begin{ttfamily}
protected procedure SetFileName(a{\_}FileName: AnsiString); Virtual;\end{ttfamily}


\end{flushleft}
\end{list}
\paragraph*{GetFileName}\hspace*{\fill}

\begin{list}{}{
\settowidth{\tmplength}{\textbf{Declaração}}
\setlength{\itemindent}{0cm}
\setlength{\listparindent}{0cm}
\setlength{\leftmargin}{\evensidemargin}
\addtolength{\leftmargin}{\tmplength}
\settowidth{\labelsep}{X}
\addtolength{\leftmargin}{\labelsep}
\setlength{\labelwidth}{\tmplength}
}
\begin{flushleft}
\item[\textbf{Declaração}\hfill]
\begin{ttfamily}
protected function GetFileName: AnsiString; Virtual;\end{ttfamily}


\end{flushleft}
\end{list}
\chapter{Unit mi.rtl.Objects.Methods.StreamBase.Stream.FileStream}
\section{Descrição}
\begin{itemize}
\item A Unit \textbf{\begin{ttfamily}mi.rtl.Objects.Methods.StreamBase.Stream.FileStream\end{ttfamily}} implementa a classe \begin{ttfamily}TFileStream\end{ttfamily}(\ref{mi.rtl.Objects.Methods.StreamBase.Stream.FileStream.TFileStream}) do pacote \textbf{\begin{ttfamily}mi.rtl\end{ttfamily}(\ref{mi.rtl})}.

\begin{itemize}
\item \textbf{NOTAS} \begin{itemize}
\item Implementa banco um fluxo de dados em disco.
\end{itemize}
\item \textbf{VERSÃO} \begin{itemize}
\item Alpha {-} 0.5.0.687
\end{itemize}
\item \textbf{HISTÓRICO} \begin{itemize}
\item Criado por: Paulo Sérgio da Silva Pacheco e{-}mail: paulosspacheco@yahoo.com.br \begin{itemize}
\item \textbf{22/11/2021} \begin{itemize}
\item 17:00 a 19:45 Criar a unit \begin{ttfamily}mi.rtl.Objects.Methods.StreamBase.Stream.FileStream\end{ttfamily}
\end{itemize}
\item \textbf{29/11/2021} \begin{itemize}
\item 10:10 a 12:01 {-} t12 Documentar a classe \begin{ttfamily}TFileStream\end{ttfamily}(\ref{mi.rtl.Objects.Methods.StreamBase.Stream.FileStream.TFileStream}). Exemplo 01: Test{\_}FileStream{\_}com{\_}header
\item 13:50 a 14:33 {-} t12 Documentar a classe \begin{ttfamily}TFileStream\end{ttfamily}(\ref{mi.rtl.Objects.Methods.StreamBase.Stream.FileStream.TFileStream}). Exemplo 01: Test{\_}FileStream{\_}sem{\_}header
\end{itemize}
\item \textbf{21/12/2021} \begin{itemize}
\item 15:30 a 16:20 {-} T12 Transferir os métodos de TDosStream para TFileStrem.
\end{itemize}
\end{itemize}
\end{itemize}
\item \textbf{CÓDIGO FONTE}: \begin{itemize}
\item 
\end{itemize}
\end{itemize}
\end{itemize}
\section{Uses}
\begin{itemize}
\item \begin{ttfamily}Classes\end{ttfamily}\item \begin{ttfamily}SysUtils\end{ttfamily}\item \begin{ttfamily}mi.rtl.types\end{ttfamily}(\ref{mi.rtl.Types})\item \begin{ttfamily}mi.rtl.consts\end{ttfamily}(\ref{mi.rtl.Consts})\item \begin{ttfamily}mi.rtl.files\end{ttfamily}(\ref{mi.rtl.files})\item \begin{ttfamily}mi.rtl.objects.types\end{ttfamily}(\ref{mi.rtl.objects.types})\item \begin{ttfamily}mi.rtl.objects.Methods\end{ttfamily}(\ref{mi.rtl.Objects.Methods})\item \begin{ttfamily}mi.rtl.objects.methods.StreamBase\end{ttfamily}(\ref{mi.rtl.Objects.Methods.StreamBase})\item \begin{ttfamily}mi.rtl.objects.methods.StreamBase.Stream\end{ttfamily}(\ref{mi.rtl.Objects.Methods.StreamBase.Stream})\item \begin{ttfamily}mi.rtl.objects.consts.MI{\_}MsgBox\end{ttfamily}\end{itemize}
\section{Visão Geral}
\begin{description}
\item[\texttt{\begin{ttfamily}TFileStream\end{ttfamily} Classe}]
\end{description}
\section{Classes, Interfaces, Objetos e Registros}
\subsection*{TFileStream Classe}
\subsubsection*{\large{\textbf{Hierarquia}}\normalsize\hspace{1ex}\hfill}
TFileStream {$>$} \begin{ttfamily}TStream\end{ttfamily}(\ref{mi.rtl.Objects.Methods.StreamBase.Stream.TStream}) {$>$} \begin{ttfamily}TStreamBase\end{ttfamily}(\ref{mi.rtl.Objects.Methods.StreamBase.TStreamBase}) {$>$} \begin{ttfamily}TObjectsMethods\end{ttfamily}(\ref{mi.rtl.Objects.Methods.TObjectsMethods}) {$>$} \begin{ttfamily}TObjectsConsts\end{ttfamily}(\ref{mi.rtl.Objects.Consts.TObjectsConsts}) {$>$} 
TObjectsTypes
\subsubsection*{\large{\textbf{Descrição}}\normalsize\hspace{1ex}\hfill}
\begin{itemize}
\item A classe \textbf{\begin{ttfamily}TFileStream\end{ttfamily}} é usada para criar um fluxo de dados em disco usada no banco de dados mar{-}icarai, onde é possível adicionar um registro no início do arquivo de tamanho maior ou menor que os registros seguintes ao registro zero.

\begin{itemize}
\item \textbf{EXEMPLOS} \begin{itemize}
\item Test{\_}FileStream{\_}sem{\_}header
\end{itemize} \texttt{\\\nopagebreak[3]
\\\nopagebreak[3]
}\textbf{Procedure}\texttt{~TMi{\_}Rtl{\_}Tests.Test{\_}FileStream{\_}sem{\_}header;\\\nopagebreak[3]
~~}\textbf{type}\texttt{\\\nopagebreak[3]
~~~TAluno~=~}\textbf{record}\texttt{\\\nopagebreak[3]
~~~~~~~~~~~~~~Id~:~integer;\\\nopagebreak[3]
~~~~~~~~~~~~~~nome~:~}\textbf{string}\texttt{[35];\\\nopagebreak[3]
~~~~~~~~~~~~}\textbf{end}\texttt{;\\\nopagebreak[3]
\\\nopagebreak[3]
~~}\textbf{var}\texttt{\\\nopagebreak[3]
~~~~FileStream{\_}Alunos~:~TObjectss.TFileStream;\\\nopagebreak[3]
~~~~aluno~:~TAluno;\textit{//Registro~de~aluno}\\\nopagebreak[3]
\\\nopagebreak[3]
~~~~nr~:~longint;~\textit{//Número~do~registro.}\\\nopagebreak[3]
~~~~n~~:~longint;~\textit{//Contador}\\\nopagebreak[3]
\\\nopagebreak[3]
}\textbf{begin}\texttt{\\\nopagebreak[3]
~~}\textbf{with}\texttt{~TObjectss~}\textbf{do}\texttt{\\\nopagebreak[3]
~~}\textbf{try}\texttt{\\\nopagebreak[3]
~~~~fillchar(aluno,sizeof(aluno),'~');\\\nopagebreak[3]
\\\nopagebreak[3]
~~~~}\textbf{if}\texttt{~TObjectss.FileExists(expandFileName('aluno.txt'))\\\nopagebreak[3]
~~~~}\textbf{then}\texttt{~FileStream{\_}Alunos~:=~TFileStream.Create(expandFileName('aluno.txt'),fileMode)\\\nopagebreak[3]
~~~~}\textbf{else}\texttt{~FileStream{\_}Alunos~:=~TFileStream.Create(expandFileName('aluno.txt'),fileMode,fmCreate~);\\\nopagebreak[3]
\\\nopagebreak[3]
~~~~}\textbf{with}\texttt{~aluno,FileStream{\_}Alunos~}\textbf{do}\texttt{\\\nopagebreak[3]
~~~~}\textbf{if}\texttt{~status~=~StOk~}\textbf{then}\texttt{\\\nopagebreak[3]
~~~~}\textbf{begin}\texttt{\\\nopagebreak[3]
~~~~~~\textit{//Define~o~tamanho~do~registro}\\\nopagebreak[3]
~~~~~~recSize~:=~sizeof(aluno);\\\nopagebreak[3]
\\\nopagebreak[3]
~~~~~~\textit{//Adiciona~o~registro~0;}\\\nopagebreak[3]
~~~~~~}\textbf{if}\texttt{~status~=~StOk~}\textbf{then}\texttt{\\\nopagebreak[3]
~~~~~~}\textbf{begin}\texttt{\\\nopagebreak[3]
~~~~~~~~n~:=~0;\\\nopagebreak[3]
~~~~~~~~Id:=~n;\\\nopagebreak[3]
~~~~~~~~nome:=~'Paulo~Sergio';\\\nopagebreak[3]
~~~~~~~~PutRec(n,aluno);\\\nopagebreak[3]
~~~~~~}\textbf{end}\texttt{;\\\nopagebreak[3]
\\\nopagebreak[3]
~~~~~~\textit{//Adiciona~o~registro~1;}\\\nopagebreak[3]
~~~~~~}\textbf{if}\texttt{~status~=~StOk~}\textbf{then}\texttt{\\\nopagebreak[3]
~~~~~~}\textbf{begin}\texttt{\\\nopagebreak[3]
~~~~~~~~inc(n);\\\nopagebreak[3]
~~~~~~~~Id:=~n;\\\nopagebreak[3]
~~~~~~~~nome:=~'George~Bruno';\\\nopagebreak[3]
~~~~~~~~PutRec(n,aluno);\\\nopagebreak[3]
~~~~~~}\textbf{end}\texttt{;\\\nopagebreak[3]
\\\nopagebreak[3]
~~~~~~\textit{//~Ler~e~imprime~os~registros~salvos~acima.}\\\nopagebreak[3]
~~~~~~}\textbf{if}\texttt{~status~=~StOk~}\textbf{then}\texttt{\\\nopagebreak[3]
~~~~~~}\textbf{begin}\texttt{\\\nopagebreak[3]
~~~~~~~~}\textbf{for}\texttt{~nr~:=~0~}\textbf{to}\texttt{~n~}\textbf{do}\texttt{\\\nopagebreak[3]
~~~~~~~~}\textbf{begin}\texttt{\\\nopagebreak[3]
~~~~~~~~~~GetRec(nr,aluno);\\\nopagebreak[3]
~~~~~~~~~~}\textbf{if}\texttt{~Status~=~Stok\\\nopagebreak[3]
~~~~~~~~~~}\textbf{then}\texttt{~}\textbf{begin}\texttt{\\\nopagebreak[3]
~~~~~~~~~~~~~~~~~SysMessageBox('Nr~='+intToStr(nr)+~~~~';~id~='+intToStr(aluno.id)+';~Aluno~='+al\\\nopagebreak[3]
~~~~~~~~~~~~~~~~~~~~~~~~~~~~~~~,'Test{\_}FileStream{\_}sem{\_}header'\\\nopagebreak[3]
~~~~~~~~~~~~~~~~~~~~~~~~~~~~~~~,false);\\\nopagebreak[3]
\\\nopagebreak[3]
~~~~~~~~~~~~~~~}\textbf{end}\texttt{\\\nopagebreak[3]
~~~~~~~~~~}\textbf{else}\texttt{~break;\\\nopagebreak[3]
~~~~~~~~}\textbf{end}\texttt{;\\\nopagebreak[3]
~~~~~~}\textbf{end}\texttt{;\textit{//if}\\\nopagebreak[3]
~~~~}\textbf{end}\texttt{;~\textit{//with}\\\nopagebreak[3]
\\\nopagebreak[3]
~~~~}\textbf{with}\texttt{~FileStream{\_}Alunos~}\textbf{do}\texttt{\\\nopagebreak[3]
~~~~~~}\textbf{if}\texttt{~status~{$<$}{$>$}~StOk\\\nopagebreak[3]
~~~~~~}\textbf{then}\texttt{~SysMessageBox(ErrorMessage(ErrorInfo),'Test{\_}FileStream{\_}sem{\_}header',true);\\\nopagebreak[3]
\\\nopagebreak[3]
~~}\textbf{finally}\texttt{\\\nopagebreak[3]
~~~~FileStream{\_}Alunos.Destroy;\\\nopagebreak[3]
~~}\textbf{end}\texttt{;\\\nopagebreak[3]
\\\nopagebreak[3]
}\textbf{end}\texttt{;\\
}

\begin{itemize}
\item Test{\_}FileStream{\_}com{\_}header
\end{itemize} \texttt{\\\nopagebreak[3]
\\\nopagebreak[3]
}\textbf{procedure}\texttt{~TMi{\_}Rtl{\_}Tests.Test{\_}FileStream{\_}com{\_}header;\\\nopagebreak[3]
\\\nopagebreak[3]
~~}\textbf{type}\texttt{\\\nopagebreak[3]
\\\nopagebreak[3]
~~~\textit{//~Tipo~de~registro~1~ao~final~do~arquivo:}\\\nopagebreak[3]
~~~TAluno~=~}\textbf{record}\texttt{\\\nopagebreak[3]
~~~~~~~~~~~~~~Id~:~integer;\\\nopagebreak[3]
~~~~~~~~~~~~~~nome~:~}\textbf{string}\texttt{[35];\\\nopagebreak[3]
~~~~~~~~~~~~}\textbf{end}\texttt{;\\\nopagebreak[3]
\\\nopagebreak[3]
~~~\textit{//~Tipo~de~registro~a~ser~usado~no~registro~zero~do~arquivo.}\\\nopagebreak[3]
\\\nopagebreak[3]
~~~THeadAlunos~=~}\textbf{record}\texttt{\\\nopagebreak[3]
~~~~~~~~~~~~~~~~~~~TotalDeAlunos:longint;\\\nopagebreak[3]
~~~~~~~~~~~~~~~~~}\textbf{end}\texttt{;\\\nopagebreak[3]
\\\nopagebreak[3]
~~}\textbf{var}\texttt{\\\nopagebreak[3]
~~~~FileStream{\_}Alunos~:~TObjectss.TFileStream;\\\nopagebreak[3]
~~~~aluno~~~~~:~TAluno;\textit{//Registro~de~aluno}\\\nopagebreak[3]
~~~~headAluno~:~THeadAlunos;\\\nopagebreak[3]
\\\nopagebreak[3]
~~~~nr~:~longint;~\textit{//Número~do~registro.}\\\nopagebreak[3]
~~~~n~~:~longint;~\textit{//Contador}\\\nopagebreak[3]
\\\nopagebreak[3]
\textit{//Início~da~procedure}\\\nopagebreak[3]
}\textbf{begin}\texttt{\\\nopagebreak[3]
~}\textbf{with}\texttt{~TObjectss~}\textbf{do}\texttt{\\\nopagebreak[3]
~}\textbf{try}\texttt{\\\nopagebreak[3]
~~~fillchar(aluno,sizeof(aluno),'~');\\\nopagebreak[3]
\\\nopagebreak[3]
~~~}\textbf{if}\texttt{~TObjectss.FileExists(expandFileName('aluno.txt'))\\\nopagebreak[3]
~~~}\textbf{then}\texttt{~FileStream{\_}Alunos~:=~TFileStream.Create(expandFileName('aluno.txt'),fileMode)\\\nopagebreak[3]
~~~}\textbf{else}\texttt{~FileStream{\_}Alunos~:=~TFileStream.Create(expandFileName('aluno.txt'),fileMode,fmCreate~);\\\nopagebreak[3]
\\\nopagebreak[3]
~~~}\textbf{with}\texttt{~aluno,FileStream{\_}Alunos~}\textbf{do}\texttt{\\\nopagebreak[3]
~~~}\textbf{if}\texttt{~status~=~StOk~}\textbf{then}\texttt{\\\nopagebreak[3]
~~~}\textbf{begin}\texttt{\\\nopagebreak[3]
~~~~~\textit{//Define~o~tamanho~do~registro~zero}\\\nopagebreak[3]
~~~~~baseSize~:=~sizeof(headAluno);\\\nopagebreak[3]
\\\nopagebreak[3]
~~~~~\textit{//Define~o~tamanho~do~registro}\\\nopagebreak[3]
~~~~~recSize~:=~sizeof(aluno);\\\nopagebreak[3]
\\\nopagebreak[3]
~~~~~headAluno.TotalDeAlunos~:=~0;\\\nopagebreak[3]
~~~~~PutRecBase(headAluno);\\\nopagebreak[3]
\\\nopagebreak[3]
~~~~~\textit{//Adiciona~o~registro~0;}\\\nopagebreak[3]
~~~~~}\textbf{if}\texttt{~status~=~StOk~}\textbf{then}\texttt{\\\nopagebreak[3]
~~~~~}\textbf{begin}\texttt{\\\nopagebreak[3]
~~~~~~~inc(headAluno.TotalDeAlunos);\\\nopagebreak[3]
~~~~~~~n~:=~headAluno.TotalDeAlunos;\\\nopagebreak[3]
~~~~~~~Id:=~n;\\\nopagebreak[3]
~~~~~~~nome:=~'Paulo~Sergio';\\\nopagebreak[3]
~~~~~~~PutRec(n,aluno);\\\nopagebreak[3]
~~~~~~~PutRecBase(headAluno);\\\nopagebreak[3]
~~~~~}\textbf{end}\texttt{;\\\nopagebreak[3]
\\\nopagebreak[3]
~~~~~\textit{//Adiciona~o~registro~1;}\\\nopagebreak[3]
~~~~~}\textbf{if}\texttt{~status~=~StOk~}\textbf{then}\texttt{\\\nopagebreak[3]
~~~~~}\textbf{begin}\texttt{\\\nopagebreak[3]
~~~~~~~inc(headAluno.TotalDeAlunos);\\\nopagebreak[3]
~~~~~~~n~:=~headAluno.TotalDeAlunos;\\\nopagebreak[3]
~~~~~~~nome:=~'George~Bruno';\\\nopagebreak[3]
~~~~~~~PutRec(n,aluno);\\\nopagebreak[3]
~~~~~~~PutRecBase(headAluno);\\\nopagebreak[3]
~~~~~}\textbf{end}\texttt{;\\\nopagebreak[3]
\\\nopagebreak[3]
~~~~~\textit{//~Ler~e~imprime~os~registros~salvos~acima.}\\\nopagebreak[3]
~~~~~}\textbf{if}\texttt{~status~=~StOk~}\textbf{then}\texttt{\\\nopagebreak[3]
~~~~~}\textbf{begin}\texttt{\\\nopagebreak[3]
~~~~~~~getRecBase(headAluno);\\\nopagebreak[3]
~~~~~~~}\textbf{if}\texttt{~status~=~StOk~}\textbf{then}\texttt{\\\nopagebreak[3]
~~~~~~~}\textbf{begin}\texttt{\\\nopagebreak[3]
~~~~~~~~~SysMessageBox('Número~de~registros='+intToStr(headAluno.TotalDeAlunos)\\\nopagebreak[3]
~~~~~~~~~~~~~~~~~~~~~~~,'Test{\_}FileStream{\_}sem{\_}header'\\\nopagebreak[3]
~~~~~~~~~~~~~~~~~~~~~~~,false);\\\nopagebreak[3]
\\\nopagebreak[3]
~~~~~~~~~}\textbf{for}\texttt{~nr~:=~1~}\textbf{to}\texttt{~headAluno.TotalDeAlunos~}\textbf{do}\texttt{\\\nopagebreak[3]
~~~~~~~~~}\textbf{begin}\texttt{\\\nopagebreak[3]
~~~~~~~~~~~GetRec(nr,aluno);\\\nopagebreak[3]
~~~~~~~~~~~}\textbf{if}\texttt{~Status~=~Stok~}\textbf{then}\texttt{\\\nopagebreak[3]
~~~~~~~~~~~}\textbf{begin}\texttt{\\\nopagebreak[3]
~~~~~~~~~~~~~SysMessageBox('Nr~='+intToStr(nr)+~~~~';~id~='+intToStr(aluno.id)+';~Aluno~='+aluno.nome\\\nopagebreak[3]
~~~~~~~~~~~~~~~~~~~~~~~~~~,'Test{\_}FileStream{\_}sem{\_}header'\\\nopagebreak[3]
~~~~~~~~~~~~~~~~~~~~~~~~~~,false);\\\nopagebreak[3]
\\\nopagebreak[3]
~~~~~~~~~~~}\textbf{end}\texttt{~}\textbf{else}\texttt{~break;\\\nopagebreak[3]
~~~~~~~~~}\textbf{end}\texttt{;\\\nopagebreak[3]
~~~~~~~}\textbf{end}\texttt{;\\\nopagebreak[3]
~~~~~}\textbf{end}\texttt{;\\\nopagebreak[3]
~~~}\textbf{end}\texttt{;~\textit{//with}\\\nopagebreak[3]
\\\nopagebreak[3]
~~~}\textbf{with}\texttt{~FileStream{\_}Alunos~}\textbf{do}\texttt{\\\nopagebreak[3]
~~~~~}\textbf{if}\texttt{~status~{$<$}{$>$}~StOk\\\nopagebreak[3]
~~~~~}\textbf{then}\texttt{~SysMessageBox(ErrorMessage(ErrorInfo),'Test{\_}FileStream{\_}sem{\_}header',true);\\\nopagebreak[3]
\\\nopagebreak[3]
~}\textbf{finally}\texttt{\\\nopagebreak[3]
~~~FileStream{\_}Alunos.Destroy;\\\nopagebreak[3]
~}\textbf{end}\texttt{;\\\nopagebreak[3]
}\textbf{end}\texttt{;\\
}
\end{itemize}
\end{itemize}\subsubsection*{\large{\textbf{Campos}}\normalsize\hspace{1ex}\hfill}
\paragraph*{{\_}ShareModeAnt}\hspace*{\fill}

\begin{list}{}{
\settowidth{\tmplength}{\textbf{Declaração}}
\setlength{\itemindent}{0cm}
\setlength{\listparindent}{0cm}
\setlength{\leftmargin}{\evensidemargin}
\addtolength{\leftmargin}{\tmplength}
\settowidth{\labelsep}{X}
\addtolength{\leftmargin}{\labelsep}
\setlength{\labelwidth}{\tmplength}
}
\begin{flushleft}
\item[\textbf{Declaração}\hfill]
\begin{ttfamily}
protected {\_}ShareModeAnt: CARDINAL;\end{ttfamily}


\end{flushleft}
\end{list}
\paragraph*{Handle}\hspace*{\fill}

\begin{list}{}{
\settowidth{\tmplength}{\textbf{Declaração}}
\setlength{\itemindent}{0cm}
\setlength{\listparindent}{0cm}
\setlength{\leftmargin}{\evensidemargin}
\addtolength{\leftmargin}{\tmplength}
\settowidth{\labelsep}{X}
\addtolength{\leftmargin}{\labelsep}
\setlength{\labelwidth}{\tmplength}
}
\begin{flushleft}
\item[\textbf{Declaração}\hfill]
\begin{ttfamily}
public Handle: THandle;\end{ttfamily}


\end{flushleft}
\par
\item[\textbf{Descrição}]
DOS file \begin{ttfamily}handle\end{ttfamily}

\end{list}
\subsubsection*{\large{\textbf{Métodos}}\normalsize\hspace{1ex}\hfill}
\paragraph*{SetShareMode}\hspace*{\fill}

\begin{list}{}{
\settowidth{\tmplength}{\textbf{Declaração}}
\setlength{\itemindent}{0cm}
\setlength{\listparindent}{0cm}
\setlength{\leftmargin}{\evensidemargin}
\addtolength{\leftmargin}{\tmplength}
\settowidth{\labelsep}{X}
\addtolength{\leftmargin}{\labelsep}
\setlength{\labelwidth}{\tmplength}
}
\begin{flushleft}
\item[\textbf{Declaração}\hfill]
\begin{ttfamily}
protected procedure SetShareMode(Const a{\_}ShareMode:CARDINAL); override;\end{ttfamily}


\end{flushleft}
\end{list}
\paragraph*{SetFileName}\hspace*{\fill}

\begin{list}{}{
\settowidth{\tmplength}{\textbf{Declaração}}
\setlength{\itemindent}{0cm}
\setlength{\listparindent}{0cm}
\setlength{\leftmargin}{\evensidemargin}
\addtolength{\leftmargin}{\tmplength}
\settowidth{\labelsep}{X}
\addtolength{\leftmargin}{\labelsep}
\setlength{\labelwidth}{\tmplength}
}
\begin{flushleft}
\item[\textbf{Declaração}\hfill]
\begin{ttfamily}
protected procedure SetFileName(a{\_}FileName: AnsiString); Override;\end{ttfamily}


\end{flushleft}
\end{list}
\paragraph*{Create}\hspace*{\fill}

\begin{list}{}{
\settowidth{\tmplength}{\textbf{Declaração}}
\setlength{\itemindent}{0cm}
\setlength{\listparindent}{0cm}
\setlength{\leftmargin}{\evensidemargin}
\addtolength{\leftmargin}{\tmplength}
\settowidth{\labelsep}{X}
\addtolength{\leftmargin}{\labelsep}
\setlength{\labelwidth}{\tmplength}
}
\begin{flushleft}
\item[\textbf{Declaração}\hfill]
\begin{ttfamily}
public CONSTRUCTOR Create(aFName: AnsiString; aFileMode: Word;aShareMode:Cardinal); overload; virtual;\end{ttfamily}


\end{flushleft}
\end{list}
\paragraph*{Create}\hspace*{\fill}

\begin{list}{}{
\settowidth{\tmplength}{\textbf{Declaração}}
\setlength{\itemindent}{0cm}
\setlength{\listparindent}{0cm}
\setlength{\leftmargin}{\evensidemargin}
\addtolength{\leftmargin}{\tmplength}
\settowidth{\labelsep}{X}
\addtolength{\leftmargin}{\labelsep}
\setlength{\labelwidth}{\tmplength}
}
\begin{flushleft}
\item[\textbf{Declaração}\hfill]
\begin{ttfamily}
public CONSTRUCTOR Create(aFName: AnsiString; aFileMode: Word); overload; virtual;\end{ttfamily}


\end{flushleft}
\end{list}
\paragraph*{Create}\hspace*{\fill}

\begin{list}{}{
\settowidth{\tmplength}{\textbf{Declaração}}
\setlength{\itemindent}{0cm}
\setlength{\listparindent}{0cm}
\setlength{\leftmargin}{\evensidemargin}
\addtolength{\leftmargin}{\tmplength}
\settowidth{\labelsep}{X}
\addtolength{\leftmargin}{\labelsep}
\setlength{\labelwidth}{\tmplength}
}
\begin{flushleft}
\item[\textbf{Declaração}\hfill]
\begin{ttfamily}
public CONSTRUCTOR Create(aFileName: AnsiString; aFileMode: Word;Size: Sw{\_}Word;a{\_}BaseSize,a{\_}RecSize:Longint); overload; virtual;\end{ttfamily}


\end{flushleft}
\end{list}
\paragraph*{GetDriveType}\hspace*{\fill}

\begin{list}{}{
\settowidth{\tmplength}{\textbf{Declaração}}
\setlength{\itemindent}{0cm}
\setlength{\listparindent}{0cm}
\setlength{\leftmargin}{\evensidemargin}
\addtolength{\leftmargin}{\tmplength}
\settowidth{\labelsep}{X}
\addtolength{\leftmargin}{\labelsep}
\setlength{\labelwidth}{\tmplength}
}
\begin{flushleft}
\item[\textbf{Declaração}\hfill]
\begin{ttfamily}
public Function GetDriveType:TDriveType; Override;\end{ttfamily}


\end{flushleft}
\end{list}
\paragraph*{Destroy}\hspace*{\fill}

\begin{list}{}{
\settowidth{\tmplength}{\textbf{Declaração}}
\setlength{\itemindent}{0cm}
\setlength{\listparindent}{0cm}
\setlength{\leftmargin}{\evensidemargin}
\addtolength{\leftmargin}{\tmplength}
\settowidth{\labelsep}{X}
\addtolength{\leftmargin}{\labelsep}
\setlength{\labelwidth}{\tmplength}
}
\begin{flushleft}
\item[\textbf{Declaração}\hfill]
\begin{ttfamily}
public DESTRUCTOR Destroy; Override;\end{ttfamily}


\end{flushleft}
\end{list}
\paragraph*{Truncate}\hspace*{\fill}

\begin{list}{}{
\settowidth{\tmplength}{\textbf{Declaração}}
\setlength{\itemindent}{0cm}
\setlength{\listparindent}{0cm}
\setlength{\leftmargin}{\evensidemargin}
\addtolength{\leftmargin}{\tmplength}
\settowidth{\labelsep}{X}
\addtolength{\leftmargin}{\labelsep}
\setlength{\labelwidth}{\tmplength}
}
\begin{flushleft}
\item[\textbf{Declaração}\hfill]
\begin{ttfamily}
public PROCEDURE Truncate; Overload; Override;\end{ttfamily}


\end{flushleft}
\end{list}
\paragraph*{Seek}\hspace*{\fill}

\begin{list}{}{
\settowidth{\tmplength}{\textbf{Declaração}}
\setlength{\itemindent}{0cm}
\setlength{\listparindent}{0cm}
\setlength{\leftmargin}{\evensidemargin}
\addtolength{\leftmargin}{\tmplength}
\settowidth{\labelsep}{X}
\addtolength{\leftmargin}{\labelsep}
\setlength{\labelwidth}{\tmplength}
}
\begin{flushleft}
\item[\textbf{Declaração}\hfill]
\begin{ttfamily}
public procedure Seek(NR: LongInt;a{\_}RecSize:Longint); Overload; override;\end{ttfamily}


\end{flushleft}
\end{list}
\paragraph*{Open}\hspace*{\fill}

\begin{list}{}{
\settowidth{\tmplength}{\textbf{Declaração}}
\setlength{\itemindent}{0cm}
\setlength{\listparindent}{0cm}
\setlength{\leftmargin}{\evensidemargin}
\addtolength{\leftmargin}{\tmplength}
\settowidth{\labelsep}{X}
\addtolength{\leftmargin}{\labelsep}
\setlength{\labelwidth}{\tmplength}
}
\begin{flushleft}
\item[\textbf{Declaração}\hfill]
\begin{ttfamily}
public PROCEDURE Open; overload; Override;\end{ttfamily}


\end{flushleft}
\end{list}
\paragraph*{Open}\hspace*{\fill}

\begin{list}{}{
\settowidth{\tmplength}{\textbf{Declaração}}
\setlength{\itemindent}{0cm}
\setlength{\listparindent}{0cm}
\setlength{\leftmargin}{\evensidemargin}
\addtolength{\leftmargin}{\tmplength}
\settowidth{\labelsep}{X}
\addtolength{\leftmargin}{\labelsep}
\setlength{\labelwidth}{\tmplength}
}
\begin{flushleft}
\item[\textbf{Declaração}\hfill]
\begin{ttfamily}
public PROCEDURE Open(aFileMode: Word;aShareMode:Cardinal); Overload; Virtual;\end{ttfamily}


\end{flushleft}
\end{list}
\paragraph*{Close}\hspace*{\fill}

\begin{list}{}{
\settowidth{\tmplength}{\textbf{Declaração}}
\setlength{\itemindent}{0cm}
\setlength{\listparindent}{0cm}
\setlength{\leftmargin}{\evensidemargin}
\addtolength{\leftmargin}{\tmplength}
\settowidth{\labelsep}{X}
\addtolength{\leftmargin}{\labelsep}
\setlength{\labelwidth}{\tmplength}
}
\begin{flushleft}
\item[\textbf{Declaração}\hfill]
\begin{ttfamily}
public PROCEDURE Close; Override;\end{ttfamily}


\end{flushleft}
\end{list}
\paragraph*{Reset}\hspace*{\fill}

\begin{list}{}{
\settowidth{\tmplength}{\textbf{Declaração}}
\setlength{\itemindent}{0cm}
\setlength{\listparindent}{0cm}
\setlength{\leftmargin}{\evensidemargin}
\addtolength{\leftmargin}{\tmplength}
\settowidth{\labelsep}{X}
\addtolength{\leftmargin}{\labelsep}
\setlength{\labelwidth}{\tmplength}
}
\begin{flushleft}
\item[\textbf{Declaração}\hfill]
\begin{ttfamily}
public PROCEDURE Reset; Overload; Override;\end{ttfamily}


\end{flushleft}
\end{list}
\paragraph*{Reset}\hspace*{\fill}

\begin{list}{}{
\settowidth{\tmplength}{\textbf{Declaração}}
\setlength{\itemindent}{0cm}
\setlength{\listparindent}{0cm}
\setlength{\leftmargin}{\evensidemargin}
\addtolength{\leftmargin}{\tmplength}
\settowidth{\labelsep}{X}
\addtolength{\leftmargin}{\labelsep}
\setlength{\labelwidth}{\tmplength}
}
\begin{flushleft}
\item[\textbf{Declaração}\hfill]
\begin{ttfamily}
public PROCEDURE Reset(aFileMode: Word;aShareMode : Cardinal); Overload; override;\end{ttfamily}


\end{flushleft}
\end{list}
\paragraph*{Rewrite}\hspace*{\fill}

\begin{list}{}{
\settowidth{\tmplength}{\textbf{Declaração}}
\setlength{\itemindent}{0cm}
\setlength{\listparindent}{0cm}
\setlength{\leftmargin}{\evensidemargin}
\addtolength{\leftmargin}{\tmplength}
\settowidth{\labelsep}{X}
\addtolength{\leftmargin}{\labelsep}
\setlength{\labelwidth}{\tmplength}
}
\begin{flushleft}
\item[\textbf{Declaração}\hfill]
\begin{ttfamily}
public PROCEDURE Rewrite; Overload; Override;\end{ttfamily}


\end{flushleft}
\end{list}
\paragraph*{Rewrite}\hspace*{\fill}

\begin{list}{}{
\settowidth{\tmplength}{\textbf{Declaração}}
\setlength{\itemindent}{0cm}
\setlength{\listparindent}{0cm}
\setlength{\leftmargin}{\evensidemargin}
\addtolength{\leftmargin}{\tmplength}
\settowidth{\labelsep}{X}
\addtolength{\leftmargin}{\labelsep}
\setlength{\labelwidth}{\tmplength}
}
\begin{flushleft}
\item[\textbf{Declaração}\hfill]
\begin{ttfamily}
public procedure Rewrite(aFileMode: Word;aShareMode : Cardinal); Overload; Override;\end{ttfamily}


\end{flushleft}
\end{list}
\paragraph*{Read}\hspace*{\fill}

\begin{list}{}{
\settowidth{\tmplength}{\textbf{Declaração}}
\setlength{\itemindent}{0cm}
\setlength{\listparindent}{0cm}
\setlength{\leftmargin}{\evensidemargin}
\addtolength{\leftmargin}{\tmplength}
\settowidth{\labelsep}{X}
\addtolength{\leftmargin}{\labelsep}
\setlength{\labelwidth}{\tmplength}
}
\begin{flushleft}
\item[\textbf{Declaração}\hfill]
\begin{ttfamily}
public PROCEDURE Read(Var Buf; Count: Sw{\_}Word); Overload; Override;\end{ttfamily}


\end{flushleft}
\end{list}
\paragraph*{Write}\hspace*{\fill}

\begin{list}{}{
\settowidth{\tmplength}{\textbf{Declaração}}
\setlength{\itemindent}{0cm}
\setlength{\listparindent}{0cm}
\setlength{\leftmargin}{\evensidemargin}
\addtolength{\leftmargin}{\tmplength}
\settowidth{\labelsep}{X}
\addtolength{\leftmargin}{\labelsep}
\setlength{\labelwidth}{\tmplength}
}
\begin{flushleft}
\item[\textbf{Declaração}\hfill]
\begin{ttfamily}
public PROCEDURE Write(Var Buf; Count: Sw{\_}Word); Overload; Override;\end{ttfamily}


\end{flushleft}
\end{list}
\paragraph*{GetSize}\hspace*{\fill}

\begin{list}{}{
\settowidth{\tmplength}{\textbf{Declaração}}
\setlength{\itemindent}{0cm}
\setlength{\listparindent}{0cm}
\setlength{\leftmargin}{\evensidemargin}
\addtolength{\leftmargin}{\tmplength}
\settowidth{\labelsep}{X}
\addtolength{\leftmargin}{\labelsep}
\setlength{\labelwidth}{\tmplength}
}
\begin{flushleft}
\item[\textbf{Declaração}\hfill]
\begin{ttfamily}
public FUNCTION GetSize: LongInt; Override;\end{ttfamily}


\end{flushleft}
\end{list}
\paragraph*{CloseOpen}\hspace*{\fill}

\begin{list}{}{
\settowidth{\tmplength}{\textbf{Declaração}}
\setlength{\itemindent}{0cm}
\setlength{\listparindent}{0cm}
\setlength{\leftmargin}{\evensidemargin}
\addtolength{\leftmargin}{\tmplength}
\settowidth{\labelsep}{X}
\addtolength{\leftmargin}{\labelsep}
\setlength{\labelwidth}{\tmplength}
}
\begin{flushleft}
\item[\textbf{Declaração}\hfill]
\begin{ttfamily}
public Function CloseOpen:Integer; Override;\end{ttfamily}


\end{flushleft}
\par
\item[\textbf{Descrição}]
\begin{itemize}
\item O método \begin{ttfamily}CloseOpen\end{ttfamily} é usado para obrigar o windows a descarregar o buffer do arquivo.

\begin{itemize}
\item \textbf{NOTA} \begin{itemize}
\item O linux não tem a função dulicateHandle.
\end{itemize}
\end{itemize}
\end{itemize}

\end{list}
\paragraph*{Flush{\_}Disk}\hspace*{\fill}

\begin{list}{}{
\settowidth{\tmplength}{\textbf{Declaração}}
\setlength{\itemindent}{0cm}
\setlength{\listparindent}{0cm}
\setlength{\leftmargin}{\evensidemargin}
\addtolength{\leftmargin}{\tmplength}
\settowidth{\labelsep}{X}
\addtolength{\leftmargin}{\labelsep}
\setlength{\labelwidth}{\tmplength}
}
\begin{flushleft}
\item[\textbf{Declaração}\hfill]
\begin{ttfamily}
public Function Flush{\_}Disk:Integer; Override;\end{ttfamily}


\end{flushleft}
\end{list}
\paragraph*{Flush}\hspace*{\fill}

\begin{list}{}{
\settowidth{\tmplength}{\textbf{Declaração}}
\setlength{\itemindent}{0cm}
\setlength{\listparindent}{0cm}
\setlength{\leftmargin}{\evensidemargin}
\addtolength{\leftmargin}{\tmplength}
\settowidth{\labelsep}{X}
\addtolength{\leftmargin}{\labelsep}
\setlength{\labelwidth}{\tmplength}
}
\begin{flushleft}
\item[\textbf{Declaração}\hfill]
\begin{ttfamily}
public PROCEDURE Flush; Override;\end{ttfamily}


\end{flushleft}
\end{list}
\paragraph*{IsFileOpen}\hspace*{\fill}

\begin{list}{}{
\settowidth{\tmplength}{\textbf{Declaração}}
\setlength{\itemindent}{0cm}
\setlength{\listparindent}{0cm}
\setlength{\leftmargin}{\evensidemargin}
\addtolength{\leftmargin}{\tmplength}
\settowidth{\labelsep}{X}
\addtolength{\leftmargin}{\labelsep}
\setlength{\labelwidth}{\tmplength}
}
\begin{flushleft}
\item[\textbf{Declaração}\hfill]
\begin{ttfamily}
public Function IsFileOpen:Boolean; Override;\end{ttfamily}


\end{flushleft}
\end{list}
\paragraph*{DeleteFile}\hspace*{\fill}

\begin{list}{}{
\settowidth{\tmplength}{\textbf{Declaração}}
\setlength{\itemindent}{0cm}
\setlength{\listparindent}{0cm}
\setlength{\leftmargin}{\evensidemargin}
\addtolength{\leftmargin}{\tmplength}
\settowidth{\labelsep}{X}
\addtolength{\leftmargin}{\labelsep}
\setlength{\labelwidth}{\tmplength}
}
\begin{flushleft}
\item[\textbf{Declaração}\hfill]
\begin{ttfamily}
public Procedure DeleteFile;\end{ttfamily}


\end{flushleft}
\end{list}
\paragraph*{CreateFileStream}\hspace*{\fill}

\begin{list}{}{
\settowidth{\tmplength}{\textbf{Declaração}}
\setlength{\itemindent}{0cm}
\setlength{\listparindent}{0cm}
\setlength{\leftmargin}{\evensidemargin}
\addtolength{\leftmargin}{\tmplength}
\settowidth{\labelsep}{X}
\addtolength{\leftmargin}{\labelsep}
\setlength{\labelwidth}{\tmplength}
}
\begin{flushleft}
\item[\textbf{Declaração}\hfill]
\begin{ttfamily}
public function CreateFileStream(aFName: AnsiString; aFileMode: Word) : TFileStream; Virtual;\end{ttfamily}


\end{flushleft}
\end{list}
\paragraph*{SaveToFile}\hspace*{\fill}

\begin{list}{}{
\settowidth{\tmplength}{\textbf{Declaração}}
\setlength{\itemindent}{0cm}
\setlength{\listparindent}{0cm}
\setlength{\leftmargin}{\evensidemargin}
\addtolength{\leftmargin}{\tmplength}
\settowidth{\labelsep}{X}
\addtolength{\leftmargin}{\labelsep}
\setlength{\labelwidth}{\tmplength}
}
\begin{flushleft}
\item[\textbf{Declaração}\hfill]
\begin{ttfamily}
public function SaveToFile(aFileName:AnsiString):Boolean; Overload; Virtual;\end{ttfamily}


\end{flushleft}
\end{list}
\paragraph*{SaveToFile}\hspace*{\fill}

\begin{list}{}{
\settowidth{\tmplength}{\textbf{Declaração}}
\setlength{\itemindent}{0cm}
\setlength{\listparindent}{0cm}
\setlength{\leftmargin}{\evensidemargin}
\addtolength{\leftmargin}{\tmplength}
\settowidth{\labelsep}{X}
\addtolength{\leftmargin}{\labelsep}
\setlength{\labelwidth}{\tmplength}
}
\begin{flushleft}
\item[\textbf{Declaração}\hfill]
\begin{ttfamily}
public function SaveToFile:Boolean; Overload; Virtual;\end{ttfamily}


\end{flushleft}
\end{list}
\paragraph*{LoadFromFile}\hspace*{\fill}

\begin{list}{}{
\settowidth{\tmplength}{\textbf{Declaração}}
\setlength{\itemindent}{0cm}
\setlength{\listparindent}{0cm}
\setlength{\leftmargin}{\evensidemargin}
\addtolength{\leftmargin}{\tmplength}
\settowidth{\labelsep}{X}
\addtolength{\leftmargin}{\labelsep}
\setlength{\labelwidth}{\tmplength}
}
\begin{flushleft}
\item[\textbf{Declaração}\hfill]
\begin{ttfamily}
public function LoadFromFile(aFileName:AnsiString):Boolean; Overload; virtual;\end{ttfamily}


\end{flushleft}
\end{list}
\chapter{Unit mi.rtl.Objects.Methods.StreamBase.Stream.MemoryStream}
\section{Descrição}
\begin{itemize}
\item A Unit \textbf{\begin{ttfamily}mi.rtl.Objects.Methods.StreamBase.Stream.MemoryStream\end{ttfamily}} implementa a classe \begin{ttfamily}TMemoryStream\end{ttfamily}(\ref{mi.rtl.Objects.Methods.StreamBase.Stream.MemoryStream.TMemoryStream}) do pacote \textbf{\begin{ttfamily}mi.rtl\end{ttfamily}(\ref{mi.rtl})}.

\begin{itemize}
\item \textbf{NOTAS} \begin{itemize}
\item Implementa um fluxo de dados em memória.
\end{itemize}
\item \textbf{VERSÃO} \begin{itemize}
\item Alpha {-} 0.5.0.687
\end{itemize}
\item \textbf{HISTÓRICO} \begin{itemize}
\item Criado por: Paulo Sérgio da Silva Pacheco e{-}mail: paulosspacheco@yahoo.com.br \begin{itemize}
\item \textbf{23/11/2021} \begin{itemize}
\item 06:10 a 07:17 {-} Criar a unit \begin{ttfamily}mi.rtl.Objects.Methods.StreamBase.Stream.MemoryStream\end{ttfamily}
\item 07:43 a {-} Documentar a unit \begin{ttfamily}mi.rtl.Objects.Methods.StreamBase.Stream.MemoryStream\end{ttfamily}.
\end{itemize}
\end{itemize}
\end{itemize}
\item \textbf{CÓDIGO FONTE}: \begin{itemize}
\item 
\end{itemize}
\end{itemize}
\end{itemize}
\section{Uses}
\begin{itemize}
\item \begin{ttfamily}Classes\end{ttfamily}\item \begin{ttfamily}SysUtils\end{ttfamily}\item \begin{ttfamily}mi.rtl.files\end{ttfamily}(\ref{mi.rtl.files})\item \begin{ttfamily}mi.rtl.objects.methods.StreamBase.Stream\end{ttfamily}(\ref{mi.rtl.Objects.Methods.StreamBase.Stream})\item \begin{ttfamily}mi.rtl.objects.methods.StreamBase.Stream.FileStream\end{ttfamily}(\ref{mi.rtl.Objects.Methods.StreamBase.Stream.FileStream})\end{itemize}
\section{Visão Geral}
\begin{description}
\item[\texttt{\begin{ttfamily}TMemoryStream\end{ttfamily} Classe}]
\end{description}
\section{Classes, Interfaces, Objetos e Registros}
\subsection*{TMemoryStream Classe}
\subsubsection*{\large{\textbf{Hierarquia}}\normalsize\hspace{1ex}\hfill}
TMemoryStream {$>$} \begin{ttfamily}TStream\end{ttfamily}(\ref{mi.rtl.Objects.Methods.StreamBase.Stream.TStream}) {$>$} \begin{ttfamily}TStreamBase\end{ttfamily}(\ref{mi.rtl.Objects.Methods.StreamBase.TStreamBase}) {$>$} \begin{ttfamily}TObjectsMethods\end{ttfamily}(\ref{mi.rtl.Objects.Methods.TObjectsMethods}) {$>$} \begin{ttfamily}TObjectsConsts\end{ttfamily}(\ref{mi.rtl.Objects.Consts.TObjectsConsts}) {$>$} 
TObjectsTypes
\subsubsection*{\large{\textbf{Descrição}}\normalsize\hspace{1ex}\hfill}
\begin{itemize}
\item A classe \begin{ttfamily}TMemoryStream\end{ttfamily} é usada para gerenciar um fluxo de dados em memória.

\begin{itemize}
\item \textbf{NOTA} \begin{itemize}
\item Todas as alterações aqui devem ser completamente transparentes para os códigos existentes. Basicamente, os blocos de memória não precisam ser segmentos de base mas isso significa que nossa lista se torna blocos de memória em vez de segmentos. O stream também se expandirá como os outros streams padrão
\end{itemize}
\end{itemize}
\end{itemize}\subsubsection*{\large{\textbf{Campos}}\normalsize\hspace{1ex}\hfill}
\paragraph*{BlkCount}\hspace*{\fill}

\begin{list}{}{
\settowidth{\tmplength}{\textbf{Declaração}}
\setlength{\itemindent}{0cm}
\setlength{\listparindent}{0cm}
\setlength{\leftmargin}{\evensidemargin}
\addtolength{\leftmargin}{\tmplength}
\settowidth{\labelsep}{X}
\addtolength{\leftmargin}{\labelsep}
\setlength{\labelwidth}{\tmplength}
}
\begin{flushleft}
\item[\textbf{Declaração}\hfill]
\begin{ttfamily}
public BlkCount: Sw{\_}Word;\end{ttfamily}


\end{flushleft}
\par
\item[\textbf{Descrição}]
Number of segments

\end{list}
\paragraph*{BlkSize}\hspace*{\fill}

\begin{list}{}{
\settowidth{\tmplength}{\textbf{Declaração}}
\setlength{\itemindent}{0cm}
\setlength{\listparindent}{0cm}
\setlength{\leftmargin}{\evensidemargin}
\addtolength{\leftmargin}{\tmplength}
\settowidth{\labelsep}{X}
\addtolength{\leftmargin}{\labelsep}
\setlength{\labelwidth}{\tmplength}
}
\begin{flushleft}
\item[\textbf{Declaração}\hfill]
\begin{ttfamily}
public BlkSize: Word;\end{ttfamily}


\end{flushleft}
\par
\item[\textbf{Descrição}]
Memory block size

\end{list}
\paragraph*{MemSize}\hspace*{\fill}

\begin{list}{}{
\settowidth{\tmplength}{\textbf{Declaração}}
\setlength{\itemindent}{0cm}
\setlength{\listparindent}{0cm}
\setlength{\leftmargin}{\evensidemargin}
\addtolength{\leftmargin}{\tmplength}
\settowidth{\labelsep}{X}
\addtolength{\leftmargin}{\labelsep}
\setlength{\labelwidth}{\tmplength}
}
\begin{flushleft}
\item[\textbf{Declaração}\hfill]
\begin{ttfamily}
public MemSize: LongInt;\end{ttfamily}


\end{flushleft}
\par
\item[\textbf{Descrição}]
Memory alloc size

\end{list}
\paragraph*{BlkList}\hspace*{\fill}

\begin{list}{}{
\settowidth{\tmplength}{\textbf{Declaração}}
\setlength{\itemindent}{0cm}
\setlength{\listparindent}{0cm}
\setlength{\leftmargin}{\evensidemargin}
\addtolength{\leftmargin}{\tmplength}
\settowidth{\labelsep}{X}
\addtolength{\leftmargin}{\labelsep}
\setlength{\labelwidth}{\tmplength}
}
\begin{flushleft}
\item[\textbf{Declaração}\hfill]
\begin{ttfamily}
public BlkList: PPointerArray;\end{ttfamily}


\end{flushleft}
\par
\item[\textbf{Descrição}]
Memory block list

\end{list}
\paragraph*{Handle}\hspace*{\fill}

\begin{list}{}{
\settowidth{\tmplength}{\textbf{Declaração}}
\setlength{\itemindent}{0cm}
\setlength{\listparindent}{0cm}
\setlength{\leftmargin}{\evensidemargin}
\addtolength{\leftmargin}{\tmplength}
\settowidth{\labelsep}{X}
\addtolength{\leftmargin}{\labelsep}
\setlength{\labelwidth}{\tmplength}
}
\begin{flushleft}
\item[\textbf{Declaração}\hfill]
\begin{ttfamily}
public Handle: THandle;\end{ttfamily}


\end{flushleft}
\par
\item[\textbf{Descrição}]
Quando \begin{ttfamily}Handle\end{ttfamily}=HANDLE{\_}INVALID o bloco de memória não foi alocado

\end{list}
\subsubsection*{\large{\textbf{Métodos}}\normalsize\hspace{1ex}\hfill}
\paragraph*{LoadFromFile}\hspace*{\fill}

\begin{list}{}{
\settowidth{\tmplength}{\textbf{Declaração}}
\setlength{\itemindent}{0cm}
\setlength{\listparindent}{0cm}
\setlength{\leftmargin}{\evensidemargin}
\addtolength{\leftmargin}{\tmplength}
\settowidth{\labelsep}{X}
\addtolength{\leftmargin}{\labelsep}
\setlength{\labelwidth}{\tmplength}
}
\begin{flushleft}
\item[\textbf{Declaração}\hfill]
\begin{ttfamily}
protected function LoadFromFile(aFileName:AnsiString):Boolean; Virtual;\end{ttfamily}


\end{flushleft}
\end{list}
\paragraph*{SetBufSize}\hspace*{\fill}

\begin{list}{}{
\settowidth{\tmplength}{\textbf{Declaração}}
\setlength{\itemindent}{0cm}
\setlength{\listparindent}{0cm}
\setlength{\leftmargin}{\evensidemargin}
\addtolength{\leftmargin}{\tmplength}
\settowidth{\labelsep}{X}
\addtolength{\leftmargin}{\labelsep}
\setlength{\labelwidth}{\tmplength}
}
\begin{flushleft}
\item[\textbf{Declaração}\hfill]
\begin{ttfamily}
public Function SetBufSize(Const aBufSize : Sw{\_}Word):Sw{\_}Word; Override;\end{ttfamily}


\end{flushleft}
\end{list}
\paragraph*{SetBufSize}\hspace*{\fill}

\begin{list}{}{
\settowidth{\tmplength}{\textbf{Declaração}}
\setlength{\itemindent}{0cm}
\setlength{\listparindent}{0cm}
\setlength{\leftmargin}{\evensidemargin}
\addtolength{\leftmargin}{\tmplength}
\settowidth{\labelsep}{X}
\addtolength{\leftmargin}{\labelsep}
\setlength{\labelwidth}{\tmplength}
}
\begin{flushleft}
\item[\textbf{Declaração}\hfill]
\begin{ttfamily}
public Function SetBufSize(ALimit: LongInt; ABlockSize: Word):Sw{\_}Word; Overload; Virtual;\end{ttfamily}


\end{flushleft}
\end{list}
\paragraph*{SetFileName}\hspace*{\fill}

\begin{list}{}{
\settowidth{\tmplength}{\textbf{Declaração}}
\setlength{\itemindent}{0cm}
\setlength{\listparindent}{0cm}
\setlength{\leftmargin}{\evensidemargin}
\addtolength{\leftmargin}{\tmplength}
\settowidth{\labelsep}{X}
\addtolength{\leftmargin}{\labelsep}
\setlength{\labelwidth}{\tmplength}
}
\begin{flushleft}
\item[\textbf{Declaração}\hfill]
\begin{ttfamily}
public Procedure SetFileName(a{\_}FileName: AnsiString); Override;\end{ttfamily}


\end{flushleft}
\end{list}
\paragraph*{Create}\hspace*{\fill}

\begin{list}{}{
\settowidth{\tmplength}{\textbf{Declaração}}
\setlength{\itemindent}{0cm}
\setlength{\listparindent}{0cm}
\setlength{\leftmargin}{\evensidemargin}
\addtolength{\leftmargin}{\tmplength}
\settowidth{\labelsep}{X}
\addtolength{\leftmargin}{\labelsep}
\setlength{\labelwidth}{\tmplength}
}
\begin{flushleft}
\item[\textbf{Declaração}\hfill]
\begin{ttfamily}
public CONSTRUCTOR Create(ALimit, ABlockSize: Longint); overload; virtual;\end{ttfamily}


\end{flushleft}
\end{list}
\paragraph*{Destroy}\hspace*{\fill}

\begin{list}{}{
\settowidth{\tmplength}{\textbf{Declaração}}
\setlength{\itemindent}{0cm}
\setlength{\listparindent}{0cm}
\setlength{\leftmargin}{\evensidemargin}
\addtolength{\leftmargin}{\tmplength}
\settowidth{\labelsep}{X}
\addtolength{\leftmargin}{\labelsep}
\setlength{\labelwidth}{\tmplength}
}
\begin{flushleft}
\item[\textbf{Declaração}\hfill]
\begin{ttfamily}
public DESTRUCTOR Destroy; Override;\end{ttfamily}


\end{flushleft}
\end{list}
\paragraph*{Truncate}\hspace*{\fill}

\begin{list}{}{
\settowidth{\tmplength}{\textbf{Declaração}}
\setlength{\itemindent}{0cm}
\setlength{\listparindent}{0cm}
\setlength{\leftmargin}{\evensidemargin}
\addtolength{\leftmargin}{\tmplength}
\settowidth{\labelsep}{X}
\addtolength{\leftmargin}{\labelsep}
\setlength{\labelwidth}{\tmplength}
}
\begin{flushleft}
\item[\textbf{Declaração}\hfill]
\begin{ttfamily}
public PROCEDURE Truncate; Override;\end{ttfamily}


\end{flushleft}
\end{list}
\paragraph*{Read}\hspace*{\fill}

\begin{list}{}{
\settowidth{\tmplength}{\textbf{Declaração}}
\setlength{\itemindent}{0cm}
\setlength{\listparindent}{0cm}
\setlength{\leftmargin}{\evensidemargin}
\addtolength{\leftmargin}{\tmplength}
\settowidth{\labelsep}{X}
\addtolength{\leftmargin}{\labelsep}
\setlength{\labelwidth}{\tmplength}
}
\begin{flushleft}
\item[\textbf{Declaração}\hfill]
\begin{ttfamily}
public PROCEDURE Read(Var Buf; Count: Sw{\_}Word;Var BytesRead:Sw{\_}Word); Overload; override;\end{ttfamily}


\end{flushleft}
\end{list}
\paragraph*{Read}\hspace*{\fill}

\begin{list}{}{
\settowidth{\tmplength}{\textbf{Declaração}}
\setlength{\itemindent}{0cm}
\setlength{\listparindent}{0cm}
\setlength{\leftmargin}{\evensidemargin}
\addtolength{\leftmargin}{\tmplength}
\settowidth{\labelsep}{X}
\addtolength{\leftmargin}{\labelsep}
\setlength{\labelwidth}{\tmplength}
}
\begin{flushleft}
\item[\textbf{Declaração}\hfill]
\begin{ttfamily}
public PROCEDURE Read(Var Buf; Count: Sw{\_}Word); Overload; Override;\end{ttfamily}


\end{flushleft}
\end{list}
\paragraph*{Write}\hspace*{\fill}

\begin{list}{}{
\settowidth{\tmplength}{\textbf{Declaração}}
\setlength{\itemindent}{0cm}
\setlength{\listparindent}{0cm}
\setlength{\leftmargin}{\evensidemargin}
\addtolength{\leftmargin}{\tmplength}
\settowidth{\labelsep}{X}
\addtolength{\leftmargin}{\labelsep}
\setlength{\labelwidth}{\tmplength}
}
\begin{flushleft}
\item[\textbf{Declaração}\hfill]
\begin{ttfamily}
public PROCEDURE Write(Var Buf; Count: Sw{\_}Word;Var BytesWrite:Sw{\_}Word); Overload; override;\end{ttfamily}


\end{flushleft}
\end{list}
\paragraph*{Write}\hspace*{\fill}

\begin{list}{}{
\settowidth{\tmplength}{\textbf{Declaração}}
\setlength{\itemindent}{0cm}
\setlength{\listparindent}{0cm}
\setlength{\leftmargin}{\evensidemargin}
\addtolength{\leftmargin}{\tmplength}
\settowidth{\labelsep}{X}
\addtolength{\leftmargin}{\labelsep}
\setlength{\labelwidth}{\tmplength}
}
\begin{flushleft}
\item[\textbf{Declaração}\hfill]
\begin{ttfamily}
public PROCEDURE Write(Var Buf; Count: Sw{\_}Word); Overload; Override;\end{ttfamily}


\end{flushleft}
\end{list}
\chapter{Unit mi.rtl.objects.methods.StreamBase.Stream.MemoryStream.BufferMemory}
\section{Descrição}
\begin{itemize}
\item A Unit \textbf{\begin{ttfamily}mi.rtl.objects.methods.StreamBase.Stream.MemoryStream.BufferMemory\end{ttfamily}} implementa a classe \begin{ttfamily}TBufferMemory\end{ttfamily}(\ref{mi.rtl.objects.methods.StreamBase.Stream.MemoryStream.BufferMemory.TBufferMemory}).

\begin{itemize}
\item \textbf{VERSÃO} \begin{itemize}
\item Alpha {-} 0.5.0.687
\end{itemize}
\item \textbf{HISTÓRICO} \begin{itemize}
\item Criado por: Paulo Sérgio da Silva Pacheco e{-}mail: paulosspacheco@yahoo.com.br \begin{itemize}
\item \textbf{23/11/2021} \begin{itemize}
\item 12:55 a 14:30 {-} Criar a unit \begin{ttfamily}mi.rtl.objects.methods.StreamBase.Stream.MemoryStream.BufferMemory\end{ttfamily}
\item 14:30 a 19:35 {-} Criar um exemplo de como usar a classe \begin{ttfamily}TBufferMemory\end{ttfamily}(\ref{mi.rtl.objects.methods.StreamBase.Stream.MemoryStream.BufferMemory.TBufferMemory})
\item 21:35 a 22:44 {-} Documentar a classe \begin{ttfamily}TBufferMemory\end{ttfamily}(\ref{mi.rtl.objects.methods.StreamBase.Stream.MemoryStream.BufferMemory.TBufferMemory})
\end{itemize}
\item \textbf{29/11/2021} \begin{itemize}
\item 14:45 a 15:10 \begin{itemize}
\item Criar exemplo TMi{\_}Rtl{\_}Tests.Test{\_}TBufferMemory{\_}sem{\_}header;
\item Criar exemplo TMi{\_}Rtl{\_}Tests.Test{\_}TBufferMemory{\_}com{\_}header;
\end{itemize}
\end{itemize}
\end{itemize}
\end{itemize}
\item \textbf{CÓDIGO FONTE}: \begin{itemize}
\item 
\end{itemize}
\end{itemize}
\end{itemize}
\section{Uses}
\begin{itemize}
\item \begin{ttfamily}Classes\end{ttfamily}\item \begin{ttfamily}SysUtils\end{ttfamily}\item \begin{ttfamily}mi.rtl.objects.methods.StreamBase.Stream.MemoryStream\end{ttfamily}(\ref{mi.rtl.Objects.Methods.StreamBase.Stream.MemoryStream})\end{itemize}
\section{Visão Geral}
\begin{description}
\item[\texttt{\begin{ttfamily}TBufferMemory\end{ttfamily} Classe}]
\end{description}
\section{Classes, Interfaces, Objetos e Registros}
\subsection*{TBufferMemory Classe}
\subsubsection*{\large{\textbf{Hierarquia}}\normalsize\hspace{1ex}\hfill}
TBufferMemory {$>$} \begin{ttfamily}TMemoryStream\end{ttfamily}(\ref{mi.rtl.Objects.Methods.StreamBase.Stream.MemoryStream.TMemoryStream}) {$>$} \begin{ttfamily}TStream\end{ttfamily}(\ref{mi.rtl.Objects.Methods.StreamBase.Stream.TStream}) {$>$} \begin{ttfamily}TStreamBase\end{ttfamily}(\ref{mi.rtl.Objects.Methods.StreamBase.TStreamBase}) {$>$} \begin{ttfamily}TObjectsMethods\end{ttfamily}(\ref{mi.rtl.Objects.Methods.TObjectsMethods}) {$>$} \begin{ttfamily}TObjectsConsts\end{ttfamily}(\ref{mi.rtl.Objects.Consts.TObjectsConsts}) {$>$} 
TObjectsTypes
\subsubsection*{\large{\textbf{Descrição}}\normalsize\hspace{1ex}\hfill}
\begin{itemize}
\item A class \begin{ttfamily}TBufferMemory\end{ttfamily} cria um \textbf{array of record} em memória usando os métodos os \begin{ttfamily}seek\end{ttfamily}(\ref{mi.rtl.objects.methods.StreamBase.Stream.MemoryStream.BufferMemory.TBufferMemory-Seek}), \begin{ttfamily}PutREc\end{ttfamily}(\ref{mi.rtl.Objects.Methods.StreamBase.Stream.TStream-PutRec}), \begin{ttfamily}GetRec\end{ttfamily}(\ref{mi.rtl.Objects.Methods.StreamBase.Stream.TStream-GetRec})

\begin{itemize}
\item \textbf{NOTA} \begin{itemize}
\item Uso a classe \begin{ttfamily}TBufferMemory\end{ttfamily} para criar arquivos em memória no banco de dados \textbf{Tb{\_}Access.pas}
\end{itemize}
\item \textbf{EXEMPLO} \begin{itemize}
\item Exemplo de como gravar um registro \textbf{sem header em memória}.

\texttt{\\\nopagebreak[3]
\\\nopagebreak[3]
}\textbf{Procedure}\texttt{~TMi{\_}Rtl{\_}Tests.Test{\_}TBufferMemory{\_}sem{\_}header;\\\nopagebreak[3]
~~}\textbf{type}\texttt{\\\nopagebreak[3]
~~~TAluno~=~}\textbf{record}\texttt{\\\nopagebreak[3]
~~~~~~~~~~~~~~Id~:~integer;\\\nopagebreak[3]
~~~~~~~~~~~~~~nome~:~}\textbf{string}\texttt{[35];\\\nopagebreak[3]
~~~~~~~~~~~~}\textbf{end}\texttt{;\\\nopagebreak[3]
\\\nopagebreak[3]
~~}\textbf{var}\texttt{\\\nopagebreak[3]
~~~FileMemory{\_}Alunos~:~TObjectss.TBufferMemory;\\\nopagebreak[3]
~~~Aluno~~~~~~~~~~:~TAluno;\\\nopagebreak[3]
\\\nopagebreak[3]
~~~nr~:~longint;\\\nopagebreak[3]
~~~n~~:~longint;\\\nopagebreak[3]
\\\nopagebreak[3]
}\textbf{begin}\texttt{\\\nopagebreak[3]
~~}\textbf{with}\texttt{~TObjectss~}\textbf{do}\texttt{\\\nopagebreak[3]
~~}\textbf{try}\texttt{\\\nopagebreak[3]
~~~~FileMemory{\_}Alunos~:=~TObjectss.TBufferMemory.Create(sizeof(aluno));\\\nopagebreak[3]
~~~~}\textbf{with}\texttt{~aluno,FileMemory{\_}Alunos~}\textbf{do}\texttt{\\\nopagebreak[3]
~~~~}\textbf{begin}\texttt{\\\nopagebreak[3]
~~~~~~}\textbf{if}\texttt{~status~=~StOk~}\textbf{then}\texttt{\\\nopagebreak[3]
~~~~~~}\textbf{begin}\texttt{\\\nopagebreak[3]
~~~~~~~~n~:=~1;\\\nopagebreak[3]
~~~~~~~~Id:=~n;\\\nopagebreak[3]
~~~~~~~~nome:=~'Paulo~Sérgio';\\\nopagebreak[3]
~~~~~~~~PutRec(id,aluno);\\\nopagebreak[3]
~~~~~~}\textbf{end}\texttt{;\\\nopagebreak[3]
\\\nopagebreak[3]
~~~~~~}\textbf{if}\texttt{~status~=~StOk~}\textbf{then}\texttt{\\\nopagebreak[3]
~~~~~~}\textbf{begin}\texttt{\\\nopagebreak[3]
~~~~~~~~inc(n);\\\nopagebreak[3]
~~~~~~~~Id:=~n;\\\nopagebreak[3]
~~~~~~~~nome:=~'George~Bruno';\\\nopagebreak[3]
~~~~~~~~PutRec(id,aluno);\\\nopagebreak[3]
~~~~~~}\textbf{end}\texttt{;\\\nopagebreak[3]
\\\nopagebreak[3]
~~~~~~}\textbf{if}\texttt{~status~=~StOk~}\textbf{then}\texttt{\\\nopagebreak[3]
~~~~~~}\textbf{begin}\texttt{\\\nopagebreak[3]
~~~~~~~~}\textbf{for}\texttt{~nr~:=~1~}\textbf{to}\texttt{~n~}\textbf{do}\texttt{\\\nopagebreak[3]
~~~~~~~~}\textbf{begin}\texttt{\\\nopagebreak[3]
~~~~~~~~~~GetRec(nr,aluno);\\\nopagebreak[3]
~~~~~~~~~~}\textbf{if}\texttt{~status~=~StOk\\\nopagebreak[3]
~~~~~~~~~~}\textbf{then}\texttt{~SysMessageBox('Nr~='+intToStr(nr)+\\\nopagebreak[3]
~~~~~~~~~~~~~~~~~~~~~~~~~~~~~';~id~='+intToStr(Aluno.id)+\\\nopagebreak[3]
~~~~~~~~~~~~~~~~~~~~~~~~~~~~~';~Aluno~='+Aluno.nome\\\nopagebreak[3]
~~~~~~~~~~~~~~~~~~~~~~~~~~~~~,'Test{\_}FileStream{\_}sem{\_}header',false)\\\nopagebreak[3]
~~~~~~~~~~}\textbf{else}\texttt{~break;\\\nopagebreak[3]
~~~~~~~~}\textbf{end}\texttt{;\\\nopagebreak[3]
~~~~~~}\textbf{end}\texttt{;\\\nopagebreak[3]
\\\nopagebreak[3]
~~~~}\textbf{end}\texttt{;\\\nopagebreak[3]
\\\nopagebreak[3]
~~}\textbf{finally}\texttt{\\\nopagebreak[3]
~~~~FileMemory{\_}Alunos.Destroy;\\\nopagebreak[3]
~~}\textbf{end}\texttt{;\\\nopagebreak[3]
\\\nopagebreak[3]
}\textbf{end}\texttt{;\\
}
\item Exemplo de como gravar um registro \textbf{com header em memória}.

\texttt{\\\nopagebreak[3]
\\\nopagebreak[3]
}\textbf{Procedure}\texttt{~Test{\_}TBufferMemory{\_}com{\_}header;\\\nopagebreak[3]
~~}\textbf{type}\texttt{\\\nopagebreak[3]
~~~~TAluno~=~}\textbf{record}\texttt{\\\nopagebreak[3]
~~~~~~~~~~~~~~~Id~:~integer;\\\nopagebreak[3]
~~~~~~~~~~~~~~~nome~:~}\textbf{string}\texttt{[35];\\\nopagebreak[3]
~~~~~~~~~~~~~}\textbf{end}\texttt{;\\\nopagebreak[3]
~~}\textbf{type}\texttt{\\\nopagebreak[3]
~~~~THeadAlunos~=~}\textbf{record}\texttt{\\\nopagebreak[3]
~~~~~~~~~~~~~~~~~~~~TotalDeAlunos:longint;\\\nopagebreak[3]
~~~~~~~~~~~~~~~~~~}\textbf{end}\texttt{;\\\nopagebreak[3]
\\\nopagebreak[3]
~~}\textbf{var}\texttt{\\\nopagebreak[3]
~~~~TBufferMemory{\_}Alunos~:~TObjectss.TBufferMemory;\\\nopagebreak[3]
~~~~HeadAlunos~:~THeadAlunos;\\\nopagebreak[3]
~~~~Aluno~~~~~~~~~~~~~:~TAluno;\\\nopagebreak[3]
~~~~nr~:~longint;\\\nopagebreak[3]
~~~~n~~:~longint;\\\nopagebreak[3]
\\\nopagebreak[3]
}\textbf{begin}\texttt{\\\nopagebreak[3]
~~}\textbf{with}\texttt{~TObjectss~}\textbf{do}\texttt{\\\nopagebreak[3]
~~}\textbf{try}\texttt{\\\nopagebreak[3]
~~~~TBufferMemory{\_}Alunos~:=~TBufferMemory.Create(sizeof(HeadAlunos),sizeof(aluno));\\\nopagebreak[3]
\\\nopagebreak[3]
~~~~}\textbf{with}\texttt{~aluno,TBufferMemory{\_}Alunos~}\textbf{do}\texttt{\\\nopagebreak[3]
~~~~}\textbf{if}\texttt{~status~=~StOk~}\textbf{then}\texttt{\\\nopagebreak[3]
~~~~}\textbf{begin}\texttt{\\\nopagebreak[3]
~~~~~~HeadAlunos.TotalDeAlunos:=~0;\\\nopagebreak[3]
~~~~~~PutRecBase(HeadAlunos);~\textit{//~Grava~o~header}\\\nopagebreak[3]
\\\nopagebreak[3]
~~~~~~}\textbf{if}\texttt{~status~=~StOk~}\textbf{then}\texttt{\\\nopagebreak[3]
~~~~~~}\textbf{begin}\texttt{\\\nopagebreak[3]
~~~~~~~~inc(HeadAlunos.TotalDeAlunos);\\\nopagebreak[3]
~~~~~~~~Id:=~HeadAlunos.TotalDeAlunos;\\\nopagebreak[3]
~~~~~~~~nome:=~'Paulo~Sérgio~da~Silva~Pacheco';\\\nopagebreak[3]
~~~~~~~~PutRec(id,aluno);\\\nopagebreak[3]
~~~~~~~~}\textbf{if}\texttt{~status~=~StOk\\\nopagebreak[3]
~~~~~~~~}\textbf{then}\texttt{~PutRecBase(HeadAlunos);~\textit{//~Grava~o~header}\\\nopagebreak[3]
~~~~~~}\textbf{end}\texttt{;\\\nopagebreak[3]
\\\nopagebreak[3]
\\\nopagebreak[3]
~~~~~~}\textbf{if}\texttt{~status~=~StOk~}\textbf{then}\texttt{\\\nopagebreak[3]
~~~~~~}\textbf{begin}\texttt{\\\nopagebreak[3]
~~~~~~~~inc(HeadAlunos.TotalDeAlunos);\\\nopagebreak[3]
~~~~~~~~Id:=~HeadAlunos.TotalDeAlunos;\\\nopagebreak[3]
~~~~~~~~nome:=~'George~Bruno~Melo~Pacheco';\\\nopagebreak[3]
\\\nopagebreak[3]
~~~~~~~~PutRec(id,aluno);\\\nopagebreak[3]
~~~~~~~~}\textbf{if}\texttt{~status~=~StOk\\\nopagebreak[3]
~~~~~~~~}\textbf{then}\texttt{~PutRecBase(HeadAlunos);~\textit{//~Grava~o~header}\\\nopagebreak[3]
~~~~~~}\textbf{end}\texttt{;\\\nopagebreak[3]
\\\nopagebreak[3]
~~~~~~}\textbf{if}\texttt{~status~=~StOk~}\textbf{then}\texttt{\\\nopagebreak[3]
~~~~~~}\textbf{begin}\texttt{\\\nopagebreak[3]
~~~~~~~~GetRecBase(n);\\\nopagebreak[3]
~~~~~~~~}\textbf{if}\texttt{~status~=~StOk\\\nopagebreak[3]
~~~~~~~~}\textbf{then}\texttt{\\\nopagebreak[3]
~~~~~~~~}\textbf{begin}\texttt{\\\nopagebreak[3]
~~~~~~~~~~\textit{//Imprime~o~número~de~elemntos~adicionado~ao~stream}\\\nopagebreak[3]
~~~~~~~~~~SysMessageBox('Número~de~registros:~'+intToStr(n)\\\nopagebreak[3]
~~~~~~~~~~~~~~~~~~~~~~~~~,\\\nopagebreak[3]
~~~~~~~~~~~~~~~~~~~~~~~~~'Test{\_}FileStream{\_}sem{\_}header',false);\\\nopagebreak[3]
\\\nopagebreak[3]
~~~~~~~~~~\textit{//~Ler~e~imprime~os~registros.}\\\nopagebreak[3]
~~~~~~~~~~}\textbf{for}\texttt{~nr~:=~1~}\textbf{to}\texttt{~n~}\textbf{do}\texttt{\\\nopagebreak[3]
~~~~~~~~~~}\textbf{begin}\texttt{\\\nopagebreak[3]
~~~~~~~~~~~~~~GetRec(nr,aluno);\\\nopagebreak[3]
~~~~~~~~~~~~~~}\textbf{if}\texttt{~status~=~StOk\\\nopagebreak[3]
~~~~~~~~~~~~~~}\textbf{then}\texttt{~SysMessageBox('Nr~='+intToStr(nr)+\\\nopagebreak[3]
~~~~~~~~~~~~~~~~~~~~~~~~~~~~~~~~~';~id~='+intToStr(Aluno.id)+\\\nopagebreak[3]
~~~~~~~~~~~~~~~~~~~~~~~~~~~~~~~~~';~Aluno~='+Aluno.nome\\\nopagebreak[3]
~~~~~~~~~~~~~~~~~~~~~~~~~~~~~~~~~,\\\nopagebreak[3]
~~~~~~~~~~~~~~~~~~~~~~~~~~~~~~~~~'Test{\_}FileStream{\_}sem{\_}header',false)\\\nopagebreak[3]
~~~~~~~~~~~~~~}\textbf{else}\texttt{~Break;\\\nopagebreak[3]
~~~~~~~~~~}\textbf{end}\texttt{;\\\nopagebreak[3]
\\\nopagebreak[3]
~~~~~~~~~~}\textbf{if}\texttt{~status~{$<$}{$>$}~StOk\\\nopagebreak[3]
~~~~~~~~~~}\textbf{then}\texttt{~SysMessageBox(errorMessage(errorInfo)\\\nopagebreak[3]
~~~~~~~~~~~~~~~~~~~~~~~~~~~~~,\\\nopagebreak[3]
~~~~~~~~~~~~~~~~~~~~~~~~~~~~~'Test{\_}FileStream{\_}sem{\_}header',false)\\\nopagebreak[3]
\\\nopagebreak[3]
~~~~~~~~}\textbf{end}\texttt{;\\\nopagebreak[3]
~~~~~~}\textbf{end}\texttt{;\\\nopagebreak[3]
\\\nopagebreak[3]
~~~~~~}\textbf{if}\texttt{~status~{$<$}{$>$}~StOk\\\nopagebreak[3]
~~~~~~}\textbf{then}\texttt{~SysMessageBox(errorMessage(errorInfo)\\\nopagebreak[3]
~~~~~~~~~~~~~~~~~~~~~~~~~,\\\nopagebreak[3]
~~~~~~~~~~~~~~~~~~~~~~~~~'Test{\_}FileStream{\_}sem{\_}header',false)\\\nopagebreak[3]
\\\nopagebreak[3]
~~~~}\textbf{end}\texttt{;\\\nopagebreak[3]
\\\nopagebreak[3]
~~}\textbf{finally}\texttt{\\\nopagebreak[3]
~~~~TBufferMemory{\_}Alunos.Destroy;\\\nopagebreak[3]
~~}\textbf{end}\texttt{;\\\nopagebreak[3]
\\\nopagebreak[3]
}\textbf{end}\texttt{;\\
}
\end{itemize}
\end{itemize}
\end{itemize}\subsubsection*{\large{\textbf{Métodos}}\normalsize\hspace{1ex}\hfill}
\paragraph*{Create}\hspace*{\fill}

\begin{list}{}{
\settowidth{\tmplength}{\textbf{Declaração}}
\setlength{\itemindent}{0cm}
\setlength{\listparindent}{0cm}
\setlength{\leftmargin}{\evensidemargin}
\addtolength{\leftmargin}{\tmplength}
\settowidth{\labelsep}{X}
\addtolength{\leftmargin}{\labelsep}
\setlength{\labelwidth}{\tmplength}
}
\begin{flushleft}
\item[\textbf{Declaração}\hfill]
\begin{ttfamily}
public CONSTRUCTOR Create(a{\_}BaseSize,a{\_}RecSize:Longint); overload; override;\end{ttfamily}


\end{flushleft}
\par
\item[\textbf{Descrição}]
\begin{itemize}
\item O constructor \textbf{\begin{ttfamily}Create\end{ttfamily}} cria um stream de um \textbf{array of record} em memória onde a mesma será gravado após o header passado pelo parâmetro a{\_}BaseSize.;

\begin{itemize}
\item \textbf{PARÂMETROS} \begin{itemize}
\item \textbf{a{\_}BaseSize} {-} Tamanho do registro usado no registro de posição zero
\item \textbf{a{\_}RecSize} {-} Tamanho do registro depois do registro usado na posição depois da base;
\end{itemize}
\end{itemize}
\end{itemize}

\end{list}
\paragraph*{Create}\hspace*{\fill}

\begin{list}{}{
\settowidth{\tmplength}{\textbf{Declaração}}
\setlength{\itemindent}{0cm}
\setlength{\listparindent}{0cm}
\setlength{\leftmargin}{\evensidemargin}
\addtolength{\leftmargin}{\tmplength}
\settowidth{\labelsep}{X}
\addtolength{\leftmargin}{\labelsep}
\setlength{\labelwidth}{\tmplength}
}
\begin{flushleft}
\item[\textbf{Declaração}\hfill]
\begin{ttfamily}
public CONSTRUCTOR Create(a{\_}RecSize:Longint); overload; virtual;\end{ttfamily}


\end{flushleft}
\par
\item[\textbf{Descrição}]
\begin{itemize}
\item O constructor \textbf{\begin{ttfamily}Create\end{ttfamily}} cria um stream de um \textbf{array of record} em memória onde a mesma será gravado após ao início do bloco em memória obs: \begin{ttfamily}BaseSize\end{ttfamily}(\ref{mi.rtl.Objects.Methods.StreamBase.Stream.TStream-BaseSize})=0.;
\end{itemize}

\end{list}
\paragraph*{Seek}\hspace*{\fill}

\begin{list}{}{
\settowidth{\tmplength}{\textbf{Declaração}}
\setlength{\itemindent}{0cm}
\setlength{\listparindent}{0cm}
\setlength{\leftmargin}{\evensidemargin}
\addtolength{\leftmargin}{\tmplength}
\settowidth{\labelsep}{X}
\addtolength{\leftmargin}{\labelsep}
\setlength{\labelwidth}{\tmplength}
}
\begin{flushleft}
\item[\textbf{Declaração}\hfill]
\begin{ttfamily}
public PROCEDURE Seek(NR: LongInt); Overload; override;\end{ttfamily}


\end{flushleft}
\end{list}
\paragraph*{Error}\hspace*{\fill}

\begin{list}{}{
\settowidth{\tmplength}{\textbf{Declaração}}
\setlength{\itemindent}{0cm}
\setlength{\listparindent}{0cm}
\setlength{\leftmargin}{\evensidemargin}
\addtolength{\leftmargin}{\tmplength}
\settowidth{\labelsep}{X}
\addtolength{\leftmargin}{\labelsep}
\setlength{\labelwidth}{\tmplength}
}
\begin{flushleft}
\item[\textbf{Declaração}\hfill]
\begin{ttfamily}
public PROCEDURE Error(Code, Info: Integer); Override;\end{ttfamily}


\end{flushleft}
\end{list}
\chapter{Unit mi.rtl.Objects.Methods.System}
\section{Descrição}
\textbf{}\textbf{}\textbf{}\textbf{}\textbf{}\textbf{}\textbf{}\textbf{}\textbf{}\textbf{}\textbf{}\textbf{}\textbf{}\textbf{}\textbf{}\textbf{}\textbf{}\textbf{}\textbf{}\textbf{}\textbf{}\textbf{} SISTEMA : Nort Soft Data Base MODULO : MARICARAY AUTOR : Paulo Pacheco --------- HISTORIA --------- DATA HARA HORA OCORRENCIA -------- ----- ----- ------------------------------------------------------------------ 01/08/02 08:00 Implementacao inicial 08/08/02 23:00 Implementacao Final 25/01/22 Convertido para lazarus \textbf{}\textbf{}\textbf{}\textbf{}\textbf{}\textbf{}\textbf{}\textbf{}\textbf{}\textbf{}\textbf{}\textbf{}\textbf{}\textbf{}\textbf{}\textbf{}\textbf{}\textbf{}\textbf{}\textbf{}\textbf{}\textbf{}*
\section{Uses}
\begin{itemize}
\item \begin{ttfamily}Classes\end{ttfamily}\item \begin{ttfamily}SysUtils\end{ttfamily}\item \begin{ttfamily}crt\end{ttfamily}\item \begin{ttfamily}mi.rtl.objects.Methods\end{ttfamily}(\ref{mi.rtl.Objects.Methods})\item \begin{ttfamily}mi.rtl.objects.Methods.Exception\end{ttfamily}(\ref{mi.rtl.Objects.Methods.Exception})\item \begin{ttfamily}mi.rtl.objects.methods.StreamBase.Stream\end{ttfamily}(\ref{mi.rtl.Objects.Methods.StreamBase.Stream})\item \begin{ttfamily}mi.rtl.objects.methods.StreamBase.Stream.FileStream\end{ttfamily}(\ref{mi.rtl.Objects.Methods.StreamBase.Stream.FileStream})\item \begin{ttfamily}mi.rtl.objects.methods.StreamBase.Stream.MemoryStream.BufferMemory\end{ttfamily}(\ref{mi.rtl.objects.methods.StreamBase.Stream.MemoryStream.BufferMemory})\item \begin{ttfamily}mi.rtl.objects.methods.Collection.FilesStreams\end{ttfamily}(\ref{mi.rtl.Objects.Methods.Collection.FilesStreams})\end{itemize}
\section{Visão Geral}
\begin{description}
\item[\texttt{\begin{ttfamily}TObjectsSystem\end{ttfamily} Classe}]
\end{description}
\section{Classes, Interfaces, Objetos e Registros}
\subsection*{TObjectsSystem Classe}
\subsubsection*{\large{\textbf{Hierarquia}}\normalsize\hspace{1ex}\hfill}
TObjectsSystem {$>$} \begin{ttfamily}TObjectsMethods\end{ttfamily}(\ref{mi.rtl.Objects.Methods.TObjectsMethods}) {$>$} \begin{ttfamily}TObjectsConsts\end{ttfamily}(\ref{mi.rtl.Objects.Consts.TObjectsConsts}) {$>$} 
TObjectsTypes
\subsubsection*{\large{\textbf{Descrição}}\normalsize\hspace{1ex}\hfill}
no description available, TObjectsMethods description follows\begin{itemize}
\item A classe \textbf{\begin{ttfamily}TObjectsMethods\end{ttfamily}} implementa os método de classe comum a todas as classes de TObjects do pacote \textbf{\begin{ttfamily}mi.rtl\end{ttfamily}(\ref{mi.rtl})}.
\end{itemize}\subsubsection*{\large{\textbf{Campos}}\normalsize\hspace{1ex}\hfill}
\paragraph*{BlocksRead}\hspace*{\fill}

\begin{list}{}{
\settowidth{\tmplength}{\textbf{Declaração}}
\setlength{\itemindent}{0cm}
\setlength{\listparindent}{0cm}
\setlength{\leftmargin}{\evensidemargin}
\addtolength{\leftmargin}{\tmplength}
\settowidth{\labelsep}{X}
\addtolength{\leftmargin}{\labelsep}
\setlength{\labelwidth}{\tmplength}
}
\begin{flushleft}
\item[\textbf{Declaração}\hfill]
\begin{ttfamily}
public const BlocksRead    : Word = 0;\end{ttfamily}


\end{flushleft}
\end{list}
\paragraph*{BlocksWrite}\hspace*{\fill}

\begin{list}{}{
\settowidth{\tmplength}{\textbf{Declaração}}
\setlength{\itemindent}{0cm}
\setlength{\listparindent}{0cm}
\setlength{\leftmargin}{\evensidemargin}
\addtolength{\leftmargin}{\tmplength}
\settowidth{\labelsep}{X}
\addtolength{\leftmargin}{\labelsep}
\setlength{\labelwidth}{\tmplength}
}
\begin{flushleft}
\item[\textbf{Declaração}\hfill]
\begin{ttfamily}
public const BlocksWrite   : Word = 0;\end{ttfamily}


\end{flushleft}
\end{list}
\subsubsection*{\large{\textbf{Métodos}}\normalsize\hspace{1ex}\hfill}
\paragraph*{FlushDOSFile}\hspace*{\fill}

\begin{list}{}{
\settowidth{\tmplength}{\textbf{Declaração}}
\setlength{\itemindent}{0cm}
\setlength{\listparindent}{0cm}
\setlength{\leftmargin}{\evensidemargin}
\addtolength{\leftmargin}{\tmplength}
\settowidth{\labelsep}{X}
\addtolength{\leftmargin}{\labelsep}
\setlength{\labelwidth}{\tmplength}
}
\begin{flushleft}
\item[\textbf{Declaração}\hfill]
\begin{ttfamily}
public class function FlushDOSFile(VAR F : File):Boolean;\end{ttfamily}


\end{flushleft}
\end{list}
\paragraph*{BlockRead}\hspace*{\fill}

\begin{list}{}{
\settowidth{\tmplength}{\textbf{Declaração}}
\setlength{\itemindent}{0cm}
\setlength{\listparindent}{0cm}
\setlength{\leftmargin}{\evensidemargin}
\addtolength{\leftmargin}{\tmplength}
\settowidth{\labelsep}{X}
\addtolength{\leftmargin}{\labelsep}
\setlength{\labelwidth}{\tmplength}
}
\begin{flushleft}
\item[\textbf{Declaração}\hfill]
\begin{ttfamily}
public class function BlockRead(var F: File; var Buf; Count: Word):Word;\end{ttfamily}


\end{flushleft}
\end{list}
\paragraph*{BlockWrite}\hspace*{\fill}

\begin{list}{}{
\settowidth{\tmplength}{\textbf{Declaração}}
\setlength{\itemindent}{0cm}
\setlength{\listparindent}{0cm}
\setlength{\leftmargin}{\evensidemargin}
\addtolength{\leftmargin}{\tmplength}
\settowidth{\labelsep}{X}
\addtolength{\leftmargin}{\labelsep}
\setlength{\labelwidth}{\tmplength}
}
\begin{flushleft}
\item[\textbf{Declaração}\hfill]
\begin{ttfamily}
public class function BlockWrite(var F: File; var Buf; Count: Word):Word;\end{ttfamily}


\end{flushleft}
\end{list}
\paragraph*{Seek}\hspace*{\fill}

\begin{list}{}{
\settowidth{\tmplength}{\textbf{Declaração}}
\setlength{\itemindent}{0cm}
\setlength{\listparindent}{0cm}
\setlength{\leftmargin}{\evensidemargin}
\addtolength{\leftmargin}{\tmplength}
\settowidth{\labelsep}{X}
\addtolength{\leftmargin}{\labelsep}
\setlength{\labelwidth}{\tmplength}
}
\begin{flushleft}
\item[\textbf{Declaração}\hfill]
\begin{ttfamily}
public class function Seek(VAR F:FILE ;Const NR: Longint):SmallInt ;\end{ttfamily}


\end{flushleft}
\end{list}
\paragraph*{AppendText}\hspace*{\fill}

\begin{list}{}{
\settowidth{\tmplength}{\textbf{Declaração}}
\setlength{\itemindent}{0cm}
\setlength{\listparindent}{0cm}
\setlength{\leftmargin}{\evensidemargin}
\addtolength{\leftmargin}{\tmplength}
\settowidth{\labelsep}{X}
\addtolength{\leftmargin}{\labelsep}
\setlength{\labelwidth}{\tmplength}
}
\begin{flushleft}
\item[\textbf{Declaração}\hfill]
\begin{ttfamily}
public class function AppendText(VAR F:Text;AFileMode:Word):SmallInt ;\end{ttfamily}


\end{flushleft}
\end{list}
\paragraph*{Rewrite}\hspace*{\fill}

\begin{list}{}{
\settowidth{\tmplength}{\textbf{Declaração}}
\setlength{\itemindent}{0cm}
\setlength{\listparindent}{0cm}
\setlength{\leftmargin}{\evensidemargin}
\addtolength{\leftmargin}{\tmplength}
\settowidth{\labelsep}{X}
\addtolength{\leftmargin}{\labelsep}
\setlength{\labelwidth}{\tmplength}
}
\begin{flushleft}
\item[\textbf{Declaração}\hfill]
\begin{ttfamily}
public class function Rewrite(VAR F:Text;AFileMode:SmallWord):SmallInt ; overload;\end{ttfamily}


\end{flushleft}
\end{list}
\paragraph*{Rewrite}\hspace*{\fill}

\begin{list}{}{
\settowidth{\tmplength}{\textbf{Declaração}}
\setlength{\itemindent}{0cm}
\setlength{\listparindent}{0cm}
\setlength{\leftmargin}{\evensidemargin}
\addtolength{\leftmargin}{\tmplength}
\settowidth{\labelsep}{X}
\addtolength{\leftmargin}{\labelsep}
\setlength{\labelwidth}{\tmplength}
}
\begin{flushleft}
\item[\textbf{Declaração}\hfill]
\begin{ttfamily}
public class function Rewrite(VAR F:File;Const aRecLen:Integer;AFileMode:SmallWord):SmallInt ; overload;\end{ttfamily}


\end{flushleft}
\end{list}
\paragraph*{Close}\hspace*{\fill}

\begin{list}{}{
\settowidth{\tmplength}{\textbf{Declaração}}
\setlength{\itemindent}{0cm}
\setlength{\listparindent}{0cm}
\setlength{\leftmargin}{\evensidemargin}
\addtolength{\leftmargin}{\tmplength}
\settowidth{\labelsep}{X}
\addtolength{\leftmargin}{\labelsep}
\setlength{\labelwidth}{\tmplength}
}
\begin{flushleft}
\item[\textbf{Declaração}\hfill]
\begin{ttfamily}
public class function Close(Var F:File):Boolean; Overload;\end{ttfamily}


\end{flushleft}
\end{list}
\paragraph*{Close}\hspace*{\fill}

\begin{list}{}{
\settowidth{\tmplength}{\textbf{Declaração}}
\setlength{\itemindent}{0cm}
\setlength{\listparindent}{0cm}
\setlength{\leftmargin}{\evensidemargin}
\addtolength{\leftmargin}{\tmplength}
\settowidth{\labelsep}{X}
\addtolength{\leftmargin}{\labelsep}
\setlength{\labelwidth}{\tmplength}
}
\begin{flushleft}
\item[\textbf{Declaração}\hfill]
\begin{ttfamily}
public class function Close(Var F:Text):Boolean; Overload;\end{ttfamily}


\end{flushleft}
\end{list}
\paragraph*{Reset}\hspace*{\fill}

\begin{list}{}{
\settowidth{\tmplength}{\textbf{Declaração}}
\setlength{\itemindent}{0cm}
\setlength{\listparindent}{0cm}
\setlength{\leftmargin}{\evensidemargin}
\addtolength{\leftmargin}{\tmplength}
\settowidth{\labelsep}{X}
\addtolength{\leftmargin}{\labelsep}
\setlength{\labelwidth}{\tmplength}
}
\begin{flushleft}
\item[\textbf{Declaração}\hfill]
\begin{ttfamily}
public class function Reset(VAR F:FILE ;Const aRecLen:Integer;Const AFileMode:SmallWord):Integer ; Overload;\end{ttfamily}


\end{flushleft}
\end{list}
\paragraph*{OpenText}\hspace*{\fill}

\begin{list}{}{
\settowidth{\tmplength}{\textbf{Declaração}}
\setlength{\itemindent}{0cm}
\setlength{\listparindent}{0cm}
\setlength{\leftmargin}{\evensidemargin}
\addtolength{\leftmargin}{\tmplength}
\settowidth{\labelsep}{X}
\addtolength{\leftmargin}{\labelsep}
\setlength{\labelwidth}{\tmplength}
}
\begin{flushleft}
\item[\textbf{Declaração}\hfill]
\begin{ttfamily}
public class function OpenText(VAR F:Text;Mode: Word):SmallInt ;\end{ttfamily}


\end{flushleft}
\end{list}
\paragraph*{Reset}\hspace*{\fill}

\begin{list}{}{
\settowidth{\tmplength}{\textbf{Declaração}}
\setlength{\itemindent}{0cm}
\setlength{\listparindent}{0cm}
\setlength{\leftmargin}{\evensidemargin}
\addtolength{\leftmargin}{\tmplength}
\settowidth{\labelsep}{X}
\addtolength{\leftmargin}{\labelsep}
\setlength{\labelwidth}{\tmplength}
}
\begin{flushleft}
\item[\textbf{Declaração}\hfill]
\begin{ttfamily}
public class function Reset(VAR F:Text ;Const AFileMode:SmallWord):SmallInt ; Overload;\end{ttfamily}


\end{flushleft}
\end{list}
\paragraph*{FileFlushBuffers}\hspace*{\fill}

\begin{list}{}{
\settowidth{\tmplength}{\textbf{Declaração}}
\setlength{\itemindent}{0cm}
\setlength{\listparindent}{0cm}
\setlength{\leftmargin}{\evensidemargin}
\addtolength{\leftmargin}{\tmplength}
\settowidth{\labelsep}{X}
\addtolength{\leftmargin}{\labelsep}
\setlength{\labelwidth}{\tmplength}
}
\begin{flushleft}
\item[\textbf{Declaração}\hfill]
\begin{ttfamily}
public class function FileFlushBuffers(VAR F : File):Boolean; overload;\end{ttfamily}


\end{flushleft}
\end{list}
\paragraph*{FileFlushBuffers}\hspace*{\fill}

\begin{list}{}{
\settowidth{\tmplength}{\textbf{Declaração}}
\setlength{\itemindent}{0cm}
\setlength{\listparindent}{0cm}
\setlength{\leftmargin}{\evensidemargin}
\addtolength{\leftmargin}{\tmplength}
\settowidth{\labelsep}{X}
\addtolength{\leftmargin}{\labelsep}
\setlength{\labelwidth}{\tmplength}
}
\begin{flushleft}
\item[\textbf{Declaração}\hfill]
\begin{ttfamily}
public class function FileFlushBuffers(VAR F : File; const Ok{\_}FileFlushBuffers : Boolean):Boolean; Overload;\end{ttfamily}


\end{flushleft}
\end{list}
\paragraph*{Size{\_}LinFeed{\_}Text}\hspace*{\fill}

\begin{list}{}{
\settowidth{\tmplength}{\textbf{Declaração}}
\setlength{\itemindent}{0cm}
\setlength{\listparindent}{0cm}
\setlength{\leftmargin}{\evensidemargin}
\addtolength{\leftmargin}{\tmplength}
\settowidth{\labelsep}{X}
\addtolength{\leftmargin}{\labelsep}
\setlength{\labelwidth}{\tmplength}
}
\begin{flushleft}
\item[\textbf{Declaração}\hfill]
\begin{ttfamily}
public class function Size{\_}LinFeed{\_}Text(aFileName : AnsiString):SmallInt ;\end{ttfamily}


\end{flushleft}
\end{list}
\paragraph*{FTempoDeTentativas}\hspace*{\fill}

\begin{list}{}{
\settowidth{\tmplength}{\textbf{Declaração}}
\setlength{\itemindent}{0cm}
\setlength{\listparindent}{0cm}
\setlength{\leftmargin}{\evensidemargin}
\addtolength{\leftmargin}{\tmplength}
\settowidth{\labelsep}{X}
\addtolength{\leftmargin}{\labelsep}
\setlength{\labelwidth}{\tmplength}
}
\begin{flushleft}
\item[\textbf{Declaração}\hfill]
\begin{ttfamily}
public class procedure FTempoDeTentativas(Const HcHelp:SmallInt);\end{ttfamily}


\end{flushleft}
\end{list}
\paragraph*{Is{\_}TFileOpen}\hspace*{\fill}

\begin{list}{}{
\settowidth{\tmplength}{\textbf{Declaração}}
\setlength{\itemindent}{0cm}
\setlength{\listparindent}{0cm}
\setlength{\leftmargin}{\evensidemargin}
\addtolength{\leftmargin}{\tmplength}
\settowidth{\labelsep}{X}
\addtolength{\leftmargin}{\labelsep}
\setlength{\labelwidth}{\tmplength}
}
\begin{flushleft}
\item[\textbf{Declaração}\hfill]
\begin{ttfamily}
public class function Is{\_}TFileOpen(const a{\_}TFile : TStream):Boolean;\end{ttfamily}


\end{flushleft}
\end{list}
\paragraph*{CopyFiles}\hspace*{\fill}

\begin{list}{}{
\settowidth{\tmplength}{\textbf{Declaração}}
\setlength{\itemindent}{0cm}
\setlength{\listparindent}{0cm}
\setlength{\leftmargin}{\evensidemargin}
\addtolength{\leftmargin}{\tmplength}
\settowidth{\labelsep}{X}
\addtolength{\leftmargin}{\labelsep}
\setlength{\labelwidth}{\tmplength}
}
\begin{flushleft}
\item[\textbf{Declaração}\hfill]
\begin{ttfamily}
public class function CopyFiles(const SourceName, TargetName: AnsiString):Integer;\end{ttfamily}


\end{flushleft}
\end{list}
\paragraph*{DeleteFiles}\hspace*{\fill}

\begin{list}{}{
\settowidth{\tmplength}{\textbf{Declaração}}
\setlength{\itemindent}{0cm}
\setlength{\listparindent}{0cm}
\setlength{\leftmargin}{\evensidemargin}
\addtolength{\leftmargin}{\tmplength}
\settowidth{\labelsep}{X}
\addtolength{\leftmargin}{\labelsep}
\setlength{\labelwidth}{\tmplength}
}
\begin{flushleft}
\item[\textbf{Declaração}\hfill]
\begin{ttfamily}
public class function DeleteFiles(const SourceName:AnsiString):Integer;\end{ttfamily}


\end{flushleft}
\end{list}
\paragraph*{Existe{\_}Espaco{\_}em{\_}Dobro}\hspace*{\fill}

\begin{list}{}{
\settowidth{\tmplength}{\textbf{Declaração}}
\setlength{\itemindent}{0cm}
\setlength{\listparindent}{0cm}
\setlength{\leftmargin}{\evensidemargin}
\addtolength{\leftmargin}{\tmplength}
\settowidth{\labelsep}{X}
\addtolength{\leftmargin}{\labelsep}
\setlength{\labelwidth}{\tmplength}
}
\begin{flushleft}
\item[\textbf{Declaração}\hfill]
\begin{ttfamily}
public class function Existe{\_}Espaco{\_}em{\_}Dobro: Boolean;\end{ttfamily}


\end{flushleft}
\end{list}
\chapter{Unit mi.rtl.objects.types}
\section{Descrição}
\begin{itemize}
\item A Unit \textbf{\begin{ttfamily}mi.rtl.objects.types\end{ttfamily}} implementa a classe \textbf{TObjectsTypes} .

\begin{itemize}
\item \textbf{NOTAS} \begin{itemize}
\item Esta unit foi testada nas plataformas: win32, win64 e linux.
\end{itemize}
\item \textbf{VERSÃO} \begin{itemize}
\item Alpha {-} 0.5.0.687
\end{itemize}
\item \textbf{CÓDIGO FONTE}: \begin{itemize}
\item 
\end{itemize}
\item \textbf{HISTÓRICO} \begin{itemize}
\item Criado por: Paulo Sérgio da Silva Pacheco e{-}mail: paulosspacheco@yahoo.com.br \begin{itemize}
\item \textbf{17/11/2021} 20:30 a 22:49 {-} Criada a classe \textbf{TObjectsTypes}. Falta conclui...
\item \textbf{18/11/2021} 09:05 {-} Concluir a classe \textbf{TObjectsTypes}.
\item \textbf{15/12/2021} 15:00 a 15:15 {-} Revisar a documentação da unidade.
\end{itemize}
\end{itemize}
\end{itemize}
\end{itemize}
\section{Uses}
\begin{itemize}
\item \begin{ttfamily}Classes\end{ttfamily}\item \begin{ttfamily}SysUtils\end{ttfamily}\item \begin{ttfamily}mi.rtl.files\end{ttfamily}(\ref{mi.rtl.files})\end{itemize}
\section{Visão Geral}
\begin{description}
\item[\texttt{\begin{ttfamily}DummyClass\end{ttfamily} Classe}]
\end{description}
\section{Classes, Interfaces, Objetos e Registros}
\subsection*{DummyClass Classe}
\subsubsection*{\large{\textbf{Hierarquia}}\normalsize\hspace{1ex}\hfill}
DummyClass {$>$} TObject
\subsubsection*{\large{\textbf{Descrição}}\normalsize\hspace{1ex}\hfill}
\begin{itemize}
\item Internal Class
\end{itemize}\subsubsection*{\large{\textbf{Campos}}\normalsize\hspace{1ex}\hfill}
\paragraph*{Data}\hspace*{\fill}

\begin{list}{}{
\settowidth{\tmplength}{\textbf{Declaração}}
\setlength{\itemindent}{0cm}
\setlength{\listparindent}{0cm}
\setlength{\leftmargin}{\evensidemargin}
\addtolength{\leftmargin}{\tmplength}
\settowidth{\labelsep}{X}
\addtolength{\leftmargin}{\labelsep}
\setlength{\labelwidth}{\tmplength}
}
\begin{flushleft}
\item[\textbf{Declaração}\hfill]
\begin{ttfamily}
public Data: Record\end{ttfamily}


\end{flushleft}
\end{list}
\chapter{Unit mi.rtl.Objectss}
\section{Descrição}
\begin{itemize}
\item A Unit \textbf{\begin{ttfamily}mi.rtl.Objectss\end{ttfamily}} reune todas as classes base pacote \textbf{\begin{ttfamily}mi.rtl\end{ttfamily}(\ref{mi.rtl})}.

\begin{itemize}
\item \textbf{NOTAS} \begin{itemize}
\item Esta unit foi testada nas plataformas: win32, win64 e linux.
\end{itemize}
\item \textbf{VERSÃO} \begin{itemize}
\item Alpha {-} 0.5.0.687
\end{itemize}
\item \textbf{HISTÓRICO} \begin{itemize}
\item Criado por: Paulo Sérgio da Silva Pacheco e{-}mail: paulosspacheco@yahoo.com.br \begin{itemize}
\item \textbf{20/11/2021} 9:10 a ??: Criar a unit mi.rtl.objects.pas {-}
\end{itemize}
\end{itemize}
\item \textbf{CÓDIGO FONTE}: \begin{itemize}
\item 
\end{itemize}
\end{itemize}
\end{itemize}
\section{Uses}
\begin{itemize}
\item \begin{ttfamily}Classes\end{ttfamily}\item \begin{ttfamily}SysUtils\end{ttfamily}\item \begin{ttfamily}mi.rtl.Objects.Methods.Paramexecucao.Application\end{ttfamily}(\ref{mi.rtl.Objects.Methods.Paramexecucao.Application})\item \begin{ttfamily}mi.rtl.types\end{ttfamily}(\ref{mi.rtl.Types})\item \begin{ttfamily}mi.rtl.Consts\end{ttfamily}(\ref{mi.rtl.Consts})\item \begin{ttfamily}mi.rtl.Consts.StringListBase\end{ttfamily}(\ref{mi.rtl.Consts.StringListBase})\item \begin{ttfamily}mi.rtl.Consts.StringList\end{ttfamily}(\ref{mi.rtl.Consts.StringList})\item \begin{ttfamily}mi.rtl.files\end{ttfamily}(\ref{mi.rtl.files})\item \begin{ttfamily}mi.rtl.objects.types\end{ttfamily}(\ref{mi.rtl.objects.types})\item \begin{ttfamily}mi.rtl.objects.consts\end{ttfamily}(\ref{mi.rtl.Objects.Consts})\item \begin{ttfamily}mi.rtl.objects.consts.MI{\_}MsgBox\end{ttfamily}\item \begin{ttfamily}mi.rtl.objects.consts.progressdlg{\_}if\end{ttfamily}(\ref{mi.rtl.Objects.Consts.ProgressDlg_If})\item \begin{ttfamily}mi.rtl.objects.Methods\end{ttfamily}(\ref{mi.rtl.Objects.Methods})\item \begin{ttfamily}mi.rtl.objects.Methods.Dates\end{ttfamily}(\ref{mi.rtl.objects.Methods.dates})\item \begin{ttfamily}mi.rtl.objects.methods.ParamExecucao\end{ttfamily}(\ref{mi.rtl.Objects.Methods.Paramexecucao})\item \begin{ttfamily}mi.rtl.objects.Methods.Exception\end{ttfamily}(\ref{mi.rtl.Objects.Methods.Exception})\item \begin{ttfamily}mi.rtl.objects.methods.StreamBase\end{ttfamily}(\ref{mi.rtl.Objects.Methods.StreamBase})\item \begin{ttfamily}mi.rtl.objects.methods.StreamBase.Stream\end{ttfamily}(\ref{mi.rtl.Objects.Methods.StreamBase.Stream})\item \begin{ttfamily}mi.rtl.objects.methods.StreamBase.Stream.MemoryStream\end{ttfamily}(\ref{mi.rtl.Objects.Methods.StreamBase.Stream.MemoryStream})\item \begin{ttfamily}mi.rtl.objects.methods.StreamBase.Stream.MemoryStream.BufferMemory\end{ttfamily}(\ref{mi.rtl.objects.methods.StreamBase.Stream.MemoryStream.BufferMemory})\item \begin{ttfamily}mi.rtl.objects.methods.StreamBase.Stream.FileStream\end{ttfamily}(\ref{mi.rtl.Objects.Methods.StreamBase.Stream.FileStream})\item \begin{ttfamily}mi.rtl.objects.methods.Collection\end{ttfamily}(\ref{mi.rtl.Objects.Methods.Collection})\item \begin{ttfamily}mi.rtl.objects.methods.Collection.SortedCollection\end{ttfamily}(\ref{mi.rtl.Objects.Methods.Collection.SortedCollection})\item \begin{ttfamily}mi.rtl.objects.methods.Collection.SortedCollection.StrCollection\end{ttfamily}(\ref{mi.rtl.Objects.Methods.Collection.SortedCollection.StrCollection})\item \begin{ttfamily}mi.rtl.objects.methods.Collection.SortedCollection.stringCollection\end{ttfamily}(\ref{mi.rtl.Objects.Methods.Collection.SortedCollection.StringCollection})\item \begin{ttfamily}mi.rtl.objects.methods.Collection.SortedCollection.stringcollection.CollectionString\end{ttfamily}(\ref{mi.rtl.Objects.Methods.Collection.Sortedcollection.Stringcollection.Collectionstring})\item \begin{ttfamily}mi.rtl.objects.methods.Collection.FilesStreams\end{ttfamily}(\ref{mi.rtl.Objects.Methods.Collection.FilesStreams})\item \begin{ttfamily}mi.rtl.objects.methods.db.tb{\_}access\end{ttfamily}(\ref{mi.rtl.Objects.Methods.Db.Tb_Access})\item \begin{ttfamily}mi.rtl.objects.methods.db.tb{\_}{\_}access\end{ttfamily}(\ref{mi.rtl.Objects.Methods.Db.Tb__Access})\item \begin{ttfamily}mi.rtl.objects.methods.db.tb{\_}{\_}{\_}access\end{ttfamily}(\ref{mi.rtl.Objects.Methods.Db.Tb___Access})\end{itemize}
\section{Visão Geral}
\begin{description}
\item[\texttt{\begin{ttfamily}TObjectss\end{ttfamily} Classe}]
\end{description}
\section{Classes, Interfaces, Objetos e Registros}
\subsection*{TObjectss Classe}
\subsubsection*{\large{\textbf{Hierarquia}}\normalsize\hspace{1ex}\hfill}
TObjectss {$>$} %%%%Descrição
\subsubsection*{\large{\textbf{Campos}}\normalsize\hspace{1ex}\hfill}
\paragraph*{Application}\hspace*{\fill}

\begin{list}{}{
\settowidth{\tmplength}{\textbf{Declaração}}
\setlength{\itemindent}{0cm}
\setlength{\listparindent}{0cm}
\setlength{\leftmargin}{\evensidemargin}
\addtolength{\leftmargin}{\tmplength}
\settowidth{\labelsep}{X}
\addtolength{\leftmargin}{\labelsep}
\setlength{\labelwidth}{\tmplength}
}
\begin{flushleft}
\item[\textbf{Declaração}\hfill]
\begin{ttfamily}
public const Application : TApplication =  nil;\end{ttfamily}


\end{flushleft}
\end{list}
\subsubsection*{\large{\textbf{Métodos}}\normalsize\hspace{1ex}\hfill}
\paragraph*{Set{\_}MI{\_}MsgBox}\hspace*{\fill}

\begin{list}{}{
\settowidth{\tmplength}{\textbf{Declaração}}
\setlength{\itemindent}{0cm}
\setlength{\listparindent}{0cm}
\setlength{\leftmargin}{\evensidemargin}
\addtolength{\leftmargin}{\tmplength}
\settowidth{\labelsep}{X}
\addtolength{\leftmargin}{\labelsep}
\setlength{\labelwidth}{\tmplength}
}
\begin{flushleft}
\item[\textbf{Declaração}\hfill]
\begin{ttfamily}
public Class Procedure Set{\_}MI{\_}MsgBox(aMI{\_}MsgBox: TMI{\_}MsgBox); Virtual;\end{ttfamily}


\end{flushleft}
\end{list}
\paragraph*{ProcStreamError}\hspace*{\fill}

\begin{list}{}{
\settowidth{\tmplength}{\textbf{Declaração}}
\setlength{\itemindent}{0cm}
\setlength{\listparindent}{0cm}
\setlength{\leftmargin}{\evensidemargin}
\addtolength{\leftmargin}{\tmplength}
\settowidth{\labelsep}{X}
\addtolength{\leftmargin}{\labelsep}
\setlength{\labelwidth}{\tmplength}
}
\begin{flushleft}
\item[\textbf{Declaração}\hfill]
\begin{ttfamily}
public class Procedure ProcStreamError(Const S: TStreambase);\end{ttfamily}


\end{flushleft}
\end{list}
\paragraph*{StrToSItem}\hspace*{\fill}

\begin{list}{}{
\settowidth{\tmplength}{\textbf{Declaração}}
\setlength{\itemindent}{0cm}
\setlength{\listparindent}{0cm}
\setlength{\leftmargin}{\evensidemargin}
\addtolength{\leftmargin}{\tmplength}
\settowidth{\labelsep}{X}
\addtolength{\leftmargin}{\labelsep}
\setlength{\labelwidth}{\tmplength}
}
\begin{flushleft}
\item[\textbf{Declaração}\hfill]
\begin{ttfamily}
public class Function StrToSItem(Const StrMsg:AnsiString; Colunas : byte;Alinhamento:TAlinhamento):PSItem;\end{ttfamily}


\end{flushleft}
\end{list}
\paragraph*{WriteSItems}\hspace*{\fill}

\begin{list}{}{
\settowidth{\tmplength}{\textbf{Declaração}}
\setlength{\itemindent}{0cm}
\setlength{\listparindent}{0cm}
\setlength{\leftmargin}{\evensidemargin}
\addtolength{\leftmargin}{\tmplength}
\settowidth{\labelsep}{X}
\addtolength{\leftmargin}{\labelsep}
\setlength{\labelwidth}{\tmplength}
}
\begin{flushleft}
\item[\textbf{Declaração}\hfill]
\begin{ttfamily}
public class procedure WriteSItems(var S: TCollectionString; Const Items: PSItem);\end{ttfamily}


\end{flushleft}
\end{list}
\paragraph*{PSItem{\_}ListaDeMsgErro}\hspace*{\fill}

\begin{list}{}{
\settowidth{\tmplength}{\textbf{Declaração}}
\setlength{\itemindent}{0cm}
\setlength{\listparindent}{0cm}
\setlength{\leftmargin}{\evensidemargin}
\addtolength{\leftmargin}{\tmplength}
\settowidth{\labelsep}{X}
\addtolength{\leftmargin}{\labelsep}
\setlength{\labelwidth}{\tmplength}
}
\begin{flushleft}
\item[\textbf{Declaração}\hfill]
\begin{ttfamily}
public class Function PSItem{\_}ListaDeMsgErro:PSItem; override;\end{ttfamily}


\end{flushleft}
\end{list}
\paragraph*{MessageError}\hspace*{\fill}

\begin{list}{}{
\settowidth{\tmplength}{\textbf{Declaração}}
\setlength{\itemindent}{0cm}
\setlength{\listparindent}{0cm}
\setlength{\leftmargin}{\evensidemargin}
\addtolength{\leftmargin}{\tmplength}
\settowidth{\labelsep}{X}
\addtolength{\leftmargin}{\labelsep}
\setlength{\labelwidth}{\tmplength}
}
\begin{flushleft}
\item[\textbf{Declaração}\hfill]
\begin{ttfamily}
public class Procedure MessageError; override;\end{ttfamily}


\end{flushleft}
\end{list}
\chapter{Unit mi.rtl.Types}
\section{Descrição}
\begin{itemize}
\item A Unit \textbf{\begin{ttfamily}mi.rtl.Types\end{ttfamily}} reune os tipos globais usados pelo pacote \textbf{\begin{ttfamily}mi.rtl\end{ttfamily}(\ref{mi.rtl})}. Esta unit foi testada nas plataformas: no linux.

\begin{itemize}
\item \textbf{NOTA} \begin{itemize}
\item O Método \textbf{TTypes.TPointer.Get{\_}Mem} ignora alocação de memória real porque não sei como fazer nas plataformas diferentes do Windows.
\end{itemize}
\item \textbf{VERSÃO} \begin{itemize}
\item Alpha {-} 0.5.0.687
\end{itemize}
\item \textbf{CÓDIGO FONTE}: \begin{itemize}
\item 
\end{itemize}
\item \textbf{HISTÓRICO} \begin{itemize}
\item Criado por: Paulo Sérgio da Silva Pacheco e{-}mail: paulosspacheco@yahoo.com.br \begin{itemize}
\item \textbf{Period} : June to September of 2001)
\item \textbf{14/09/2001} : I begin of the version: Windows 98
\item \textbf{29/10/2021} : Portado para o compilador free pascal para os sistemas operacionais: 1. x86{\_}64{-}linux 2. x86{\_}64{-}win64 3. i386{-}win32
\item \textbf{02/11/2021} : Trabalhei na documentação com pasdoc.
\item \textbf{12/11/2021} \begin{itemize}
\item A Unit \textbf{mi.rtl.types} foi convertida para unit \textbf{mi.types}.
\item Criado a class \textbf{\begin{ttfamily}TTypes\end{ttfamily}(\ref{mi.rtl.Types.TTypes})} com todos os tipos definidos em \textbf{mi.rtl.types} com objetivo de encapsular os tipos globais do pacote \begin{ttfamily}mi.rtl\end{ttfamily}(\ref{mi.rtl}).
\end{itemize}
\item \textbf{13/11/2021} \begin{itemize}
\item Documentação da unit \begin{ttfamily}mi.rtl.Types\end{ttfamily}.
\end{itemize}
\item \textbf{15/12/2021} \begin{itemize}
\item Criado o tipo registro TIndentificação.
\end{itemize}
\end{itemize}
\end{itemize}
\end{itemize}
\end{itemize}
\section{Uses}
\begin{itemize}
\item \begin{ttfamily}Classes\end{ttfamily}\item \begin{ttfamily}Dos\end{ttfamily}\item \begin{ttfamily}SysUtils\end{ttfamily}\end{itemize}
\section{Visão Geral}
\begin{description}
\item[\texttt{\begin{ttfamily}TTypes\end{ttfamily} Classe}]
\end{description}
\section{Classes, Interfaces, Objetos e Registros}
\subsection*{TTypes Classe}
\subsubsection*{\large{\textbf{Hierarquia}}\normalsize\hspace{1ex}\hfill}
TTypes {$>$} TComponent
\subsubsection*{\large{\textbf{Descrição}}\normalsize\hspace{1ex}\hfill}
A classe \textbf{\begin{ttfamily}TTypes\end{ttfamily}} declara todos os tipos globais do pacote MarIcarai\subsubsection*{\large{\textbf{Campos}}\normalsize\hspace{1ex}\hfill}
\paragraph*{Alias}\hspace*{\fill}

\begin{list}{}{
\settowidth{\tmplength}{\textbf{Declaração}}
\setlength{\itemindent}{0cm}
\setlength{\listparindent}{0cm}
\setlength{\leftmargin}{\evensidemargin}
\addtolength{\leftmargin}{\tmplength}
\settowidth{\labelsep}{X}
\addtolength{\leftmargin}{\labelsep}
\setlength{\labelwidth}{\tmplength}
}
\begin{flushleft}
\item[\textbf{Declaração}\hfill]
\begin{ttfamily}
public Alias: AnsiString;\end{ttfamily}


\end{flushleft}
\end{list}
\paragraph*{ok{\_}Set{\_}Transaction}\hspace*{\fill}

\begin{list}{}{
\settowidth{\tmplength}{\textbf{Declaração}}
\setlength{\itemindent}{0cm}
\setlength{\listparindent}{0cm}
\setlength{\leftmargin}{\evensidemargin}
\addtolength{\leftmargin}{\tmplength}
\settowidth{\labelsep}{X}
\addtolength{\leftmargin}{\labelsep}
\setlength{\labelwidth}{\tmplength}
}
\begin{flushleft}
\item[\textbf{Declaração}\hfill]
\begin{ttfamily}
public const ok{\_}Set{\_}Transaction   : BOOLEAN = false;\end{ttfamily}


\end{flushleft}
\par
\item[\textbf{Descrição}]
\begin{itemize}
\item A constant \textbf{\begin{ttfamily}ok{\_}Set{\_}Transaction\end{ttfamily}} indica se o processo está dentro de uma transação.
\end{itemize}

\end{list}
\paragraph*{MAX{\_}BYTE}\hspace*{\fill}

\begin{list}{}{
\settowidth{\tmplength}{\textbf{Declaração}}
\setlength{\itemindent}{0cm}
\setlength{\listparindent}{0cm}
\setlength{\leftmargin}{\evensidemargin}
\addtolength{\leftmargin}{\tmplength}
\settowidth{\labelsep}{X}
\addtolength{\leftmargin}{\labelsep}
\setlength{\labelwidth}{\tmplength}
}
\begin{flushleft}
\item[\textbf{Declaração}\hfill]
\begin{ttfamily}
public const MAX{\_}BYTE         = high(SmallWord);\end{ttfamily}


\end{flushleft}
\end{list}
\paragraph*{MAX{\_}ARRAY{\_}BYTE}\hspace*{\fill}

\begin{list}{}{
\settowidth{\tmplength}{\textbf{Declaração}}
\setlength{\itemindent}{0cm}
\setlength{\listparindent}{0cm}
\setlength{\leftmargin}{\evensidemargin}
\addtolength{\leftmargin}{\tmplength}
\settowidth{\labelsep}{X}
\addtolength{\leftmargin}{\labelsep}
\setlength{\labelwidth}{\tmplength}
}
\begin{flushleft}
\item[\textbf{Declaração}\hfill]
\begin{ttfamily}
public const MAX{\_}ARRAY{\_}BYTE   = MAX{\_}BYTE div sizeof(byte);\end{ttfamily}


\end{flushleft}
\end{list}
\paragraph*{MAX{\_}INT}\hspace*{\fill}

\begin{list}{}{
\settowidth{\tmplength}{\textbf{Declaração}}
\setlength{\itemindent}{0cm}
\setlength{\listparindent}{0cm}
\setlength{\leftmargin}{\evensidemargin}
\addtolength{\leftmargin}{\tmplength}
\settowidth{\labelsep}{X}
\addtolength{\leftmargin}{\labelsep}
\setlength{\labelwidth}{\tmplength}
}
\begin{flushleft}
\item[\textbf{Declaração}\hfill]
\begin{ttfamily}
public const MAX{\_}INT          = high(Integer);\end{ttfamily}


\end{flushleft}
\end{list}
\paragraph*{MAX{\_}ARRAY{\_}INT}\hspace*{\fill}

\begin{list}{}{
\settowidth{\tmplength}{\textbf{Declaração}}
\setlength{\itemindent}{0cm}
\setlength{\listparindent}{0cm}
\setlength{\leftmargin}{\evensidemargin}
\addtolength{\leftmargin}{\tmplength}
\settowidth{\labelsep}{X}
\addtolength{\leftmargin}{\labelsep}
\setlength{\labelwidth}{\tmplength}
}
\begin{flushleft}
\item[\textbf{Declaração}\hfill]
\begin{ttfamily}
public const MAX{\_}ARRAY{\_}INT    = MAX{\_}INT div sizeof(integer);\end{ttfamily}


\end{flushleft}
\end{list}
\paragraph*{MAX{\_}SMALL{\_}INT}\hspace*{\fill}

\begin{list}{}{
\settowidth{\tmplength}{\textbf{Declaração}}
\setlength{\itemindent}{0cm}
\setlength{\listparindent}{0cm}
\setlength{\leftmargin}{\evensidemargin}
\addtolength{\leftmargin}{\tmplength}
\settowidth{\labelsep}{X}
\addtolength{\leftmargin}{\labelsep}
\setlength{\labelwidth}{\tmplength}
}
\begin{flushleft}
\item[\textbf{Declaração}\hfill]
\begin{ttfamily}
public const MAX{\_}SMALL{\_}INT    = high(SmallInt);\end{ttfamily}


\end{flushleft}
\end{list}
\paragraph*{MAX{\_}ARRAY{\_}SMALL{\_}INT}\hspace*{\fill}

\begin{list}{}{
\settowidth{\tmplength}{\textbf{Declaração}}
\setlength{\itemindent}{0cm}
\setlength{\listparindent}{0cm}
\setlength{\leftmargin}{\evensidemargin}
\addtolength{\leftmargin}{\tmplength}
\settowidth{\labelsep}{X}
\addtolength{\leftmargin}{\labelsep}
\setlength{\labelwidth}{\tmplength}
}
\begin{flushleft}
\item[\textbf{Declaração}\hfill]
\begin{ttfamily}
public const MAX{\_}ARRAY{\_}SMALL{\_}INT  = MAX{\_}SMALL{\_}INT div sizeof(SmallInt);\end{ttfamily}


\end{flushleft}
\end{list}
\paragraph*{MAX{\_}LONG{\_}INT}\hspace*{\fill}

\begin{list}{}{
\settowidth{\tmplength}{\textbf{Declaração}}
\setlength{\itemindent}{0cm}
\setlength{\listparindent}{0cm}
\setlength{\leftmargin}{\evensidemargin}
\addtolength{\leftmargin}{\tmplength}
\settowidth{\labelsep}{X}
\addtolength{\leftmargin}{\labelsep}
\setlength{\labelwidth}{\tmplength}
}
\begin{flushleft}
\item[\textbf{Declaração}\hfill]
\begin{ttfamily}
public const MAX{\_}LONG{\_}INT     = high(LongInt);\end{ttfamily}


\end{flushleft}
\end{list}
\paragraph*{MAX{\_}ARRAY{\_}LONG{\_}INT}\hspace*{\fill}

\begin{list}{}{
\settowidth{\tmplength}{\textbf{Declaração}}
\setlength{\itemindent}{0cm}
\setlength{\listparindent}{0cm}
\setlength{\leftmargin}{\evensidemargin}
\addtolength{\leftmargin}{\tmplength}
\settowidth{\labelsep}{X}
\addtolength{\leftmargin}{\labelsep}
\setlength{\labelwidth}{\tmplength}
}
\begin{flushleft}
\item[\textbf{Declaração}\hfill]
\begin{ttfamily}
public const MAX{\_}ARRAY{\_}LONG{\_}INT  = MAX{\_}LONG{\_}INT div sizeof(Longint);\end{ttfamily}


\end{flushleft}
\end{list}
\paragraph*{MAX{\_}WORD}\hspace*{\fill}

\begin{list}{}{
\settowidth{\tmplength}{\textbf{Declaração}}
\setlength{\itemindent}{0cm}
\setlength{\listparindent}{0cm}
\setlength{\leftmargin}{\evensidemargin}
\addtolength{\leftmargin}{\tmplength}
\settowidth{\labelsep}{X}
\addtolength{\leftmargin}{\labelsep}
\setlength{\labelwidth}{\tmplength}
}
\begin{flushleft}
\item[\textbf{Declaração}\hfill]
\begin{ttfamily}
public const MAX{\_}WORD         = high(Word);\end{ttfamily}


\end{flushleft}
\end{list}
\paragraph*{MAX{\_}ARRAY{\_}WORD}\hspace*{\fill}

\begin{list}{}{
\settowidth{\tmplength}{\textbf{Declaração}}
\setlength{\itemindent}{0cm}
\setlength{\listparindent}{0cm}
\setlength{\leftmargin}{\evensidemargin}
\addtolength{\leftmargin}{\tmplength}
\settowidth{\labelsep}{X}
\addtolength{\leftmargin}{\labelsep}
\setlength{\labelwidth}{\tmplength}
}
\begin{flushleft}
\item[\textbf{Declaração}\hfill]
\begin{ttfamily}
public const MAX{\_}ARRAY{\_}WORD  = MAX{\_}WORD div sizeof(word);\end{ttfamily}


\end{flushleft}
\end{list}
\paragraph*{MAX{\_}SMALL{\_}WORD}\hspace*{\fill}

\begin{list}{}{
\settowidth{\tmplength}{\textbf{Declaração}}
\setlength{\itemindent}{0cm}
\setlength{\listparindent}{0cm}
\setlength{\leftmargin}{\evensidemargin}
\addtolength{\leftmargin}{\tmplength}
\settowidth{\labelsep}{X}
\addtolength{\leftmargin}{\labelsep}
\setlength{\labelwidth}{\tmplength}
}
\begin{flushleft}
\item[\textbf{Declaração}\hfill]
\begin{ttfamily}
public const MAX{\_}SMALL{\_}WORD   = high(system.word);\end{ttfamily}


\end{flushleft}
\end{list}
\paragraph*{MAX{\_}ARRAY{\_}SMALL{\_}WORD}\hspace*{\fill}

\begin{list}{}{
\settowidth{\tmplength}{\textbf{Declaração}}
\setlength{\itemindent}{0cm}
\setlength{\listparindent}{0cm}
\setlength{\leftmargin}{\evensidemargin}
\addtolength{\leftmargin}{\tmplength}
\settowidth{\labelsep}{X}
\addtolength{\leftmargin}{\labelsep}
\setlength{\labelwidth}{\tmplength}
}
\begin{flushleft}
\item[\textbf{Declaração}\hfill]
\begin{ttfamily}
public const MAX{\_}ARRAY{\_}SMALL{\_}WORD  = MAX{\_}SMALL{\_}WORD div sizeof(system.word);\end{ttfamily}


\end{flushleft}
\end{list}
\paragraph*{MAX{\_}LONG{\_}WORD}\hspace*{\fill}

\begin{list}{}{
\settowidth{\tmplength}{\textbf{Declaração}}
\setlength{\itemindent}{0cm}
\setlength{\listparindent}{0cm}
\setlength{\leftmargin}{\evensidemargin}
\addtolength{\leftmargin}{\tmplength}
\settowidth{\labelsep}{X}
\addtolength{\leftmargin}{\labelsep}
\setlength{\labelwidth}{\tmplength}
}
\begin{flushleft}
\item[\textbf{Declaração}\hfill]
\begin{ttfamily}
public const MAX{\_}LONG{\_}WORD    = high(LongWord);\end{ttfamily}


\end{flushleft}
\end{list}
\paragraph*{MAX{\_}ARRAY{\_}LONG{\_}WORD}\hspace*{\fill}

\begin{list}{}{
\settowidth{\tmplength}{\textbf{Declaração}}
\setlength{\itemindent}{0cm}
\setlength{\listparindent}{0cm}
\setlength{\leftmargin}{\evensidemargin}
\addtolength{\leftmargin}{\tmplength}
\settowidth{\labelsep}{X}
\addtolength{\leftmargin}{\labelsep}
\setlength{\labelwidth}{\tmplength}
}
\begin{flushleft}
\item[\textbf{Declaração}\hfill]
\begin{ttfamily}
public const MAX{\_}ARRAY{\_}LONG{\_}WORD  = MAX{\_}LONG{\_}WORD div sizeof(LongWord);\end{ttfamily}


\end{flushleft}
\end{list}
\paragraph*{MAX{\_}POINTER}\hspace*{\fill}

\begin{list}{}{
\settowidth{\tmplength}{\textbf{Declaração}}
\setlength{\itemindent}{0cm}
\setlength{\listparindent}{0cm}
\setlength{\leftmargin}{\evensidemargin}
\addtolength{\leftmargin}{\tmplength}
\settowidth{\labelsep}{X}
\addtolength{\leftmargin}{\labelsep}
\setlength{\labelwidth}{\tmplength}
}
\begin{flushleft}
\item[\textbf{Declaração}\hfill]
\begin{ttfamily}
public const MAX{\_}POINTER        = MAX{\_}ARRAY{\_}WORD;\end{ttfamily}


\end{flushleft}
\par
\item[\textbf{Descrição}]
O ideal seria memAvail, porém esta função não é multiplataforma;

\end{list}
\paragraph*{MAX{\_}ARRAY{\_}PTR}\hspace*{\fill}

\begin{list}{}{
\settowidth{\tmplength}{\textbf{Declaração}}
\setlength{\itemindent}{0cm}
\setlength{\listparindent}{0cm}
\setlength{\leftmargin}{\evensidemargin}
\addtolength{\leftmargin}{\tmplength}
\settowidth{\labelsep}{X}
\addtolength{\leftmargin}{\labelsep}
\setlength{\labelwidth}{\tmplength}
}
\begin{flushleft}
\item[\textbf{Declaração}\hfill]
\begin{ttfamily}
public const MAX{\_}ARRAY{\_}PTR      = MAX{\_}POINTER  div sizeof(Pointer);\end{ttfamily}


\end{flushleft}
\end{list}
\paragraph*{FileNameLen}\hspace*{\fill}

\begin{list}{}{
\settowidth{\tmplength}{\textbf{Declaração}}
\setlength{\itemindent}{0cm}
\setlength{\listparindent}{0cm}
\setlength{\leftmargin}{\evensidemargin}
\addtolength{\leftmargin}{\tmplength}
\settowidth{\labelsep}{X}
\addtolength{\leftmargin}{\labelsep}
\setlength{\labelwidth}{\tmplength}
}
\begin{flushleft}
\item[\textbf{Declaração}\hfill]
\begin{ttfamily}
public const FileNameLen : integer = Dos.FileNameLen;\end{ttfamily}


\end{flushleft}
\par
\item[\textbf{Descrição}]
Usado para compatibilidade com o passado;

\end{list}
\paragraph*{evNothing}\hspace*{\fill}

\begin{list}{}{
\settowidth{\tmplength}{\textbf{Declaração}}
\setlength{\itemindent}{0cm}
\setlength{\listparindent}{0cm}
\setlength{\leftmargin}{\evensidemargin}
\addtolength{\leftmargin}{\tmplength}
\settowidth{\labelsep}{X}
\addtolength{\leftmargin}{\labelsep}
\setlength{\labelwidth}{\tmplength}
}
\begin{flushleft}
\item[\textbf{Declaração}\hfill]
\begin{ttfamily}
public const evNothing   = {\$}0000;\end{ttfamily}


\end{flushleft}
\end{list}
\paragraph*{evMouseDown}\hspace*{\fill}

\begin{list}{}{
\settowidth{\tmplength}{\textbf{Declaração}}
\setlength{\itemindent}{0cm}
\setlength{\listparindent}{0cm}
\setlength{\leftmargin}{\evensidemargin}
\addtolength{\leftmargin}{\tmplength}
\settowidth{\labelsep}{X}
\addtolength{\leftmargin}{\labelsep}
\setlength{\labelwidth}{\tmplength}
}
\begin{flushleft}
\item[\textbf{Declaração}\hfill]
\begin{ttfamily}
public const evMouseDown = {\$}0001;\end{ttfamily}


\end{flushleft}
\end{list}
\paragraph*{evMouseUp}\hspace*{\fill}

\begin{list}{}{
\settowidth{\tmplength}{\textbf{Declaração}}
\setlength{\itemindent}{0cm}
\setlength{\listparindent}{0cm}
\setlength{\leftmargin}{\evensidemargin}
\addtolength{\leftmargin}{\tmplength}
\settowidth{\labelsep}{X}
\addtolength{\leftmargin}{\labelsep}
\setlength{\labelwidth}{\tmplength}
}
\begin{flushleft}
\item[\textbf{Declaração}\hfill]
\begin{ttfamily}
public const evMouseUp   = {\$}0002;\end{ttfamily}


\end{flushleft}
\end{list}
\paragraph*{evMouseMove}\hspace*{\fill}

\begin{list}{}{
\settowidth{\tmplength}{\textbf{Declaração}}
\setlength{\itemindent}{0cm}
\setlength{\listparindent}{0cm}
\setlength{\leftmargin}{\evensidemargin}
\addtolength{\leftmargin}{\tmplength}
\settowidth{\labelsep}{X}
\addtolength{\leftmargin}{\labelsep}
\setlength{\labelwidth}{\tmplength}
}
\begin{flushleft}
\item[\textbf{Declaração}\hfill]
\begin{ttfamily}
public const evMouseMove = {\$}0004;\end{ttfamily}


\end{flushleft}
\end{list}
\paragraph*{evMouseAuto}\hspace*{\fill}

\begin{list}{}{
\settowidth{\tmplength}{\textbf{Declaração}}
\setlength{\itemindent}{0cm}
\setlength{\listparindent}{0cm}
\setlength{\leftmargin}{\evensidemargin}
\addtolength{\leftmargin}{\tmplength}
\settowidth{\labelsep}{X}
\addtolength{\leftmargin}{\labelsep}
\setlength{\labelwidth}{\tmplength}
}
\begin{flushleft}
\item[\textbf{Declaração}\hfill]
\begin{ttfamily}
public const evMouseAuto = {\$}0008;\end{ttfamily}


\end{flushleft}
\end{list}
\paragraph*{evKeyDown}\hspace*{\fill}

\begin{list}{}{
\settowidth{\tmplength}{\textbf{Declaração}}
\setlength{\itemindent}{0cm}
\setlength{\listparindent}{0cm}
\setlength{\leftmargin}{\evensidemargin}
\addtolength{\leftmargin}{\tmplength}
\settowidth{\labelsep}{X}
\addtolength{\leftmargin}{\labelsep}
\setlength{\labelwidth}{\tmplength}
}
\begin{flushleft}
\item[\textbf{Declaração}\hfill]
\begin{ttfamily}
public const evKeyDown   = {\$}0010;\end{ttfamily}


\end{flushleft}
\end{list}
\paragraph*{evCommand}\hspace*{\fill}

\begin{list}{}{
\settowidth{\tmplength}{\textbf{Declaração}}
\setlength{\itemindent}{0cm}
\setlength{\listparindent}{0cm}
\setlength{\leftmargin}{\evensidemargin}
\addtolength{\leftmargin}{\tmplength}
\settowidth{\labelsep}{X}
\addtolength{\leftmargin}{\labelsep}
\setlength{\labelwidth}{\tmplength}
}
\begin{flushleft}
\item[\textbf{Declaração}\hfill]
\begin{ttfamily}
public const evCommand   = {\$}0100;\end{ttfamily}


\end{flushleft}
\end{list}
\paragraph*{evBroadcast}\hspace*{\fill}

\begin{list}{}{
\settowidth{\tmplength}{\textbf{Declaração}}
\setlength{\itemindent}{0cm}
\setlength{\listparindent}{0cm}
\setlength{\leftmargin}{\evensidemargin}
\addtolength{\leftmargin}{\tmplength}
\settowidth{\labelsep}{X}
\addtolength{\leftmargin}{\labelsep}
\setlength{\labelwidth}{\tmplength}
}
\begin{flushleft}
\item[\textbf{Declaração}\hfill]
\begin{ttfamily}
public const evBroadcast = {\$}0200;\end{ttfamily}


\end{flushleft}
\end{list}
\paragraph*{EvAplCliSvr}\hspace*{\fill}

\begin{list}{}{
\settowidth{\tmplength}{\textbf{Declaração}}
\setlength{\itemindent}{0cm}
\setlength{\listparindent}{0cm}
\setlength{\leftmargin}{\evensidemargin}
\addtolength{\leftmargin}{\tmplength}
\settowidth{\labelsep}{X}
\addtolength{\leftmargin}{\labelsep}
\setlength{\labelwidth}{\tmplength}
}
\begin{flushleft}
\item[\textbf{Declaração}\hfill]
\begin{ttfamily}
public const EvAplCliSvr = {\$}0400;\end{ttfamily}


\end{flushleft}
\end{list}
\paragraph*{evMouse}\hspace*{\fill}

\begin{list}{}{
\settowidth{\tmplength}{\textbf{Declaração}}
\setlength{\itemindent}{0cm}
\setlength{\listparindent}{0cm}
\setlength{\leftmargin}{\evensidemargin}
\addtolength{\leftmargin}{\tmplength}
\settowidth{\labelsep}{X}
\addtolength{\leftmargin}{\labelsep}
\setlength{\labelwidth}{\tmplength}
}
\begin{flushleft}
\item[\textbf{Declaração}\hfill]
\begin{ttfamily}
public const evMouse     = {\$}000F;\end{ttfamily}


\end{flushleft}
\end{list}
\paragraph*{evKeyboard}\hspace*{\fill}

\begin{list}{}{
\settowidth{\tmplength}{\textbf{Declaração}}
\setlength{\itemindent}{0cm}
\setlength{\listparindent}{0cm}
\setlength{\leftmargin}{\evensidemargin}
\addtolength{\leftmargin}{\tmplength}
\settowidth{\labelsep}{X}
\addtolength{\leftmargin}{\labelsep}
\setlength{\labelwidth}{\tmplength}
}
\begin{flushleft}
\item[\textbf{Declaração}\hfill]
\begin{ttfamily}
public const evKeyboard  = {\$}0010;\end{ttfamily}


\end{flushleft}
\end{list}
\paragraph*{evMessage}\hspace*{\fill}

\begin{list}{}{
\settowidth{\tmplength}{\textbf{Declaração}}
\setlength{\itemindent}{0cm}
\setlength{\listparindent}{0cm}
\setlength{\leftmargin}{\evensidemargin}
\addtolength{\leftmargin}{\tmplength}
\settowidth{\labelsep}{X}
\addtolength{\leftmargin}{\labelsep}
\setlength{\labelwidth}{\tmplength}
}
\begin{flushleft}
\item[\textbf{Declaração}\hfill]
\begin{ttfamily}
public const evMessage   = {\$}FF00;\end{ttfamily}


\end{flushleft}
\end{list}
\paragraph*{SizeOffldCluster}\hspace*{\fill}

\begin{list}{}{
\settowidth{\tmplength}{\textbf{Declaração}}
\setlength{\itemindent}{0cm}
\setlength{\listparindent}{0cm}
\setlength{\leftmargin}{\evensidemargin}
\addtolength{\leftmargin}{\tmplength}
\settowidth{\labelsep}{X}
\addtolength{\leftmargin}{\labelsep}
\setlength{\labelwidth}{\tmplength}
}
\begin{flushleft}
\item[\textbf{Declaração}\hfill]
\begin{ttfamily}
public const SizeOffldCluster    : TSizeOffldCluster = sizeof(TSizeOffldCluster);\end{ttfamily}


\end{flushleft}
\end{list}
\paragraph*{SizeOffldDbCluster}\hspace*{\fill}

\begin{list}{}{
\settowidth{\tmplength}{\textbf{Declaração}}
\setlength{\itemindent}{0cm}
\setlength{\listparindent}{0cm}
\setlength{\leftmargin}{\evensidemargin}
\addtolength{\leftmargin}{\tmplength}
\settowidth{\labelsep}{X}
\addtolength{\leftmargin}{\labelsep}
\setlength{\labelwidth}{\tmplength}
}
\begin{flushleft}
\item[\textbf{Declaração}\hfill]
\begin{ttfamily}
public const SizeOffldDbCluster  = 50;\end{ttfamily}


\end{flushleft}
\end{list}
\subsubsection*{\large{\textbf{Métodos}}\normalsize\hspace{1ex}\hfill}
\paragraph*{Create}\hspace*{\fill}

\begin{list}{}{
\settowidth{\tmplength}{\textbf{Declaração}}
\setlength{\itemindent}{0cm}
\setlength{\listparindent}{0cm}
\setlength{\leftmargin}{\evensidemargin}
\addtolength{\leftmargin}{\tmplength}
\settowidth{\labelsep}{X}
\addtolength{\leftmargin}{\labelsep}
\setlength{\labelwidth}{\tmplength}
}
\begin{flushleft}
\item[\textbf{Declaração}\hfill]
\begin{ttfamily}
public constructor Create(aowner:TComponent); Overload; Override;\end{ttfamily}


\end{flushleft}
\end{list}
\paragraph*{CheckEmpty}\hspace*{\fill}

\begin{list}{}{
\settowidth{\tmplength}{\textbf{Declaração}}
\setlength{\itemindent}{0cm}
\setlength{\listparindent}{0cm}
\setlength{\leftmargin}{\evensidemargin}
\addtolength{\leftmargin}{\tmplength}
\settowidth{\labelsep}{X}
\addtolength{\leftmargin}{\labelsep}
\setlength{\labelwidth}{\tmplength}
}
\begin{flushleft}
\item[\textbf{Declaração}\hfill]
\begin{ttfamily}
public class procedure CheckEmpty(Var Rect: TTypes.TRect);\end{ttfamily}


\end{flushleft}
\end{list}
\chapter{Unit mi.ui.dialogs}
\section{Descrição}
\begin{itemize}
\item A unit \textbf{\begin{ttfamily}mi.ui.dialogs\end{ttfamily}} implementa a classe \begin{ttfamily}TDialogs\end{ttfamily}(\ref{mi.ui.dialogs.TDialogs}) do pacote mi.ui.

\begin{itemize}
\item \textbf{VERSÃO}: \begin{itemize}
\item Alpha {-} 0.5.0.687
\end{itemize}
\item \textbf{CÓDIGO FONTE}: \begin{itemize}
\item 
\end{itemize}
\item \textbf{HISTÓRICO} \begin{itemize}
\item Criado por: Paulo Sérgio da Silva Pacheco e{-}mail: paulosspacheco@yahoo.com.br \begin{itemize}
\item 2021{-}12{-}02 \begin{itemize}
\item 23:00 a 23:35 {-} Criado a unit \textbf{\begin{ttfamily}mi.ui.dialogs\end{ttfamily}} e implementação da classe \textbf{\begin{ttfamily}TDialogs\end{ttfamily}(\ref{mi.ui.dialogs.TDialogs})}
\end{itemize}
\end{itemize}
\item 2021{-}12{-}03 \begin{itemize}
\item 09:40 a 12:00 \begin{itemize}
\item Criar método de classe Confirm();
\item Criar método de classe Prompt();
\item Criar método de classe Password();
\end{itemize}
\end{itemize}
\item \textbf{2021{-}12{-}04} \begin{itemize}
\item 15:11 a 16:40 \begin{itemize}
\item Criar exemplo TForm1.Test{\_}tobjects{\_}dlgs{\_}Confirm;
\item Criar exemplo TForm1.Test{\_}tobjects{\_}dlgs{\_}Prompt;
\item Criar exemplo TForm1.Test{\_}tobjects{\_}dlgs{\_}password;
\end{itemize}
\end{itemize}
\end{itemize}
\end{itemize}
\end{itemize}
\section{Uses}
\begin{itemize}
\item \begin{ttfamily}Classes\end{ttfamily}\item \begin{ttfamily}SysUtils\end{ttfamily}\item \begin{ttfamily}Forms\end{ttfamily}\item \begin{ttfamily}Dialogs\end{ttfamily}\item \begin{ttfamily}Graphics\end{ttfamily}\item \begin{ttfamily}StdCtrls\end{ttfamily}\item \begin{ttfamily}mi.rtl.objects.consts\end{ttfamily}(\ref{mi.rtl.Objects.Consts})\item \begin{ttfamily}mi.rtl.objects.Methods\end{ttfamily}(\ref{mi.rtl.Objects.Methods})\item \begin{ttfamily}mi.rtl.objects.consts.dialogs\end{ttfamily}\end{itemize}
\section{Visão Geral}
\begin{description}
\item[\texttt{\begin{ttfamily}TDialogs\end{ttfamily} Classe}]
\end{description}
\section{Classes, Interfaces, Objetos e Registros}
\subsection*{TDialogs Classe}
\subsubsection*{\large{\textbf{Hierarquia}}\normalsize\hspace{1ex}\hfill}
TDialogs {$>$} mi.rtl.objects.consts.dialogs.TDialogs
%%%%Descrição
\subsubsection*{\large{\textbf{Métodos}}\normalsize\hspace{1ex}\hfill}
\paragraph*{Create}\hspace*{\fill}

\begin{list}{}{
\settowidth{\tmplength}{\textbf{Declaração}}
\setlength{\itemindent}{0cm}
\setlength{\listparindent}{0cm}
\setlength{\leftmargin}{\evensidemargin}
\addtolength{\leftmargin}{\tmplength}
\settowidth{\labelsep}{X}
\addtolength{\leftmargin}{\labelsep}
\setlength{\labelwidth}{\tmplength}
}
\begin{flushleft}
\item[\textbf{Declaração}\hfill]
\begin{ttfamily}
public constructor Create(aOwner: TObjectsConsts); overload; override;\end{ttfamily}


\end{flushleft}
\end{list}
\paragraph*{CreateMessageDialog}\hspace*{\fill}

\begin{list}{}{
\settowidth{\tmplength}{\textbf{Declaração}}
\setlength{\itemindent}{0cm}
\setlength{\listparindent}{0cm}
\setlength{\leftmargin}{\evensidemargin}
\addtolength{\leftmargin}{\tmplength}
\settowidth{\labelsep}{X}
\addtolength{\leftmargin}{\labelsep}
\setlength{\labelwidth}{\tmplength}
}
\begin{flushleft}
\item[\textbf{Declaração}\hfill]
\begin{ttfamily}
public function CreateMessageDialog( aCaption, aMsg: string; DlgType: TMsgDlgType; Buttons: TMsgDlgButtons): integer; overload;\end{ttfamily}


\end{flushleft}
\par
\item[\textbf{Descrição}]
\begin{itemize}
\item O método \textbf{\begin{ttfamily}CreateMessageDialog\end{ttfamily}} mostra uma mensagem formatada onde a função reconhece {\^{}}M para passagem de linha, {\^{}}J retorno do carro e {\^{}}C.

\begin{itemize}
\item \textbf{NOTA} \begin{itemize}
\item O texto entre {\^{}}C vai ficar alinhado no centro do topo do formulário.
\end{itemize}
\item \textbf{EXEMPLO}

\texttt{}
\end{itemize}
\end{itemize}

\end{list}
\paragraph*{Alert}\hspace*{\fill}

\begin{list}{}{
\settowidth{\tmplength}{\textbf{Declaração}}
\setlength{\itemindent}{0cm}
\setlength{\listparindent}{0cm}
\setlength{\leftmargin}{\evensidemargin}
\addtolength{\leftmargin}{\tmplength}
\settowidth{\labelsep}{X}
\addtolength{\leftmargin}{\labelsep}
\setlength{\labelwidth}{\tmplength}
}
\begin{flushleft}
\item[\textbf{Declaração}\hfill]
\begin{ttfamily}
public Procedure Alert(aTitle: AnsiString;aMsg:AnsiString); override;\end{ttfamily}


\end{flushleft}
\par
\item[\textbf{Descrição}]
\begin{itemize}
\item A procedure \textbf{\begin{ttfamily}Alert\end{ttfamily}} executa um dialogo com botão \textbf{OK}
\end{itemize}

\end{list}
\paragraph*{Confirm}\hspace*{\fill}

\begin{list}{}{
\settowidth{\tmplength}{\textbf{Declaração}}
\setlength{\itemindent}{0cm}
\setlength{\listparindent}{0cm}
\setlength{\leftmargin}{\evensidemargin}
\addtolength{\leftmargin}{\tmplength}
\settowidth{\labelsep}{X}
\addtolength{\leftmargin}{\labelsep}
\setlength{\labelwidth}{\tmplength}
}
\begin{flushleft}
\item[\textbf{Declaração}\hfill]
\begin{ttfamily}
public Function Confirm(aTitle: AnsiString;aPergunta:AnsiString):Boolean; override;\end{ttfamily}


\end{flushleft}
\par
\item[\textbf{Descrição}]
\begin{itemize}
\item A procedure \textbf{\begin{ttfamily}Confirm\end{ttfamily}} executa um diálogo com dois botões: \textbf{OK} e \textbf{Cancel}

\begin{itemize}
\item \textbf{RETORNA:} \begin{itemize}
\item \textbf{True} : Se o botão \textbf{OK} foi pŕessionando;
\item \textbf{False} : Se o botão \textbf{Cancel} foi pŕessionando.
\end{itemize}
\item \textbf{EXEMPLO}

\texttt{\\\nopagebreak[3]
\\\nopagebreak[3]
}\textbf{procedure}\texttt{~TForm1.Test{\_}tobjects{\_}dlgs{\_}Confirm;\\\nopagebreak[3]
}\textbf{begin}\texttt{\\\nopagebreak[3]
~~}\textbf{with}\texttt{~TObjectss.dlgs~}\textbf{do}\texttt{\\\nopagebreak[3]
~~~~}\textbf{if}\texttt{~Confirm('Test{\_}tobjects{\_}dlgs{\_}Confirm','Continua~o~processamento?')\\\nopagebreak[3]
~~~~}\textbf{then}\texttt{~Alert('Test{\_}tobjects{\_}dlgs{\_}Confirm','Confirmado~a~ação!')\\\nopagebreak[3]
~~~~}\textbf{else}\texttt{~Alert('Test{\_}tobjects{\_}dlgs{\_}Confirm','Não~confirmado~a~ação!');\\\nopagebreak[3]
}\textbf{end}\texttt{;\\
}
\end{itemize}
\end{itemize}

\end{list}
\paragraph*{Prompt}\hspace*{\fill}

\begin{list}{}{
\settowidth{\tmplength}{\textbf{Declaração}}
\setlength{\itemindent}{0cm}
\setlength{\listparindent}{0cm}
\setlength{\leftmargin}{\evensidemargin}
\addtolength{\leftmargin}{\tmplength}
\settowidth{\labelsep}{X}
\addtolength{\leftmargin}{\labelsep}
\setlength{\labelwidth}{\tmplength}
}
\begin{flushleft}
\item[\textbf{Declaração}\hfill]
\begin{ttfamily}
public Function Prompt(aTitle: AnsiString;aPergunta:AnsiString;Var aResult: AnsiString):Boolean; override;\end{ttfamily}


\end{flushleft}
\par
\item[\textbf{Descrição}]
\begin{itemize}
\item A função \textbf{\begin{ttfamily}Prompt\end{ttfamily}} mostra um dialogo com dois botões \textbf{OK} e \textbf{Cancel} e uma entrada de dados solicitando que o usuário digite um valor.

\begin{itemize}
\item \textbf{RETORNA:} \begin{itemize}
\item \textbf{True} : Se o botão \textbf{ok} foi pŕessionando;
\item \textbf{False} : Se o botão \textbf{cancel} foi pŕessionando.
\item \textbf{aResult} : Retorna a string digitada no formulário;
\end{itemize}
\item \textbf{EXEMPLO}

\texttt{\\\nopagebreak[3]
\\\nopagebreak[3]
}\textbf{procedure}\texttt{~TForm1.Test{\_}tobjects{\_}dlgs{\_}Prompt;\\\nopagebreak[3]
~~}\textbf{var}\texttt{~idade,fmt~:~}\textbf{string}\texttt{;\\\nopagebreak[3]
}\textbf{begin}\texttt{\\\nopagebreak[3]
~~idade~:=~'';\\\nopagebreak[3]
~~}\textbf{with}\texttt{~TObjectss.dlgs~}\textbf{do}\texttt{\\\nopagebreak[3]
~~~~}\textbf{if}\texttt{~Prompt('Test~de~Dlgs.Prompt','Qual~a~sua~idade',idade)\\\nopagebreak[3]
~~~~}\textbf{then}\texttt{~}\textbf{begin}\texttt{\\\nopagebreak[3]
~~~~~~~~~~~fmt~:=~format('Idade~digitada:~{\%}s~~~'+{\^{}}M+\\\nopagebreak[3]
~~~~~~~~~~~~~~~~~~~~~~~~~'Idade~de~meu~pai~é~{\%}d~',[idade,102]);\\\nopagebreak[3]
~~~~~~~~~~~Alert('Test~de~Dlgs.Prompt',fmt)~\textit{//}\\\nopagebreak[3]
~~~~~~~~~}\textbf{end}\texttt{\\\nopagebreak[3]
~~~~}\textbf{else}\texttt{~Alert('Test~de~Dlgs.Prompt','Ok.~Respeito~sua~privacidade.');\\\nopagebreak[3]
}\textbf{end}\texttt{;\\
}
\end{itemize}
\end{itemize}

\end{list}
\paragraph*{GetPassword}\hspace*{\fill}

\begin{list}{}{
\settowidth{\tmplength}{\textbf{Declaração}}
\setlength{\itemindent}{0cm}
\setlength{\listparindent}{0cm}
\setlength{\leftmargin}{\evensidemargin}
\addtolength{\leftmargin}{\tmplength}
\settowidth{\labelsep}{X}
\addtolength{\leftmargin}{\labelsep}
\setlength{\labelwidth}{\tmplength}
}
\begin{flushleft}
\item[\textbf{Declaração}\hfill]
\begin{ttfamily}
public Function GetPassword(aTitle: AnsiString; var apassword:AnsiString):Boolean; Overload; override;\end{ttfamily}


\end{flushleft}
\par
\item[\textbf{Descrição}]
\begin{itemize}
\item A função \textbf{\begin{ttfamily}GetPassword\end{ttfamily}} mostra um diálogo para receber um valor sem mostrar o que foi digitado. O formulário possui dois botões \textbf{OK} e \textbf{Cancel}

\begin{itemize}
\item \textbf{RETORNA:} \begin{itemize}
\item \textbf{True} : Se o botão \textbf{ok} foi pressionado;
\item \textbf{False} : Se o botão \textbf{cancel} foi pressionado.
\item \textbf{apassword} : Retorna a string com a senha do usuário.
\end{itemize}
\end{itemize}
\end{itemize}

\end{list}
\paragraph*{GetPassword}\hspace*{\fill}

\begin{list}{}{
\settowidth{\tmplength}{\textbf{Declaração}}
\setlength{\itemindent}{0cm}
\setlength{\listparindent}{0cm}
\setlength{\leftmargin}{\evensidemargin}
\addtolength{\leftmargin}{\tmplength}
\settowidth{\labelsep}{X}
\addtolength{\leftmargin}{\labelsep}
\setlength{\labelwidth}{\tmplength}
}
\begin{flushleft}
\item[\textbf{Declaração}\hfill]
\begin{ttfamily}
public Function GetPassword(aTitle: AnsiString; var aUsername:AnsiString; var apassword:AnsiString):Boolean; Overload; override;\end{ttfamily}


\end{flushleft}
\par
\item[\textbf{Descrição}]
\begin{itemize}
\item A função \textbf{\begin{ttfamily}GetPassword\end{ttfamily}} mostra um dialogo solicitando o login do usuário e a senha e dois botões \textbf{OK} e \textbf{Cancel}

\begin{itemize}
\item \textbf{RETORNA:} \begin{itemize}
\item \textbf{True} : Se o botão \textbf{ok} foi pŕessionando;
\item \textbf{False} : Se o botão \textbf{cancel} foi pŕessionando.
\item \textbf{aUsername} : Retorna a string com nome do usuário.
\item \textbf{apassword} : Retorna a string com a senha do usuário.
\end{itemize}
\item \textbf{EXEMPLO}

\texttt{\\\nopagebreak[3]
\\\nopagebreak[3]
}\textbf{procedure}\texttt{~TForm1.Test{\_}tobjects{\_}dlgs{\_}password;\\\nopagebreak[3]
~~}\textbf{Var}\texttt{\\\nopagebreak[3]
~~~~s,u~:~}\textbf{string}\texttt{;\\\nopagebreak[3]
}\textbf{begin}\texttt{\\\nopagebreak[3]
~~s~:=~'';\\\nopagebreak[3]
~~}\textbf{with}\texttt{~TObjectss.dlgs~}\textbf{do}\texttt{\\\nopagebreak[3]
~~~~}\textbf{if}\texttt{~GetPassword('Password',u,s)\\\nopagebreak[3]
~~~~}\textbf{then}\texttt{~Alert('Password','A~senha~digitada~é:~'+S)\\\nopagebreak[3]
~~~~}\textbf{else}\texttt{~Alert('Password','Senha~não~informada');\\\nopagebreak[3]
\\\nopagebreak[3]
}\textbf{end}\texttt{;\\
}
\end{itemize}
\end{itemize}

\end{list}
\section{Constantes}
\subsection*{{\_}Dialogs}
\begin{list}{}{
\settowidth{\tmplength}{\textbf{Declaração}}
\setlength{\itemindent}{0cm}
\setlength{\listparindent}{0cm}
\setlength{\leftmargin}{\evensidemargin}
\addtolength{\leftmargin}{\tmplength}
\settowidth{\labelsep}{X}
\addtolength{\leftmargin}{\labelsep}
\setlength{\labelwidth}{\tmplength}
}
\begin{flushleft}
\item[\textbf{Declaração}\hfill]
\begin{ttfamily}
{\_}Dialogs : mi.ui.Dialogs.TDialogs = nil;\end{ttfamily}


\end{flushleft}
\end{list}
\chapter{Unit mi.ui.lcl.form}
\section{Uses}
\begin{itemize}
\item \begin{ttfamily}uMi{\_}ui{\_}scrollbox{\_}lcl\end{ttfamily}(\ref{uMi_ui_scrollbox_lcl})\item \begin{ttfamily}umi{\_}ui{\_}dmxscroller{\_}form{\_}lcl{\_}attributes\end{ttfamily}(\ref{umi_ui_dmxscroller_form_lcl_attributes})\item \begin{ttfamily}umi{\_}ui{\_}bitbtn{\_}lcl\end{ttfamily}(\ref{umi_ui_bitbtn_lcl})\item \begin{ttfamily}umi{\_}ui{\_}button{\_}lcl\end{ttfamily}(\ref{umi_ui_button_lcl})\item \begin{ttfamily}umi{\_}ui{\_}checkbox{\_}lcl\end{ttfamily}(\ref{umi_ui_checkbox_lcl})\item \begin{ttfamily}umi{\_}ui{\_}radiogroup{\_}lcl\end{ttfamily}(\ref{umi_ui_radiogroup_lcl})\item \begin{ttfamily}uMi{\_}ui{\_}ComboBox{\_}lcl\end{ttfamily}(\ref{uMi_ui_ComboBox_lcl})\item \begin{ttfamily}uMi{\_}Ui{\_}DBCheckBox{\_}Lcl\end{ttfamily}(\ref{uMi_Ui_DBCheckBox_Lcl})\item \begin{ttfamily}uMi{\_}Ui{\_}DbComboBox{\_}lcl\end{ttfamily}(\ref{uMi_Ui_DbComboBox_lcl})\item \begin{ttfamily}uMI{\_}ui{\_}DbEdit{\_}LCL\end{ttfamily}(\ref{uMI_ui_DbEdit_LCL})\item \begin{ttfamily}umi{\_}ui{\_}dblookupComboBox{\_}lcl\end{ttfamily}(\ref{umi_ui_dblookupComboBox_lcl})\item \begin{ttfamily}uMI{\_}ui{\_}DbRadioGroup{\_}Lcl\end{ttfamily}(\ref{uMI_ui_DbRadioGroup_Lcl})\item \begin{ttfamily}uMi{\_}ui{\_}Label{\_}lcl\end{ttfamily}(\ref{uMi_ui_Label_lcl})\item \begin{ttfamily}uMi{\_}ui{\_}maskedit{\_}lcl\end{ttfamily}(\ref{uMi_ui_maskedit_lcl})\item \begin{ttfamily}uMi{\_}ui{\_}Dmxscroller{\_}form{\_}lcl\end{ttfamily}\item \begin{ttfamily}umi{\_}ui{\_}InputBox{\_}lcl\end{ttfamily}(\ref{umi_ui_InputBox_lcl})\item \begin{ttfamily}umi{\_}ui{\_}dmxscroller{\_}form{\_}lcl{\_}ds\end{ttfamily}(\ref{umi_ui_dmxscroller_form_lcl_ds})\item \begin{ttfamily}LazarusPackageIntf\end{ttfamily}\end{itemize}
\chapter{Unit mi{\_}rtl{\_}ui{\_}custom{\_}application}
\section{Descrição}
A unit \textbf{\begin{ttfamily}mi{\_}rtl{\_}ui{\_}custom{\_}application\end{ttfamily}} implementa a classe \begin{ttfamily}TMI{\_}ui{\_}Custom{\_}Application\end{ttfamily}(\ref{mi_rtl_ui_custom_application.TMI_ui_Custom_Application}).

\begin{itemize}
\item \textbf{VERSÃO} \begin{itemize}
\item Alpha {-} 0.5.0.687
\end{itemize}
\item \textbf{CÓDIGO FONTE}: \begin{itemize}
\item 
\end{itemize}
\item \textbf{PENDÊNCIAS}
\item \textbf{REFERÊNCIA} \begin{itemize}
\item [PostgresSql MULTIBYTE{-}CHARSET{-}SUPPORTED](https://www.postgresql.org/docs/current/multibyte.html{\#}MULTIBYTE-CHARSET-SUPPORTED)
\item [Documento oficial do componente \textbf{sqldb}](https://www.freepascal.org/docs-html/fcl/sqldb/index.html)
\item [Exemplos de uso do \textbf{SqlDb}](https://www.freepascal.org/docs-html/fcl/sqldb/usingsqldb.html)
\item [SqlDBHowto](https://wiki.freepascal.org/SqlDBHowto)
\item [tsqlquery.insertsql](https://www.freepascal.org/docs-html/fcl/sqldb/tsqlquery.insertsql.html)
\end{itemize}
\item \textbf{HISTÓRICO} \begin{itemize}
\item Criado por: Paulo Sérgio da Silva Pacheco paulosspacheco@yahoo.com.br) \begin{itemize}
\item \textbf{2022{-}03{-}29 16:06} \begin{itemize}
\item Criar a unit \textbf{\begin{ttfamily}mi{\_}rtl{\_}ui{\_}custom{\_}application\end{ttfamily}} e analisar o que preciso fazer para integrar com a unit \textbf{\begin{ttfamily}mi{\_}ui{\_}Dmxscroller{\_}sql\end{ttfamily}(\ref{mi_ui_Dmxscroller_sql})} ✅
\end{itemize}
\item \textbf{2022{-}04{-}06 15:40} \begin{itemize}
\item Implementar o evento Get{\_}ParametersCloseQuery e salvar o formulário quando ele for executado. ✅
\item No evento Get{\_}ParametersCloseQuery Checar se o usuário é válido. ✅
\item Criar método DoOnValidUser ✅
\item Criar método \textbf{Get{\_}ParametersCloseQuery} para executar o evento DoOnValidUser. ✅
\end{itemize}
\item \textbf{2022{-}04{-}07 08:43} \begin{itemize}
\item Cada banco de dados SQL tem alguns parâmetros básicos para sua conexão: \begin{itemize}
\item Criar as propriedades de \begin{ttfamily}TMI{\_}ui{\_}Custom{\_}Application\end{ttfamily}(\ref{mi_rtl_ui_custom_application.TMI_ui_Custom_Application}) para que o usuário informe esses parâmetros: \begin{itemize}
\item Banco de dados PostgresSQL \begin{itemize}
\item CharSet = 'UTF8'; ✅
\item ConnectorType:='PostgreSQL'; ✅
\item HostName := '127.0.0.1'; ✅
\item UserName := 'postgres'; ✅
\item Password := 'masterkey'; ✅
\item DatabaseName:= 'maricarai'; ✅
\item DirDatabaseName:= './'; ✅
\end{itemize}
\item connected :Boolean ✅
\item Options : TSQLConnectorOptions ✅
\end{itemize}
\item Documentar as propriedade criadas hoje ✅.
\end{itemize}
\end{itemize}
\item \textbf{2022{-}04{-}08} \begin{itemize}
\item \textbf{09:00} \begin{itemize}
\item Escrever a descrição da classe \begin{ttfamily}TMI{\_}ui{\_}Custom{\_}Application\end{ttfamily}(\ref{mi_rtl_ui_custom_application.TMI_ui_Custom_Application}). ✅
\end{itemize}
\item \textbf{11:18} \begin{itemize}
\item Alterar o nome da propriedade \textbf{Options} para \textbf{SQLConnectorOptions}. ✅
\item Criar a propriedade \textbf{SQLTransactionOptions} ✅
\end{itemize}
\item \textbf{14:22} \begin{itemize}
\item Documentar as propriedades da classe \begin{ttfamily}TMi{\_}ui{\_}Custom{\_}Application\end{ttfamily}(\ref{mi_rtl_ui_custom_application.TMI_ui_Custom_Application}). ✅
\end{itemize}
\item \textbf{21:40} \begin{itemize}
\item Em \begin{ttfamily}TMI{\_}ui{\_}Custom{\_}Application.Get{\_}ParametersCloseQuery\end{ttfamily}(\ref{mi_rtl_ui_custom_application.TMI_ui_Custom_Application-Get_ParametersCloseQuery}) antes de checar se os parãmetros são válidos, transferir os campos do formulários para as propriedades equivalentes.
\end{itemize}
\end{itemize}
\item \textbf{2022{-}04{-}14 14:58} \begin{itemize}
\item Criar a constante \textbf{OkCreateDataBase} e o método \textbf{CreateDataBase}. ✅
\end{itemize}
\item \textbf{2022{-}04{-}15 10:00} \begin{itemize}
\item Criar método \textbf{NameDataBase} que retorna o nome do database porque o nome do dataBase é diferente em cada banco de dados. O postres usa um nome simples e o ip para acessar o banco, o SqLite3 usa o nome da pasta + nome do database + ext. ✅
\end{itemize}
\end{itemize}
\end{itemize}
\end{itemize}
\section{Uses}
\begin{itemize}
\item \begin{ttfamily}Classes\end{ttfamily}\item \begin{ttfamily}SysUtils\end{ttfamily}\item \begin{ttfamily}SqlDb\end{ttfamily}\item \begin{ttfamily}DB\end{ttfamily}\item \begin{ttfamily}BufDataset\end{ttfamily}\item \begin{ttfamily}PQConnection\end{ttfamily}\item \begin{ttfamily}CustApp\end{ttfamily}\item \begin{ttfamily}mi.rtl.Types\end{ttfamily}(\ref{mi.rtl.Types})\item \begin{ttfamily}mi{\_}rtl{\_}ui{\_}Dmxscroller\end{ttfamily}(\ref{mi_rtl_ui_Dmxscroller})\item \begin{ttfamily}mi.rtl.Objects.Methods.Paramexecucao.Application\end{ttfamily}(\ref{mi.rtl.Objects.Methods.Paramexecucao.Application})\end{itemize}
\section{Visão Geral}
\begin{description}
\item[\texttt{\begin{ttfamily}TMI{\_}ui{\_}Custom{\_}Application\end{ttfamily} Classe}]
\end{description}
\begin{description}
\item[\texttt{Mi{\_}ui{\_}Custom{\_}Application}]
\item[\texttt{Set{\_}Mi{\_}ui{\_}Custom{\_}Application}]
\end{description}
\section{Classes, Interfaces, Objetos e Registros}
\subsection*{TMI{\_}ui{\_}Custom{\_}Application Classe}
\subsubsection*{\large{\textbf{Hierarquia}}\normalsize\hspace{1ex}\hfill}
TMI{\_}ui{\_}Custom{\_}Application {$>$} \begin{ttfamily}TApplication\end{ttfamily}(\ref{mi.rtl.Objects.Methods.Paramexecucao.Application.TApplication}) {$>$} \begin{ttfamily}TApplicationConsts\end{ttfamily}(\ref{mi.rtl.Objects.Methods.Paramexecucao.Application.TApplicationConsts}) {$>$} \begin{ttfamily}TApplication{\_}type\end{ttfamily}(\ref{mi.rtl.Objects.Methods.Paramexecucao.Application.TApplication_type}) {$>$} \begin{ttfamily}TApplicationAbstract\end{ttfamily}(\ref{mi.rtl.ApplicationAbstract.TApplicationAbstract}) {$>$} 
TCustomApplication
\subsubsection*{\large{\textbf{Descrição}}\normalsize\hspace{1ex}\hfill}
no description available, TApplication description followsno description available, TApplicationConsts description followsno description available, TApplication{\_}type description followsA class \textit{\begin{ttfamily}TApplication{\_}type\end{ttfamily}}* é usada para capsular todas as variáveis globais do projeto e gerenciar o ciclo de vida do aplicativo\subsubsection*{\large{\textbf{Propriedades}}\normalsize\hspace{1ex}\hfill}
\paragraph*{SQLConnectorOptions}\hspace*{\fill}

\begin{list}{}{
\settowidth{\tmplength}{\textbf{Declaração}}
\setlength{\itemindent}{0cm}
\setlength{\listparindent}{0cm}
\setlength{\leftmargin}{\evensidemargin}
\addtolength{\leftmargin}{\tmplength}
\settowidth{\labelsep}{X}
\addtolength{\leftmargin}{\labelsep}
\setlength{\labelwidth}{\tmplength}
}
\begin{flushleft}
\item[\textbf{Declaração}\hfill]
\begin{ttfamily}
published property SQLConnectorOptions : TSQLConnectionOptions Read {\_}SQLConnectorOptions write SetSQLConnectorOptions default [];\end{ttfamily}


\end{flushleft}
\par
\item[\textbf{Descrição}]
A propriedade \textbf{\begin{ttfamily}SQLConnectorOptions\end{ttfamily}} é usada para controlar o comportamento do SqlDb para esta conexão.

\begin{itemize}
\item As seguintes opções podem ser definidas: \begin{itemize}
\item Type TSQLConnectionOption = (scoExplicitConnect, scoApplyUpdatesChecksRowsAffected);
\item \textbf{ONDE}: \begin{itemize}
\item \textbf{scoExplicitConnect} : \begin{itemize}
\item Quando definido, a conexão deve ser feita explicitamente.
\item O comportamento padrão é para \textbf{TSQLQuery} abrir implicitamente a conexão conforme necessário.
\end{itemize}
\item \textbf{scoApplyUpdatesChecksRowsAffected} : \begin{itemize}
\item Quando definido, sempre que uma instrução SQL de atualização é executada durante \textbf{ApplyOptions} de um conjunto de dados, \textbf{RowsAffected} é verificado e deve ser igual a 1.
\end{itemize}
\end{itemize}
\end{itemize}
\item \textbf{REFERÊNCIAS} \begin{itemize}
\item [tsqltransaction.options](https://www.freepascal.org/docs-html/fcl/sqldb/tsqltransaction.options.html)
\end{itemize}
\end{itemize}

\end{list}
\paragraph*{SQLTransactionOptions}\hspace*{\fill}

\begin{list}{}{
\settowidth{\tmplength}{\textbf{Declaração}}
\setlength{\itemindent}{0cm}
\setlength{\listparindent}{0cm}
\setlength{\leftmargin}{\evensidemargin}
\addtolength{\leftmargin}{\tmplength}
\settowidth{\labelsep}{X}
\addtolength{\leftmargin}{\labelsep}
\setlength{\labelwidth}{\tmplength}
}
\begin{flushleft}
\item[\textbf{Declaração}\hfill]
\begin{ttfamily}
published property SQLTransactionOptions : TSQLTransactionOptions Read {\_}SQLTransactionOptions write SetSQLTransactionOptions default [];\end{ttfamily}


\end{flushleft}
\par
\item[\textbf{Descrição}]
A propriedade \textbf{\begin{ttfamily}SQLTransactionOptions\end{ttfamily}} é usada para controlar o comportamento do SqlDb para esta transação.

\begin{itemize}
\item As seguintes opções podem ser definidas: \begin{itemize}
\item Type TSQLTransactionOption = (stoUseImplicit, stoExplicitStart);
\item \textbf{ONDE}: \begin{itemize}
\item \textbf{stoUseImplicit} : \begin{itemize}
\item Use o suporte a transações implícitas do mecanismo de banco de dados. Isso significa que nenhum comando explícito de início e parada de transação será enviado ao servidor quando os métodos \textbf{Commit} ou \textbf{Rollback} forem chamados (tornando{-}os efetivamente sem operação no nível do banco de dados).
\end{itemize}
\item \textbf{stoExplicitStart} \begin{itemize}
\item Quando definido, sempre que uma instrução SQL é executada, a transação deve ter sido iniciada explicitamente. O comportamento padrão é que \textbf{TSQLStatement} ou \textbf{TSQLQuery} iniciem a transação conforme necessário.
\end{itemize}
\end{itemize}
\end{itemize}
\item \textbf{REFERÊNCIAS} \begin{itemize}
\item [tsqltransaction.options](https://www.freepascal.org/docs-html/fcl/sqldb/tsqltransaction.options.html)
\end{itemize}
\end{itemize}

\end{list}
\paragraph*{Connected}\hspace*{\fill}

\begin{list}{}{
\settowidth{\tmplength}{\textbf{Declaração}}
\setlength{\itemindent}{0cm}
\setlength{\listparindent}{0cm}
\setlength{\leftmargin}{\evensidemargin}
\addtolength{\leftmargin}{\tmplength}
\settowidth{\labelsep}{X}
\addtolength{\leftmargin}{\labelsep}
\setlength{\labelwidth}{\tmplength}
}
\begin{flushleft}
\item[\textbf{Declaração}\hfill]
\begin{ttfamily}
published property Connected : Boolean Read GetConnected write SetConnected;\end{ttfamily}


\end{flushleft}
\par
\item[\textbf{Descrição}]
A propriedade \textbf{\begin{ttfamily}Connected\end{ttfamily}} conecta ao banco de dados selecionado.

\begin{itemize}
\item True = Conecta ao banco;
\item False = Desconecta do banco;
\end{itemize}

\end{list}
\paragraph*{ConnectorType}\hspace*{\fill}

\begin{list}{}{
\settowidth{\tmplength}{\textbf{Declaração}}
\setlength{\itemindent}{0cm}
\setlength{\listparindent}{0cm}
\setlength{\leftmargin}{\evensidemargin}
\addtolength{\leftmargin}{\tmplength}
\settowidth{\labelsep}{X}
\addtolength{\leftmargin}{\labelsep}
\setlength{\labelwidth}{\tmplength}
}
\begin{flushleft}
\item[\textbf{Declaração}\hfill]
\begin{ttfamily}
published property ConnectorType : TUiDmxScroller.TConnectorType Read {\_}ConnectorType write {\_}ConnectorType;\end{ttfamily}


\end{flushleft}
\par
\item[\textbf{Descrição}]
O evento \textbf{\begin{ttfamily}ConnectorType\end{ttfamily}} seleciona o tipo de banco de dados a ser conectado

\end{list}
\paragraph*{HostName}\hspace*{\fill}

\begin{list}{}{
\settowidth{\tmplength}{\textbf{Declaração}}
\setlength{\itemindent}{0cm}
\setlength{\listparindent}{0cm}
\setlength{\leftmargin}{\evensidemargin}
\addtolength{\leftmargin}{\tmplength}
\settowidth{\labelsep}{X}
\addtolength{\leftmargin}{\labelsep}
\setlength{\labelwidth}{\tmplength}
}
\begin{flushleft}
\item[\textbf{Declaração}\hfill]
\begin{ttfamily}
published property HostName : AnsiString Read GetHostName write SetHostName;\end{ttfamily}


\end{flushleft}
\par
\item[\textbf{Descrição}]
A propriedade \textbf{\begin{ttfamily}HostName\end{ttfamily}} informa ao \textbf{\begin{ttfamily}SQLConnector\end{ttfamily}(\ref{mi_rtl_ui_custom_application.TMI_ui_Custom_Application-SQLConnector})} o \textbf{IP} ou domínimo onde o banco de dados foi hospedado.

\end{list}
\paragraph*{DirDataBaseName}\hspace*{\fill}

\begin{list}{}{
\settowidth{\tmplength}{\textbf{Declaração}}
\setlength{\itemindent}{0cm}
\setlength{\listparindent}{0cm}
\setlength{\leftmargin}{\evensidemargin}
\addtolength{\leftmargin}{\tmplength}
\settowidth{\labelsep}{X}
\addtolength{\leftmargin}{\labelsep}
\setlength{\labelwidth}{\tmplength}
}
\begin{flushleft}
\item[\textbf{Declaração}\hfill]
\begin{ttfamily}
published property DirDataBaseName : AnsiString Read GetDirDataBaseName write SetDirDataBaseName;\end{ttfamily}


\end{flushleft}
\par
\item[\textbf{Descrição}]
A propriedade \textbf{\begin{ttfamily}DirDataBaseName\end{ttfamily}} contém a pasta do HD do servidor onde o banco de banco foi hospedado.

\begin{itemize}
\item \textbf{Não foi implementado ainda} \begin{itemize}
\item Preciso de mais informações de como alterar a pasta dos bancos de dados PostgreSQL e SQLite3.
\end{itemize}
\end{itemize}

\end{list}
\paragraph*{DatabaseName}\hspace*{\fill}

\begin{list}{}{
\settowidth{\tmplength}{\textbf{Declaração}}
\setlength{\itemindent}{0cm}
\setlength{\listparindent}{0cm}
\setlength{\leftmargin}{\evensidemargin}
\addtolength{\leftmargin}{\tmplength}
\settowidth{\labelsep}{X}
\addtolength{\leftmargin}{\labelsep}
\setlength{\labelwidth}{\tmplength}
}
\begin{flushleft}
\item[\textbf{Declaração}\hfill]
\begin{ttfamily}
published property DatabaseName : AnsiString Read GetDatabaseName write SetDatabaseName;\end{ttfamily}


\end{flushleft}
\par
\item[\textbf{Descrição}]
A propriedade \textbf{\begin{ttfamily}DatabaseName\end{ttfamily}} contém o nome do Banco de Dados dentro do PostegresSQL ou do SQLite3.

\end{list}
\paragraph*{UserName}\hspace*{\fill}

\begin{list}{}{
\settowidth{\tmplength}{\textbf{Declaração}}
\setlength{\itemindent}{0cm}
\setlength{\listparindent}{0cm}
\setlength{\leftmargin}{\evensidemargin}
\addtolength{\leftmargin}{\tmplength}
\settowidth{\labelsep}{X}
\addtolength{\leftmargin}{\labelsep}
\setlength{\labelwidth}{\tmplength}
}
\begin{flushleft}
\item[\textbf{Declaração}\hfill]
\begin{ttfamily}
published property UserName : AnsiString Read GetUserName write SetUserName;\end{ttfamily}


\end{flushleft}
\par
\item[\textbf{Descrição}]
A propriedade \textbf{\begin{ttfamily}UserName\end{ttfamily}} contém o nome do usuário conectado ao banco de dados.

\end{list}
\paragraph*{Password}\hspace*{\fill}

\begin{list}{}{
\settowidth{\tmplength}{\textbf{Declaração}}
\setlength{\itemindent}{0cm}
\setlength{\listparindent}{0cm}
\setlength{\leftmargin}{\evensidemargin}
\addtolength{\leftmargin}{\tmplength}
\settowidth{\labelsep}{X}
\addtolength{\leftmargin}{\labelsep}
\setlength{\labelwidth}{\tmplength}
}
\begin{flushleft}
\item[\textbf{Declaração}\hfill]
\begin{ttfamily}
published property Password : AnsiString Read GetPassword write SetPassword;\end{ttfamily}


\end{flushleft}
\par
\item[\textbf{Descrição}]
A propriedade \textbf{\begin{ttfamily}Password\end{ttfamily}} contém a senha do usuário conectado ao banco de dados.

\end{list}
\paragraph*{CharSet}\hspace*{\fill}

\begin{list}{}{
\settowidth{\tmplength}{\textbf{Declaração}}
\setlength{\itemindent}{0cm}
\setlength{\listparindent}{0cm}
\setlength{\leftmargin}{\evensidemargin}
\addtolength{\leftmargin}{\tmplength}
\settowidth{\labelsep}{X}
\addtolength{\leftmargin}{\labelsep}
\setlength{\labelwidth}{\tmplength}
}
\begin{flushleft}
\item[\textbf{Declaração}\hfill]
\begin{ttfamily}
published property CharSet : AnsiString Read GetCharSet write SetCharSet;\end{ttfamily}


\end{flushleft}
\par
\item[\textbf{Descrição}]
A propriedade \textbf{\begin{ttfamily}CharSet\end{ttfamily}} é usada para definir o tipo de caractere do banco de dados.

\begin{itemize}
\item \textbf{NOTA} \begin{itemize}
\item Deve ser informado em tempo de designe do projeto.
\end{itemize}
\item \textbf{REFERÊNCIAS} \begin{itemize}
\item [\begin{ttfamily}CHARSET\end{ttfamily}{-}TABLE](https://www.postgresql.org/docs/current/multibyte.html{\#}CHARSET-TABLE)
\end{itemize}
\end{itemize}

\end{list}
\paragraph*{onValidUser}\hspace*{\fill}

\begin{list}{}{
\settowidth{\tmplength}{\textbf{Declaração}}
\setlength{\itemindent}{0cm}
\setlength{\listparindent}{0cm}
\setlength{\leftmargin}{\evensidemargin}
\addtolength{\leftmargin}{\tmplength}
\settowidth{\labelsep}{X}
\addtolength{\leftmargin}{\labelsep}
\setlength{\labelwidth}{\tmplength}
}
\begin{flushleft}
\item[\textbf{Declaração}\hfill]
\begin{ttfamily}
published property onValidUser : TOnValidUser Read {\_}OnValidUser write {\_}onValidUser;\end{ttfamily}


\end{flushleft}
\par
\item[\textbf{Descrição}]
O evento \textbf{\begin{ttfamily}onValidUser\end{ttfamily}} é disparado toda vez que o \begin{ttfamily}TUiDmxScroller\end{ttfamily}(\ref{mi_rtl_ui_Dmxscroller.TUiDmxScroller}) ativado.

\end{list}
\paragraph*{SQLConnector}\hspace*{\fill}

\begin{list}{}{
\settowidth{\tmplength}{\textbf{Declaração}}
\setlength{\itemindent}{0cm}
\setlength{\listparindent}{0cm}
\setlength{\leftmargin}{\evensidemargin}
\addtolength{\leftmargin}{\tmplength}
\settowidth{\labelsep}{X}
\addtolength{\leftmargin}{\labelsep}
\setlength{\labelwidth}{\tmplength}
}
\begin{flushleft}
\item[\textbf{Declaração}\hfill]
\begin{ttfamily}
published property SQLConnector : TSQLConnector read {\_}SQLConnector;\end{ttfamily}


\end{flushleft}
\par
\item[\textbf{Descrição}]
A propriedade \textbf{\begin{ttfamily}SQLConnector\end{ttfamily}} é um componente conector de banco de dados versátil para uso com qualquer banco de dados suportado.

\begin{itemize}
\item A incluir uma aplicação \textbf{TMi{\_}UI{\_}Application} na aplicação corrente automáticamente é disponibilizado um conector de acesso o banco de dados.
\item \textbf{REFERÊNCIAS} \begin{itemize}
\item [TSQLConnector](https://wiki.freepascal.org/TSQLConnector)
\item [sqldb/tsqlconnector](https://www.freepascal.org/docs-html/fcl/sqldb/tsqlconnector.html)
\end{itemize}
\end{itemize}

\end{list}
\paragraph*{SQLTransaction}\hspace*{\fill}

\begin{list}{}{
\settowidth{\tmplength}{\textbf{Declaração}}
\setlength{\itemindent}{0cm}
\setlength{\listparindent}{0cm}
\setlength{\leftmargin}{\evensidemargin}
\addtolength{\leftmargin}{\tmplength}
\settowidth{\labelsep}{X}
\addtolength{\leftmargin}{\labelsep}
\setlength{\labelwidth}{\tmplength}
}
\begin{flushleft}
\item[\textbf{Declaração}\hfill]
\begin{ttfamily}
published property SQLTransaction : TSQLTransaction read {\_}SQLTransaction;\end{ttfamily}


\end{flushleft}
\par
\item[\textbf{Descrição}]
A propriedade \textbf{\begin{ttfamily}SQLTransaction\end{ttfamily}} representa uma transação no banco de dados na qual um TSQLQuery é tratado.

\begin{itemize}
\item Na prática, pelo menos uma transação precisa estar ativa para um banco de dados, mesmo que você a utilize apenas para leitura de dados.
\item \textbf{NOTAS} \begin{itemize}
\item Ao usar uma única transação, defina a propriedade TConnection. Transaction para a transação para definir a transação padrão para o banco de dados; a propriedade TSQLTransaction.Database correspondente deve apontar automaticamente para a conexão.
\item Ao ativar uma TSQLTransaction o método \textbf{StartTransaction} inicia uma transação; chamar o método \textbf{Commit} ou o método \textbf{RollBack} confirma (salva) ou reverte (esquece/aborta) a transação. \begin{itemize}
\item Você deve cercar suas transações de banco de dados com eles, a menos que use as propriedades \textbf{Autocommit} ou \textbf{CommitRetaining}.
\end{itemize}
\end{itemize}
\item \textbf{REFERÊNCIAS} \begin{itemize}
\item [tsqltransaction](https://www.freepascal.org/docs-html/fcl/sqldb/tsqltransaction.html)
\end{itemize}
\end{itemize}

\end{list}
\paragraph*{Get{\_}Parameters}\hspace*{\fill}

\begin{list}{}{
\settowidth{\tmplength}{\textbf{Declaração}}
\setlength{\itemindent}{0cm}
\setlength{\listparindent}{0cm}
\setlength{\leftmargin}{\evensidemargin}
\addtolength{\leftmargin}{\tmplength}
\settowidth{\labelsep}{X}
\addtolength{\leftmargin}{\labelsep}
\setlength{\labelwidth}{\tmplength}
}
\begin{flushleft}
\item[\textbf{Declaração}\hfill]
\begin{ttfamily}
published property Get{\_}Parameters : TUiDmxScroller read {\_}Get{\_}Parameters;\end{ttfamily}


\end{flushleft}
\par
\item[\textbf{Descrição}]
A propriedade \textbf{\begin{ttfamily}Get{\_}Parameters\end{ttfamily}} contém o formulário para ler os parâmetros de conexão com o banco de dados.

\end{list}
\subsubsection*{\large{\textbf{Campos}}\normalsize\hspace{1ex}\hfill}
\paragraph*{BufDataSet1}\hspace*{\fill}

\begin{list}{}{
\settowidth{\tmplength}{\textbf{Declaração}}
\setlength{\itemindent}{0cm}
\setlength{\listparindent}{0cm}
\setlength{\leftmargin}{\evensidemargin}
\addtolength{\leftmargin}{\tmplength}
\settowidth{\labelsep}{X}
\addtolength{\leftmargin}{\labelsep}
\setlength{\labelwidth}{\tmplength}
}
\begin{flushleft}
\item[\textbf{Declaração}\hfill]
\begin{ttfamily}
public BufDataSet1: TBufDataSet;\end{ttfamily}


\end{flushleft}
\par
\item[\textbf{Descrição}]
O atributo \textbf{\begin{ttfamily}BufDataSet1\end{ttfamily}} é usado para salvar em disco local no arquivo \textbf{\begin{ttfamily}FileName{\_}Parameters\end{ttfamily}(\ref{mi_rtl_ui_custom_application.TMI_ui_Custom_Application-FileName_Parameters})} os parâmetros informados pelo formulário \textbf{\begin{ttfamily}Get{\_}Parameters\end{ttfamily}(\ref{mi_rtl_ui_custom_application.TMI_ui_Custom_Application-Get_Parameters})}

\end{list}
\paragraph*{DataSource1}\hspace*{\fill}

\begin{list}{}{
\settowidth{\tmplength}{\textbf{Declaração}}
\setlength{\itemindent}{0cm}
\setlength{\listparindent}{0cm}
\setlength{\leftmargin}{\evensidemargin}
\addtolength{\leftmargin}{\tmplength}
\settowidth{\labelsep}{X}
\addtolength{\leftmargin}{\labelsep}
\setlength{\labelwidth}{\tmplength}
}
\begin{flushleft}
\item[\textbf{Declaração}\hfill]
\begin{ttfamily}
public DataSource1: TDataSource;\end{ttfamily}


\end{flushleft}
\par
\item[\textbf{Descrição}]
O atributo \textbf{\begin{ttfamily}DataSource1\end{ttfamily}} permite integrar os dados da classe \textbf{TMiDmxScroller} com os componentes da LCL com DbGrid, DbEdit etc...

\end{list}
\paragraph*{OkCreateDataBase}\hspace*{\fill}

\begin{list}{}{
\settowidth{\tmplength}{\textbf{Declaração}}
\setlength{\itemindent}{0cm}
\setlength{\listparindent}{0cm}
\setlength{\leftmargin}{\evensidemargin}
\addtolength{\leftmargin}{\tmplength}
\settowidth{\labelsep}{X}
\addtolength{\leftmargin}{\labelsep}
\setlength{\labelwidth}{\tmplength}
}
\begin{flushleft}
\item[\textbf{Declaração}\hfill]
\begin{ttfamily}
public const OkCreateDataBase : boolean = false;\end{ttfamily}


\end{flushleft}
\par
\item[\textbf{Descrição}]
A constante \textbf{\begin{ttfamily}OkCreateDataBase\end{ttfamily}} se \textbf{true} executa o método \textbf{\begin{ttfamily}CreateDataBase\end{ttfamily}(\ref{mi_rtl_ui_custom_application.TMI_ui_Custom_Application-CreateDataBase})} se ExisteCreateDataBase = false

\end{list}
\paragraph*{Const{\_}ConnectorType}\hspace*{\fill}

\begin{list}{}{
\settowidth{\tmplength}{\textbf{Declaração}}
\setlength{\itemindent}{0cm}
\setlength{\listparindent}{0cm}
\setlength{\leftmargin}{\evensidemargin}
\addtolength{\leftmargin}{\tmplength}
\settowidth{\labelsep}{X}
\addtolength{\leftmargin}{\labelsep}
\setlength{\labelwidth}{\tmplength}
}
\begin{flushleft}
\item[\textbf{Declaração}\hfill]
\begin{ttfamily}
public const Const{\_}ConnectorType : Array[TUiDmxScroller.TConnectorType] of AnsiString =('PostgreSQL','SqLite3');\end{ttfamily}


\end{flushleft}
\par
\item[\textbf{Descrição}]
A constante \textbf{\begin{ttfamily}Const{\_}ConnectorType\end{ttfamily}} contém a lista de nomes dos tipo de bancos de dados testados pelo componente \textbf{\begin{ttfamily}TMI{\_}ui{\_}Custom{\_}Application\end{ttfamily}(\ref{mi_rtl_ui_custom_application.TMI_ui_Custom_Application})}

\end{list}
\paragraph*{FileName{\_}Parameters}\hspace*{\fill}

\begin{list}{}{
\settowidth{\tmplength}{\textbf{Declaração}}
\setlength{\itemindent}{0cm}
\setlength{\listparindent}{0cm}
\setlength{\leftmargin}{\evensidemargin}
\addtolength{\leftmargin}{\tmplength}
\settowidth{\labelsep}{X}
\addtolength{\leftmargin}{\labelsep}
\setlength{\labelwidth}{\tmplength}
}
\begin{flushleft}
\item[\textbf{Declaração}\hfill]
\begin{ttfamily}
public const FileName{\_}Parameters : AnsiString = '';\end{ttfamily}


\end{flushleft}
\par
\item[\textbf{Descrição}]
A constante \textbf{\begin{ttfamily}FileName{\_}Parameters\end{ttfamily}} contém o nome do arquivo de parâmetros \begin{itemize}
\item A constante \textbf{\begin{ttfamily}FileName{\_}Parameters\end{ttfamily}} é inicializado em \textbf{\begin{ttfamily}TMI{\_}ui{\_}Custom{\_}Application.create\end{ttfamily}(\ref{mi_rtl_ui_custom_application.TMI_ui_Custom_Application-Create})} onde: \begin{itemize}
\item \begin{ttfamily}FileName{\_}Parameters\end{ttfamily} := ParamStr(0)+'{\_}Parameters.bds';
\end{itemize}
\end{itemize}

\end{list}
\paragraph*{{\_}Get{\_}Parameters}\hspace*{\fill}

\begin{list}{}{
\settowidth{\tmplength}{\textbf{Declaração}}
\setlength{\itemindent}{0cm}
\setlength{\listparindent}{0cm}
\setlength{\leftmargin}{\evensidemargin}
\addtolength{\leftmargin}{\tmplength}
\settowidth{\labelsep}{X}
\addtolength{\leftmargin}{\labelsep}
\setlength{\labelwidth}{\tmplength}
}
\begin{flushleft}
\item[\textbf{Declaração}\hfill]
\begin{ttfamily}
protected {\_}Get{\_}Parameters: TUiDmxScroller;\end{ttfamily}


\end{flushleft}
\par
\item[\textbf{Descrição}]
Este atributo é usado pelas classes filhas para implementar classes herdadas de \textbf{\begin{ttfamily}TUiDmxScroller\end{ttfamily}(\ref{mi_rtl_ui_Dmxscroller.TUiDmxScroller})}.

\begin{itemize}
\item No momento (08/04/22 a classe que herdade é: \textbf{TUiDmxScroller{\_}form})
\end{itemize}

\end{list}
\subsubsection*{\large{\textbf{Métodos}}\normalsize\hspace{1ex}\hfill}
\paragraph*{ExistDataBase}\hspace*{\fill}

\begin{list}{}{
\settowidth{\tmplength}{\textbf{Declaração}}
\setlength{\itemindent}{0cm}
\setlength{\listparindent}{0cm}
\setlength{\leftmargin}{\evensidemargin}
\addtolength{\leftmargin}{\tmplength}
\settowidth{\labelsep}{X}
\addtolength{\leftmargin}{\labelsep}
\setlength{\labelwidth}{\tmplength}
}
\begin{flushleft}
\item[\textbf{Declaração}\hfill]
\begin{ttfamily}
public function ExistDataBase:Boolean;\end{ttfamily}


\end{flushleft}
\par
\item[\textbf{Descrição}]
O Método \textbf{\begin{ttfamily}ExistDataBase\end{ttfamily}} retorna \textbf{true} se o banco de dados existe e \textbf{false} se não existir.

\end{list}
\paragraph*{CreateDataBase}\hspace*{\fill}

\begin{list}{}{
\settowidth{\tmplength}{\textbf{Declaração}}
\setlength{\itemindent}{0cm}
\setlength{\listparindent}{0cm}
\setlength{\leftmargin}{\evensidemargin}
\addtolength{\leftmargin}{\tmplength}
\settowidth{\labelsep}{X}
\addtolength{\leftmargin}{\labelsep}
\setlength{\labelwidth}{\tmplength}
}
\begin{flushleft}
\item[\textbf{Declaração}\hfill]
\begin{ttfamily}
public function CreateDataBase:boolean;\end{ttfamily}


\end{flushleft}
\par
\item[\textbf{Descrição}]
O método \textbf{\begin{ttfamily}CreateDataBase\end{ttfamily}} cria o banco de dados se a constante \begin{ttfamily}OkCreateDataBase\end{ttfamily}(\ref{mi_rtl_ui_custom_application.TMI_ui_Custom_Application-OkCreateDataBase}) = true

\end{list}
\paragraph*{Get{\_}ParametersEnter}\hspace*{\fill}

\begin{list}{}{
\settowidth{\tmplength}{\textbf{Declaração}}
\setlength{\itemindent}{0cm}
\setlength{\listparindent}{0cm}
\setlength{\leftmargin}{\evensidemargin}
\addtolength{\leftmargin}{\tmplength}
\settowidth{\labelsep}{X}
\addtolength{\leftmargin}{\labelsep}
\setlength{\labelwidth}{\tmplength}
}
\begin{flushleft}
\item[\textbf{Declaração}\hfill]
\begin{ttfamily}
public procedure Get{\_}ParametersEnter(aDmxScroller: TUiDmxScroller);\end{ttfamily}


\end{flushleft}
\par
\item[\textbf{Descrição}]
O método \textbf{\begin{ttfamily}Get{\_}ParametersEnter\end{ttfamily}} é usado pela classe \textbf{\begin{ttfamily}Get{\_}Parameters\end{ttfamily}(\ref{mi_rtl_ui_custom_application.TMI_ui_Custom_Application-Get_Parameters})}. \begin{itemize}
\item Esse evento cria o arquivo de parâmetros usando os dados das propriedades de \textbf{\begin{ttfamily}TMI{\_}ui{\_}Custom{\_}Application\end{ttfamily}(\ref{mi_rtl_ui_custom_application.TMI_ui_Custom_Application})} definidas no tempo de projeto.
\end{itemize}

\end{list}
\paragraph*{Get{\_}ParametersExit}\hspace*{\fill}

\begin{list}{}{
\settowidth{\tmplength}{\textbf{Declaração}}
\setlength{\itemindent}{0cm}
\setlength{\listparindent}{0cm}
\setlength{\leftmargin}{\evensidemargin}
\addtolength{\leftmargin}{\tmplength}
\settowidth{\labelsep}{X}
\addtolength{\leftmargin}{\labelsep}
\setlength{\labelwidth}{\tmplength}
}
\begin{flushleft}
\item[\textbf{Declaração}\hfill]
\begin{ttfamily}
public procedure Get{\_}ParametersExit(aDmxScroller: TUiDmxScroller);\end{ttfamily}


\end{flushleft}
\end{list}
\paragraph*{Get{\_}ParametersCloseQuery}\hspace*{\fill}

\begin{list}{}{
\settowidth{\tmplength}{\textbf{Declaração}}
\setlength{\itemindent}{0cm}
\setlength{\listparindent}{0cm}
\setlength{\leftmargin}{\evensidemargin}
\addtolength{\leftmargin}{\tmplength}
\settowidth{\labelsep}{X}
\addtolength{\leftmargin}{\labelsep}
\setlength{\labelwidth}{\tmplength}
}
\begin{flushleft}
\item[\textbf{Declaração}\hfill]
\begin{ttfamily}
public Procedure Get{\_}ParametersCloseQuery(aDmxScroller:TUiDmxScroller; var CanClose:boolean);\end{ttfamily}


\end{flushleft}
\par
\item[\textbf{Descrição}]
O método \textbf{\begin{ttfamily}Get{\_}ParametersCloseQuery\end{ttfamily}} é usado para confirmar o fechamento do formulário \begin{ttfamily}Get{\_}Parameters\end{ttfamily}(\ref{mi_rtl_ui_custom_application.TMI_ui_Custom_Application-Get_Parameters}) com botão \textbf{MrOK} caso os campos de \textbf{\begin{ttfamily}Get{\_}Parameters\end{ttfamily}(\ref{mi_rtl_ui_custom_application.TMI_ui_Custom_Application-Get_Parameters})} sejam válidos.

\begin{itemize}
\item \textbf{NOTA} \begin{itemize}
\item Método \textbf{\begin{ttfamily}Get{\_}ParametersCloseQuery\end{ttfamily}} executa o evento \textbf{\begin{ttfamily}DoOnValidUser\end{ttfamily}(\ref{mi_rtl_ui_custom_application.TMI_ui_Custom_Application-DoOnValidUser})}, se o mesmo for assinalado na aplicação com objetivo de não permitir fechar o formulário modal com botão MrOK caso \textbf{ \begin{ttfamily}DoOnValidUser\end{ttfamily}(\ref{mi_rtl_ui_custom_application.TMI_ui_Custom_Application-DoOnValidUser})} retornar false.
\item Pode ser usado para checar se usuário e senha são válidos bem como se os parâmetros estão compatíveis com os bancos de dados instalados.
\end{itemize}
\end{itemize}

\end{list}
\paragraph*{Login{\_}GetTemplate}\hspace*{\fill}

\begin{list}{}{
\settowidth{\tmplength}{\textbf{Declaração}}
\setlength{\itemindent}{0cm}
\setlength{\listparindent}{0cm}
\setlength{\leftmargin}{\evensidemargin}
\addtolength{\leftmargin}{\tmplength}
\settowidth{\labelsep}{X}
\addtolength{\leftmargin}{\labelsep}
\setlength{\labelwidth}{\tmplength}
}
\begin{flushleft}
\item[\textbf{Declaração}\hfill]
\begin{ttfamily}
public Function Login{\_}GetTemplate( aNext : PSItem ) : PSItem;\end{ttfamily}


\end{flushleft}
\par
\item[\textbf{Descrição}]
O método \textbf{\begin{ttfamily}Login{\_}GetTemplate\end{ttfamily}} retorna um Template usado para criar o formulário de entrada de dados para a conexão.

\end{list}
\paragraph*{NameDataBase}\hspace*{\fill}

\begin{list}{}{
\settowidth{\tmplength}{\textbf{Declaração}}
\setlength{\itemindent}{0cm}
\setlength{\listparindent}{0cm}
\setlength{\leftmargin}{\evensidemargin}
\addtolength{\leftmargin}{\tmplength}
\settowidth{\labelsep}{X}
\addtolength{\leftmargin}{\labelsep}
\setlength{\labelwidth}{\tmplength}
}
\begin{flushleft}
\item[\textbf{Declaração}\hfill]
\begin{ttfamily}
published Function NameDataBase:AnsiString;\end{ttfamily}


\end{flushleft}
\par
\item[\textbf{Descrição}]
O método \textbf{\begin{ttfamily}NameDataBase\end{ttfamily}} retorna o nome do banco de dados de acordo com o tipo de banco de dados.

\end{list}
\paragraph*{DoOnValidUser}\hspace*{\fill}

\begin{list}{}{
\settowidth{\tmplength}{\textbf{Declaração}}
\setlength{\itemindent}{0cm}
\setlength{\listparindent}{0cm}
\setlength{\leftmargin}{\evensidemargin}
\addtolength{\leftmargin}{\tmplength}
\settowidth{\labelsep}{X}
\addtolength{\leftmargin}{\labelsep}
\setlength{\labelwidth}{\tmplength}
}
\begin{flushleft}
\item[\textbf{Declaração}\hfill]
\begin{ttfamily}
public function DoOnValidUser(aDmxScroller:TUiDmxScroller;aUserName:AnsiString;aPassword:AnsiString):boolean; virtual;\end{ttfamily}


\end{flushleft}
\par
\item[\textbf{Descrição}]
O método \textbf{\begin{ttfamily}DoOnValidUser\end{ttfamily}} executa o evento \textbf{\begin{ttfamily}onValidUser\end{ttfamily}(\ref{mi_rtl_ui_custom_application.TMI_ui_Custom_Application-onValidUser})} se o mesmo for assinalado na aplicação ou retorna true se \textbf{\begin{ttfamily}onValidUser\end{ttfamily}(\ref{mi_rtl_ui_custom_application.TMI_ui_Custom_Application-onValidUser}) = nil}

\end{list}
\paragraph*{Create{\_}Get{\_}Parameters}\hspace*{\fill}

\begin{list}{}{
\settowidth{\tmplength}{\textbf{Declaração}}
\setlength{\itemindent}{0cm}
\setlength{\listparindent}{0cm}
\setlength{\leftmargin}{\evensidemargin}
\addtolength{\leftmargin}{\tmplength}
\settowidth{\labelsep}{X}
\addtolength{\leftmargin}{\labelsep}
\setlength{\labelwidth}{\tmplength}
}
\begin{flushleft}
\item[\textbf{Declaração}\hfill]
\begin{ttfamily}
protected procedure Create{\_}Get{\_}Parameters; virtual; Abstract;\end{ttfamily}


\end{flushleft}
\par
\item[\textbf{Descrição}]
O método \textbf{\begin{ttfamily}Create{\_}Get{\_}Parameters\end{ttfamily}} deve ser implementado para criar classe TUiDmxScroller{\_}form{\_}lcl ou TUiDmxScroller{\_}form{\_}HTML{\_}Angular4.

\end{list}
\paragraph*{Create}\hspace*{\fill}

\begin{list}{}{
\settowidth{\tmplength}{\textbf{Declaração}}
\setlength{\itemindent}{0cm}
\setlength{\listparindent}{0cm}
\setlength{\leftmargin}{\evensidemargin}
\addtolength{\leftmargin}{\tmplength}
\settowidth{\labelsep}{X}
\addtolength{\leftmargin}{\labelsep}
\setlength{\labelwidth}{\tmplength}
}
\begin{flushleft}
\item[\textbf{Declaração}\hfill]
\begin{ttfamily}
public constructor Create(AOwner: TComponent); override;\end{ttfamily}


\end{flushleft}
\par
\item[\textbf{Descrição}]
O constructor \textbf{\begin{ttfamily}Create\end{ttfamily}} cria os componentes \textbf{\begin{ttfamily}SQLConnector\end{ttfamily}(\ref{mi_rtl_ui_custom_application.TMI_ui_Custom_Application-SQLConnector})}, \textbf{\begin{ttfamily}SQLTransaction\end{ttfamily}(\ref{mi_rtl_ui_custom_application.TMI_ui_Custom_Application-SQLTransaction})}, \textbf{\begin{ttfamily}BufDataSet1\end{ttfamily}(\ref{mi_rtl_ui_custom_application.TMI_ui_Custom_Application-BufDataSet1})}, \textbf{\begin{ttfamily}DataSource1\end{ttfamily}(\ref{mi_rtl_ui_custom_application.TMI_ui_Custom_Application-DataSource1})}, Inicia a constante \textbf{\begin{ttfamily}FileName{\_}Parameters\end{ttfamily}(\ref{mi_rtl_ui_custom_application.TMI_ui_Custom_Application-FileName_Parameters})}, executa o método \begin{ttfamily}Create{\_}Get{\_}Parameters\end{ttfamily}(\ref{mi_rtl_ui_custom_application.TMI_ui_Custom_Application-Create_Get_Parameters}), inicializa \textbf{\begin{ttfamily}charSet\end{ttfamily}(\ref{mi_rtl_ui_custom_application.TMI_ui_Custom_Application-CharSet})} e liga os componentes \begin{ttfamily}SQLConnector\end{ttfamily}(\ref{mi_rtl_ui_custom_application.TMI_ui_Custom_Application-SQLConnector}) com \begin{ttfamily}SQLTransaction\end{ttfamily}(\ref{mi_rtl_ui_custom_application.TMI_ui_Custom_Application-SQLTransaction}) e os componentes \textbf{DataSource1.DataSet := \begin{ttfamily}BufDataset1\end{ttfamily}(\ref{mi_rtl_ui_custom_application.TMI_ui_Custom_Application-BufDataSet1})}.

\end{list}
\paragraph*{Destroy}\hspace*{\fill}

\begin{list}{}{
\settowidth{\tmplength}{\textbf{Declaração}}
\setlength{\itemindent}{0cm}
\setlength{\listparindent}{0cm}
\setlength{\leftmargin}{\evensidemargin}
\addtolength{\leftmargin}{\tmplength}
\settowidth{\labelsep}{X}
\addtolength{\leftmargin}{\labelsep}
\setlength{\labelwidth}{\tmplength}
}
\begin{flushleft}
\item[\textbf{Declaração}\hfill]
\begin{ttfamily}
public destructor Destroy; override;\end{ttfamily}


\end{flushleft}
\par
\item[\textbf{Descrição}]
O destructor \textbf{\begin{ttfamily}Destroy\end{ttfamily}} destrói as classes criadas pelo constructor da classe

\end{list}
\section{Funções e Procedimentos}
\subsection*{Mi{\_}ui{\_}Custom{\_}Application}
\begin{list}{}{
\settowidth{\tmplength}{\textbf{Declaração}}
\setlength{\itemindent}{0cm}
\setlength{\listparindent}{0cm}
\setlength{\leftmargin}{\evensidemargin}
\addtolength{\leftmargin}{\tmplength}
\settowidth{\labelsep}{X}
\addtolength{\leftmargin}{\labelsep}
\setlength{\labelwidth}{\tmplength}
}
\begin{flushleft}
\item[\textbf{Declaração}\hfill]
\begin{ttfamily}
function Mi{\_}ui{\_}Custom{\_}Application: TMI{\_}ui{\_}Custom{\_}Application;\end{ttfamily}


\end{flushleft}
\par
\item[\textbf{Descrição}]
A função \textbf{\begin{ttfamily}Mi{\_}ui{\_}Custom{\_}Application\end{ttfamily}} retorna a ultima instância de \textbf{\begin{ttfamily}TMI{\_}ui{\_}Custom{\_}Application\end{ttfamily}(\ref{mi_rtl_ui_custom_application.TMI_ui_Custom_Application})} criada no sistema

\end{list}
\subsection*{Set{\_}Mi{\_}ui{\_}Custom{\_}Application}
\begin{list}{}{
\settowidth{\tmplength}{\textbf{Declaração}}
\setlength{\itemindent}{0cm}
\setlength{\listparindent}{0cm}
\setlength{\leftmargin}{\evensidemargin}
\addtolength{\leftmargin}{\tmplength}
\settowidth{\labelsep}{X}
\addtolength{\leftmargin}{\labelsep}
\setlength{\labelwidth}{\tmplength}
}
\begin{flushleft}
\item[\textbf{Declaração}\hfill]
\begin{ttfamily}
Function Set{\_}Mi{\_}ui{\_}Custom{\_}Application(aMi{\_}ui{\_}Custom{\_}Application : TMI{\_}ui{\_}Custom{\_}Application): TMI{\_}ui{\_}Custom{\_}Application;\end{ttfamily}


\end{flushleft}
\par
\item[\textbf{Descrição}]
A função \textbf{\begin{ttfamily}Set{\_}Mi{\_}ui{\_}Custom{\_}Application\end{ttfamily}} seta a ultima instância de \textbf{\begin{ttfamily}TMI{\_}ui{\_}Custom{\_}Application\end{ttfamily}(\ref{mi_rtl_ui_custom_application.TMI_ui_Custom_Application})} criada no sistema e retorna aplicação selecionada anteriormente.

\end{list}
\section{Tipos}
\subsection*{TOnValidUser}
\begin{list}{}{
\settowidth{\tmplength}{\textbf{Declaração}}
\setlength{\itemindent}{0cm}
\setlength{\listparindent}{0cm}
\setlength{\leftmargin}{\evensidemargin}
\addtolength{\leftmargin}{\tmplength}
\settowidth{\labelsep}{X}
\addtolength{\leftmargin}{\labelsep}
\setlength{\labelwidth}{\tmplength}
}
\begin{flushleft}
\item[\textbf{Declaração}\hfill]
\begin{ttfamily}
TOnValidUser = function (aDmxScroller:TUiDmxScroller;aUserName:AnsiString;aPassword:AnsiString):boolean of Object;\end{ttfamily}


\end{flushleft}
\par
\item[\textbf{Descrição}]
O tipo \textbf{\begin{ttfamily}TOnValidUser\end{ttfamily}} é usado no evento OnValidUser

\end{list}
\chapter{Unit mi{\_}rtl{\_}ui{\_}Dmxscroller}
\section{Descrição}
A unit \textbf{\begin{ttfamily}mi{\_}rtl{\_}ui{\_}Dmxscroller\end{ttfamily}} implementa a classe \begin{ttfamily}TUiDmxScroller\end{ttfamily}(\ref{mi_rtl_ui_Dmxscroller.TUiDmxScroller}) e registro \begin{ttfamily}TDmxFieldRec\end{ttfamily}(\ref{mi_rtl_ui_Dmxscroller.TDmxFieldRec}).

\begin{itemize}
\item \textbf{VERSÃO} \begin{itemize}
\item Alpha {-} 0.5.0.687
\end{itemize}
\item \textbf{HISTÓRICO} \begin{itemize}
\item 
\end{itemize}
\item \textbf{CÓDIGO FONTE}: \begin{itemize}
\item 
\item \textbf{PENDÊNCIAS} \begin{itemize}
\item T12 Quando uma linha em um label tem muitos caracteres de 2 bytes os últimos não são interpretados.
\item T12 Implementar o campo FldLink. (Esse campo executa um ação usando controle TStaticText.
\item T12 O controle TComboBox da LCL alterar o tamanho da fonte courie New caso o tema do sistema mude. \begin{itemize}
\item Pesquisar sobre o assunto.
\end{itemize}
\item T12 No método SetString em caso de erro de gera exceção informando valor máximo do campo e não o valor digitado.
\item T12 Implementar o evento OnChange em todos os controles, visto que o mesmo é mais \begin{ttfamily}f\end{ttfamily}(\ref{testForm-f})ácil criar lógica de negócios visto que o mesmo só é executado se o campo for modificado.
\item T12 Implementar a possibilidade das fontes do label ser personalizada baseado em um estilo que pode ser uma variável global.

\begin{itemize}
\item Suponha que {\^{}}Z = {$<$}h1{$>$} Título e {\^{}}D = {$<$}B{$>$} de negrito então o sistema informa a TDmxFieldRec.Style = nome do estilo onde nome do estilo = 'Font = FonteX; Size= XX; etc.. '

\begin{itemize}
\item Exemplo:
\end{itemize} ~{\^{}}ZCADASTRO DE ALUNOS~

~{\^{}}DÑome do Aluno:~{\textbackslash}ssssssssss
\end{itemize}
\end{itemize}\begin{itemize}
\item T12 Na construção do formulário LCL setar o campo PDmxFieldRec.LinkEdit;
\item T12 Implementar o método: function FieldByNum(aFieldnum:Integer):\begin{ttfamily}PDmxFieldRec\end{ttfamily}(\ref{mi_rtl_ui_Dmxscroller-pDmxFieldRec});
\item T12 Implementar a edição \textbf{\begin{ttfamily}FldBoolean\end{ttfamily}(\ref{mi_rtl_ui_dmxscroller_form-fldBoolean})}. \begin{itemize}
\item Os campo Boolean deve ser editados como uma campo enumerado onde: \begin{itemize}
\item 0 {-} False; não
\item 1 = True; sim
\end{itemize}
\end{itemize}
\item T12 O campo \begin{ttfamily}fld{\_}LHora\end{ttfamily}(\ref{mi_rtl_ui_dmxscroller_form-fld_LHora}) não inicializado antes de compactar a hora.
\item T12 Quando o usuário teclar tab para passar o campo e o campo seguinte não estiver visível o sistema deve passar a página do controle parent.
\item T12 Implementar a edição de campo \textbf{\begin{ttfamily}FldMemo\end{ttfamily}(\ref{mi_rtl_ui_dmxscroller_form-FldMemo})}.
\item T!2 Implementar a campo \textbf{\begin{ttfamily}fldBLOb\end{ttfamily}(\ref{mi_rtl_ui_dmxscroller_form-fldBLOb})};
\item t12 Implementar a edição de \textbf{\begin{ttfamily}fldHexValue\end{ttfamily}(\ref{mi_rtl_ui_dmxscroller_form-fldHexValue})}. \begin{itemize}
\item O campo Hexadecimal deve ser campo longint mais a edição é uma string comum . \begin{ttfamily}FldStr\end{ttfamily}(\ref{mi_rtl_ui_dmxscroller_form-fldSTR})
\end{itemize}
\item T12 Implementar a propriedade AlignmentLabels := taCenter; AlignmentLabels := taLeftJustify; AlignmentLabels := taRightJustify ;
\item T12 Implementar a execução do evento do tipo CharExecProc quando a tecla F7 é pressionada.
\item T12 Criar opção para gerar cliente HTML a partir de TDmxScroller \begin{itemize}
\item Referência: [Componente que espoe dados para o browser](https://wiki.freepascal.org/SqlDbRestBridge{\#}Purpose)
\end{itemize}
\item T12 O grupo \begin{ttfamily}TMi{\_}RadioGroup{\_}Lcl\end{ttfamily}(\ref{umi_ui_radiogroup_lcl.TMI_RadioGroup_LCL}) não é selecionado com a tecla na tecla \textbf{TAB} \begin{itemize}
\item Quando os botões TRadioButton estão dentro do TRadioGroup a propriedade TRadioGroup.TabStop não funciona.
\end{itemize}
\item T12 Nosso código só é executado com o editor de propriedade. Se não estamos no editor de propriedade então não temos controle do código no modo design. Qual o meu problema: O formulário deve ser criado em tempo de execução, porém eu queria ver como ele estava ficando sem precisar compilar e executar o código, por isso coloquei o código em um stringList e ao ativar o objeto, o formulário é criado. Porém esses objetos criados no designer não podem ficar no arquivo de recursos porque quando for executado vai haver duplicidade. \begin{itemize}
\item Quando eu desativo o objetos todos os objetos que ele criou são excluídos do arquivo de recursos.
\item Isso eu já faço agora, mais quando distribuir o componente as pessoas vão deixar esses componente usado no teste e ao executar vai haver error.
\item Por isso eu queria que caso a propriedade active tivesse em true eu queria que ela ficasse em false.
\end{itemize}
\end{itemize}
\end{itemize}
\item \textbf{CONCLUÍDO} \begin{itemize}
\item T12 O campo FldCheckBox não está funcionando o flag \begin{ttfamily}charHint\end{ttfamily}(\ref{mi_rtl_ui_dmxscroller_form-CharHint}) ✅️.
\item T12 Implementar o controle ChatHint no Template para seja possível passar um documento markdown pelo Template; ✅️.
\item T12 Ao executar o evento OnExit é necessário o redraw em de todos os campo caso haja alteração ao retorna da chamada. ✅️.
\item T12 O componente TMi{\_}ui{\_}Button{\_}lcl não está na lista dos campos selecionados na tecla tab. ✅
\item T12 Os campos \begin{ttfamily}FldEnum\end{ttfamily}(\ref{mi_rtl_ui_dmxscroller_form-fldENUM}) não estão mostrando o help. ✅
\item T12 Criar a propriedade Locked; ✅
\item T12 No pacote mi.rtl.ui, transferir toda dependência do pacote LCL para o pacote mi.rtl.form.
\end{itemize}
\end{itemize}
\section{Uses}
\begin{itemize}
\item \begin{ttfamily}Classes\end{ttfamily}\item \begin{ttfamily}SysUtils\end{ttfamily}\item \begin{ttfamily}db\end{ttfamily}\item \begin{ttfamily}BufDataset\end{ttfamily}\item \begin{ttfamily}SqlDb\end{ttfamily}\item \begin{ttfamily}mi.rtl.Objects.Consts.Mi{\_}MsgBox\end{ttfamily}\item \begin{ttfamily}mi.rtl.objects.Methods.dates\end{ttfamily}(\ref{mi.rtl.objects.Methods.dates})\item \begin{ttfamily}mi{\_}rtl{\_}ui{\_}Types\end{ttfamily}(\ref{mi_rtl_ui_types})\item \begin{ttfamily}mi.rtl.Consts\end{ttfamily}(\ref{mi.rtl.Consts})\item \begin{ttfamily}mi{\_}rtl{\_}ui{\_}methods\end{ttfamily}(\ref{mi_rtl_ui_methods})\end{itemize}
\section{Visão Geral}
\begin{description}
\item[\texttt{\begin{ttfamily}TFldEnum{\_}Lookup\end{ttfamily} Classe}]
\item[\texttt{\begin{ttfamily}TDmxFieldRec\end{ttfamily} Registro}]
\item[\texttt{\begin{ttfamily}TUiDmxScroller\end{ttfamily} Classe}]
\end{description}
\section{Classes, Interfaces, Objetos e Registros}
\subsection*{TFldEnum{\_}Lookup Classe}
\subsubsection*{\large{\textbf{Hierarquia}}\normalsize\hspace{1ex}\hfill}
TFldEnum{\_}Lookup {$>$} TComponent
\subsubsection*{\large{\textbf{Descrição}}\normalsize\hspace{1ex}\hfill}
A classe \textbf{\begin{ttfamily}TFldEnum{\_}Lookup\end{ttfamily}} é usada para implementar campo ComboBox quando TDmxScroller estiver TDataSource {$<$}{$>$} nil porque o Lazarus espera em campos ComboBox um string e não o índice da lista de strings.\subsubsection*{\large{\textbf{Propriedades}}\normalsize\hspace{1ex}\hfill}
\paragraph*{DmxFieldRec}\hspace*{\fill}

\begin{list}{}{
\settowidth{\tmplength}{\textbf{Declaração}}
\setlength{\itemindent}{0cm}
\setlength{\listparindent}{0cm}
\setlength{\leftmargin}{\evensidemargin}
\addtolength{\leftmargin}{\tmplength}
\settowidth{\labelsep}{X}
\addtolength{\leftmargin}{\labelsep}
\setlength{\labelwidth}{\tmplength}
}
\begin{flushleft}
\item[\textbf{Declaração}\hfill]
\begin{ttfamily}
public property DmxFieldRec : pDmxFieldRec read {\_}DmxFieldRec Write SeTDmxFieldRec;\end{ttfamily}


\end{flushleft}
\par
\item[\textbf{Descrição}]
A propriedade \textbf{\begin{ttfamily}DmxFieldRec\end{ttfamily}} contém o campo comboBox se ser editado

\end{list}
\subsubsection*{\large{\textbf{Campos}}\normalsize\hspace{1ex}\hfill}
\paragraph*{BufDataSet}\hspace*{\fill}

\begin{list}{}{
\settowidth{\tmplength}{\textbf{Declaração}}
\setlength{\itemindent}{0cm}
\setlength{\listparindent}{0cm}
\setlength{\leftmargin}{\evensidemargin}
\addtolength{\leftmargin}{\tmplength}
\settowidth{\labelsep}{X}
\addtolength{\leftmargin}{\labelsep}
\setlength{\labelwidth}{\tmplength}
}
\begin{flushleft}
\item[\textbf{Declaração}\hfill]
\begin{ttfamily}
public BufDataSet: TBufDataSet;\end{ttfamily}


\end{flushleft}
\par
\item[\textbf{Descrição}]
O atributo \textbf{\begin{ttfamily}BufDataSet\end{ttfamily}} contém o arquivo em memória das opções do campo ComboBox sendo editado.

\end{list}
\paragraph*{DataSource}\hspace*{\fill}

\begin{list}{}{
\settowidth{\tmplength}{\textbf{Declaração}}
\setlength{\itemindent}{0cm}
\setlength{\listparindent}{0cm}
\setlength{\leftmargin}{\evensidemargin}
\addtolength{\leftmargin}{\tmplength}
\settowidth{\labelsep}{X}
\addtolength{\leftmargin}{\labelsep}
\setlength{\labelwidth}{\tmplength}
}
\begin{flushleft}
\item[\textbf{Declaração}\hfill]
\begin{ttfamily}
public DataSource: TDataSource;\end{ttfamily}


\end{flushleft}
\par
\item[\textbf{Descrição}]
O atributo \textbf{\begin{ttfamily}DataSource\end{ttfamily}} é a fonte de dados associado a \begin{ttfamily}TFldEnum{\_}Lookup.BufDataSet\end{ttfamily}(\ref{mi_rtl_ui_Dmxscroller.TFldEnum_Lookup-BufDataSet}) do campo sendo editado.

\end{list}
\paragraph*{KeyField}\hspace*{\fill}

\begin{list}{}{
\settowidth{\tmplength}{\textbf{Declaração}}
\setlength{\itemindent}{0cm}
\setlength{\listparindent}{0cm}
\setlength{\leftmargin}{\evensidemargin}
\addtolength{\leftmargin}{\tmplength}
\settowidth{\labelsep}{X}
\addtolength{\leftmargin}{\labelsep}
\setlength{\labelwidth}{\tmplength}
}
\begin{flushleft}
\item[\textbf{Declaração}\hfill]
\begin{ttfamily}
public KeyField: AnsiString;\end{ttfamily}


\end{flushleft}
\par
\item[\textbf{Descrição}]
O atributo \textbf{\begin{ttfamily}KeyField\end{ttfamily}} contém o nome do campo chave da tabela associada.

\end{list}
\paragraph*{ListField}\hspace*{\fill}

\begin{list}{}{
\settowidth{\tmplength}{\textbf{Declaração}}
\setlength{\itemindent}{0cm}
\setlength{\listparindent}{0cm}
\setlength{\leftmargin}{\evensidemargin}
\addtolength{\leftmargin}{\tmplength}
\settowidth{\labelsep}{X}
\addtolength{\leftmargin}{\labelsep}
\setlength{\labelwidth}{\tmplength}
}
\begin{flushleft}
\item[\textbf{Declaração}\hfill]
\begin{ttfamily}
public ListField: AnsiString;\end{ttfamily}


\end{flushleft}
\par
\item[\textbf{Descrição}]
O atributo \textbf{\begin{ttfamily}ListField\end{ttfamily}} contém o nome do campo da tabela associada a ser visualizado.

\end{list}
\subsubsection*{\large{\textbf{Métodos}}\normalsize\hspace{1ex}\hfill}
\paragraph*{create}\hspace*{\fill}

\begin{list}{}{
\settowidth{\tmplength}{\textbf{Declaração}}
\setlength{\itemindent}{0cm}
\setlength{\listparindent}{0cm}
\setlength{\leftmargin}{\evensidemargin}
\addtolength{\leftmargin}{\tmplength}
\settowidth{\labelsep}{X}
\addtolength{\leftmargin}{\labelsep}
\setlength{\labelwidth}{\tmplength}
}
\begin{flushleft}
\item[\textbf{Declaração}\hfill]
\begin{ttfamily}
public constructor create(aDmxFieldRec : pDmxFieldRec); overload;\end{ttfamily}


\end{flushleft}
\par
\item[\textbf{Descrição}]
O constructor \textbf{\begin{ttfamily}create\end{ttfamily}} cria os campos TBufDataSet e TDataSource do campo \begin{ttfamily}TFldEnum{\_}Lookup\end{ttfamily}(\ref{mi_rtl_ui_Dmxscroller.TFldEnum_Lookup})

\end{list}
\paragraph*{destroy}\hspace*{\fill}

\begin{list}{}{
\settowidth{\tmplength}{\textbf{Declaração}}
\setlength{\itemindent}{0cm}
\setlength{\listparindent}{0cm}
\setlength{\leftmargin}{\evensidemargin}
\addtolength{\leftmargin}{\tmplength}
\settowidth{\labelsep}{X}
\addtolength{\leftmargin}{\labelsep}
\setlength{\labelwidth}{\tmplength}
}
\begin{flushleft}
\item[\textbf{Declaração}\hfill]
\begin{ttfamily}
public destructor destroy; override;\end{ttfamily}


\end{flushleft}
\par
\item[\textbf{Descrição}]
O destructor \textbf{\begin{ttfamily}destroy\end{ttfamily}} destrói os campos TBufDataSet e TDataSource do campo \begin{ttfamily}TFldEnum{\_}Lookup\end{ttfamily}(\ref{mi_rtl_ui_Dmxscroller.TFldEnum_Lookup})

\end{list}
\subsection*{TDmxFieldRec Registro}
\subsubsection*{\large{\textbf{Descrição}}\normalsize\hspace{1ex}\hfill}
O registro \textbf{\begin{ttfamily}TDmxFieldRec\end{ttfamily}} é usado para guardar as informações passadas pelos Templates das strings.

\begin{itemize}
\item \textbf{REFERÊNCIA} \begin{itemize}
\item [Estrutura record e object]https://wiki.freepascal.org/Record
\end{itemize}
\item A aparência padrão dessas visualizações geralmente é orientada por coluna/linha, com exceção de exibições do tipo formulário e campos únicos.
\item Você declara uma estrutura de registro para o procedimento de inicialização do \textbf{tvDMX} em um modelo string – que também determina o formato de exibição. (Você verá mais tarde como o \textbf{tvDMX} pode ser usado para trabalhar com formulários ou editores de campo.)
\item \textbf{EXEMPLO}

\begin{itemize}
\item O \begin{ttfamily}Template\end{ttfamily}(\ref{mi_rtl_ui_Dmxscroller.TDmxFieldRec-Template}) '{\textbackslash} sssssssss`sssssssssss {\textbackslash} iiii {\textbackslash} rrr.rr' representa o registro: \begin{itemize}
\item \textbf{CÓDIGO PASCAL}

\texttt{\\\nopagebreak[3]
\\\nopagebreak[3]
}\textbf{type}\texttt{\\\nopagebreak[3]
\\\nopagebreak[3]
~~TRecord~=~}\textbf{Record}\texttt{\\\nopagebreak[3]
~~~~~~~~~~~~~~nome~:~}\textbf{String}\texttt{~[20];\\\nopagebreak[3]
~~~~~~~~~~~~~~Ano~~:~Integer;\\\nopagebreak[3]
~~~~~~~~~~~~~~Valor~:~Real;\\\nopagebreak[3]
~~~~~~~~~~~~}\textbf{end}\texttt{;\\
}
\item \textbf{NOTA:} \begin{itemize}
\item A letra ( \textbf{s} ) minúsculo aceita qualquer número e letras maiúsculas e minúsculas;
\item A letra ( \textbf{i} ) representa um número inteiro com 2 bytes com edição em 4 posições (0 a 9999);
\item A letra ( \textbf{r} ) representa um número real com 8 bytes com edição em 5 posições (0 a 999.99)
\item O símbolo ( \textbf{`} ) crase é usado para informar que a parte do texto depois deste sinal deve ser omitida da visão.
\item A símbolo ( \textbf{' {\textbackslash} '} ) barra invertida deve ser usada como delimitador de campo e é exibida como um espaço em branco.
\item O símbolo ( \textbf{~} ) til deve ser usado para separar rótulos dos campos de dados.
\end{itemize}
\end{itemize}
\end{itemize}
\item \textbf{ATENÇÃO} \begin{itemize}
\item O registro \textbf{\begin{ttfamily}TDmxFieldRec\end{ttfamily}} não pode ser \textbf{class} e nem conter \textbf{métodos virtuais}, porque este registro e alocado com as funções \textbf{new} e \textbf{dispose}.
\end{itemize}
\end{itemize}\subsubsection*{\large{\textbf{Propriedades}}\normalsize\hspace{1ex}\hfill}
\paragraph*{FieldName}\hspace*{\fill}

\begin{list}{}{
\settowidth{\tmplength}{\textbf{Declaração}}
\setlength{\itemindent}{0cm}
\setlength{\listparindent}{0cm}
\setlength{\leftmargin}{\evensidemargin}
\addtolength{\leftmargin}{\tmplength}
\settowidth{\labelsep}{X}
\addtolength{\leftmargin}{\labelsep}
\setlength{\labelwidth}{\tmplength}
}
\begin{flushleft}
\item[\textbf{Declaração}\hfill]
\begin{ttfamily}
public property FieldName : AnsiString  read {\_}FieldName write SetFieldName;\end{ttfamily}


\end{flushleft}
\par
\item[\textbf{Descrição}]
O campo \textbf{\begin{ttfamily}FieldName\end{ttfamily}} guarda o nome do campo e deve ser inicializado em CreateStruct

\end{list}
\paragraph*{ID{\_}Dynamic}\hspace*{\fill}

\begin{list}{}{
\settowidth{\tmplength}{\textbf{Declaração}}
\setlength{\itemindent}{0cm}
\setlength{\listparindent}{0cm}
\setlength{\leftmargin}{\evensidemargin}
\addtolength{\leftmargin}{\tmplength}
\settowidth{\labelsep}{X}
\addtolength{\leftmargin}{\labelsep}
\setlength{\labelwidth}{\tmplength}
}
\begin{flushleft}
\item[\textbf{Declaração}\hfill]
\begin{ttfamily}
public property ID{\_}Dynamic : AnsiString Read {\_}ID{\_}Dynamic Write {\_}ID{\_}Dynamic;\end{ttfamily}


\end{flushleft}
\end{list}
\paragraph*{owner}\hspace*{\fill}

\begin{list}{}{
\settowidth{\tmplength}{\textbf{Declaração}}
\setlength{\itemindent}{0cm}
\setlength{\listparindent}{0cm}
\setlength{\leftmargin}{\evensidemargin}
\addtolength{\leftmargin}{\tmplength}
\settowidth{\labelsep}{X}
\addtolength{\leftmargin}{\labelsep}
\setlength{\labelwidth}{\tmplength}
}
\begin{flushleft}
\item[\textbf{Declaração}\hfill]
\begin{ttfamily}
public property owner : TUiDmxScroller read {\_}owner write Set{\_}owner;\end{ttfamily}


\end{flushleft}
\end{list}
\paragraph*{FieldAltered}\hspace*{\fill}

\begin{list}{}{
\settowidth{\tmplength}{\textbf{Declaração}}
\setlength{\itemindent}{0cm}
\setlength{\listparindent}{0cm}
\setlength{\leftmargin}{\evensidemargin}
\addtolength{\leftmargin}{\tmplength}
\settowidth{\labelsep}{X}
\addtolength{\leftmargin}{\labelsep}
\setlength{\labelwidth}{\tmplength}
}
\begin{flushleft}
\item[\textbf{Declaração}\hfill]
\begin{ttfamily}
public property FieldAltered : Boolean read GetFieldAltered write {\_}FieldAltered;\end{ttfamily}


\end{flushleft}
\par
\item[\textbf{Descrição}]
A propriedade \textbf{\begin{ttfamily}FieldAltered\end{ttfamily}} Indica que o campo foi alterado. Deve ser atualizado na visão caso a tabela esteja em modo de edição.

\end{list}
\paragraph*{OkSpc}\hspace*{\fill}

\begin{list}{}{
\settowidth{\tmplength}{\textbf{Declaração}}
\setlength{\itemindent}{0cm}
\setlength{\listparindent}{0cm}
\setlength{\leftmargin}{\evensidemargin}
\addtolength{\leftmargin}{\tmplength}
\settowidth{\labelsep}{X}
\addtolength{\leftmargin}{\labelsep}
\setlength{\labelwidth}{\tmplength}
}
\begin{flushleft}
\item[\textbf{Declaração}\hfill]
\begin{ttfamily}
public property OkSpc : Boolean read {\_}OkSpc write SetOkSpc;\end{ttfamily}


\end{flushleft}
\end{list}
\paragraph*{OkMask}\hspace*{\fill}

\begin{list}{}{
\settowidth{\tmplength}{\textbf{Declaração}}
\setlength{\itemindent}{0cm}
\setlength{\listparindent}{0cm}
\setlength{\leftmargin}{\evensidemargin}
\addtolength{\leftmargin}{\tmplength}
\settowidth{\labelsep}{X}
\addtolength{\leftmargin}{\labelsep}
\setlength{\labelwidth}{\tmplength}
}
\begin{flushleft}
\item[\textbf{Declaração}\hfill]
\begin{ttfamily}
public property OkMask : Boolean read {\_}OkMask write {\_}okMask;\end{ttfamily}


\end{flushleft}
\par
\item[\textbf{Descrição}]
O método \textbf{\begin{ttfamily}OkMask\end{ttfamily}} é usado para habilitar ou não em GetString a mascara em campos numéricos.

\end{list}
\paragraph*{AsString}\hspace*{\fill}

\begin{list}{}{
\settowidth{\tmplength}{\textbf{Declaração}}
\setlength{\itemindent}{0cm}
\setlength{\listparindent}{0cm}
\setlength{\leftmargin}{\evensidemargin}
\addtolength{\leftmargin}{\tmplength}
\settowidth{\labelsep}{X}
\addtolength{\leftmargin}{\labelsep}
\setlength{\labelwidth}{\tmplength}
}
\begin{flushleft}
\item[\textbf{Declaração}\hfill]
\begin{ttfamily}
public property AsString : AnsiString read GetAsString write SetAsString;\end{ttfamily}


\end{flushleft}
\end{list}
\paragraph*{Value}\hspace*{\fill}

\begin{list}{}{
\settowidth{\tmplength}{\textbf{Declaração}}
\setlength{\itemindent}{0cm}
\setlength{\listparindent}{0cm}
\setlength{\leftmargin}{\evensidemargin}
\addtolength{\leftmargin}{\tmplength}
\settowidth{\labelsep}{X}
\addtolength{\leftmargin}{\labelsep}
\setlength{\labelwidth}{\tmplength}
}
\begin{flushleft}
\item[\textbf{Declaração}\hfill]
\begin{ttfamily}
public property Value : Variant Read GetValue write SetValue;\end{ttfamily}


\end{flushleft}
\end{list}
\paragraph*{FldOrigin{\_}Y}\hspace*{\fill}

\begin{list}{}{
\settowidth{\tmplength}{\textbf{Declaração}}
\setlength{\itemindent}{0cm}
\setlength{\listparindent}{0cm}
\setlength{\leftmargin}{\evensidemargin}
\addtolength{\leftmargin}{\tmplength}
\settowidth{\labelsep}{X}
\addtolength{\leftmargin}{\labelsep}
\setlength{\labelwidth}{\tmplength}
}
\begin{flushleft}
\item[\textbf{Declaração}\hfill]
\begin{ttfamily}
public property FldOrigin{\_}Y: Integer Read GetFldOrigin{\_}Y Write {\_}FldOrigin{\_}Y;\end{ttfamily}


\end{flushleft}
\end{list}
\paragraph*{FldOrigin}\hspace*{\fill}

\begin{list}{}{
\settowidth{\tmplength}{\textbf{Declaração}}
\setlength{\itemindent}{0cm}
\setlength{\listparindent}{0cm}
\setlength{\leftmargin}{\evensidemargin}
\addtolength{\leftmargin}{\tmplength}
\settowidth{\labelsep}{X}
\addtolength{\leftmargin}{\labelsep}
\setlength{\labelwidth}{\tmplength}
}
\begin{flushleft}
\item[\textbf{Declaração}\hfill]
\begin{ttfamily}
public property FldOrigin  : TPoint read getFldOrigin;\end{ttfamily}


\end{flushleft}
\end{list}
\paragraph*{vidis{\_}OnEnter}\hspace*{\fill}

\begin{list}{}{
\settowidth{\tmplength}{\textbf{Declaração}}
\setlength{\itemindent}{0cm}
\setlength{\listparindent}{0cm}
\setlength{\leftmargin}{\evensidemargin}
\addtolength{\leftmargin}{\tmplength}
\settowidth{\labelsep}{X}
\addtolength{\leftmargin}{\labelsep}
\setlength{\labelwidth}{\tmplength}
}
\begin{flushleft}
\item[\textbf{Declaração}\hfill]
\begin{ttfamily}
public property vidis{\_}OnEnter: Boolean Read Getvidis{\_}OnEnter   Write  Setvidis{\_}OnEnter;\end{ttfamily}


\end{flushleft}
\par
\item[\textbf{Descrição}]
A propriedade \textbf{\begin{ttfamily}vidis{\_}OnEnter\end{ttfamily}} usado para evitar reentrância do evento \begin{ttfamily}DoOnEnter\end{ttfamily}(\ref{mi_rtl_ui_Dmxscroller.TDmxFieldRec-DoOnEnter})()

\end{list}
\paragraph*{vidis{\_}OnExit}\hspace*{\fill}

\begin{list}{}{
\settowidth{\tmplength}{\textbf{Declaração}}
\setlength{\itemindent}{0cm}
\setlength{\listparindent}{0cm}
\setlength{\leftmargin}{\evensidemargin}
\addtolength{\leftmargin}{\tmplength}
\settowidth{\labelsep}{X}
\addtolength{\leftmargin}{\labelsep}
\setlength{\labelwidth}{\tmplength}
}
\begin{flushleft}
\item[\textbf{Declaração}\hfill]
\begin{ttfamily}
public property vidis{\_}OnExit: Boolean Read Getvidis{\_}OnExit   Write  Setvidis{\_}OnExit;\end{ttfamily}


\end{flushleft}
\par
\item[\textbf{Descrição}]
A propriedade \textbf{\begin{ttfamily}vidis{\_}OnExit\end{ttfamily}} é usado para evitar reentrância do evento \begin{ttfamily}DoOnExit\end{ttfamily}(\ref{mi_rtl_ui_Dmxscroller.TDmxFieldRec-DoOnExit})()

\end{list}
\subsubsection*{\large{\textbf{Campos}}\normalsize\hspace{1ex}\hfill}
\paragraph*{LinkEdit}\hspace*{\fill}

\begin{list}{}{
\settowidth{\tmplength}{\textbf{Declaração}}
\setlength{\itemindent}{0cm}
\setlength{\listparindent}{0cm}
\setlength{\leftmargin}{\evensidemargin}
\addtolength{\leftmargin}{\tmplength}
\settowidth{\labelsep}{X}
\addtolength{\leftmargin}{\labelsep}
\setlength{\labelwidth}{\tmplength}
}
\begin{flushleft}
\item[\textbf{Declaração}\hfill]
\begin{ttfamily}
public LinkEdit:  TComponent;\end{ttfamily}


\end{flushleft}
\par
\item[\textbf{Descrição}]
Componente corrente que está editando esse campo.

\end{list}
\paragraph*{Alias}\hspace*{\fill}

\begin{list}{}{
\settowidth{\tmplength}{\textbf{Declaração}}
\setlength{\itemindent}{0cm}
\setlength{\listparindent}{0cm}
\setlength{\leftmargin}{\evensidemargin}
\addtolength{\leftmargin}{\tmplength}
\settowidth{\labelsep}{X}
\addtolength{\leftmargin}{\labelsep}
\setlength{\labelwidth}{\tmplength}
}
\begin{flushleft}
\item[\textbf{Declaração}\hfill]
\begin{ttfamily}
public Alias: AnsiString;\end{ttfamily}


\end{flushleft}
\par
\item[\textbf{Descrição}]
O campo \textbf{\begin{ttfamily}Alias\end{ttfamily}} é usado para associar label ao corrente campo.

\begin{itemize}
\item \textbf{NOTA} \begin{itemize}
\item Esse campo foi necessário para implementar campos do tipo boolean [X] por que o mesmo sempre vem associado a um rótulos amigável e o controle checkbox precisa dele.
\end{itemize}
\item \textbf{EXEMPLO} \begin{itemize}
\item \begin{ttfamily}Template\end{ttfamily}(\ref{mi_rtl_ui_Dmxscroller.TDmxFieldRec-Template}) de um botão checkbox:

\texttt{\\\nopagebreak[3]
\\\nopagebreak[3]
}\textbf{Resourcestring}\texttt{\\\nopagebreak[3]
~~tmp{\_}Aceita~=~'{\textbackslash}X~Aceita~o~contrato~+ChFN+'Aceita{\_}contrato'+CharHint+'Aceita~os~termos~}\textbf{do}\texttt{~contrato?\\
}
\end{itemize}
\end{itemize}

\end{list}
\paragraph*{Template{\_}org}\hspace*{\fill}

\begin{list}{}{
\settowidth{\tmplength}{\textbf{Declaração}}
\setlength{\itemindent}{0cm}
\setlength{\listparindent}{0cm}
\setlength{\leftmargin}{\evensidemargin}
\addtolength{\leftmargin}{\tmplength}
\settowidth{\labelsep}{X}
\addtolength{\leftmargin}{\labelsep}
\setlength{\labelwidth}{\tmplength}
}
\begin{flushleft}
\item[\textbf{Declaração}\hfill]
\begin{ttfamily}
public Template{\_}org: AnsiString;\end{ttfamily}


\end{flushleft}
\par
\item[\textbf{Descrição}]
O campo \textbf{\begin{ttfamily}Template{\_}org\end{ttfamily}} guarda o modelo original do \begin{ttfamily}Template\end{ttfamily}(\ref{mi_rtl_ui_Dmxscroller.TDmxFieldRec-Template}) e deve ser inicializado em CreateStruct

\end{list}
\paragraph*{Next}\hspace*{\fill}

\begin{list}{}{
\settowidth{\tmplength}{\textbf{Declaração}}
\setlength{\itemindent}{0cm}
\setlength{\listparindent}{0cm}
\setlength{\leftmargin}{\evensidemargin}
\addtolength{\leftmargin}{\tmplength}
\settowidth{\labelsep}{X}
\addtolength{\leftmargin}{\labelsep}
\setlength{\labelwidth}{\tmplength}
}
\begin{flushleft}
\item[\textbf{Declaração}\hfill]
\begin{ttfamily}
public Next:  pDmxFieldRec;\end{ttfamily}


\end{flushleft}
\par
\item[\textbf{Descrição}]
Próximo campo

\end{list}
\paragraph*{RSelf}\hspace*{\fill}

\begin{list}{}{
\settowidth{\tmplength}{\textbf{Declaração}}
\setlength{\itemindent}{0cm}
\setlength{\listparindent}{0cm}
\setlength{\leftmargin}{\evensidemargin}
\addtolength{\leftmargin}{\tmplength}
\settowidth{\labelsep}{X}
\addtolength{\leftmargin}{\labelsep}
\setlength{\labelwidth}{\tmplength}
}
\begin{flushleft}
\item[\textbf{Declaração}\hfill]
\begin{ttfamily}
public RSelf:  pDmxFieldRec;\end{ttfamily}


\end{flushleft}
\par
\item[\textbf{Descrição}]
Usado para referenciar{-}se a si mesmo.

\end{list}
\paragraph*{Prev}\hspace*{\fill}

\begin{list}{}{
\settowidth{\tmplength}{\textbf{Declaração}}
\setlength{\itemindent}{0cm}
\setlength{\listparindent}{0cm}
\setlength{\leftmargin}{\evensidemargin}
\addtolength{\leftmargin}{\tmplength}
\settowidth{\labelsep}{X}
\addtolength{\leftmargin}{\labelsep}
\setlength{\labelwidth}{\tmplength}
}
\begin{flushleft}
\item[\textbf{Declaração}\hfill]
\begin{ttfamily}
public Prev:  pDmxFieldRec;\end{ttfamily}


\end{flushleft}
\par
\item[\textbf{Descrição}]
Campo anterior

\end{list}
\paragraph*{access}\hspace*{\fill}

\begin{list}{}{
\settowidth{\tmplength}{\textbf{Declaração}}
\setlength{\itemindent}{0cm}
\setlength{\listparindent}{0cm}
\setlength{\leftmargin}{\evensidemargin}
\addtolength{\leftmargin}{\tmplength}
\settowidth{\labelsep}{X}
\addtolength{\leftmargin}{\labelsep}
\setlength{\labelwidth}{\tmplength}
}
\begin{flushleft}
\item[\textbf{Declaração}\hfill]
\begin{ttfamily}
public access:  byte;\end{ttfamily}


\end{flushleft}
\par
\item[\textbf{Descrição}]
read{-}only, hidden, skip, accSpecX

\end{list}
\paragraph*{Fieldnum}\hspace*{\fill}

\begin{list}{}{
\settowidth{\tmplength}{\textbf{Declaração}}
\setlength{\itemindent}{0cm}
\setlength{\listparindent}{0cm}
\setlength{\leftmargin}{\evensidemargin}
\addtolength{\leftmargin}{\tmplength}
\settowidth{\labelsep}{X}
\addtolength{\leftmargin}{\labelsep}
\setlength{\labelwidth}{\tmplength}
}
\begin{flushleft}
\item[\textbf{Declaração}\hfill]
\begin{ttfamily}
public Fieldnum:  Integer;\end{ttfamily}


\end{flushleft}
\par
\item[\textbf{Descrição}]
Número do campo, varia de 1 a totalFields (Se zero (0) é porque trata{-}se um rótulos)

\end{list}
\paragraph*{ScreenTab}\hspace*{\fill}

\begin{list}{}{
\settowidth{\tmplength}{\textbf{Declaração}}
\setlength{\itemindent}{0cm}
\setlength{\listparindent}{0cm}
\setlength{\leftmargin}{\evensidemargin}
\addtolength{\leftmargin}{\tmplength}
\settowidth{\labelsep}{X}
\addtolength{\leftmargin}{\labelsep}
\setlength{\labelwidth}{\tmplength}
}
\begin{flushleft}
\item[\textbf{Declaração}\hfill]
\begin{ttfamily}
public ScreenTab:  integer;\end{ttfamily}


\end{flushleft}
\par
\item[\textbf{Descrição}]
Override column num.

\end{list}
\paragraph*{ColumnWid}\hspace*{\fill}

\begin{list}{}{
\settowidth{\tmplength}{\textbf{Declaração}}
\setlength{\itemindent}{0cm}
\setlength{\listparindent}{0cm}
\setlength{\leftmargin}{\evensidemargin}
\addtolength{\leftmargin}{\tmplength}
\settowidth{\labelsep}{X}
\addtolength{\leftmargin}{\labelsep}
\setlength{\labelwidth}{\tmplength}
}
\begin{flushleft}
\item[\textbf{Declaração}\hfill]
\begin{ttfamily}
public ColumnWid:  byte;\end{ttfamily}


\end{flushleft}
\par
\item[\textbf{Descrição}]
width of Field column

\end{list}
\paragraph*{ShownWid}\hspace*{\fill}

\begin{list}{}{
\settowidth{\tmplength}{\textbf{Declaração}}
\setlength{\itemindent}{0cm}
\setlength{\listparindent}{0cm}
\setlength{\leftmargin}{\evensidemargin}
\addtolength{\leftmargin}{\tmplength}
\settowidth{\labelsep}{X}
\addtolength{\leftmargin}{\labelsep}
\setlength{\labelwidth}{\tmplength}
}
\begin{flushleft}
\item[\textbf{Declaração}\hfill]
\begin{ttfamily}
public ShownWid:  byte;\end{ttfamily}


\end{flushleft}
\par
\item[\textbf{Descrição}]
visible width of column

\end{list}
\paragraph*{TypeCode}\hspace*{\fill}

\begin{list}{}{
\settowidth{\tmplength}{\textbf{Declaração}}
\setlength{\itemindent}{0cm}
\setlength{\listparindent}{0cm}
\setlength{\leftmargin}{\evensidemargin}
\addtolength{\leftmargin}{\tmplength}
\settowidth{\labelsep}{X}
\addtolength{\leftmargin}{\labelsep}
\setlength{\labelwidth}{\tmplength}
}
\begin{flushleft}
\item[\textbf{Declaração}\hfill]
\begin{ttfamily}
public TypeCode:  AnsiChar;\end{ttfamily}


\end{flushleft}
\par
\item[\textbf{Descrição}]
's', 'r', etc.

\end{list}
\paragraph*{FldEnum{\_}Lookup}\hspace*{\fill}

\begin{list}{}{
\settowidth{\tmplength}{\textbf{Declaração}}
\setlength{\itemindent}{0cm}
\setlength{\listparindent}{0cm}
\setlength{\leftmargin}{\evensidemargin}
\addtolength{\leftmargin}{\tmplength}
\settowidth{\labelsep}{X}
\addtolength{\leftmargin}{\labelsep}
\setlength{\labelwidth}{\tmplength}
}
\begin{flushleft}
\item[\textbf{Declaração}\hfill]
\begin{ttfamily}
public FldEnum{\_}Lookup:TFldEnum{\_}Lookup;\end{ttfamily}


\end{flushleft}
\end{list}
\paragraph*{FillValue}\hspace*{\fill}

\begin{list}{}{
\settowidth{\tmplength}{\textbf{Declaração}}
\setlength{\itemindent}{0cm}
\setlength{\listparindent}{0cm}
\setlength{\leftmargin}{\evensidemargin}
\addtolength{\leftmargin}{\tmplength}
\settowidth{\labelsep}{X}
\addtolength{\leftmargin}{\labelsep}
\setlength{\labelwidth}{\tmplength}
}
\begin{flushleft}
\item[\textbf{Declaração}\hfill]
\begin{ttfamily}
public FillValue:  AnsiChar;\end{ttfamily}


\end{flushleft}
\par
\item[\textbf{Descrição}]
If the Field is numeric, fill in with '{\#}0' if it's alphanumeric, fill in with ' '

\end{list}
\paragraph*{UpperLimit}\hspace*{\fill}

\begin{list}{}{
\settowidth{\tmplength}{\textbf{Declaração}}
\setlength{\itemindent}{0cm}
\setlength{\listparindent}{0cm}
\setlength{\leftmargin}{\evensidemargin}
\addtolength{\leftmargin}{\tmplength}
\settowidth{\labelsep}{X}
\addtolength{\leftmargin}{\labelsep}
\setlength{\labelwidth}{\tmplength}
}
\begin{flushleft}
\item[\textbf{Declaração}\hfill]
\begin{ttfamily}
public UpperLimit:  byte;\end{ttfamily}


\end{flushleft}
\par
\item[\textbf{Descrição}]
maximum \begin{ttfamily}value\end{ttfamily}(\ref{mi_rtl_ui_Dmxscroller.TDmxFieldRec-Value}) limit

\end{list}
\paragraph*{ShowZeroes}\hspace*{\fill}

\begin{list}{}{
\settowidth{\tmplength}{\textbf{Declaração}}
\setlength{\itemindent}{0cm}
\setlength{\listparindent}{0cm}
\setlength{\leftmargin}{\evensidemargin}
\addtolength{\leftmargin}{\tmplength}
\settowidth{\labelsep}{X}
\addtolength{\leftmargin}{\labelsep}
\setlength{\labelwidth}{\tmplength}
}
\begin{flushleft}
\item[\textbf{Declaração}\hfill]
\begin{ttfamily}
public ShowZeroes:  boolean;\end{ttfamily}


\end{flushleft}
\par
\item[\textbf{Descrição}]
display zero values

\end{list}
\paragraph*{TrueLen}\hspace*{\fill}

\begin{list}{}{
\settowidth{\tmplength}{\textbf{Declaração}}
\setlength{\itemindent}{0cm}
\setlength{\listparindent}{0cm}
\setlength{\leftmargin}{\evensidemargin}
\addtolength{\leftmargin}{\tmplength}
\settowidth{\labelsep}{X}
\addtolength{\leftmargin}{\labelsep}
\setlength{\labelwidth}{\tmplength}
}
\begin{flushleft}
\item[\textbf{Declaração}\hfill]
\begin{ttfamily}
public TrueLen:  byte;\end{ttfamily}


\end{flushleft}
\par
\item[\textbf{Descrição}]
unformatted text length

\end{list}
\paragraph*{Parenthesis}\hspace*{\fill}

\begin{list}{}{
\settowidth{\tmplength}{\textbf{Declaração}}
\setlength{\itemindent}{0cm}
\setlength{\listparindent}{0cm}
\setlength{\leftmargin}{\evensidemargin}
\addtolength{\leftmargin}{\tmplength}
\settowidth{\labelsep}{X}
\addtolength{\leftmargin}{\labelsep}
\setlength{\labelwidth}{\tmplength}
}
\begin{flushleft}
\item[\textbf{Declaração}\hfill]
\begin{ttfamily}
public Parenthesis:  boolean;\end{ttfamily}


\end{flushleft}
\par
\item[\textbf{Descrição}]
'('/')' AnsiCharacters

\end{list}
\paragraph*{Decimals}\hspace*{\fill}

\begin{list}{}{
\settowidth{\tmplength}{\textbf{Declaração}}
\setlength{\itemindent}{0cm}
\setlength{\listparindent}{0cm}
\setlength{\leftmargin}{\evensidemargin}
\addtolength{\leftmargin}{\tmplength}
\settowidth{\labelsep}{X}
\addtolength{\leftmargin}{\labelsep}
\setlength{\labelwidth}{\tmplength}
}
\begin{flushleft}
\item[\textbf{Declaração}\hfill]
\begin{ttfamily}
public Decimals:  byte;\end{ttfamily}


\end{flushleft}
\par
\item[\textbf{Descrição}]
decimal point or cluster \begin{ttfamily}value\end{ttfamily}(\ref{mi_rtl_ui_Dmxscroller.TDmxFieldRec-Value})

\end{list}
\paragraph*{FieldSize}\hspace*{\fill}

\begin{list}{}{
\settowidth{\tmplength}{\textbf{Declaração}}
\setlength{\itemindent}{0cm}
\setlength{\listparindent}{0cm}
\setlength{\leftmargin}{\evensidemargin}
\addtolength{\leftmargin}{\tmplength}
\settowidth{\labelsep}{X}
\addtolength{\leftmargin}{\labelsep}
\setlength{\labelwidth}{\tmplength}
}
\begin{flushleft}
\item[\textbf{Declaração}\hfill]
\begin{ttfamily}
public FieldSize:  integer;\end{ttfamily}


\end{flushleft}
\par
\item[\textbf{Descrição}]
sizeof (datatype)

\end{list}
\paragraph*{DataTab}\hspace*{\fill}

\begin{list}{}{
\settowidth{\tmplength}{\textbf{Declaração}}
\setlength{\itemindent}{0cm}
\setlength{\listparindent}{0cm}
\setlength{\leftmargin}{\evensidemargin}
\addtolength{\leftmargin}{\tmplength}
\settowidth{\labelsep}{X}
\addtolength{\leftmargin}{\labelsep}
\setlength{\labelwidth}{\tmplength}
}
\begin{flushleft}
\item[\textbf{Declaração}\hfill]
\begin{ttfamily}
public DataTab:  integer;\end{ttfamily}


\end{flushleft}
\par
\item[\textbf{Descrição}]
position in record

\end{list}
\paragraph*{Template}\hspace*{\fill}

\begin{list}{}{
\settowidth{\tmplength}{\textbf{Declaração}}
\setlength{\itemindent}{0cm}
\setlength{\listparindent}{0cm}
\setlength{\leftmargin}{\evensidemargin}
\addtolength{\leftmargin}{\tmplength}
\settowidth{\labelsep}{X}
\addtolength{\leftmargin}{\labelsep}
\setlength{\labelwidth}{\tmplength}
}
\begin{flushleft}
\item[\textbf{Declaração}\hfill]
\begin{ttfamily}
public Template:  ptString;\end{ttfamily}


\end{flushleft}
\par
\item[\textbf{Descrição}]
Field \begin{ttfamily}Template\end{ttfamily}

\end{list}
\paragraph*{ListComboBox}\hspace*{\fill}

\begin{list}{}{
\settowidth{\tmplength}{\textbf{Declaração}}
\setlength{\itemindent}{0cm}
\setlength{\listparindent}{0cm}
\setlength{\leftmargin}{\evensidemargin}
\addtolength{\leftmargin}{\tmplength}
\settowidth{\labelsep}{X}
\addtolength{\leftmargin}{\labelsep}
\setlength{\labelwidth}{\tmplength}
}
\begin{flushleft}
\item[\textbf{Declaração}\hfill]
\begin{ttfamily}
public ListComboBox: PSItem;\end{ttfamily}


\end{flushleft}
\par
\item[\textbf{Descrição}]
O atributo \textbf{\begin{ttfamily}ListComboBox\end{ttfamily}} contém uma lista de opções possíveis para o campo.

\begin{itemize}
\item Nota: \begin{itemize}
\item Após caractere \textbf{CharListComboBox} contém um ponteiro para uma lista de opções do mesmo tipo de campo. \begin{itemize}
\item Exemplo: \texttt{\\\nopagebreak[3]
\\\nopagebreak[3]
}\textbf{Const}\texttt{\\\nopagebreak[3]
~~~'~Dia~de~vencimento:~{\textbackslash}Ssssss'+ChFN+'Dia'+CreateOptions(accNormal,~1,\\\nopagebreak[3]
~~~~~~NewSItem('Dia~10',\\\nopagebreak[3]
~~~~~~NewSItem('Dia~15',\\\nopagebreak[3]
~~~~~~NewSItem('Dia~20',\\\nopagebreak[3]
~~~~~~NewSItem('Dia~25',\\\nopagebreak[3]
~~~~~~~~~~~~~~~}\textbf{nil}\texttt{)))));\\
}
\end{itemize}
\end{itemize}
\end{itemize}

\end{list}
\paragraph*{ListComboBox{\_}Default}\hspace*{\fill}

\begin{list}{}{
\settowidth{\tmplength}{\textbf{Declaração}}
\setlength{\itemindent}{0cm}
\setlength{\listparindent}{0cm}
\setlength{\leftmargin}{\evensidemargin}
\addtolength{\leftmargin}{\tmplength}
\settowidth{\labelsep}{X}
\addtolength{\leftmargin}{\labelsep}
\setlength{\labelwidth}{\tmplength}
}
\begin{flushleft}
\item[\textbf{Declaração}\hfill]
\begin{ttfamily}
public ListComboBox{\_}Default: Longint;\end{ttfamily}


\end{flushleft}
\par
\item[\textbf{Descrição}]
O Atributo \textbf{\begin{ttfamily}ListComboBox{\_}Default\end{ttfamily}}* é usado guardar o valor padrão para a lista do BomboBox ou LookupBox

\begin{itemize}
\item Exemplo para selecionar "Dia 20" da lista. \begin{itemize}
\item O número \textbf{2} representa o terceiro item da lista. \texttt{~~~\\\nopagebreak[3]
\\\nopagebreak[3]
}\textbf{Const}\texttt{~~~~\\\nopagebreak[3]
~~~'~Vencimento:~{\textbackslash}Ssssss'+ChFN+'Dia'+CreateOptions(accNormal,~2,~~~~\\\nopagebreak[3]
~~~~~~NewSItem('Dia~10',~~~~\\\nopagebreak[3]
~~~~~~NewSItem('Dia~15',~~~~\\\nopagebreak[3]
~~~~~~NewSItem('Dia~20',~~~~\\\nopagebreak[3]
~~~~~~NewSItem('Dia~25',~~~~\\\nopagebreak[3]
~~~~~~~~~~~~~~~}\textbf{nil}\texttt{)))));\\
}
\end{itemize}
\end{itemize}

\end{list}
\paragraph*{ExecAction}\hspace*{\fill}

\begin{list}{}{
\settowidth{\tmplength}{\textbf{Declaração}}
\setlength{\itemindent}{0cm}
\setlength{\listparindent}{0cm}
\setlength{\leftmargin}{\evensidemargin}
\addtolength{\leftmargin}{\tmplength}
\settowidth{\labelsep}{X}
\addtolength{\leftmargin}{\labelsep}
\setlength{\labelwidth}{\tmplength}
}
\begin{flushleft}
\item[\textbf{Declaração}\hfill]
\begin{ttfamily}
public ExecAction: AnsiString;\end{ttfamily}


\end{flushleft}
\par
\item[\textbf{Descrição}]
O campo \textbf{\begin{ttfamily}ExecAction\end{ttfamily}} é inicializado no interpretador de \begin{ttfamily}Template\end{ttfamily}(\ref{mi_rtl_ui_Dmxscroller.TDmxFieldRec-Template}) quando o caractere \textbf{\begin{ttfamily}CharExecAction\end{ttfamily}(\ref{mi_rtl_ui_dmxscroller_form-CharExecAction})} é encontrado.

\begin{itemize}
\item \textbf{EXEMPLO DE USO DE AÇÕES NO \begin{ttfamily}TEMPLATE\end{ttfamily}(\ref{mi_rtl_ui_Dmxscroller.TDmxFieldRec-Template})} \begin{enumerate}
\setcounter{enumi}{0} \setcounter{enumii}{0} \setcounter{enumiii}{0} \setcounter{enumiv}{0} 
\item Se o atributo \textbf{\begin{ttfamily}Fieldnum\end{ttfamily}(\ref{mi_rtl_ui_Dmxscroller.TDmxFieldRec-Fieldnum})} do campo for diferente de zero, então o \textbf{rótulo} do botão associado a ação será o caracteres 🔍 e a ação pode atualizar o buffer do campo. \begin{itemize}
\item No exemplo a seguir a função CreateExecAction retorna a string \textbf{\begin{ttfamily}chFN\end{ttfamily}(\ref{mi_rtl_ui_dmxscroller_form-ChFN})+aFieldName+'~ 🔍~'+ChEA+(aFieldName+'.'+aExecAction)}.
\item O interpretador de \begin{ttfamily}Template\end{ttfamily}(\ref{mi_rtl_ui_Dmxscroller.TDmxFieldRec-Template}) atualiza a string \begin{ttfamily}LinkExecAction\end{ttfamily}(\ref{mi_rtl_ui_Dmxscroller.TDmxFieldRec-LinkExecAction}) caso o o ponto seja encontrado no \begin{ttfamily}ExecAction\end{ttfamily} do Label.

\texttt{\\\nopagebreak[3]
\\\nopagebreak[3]
Result~:=~NewSItem('~Cliente:~'+'{\textbackslash}LLLLL'+CreateExecAction('Cliente',Pesquisa.}\textbf{Name}\texttt{),}\textbf{nil}\texttt{);\\
}
\end{itemize}
\setcounter{enumi}{1} \setcounter{enumii}{1} \setcounter{enumiii}{1} \setcounter{enumiv}{1} 
\item Se o atributo \textbf{\begin{ttfamily}Fieldnum\end{ttfamily}(\ref{mi_rtl_ui_Dmxscroller.TDmxFieldRec-Fieldnum})} do campo for igual a zero, então a rótulo do botão será o rótulo do campo. \begin{itemize}
\item No exemplo a seguir um rótulo de novo cliente (icons 🆕) e um botão ok (icons 🆗)

\texttt{\\\nopagebreak[3]
\\\nopagebreak[3]
NewSItem('~~🆕~{\&}Novo~cliente:~'+CharExecAction+Action{\_}Novo.}\textbf{name}\texttt{+\\\nopagebreak[3]
~~~~~~~~~'~~~~~~~🆗~~'+CharExecAction+Action{\_}Ok.}\textbf{name}\texttt{)\\
}
\end{itemize}
\end{enumerate}
\end{itemize}

\end{list}
\paragraph*{LinkExecAction}\hspace*{\fill}

\begin{list}{}{
\settowidth{\tmplength}{\textbf{Declaração}}
\setlength{\itemindent}{0cm}
\setlength{\listparindent}{0cm}
\setlength{\leftmargin}{\evensidemargin}
\addtolength{\leftmargin}{\tmplength}
\settowidth{\labelsep}{X}
\addtolength{\leftmargin}{\labelsep}
\setlength{\labelwidth}{\tmplength}
}
\begin{flushleft}
\item[\textbf{Declaração}\hfill]
\begin{ttfamily}
public LinkExecAction: pDmxFieldRec;\end{ttfamily}


\end{flushleft}
\par
\item[\textbf{Descrição}]
O atributo \textbf{\begin{ttfamily}LinkExecAction\end{ttfamily}} é atualizado com o ponteiro do campo passado por \textbf{\begin{ttfamily}execAction\end{ttfamily}(\ref{mi_rtl_ui_Dmxscroller.TDmxFieldRec-ExecAction})}.

\begin{itemize}
\item O Interpretador de \begin{ttfamily}Template\end{ttfamily}(\ref{mi_rtl_ui_Dmxscroller.TDmxFieldRec-Template}) deve pegar o campo usando a função FieldByName(aFieldName passado em \begin{ttfamily}execAction\end{ttfamily}(\ref{mi_rtl_ui_Dmxscroller.TDmxFieldRec-ExecAction})), quando \begin{ttfamily}execAction\end{ttfamily}(\ref{mi_rtl_ui_Dmxscroller.TDmxFieldRec-ExecAction}) tiver um ponto antes do nome da ação. \begin{itemize}
\item Ex: \textbf{(aFieldName.aExecAction)}.
\end{itemize}
\end{itemize} \texttt{\\\nopagebreak[3]
\\\nopagebreak[3]
Result~:=~NewSItem('~Cliente:~'+'{\textbackslash}LLLLL'+CreateExecAction('Cliente',Pesquisa.}\textbf{Name}\texttt{),}\textbf{nil}\texttt{);\\
}

\end{list}
\paragraph*{CharShowPassword}\hspace*{\fill}

\begin{list}{}{
\settowidth{\tmplength}{\textbf{Declaração}}
\setlength{\itemindent}{0cm}
\setlength{\listparindent}{0cm}
\setlength{\leftmargin}{\evensidemargin}
\addtolength{\leftmargin}{\tmplength}
\settowidth{\labelsep}{X}
\addtolength{\leftmargin}{\labelsep}
\setlength{\labelwidth}{\tmplength}
}
\begin{flushleft}
\item[\textbf{Declaração}\hfill]
\begin{ttfamily}
public CharShowPassword:  AnsiChar;\end{ttfamily}


\end{flushleft}
\end{list}
\paragraph*{{\_}DateMask}\hspace*{\fill}

\begin{list}{}{
\settowidth{\tmplength}{\textbf{Declaração}}
\setlength{\itemindent}{0cm}
\setlength{\listparindent}{0cm}
\setlength{\leftmargin}{\evensidemargin}
\addtolength{\leftmargin}{\tmplength}
\settowidth{\labelsep}{X}
\addtolength{\leftmargin}{\labelsep}
\setlength{\labelwidth}{\tmplength}
}
\begin{flushleft}
\item[\textbf{Declaração}\hfill]
\begin{ttfamily}
public {\_}DateMask: TDates.TDateMask;\end{ttfamily}


\end{flushleft}
\end{list}
\paragraph*{{\_}HourMask}\hspace*{\fill}

\begin{list}{}{
\settowidth{\tmplength}{\textbf{Declaração}}
\setlength{\itemindent}{0cm}
\setlength{\listparindent}{0cm}
\setlength{\leftmargin}{\evensidemargin}
\addtolength{\leftmargin}{\tmplength}
\settowidth{\labelsep}{X}
\addtolength{\leftmargin}{\labelsep}
\setlength{\labelwidth}{\tmplength}
}
\begin{flushleft}
\item[\textbf{Declaração}\hfill]
\begin{ttfamily}
public {\_}HourMask: TDates.THourMask;\end{ttfamily}


\end{flushleft}
\end{list}
\paragraph*{QuitFieldAltomatic}\hspace*{\fill}

\begin{list}{}{
\settowidth{\tmplength}{\textbf{Declaração}}
\setlength{\itemindent}{0cm}
\setlength{\listparindent}{0cm}
\setlength{\leftmargin}{\evensidemargin}
\addtolength{\leftmargin}{\tmplength}
\settowidth{\labelsep}{X}
\addtolength{\leftmargin}{\labelsep}
\setlength{\labelwidth}{\tmplength}
}
\begin{flushleft}
\item[\textbf{Declaração}\hfill]
\begin{ttfamily}
public QuitFieldAltomatic: Boolean;\end{ttfamily}


\end{flushleft}
\end{list}
\paragraph*{CurPos}\hspace*{\fill}

\begin{list}{}{
\settowidth{\tmplength}{\textbf{Declaração}}
\setlength{\itemindent}{0cm}
\setlength{\listparindent}{0cm}
\setlength{\leftmargin}{\evensidemargin}
\addtolength{\leftmargin}{\tmplength}
\settowidth{\labelsep}{X}
\addtolength{\leftmargin}{\labelsep}
\setlength{\labelwidth}{\tmplength}
}
\begin{flushleft}
\item[\textbf{Declaração}\hfill]
\begin{ttfamily}
public CurPos: integer;\end{ttfamily}


\end{flushleft}
\par
\item[\textbf{Descrição}]
Posição do curso quando este campo estiver sendo editado.

\end{list}
\paragraph*{SelStart}\hspace*{\fill}

\begin{list}{}{
\settowidth{\tmplength}{\textbf{Declaração}}
\setlength{\itemindent}{0cm}
\setlength{\listparindent}{0cm}
\setlength{\leftmargin}{\evensidemargin}
\addtolength{\leftmargin}{\tmplength}
\settowidth{\labelsep}{X}
\addtolength{\leftmargin}{\labelsep}
\setlength{\labelwidth}{\tmplength}
}
\begin{flushleft}
\item[\textbf{Declaração}\hfill]
\begin{ttfamily}
public SelStart: Integer;\end{ttfamily}


\end{flushleft}
\par
\item[\textbf{Descrição}]
Posição do início da seleção quando este campo estiver sendo editado.

\end{list}
\paragraph*{SelEnd}\hspace*{\fill}

\begin{list}{}{
\settowidth{\tmplength}{\textbf{Declaração}}
\setlength{\itemindent}{0cm}
\setlength{\listparindent}{0cm}
\setlength{\leftmargin}{\evensidemargin}
\addtolength{\leftmargin}{\tmplength}
\settowidth{\labelsep}{X}
\addtolength{\leftmargin}{\labelsep}
\setlength{\labelwidth}{\tmplength}
}
\begin{flushleft}
\item[\textbf{Declaração}\hfill]
\begin{ttfamily}
public SelEnd: Integer;\end{ttfamily}


\end{flushleft}
\par
\item[\textbf{Descrição}]
Posição do fim da seleção quando este campo estiver sendo editado.

\end{list}
\paragraph*{{\_}FieldAltered}\hspace*{\fill}

\begin{list}{}{
\settowidth{\tmplength}{\textbf{Declaração}}
\setlength{\itemindent}{0cm}
\setlength{\listparindent}{0cm}
\setlength{\leftmargin}{\evensidemargin}
\addtolength{\leftmargin}{\tmplength}
\settowidth{\labelsep}{X}
\addtolength{\leftmargin}{\labelsep}
\setlength{\labelwidth}{\tmplength}
}
\begin{flushleft}
\item[\textbf{Declaração}\hfill]
\begin{ttfamily}
public {\_}FieldAltered: Boolean;\end{ttfamily}


\end{flushleft}
\end{list}
\paragraph*{HelpCtx{\_}Hint}\hspace*{\fill}

\begin{list}{}{
\settowidth{\tmplength}{\textbf{Declaração}}
\setlength{\itemindent}{0cm}
\setlength{\listparindent}{0cm}
\setlength{\leftmargin}{\evensidemargin}
\addtolength{\leftmargin}{\tmplength}
\settowidth{\labelsep}{X}
\addtolength{\leftmargin}{\labelsep}
\setlength{\labelwidth}{\tmplength}
}
\begin{flushleft}
\item[\textbf{Declaração}\hfill]
\begin{ttfamily}
public HelpCtx{\_}Hint: AnsiString;\end{ttfamily}


\end{flushleft}
\par
\item[\textbf{Descrição}]
O campo \textbf{\begin{ttfamily}HelpCtx{\_}Hint\end{ttfamily}} contém a documentação resumida do registro.

\end{list}
\paragraph*{HelpCtx{\_}Porque}\hspace*{\fill}

\begin{list}{}{
\settowidth{\tmplength}{\textbf{Declaração}}
\setlength{\itemindent}{0cm}
\setlength{\listparindent}{0cm}
\setlength{\leftmargin}{\evensidemargin}
\addtolength{\leftmargin}{\tmplength}
\settowidth{\labelsep}{X}
\addtolength{\leftmargin}{\labelsep}
\setlength{\labelwidth}{\tmplength}
}
\begin{flushleft}
\item[\textbf{Declaração}\hfill]
\begin{ttfamily}
public HelpCtx{\_}Porque: AnsiString;\end{ttfamily}


\end{flushleft}
\par
\item[\textbf{Descrição}]
Por que preciso deste campo?

\end{list}
\paragraph*{HelpCtx{\_}Onde}\hspace*{\fill}

\begin{list}{}{
\settowidth{\tmplength}{\textbf{Declaração}}
\setlength{\itemindent}{0cm}
\setlength{\listparindent}{0cm}
\setlength{\leftmargin}{\evensidemargin}
\addtolength{\leftmargin}{\tmplength}
\settowidth{\labelsep}{X}
\addtolength{\leftmargin}{\labelsep}
\setlength{\labelwidth}{\tmplength}
}
\begin{flushleft}
\item[\textbf{Declaração}\hfill]
\begin{ttfamily}
public HelpCtx{\_}Onde: AnsiString;\end{ttfamily}


\end{flushleft}
\par
\item[\textbf{Descrição}]
Onde esse campo será usado?

\end{list}
\paragraph*{{\_}OkSpcAnt}\hspace*{\fill}

\begin{list}{}{
\settowidth{\tmplength}{\textbf{Declaração}}
\setlength{\itemindent}{0cm}
\setlength{\listparindent}{0cm}
\setlength{\leftmargin}{\evensidemargin}
\addtolength{\leftmargin}{\tmplength}
\settowidth{\labelsep}{X}
\addtolength{\leftmargin}{\labelsep}
\setlength{\labelwidth}{\tmplength}
}
\begin{flushleft}
\item[\textbf{Declaração}\hfill]
\begin{ttfamily}
public {\_}OkSpcAnt: Boolean;\end{ttfamily}


\end{flushleft}
\par
\item[\textbf{Descrição}]
Salva o valor de {\_}OkSpc antes de setar com aOkSpc

\end{list}
\paragraph*{ProviderFlags}\hspace*{\fill}

\begin{list}{}{
\settowidth{\tmplength}{\textbf{Declaração}}
\setlength{\itemindent}{0cm}
\setlength{\listparindent}{0cm}
\setlength{\leftmargin}{\evensidemargin}
\addtolength{\leftmargin}{\tmplength}
\settowidth{\labelsep}{X}
\addtolength{\leftmargin}{\labelsep}
\setlength{\labelwidth}{\tmplength}
}
\begin{flushleft}
\item[\textbf{Declaração}\hfill]
\begin{ttfamily}
public ProviderFlags: TUiTypes.TMiProviderFlags;\end{ttfamily}


\end{flushleft}
\par
\item[\textbf{Descrição}]
O atributo \textbf{\begin{ttfamily}ProviderFlags\end{ttfamily}} é usado nos métodos \textbf{TUiDmxScroller{\_}sql.CreateTables} e \textbf{TUiDmxScroller{\_}sql.CreateBufDataset{\_}FieldDefs} para integração do componente \textbf{TDmxScroller} com o componente TSqlDbConnector.

\end{list}
\paragraph*{ForeignKey}\hspace*{\fill}

\begin{list}{}{
\settowidth{\tmplength}{\textbf{Declaração}}
\setlength{\itemindent}{0cm}
\setlength{\listparindent}{0cm}
\setlength{\leftmargin}{\evensidemargin}
\addtolength{\leftmargin}{\tmplength}
\settowidth{\labelsep}{X}
\addtolength{\leftmargin}{\labelsep}
\setlength{\labelwidth}{\tmplength}
}
\begin{flushleft}
\item[\textbf{Declaração}\hfill]
\begin{ttfamily}
public ForeignKey: TuiTypes.TForeignKey;\end{ttfamily}


\end{flushleft}
\par
\item[\textbf{Descrição}]
O atributo \textbf{\begin{ttfamily}ForeignKey\end{ttfamily}} é usado para criar chave estrangeira e os relacionamentos

\end{list}
\paragraph*{KeyForeign}\hspace*{\fill}

\begin{list}{}{
\settowidth{\tmplength}{\textbf{Declaração}}
\setlength{\itemindent}{0cm}
\setlength{\listparindent}{0cm}
\setlength{\leftmargin}{\evensidemargin}
\addtolength{\leftmargin}{\tmplength}
\settowidth{\labelsep}{X}
\addtolength{\leftmargin}{\labelsep}
\setlength{\labelwidth}{\tmplength}
}
\begin{flushleft}
\item[\textbf{Declaração}\hfill]
\begin{ttfamily}
public KeyForeign: AnsiString;\end{ttfamily}


\end{flushleft}
\par
\item[\textbf{Descrição}]
O atributo \textbf{\begin{ttfamily}KeyForeign\end{ttfamily}} contém uma string com o nome da tabela estrangeira e a lista de campos relacionados

\begin{itemize}
\item \textbf{EXEMPLO} \begin{itemize}
\item CIDADES,ESTADO;CIDADE \begin{itemize}
\item CIDADES = tabela estrangeira
\item ESTADO = Estado da cidade.
\item CIDADE = Cidade do estado.
\end{itemize}
\end{itemize}
\end{itemize}

\end{list}
\subsubsection*{\large{\textbf{Métodos}}\normalsize\hspace{1ex}\hfill}
\paragraph*{StrNumberValid}\hspace*{\fill}

\begin{list}{}{
\settowidth{\tmplength}{\textbf{Declaração}}
\setlength{\itemindent}{0cm}
\setlength{\listparindent}{0cm}
\setlength{\leftmargin}{\evensidemargin}
\addtolength{\leftmargin}{\tmplength}
\settowidth{\labelsep}{X}
\addtolength{\leftmargin}{\labelsep}
\setlength{\labelwidth}{\tmplength}
}
\begin{flushleft}
\item[\textbf{Declaração}\hfill]
\begin{ttfamily}
public function StrNumberValid(S: AnsiString):AnsiString;\end{ttfamily}


\end{flushleft}
\end{list}
\paragraph*{GetAsStringFromBuffer}\hspace*{\fill}

\begin{list}{}{
\settowidth{\tmplength}{\textbf{Declaração}}
\setlength{\itemindent}{0cm}
\setlength{\listparindent}{0cm}
\setlength{\leftmargin}{\evensidemargin}
\addtolength{\leftmargin}{\tmplength}
\settowidth{\labelsep}{X}
\addtolength{\leftmargin}{\labelsep}
\setlength{\labelwidth}{\tmplength}
}
\begin{flushleft}
\item[\textbf{Declaração}\hfill]
\begin{ttfamily}
public Function GetAsStringFromBuffer(aWorkingData : pointer):AnsiString;\end{ttfamily}


\end{flushleft}
\end{list}
\paragraph*{SetAsString}\hspace*{\fill}

\begin{list}{}{
\settowidth{\tmplength}{\textbf{Declaração}}
\setlength{\itemindent}{0cm}
\setlength{\listparindent}{0cm}
\setlength{\leftmargin}{\evensidemargin}
\addtolength{\leftmargin}{\tmplength}
\settowidth{\labelsep}{X}
\addtolength{\leftmargin}{\labelsep}
\setlength{\labelwidth}{\tmplength}
}
\begin{flushleft}
\item[\textbf{Declaração}\hfill]
\begin{ttfamily}
public Procedure SetAsString(S:AnsiString);\end{ttfamily}


\end{flushleft}
\end{list}
\paragraph*{GetAsString}\hspace*{\fill}

\begin{list}{}{
\settowidth{\tmplength}{\textbf{Declaração}}
\setlength{\itemindent}{0cm}
\setlength{\listparindent}{0cm}
\setlength{\leftmargin}{\evensidemargin}
\addtolength{\leftmargin}{\tmplength}
\settowidth{\labelsep}{X}
\addtolength{\leftmargin}{\labelsep}
\setlength{\labelwidth}{\tmplength}
}
\begin{flushleft}
\item[\textbf{Declaração}\hfill]
\begin{ttfamily}
public Function GetAsString:AnsiString;\end{ttfamily}


\end{flushleft}
\end{list}
\paragraph*{IsInputText}\hspace*{\fill}

\begin{list}{}{
\settowidth{\tmplength}{\textbf{Declaração}}
\setlength{\itemindent}{0cm}
\setlength{\listparindent}{0cm}
\setlength{\leftmargin}{\evensidemargin}
\addtolength{\leftmargin}{\tmplength}
\settowidth{\labelsep}{X}
\addtolength{\leftmargin}{\labelsep}
\setlength{\labelwidth}{\tmplength}
}
\begin{flushleft}
\item[\textbf{Declaração}\hfill]
\begin{ttfamily}
public Function IsInputText:Boolean;\end{ttfamily}


\end{flushleft}
\end{list}
\paragraph*{SItemsLen}\hspace*{\fill}

\begin{list}{}{
\settowidth{\tmplength}{\textbf{Declaração}}
\setlength{\itemindent}{0cm}
\setlength{\listparindent}{0cm}
\setlength{\leftmargin}{\evensidemargin}
\addtolength{\leftmargin}{\tmplength}
\settowidth{\labelsep}{X}
\addtolength{\leftmargin}{\labelsep}
\setlength{\labelwidth}{\tmplength}
}
\begin{flushleft}
\item[\textbf{Declaração}\hfill]
\begin{ttfamily}
public function SItemsLen(S: PSItem) : SmallInt;\end{ttfamily}


\end{flushleft}
\end{list}
\paragraph*{MaxItemStrLen}\hspace*{\fill}

\begin{list}{}{
\settowidth{\tmplength}{\textbf{Declaração}}
\setlength{\itemindent}{0cm}
\setlength{\listparindent}{0cm}
\setlength{\leftmargin}{\evensidemargin}
\addtolength{\leftmargin}{\tmplength}
\settowidth{\labelsep}{X}
\addtolength{\leftmargin}{\labelsep}
\setlength{\labelwidth}{\tmplength}
}
\begin{flushleft}
\item[\textbf{Declaração}\hfill]
\begin{ttfamily}
public function MaxItemStrLen(AItems: PSItem) : integer;\end{ttfamily}


\end{flushleft}
\end{list}
\paragraph*{GetMaxLength}\hspace*{\fill}

\begin{list}{}{
\settowidth{\tmplength}{\textbf{Declaração}}
\setlength{\itemindent}{0cm}
\setlength{\listparindent}{0cm}
\setlength{\leftmargin}{\evensidemargin}
\addtolength{\leftmargin}{\tmplength}
\settowidth{\labelsep}{X}
\addtolength{\leftmargin}{\labelsep}
\setlength{\labelwidth}{\tmplength}
}
\begin{flushleft}
\item[\textbf{Declaração}\hfill]
\begin{ttfamily}
public Function GetMaxLength():integer;\end{ttfamily}


\end{flushleft}
\end{list}
\paragraph*{IsStaticText}\hspace*{\fill}

\begin{list}{}{
\settowidth{\tmplength}{\textbf{Declaração}}
\setlength{\itemindent}{0cm}
\setlength{\listparindent}{0cm}
\setlength{\leftmargin}{\evensidemargin}
\addtolength{\leftmargin}{\tmplength}
\settowidth{\labelsep}{X}
\addtolength{\leftmargin}{\labelsep}
\setlength{\labelwidth}{\tmplength}
}
\begin{flushleft}
\item[\textbf{Declaração}\hfill]
\begin{ttfamily}
public function IsStaticText:Boolean;\end{ttfamily}


\end{flushleft}
\end{list}
\paragraph*{IsInputRadio}\hspace*{\fill}

\begin{list}{}{
\settowidth{\tmplength}{\textbf{Declaração}}
\setlength{\itemindent}{0cm}
\setlength{\listparindent}{0cm}
\setlength{\leftmargin}{\evensidemargin}
\addtolength{\leftmargin}{\tmplength}
\settowidth{\labelsep}{X}
\addtolength{\leftmargin}{\labelsep}
\setlength{\labelwidth}{\tmplength}
}
\begin{flushleft}
\item[\textbf{Declaração}\hfill]
\begin{ttfamily}
public function IsInputRadio:Boolean;\end{ttfamily}


\end{flushleft}
\end{list}
\paragraph*{IsInputDbRadio}\hspace*{\fill}

\begin{list}{}{
\settowidth{\tmplength}{\textbf{Declaração}}
\setlength{\itemindent}{0cm}
\setlength{\listparindent}{0cm}
\setlength{\leftmargin}{\evensidemargin}
\addtolength{\leftmargin}{\tmplength}
\settowidth{\labelsep}{X}
\addtolength{\leftmargin}{\labelsep}
\setlength{\labelwidth}{\tmplength}
}
\begin{flushleft}
\item[\textbf{Declaração}\hfill]
\begin{ttfamily}
public function IsInputDbRadio:Boolean;\end{ttfamily}


\end{flushleft}
\end{list}
\paragraph*{IsInputCheckbox}\hspace*{\fill}

\begin{list}{}{
\settowidth{\tmplength}{\textbf{Declaração}}
\setlength{\itemindent}{0cm}
\setlength{\listparindent}{0cm}
\setlength{\leftmargin}{\evensidemargin}
\addtolength{\leftmargin}{\tmplength}
\settowidth{\labelsep}{X}
\addtolength{\leftmargin}{\labelsep}
\setlength{\labelwidth}{\tmplength}
}
\begin{flushleft}
\item[\textbf{Declaração}\hfill]
\begin{ttfamily}
public function IsInputCheckbox:Boolean;\end{ttfamily}


\end{flushleft}
\end{list}
\paragraph*{isInputPassword}\hspace*{\fill}

\begin{list}{}{
\settowidth{\tmplength}{\textbf{Declaração}}
\setlength{\itemindent}{0cm}
\setlength{\listparindent}{0cm}
\setlength{\leftmargin}{\evensidemargin}
\addtolength{\leftmargin}{\tmplength}
\settowidth{\labelsep}{X}
\addtolength{\leftmargin}{\labelsep}
\setlength{\labelwidth}{\tmplength}
}
\begin{flushleft}
\item[\textbf{Declaração}\hfill]
\begin{ttfamily}
public function isInputPassword:Boolean;\end{ttfamily}


\end{flushleft}
\end{list}
\paragraph*{IsInputHidden}\hspace*{\fill}

\begin{list}{}{
\settowidth{\tmplength}{\textbf{Declaração}}
\setlength{\itemindent}{0cm}
\setlength{\listparindent}{0cm}
\setlength{\leftmargin}{\evensidemargin}
\addtolength{\leftmargin}{\tmplength}
\settowidth{\labelsep}{X}
\addtolength{\leftmargin}{\labelsep}
\setlength{\labelwidth}{\tmplength}
}
\begin{flushleft}
\item[\textbf{Declaração}\hfill]
\begin{ttfamily}
public function IsInputHidden:Boolean;\end{ttfamily}


\end{flushleft}
\end{list}
\paragraph*{IsSelect}\hspace*{\fill}

\begin{list}{}{
\settowidth{\tmplength}{\textbf{Declaração}}
\setlength{\itemindent}{0cm}
\setlength{\listparindent}{0cm}
\setlength{\leftmargin}{\evensidemargin}
\addtolength{\leftmargin}{\tmplength}
\settowidth{\labelsep}{X}
\addtolength{\leftmargin}{\labelsep}
\setlength{\labelwidth}{\tmplength}
}
\begin{flushleft}
\item[\textbf{Declaração}\hfill]
\begin{ttfamily}
public function IsSelect:Boolean;\end{ttfamily}


\end{flushleft}
\par
\item[\textbf{Descrição}]
O objeto filho que implementar um ISelect deve anular e retornar a interface ISelect;

\end{list}
\paragraph*{IsComboBox}\hspace*{\fill}

\begin{list}{}{
\settowidth{\tmplength}{\textbf{Declaração}}
\setlength{\itemindent}{0cm}
\setlength{\listparindent}{0cm}
\setlength{\leftmargin}{\evensidemargin}
\addtolength{\leftmargin}{\tmplength}
\settowidth{\labelsep}{X}
\addtolength{\leftmargin}{\labelsep}
\setlength{\labelwidth}{\tmplength}
}
\begin{flushleft}
\item[\textbf{Declaração}\hfill]
\begin{ttfamily}
public function IsComboBox:Boolean;\end{ttfamily}


\end{flushleft}
\par
\item[\textbf{Descrição}]
Usado quando trata{-}se de campos enumerados.

\end{list}
\paragraph*{FirstField}\hspace*{\fill}

\begin{list}{}{
\settowidth{\tmplength}{\textbf{Declaração}}
\setlength{\itemindent}{0cm}
\setlength{\listparindent}{0cm}
\setlength{\leftmargin}{\evensidemargin}
\addtolength{\leftmargin}{\tmplength}
\settowidth{\labelsep}{X}
\addtolength{\leftmargin}{\labelsep}
\setlength{\labelwidth}{\tmplength}
}
\begin{flushleft}
\item[\textbf{Declaração}\hfill]
\begin{ttfamily}
public function FirstField: pDmxFieldRec;\end{ttfamily}


\end{flushleft}
\end{list}
\paragraph*{LastField}\hspace*{\fill}

\begin{list}{}{
\settowidth{\tmplength}{\textbf{Declaração}}
\setlength{\itemindent}{0cm}
\setlength{\listparindent}{0cm}
\setlength{\leftmargin}{\evensidemargin}
\addtolength{\leftmargin}{\tmplength}
\settowidth{\labelsep}{X}
\addtolength{\leftmargin}{\labelsep}
\setlength{\labelwidth}{\tmplength}
}
\begin{flushleft}
\item[\textbf{Declaração}\hfill]
\begin{ttfamily}
public function LastField: pDmxFieldRec;\end{ttfamily}


\end{flushleft}
\end{list}
\paragraph*{NextField}\hspace*{\fill}

\begin{list}{}{
\settowidth{\tmplength}{\textbf{Declaração}}
\setlength{\itemindent}{0cm}
\setlength{\listparindent}{0cm}
\setlength{\leftmargin}{\evensidemargin}
\addtolength{\leftmargin}{\tmplength}
\settowidth{\labelsep}{X}
\addtolength{\leftmargin}{\labelsep}
\setlength{\labelwidth}{\tmplength}
}
\begin{flushleft}
\item[\textbf{Declaração}\hfill]
\begin{ttfamily}
public function NextField: pDmxFieldRec;\end{ttfamily}


\end{flushleft}
\end{list}
\paragraph*{PrevField}\hspace*{\fill}

\begin{list}{}{
\settowidth{\tmplength}{\textbf{Declaração}}
\setlength{\itemindent}{0cm}
\setlength{\listparindent}{0cm}
\setlength{\leftmargin}{\evensidemargin}
\addtolength{\leftmargin}{\tmplength}
\settowidth{\labelsep}{X}
\addtolength{\leftmargin}{\labelsep}
\setlength{\labelwidth}{\tmplength}
}
\begin{flushleft}
\item[\textbf{Declaração}\hfill]
\begin{ttfamily}
public function PrevField: pDmxFieldRec;\end{ttfamily}


\end{flushleft}
\end{list}
\paragraph*{SelectFirstField}\hspace*{\fill}

\begin{list}{}{
\settowidth{\tmplength}{\textbf{Declaração}}
\setlength{\itemindent}{0cm}
\setlength{\listparindent}{0cm}
\setlength{\leftmargin}{\evensidemargin}
\addtolength{\leftmargin}{\tmplength}
\settowidth{\labelsep}{X}
\addtolength{\leftmargin}{\labelsep}
\setlength{\labelwidth}{\tmplength}
}
\begin{flushleft}
\item[\textbf{Declaração}\hfill]
\begin{ttfamily}
public Function SelectFirstField: pDmxFieldRec;\end{ttfamily}


\end{flushleft}
\end{list}
\paragraph*{SelectLastField}\hspace*{\fill}

\begin{list}{}{
\settowidth{\tmplength}{\textbf{Declaração}}
\setlength{\itemindent}{0cm}
\setlength{\listparindent}{0cm}
\setlength{\leftmargin}{\evensidemargin}
\addtolength{\leftmargin}{\tmplength}
\settowidth{\labelsep}{X}
\addtolength{\leftmargin}{\labelsep}
\setlength{\labelwidth}{\tmplength}
}
\begin{flushleft}
\item[\textbf{Declaração}\hfill]
\begin{ttfamily}
public Function SelectLastField: pDmxFieldRec;\end{ttfamily}


\end{flushleft}
\end{list}
\paragraph*{Select}\hspace*{\fill}

\begin{list}{}{
\settowidth{\tmplength}{\textbf{Declaração}}
\setlength{\itemindent}{0cm}
\setlength{\listparindent}{0cm}
\setlength{\leftmargin}{\evensidemargin}
\addtolength{\leftmargin}{\tmplength}
\settowidth{\labelsep}{X}
\addtolength{\leftmargin}{\labelsep}
\setlength{\labelwidth}{\tmplength}
}
\begin{flushleft}
\item[\textbf{Declaração}\hfill]
\begin{ttfamily}
public Procedure Select;\end{ttfamily}


\end{flushleft}
\end{list}
\paragraph*{GetCount{\_}Cluster}\hspace*{\fill}

\begin{list}{}{
\settowidth{\tmplength}{\textbf{Declaração}}
\setlength{\itemindent}{0cm}
\setlength{\listparindent}{0cm}
\setlength{\leftmargin}{\evensidemargin}
\addtolength{\leftmargin}{\tmplength}
\settowidth{\labelsep}{X}
\addtolength{\leftmargin}{\labelsep}
\setlength{\labelwidth}{\tmplength}
}
\begin{flushleft}
\item[\textbf{Declaração}\hfill]
\begin{ttfamily}
public Function GetCount{\_}Cluster:Integer;\end{ttfamily}


\end{flushleft}
\end{list}
\paragraph*{GetValue{\_}Cluster}\hspace*{\fill}

\begin{list}{}{
\settowidth{\tmplength}{\textbf{Declaração}}
\setlength{\itemindent}{0cm}
\setlength{\listparindent}{0cm}
\setlength{\leftmargin}{\evensidemargin}
\addtolength{\leftmargin}{\tmplength}
\settowidth{\labelsep}{X}
\addtolength{\leftmargin}{\labelsep}
\setlength{\labelwidth}{\tmplength}
}
\begin{flushleft}
\item[\textbf{Declaração}\hfill]
\begin{ttfamily}
public Function GetValue{\_}Cluster(aItem: Integer):AnsiString;\end{ttfamily}


\end{flushleft}
\end{list}
\paragraph*{SetValue{\_}Cluster}\hspace*{\fill}

\begin{list}{}{
\settowidth{\tmplength}{\textbf{Declaração}}
\setlength{\itemindent}{0cm}
\setlength{\listparindent}{0cm}
\setlength{\leftmargin}{\evensidemargin}
\addtolength{\leftmargin}{\tmplength}
\settowidth{\labelsep}{X}
\addtolength{\leftmargin}{\labelsep}
\setlength{\labelwidth}{\tmplength}
}
\begin{flushleft}
\item[\textbf{Declaração}\hfill]
\begin{ttfamily}
public Procedure SetValue{\_}Cluster(aItem:Integer;wValue:AnsiString);\end{ttfamily}


\end{flushleft}
\end{list}
\paragraph*{GetChecked{\_}Cluster}\hspace*{\fill}

\begin{list}{}{
\settowidth{\tmplength}{\textbf{Declaração}}
\setlength{\itemindent}{0cm}
\setlength{\listparindent}{0cm}
\setlength{\leftmargin}{\evensidemargin}
\addtolength{\leftmargin}{\tmplength}
\settowidth{\labelsep}{X}
\addtolength{\leftmargin}{\labelsep}
\setlength{\labelwidth}{\tmplength}
}
\begin{flushleft}
\item[\textbf{Declaração}\hfill]
\begin{ttfamily}
public Function GetChecked{\_}Cluster( aItem: Integer):Boolean;\end{ttfamily}


\end{flushleft}
\end{list}
\paragraph*{SetChecked{\_}Cluster}\hspace*{\fill}

\begin{list}{}{
\settowidth{\tmplength}{\textbf{Declaração}}
\setlength{\itemindent}{0cm}
\setlength{\listparindent}{0cm}
\setlength{\leftmargin}{\evensidemargin}
\addtolength{\leftmargin}{\tmplength}
\settowidth{\labelsep}{X}
\addtolength{\leftmargin}{\labelsep}
\setlength{\labelwidth}{\tmplength}
}
\begin{flushleft}
\item[\textbf{Declaração}\hfill]
\begin{ttfamily}
public Procedure SetChecked{\_}Cluster( aItem : Integer;aValue:Boolean);\end{ttfamily}


\end{flushleft}
\end{list}
\paragraph*{GetCount{\_}InputRadio}\hspace*{\fill}

\begin{list}{}{
\settowidth{\tmplength}{\textbf{Declaração}}
\setlength{\itemindent}{0cm}
\setlength{\listparindent}{0cm}
\setlength{\leftmargin}{\evensidemargin}
\addtolength{\leftmargin}{\tmplength}
\settowidth{\labelsep}{X}
\addtolength{\leftmargin}{\labelsep}
\setlength{\labelwidth}{\tmplength}
}
\begin{flushleft}
\item[\textbf{Declaração}\hfill]
\begin{ttfamily}
public Function GetCount{\_}InputRadio:Integer;\end{ttfamily}


\end{flushleft}
\end{list}
\paragraph*{GetValue{\_}InputRadio}\hspace*{\fill}

\begin{list}{}{
\settowidth{\tmplength}{\textbf{Declaração}}
\setlength{\itemindent}{0cm}
\setlength{\listparindent}{0cm}
\setlength{\leftmargin}{\evensidemargin}
\addtolength{\leftmargin}{\tmplength}
\settowidth{\labelsep}{X}
\addtolength{\leftmargin}{\labelsep}
\setlength{\labelwidth}{\tmplength}
}
\begin{flushleft}
\item[\textbf{Declaração}\hfill]
\begin{ttfamily}
public Function GetValue{\_}InputRadio(aItem: Integer):AnsiString;\end{ttfamily}


\end{flushleft}
\end{list}
\paragraph*{SetValue{\_}InputRadio}\hspace*{\fill}

\begin{list}{}{
\settowidth{\tmplength}{\textbf{Declaração}}
\setlength{\itemindent}{0cm}
\setlength{\listparindent}{0cm}
\setlength{\leftmargin}{\evensidemargin}
\addtolength{\leftmargin}{\tmplength}
\settowidth{\labelsep}{X}
\addtolength{\leftmargin}{\labelsep}
\setlength{\labelwidth}{\tmplength}
}
\begin{flushleft}
\item[\textbf{Declaração}\hfill]
\begin{ttfamily}
public Procedure SetValue{\_}InputRadio(aItem:Integer;aValue:AnsiString);\end{ttfamily}


\end{flushleft}
\end{list}
\paragraph*{GetChecked{\_}InputRadio}\hspace*{\fill}

\begin{list}{}{
\settowidth{\tmplength}{\textbf{Declaração}}
\setlength{\itemindent}{0cm}
\setlength{\listparindent}{0cm}
\setlength{\leftmargin}{\evensidemargin}
\addtolength{\leftmargin}{\tmplength}
\settowidth{\labelsep}{X}
\addtolength{\leftmargin}{\labelsep}
\setlength{\labelwidth}{\tmplength}
}
\begin{flushleft}
\item[\textbf{Declaração}\hfill]
\begin{ttfamily}
public Function GetChecked{\_}InputRadio( aItem: Integer):Boolean;\end{ttfamily}


\end{flushleft}
\end{list}
\paragraph*{SetChecked{\_}InputRadio}\hspace*{\fill}

\begin{list}{}{
\settowidth{\tmplength}{\textbf{Declaração}}
\setlength{\itemindent}{0cm}
\setlength{\listparindent}{0cm}
\setlength{\leftmargin}{\evensidemargin}
\addtolength{\leftmargin}{\tmplength}
\settowidth{\labelsep}{X}
\addtolength{\leftmargin}{\labelsep}
\setlength{\labelwidth}{\tmplength}
}
\begin{flushleft}
\item[\textbf{Declaração}\hfill]
\begin{ttfamily}
public Procedure SetChecked{\_}InputRadio( aItem : Integer;aValue:Boolean);\end{ttfamily}


\end{flushleft}
\end{list}
\paragraph*{get{\_}Item{\_}Focused{\_}InputRadio}\hspace*{\fill}

\begin{list}{}{
\settowidth{\tmplength}{\textbf{Declaração}}
\setlength{\itemindent}{0cm}
\setlength{\listparindent}{0cm}
\setlength{\leftmargin}{\evensidemargin}
\addtolength{\leftmargin}{\tmplength}
\settowidth{\labelsep}{X}
\addtolength{\leftmargin}{\labelsep}
\setlength{\labelwidth}{\tmplength}
}
\begin{flushleft}
\item[\textbf{Declaração}\hfill]
\begin{ttfamily}
public Function get{\_}Item{\_}Focused{\_}InputRadio:Longint;\end{ttfamily}


\end{flushleft}
\end{list}
\paragraph*{GetCount{\_}InputCheckbox}\hspace*{\fill}

\begin{list}{}{
\settowidth{\tmplength}{\textbf{Declaração}}
\setlength{\itemindent}{0cm}
\setlength{\listparindent}{0cm}
\setlength{\leftmargin}{\evensidemargin}
\addtolength{\leftmargin}{\tmplength}
\settowidth{\labelsep}{X}
\addtolength{\leftmargin}{\labelsep}
\setlength{\labelwidth}{\tmplength}
}
\begin{flushleft}
\item[\textbf{Declaração}\hfill]
\begin{ttfamily}
public Function GetCount{\_}InputCheckbox:Integer;\end{ttfamily}


\end{flushleft}
\par
\item[\textbf{Descrição}]
Construção da propriedade Count \begin{itemize}
\item Objetivo: Retorna o numero de items da lista onde os itens devem ser acessados com index 0 a count{-}1
\end{itemize}

\end{list}
\paragraph*{GetValue{\_}InputCheckbox}\hspace*{\fill}

\begin{list}{}{
\settowidth{\tmplength}{\textbf{Declaração}}
\setlength{\itemindent}{0cm}
\setlength{\listparindent}{0cm}
\setlength{\leftmargin}{\evensidemargin}
\addtolength{\leftmargin}{\tmplength}
\settowidth{\labelsep}{X}
\addtolength{\leftmargin}{\labelsep}
\setlength{\labelwidth}{\tmplength}
}
\begin{flushleft}
\item[\textbf{Declaração}\hfill]
\begin{ttfamily}
public Function GetValue{\_}InputCheckbox(aItem: Integer):AnsiString;\end{ttfamily}


\end{flushleft}
\par
\item[\textbf{Descrição}]
Construção da propriedade \begin{ttfamily}Value\end{ttfamily}(\ref{mi_rtl_ui_Dmxscroller.TDmxFieldRec-Value})

\begin{itemize}
\item Objetivo: Ler o label associado a opção ou trocar seu valor.
\item Sintaxe: Setando = \begin{ttfamily}Value\end{ttfamily}(\ref{mi_rtl_ui_Dmxscroller.TDmxFieldRec-Value})[1] = 'Sim'; \begin{ttfamily}Value\end{ttfamily}(\ref{mi_rtl_ui_Dmxscroller.TDmxFieldRec-Value})[2] = 'Nao'; \begin{ttfamily}Value\end{ttfamily}(\ref{mi_rtl_ui_Dmxscroller.TDmxFieldRec-Value})[1] = 'Yes' Lendo = If LowerCase(\begin{ttfamily}Value\end{ttfamily}(\ref{mi_rtl_ui_Dmxscroller.TDmxFieldRec-Value})[1]) = 'SIM' Then;
\end{itemize}

\end{list}
\paragraph*{SetValue{\_}InputCheckbox}\hspace*{\fill}

\begin{list}{}{
\settowidth{\tmplength}{\textbf{Declaração}}
\setlength{\itemindent}{0cm}
\setlength{\listparindent}{0cm}
\setlength{\leftmargin}{\evensidemargin}
\addtolength{\leftmargin}{\tmplength}
\settowidth{\labelsep}{X}
\addtolength{\leftmargin}{\labelsep}
\setlength{\labelwidth}{\tmplength}
}
\begin{flushleft}
\item[\textbf{Declaração}\hfill]
\begin{ttfamily}
public Procedure SetValue{\_}InputCheckbox(aItem: Integer;aValue:AnsiString);\end{ttfamily}


\end{flushleft}
\end{list}
\paragraph*{GetChecked{\_}InputCheckbox}\hspace*{\fill}

\begin{list}{}{
\settowidth{\tmplength}{\textbf{Declaração}}
\setlength{\itemindent}{0cm}
\setlength{\listparindent}{0cm}
\setlength{\leftmargin}{\evensidemargin}
\addtolength{\leftmargin}{\tmplength}
\settowidth{\labelsep}{X}
\addtolength{\leftmargin}{\labelsep}
\setlength{\labelwidth}{\tmplength}
}
\begin{flushleft}
\item[\textbf{Declaração}\hfill]
\begin{ttfamily}
public Function GetChecked{\_}InputCheckbox( aItem: Integer):Boolean;\end{ttfamily}


\end{flushleft}
\par
\item[\textbf{Descrição}]
Construção da propriedade Checked {-} Sintaxe: 1 = If Checked[1] then; 2 = Checked[1] := True.

\begin{itemize}
\item Objetivo: Selecionar um item da lista de opções ou checar se a opção está selecionada
\end{itemize}

\end{list}
\paragraph*{SetChecked{\_}InputCheckbox}\hspace*{\fill}

\begin{list}{}{
\settowidth{\tmplength}{\textbf{Declaração}}
\setlength{\itemindent}{0cm}
\setlength{\listparindent}{0cm}
\setlength{\leftmargin}{\evensidemargin}
\addtolength{\leftmargin}{\tmplength}
\settowidth{\labelsep}{X}
\addtolength{\leftmargin}{\labelsep}
\setlength{\labelwidth}{\tmplength}
}
\begin{flushleft}
\item[\textbf{Declaração}\hfill]
\begin{ttfamily}
public Procedure SetChecked{\_}InputCheckbox( aItem : Integer;aValue:Boolean);\end{ttfamily}


\end{flushleft}
\end{list}
\paragraph*{GetCount{\_}Select}\hspace*{\fill}

\begin{list}{}{
\settowidth{\tmplength}{\textbf{Declaração}}
\setlength{\itemindent}{0cm}
\setlength{\listparindent}{0cm}
\setlength{\leftmargin}{\evensidemargin}
\addtolength{\leftmargin}{\tmplength}
\settowidth{\labelsep}{X}
\addtolength{\leftmargin}{\labelsep}
\setlength{\labelwidth}{\tmplength}
}
\begin{flushleft}
\item[\textbf{Declaração}\hfill]
\begin{ttfamily}
public Function GetCount{\_}Select:Variant;\end{ttfamily}


\end{flushleft}
\par
\item[\textbf{Descrição}]
Construção da propriedade Count de campos enumerados \begin{itemize}
\item Objetivo: Retorna o numero de items da lista onde os itens devem ser acessados com index 0 a count{-}1
\end{itemize}

\end{list}
\paragraph*{GetSize{\_}Select}\hspace*{\fill}

\begin{list}{}{
\settowidth{\tmplength}{\textbf{Declaração}}
\setlength{\itemindent}{0cm}
\setlength{\listparindent}{0cm}
\setlength{\leftmargin}{\evensidemargin}
\addtolength{\leftmargin}{\tmplength}
\settowidth{\labelsep}{X}
\addtolength{\leftmargin}{\labelsep}
\setlength{\labelwidth}{\tmplength}
}
\begin{flushleft}
\item[\textbf{Declaração}\hfill]
\begin{ttfamily}
public Function GetSize{\_}Select():Variant;\end{ttfamily}


\end{flushleft}
\par
\item[\textbf{Descrição}]
Número de Linhas a ser mostrada no box. Usado em campos enumerados.

\end{list}
\paragraph*{GetValue{\_}Select}\hspace*{\fill}

\begin{list}{}{
\settowidth{\tmplength}{\textbf{Declaração}}
\setlength{\itemindent}{0cm}
\setlength{\listparindent}{0cm}
\setlength{\leftmargin}{\evensidemargin}
\addtolength{\leftmargin}{\tmplength}
\settowidth{\labelsep}{X}
\addtolength{\leftmargin}{\labelsep}
\setlength{\labelwidth}{\tmplength}
}
\begin{flushleft}
\item[\textbf{Declaração}\hfill]
\begin{ttfamily}
public Function GetValue{\_}Select(aItem: Integer):AnsiString;\end{ttfamily}


\end{flushleft}
\par
\item[\textbf{Descrição}]
Construção da propriedade \begin{ttfamily}Value\end{ttfamily}(\ref{mi_rtl_ui_Dmxscroller.TDmxFieldRec-Value}) \begin{itemize}
\item Objetivo: Ler o label associado a opção ou trocar seu valor.
\item Sintaxe: Setando = \begin{ttfamily}Value\end{ttfamily}(\ref{mi_rtl_ui_Dmxscroller.TDmxFieldRec-Value})[1] = 'Sim'; \begin{ttfamily}Value\end{ttfamily}(\ref{mi_rtl_ui_Dmxscroller.TDmxFieldRec-Value})[2] = 'Nao'; \begin{ttfamily}Value\end{ttfamily}(\ref{mi_rtl_ui_Dmxscroller.TDmxFieldRec-Value})[1] = 'Yes' Lendo = If LowerCase(\begin{ttfamily}Value\end{ttfamily}(\ref{mi_rtl_ui_Dmxscroller.TDmxFieldRec-Value})[1]) = 'SIM' Then;
\end{itemize}

\end{list}
\paragraph*{SetValue{\_}Select}\hspace*{\fill}

\begin{list}{}{
\settowidth{\tmplength}{\textbf{Declaração}}
\setlength{\itemindent}{0cm}
\setlength{\listparindent}{0cm}
\setlength{\leftmargin}{\evensidemargin}
\addtolength{\leftmargin}{\tmplength}
\settowidth{\labelsep}{X}
\addtolength{\leftmargin}{\labelsep}
\setlength{\labelwidth}{\tmplength}
}
\begin{flushleft}
\item[\textbf{Declaração}\hfill]
\begin{ttfamily}
public Procedure SetValue{\_}Select(aItem: Integer;aValue:AnsiString);\end{ttfamily}


\end{flushleft}
\end{list}
\paragraph*{GetChecked{\_}Select}\hspace*{\fill}

\begin{list}{}{
\settowidth{\tmplength}{\textbf{Declaração}}
\setlength{\itemindent}{0cm}
\setlength{\listparindent}{0cm}
\setlength{\leftmargin}{\evensidemargin}
\addtolength{\leftmargin}{\tmplength}
\settowidth{\labelsep}{X}
\addtolength{\leftmargin}{\labelsep}
\setlength{\labelwidth}{\tmplength}
}
\begin{flushleft}
\item[\textbf{Declaração}\hfill]
\begin{ttfamily}
public Function GetChecked{\_}Select( aItem: Integer):Boolean;\end{ttfamily}


\end{flushleft}
\par
\item[\textbf{Descrição}]
Construção da propriedade Checked {-} Sintaxe: 1 = If Checked[1] then; 2 = Checked[1] := True.

\begin{itemize}
\item Objetivo: Selecionar um item da lista de opções ou checar se a opção está selecionada
\end{itemize}

\end{list}
\paragraph*{SetChecked{\_}Select}\hspace*{\fill}

\begin{list}{}{
\settowidth{\tmplength}{\textbf{Declaração}}
\setlength{\itemindent}{0cm}
\setlength{\listparindent}{0cm}
\setlength{\leftmargin}{\evensidemargin}
\addtolength{\leftmargin}{\tmplength}
\settowidth{\labelsep}{X}
\addtolength{\leftmargin}{\labelsep}
\setlength{\labelwidth}{\tmplength}
}
\begin{flushleft}
\item[\textbf{Declaração}\hfill]
\begin{ttfamily}
public Procedure SetChecked{\_}Select( aItem : Integer;aValue:Boolean);\end{ttfamily}


\end{flushleft}
\end{list}
\paragraph*{IsNumber}\hspace*{\fill}

\begin{list}{}{
\settowidth{\tmplength}{\textbf{Declaração}}
\setlength{\itemindent}{0cm}
\setlength{\listparindent}{0cm}
\setlength{\leftmargin}{\evensidemargin}
\addtolength{\leftmargin}{\tmplength}
\settowidth{\labelsep}{X}
\addtolength{\leftmargin}{\labelsep}
\setlength{\labelwidth}{\tmplength}
}
\begin{flushleft}
\item[\textbf{Declaração}\hfill]
\begin{ttfamily}
public Function IsNumber:Boolean;\end{ttfamily}


\end{flushleft}
\par
\item[\textbf{Descrição}]
O método \textbf{\begin{ttfamily}IsNumber\end{ttfamily}} retorna true se o campo é numérico e false se alfanumérico

\end{list}
\paragraph*{IsNumberReal}\hspace*{\fill}

\begin{list}{}{
\settowidth{\tmplength}{\textbf{Declaração}}
\setlength{\itemindent}{0cm}
\setlength{\listparindent}{0cm}
\setlength{\leftmargin}{\evensidemargin}
\addtolength{\leftmargin}{\tmplength}
\settowidth{\labelsep}{X}
\addtolength{\leftmargin}{\labelsep}
\setlength{\labelwidth}{\tmplength}
}
\begin{flushleft}
\item[\textbf{Declaração}\hfill]
\begin{ttfamily}
public Function IsNumberReal:Boolean;\end{ttfamily}


\end{flushleft}
\end{list}
\paragraph*{IsNumberInteger}\hspace*{\fill}

\begin{list}{}{
\settowidth{\tmplength}{\textbf{Declaração}}
\setlength{\itemindent}{0cm}
\setlength{\listparindent}{0cm}
\setlength{\leftmargin}{\evensidemargin}
\addtolength{\leftmargin}{\tmplength}
\settowidth{\labelsep}{X}
\addtolength{\leftmargin}{\labelsep}
\setlength{\labelwidth}{\tmplength}
}
\begin{flushleft}
\item[\textbf{Declaração}\hfill]
\begin{ttfamily}
public Function IsNumberInteger:Boolean;\end{ttfamily}


\end{flushleft}
\end{list}
\paragraph*{IsData}\hspace*{\fill}

\begin{list}{}{
\settowidth{\tmplength}{\textbf{Declaração}}
\setlength{\itemindent}{0cm}
\setlength{\listparindent}{0cm}
\setlength{\leftmargin}{\evensidemargin}
\addtolength{\leftmargin}{\tmplength}
\settowidth{\labelsep}{X}
\addtolength{\leftmargin}{\labelsep}
\setlength{\labelwidth}{\tmplength}
}
\begin{flushleft}
\item[\textbf{Declaração}\hfill]
\begin{ttfamily}
public function IsData: Boolean;\end{ttfamily}


\end{flushleft}
\end{list}
\paragraph*{IsHora}\hspace*{\fill}

\begin{list}{}{
\settowidth{\tmplength}{\textbf{Declaração}}
\setlength{\itemindent}{0cm}
\setlength{\listparindent}{0cm}
\setlength{\leftmargin}{\evensidemargin}
\addtolength{\leftmargin}{\tmplength}
\settowidth{\labelsep}{X}
\addtolength{\leftmargin}{\labelsep}
\setlength{\labelwidth}{\tmplength}
}
\begin{flushleft}
\item[\textbf{Declaração}\hfill]
\begin{ttfamily}
public function IsHora: Boolean;\end{ttfamily}


\end{flushleft}
\end{list}
\paragraph*{GetLeft}\hspace*{\fill}

\begin{list}{}{
\settowidth{\tmplength}{\textbf{Declaração}}
\setlength{\itemindent}{0cm}
\setlength{\listparindent}{0cm}
\setlength{\leftmargin}{\evensidemargin}
\addtolength{\leftmargin}{\tmplength}
\settowidth{\labelsep}{X}
\addtolength{\leftmargin}{\labelsep}
\setlength{\labelwidth}{\tmplength}
}
\begin{flushleft}
\item[\textbf{Declaração}\hfill]
\begin{ttfamily}
public Function GetLeft:Integer;\end{ttfamily}


\end{flushleft}
\end{list}
\paragraph*{GetTop}\hspace*{\fill}

\begin{list}{}{
\settowidth{\tmplength}{\textbf{Declaração}}
\setlength{\itemindent}{0cm}
\setlength{\listparindent}{0cm}
\setlength{\leftmargin}{\evensidemargin}
\addtolength{\leftmargin}{\tmplength}
\settowidth{\labelsep}{X}
\addtolength{\leftmargin}{\labelsep}
\setlength{\labelwidth}{\tmplength}
}
\begin{flushleft}
\item[\textbf{Declaração}\hfill]
\begin{ttfamily}
public Function GetTop:Integer;\end{ttfamily}


\end{flushleft}
\end{list}
\paragraph*{GetWidth}\hspace*{\fill}

\begin{list}{}{
\settowidth{\tmplength}{\textbf{Declaração}}
\setlength{\itemindent}{0cm}
\setlength{\listparindent}{0cm}
\setlength{\leftmargin}{\evensidemargin}
\addtolength{\leftmargin}{\tmplength}
\settowidth{\labelsep}{X}
\addtolength{\leftmargin}{\labelsep}
\setlength{\labelwidth}{\tmplength}
}
\begin{flushleft}
\item[\textbf{Declaração}\hfill]
\begin{ttfamily}
public Function GetWidth:Integer;\end{ttfamily}


\end{flushleft}
\end{list}
\paragraph*{GetHeight}\hspace*{\fill}

\begin{list}{}{
\settowidth{\tmplength}{\textbf{Declaração}}
\setlength{\itemindent}{0cm}
\setlength{\listparindent}{0cm}
\setlength{\leftmargin}{\evensidemargin}
\addtolength{\leftmargin}{\tmplength}
\settowidth{\labelsep}{X}
\addtolength{\leftmargin}{\labelsep}
\setlength{\labelwidth}{\tmplength}
}
\begin{flushleft}
\item[\textbf{Declaração}\hfill]
\begin{ttfamily}
public Function GetHeight:Integer;\end{ttfamily}


\end{flushleft}
\end{list}
\paragraph*{SetAccess}\hspace*{\fill}

\begin{list}{}{
\settowidth{\tmplength}{\textbf{Declaração}}
\setlength{\itemindent}{0cm}
\setlength{\listparindent}{0cm}
\setlength{\leftmargin}{\evensidemargin}
\addtolength{\leftmargin}{\tmplength}
\settowidth{\labelsep}{X}
\addtolength{\leftmargin}{\labelsep}
\setlength{\labelwidth}{\tmplength}
}
\begin{flushleft}
\item[\textbf{Declaração}\hfill]
\begin{ttfamily}
public Function SetAccess(aaccess : byte):Byte;\end{ttfamily}


\end{flushleft}
\end{list}
\paragraph*{Valid}\hspace*{\fill}

\begin{list}{}{
\settowidth{\tmplength}{\textbf{Declaração}}
\setlength{\itemindent}{0cm}
\setlength{\listparindent}{0cm}
\setlength{\leftmargin}{\evensidemargin}
\addtolength{\leftmargin}{\tmplength}
\settowidth{\labelsep}{X}
\addtolength{\leftmargin}{\labelsep}
\setlength{\labelwidth}{\tmplength}
}
\begin{flushleft}
\item[\textbf{Declaração}\hfill]
\begin{ttfamily}
public function Valid(Command: Word): Boolean;\end{ttfamily}


\end{flushleft}
\end{list}
\paragraph*{DoOnEnter}\hspace*{\fill}

\begin{list}{}{
\settowidth{\tmplength}{\textbf{Declaração}}
\setlength{\itemindent}{0cm}
\setlength{\listparindent}{0cm}
\setlength{\leftmargin}{\evensidemargin}
\addtolength{\leftmargin}{\tmplength}
\settowidth{\labelsep}{X}
\addtolength{\leftmargin}{\labelsep}
\setlength{\labelwidth}{\tmplength}
}
\begin{flushleft}
\item[\textbf{Declaração}\hfill]
\begin{ttfamily}
public procedure DoOnEnter(Sender: TObject);\end{ttfamily}


\end{flushleft}
\par
\item[\textbf{Descrição}]
O método \textbf{\begin{ttfamily}DoOnEnter\end{ttfamily}} é executado toda vez antes do controle ler do buffer do campo. \begin{itemize}
\item Se o \begin{ttfamily}TUiDmxScroller.OnEnterField\end{ttfamily}(\ref{mi_rtl_ui_Dmxscroller.TUiDmxScroller-onEnterField}) tiver assinalado o método \textbf{\begin{ttfamily}DoOnEnter\end{ttfamily}} o executa.
\end{itemize}

\end{list}
\paragraph*{DoOnExit}\hspace*{\fill}

\begin{list}{}{
\settowidth{\tmplength}{\textbf{Declaração}}
\setlength{\itemindent}{0cm}
\setlength{\listparindent}{0cm}
\setlength{\leftmargin}{\evensidemargin}
\addtolength{\leftmargin}{\tmplength}
\settowidth{\labelsep}{X}
\addtolength{\leftmargin}{\labelsep}
\setlength{\labelwidth}{\tmplength}
}
\begin{flushleft}
\item[\textbf{Declaração}\hfill]
\begin{ttfamily}
public procedure DoOnExit(Sender: TObject);\end{ttfamily}


\end{flushleft}
\par
\item[\textbf{Descrição}]
O método \textbf{\begin{ttfamily}DoOnExit\end{ttfamily}} é executado toda vez antes do controle gravar no buffer do campo. \begin{itemize}
\item Se o \begin{ttfamily}TUiDmxScroller.OnExitField\end{ttfamily}(\ref{mi_rtl_ui_Dmxscroller.TUiDmxScroller-onExitField}) tiver assinalado o método \textbf{\begin{ttfamily}DoOnExit\end{ttfamily}} o executa.
\end{itemize}

\end{list}
\subsection*{TUiDmxScroller Classe}
\subsubsection*{\large{\textbf{Hierarquia}}\normalsize\hspace{1ex}\hfill}
TUiDmxScroller {$>$} \begin{ttfamily}TUiMethods\end{ttfamily}(\ref{mi_rtl_ui_methods.TUiMethods}) {$>$} 
TUiConsts
\subsubsection*{\large{\textbf{Descrição}}\normalsize\hspace{1ex}\hfill}
A classe \textbf{\begin{ttfamily}TUiDmxScroller\end{ttfamily}} tem como objetivo criar um formulário baseado em uma lista do tipo ShortString.

\begin{itemize}
\item \textbf{NOTAS} \begin{itemize}
\item O método \begin{ttfamily}createStruct\end{ttfamily}(\ref{mi_rtl_ui_Dmxscroller.TUiDmxScroller-CreateStruct}) criar uma lista de campo tipo \begin{ttfamily}TDmxFieldRec\end{ttfamily}(\ref{mi_rtl_ui_Dmxscroller.TDmxFieldRec}) com todas as informações necessárias para criar uma tabela ou um formulário.
\item O formulário é criado com apena uma linha.
\end{itemize}
\item \textbf{EXEMPLO}: \begin{itemize}
\item Template := '~Nome~{\textbackslash}SSSSSSSSSSSSSSSSSSSS ~Idade:~{\textbackslash}BB' \begin{itemize}
\item A classe cria a lista de campos: \begin{itemize}
\item Label1: Nome
\item Field1: campo ShortString com 20 posições maiúsculas
\item Label2: Idade
\item Field2: Campo byte com duas posições
\end{itemize}
\end{itemize}
\end{itemize}
\end{itemize}\subsubsection*{\large{\textbf{Propriedades}}\normalsize\hspace{1ex}\hfill}
\paragraph*{CurrentRecord}\hspace*{\fill}

\begin{list}{}{
\settowidth{\tmplength}{\textbf{Declaração}}
\setlength{\itemindent}{0cm}
\setlength{\listparindent}{0cm}
\setlength{\leftmargin}{\evensidemargin}
\addtolength{\leftmargin}{\tmplength}
\settowidth{\labelsep}{X}
\addtolength{\leftmargin}{\labelsep}
\setlength{\labelwidth}{\tmplength}
}
\begin{flushleft}
\item[\textbf{Declaração}\hfill]
\begin{ttfamily}
public property CurrentRecord : Longint read {\_}CurrentRecord write SetCurrentRecord;\end{ttfamily}


\end{flushleft}
\end{list}
\paragraph*{Strings}\hspace*{\fill}

\begin{list}{}{
\settowidth{\tmplength}{\textbf{Declaração}}
\setlength{\itemindent}{0cm}
\setlength{\listparindent}{0cm}
\setlength{\leftmargin}{\evensidemargin}
\addtolength{\leftmargin}{\tmplength}
\settowidth{\labelsep}{X}
\addtolength{\leftmargin}{\labelsep}
\setlength{\labelwidth}{\tmplength}
}
\begin{flushleft}
\item[\textbf{Declaração}\hfill]
\begin{ttfamily}
published property Strings : TMiStringList Read GetStrings   Write  SetStrings;\end{ttfamily}


\end{flushleft}
\par
\item[\textbf{Descrição}]
A propriedade \textbf{\begin{ttfamily}Strings\end{ttfamily}} o \begin{ttfamily}Strings\end{ttfamily} é usada para editar o Template em tempo de projeto.

\end{list}
\paragraph*{TableName}\hspace*{\fill}

\begin{list}{}{
\settowidth{\tmplength}{\textbf{Declaração}}
\setlength{\itemindent}{0cm}
\setlength{\listparindent}{0cm}
\setlength{\leftmargin}{\evensidemargin}
\addtolength{\leftmargin}{\tmplength}
\settowidth{\labelsep}{X}
\addtolength{\leftmargin}{\labelsep}
\setlength{\labelwidth}{\tmplength}
}
\begin{flushleft}
\item[\textbf{Declaração}\hfill]
\begin{ttfamily}
public property TableName : String Read {\_}TableName   Write  SetTableName;\end{ttfamily}


\end{flushleft}
\par
\item[\textbf{Descrição}]
A propriedade \textbf{\begin{ttfamily}TableName\end{ttfamily}} contém o nome da tabela ou consulta no banco de dados.

\begin{itemize}
\item \textbf{NOTA} \begin{itemize}
\item A propriedade \textbf{\begin{ttfamily}TableName\end{ttfamily}} é usado no método \textbf{SetSqlBufDataset} para criação da propriedade \textbf{TCustomSQLQuery.SQL} e no método \textbf{AlterTable} para criação da tabela ou consulta no banco de dados.
\end{itemize}
\end{itemize}

\end{list}
\paragraph*{Appending}\hspace*{\fill}

\begin{list}{}{
\settowidth{\tmplength}{\textbf{Declaração}}
\setlength{\itemindent}{0cm}
\setlength{\listparindent}{0cm}
\setlength{\leftmargin}{\evensidemargin}
\addtolength{\leftmargin}{\tmplength}
\settowidth{\labelsep}{X}
\addtolength{\leftmargin}{\labelsep}
\setlength{\labelwidth}{\tmplength}
}
\begin{flushleft}
\item[\textbf{Declaração}\hfill]
\begin{ttfamily}
public property Appending : Boolean read GetAppending write SetAppending;\end{ttfamily}


\end{flushleft}
\par
\item[\textbf{Descrição}]
A propriedade \textbf{\begin{ttfamily}Appending\end{ttfamily}} é usada para saber se está editando um novo registro ou não

\begin{itemize}
\item \textbf{NOTA} \begin{itemize}
\item TRUE = Indica que um novo registro esta sendo editado.
\item False = Indica que um registro existente está sendo editado ou visualizado.
\item Obs: Deve ser atualizado na visão caso a tabela está em edição.
\end{itemize}
\end{itemize}

\end{list}
\paragraph*{DoOnNewRecord{\_}FillChar}\hspace*{\fill}

\begin{list}{}{
\settowidth{\tmplength}{\textbf{Declaração}}
\setlength{\itemindent}{0cm}
\setlength{\listparindent}{0cm}
\setlength{\leftmargin}{\evensidemargin}
\addtolength{\leftmargin}{\tmplength}
\settowidth{\labelsep}{X}
\addtolength{\leftmargin}{\labelsep}
\setlength{\labelwidth}{\tmplength}
}
\begin{flushleft}
\item[\textbf{Declaração}\hfill]
\begin{ttfamily}
public property DoOnNewRecord{\_}FillChar : Boolean Read {\_}DoOnNewRecord{\_}FillChar Write SetDoOnNewRecord{\_}FillChar default True;\end{ttfamily}


\end{flushleft}
\end{list}
\paragraph*{RecordSelected}\hspace*{\fill}

\begin{list}{}{
\settowidth{\tmplength}{\textbf{Declaração}}
\setlength{\itemindent}{0cm}
\setlength{\listparindent}{0cm}
\setlength{\leftmargin}{\evensidemargin}
\addtolength{\leftmargin}{\tmplength}
\settowidth{\labelsep}{X}
\addtolength{\leftmargin}{\labelsep}
\setlength{\labelwidth}{\tmplength}
}
\begin{flushleft}
\item[\textbf{Declaração}\hfill]
\begin{ttfamily}
protected property RecordSelected  : boolean read GetRecordSelected  Write SetRecordSelected default false;\end{ttfamily}


\end{flushleft}
\end{list}
\paragraph*{OnNewRecord}\hspace*{\fill}

\begin{list}{}{
\settowidth{\tmplength}{\textbf{Declaração}}
\setlength{\itemindent}{0cm}
\setlength{\listparindent}{0cm}
\setlength{\leftmargin}{\evensidemargin}
\addtolength{\leftmargin}{\tmplength}
\settowidth{\labelsep}{X}
\addtolength{\leftmargin}{\labelsep}
\setlength{\labelwidth}{\tmplength}
}
\begin{flushleft}
\item[\textbf{Declaração}\hfill]
\begin{ttfamily}
public property OnNewRecord  : TOnNewRecord read {\_}OnNewRecord  Write {\_}OnNewRecord;\end{ttfamily}


\end{flushleft}
\par
\item[\textbf{Descrição}]
A propriedade \textbf{\begin{ttfamily}OnNewRecord\end{ttfamily}} é executada em \begin{ttfamily}DoOnNewRecord\end{ttfamily}(\ref{mi_rtl_ui_Dmxscroller.TUiDmxScroller-DoOnNewRecord}) se a mesma for assinalada.

\end{list}
\paragraph*{onCloseQuery}\hspace*{\fill}

\begin{list}{}{
\settowidth{\tmplength}{\textbf{Declaração}}
\setlength{\itemindent}{0cm}
\setlength{\listparindent}{0cm}
\setlength{\leftmargin}{\evensidemargin}
\addtolength{\leftmargin}{\tmplength}
\settowidth{\labelsep}{X}
\addtolength{\leftmargin}{\labelsep}
\setlength{\labelwidth}{\tmplength}
}
\begin{flushleft}
\item[\textbf{Declaração}\hfill]
\begin{ttfamily}
public property onCloseQuery : TOnCloseQuery Read {\_}OnCloseQuery write {\_}onCloseQuery;\end{ttfamily}


\end{flushleft}
\par
\item[\textbf{Descrição}]
O evento \textbf{\begin{ttfamily}onCloseQuery\end{ttfamily}} é disparado toda vez que o \begin{ttfamily}TUiDmxScroller\end{ttfamily}(\ref{mi_rtl_ui_Dmxscroller.TUiDmxScroller}) é desativado.

\begin{itemize}
\item \textbf{NOTA* \begin{itemize}
\item Este evento é disparado antes de desativar a classe **\begin{ttfamily}TUiDmxScroller\end{ttfamily}(\ref{mi_rtl_ui_Dmxscroller.TUiDmxScroller})
\end{itemize}}. \begin{itemize}
\item Obs: Se o parâmetro \textbf{CanClose} for \textbf{false}, então o formulário não é desativado.
\end{itemize}
\end{itemize}

\end{list}
\paragraph*{onEnter}\hspace*{\fill}

\begin{list}{}{
\settowidth{\tmplength}{\textbf{Declaração}}
\setlength{\itemindent}{0cm}
\setlength{\listparindent}{0cm}
\setlength{\leftmargin}{\evensidemargin}
\addtolength{\leftmargin}{\tmplength}
\settowidth{\labelsep}{X}
\addtolength{\leftmargin}{\labelsep}
\setlength{\labelwidth}{\tmplength}
}
\begin{flushleft}
\item[\textbf{Declaração}\hfill]
\begin{ttfamily}
public property onEnter : TOnEnter Read {\_}OnEnter write {\_}onEnter;\end{ttfamily}


\end{flushleft}
\par
\item[\textbf{Descrição}]
O evento \textbf{\begin{ttfamily}onEnter\end{ttfamily}} é disparado toda vez que o \begin{ttfamily}TUiDmxScroller\end{ttfamily}(\ref{mi_rtl_ui_Dmxscroller.TUiDmxScroller}) ativado.

\end{list}
\paragraph*{onExit}\hspace*{\fill}

\begin{list}{}{
\settowidth{\tmplength}{\textbf{Declaração}}
\setlength{\itemindent}{0cm}
\setlength{\listparindent}{0cm}
\setlength{\leftmargin}{\evensidemargin}
\addtolength{\leftmargin}{\tmplength}
\settowidth{\labelsep}{X}
\addtolength{\leftmargin}{\labelsep}
\setlength{\labelwidth}{\tmplength}
}
\begin{flushleft}
\item[\textbf{Declaração}\hfill]
\begin{ttfamily}
public property onExit : TOnExit Read {\_}OnExit write {\_}onExit;\end{ttfamily}


\end{flushleft}
\par
\item[\textbf{Descrição}]
O evento \textbf{\begin{ttfamily}onExit\end{ttfamily}} é disparado toda vez que o \begin{ttfamily}TUiDmxScroller\end{ttfamily}(\ref{mi_rtl_ui_Dmxscroller.TUiDmxScroller}) é destivado.

\end{list}
\paragraph*{onEnterField}\hspace*{\fill}

\begin{list}{}{
\settowidth{\tmplength}{\textbf{Declaração}}
\setlength{\itemindent}{0cm}
\setlength{\listparindent}{0cm}
\setlength{\leftmargin}{\evensidemargin}
\addtolength{\leftmargin}{\tmplength}
\settowidth{\labelsep}{X}
\addtolength{\leftmargin}{\labelsep}
\setlength{\labelwidth}{\tmplength}
}
\begin{flushleft}
\item[\textbf{Declaração}\hfill]
\begin{ttfamily}
public property onEnterField : TOnEnterField Read {\_}OnEnterField write {\_}onEnterField;\end{ttfamily}


\end{flushleft}
\par
\item[\textbf{Descrição}]
O evento \textbf{\begin{ttfamily}onEnterField\end{ttfamily}} é disparado toda vez que o controle corrente recebe o foco.

\end{list}
\paragraph*{onExitField}\hspace*{\fill}

\begin{list}{}{
\settowidth{\tmplength}{\textbf{Declaração}}
\setlength{\itemindent}{0cm}
\setlength{\listparindent}{0cm}
\setlength{\leftmargin}{\evensidemargin}
\addtolength{\leftmargin}{\tmplength}
\settowidth{\labelsep}{X}
\addtolength{\leftmargin}{\labelsep}
\setlength{\labelwidth}{\tmplength}
}
\begin{flushleft}
\item[\textbf{Declaração}\hfill]
\begin{ttfamily}
public property onExitField : TOnExitField Read {\_}OnExitField write {\_}onExitField;\end{ttfamily}


\end{flushleft}
\par
\item[\textbf{Descrição}]
O evento \textbf{\begin{ttfamily}onExitField\end{ttfamily}} é disparado toda vez que o controle corrente perde o foco.

\end{list}
\paragraph*{onGetTemplate}\hspace*{\fill}

\begin{list}{}{
\settowidth{\tmplength}{\textbf{Declaração}}
\setlength{\itemindent}{0cm}
\setlength{\listparindent}{0cm}
\setlength{\leftmargin}{\evensidemargin}
\addtolength{\leftmargin}{\tmplength}
\settowidth{\labelsep}{X}
\addtolength{\leftmargin}{\labelsep}
\setlength{\labelwidth}{\tmplength}
}
\begin{flushleft}
\item[\textbf{Declaração}\hfill]
\begin{ttfamily}
public property onGetTemplate : TonGetTemplate Read {\_}onGetTemplate   Write  {\_}onGetTemplate;\end{ttfamily}


\end{flushleft}
\par
\item[\textbf{Descrição}]
O evento \textbf{\begin{ttfamily}onGetTemplate\end{ttfamily}} substitui o método \begin{ttfamily}getTemplate\end{ttfamily}(\ref{mi_rtl_ui_Dmxscroller.TUiDmxScroller-GetTemplate}) caso \begin{ttfamily}OnGetTemplate\end{ttfamily}{$<$}{$>$}nil

\end{list}
\paragraph*{onAddTemplate}\hspace*{\fill}

\begin{list}{}{
\settowidth{\tmplength}{\textbf{Declaração}}
\setlength{\itemindent}{0cm}
\setlength{\listparindent}{0cm}
\setlength{\leftmargin}{\evensidemargin}
\addtolength{\leftmargin}{\tmplength}
\settowidth{\labelsep}{X}
\addtolength{\leftmargin}{\labelsep}
\setlength{\labelwidth}{\tmplength}
}
\begin{flushleft}
\item[\textbf{Declaração}\hfill]
\begin{ttfamily}
public property onAddTemplate: TonAddTemplate read {\_}onAddTemplate write {\_}onAddTemplate;\end{ttfamily}


\end{flushleft}
\end{list}
\paragraph*{Active}\hspace*{\fill}

\begin{list}{}{
\settowidth{\tmplength}{\textbf{Declaração}}
\setlength{\itemindent}{0cm}
\setlength{\listparindent}{0cm}
\setlength{\leftmargin}{\evensidemargin}
\addtolength{\leftmargin}{\tmplength}
\settowidth{\labelsep}{X}
\addtolength{\leftmargin}{\labelsep}
\setlength{\labelwidth}{\tmplength}
}
\begin{flushleft}
\item[\textbf{Declaração}\hfill]
\begin{ttfamily}
public property Active : Boolean Read {\_}Active Write SetActive;\end{ttfamily}


\end{flushleft}
\end{list}
\paragraph*{CurrentField}\hspace*{\fill}

\begin{list}{}{
\settowidth{\tmplength}{\textbf{Declaração}}
\setlength{\itemindent}{0cm}
\setlength{\listparindent}{0cm}
\setlength{\leftmargin}{\evensidemargin}
\addtolength{\leftmargin}{\tmplength}
\settowidth{\labelsep}{X}
\addtolength{\leftmargin}{\labelsep}
\setlength{\labelwidth}{\tmplength}
}
\begin{flushleft}
\item[\textbf{Declaração}\hfill]
\begin{ttfamily}
public property CurrentField : pDmxFieldRec  read {\_}CurrentField write SetCurrentField;\end{ttfamily}


\end{flushleft}
\par
\item[\textbf{Descrição}]
A propriedade \textbf{\begin{ttfamily}CurrentField\end{ttfamily}} contem um ponteiro para o campo selecionado

\end{list}
\paragraph*{AlignmentLabels}\hspace*{\fill}

\begin{list}{}{
\settowidth{\tmplength}{\textbf{Declaração}}
\setlength{\itemindent}{0cm}
\setlength{\listparindent}{0cm}
\setlength{\leftmargin}{\evensidemargin}
\addtolength{\leftmargin}{\tmplength}
\settowidth{\labelsep}{X}
\addtolength{\leftmargin}{\labelsep}
\setlength{\labelwidth}{\tmplength}
}
\begin{flushleft}
\item[\textbf{Declaração}\hfill]
\begin{ttfamily}
public property AlignmentLabels : TAlignment read {\_}AlignmentLabels write {\_}AlignmentLabels;\end{ttfamily}


\end{flushleft}
\end{list}
\paragraph*{BufDataset}\hspace*{\fill}

\begin{list}{}{
\settowidth{\tmplength}{\textbf{Declaração}}
\setlength{\itemindent}{0cm}
\setlength{\listparindent}{0cm}
\setlength{\leftmargin}{\evensidemargin}
\addtolength{\leftmargin}{\tmplength}
\settowidth{\labelsep}{X}
\addtolength{\leftmargin}{\labelsep}
\setlength{\labelwidth}{\tmplength}
}
\begin{flushleft}
\item[\textbf{Declaração}\hfill]
\begin{ttfamily}
protected property BufDataset : TBufDataset read GetBufDataset write SetBufDataset;\end{ttfamily}


\end{flushleft}
\par
\item[\textbf{Descrição}]
A propriedade \textbf{\begin{ttfamily}BufDataset\end{ttfamily}} é usada com objetivo de integração dos dados do formulário TVDmx e os controle decentes de TDataSet.

\end{list}
\paragraph*{DataSource}\hspace*{\fill}

\begin{list}{}{
\settowidth{\tmplength}{\textbf{Declaração}}
\setlength{\itemindent}{0cm}
\setlength{\listparindent}{0cm}
\setlength{\leftmargin}{\evensidemargin}
\addtolength{\leftmargin}{\tmplength}
\settowidth{\labelsep}{X}
\addtolength{\leftmargin}{\labelsep}
\setlength{\labelwidth}{\tmplength}
}
\begin{flushleft}
\item[\textbf{Declaração}\hfill]
\begin{ttfamily}
public property DataSource : TDataSource Read {\_}DataSource   Write  {\_}DataSource;\end{ttfamily}


\end{flushleft}
\par
\item[\textbf{Descrição}]
A propriedade \textbf{\begin{ttfamily}DataSource\end{ttfamily}} permite que controles da \textbf{LCL} (Lazarus Componentes Library) possam usar os dados do componente \textbf{TDmxScroller}.

\begin{itemize}
\item \textbf{NOTA} \begin{itemize}
\item Essa integração permite que \textbf{TDmxScroller} utilize todos os componentes de banco de dados do Free Pascal.
\end{itemize}
\end{itemize}

\end{list}
\paragraph*{Locked}\hspace*{\fill}

\begin{list}{}{
\settowidth{\tmplength}{\textbf{Declaração}}
\setlength{\itemindent}{0cm}
\setlength{\listparindent}{0cm}
\setlength{\leftmargin}{\evensidemargin}
\addtolength{\leftmargin}{\tmplength}
\settowidth{\labelsep}{X}
\addtolength{\leftmargin}{\labelsep}
\setlength{\labelwidth}{\tmplength}
}
\begin{flushleft}
\item[\textbf{Declaração}\hfill]
\begin{ttfamily}
public property Locked : Boolean read {\_}Locked write SetLocked;\end{ttfamily}


\end{flushleft}
\par
\item[\textbf{Descrição}]
A propriedade \textbf{\begin{ttfamily}Locked\end{ttfamily}} deve ser redefinida na interface filha desta classe para travar o formulário se aLocked = true e destravar se aLocked = false;

\end{list}
\subsubsection*{\large{\textbf{Campos}}\normalsize\hspace{1ex}\hfill}
\paragraph*{Fields}\hspace*{\fill}

\begin{list}{}{
\settowidth{\tmplength}{\textbf{Declaração}}
\setlength{\itemindent}{0cm}
\setlength{\listparindent}{0cm}
\setlength{\leftmargin}{\evensidemargin}
\addtolength{\leftmargin}{\tmplength}
\settowidth{\labelsep}{X}
\addtolength{\leftmargin}{\labelsep}
\setlength{\labelwidth}{\tmplength}
}
\begin{flushleft}
\item[\textbf{Declaração}\hfill]
\begin{ttfamily}
public Fields: TFPList;\end{ttfamily}


\end{flushleft}
\par
\item[\textbf{Descrição}]
O atributo \textbf{\begin{ttfamily}Fields\end{ttfamily}} contém uma lista \begin{ttfamily}pDmxFieldRec\end{ttfamily}(\ref{mi_rtl_ui_Dmxscroller-pDmxFieldRec}) cujo \textbf{Fieldnum{$<$}{$>$}0}.

\begin{itemize}
\item Essa lista é atualizada em \begin{ttfamily}createStruct\end{ttfamily}(\ref{mi_rtl_ui_Dmxscroller.TUiDmxScroller-CreateStruct})
\end{itemize}

\end{list}
\paragraph*{Limit}\hspace*{\fill}

\begin{list}{}{
\settowidth{\tmplength}{\textbf{Declaração}}
\setlength{\itemindent}{0cm}
\setlength{\listparindent}{0cm}
\setlength{\leftmargin}{\evensidemargin}
\addtolength{\leftmargin}{\tmplength}
\settowidth{\labelsep}{X}
\addtolength{\leftmargin}{\labelsep}
\setlength{\labelwidth}{\tmplength}
}
\begin{flushleft}
\item[\textbf{Declaração}\hfill]
\begin{ttfamily}
public Limit: TPoint;\end{ttfamily}


\end{flushleft}
\end{list}
\paragraph*{CreateValid}\hspace*{\fill}

\begin{list}{}{
\settowidth{\tmplength}{\textbf{Declaração}}
\setlength{\itemindent}{0cm}
\setlength{\listparindent}{0cm}
\setlength{\leftmargin}{\evensidemargin}
\addtolength{\leftmargin}{\tmplength}
\settowidth{\labelsep}{X}
\addtolength{\leftmargin}{\labelsep}
\setlength{\labelwidth}{\tmplength}
}
\begin{flushleft}
\item[\textbf{Declaração}\hfill]
\begin{ttfamily}
public CreateValid: boolean ;\end{ttfamily}


\end{flushleft}
\par
\item[\textbf{Descrição}]
Deve ser true ao criar a classe

\end{list}
\paragraph*{WorkingData}\hspace*{\fill}

\begin{list}{}{
\settowidth{\tmplength}{\textbf{Declaração}}
\setlength{\itemindent}{0cm}
\setlength{\listparindent}{0cm}
\setlength{\leftmargin}{\evensidemargin}
\addtolength{\leftmargin}{\tmplength}
\settowidth{\labelsep}{X}
\addtolength{\leftmargin}{\labelsep}
\setlength{\labelwidth}{\tmplength}
}
\begin{flushleft}
\item[\textbf{Declaração}\hfill]
\begin{ttfamily}
protected WorkingData: pointer;\end{ttfamily}


\end{flushleft}
\end{list}
\paragraph*{WorkingDataOld}\hspace*{\fill}

\begin{list}{}{
\settowidth{\tmplength}{\textbf{Declaração}}
\setlength{\itemindent}{0cm}
\setlength{\listparindent}{0cm}
\setlength{\leftmargin}{\evensidemargin}
\addtolength{\leftmargin}{\tmplength}
\settowidth{\labelsep}{X}
\addtolength{\leftmargin}{\labelsep}
\setlength{\labelwidth}{\tmplength}
}
\begin{flushleft}
\item[\textbf{Declaração}\hfill]
\begin{ttfamily}
protected WorkingDataOld: pointer;\end{ttfamily}


\end{flushleft}
\end{list}
\paragraph*{DataBlockSize}\hspace*{\fill}

\begin{list}{}{
\settowidth{\tmplength}{\textbf{Declaração}}
\setlength{\itemindent}{0cm}
\setlength{\listparindent}{0cm}
\setlength{\leftmargin}{\evensidemargin}
\addtolength{\leftmargin}{\tmplength}
\settowidth{\labelsep}{X}
\addtolength{\leftmargin}{\labelsep}
\setlength{\labelwidth}{\tmplength}
}
\begin{flushleft}
\item[\textbf{Declaração}\hfill]
\begin{ttfamily}
protected DataBlockSize: longint;\end{ttfamily}


\end{flushleft}
\end{list}
\paragraph*{ActualRecordNum}\hspace*{\fill}

\begin{list}{}{
\settowidth{\tmplength}{\textbf{Declaração}}
\setlength{\itemindent}{0cm}
\setlength{\listparindent}{0cm}
\setlength{\leftmargin}{\evensidemargin}
\addtolength{\leftmargin}{\tmplength}
\settowidth{\labelsep}{X}
\addtolength{\leftmargin}{\labelsep}
\setlength{\labelwidth}{\tmplength}
}
\begin{flushleft}
\item[\textbf{Declaração}\hfill]
\begin{ttfamily}
public ActualRecordNum: longint;\end{ttfamily}


\end{flushleft}
\end{list}
\paragraph*{DMXField1}\hspace*{\fill}

\begin{list}{}{
\settowidth{\tmplength}{\textbf{Declaração}}
\setlength{\itemindent}{0cm}
\setlength{\listparindent}{0cm}
\setlength{\leftmargin}{\evensidemargin}
\addtolength{\leftmargin}{\tmplength}
\settowidth{\labelsep}{X}
\addtolength{\leftmargin}{\labelsep}
\setlength{\labelwidth}{\tmplength}
}
\begin{flushleft}
\item[\textbf{Declaração}\hfill]
\begin{ttfamily}
public DMXField1: pDmxFieldRec;\end{ttfamily}


\end{flushleft}
\par
\item[\textbf{Descrição}]
O atributo \textbf{\begin{ttfamily}DMXField1\end{ttfamily}} contém o primeiro campo da lista encandeada

\end{list}
\paragraph*{TotalFields}\hspace*{\fill}

\begin{list}{}{
\settowidth{\tmplength}{\textbf{Declaração}}
\setlength{\itemindent}{0cm}
\setlength{\listparindent}{0cm}
\setlength{\leftmargin}{\evensidemargin}
\addtolength{\leftmargin}{\tmplength}
\settowidth{\labelsep}{X}
\addtolength{\leftmargin}{\labelsep}
\setlength{\labelwidth}{\tmplength}
}
\begin{flushleft}
\item[\textbf{Declaração}\hfill]
\begin{ttfamily}
public TotalFields: integer;\end{ttfamily}


\end{flushleft}
\par
\item[\textbf{Descrição}]
O atributo \textbf{\begin{ttfamily}TotalFields\end{ttfamily}} contém o total de campos da lista apontada por \begin{ttfamily}DMXField1\end{ttfamily}(\ref{mi_rtl_ui_Dmxscroller.TUiDmxScroller-DMXField1})

\end{list}
\paragraph*{RecordSize}\hspace*{\fill}

\begin{list}{}{
\settowidth{\tmplength}{\textbf{Declaração}}
\setlength{\itemindent}{0cm}
\setlength{\listparindent}{0cm}
\setlength{\leftmargin}{\evensidemargin}
\addtolength{\leftmargin}{\tmplength}
\settowidth{\labelsep}{X}
\addtolength{\leftmargin}{\labelsep}
\setlength{\labelwidth}{\tmplength}
}
\begin{flushleft}
\item[\textbf{Declaração}\hfill]
\begin{ttfamily}
public RecordSize: integer;\end{ttfamily}


\end{flushleft}
\par
\item[\textbf{Descrição}]
O atributo \textbf{\begin{ttfamily}RecordSize\end{ttfamily}} contém o tamanho do buffer calculado por \begin{ttfamily}CreateStruct\end{ttfamily}(\ref{mi_rtl_ui_Dmxscroller.TUiDmxScroller-CreateStruct})

\end{list}
\paragraph*{FieldData}\hspace*{\fill}

\begin{list}{}{
\settowidth{\tmplength}{\textbf{Declaração}}
\setlength{\itemindent}{0cm}
\setlength{\listparindent}{0cm}
\setlength{\leftmargin}{\evensidemargin}
\addtolength{\leftmargin}{\tmplength}
\settowidth{\labelsep}{X}
\addtolength{\leftmargin}{\labelsep}
\setlength{\labelwidth}{\tmplength}
}
\begin{flushleft}
\item[\textbf{Declaração}\hfill]
\begin{ttfamily}
public FieldData: pointer;\end{ttfamily}


\end{flushleft}
\par
\item[\textbf{Descrição}]
O atributo \textbf{\begin{ttfamily}FieldData\end{ttfamily}} contém o ponteiro do buffer do corrente campo calculado pela propriedade \begin{ttfamily}CurrentField\end{ttfamily}(\ref{mi_rtl_ui_Dmxscroller.TUiDmxScroller-CurrentField})

\end{list}
\paragraph*{RecordData}\hspace*{\fill}

\begin{list}{}{
\settowidth{\tmplength}{\textbf{Declaração}}
\setlength{\itemindent}{0cm}
\setlength{\listparindent}{0cm}
\setlength{\leftmargin}{\evensidemargin}
\addtolength{\leftmargin}{\tmplength}
\settowidth{\labelsep}{X}
\addtolength{\leftmargin}{\labelsep}
\setlength{\labelwidth}{\tmplength}
}
\begin{flushleft}
\item[\textbf{Declaração}\hfill]
\begin{ttfamily}
public RecordData: pointer;\end{ttfamily}


\end{flushleft}
\end{list}
\paragraph*{WidthChar}\hspace*{\fill}

\begin{list}{}{
\settowidth{\tmplength}{\textbf{Declaração}}
\setlength{\itemindent}{0cm}
\setlength{\listparindent}{0cm}
\setlength{\leftmargin}{\evensidemargin}
\addtolength{\leftmargin}{\tmplength}
\settowidth{\labelsep}{X}
\addtolength{\leftmargin}{\labelsep}
\setlength{\labelwidth}{\tmplength}
}
\begin{flushleft}
\item[\textbf{Declaração}\hfill]
\begin{ttfamily}
public WidthChar:byte;\end{ttfamily}


\end{flushleft}
\par
\item[\textbf{Descrição}]
O atributo \textbf{\begin{ttfamily}WidthChar\end{ttfamily}} deve ser iniciado quando este controle for incluído em um TScrollingWinControl. em um controle gráfico. \begin{itemize}
\item \textbf{EXEMPLO} \begin{itemize}
\item \begin{ttfamily}WidthChar\end{ttfamily} := ((Owner as TScrollingWinControl).Canvas.TextWidth(UiDmxScroller.CharAlfanumeric) div Length(UiDmxScroller.CharAlfanumeric));
\end{itemize}
\end{itemize}

\end{list}
\paragraph*{HeightChar}\hspace*{\fill}

\begin{list}{}{
\settowidth{\tmplength}{\textbf{Declaração}}
\setlength{\itemindent}{0cm}
\setlength{\listparindent}{0cm}
\setlength{\leftmargin}{\evensidemargin}
\addtolength{\leftmargin}{\tmplength}
\settowidth{\labelsep}{X}
\addtolength{\leftmargin}{\labelsep}
\setlength{\labelwidth}{\tmplength}
}
\begin{flushleft}
\item[\textbf{Declaração}\hfill]
\begin{ttfamily}
public HeightChar:byte;\end{ttfamily}


\end{flushleft}
\par
\item[\textbf{Descrição}]
O atributo \textbf{\begin{ttfamily}HeightChar\end{ttfamily}} deve ser iniciado quando este controle for incluído em um TScrollingWinControl. \begin{itemize}
\item \textbf{EXEMPLO} \begin{itemize}
\item \begin{ttfamily}HeightChar\end{ttfamily} := (Owner as TScrollingWinControl).Canvas.TextHeight(CharAlfanumeric)+2;
\end{itemize}
\end{itemize}

\end{list}
\paragraph*{KeyAltered}\hspace*{\fill}

\begin{list}{}{
\settowidth{\tmplength}{\textbf{Declaração}}
\setlength{\itemindent}{0cm}
\setlength{\listparindent}{0cm}
\setlength{\leftmargin}{\evensidemargin}
\addtolength{\leftmargin}{\tmplength}
\settowidth{\labelsep}{X}
\addtolength{\leftmargin}{\labelsep}
\setlength{\labelwidth}{\tmplength}
}
\begin{flushleft}
\item[\textbf{Declaração}\hfill]
\begin{ttfamily}
protected KeyAltered: Boolean ;\end{ttfamily}


\end{flushleft}
\par
\item[\textbf{Descrição}]
O atributo \textbf{\begin{ttfamily}KeyAltered\end{ttfamily}} indica se algum campo da chave foi alterado é setado em changeMade

\end{list}
\paragraph*{keysPrimaryKeyComposite}\hspace*{\fill}

\begin{list}{}{
\settowidth{\tmplength}{\textbf{Declaração}}
\setlength{\itemindent}{0cm}
\setlength{\listparindent}{0cm}
\setlength{\leftmargin}{\evensidemargin}
\addtolength{\leftmargin}{\tmplength}
\settowidth{\labelsep}{X}
\addtolength{\leftmargin}{\labelsep}
\setlength{\labelwidth}{\tmplength}
}
\begin{flushleft}
\item[\textbf{Declaração}\hfill]
\begin{ttfamily}
protected keysPrimaryKeyComposite: AnsiString;\end{ttfamily}


\end{flushleft}
\par
\item[\textbf{Descrição}]
O atributo \textbf{\begin{ttfamily}keysPrimaryKeyComposite\end{ttfamily}} contém a lista de campos que pertence a chave primária \begin{itemize}
\item \textbf{EXEMPLOS}: \begin{itemize}
\item \textbf{Chave simples}: \begin{itemize}
\item 'Matricula'.
\end{itemize}
\item \textbf{Chave composta}: \begin{itemize}
\item 'Estado,Cidade'.
\end{itemize}
\end{itemize}
\item \textbf{NOTA} \begin{itemize}
\item Se pos(',',\begin{ttfamily}keysPrimaryKeyComposite\end{ttfamily}) {$<$}{$>$} 0 indica que a chave é composta.
\end{itemize}
\end{itemize}

\end{list}
\paragraph*{flagPrimaryKey{\_}AutoIncrement}\hspace*{\fill}

\begin{list}{}{
\settowidth{\tmplength}{\textbf{Declaração}}
\setlength{\itemindent}{0cm}
\setlength{\listparindent}{0cm}
\setlength{\leftmargin}{\evensidemargin}
\addtolength{\leftmargin}{\tmplength}
\settowidth{\labelsep}{X}
\addtolength{\leftmargin}{\labelsep}
\setlength{\labelwidth}{\tmplength}
}
\begin{flushleft}
\item[\textbf{Declaração}\hfill]
\begin{ttfamily}
protected flagPrimaryKey{\_}AutoIncrement:Boolean;\end{ttfamily}


\end{flushleft}
\end{list}
\paragraph*{{\_}OnCloseQuery}\hspace*{\fill}

\begin{list}{}{
\settowidth{\tmplength}{\textbf{Declaração}}
\setlength{\itemindent}{0cm}
\setlength{\listparindent}{0cm}
\setlength{\leftmargin}{\evensidemargin}
\addtolength{\leftmargin}{\tmplength}
\settowidth{\labelsep}{X}
\addtolength{\leftmargin}{\labelsep}
\setlength{\labelwidth}{\tmplength}
}
\begin{flushleft}
\item[\textbf{Declaração}\hfill]
\begin{ttfamily}
protected {\_}OnCloseQuery: TOnCloseQuery;\end{ttfamily}


\end{flushleft}
\end{list}
\paragraph*{{\_}OnEnter}\hspace*{\fill}

\begin{list}{}{
\settowidth{\tmplength}{\textbf{Declaração}}
\setlength{\itemindent}{0cm}
\setlength{\listparindent}{0cm}
\setlength{\leftmargin}{\evensidemargin}
\addtolength{\leftmargin}{\tmplength}
\settowidth{\labelsep}{X}
\addtolength{\leftmargin}{\labelsep}
\setlength{\labelwidth}{\tmplength}
}
\begin{flushleft}
\item[\textbf{Declaração}\hfill]
\begin{ttfamily}
protected {\_}OnEnter: TOnEnter;\end{ttfamily}


\end{flushleft}
\end{list}
\paragraph*{{\_}OnExit}\hspace*{\fill}

\begin{list}{}{
\settowidth{\tmplength}{\textbf{Declaração}}
\setlength{\itemindent}{0cm}
\setlength{\listparindent}{0cm}
\setlength{\leftmargin}{\evensidemargin}
\addtolength{\leftmargin}{\tmplength}
\settowidth{\labelsep}{X}
\addtolength{\leftmargin}{\labelsep}
\setlength{\labelwidth}{\tmplength}
}
\begin{flushleft}
\item[\textbf{Declaração}\hfill]
\begin{ttfamily}
protected {\_}OnExit: TOnExit;\end{ttfamily}


\end{flushleft}
\end{list}
\paragraph*{{\_}OnEnterField}\hspace*{\fill}

\begin{list}{}{
\settowidth{\tmplength}{\textbf{Declaração}}
\setlength{\itemindent}{0cm}
\setlength{\listparindent}{0cm}
\setlength{\leftmargin}{\evensidemargin}
\addtolength{\leftmargin}{\tmplength}
\settowidth{\labelsep}{X}
\addtolength{\leftmargin}{\labelsep}
\setlength{\labelwidth}{\tmplength}
}
\begin{flushleft}
\item[\textbf{Declaração}\hfill]
\begin{ttfamily}
protected {\_}OnEnterField: TOnEnterField;\end{ttfamily}


\end{flushleft}
\end{list}
\paragraph*{{\_}OnExitField}\hspace*{\fill}

\begin{list}{}{
\settowidth{\tmplength}{\textbf{Declaração}}
\setlength{\itemindent}{0cm}
\setlength{\listparindent}{0cm}
\setlength{\leftmargin}{\evensidemargin}
\addtolength{\leftmargin}{\tmplength}
\settowidth{\labelsep}{X}
\addtolength{\leftmargin}{\labelsep}
\setlength{\labelwidth}{\tmplength}
}
\begin{flushleft}
\item[\textbf{Declaração}\hfill]
\begin{ttfamily}
protected {\_}OnExitField: TOnExitField;\end{ttfamily}


\end{flushleft}
\end{list}
\paragraph*{{\_}Active}\hspace*{\fill}

\begin{list}{}{
\settowidth{\tmplength}{\textbf{Declaração}}
\setlength{\itemindent}{0cm}
\setlength{\listparindent}{0cm}
\setlength{\leftmargin}{\evensidemargin}
\addtolength{\leftmargin}{\tmplength}
\settowidth{\labelsep}{X}
\addtolength{\leftmargin}{\labelsep}
\setlength{\labelwidth}{\tmplength}
}
\begin{flushleft}
\item[\textbf{Declaração}\hfill]
\begin{ttfamily}
protected var {\_}Active: Boolean;\end{ttfamily}


\end{flushleft}
\par
\item[\textbf{Descrição}]
O atributo \textbf{name} é usado para criar a propriedade \begin{ttfamily}active\end{ttfamily}(\ref{mi_rtl_ui_Dmxscroller.TUiDmxScroller-Active}) do componente

\begin{itemize}
\item NOTAS \begin{itemize}
\item O componente só pode estar ativo se o \begin{ttfamily}GetTemplate\end{ttfamily}(\ref{mi_rtl_ui_Dmxscroller.TUiDmxScroller-GetTemplate}) {$<$}{$>$} nil .
\item O método \textbf{CreateFormLCL} só pode ser executado uma vêz.
\item Caso o \begin{ttfamily}active\end{ttfamily}(\ref{mi_rtl_ui_Dmxscroller.TUiDmxScroller-Active}) esteja \textbf{true} e o usuário seta para false o controle Owner deve ficar invisível.
\end{itemize}
\end{itemize}

\end{list}
\paragraph*{{\_}CurrentField}\hspace*{\fill}

\begin{list}{}{
\settowidth{\tmplength}{\textbf{Declaração}}
\setlength{\itemindent}{0cm}
\setlength{\listparindent}{0cm}
\setlength{\leftmargin}{\evensidemargin}
\addtolength{\leftmargin}{\tmplength}
\settowidth{\labelsep}{X}
\addtolength{\leftmargin}{\labelsep}
\setlength{\labelwidth}{\tmplength}
}
\begin{flushleft}
\item[\textbf{Declaração}\hfill]
\begin{ttfamily}
protected {\_}CurrentField: pDmxFieldRec;\end{ttfamily}


\end{flushleft}
\end{list}
\paragraph*{{\_}BufDataset}\hspace*{\fill}

\begin{list}{}{
\settowidth{\tmplength}{\textbf{Declaração}}
\setlength{\itemindent}{0cm}
\setlength{\listparindent}{0cm}
\setlength{\leftmargin}{\evensidemargin}
\addtolength{\leftmargin}{\tmplength}
\settowidth{\labelsep}{X}
\addtolength{\leftmargin}{\labelsep}
\setlength{\labelwidth}{\tmplength}
}
\begin{flushleft}
\item[\textbf{Declaração}\hfill]
\begin{ttfamily}
protected {\_}BufDataset: TBufDataset;\end{ttfamily}


\end{flushleft}
\end{list}
\paragraph*{{\_}DataSource}\hspace*{\fill}

\begin{list}{}{
\settowidth{\tmplength}{\textbf{Declaração}}
\setlength{\itemindent}{0cm}
\setlength{\listparindent}{0cm}
\setlength{\leftmargin}{\evensidemargin}
\addtolength{\leftmargin}{\tmplength}
\settowidth{\labelsep}{X}
\addtolength{\leftmargin}{\labelsep}
\setlength{\labelwidth}{\tmplength}
}
\begin{flushleft}
\item[\textbf{Declaração}\hfill]
\begin{ttfamily}
protected {\_}DataSource: TDataSource;\end{ttfamily}


\end{flushleft}
\end{list}
\paragraph*{{\_}Locked}\hspace*{\fill}

\begin{list}{}{
\settowidth{\tmplength}{\textbf{Declaração}}
\setlength{\itemindent}{0cm}
\setlength{\listparindent}{0cm}
\setlength{\leftmargin}{\evensidemargin}
\addtolength{\leftmargin}{\tmplength}
\settowidth{\labelsep}{X}
\addtolength{\leftmargin}{\labelsep}
\setlength{\labelwidth}{\tmplength}
}
\begin{flushleft}
\item[\textbf{Declaração}\hfill]
\begin{ttfamily}
public {\_}Locked: Boolean;\end{ttfamily}


\end{flushleft}
\end{list}
\subsubsection*{\large{\textbf{Métodos}}\normalsize\hspace{1ex}\hfill}
\paragraph*{SetHelpCtx{\_}Hint}\hspace*{\fill}

\begin{list}{}{
\settowidth{\tmplength}{\textbf{Declaração}}
\setlength{\itemindent}{0cm}
\setlength{\listparindent}{0cm}
\setlength{\leftmargin}{\evensidemargin}
\addtolength{\leftmargin}{\tmplength}
\settowidth{\labelsep}{X}
\addtolength{\leftmargin}{\labelsep}
\setlength{\labelwidth}{\tmplength}
}
\begin{flushleft}
\item[\textbf{Declaração}\hfill]
\begin{ttfamily}
public Function SetHelpCtx{\_}Hint(aFldNum:Integer;a{\_}HelpCtx{\_}Hint:AnsiString):pDmxFieldRec; virtual; overload;\end{ttfamily}


\end{flushleft}
\par
\item[\textbf{Descrição}]
O método \textbf{\begin{ttfamily}SetHelpCtx{\_}Hint\end{ttfamily}} inicia a documentação resumida do campo. aFldNum

\end{list}
\paragraph*{SetHelpCtx{\_}Hint}\hspace*{\fill}

\begin{list}{}{
\settowidth{\tmplength}{\textbf{Declaração}}
\setlength{\itemindent}{0cm}
\setlength{\listparindent}{0cm}
\setlength{\leftmargin}{\evensidemargin}
\addtolength{\leftmargin}{\tmplength}
\settowidth{\labelsep}{X}
\addtolength{\leftmargin}{\labelsep}
\setlength{\labelwidth}{\tmplength}
}
\begin{flushleft}
\item[\textbf{Declaração}\hfill]
\begin{ttfamily}
public Procedure SetHelpCtx{\_}Hint(apDmxFieldRec:pDmxFieldRec;a{\_}HelpCtx{\_}Hint:AnsiString); virtual; overload;\end{ttfamily}


\end{flushleft}
\par
\item[\textbf{Descrição}]
O método \textbf{\begin{ttfamily}SetHelpCtx{\_}Hint\end{ttfamily}} inicia a documentação resumida do campo passado em :apDmxFieldRec

\end{list}
\paragraph*{SetCurrentRecord}\hspace*{\fill}

\begin{list}{}{
\settowidth{\tmplength}{\textbf{Declaração}}
\setlength{\itemindent}{0cm}
\setlength{\listparindent}{0cm}
\setlength{\leftmargin}{\evensidemargin}
\addtolength{\leftmargin}{\tmplength}
\settowidth{\labelsep}{X}
\addtolength{\leftmargin}{\labelsep}
\setlength{\labelwidth}{\tmplength}
}
\begin{flushleft}
\item[\textbf{Declaração}\hfill]
\begin{ttfamily}
protected Procedure SetCurrentRecord(aCurrentRecord : Longint ); Virtual;\end{ttfamily}


\end{flushleft}
\end{list}
\paragraph*{ShowControlState}\hspace*{\fill}

\begin{list}{}{
\settowidth{\tmplength}{\textbf{Declaração}}
\setlength{\itemindent}{0cm}
\setlength{\listparindent}{0cm}
\setlength{\leftmargin}{\evensidemargin}
\addtolength{\leftmargin}{\tmplength}
\settowidth{\labelsep}{X}
\addtolength{\leftmargin}{\labelsep}
\setlength{\labelwidth}{\tmplength}
}
\begin{flushleft}
\item[\textbf{Declaração}\hfill]
\begin{ttfamily}
protected procedure ShowControlState; Virtual;\end{ttfamily}


\end{flushleft}
\end{list}
\paragraph*{UpdateBuffers}\hspace*{\fill}

\begin{list}{}{
\settowidth{\tmplength}{\textbf{Declaração}}
\setlength{\itemindent}{0cm}
\setlength{\listparindent}{0cm}
\setlength{\leftmargin}{\evensidemargin}
\addtolength{\leftmargin}{\tmplength}
\settowidth{\labelsep}{X}
\addtolength{\leftmargin}{\labelsep}
\setlength{\labelwidth}{\tmplength}
}
\begin{flushleft}
\item[\textbf{Declaração}\hfill]
\begin{ttfamily}
public Procedure UpdateBuffers; Virtual;\end{ttfamily}


\end{flushleft}
\end{list}
\paragraph*{Refresh}\hspace*{\fill}

\begin{list}{}{
\settowidth{\tmplength}{\textbf{Declaração}}
\setlength{\itemindent}{0cm}
\setlength{\listparindent}{0cm}
\setlength{\leftmargin}{\evensidemargin}
\addtolength{\leftmargin}{\tmplength}
\settowidth{\labelsep}{X}
\addtolength{\leftmargin}{\labelsep}
\setlength{\labelwidth}{\tmplength}
}
\begin{flushleft}
\item[\textbf{Declaração}\hfill]
\begin{ttfamily}
public procedure Refresh; VIRTUAL;\end{ttfamily}


\end{flushleft}
\end{list}
\paragraph*{GetAppending}\hspace*{\fill}

\begin{list}{}{
\settowidth{\tmplength}{\textbf{Declaração}}
\setlength{\itemindent}{0cm}
\setlength{\listparindent}{0cm}
\setlength{\leftmargin}{\evensidemargin}
\addtolength{\leftmargin}{\tmplength}
\settowidth{\labelsep}{X}
\addtolength{\leftmargin}{\labelsep}
\setlength{\labelwidth}{\tmplength}
}
\begin{flushleft}
\item[\textbf{Declaração}\hfill]
\begin{ttfamily}
protected Function GetAppending:Boolean; VIRTUAL;\end{ttfamily}


\end{flushleft}
\end{list}
\paragraph*{SetAppending}\hspace*{\fill}

\begin{list}{}{
\settowidth{\tmplength}{\textbf{Declaração}}
\setlength{\itemindent}{0cm}
\setlength{\listparindent}{0cm}
\setlength{\leftmargin}{\evensidemargin}
\addtolength{\leftmargin}{\tmplength}
\settowidth{\labelsep}{X}
\addtolength{\leftmargin}{\labelsep}
\setlength{\labelwidth}{\tmplength}
}
\begin{flushleft}
\item[\textbf{Declaração}\hfill]
\begin{ttfamily}
protected Procedure SetAppending(aAppending:Boolean); VIRTUAL;\end{ttfamily}


\end{flushleft}
\end{list}
\paragraph*{SetOnCalcRecord}\hspace*{\fill}

\begin{list}{}{
\settowidth{\tmplength}{\textbf{Declaração}}
\setlength{\itemindent}{0cm}
\setlength{\listparindent}{0cm}
\setlength{\leftmargin}{\evensidemargin}
\addtolength{\leftmargin}{\tmplength}
\settowidth{\labelsep}{X}
\addtolength{\leftmargin}{\labelsep}
\setlength{\labelwidth}{\tmplength}
}
\begin{flushleft}
\item[\textbf{Declaração}\hfill]
\begin{ttfamily}
protected Function SetOnCalcRecord(Const WOnCalcRecordEnable:Boolean):Boolean;\end{ttfamily}


\end{flushleft}
\end{list}
\paragraph*{GetRecordSelected}\hspace*{\fill}

\begin{list}{}{
\settowidth{\tmplength}{\textbf{Declaração}}
\setlength{\itemindent}{0cm}
\setlength{\listparindent}{0cm}
\setlength{\leftmargin}{\evensidemargin}
\addtolength{\leftmargin}{\tmplength}
\settowidth{\labelsep}{X}
\addtolength{\leftmargin}{\labelsep}
\setlength{\labelwidth}{\tmplength}
}
\begin{flushleft}
\item[\textbf{Declaração}\hfill]
\begin{ttfamily}
protected Function GetRecordSelected: boolean; Virtual;\end{ttfamily}


\end{flushleft}
\end{list}
\paragraph*{SetRecordSelected}\hspace*{\fill}

\begin{list}{}{
\settowidth{\tmplength}{\textbf{Declaração}}
\setlength{\itemindent}{0cm}
\setlength{\listparindent}{0cm}
\setlength{\leftmargin}{\evensidemargin}
\addtolength{\leftmargin}{\tmplength}
\settowidth{\labelsep}{X}
\addtolength{\leftmargin}{\labelsep}
\setlength{\labelwidth}{\tmplength}
}
\begin{flushleft}
\item[\textbf{Declaração}\hfill]
\begin{ttfamily}
protected Procedure SetRecordSelected(a{\_}RecordSelected : boolean); Virtual;\end{ttfamily}


\end{flushleft}
\end{list}
\paragraph*{ChangeMadeOnOff}\hspace*{\fill}

\begin{list}{}{
\settowidth{\tmplength}{\textbf{Declaração}}
\setlength{\itemindent}{0cm}
\setlength{\listparindent}{0cm}
\setlength{\leftmargin}{\evensidemargin}
\addtolength{\leftmargin}{\tmplength}
\settowidth{\labelsep}{X}
\addtolength{\leftmargin}{\labelsep}
\setlength{\labelwidth}{\tmplength}
}
\begin{flushleft}
\item[\textbf{Declaração}\hfill]
\begin{ttfamily}
protected procedure ChangeMadeOnOff(const aValue:Boolean);\end{ttfamily}


\end{flushleft}
\par
\item[\textbf{Descrição}]
O método \textbf{\begin{ttfamily}ChangeMadeOnOff\end{ttfamily}} seta os atributos indicativos de que o objeto foi alterado ou não.

\end{list}
\paragraph*{DoOnNewRecord}\hspace*{\fill}

\begin{list}{}{
\settowidth{\tmplength}{\textbf{Declaração}}
\setlength{\itemindent}{0cm}
\setlength{\listparindent}{0cm}
\setlength{\leftmargin}{\evensidemargin}
\addtolength{\leftmargin}{\tmplength}
\settowidth{\labelsep}{X}
\addtolength{\leftmargin}{\labelsep}
\setlength{\labelwidth}{\tmplength}
}
\begin{flushleft}
\item[\textbf{Declaração}\hfill]
\begin{ttfamily}
public Procedure DoOnNewRecord; overload; Virtual;\end{ttfamily}


\end{flushleft}
\par
\item[\textbf{Descrição}]
O método \textbf{\begin{ttfamily}DoOnNewRecord\end{ttfamily}} é usado para inicializa os parâmetros de um novo registro

\end{list}
\paragraph*{SetState}\hspace*{\fill}

\begin{list}{}{
\settowidth{\tmplength}{\textbf{Declaração}}
\setlength{\itemindent}{0cm}
\setlength{\listparindent}{0cm}
\setlength{\leftmargin}{\evensidemargin}
\addtolength{\leftmargin}{\tmplength}
\settowidth{\labelsep}{X}
\addtolength{\leftmargin}{\labelsep}
\setlength{\labelwidth}{\tmplength}
}
\begin{flushleft}
\item[\textbf{Declaração}\hfill]
\begin{ttfamily}
protected Function SetState(Const AState: Int64; Const Enable: boolean):Boolean; virtual;\end{ttfamily}


\end{flushleft}
\par
\item[\textbf{Descrição}]
O método \textbf{\begin{ttfamily}SetState\end{ttfamily}} seta o bit passado no parâmetro aState e retorna o estado anterior do mapa de bits passado por aState

\end{list}
\paragraph*{GetState}\hspace*{\fill}

\begin{list}{}{
\settowidth{\tmplength}{\textbf{Declaração}}
\setlength{\itemindent}{0cm}
\setlength{\listparindent}{0cm}
\setlength{\leftmargin}{\evensidemargin}
\addtolength{\leftmargin}{\tmplength}
\settowidth{\labelsep}{X}
\addtolength{\leftmargin}{\labelsep}
\setlength{\labelwidth}{\tmplength}
}
\begin{flushleft}
\item[\textbf{Declaração}\hfill]
\begin{ttfamily}
public function GetState(Const AState: Int64): Boolean; Virtual;\end{ttfamily}


\end{flushleft}
\par
\item[\textbf{Descrição}]
O Método \textbf{\begin{ttfamily}GetState\end{ttfamily}} recebe um mapa de bits e retorna: \begin{itemize}
\item false : Se o bite estiver desligado;
\item true ; Se o bit estiver ligado
\end{itemize}

\end{list}
\paragraph*{FieldByName}\hspace*{\fill}

\begin{list}{}{
\settowidth{\tmplength}{\textbf{Declaração}}
\setlength{\itemindent}{0cm}
\setlength{\listparindent}{0cm}
\setlength{\leftmargin}{\evensidemargin}
\addtolength{\leftmargin}{\tmplength}
\settowidth{\labelsep}{X}
\addtolength{\leftmargin}{\labelsep}
\setlength{\labelwidth}{\tmplength}
}
\begin{flushleft}
\item[\textbf{Declaração}\hfill]
\begin{ttfamily}
public function FieldByName(aName:String):PDmxFieldRec;\end{ttfamily}


\end{flushleft}
\par
\item[\textbf{Descrição}]
O método \textbf{\begin{ttfamily}FieldByName\end{ttfamily}} retorna o campo passado por aName.

\end{list}
\paragraph*{FieldByNumber}\hspace*{\fill}

\begin{list}{}{
\settowidth{\tmplength}{\textbf{Declaração}}
\setlength{\itemindent}{0cm}
\setlength{\listparindent}{0cm}
\setlength{\leftmargin}{\evensidemargin}
\addtolength{\leftmargin}{\tmplength}
\settowidth{\labelsep}{X}
\addtolength{\leftmargin}{\labelsep}
\setlength{\labelwidth}{\tmplength}
}
\begin{flushleft}
\item[\textbf{Declaração}\hfill]
\begin{ttfamily}
public function FieldByNumber(aFieldNum:Integer):PDmxFieldRec;\end{ttfamily}


\end{flushleft}
\end{list}
\paragraph*{CancelBuffers}\hspace*{\fill}

\begin{list}{}{
\settowidth{\tmplength}{\textbf{Declaração}}
\setlength{\itemindent}{0cm}
\setlength{\listparindent}{0cm}
\setlength{\leftmargin}{\evensidemargin}
\addtolength{\leftmargin}{\tmplength}
\settowidth{\labelsep}{X}
\addtolength{\leftmargin}{\labelsep}
\setlength{\labelwidth}{\tmplength}
}
\begin{flushleft}
\item[\textbf{Declaração}\hfill]
\begin{ttfamily}
public function CancelBuffers: Boolean;\end{ttfamily}


\end{flushleft}
\par
\item[\textbf{Descrição}]
O método \textbf{\begin{ttfamily}CancelBuffers\end{ttfamily}} copia o buffer do registro anterior para o buffer do registro atual

\end{list}
\paragraph*{GetBuffers}\hspace*{\fill}

\begin{list}{}{
\settowidth{\tmplength}{\textbf{Declaração}}
\setlength{\itemindent}{0cm}
\setlength{\listparindent}{0cm}
\setlength{\leftmargin}{\evensidemargin}
\addtolength{\leftmargin}{\tmplength}
\settowidth{\labelsep}{X}
\addtolength{\leftmargin}{\labelsep}
\setlength{\labelwidth}{\tmplength}
}
\begin{flushleft}
\item[\textbf{Declaração}\hfill]
\begin{ttfamily}
protected function GetBuffers:Boolean; Virtual;\end{ttfamily}


\end{flushleft}
\par
\item[\textbf{Descrição}]
O método \textbf{\begin{ttfamily}GetBuffers\end{ttfamily}} copia o buffer do registro atual para o buffer do registro anterior

\begin{itemize}
\item \textbf{OBSERVAÇÃO:} \begin{itemize}
\item O método \textbf{\begin{ttfamily}GetBuffers\end{ttfamily}} deve ser anulado para ler o buffer dos campos dos arquivos associados a classe \textbf{\begin{ttfamily}TUiDmxScroller\end{ttfamily}(\ref{mi_rtl_ui_Dmxscroller.TUiDmxScroller})} para o buffer dos campos da classe \textbf{\begin{ttfamily}TUiDmxScroller\end{ttfamily}(\ref{mi_rtl_ui_Dmxscroller.TUiDmxScroller})}
\end{itemize}
\end{itemize}

\end{list}
\paragraph*{PutBuffers}\hspace*{\fill}

\begin{list}{}{
\settowidth{\tmplength}{\textbf{Declaração}}
\setlength{\itemindent}{0cm}
\setlength{\listparindent}{0cm}
\setlength{\leftmargin}{\evensidemargin}
\addtolength{\leftmargin}{\tmplength}
\settowidth{\labelsep}{X}
\addtolength{\leftmargin}{\labelsep}
\setlength{\labelwidth}{\tmplength}
}
\begin{flushleft}
\item[\textbf{Declaração}\hfill]
\begin{ttfamily}
protected function PutBuffers:Boolean; Virtual;\end{ttfamily}


\end{flushleft}
\par
\item[\textbf{Descrição}]
O método \textbf{\begin{ttfamily}PutBuffers\end{ttfamily}} deve ser anulado para grava o buffer dos campos da classe \textbf{\begin{ttfamily}TUiDmxScroller\end{ttfamily}(\ref{mi_rtl_ui_Dmxscroller.TUiDmxScroller})} para o buffer dos campos dos arquivos associados a classe \textbf{\begin{ttfamily}TUiDmxScroller\end{ttfamily}(\ref{mi_rtl_ui_Dmxscroller.TUiDmxScroller})}

\end{list}
\paragraph*{DoOnCloseQuery}\hspace*{\fill}

\begin{list}{}{
\settowidth{\tmplength}{\textbf{Declaração}}
\setlength{\itemindent}{0cm}
\setlength{\listparindent}{0cm}
\setlength{\leftmargin}{\evensidemargin}
\addtolength{\leftmargin}{\tmplength}
\settowidth{\labelsep}{X}
\addtolength{\leftmargin}{\labelsep}
\setlength{\labelwidth}{\tmplength}
}
\begin{flushleft}
\item[\textbf{Declaração}\hfill]
\begin{ttfamily}
public Procedure DoOnCloseQuery(aDmxScroller:TUiDmxScroller ; var CanClose:boolean ); overload;\end{ttfamily}


\end{flushleft}
\end{list}
\paragraph*{DoOnCloseQuery}\hspace*{\fill}

\begin{list}{}{
\settowidth{\tmplength}{\textbf{Declaração}}
\setlength{\itemindent}{0cm}
\setlength{\listparindent}{0cm}
\setlength{\leftmargin}{\evensidemargin}
\addtolength{\leftmargin}{\tmplength}
\settowidth{\labelsep}{X}
\addtolength{\leftmargin}{\labelsep}
\setlength{\labelwidth}{\tmplength}
}
\begin{flushleft}
\item[\textbf{Declaração}\hfill]
\begin{ttfamily}
public Procedure DoOnCloseQuery(var CanClose:boolean ); overload;\end{ttfamily}


\end{flushleft}
\end{list}
\paragraph*{Scroll{\_}it{\_}inview}\hspace*{\fill}

\begin{list}{}{
\settowidth{\tmplength}{\textbf{Declaração}}
\setlength{\itemindent}{0cm}
\setlength{\listparindent}{0cm}
\setlength{\leftmargin}{\evensidemargin}
\addtolength{\leftmargin}{\tmplength}
\settowidth{\labelsep}{X}
\addtolength{\leftmargin}{\labelsep}
\setlength{\labelwidth}{\tmplength}
}
\begin{flushleft}
\item[\textbf{Declaração}\hfill]
\begin{ttfamily}
public procedure Scroll{\_}it{\_}inview(AControl: pDmxFieldRec); virtual;\end{ttfamily}


\end{flushleft}
\par
\item[\textbf{Descrição}]
O método \textbf{\begin{ttfamily}Scroll{\_}it{\_}inview\end{ttfamily}} é usado para da o scroller na janela onde esse componente for inserido. \begin{itemize}
\item \textbf{NOTA} \begin{itemize}
\item A LCL não rola a tela com a tecla tab e o controle não estiver visível.
\end{itemize}
\end{itemize}

\end{list}
\paragraph*{DoOnEnter}\hspace*{\fill}

\begin{list}{}{
\settowidth{\tmplength}{\textbf{Declaração}}
\setlength{\itemindent}{0cm}
\setlength{\listparindent}{0cm}
\setlength{\leftmargin}{\evensidemargin}
\addtolength{\leftmargin}{\tmplength}
\settowidth{\labelsep}{X}
\addtolength{\leftmargin}{\labelsep}
\setlength{\labelwidth}{\tmplength}
}
\begin{flushleft}
\item[\textbf{Declaração}\hfill]
\begin{ttfamily}
public Procedure DoOnEnter(aDmxScroller:TUiDmxScroller); Virtual;\end{ttfamily}


\end{flushleft}
\par
\item[\textbf{Descrição}]
O método \textbf{\begin{ttfamily}DoOnEnter\end{ttfamily}} Executa o evento \begin{ttfamily}onEnter\end{ttfamily}(\ref{mi_rtl_ui_Dmxscroller.TUiDmxScroller-onEnter}) se o mesmo estiver assinalado.

\end{list}
\paragraph*{DoOnExit}\hspace*{\fill}

\begin{list}{}{
\settowidth{\tmplength}{\textbf{Declaração}}
\setlength{\itemindent}{0cm}
\setlength{\listparindent}{0cm}
\setlength{\leftmargin}{\evensidemargin}
\addtolength{\leftmargin}{\tmplength}
\settowidth{\labelsep}{X}
\addtolength{\leftmargin}{\labelsep}
\setlength{\labelwidth}{\tmplength}
}
\begin{flushleft}
\item[\textbf{Declaração}\hfill]
\begin{ttfamily}
public Procedure DoOnExit(aDmxScroller:TUiDmxScroller);\end{ttfamily}


\end{flushleft}
\end{list}
\paragraph*{BeforeDestruction}\hspace*{\fill}

\begin{list}{}{
\settowidth{\tmplength}{\textbf{Declaração}}
\setlength{\itemindent}{0cm}
\setlength{\listparindent}{0cm}
\setlength{\leftmargin}{\evensidemargin}
\addtolength{\leftmargin}{\tmplength}
\settowidth{\labelsep}{X}
\addtolength{\leftmargin}{\labelsep}
\setlength{\labelwidth}{\tmplength}
}
\begin{flushleft}
\item[\textbf{Declaração}\hfill]
\begin{ttfamily}
public procedure BeforeDestruction; override;\end{ttfamily}


\end{flushleft}
\par
\item[\textbf{Descrição}]
Executado antes de construir o componente

\end{list}
\paragraph*{Create}\hspace*{\fill}

\begin{list}{}{
\settowidth{\tmplength}{\textbf{Declaração}}
\setlength{\itemindent}{0cm}
\setlength{\listparindent}{0cm}
\setlength{\leftmargin}{\evensidemargin}
\addtolength{\leftmargin}{\tmplength}
\settowidth{\labelsep}{X}
\addtolength{\leftmargin}{\labelsep}
\setlength{\labelwidth}{\tmplength}
}
\begin{flushleft}
\item[\textbf{Declaração}\hfill]
\begin{ttfamily}
public constructor Create(aOwner:TComponent); Override;\end{ttfamily}


\end{flushleft}
\par
\item[\textbf{Descrição}]
Constrói o componente

\end{list}
\paragraph*{destroy}\hspace*{\fill}

\begin{list}{}{
\settowidth{\tmplength}{\textbf{Declaração}}
\setlength{\itemindent}{0cm}
\setlength{\listparindent}{0cm}
\setlength{\leftmargin}{\evensidemargin}
\addtolength{\leftmargin}{\tmplength}
\settowidth{\labelsep}{X}
\addtolength{\leftmargin}{\labelsep}
\setlength{\labelwidth}{\tmplength}
}
\begin{flushleft}
\item[\textbf{Declaração}\hfill]
\begin{ttfamily}
public destructor destroy; override;\end{ttfamily}


\end{flushleft}
\par
\item[\textbf{Descrição}]
Destrói o componente

\end{list}
\paragraph*{CreateStruct}\hspace*{\fill}

\begin{list}{}{
\settowidth{\tmplength}{\textbf{Declaração}}
\setlength{\itemindent}{0cm}
\setlength{\listparindent}{0cm}
\setlength{\leftmargin}{\evensidemargin}
\addtolength{\leftmargin}{\tmplength}
\settowidth{\labelsep}{X}
\addtolength{\leftmargin}{\labelsep}
\setlength{\labelwidth}{\tmplength}
}
\begin{flushleft}
\item[\textbf{Declaração}\hfill]
\begin{ttfamily}
protected procedure CreateStruct(var ATemplate : TString); virtual; overload;\end{ttfamily}


\end{flushleft}
\par
\item[\textbf{Descrição}]
A procedure \textbf{\begin{ttfamily}CreateStruct\end{ttfamily}} é executado para construir a lista apontada por \begin{ttfamily}DMXField1\end{ttfamily}(\ref{mi_rtl_ui_Dmxscroller.TUiDmxScroller-DMXField1}) baseado no Template do tipo \begin{ttfamily}TString\end{ttfamily}(\ref{mi_rtl_ui_Dmxscroller-tString}).

\begin{itemize}
\item \textbf{NOTA} \begin{itemize}
\item O parâmetro aTemplate é um string com 255 caracteres, porém o mesmo pode ser encandeado usando a função \begin{ttfamily}CreateAppendFields\end{ttfamily}(\ref{mi_rtl_ui_methods.TUiMethods-CreateAppendFields}).
\item A função \begin{ttfamily}CreateAppendFields\end{ttfamily}(\ref{mi_rtl_ui_methods.TUiMethods-CreateAppendFields}) retorna a constante \textbf{\begin{ttfamily}fldAPPEND\end{ttfamily}(\ref{mi_rtl_ui_dmxscroller_form-fldAPPEND})} mais o endereço da string a ser concatenada.

\begin{itemize}
\item \textbf{EXEMPLO}

\texttt{\\\nopagebreak[3]
}\textbf{Var}\texttt{\\\nopagebreak[3]
~~S1,s2,Template~:~TString;\\\nopagebreak[3]
}\textbf{begin}\texttt{\\\nopagebreak[3]
~~S1~:=~'~Nome~do~Aluno....:~{\textbackslash}ssssssssssssssssssssssssssssssssss';\\\nopagebreak[3]
~~s2~:=~'~Endereço~do~aluno:~{\textbackslash}sssssssssssssssssssssssss';\\\nopagebreak[3]
~~Template~:=~S1+CreateAppendFields(s2);\\\nopagebreak[3]
}\textbf{end}\texttt{;\\
}
\end{itemize}
\end{itemize}
\end{itemize}

\end{list}
\paragraph*{CreateStruct}\hspace*{\fill}

\begin{list}{}{
\settowidth{\tmplength}{\textbf{Declaração}}
\setlength{\itemindent}{0cm}
\setlength{\listparindent}{0cm}
\setlength{\leftmargin}{\evensidemargin}
\addtolength{\leftmargin}{\tmplength}
\settowidth{\labelsep}{X}
\addtolength{\leftmargin}{\labelsep}
\setlength{\labelwidth}{\tmplength}
}
\begin{flushleft}
\item[\textbf{Declaração}\hfill]
\begin{ttfamily}
protected procedure CreateStruct(var ATemplate : PSItem); virtual; overload;\end{ttfamily}


\end{flushleft}
\par
\item[\textbf{Descrição}]
A procedure \textbf{\begin{ttfamily}CreateStruct\end{ttfamily}} é executado para construir a lista apontada por \begin{ttfamily}DMXField1\end{ttfamily}(\ref{mi_rtl_ui_Dmxscroller.TUiDmxScroller-DMXField1}) baseado na lista \begin{ttfamily}PSItem\end{ttfamily}(\ref{mi_rtl_ui_Dmxscroller-PSItem}).

\begin{itemize}
\item \textbf{NOTA} \begin{itemize}
\item O parâmetro aTemplate é uma lista \begin{ttfamily}PSitem\end{ttfamily}(\ref{mi_rtl_ui_Dmxscroller-PSItem}).
\item A função \begin{ttfamily}CreateTSItemFields\end{ttfamily}(\ref{mi_rtl_ui_methods.TUiMethods-CreateTSItemFields}) retorna uma lista de \begin{ttfamily}PSItem\end{ttfamily}(\ref{mi_rtl_ui_Dmxscroller-PSItem}). \begin{itemize}
\item \textbf{EXEMPLO}

\texttt{\\\nopagebreak[3]
}\textbf{Var}\texttt{\\\nopagebreak[3]
~~Template~:~PSItem;\\\nopagebreak[3]
}\textbf{begin}\texttt{\\\nopagebreak[3]
~~Template~:=~CreateTSItemFields(\\\nopagebreak[3]
~~~~~~~~~~~~~~~~~~~~~~~~NewSItem('cccccccccccccccccccccccccccccccccccccccccccccccccc',\\\nopagebreak[3]
~~~~~~~~~~~~~~~~~~~~~~~~NewSItem('cccccccccccccccccccccccccccccccccccccccccccccccccc',\\\nopagebreak[3]
~~~~~~~~~~~~~~~~~~~~~~~~NewSItem('cccccccccccccccccccccccccccccccccccccccccccccccccc',\\\nopagebreak[3]
~~~~~~~~~~~~~~~~~~~~~~~~NewSItem('cccccccccccccccccccccccccccccccccccccccccccccccccc',\\\nopagebreak[3]
~~~~~~~~~~~~~~~~~~~~~~~~NewSItem('cccccccccccccccccccccccccccccccccccccccccccccccccc',}\textbf{nil}\texttt{))))))\\\nopagebreak[3]
}\textbf{end}\texttt{;\\
}
\end{itemize}
\end{itemize}
\end{itemize}

\end{list}
\paragraph*{CreateStruct}\hspace*{\fill}

\begin{list}{}{
\settowidth{\tmplength}{\textbf{Declaração}}
\setlength{\itemindent}{0cm}
\setlength{\listparindent}{0cm}
\setlength{\leftmargin}{\evensidemargin}
\addtolength{\leftmargin}{\tmplength}
\settowidth{\labelsep}{X}
\addtolength{\leftmargin}{\labelsep}
\setlength{\labelwidth}{\tmplength}
}
\begin{flushleft}
\item[\textbf{Declaração}\hfill]
\begin{ttfamily}
protected procedure CreateStruct(); virtual; overload;\end{ttfamily}


\end{flushleft}
\par
\item[\textbf{Descrição}]
A procedure \textbf{\begin{ttfamily}CreateStruct\end{ttfamily}} interpreta o Template obtido em \begin{ttfamily}getTemplate\end{ttfamily}(\ref{mi_rtl_ui_Dmxscroller.TUiDmxScroller-GetTemplate}) e cria a lista de \textbf{\begin{ttfamily}pDmxFieldRec\end{ttfamily}(\ref{mi_rtl_ui_Dmxscroller-pDmxFieldRec})} associada ao Template.

\begin{itemize}
\item \textbf{Nota} \begin{itemize}
\item O Template pode ser obtido pela propriedade Template se Template{$<$}{$>$} '' ou retornado pelo evento \begin{ttfamily}onGetTemplate\end{ttfamily}(\ref{mi_rtl_ui_Dmxscroller.TUiDmxScroller-onGetTemplate}).
\end{itemize}
\end{itemize}

\end{list}
\paragraph*{DestroyStruct}\hspace*{\fill}

\begin{list}{}{
\settowidth{\tmplength}{\textbf{Declaração}}
\setlength{\itemindent}{0cm}
\setlength{\listparindent}{0cm}
\setlength{\leftmargin}{\evensidemargin}
\addtolength{\leftmargin}{\tmplength}
\settowidth{\labelsep}{X}
\addtolength{\leftmargin}{\labelsep}
\setlength{\labelwidth}{\tmplength}
}
\begin{flushleft}
\item[\textbf{Declaração}\hfill]
\begin{ttfamily}
protected procedure DestroyStruct; virtual;\end{ttfamily}


\end{flushleft}
\par
\item[\textbf{Descrição}]
A procedure \textbf{\begin{ttfamily}DestroyStruct\end{ttfamily}} destrói as lista criada por \begin{ttfamily}CreateStruct\end{ttfamily}(\ref{mi_rtl_ui_Dmxscroller.TUiDmxScroller-CreateStruct}) acima.

\end{list}
\paragraph*{CreateBufDataset{\_}FieldDefs}\hspace*{\fill}

\begin{list}{}{
\settowidth{\tmplength}{\textbf{Declaração}}
\setlength{\itemindent}{0cm}
\setlength{\listparindent}{0cm}
\setlength{\leftmargin}{\evensidemargin}
\addtolength{\leftmargin}{\tmplength}
\settowidth{\labelsep}{X}
\addtolength{\leftmargin}{\labelsep}
\setlength{\labelwidth}{\tmplength}
}
\begin{flushleft}
\item[\textbf{Declaração}\hfill]
\begin{ttfamily}
protected Procedure CreateBufDataset{\_}FieldDefs; virtual;\end{ttfamily}


\end{flushleft}
\par
\item[\textbf{Descrição}]
O método \textbf{\begin{ttfamily}CreateBufDataset{\_}FieldDefs\end{ttfamily}} é usado para criar os campos de \textbf{\begin{ttfamily}BufDataset\end{ttfamily}(\ref{mi_rtl_ui_Dmxscroller.TUiDmxScroller-BufDataset})}

\end{list}
\paragraph*{CreateData}\hspace*{\fill}

\begin{list}{}{
\settowidth{\tmplength}{\textbf{Declaração}}
\setlength{\itemindent}{0cm}
\setlength{\listparindent}{0cm}
\setlength{\leftmargin}{\evensidemargin}
\addtolength{\leftmargin}{\tmplength}
\settowidth{\labelsep}{X}
\addtolength{\leftmargin}{\labelsep}
\setlength{\labelwidth}{\tmplength}
}
\begin{flushleft}
\item[\textbf{Declaração}\hfill]
\begin{ttfamily}
protected procedure CreateData; Virtual;\end{ttfamily}


\end{flushleft}
\par
\item[\textbf{Descrição}]
A procedure \textbf{\begin{ttfamily}CreateData\end{ttfamily}} é usada para alocar (\begin{ttfamily}RecordSize\end{ttfamily}(\ref{mi_rtl_ui_Dmxscroller.TUiDmxScroller-RecordSize})) memória para o buffer (\begin{ttfamily}WorkingData\end{ttfamily}(\ref{mi_rtl_ui_Dmxscroller.TUiDmxScroller-WorkingData})) do registro calculado por \begin{ttfamily}createStruct\end{ttfamily}(\ref{mi_rtl_ui_Dmxscroller.TUiDmxScroller-CreateStruct})

\end{list}
\paragraph*{DestroyData}\hspace*{\fill}

\begin{list}{}{
\settowidth{\tmplength}{\textbf{Declaração}}
\setlength{\itemindent}{0cm}
\setlength{\listparindent}{0cm}
\setlength{\leftmargin}{\evensidemargin}
\addtolength{\leftmargin}{\tmplength}
\settowidth{\labelsep}{X}
\addtolength{\leftmargin}{\labelsep}
\setlength{\labelwidth}{\tmplength}
}
\begin{flushleft}
\item[\textbf{Declaração}\hfill]
\begin{ttfamily}
protected procedure DestroyData; virtual;\end{ttfamily}


\end{flushleft}
\par
\item[\textbf{Descrição}]
A procedure \textbf{\begin{ttfamily}DestroyData\end{ttfamily}} é usada para desalocar memória do buffer do registro criado por \begin{ttfamily}CreateData\end{ttfamily}(\ref{mi_rtl_ui_Dmxscroller.TUiDmxScroller-CreateData})

\end{list}
\paragraph*{GetRecordData}\hspace*{\fill}

\begin{list}{}{
\settowidth{\tmplength}{\textbf{Declaração}}
\setlength{\itemindent}{0cm}
\setlength{\listparindent}{0cm}
\setlength{\leftmargin}{\evensidemargin}
\addtolength{\leftmargin}{\tmplength}
\settowidth{\labelsep}{X}
\addtolength{\leftmargin}{\labelsep}
\setlength{\labelwidth}{\tmplength}
}
\begin{flushleft}
\item[\textbf{Declaração}\hfill]
\begin{ttfamily}
public Function GetRecordData: Pointer; virtual;\end{ttfamily}


\end{flushleft}
\par
\item[\textbf{Descrição}]
A função \textbf{\begin{ttfamily}GetRecordData\end{ttfamily}} retorna o atributo \begin{ttfamily}WorkingData\end{ttfamily}(\ref{mi_rtl_ui_Dmxscroller.TUiDmxScroller-WorkingData})

\end{list}
\paragraph*{SetLimit}\hspace*{\fill}

\begin{list}{}{
\settowidth{\tmplength}{\textbf{Declaração}}
\setlength{\itemindent}{0cm}
\setlength{\listparindent}{0cm}
\setlength{\leftmargin}{\evensidemargin}
\addtolength{\leftmargin}{\tmplength}
\settowidth{\labelsep}{X}
\addtolength{\leftmargin}{\labelsep}
\setlength{\labelwidth}{\tmplength}
}
\begin{flushleft}
\item[\textbf{Declaração}\hfill]
\begin{ttfamily}
protected Procedure SetLimit(X, Y: Integer); virtual;\end{ttfamily}


\end{flushleft}
\end{list}
\paragraph*{GetTemplate}\hspace*{\fill}

\begin{list}{}{
\settowidth{\tmplength}{\textbf{Declaração}}
\setlength{\itemindent}{0cm}
\setlength{\listparindent}{0cm}
\setlength{\leftmargin}{\evensidemargin}
\addtolength{\leftmargin}{\tmplength}
\settowidth{\labelsep}{X}
\addtolength{\leftmargin}{\labelsep}
\setlength{\labelwidth}{\tmplength}
}
\begin{flushleft}
\item[\textbf{Declaração}\hfill]
\begin{ttfamily}
protected function GetTemplate(aNext: PSItem) : PSItem; overload; virtual;\end{ttfamily}


\end{flushleft}
\par
\item[\textbf{Descrição}]
O método \textbf{\begin{ttfamily}GetTemplate\end{ttfamily}} é usado para atualizar o atributo {\_}onGetTemplate com o modelo informado pelo usuário caso o \begin{ttfamily}onGetTemplate\end{ttfamily}(\ref{mi_rtl_ui_Dmxscroller.TUiDmxScroller-onGetTemplate}) seja nil.

\begin{itemize}
\item \textbf{NOTA} \begin{enumerate}
\setcounter{enumi}{0} \setcounter{enumii}{0} \setcounter{enumiii}{0} \setcounter{enumiv}{0} 
\item O Evento {\_}onGetTemplate só é iniciado em tempo de execução por isso o formulário não pode ser criado em tempo de projeto usando o evento \textbf{\begin{ttfamily}onGetTemplate\end{ttfamily}(\ref{mi_rtl_ui_Dmxscroller.TUiDmxScroller-onGetTemplate})}.
\setcounter{enumi}{1} \setcounter{enumii}{1} \setcounter{enumiii}{1} \setcounter{enumiv}{1} 
\item As \begin{ttfamily}strings\end{ttfamily}(\ref{mi_rtl_ui_Dmxscroller.TUiDmxScroller-Strings}) do formulário também pode ser desenhado usando o evento \begin{ttfamily}OnAddTemplate\end{ttfamily}(\ref{mi_rtl_ui_Dmxscroller.TUiDmxScroller-onAddTemplate}).
\setcounter{enumi}{2} \setcounter{enumii}{2} \setcounter{enumiii}{2} \setcounter{enumiv}{2} 
\item O evento \begin{ttfamily}OnGetTemplate\end{ttfamily}(\ref{mi_rtl_ui_Dmxscroller.TUiDmxScroller-onGetTemplate}) tem prioridade em relação ao evento \begin{ttfamily}OnAddTemplate\end{ttfamily}(\ref{mi_rtl_ui_Dmxscroller.TUiDmxScroller-onAddTemplate});
\end{enumerate}
\end{itemize}

\end{list}
\paragraph*{SetActive}\hspace*{\fill}

\begin{list}{}{
\settowidth{\tmplength}{\textbf{Declaração}}
\setlength{\itemindent}{0cm}
\setlength{\listparindent}{0cm}
\setlength{\leftmargin}{\evensidemargin}
\addtolength{\leftmargin}{\tmplength}
\settowidth{\labelsep}{X}
\addtolength{\leftmargin}{\labelsep}
\setlength{\labelwidth}{\tmplength}
}
\begin{flushleft}
\item[\textbf{Declaração}\hfill]
\begin{ttfamily}
protected procedure SetActive(aActive : Boolean); virtual;\end{ttfamily}


\end{flushleft}
\par
\item[\textbf{Descrição}]
A procedure \textbf{\begin{ttfamily}SetActive\end{ttfamily}} seta a propriedade \begin{ttfamily}active\end{ttfamily}(\ref{mi_rtl_ui_Dmxscroller.TUiDmxScroller-Active}) e criar um formulário LCL ou HTML dependendo do tipo de aplicação

\end{list}
\paragraph*{SetCurrentField}\hspace*{\fill}

\begin{list}{}{
\settowidth{\tmplength}{\textbf{Declaração}}
\setlength{\itemindent}{0cm}
\setlength{\listparindent}{0cm}
\setlength{\leftmargin}{\evensidemargin}
\addtolength{\leftmargin}{\tmplength}
\settowidth{\labelsep}{X}
\addtolength{\leftmargin}{\labelsep}
\setlength{\labelwidth}{\tmplength}
}
\begin{flushleft}
\item[\textbf{Declaração}\hfill]
\begin{ttfamily}
protected Procedure SetCurrentField(aCurrentField : pDmxFieldRec);\end{ttfamily}


\end{flushleft}
\end{list}
\paragraph*{PutString}\hspace*{\fill}

\begin{list}{}{
\settowidth{\tmplength}{\textbf{Declaração}}
\setlength{\itemindent}{0cm}
\setlength{\listparindent}{0cm}
\setlength{\leftmargin}{\evensidemargin}
\addtolength{\leftmargin}{\tmplength}
\settowidth{\labelsep}{X}
\addtolength{\leftmargin}{\labelsep}
\setlength{\labelwidth}{\tmplength}
}
\begin{flushleft}
\item[\textbf{Declaração}\hfill]
\begin{ttfamily}
public Function PutString(Const OkSpc:Boolean;Const S:tString) : SmallInt; virtual; overload;\end{ttfamily}


\end{flushleft}
\par
\item[\textbf{Descrição}]
A função \textbf{\begin{ttfamily}PutString\end{ttfamily}} salva a string S no \begin{ttfamily}currentField\end{ttfamily}(\ref{mi_rtl_ui_Dmxscroller.TUiDmxScroller-CurrentField})

\begin{itemize}
\item \textbf{PARÂMETROS} \begin{itemize}
\item OkSpc : campo lógico e se \textbf{true} salva o campo preenchendo com espaço para completar a máscara.
\item S : String do tipo ShortString com conteúdo do campo.
\end{itemize}
\end{itemize}

\end{list}
\paragraph*{PutString}\hspace*{\fill}

\begin{list}{}{
\settowidth{\tmplength}{\textbf{Declaração}}
\setlength{\itemindent}{0cm}
\setlength{\listparindent}{0cm}
\setlength{\leftmargin}{\evensidemargin}
\addtolength{\leftmargin}{\tmplength}
\settowidth{\labelsep}{X}
\addtolength{\leftmargin}{\labelsep}
\setlength{\labelwidth}{\tmplength}
}
\begin{flushleft}
\item[\textbf{Declaração}\hfill]
\begin{ttfamily}
public function PutString(Const aFieldName:tString;S : ShortString):SmallInt; virtual; overload;\end{ttfamily}


\end{flushleft}
\par
\item[\textbf{Descrição}]
O método \textbf{\begin{ttfamily}PutString\end{ttfamily}} salva um string no campo passado por aFieldName.

\end{list}
\paragraph*{GetString}\hspace*{\fill}

\begin{list}{}{
\settowidth{\tmplength}{\textbf{Declaração}}
\setlength{\itemindent}{0cm}
\setlength{\listparindent}{0cm}
\setlength{\leftmargin}{\evensidemargin}
\addtolength{\leftmargin}{\tmplength}
\settowidth{\labelsep}{X}
\addtolength{\leftmargin}{\labelsep}
\setlength{\labelwidth}{\tmplength}
}
\begin{flushleft}
\item[\textbf{Declaração}\hfill]
\begin{ttfamily}
public function GetString(const aFieldName: tString):AnsiString; virtual; overload;\end{ttfamily}


\end{flushleft}
\par
\item[\textbf{Descrição}]
O método \textbf{\begin{ttfamily}GetString\end{ttfamily}} retorna um string do campo passado por aFieldName.

\end{list}
\paragraph*{GetString}\hspace*{\fill}

\begin{list}{}{
\settowidth{\tmplength}{\textbf{Declaração}}
\setlength{\itemindent}{0cm}
\setlength{\listparindent}{0cm}
\setlength{\leftmargin}{\evensidemargin}
\addtolength{\leftmargin}{\tmplength}
\settowidth{\labelsep}{X}
\addtolength{\leftmargin}{\labelsep}
\setlength{\labelwidth}{\tmplength}
}
\begin{flushleft}
\item[\textbf{Declaração}\hfill]
\begin{ttfamily}
public Function GetString(Const OkSpc:Boolean) : TString; virtual; overload;\end{ttfamily}


\end{flushleft}
\par
\item[\textbf{Descrição}]
A função \textbf{\begin{ttfamily}GetString\end{ttfamily}} retorna a string com o valor do \begin{ttfamily}currentField\end{ttfamily}(\ref{mi_rtl_ui_Dmxscroller.TUiDmxScroller-CurrentField})

\begin{itemize}
\item \textbf{PARÂMETROS} \begin{itemize}
\item OkSpc : campo lógico e se \textbf{true} retorna o campo preenchendo com espaço para completar a máscara.
\end{itemize}
\end{itemize}

\end{list}
\paragraph*{GetString}\hspace*{\fill}

\begin{list}{}{
\settowidth{\tmplength}{\textbf{Declaração}}
\setlength{\itemindent}{0cm}
\setlength{\listparindent}{0cm}
\setlength{\leftmargin}{\evensidemargin}
\addtolength{\leftmargin}{\tmplength}
\settowidth{\labelsep}{X}
\addtolength{\leftmargin}{\labelsep}
\setlength{\labelwidth}{\tmplength}
}
\begin{flushleft}
\item[\textbf{Declaração}\hfill]
\begin{ttfamily}
public Function GetString: TString; virtual; overload;\end{ttfamily}


\end{flushleft}
\par
\item[\textbf{Descrição}]
A função \textbf{\begin{ttfamily}GetString\end{ttfamily}} retorna a string com o valor do \begin{ttfamily}currentField\end{ttfamily}(\ref{mi_rtl_ui_Dmxscroller.TUiDmxScroller-CurrentField}) sem preencher com espaço para completar a máscara.

\end{list}
\paragraph*{PutString}\hspace*{\fill}

\begin{list}{}{
\settowidth{\tmplength}{\textbf{Declaração}}
\setlength{\itemindent}{0cm}
\setlength{\listparindent}{0cm}
\setlength{\leftmargin}{\evensidemargin}
\addtolength{\leftmargin}{\tmplength}
\settowidth{\labelsep}{X}
\addtolength{\leftmargin}{\labelsep}
\setlength{\labelwidth}{\tmplength}
}
\begin{flushleft}
\item[\textbf{Declaração}\hfill]
\begin{ttfamily}
public function PutString(const S : ShortString):SmallInt; virtual; overload;\end{ttfamily}


\end{flushleft}
\par
\item[\textbf{Descrição}]
A função \textbf{\begin{ttfamily}PutString\end{ttfamily}} salva a string \textbf{S} no \begin{ttfamily}currentField\end{ttfamily}(\ref{mi_rtl_ui_Dmxscroller.TUiDmxScroller-CurrentField}) usando \textbf{okspc} = false;

\begin{itemize}
\item \textbf{PARÂMETROS} \begin{itemize}
\item S : String do tipo ShortString com conteúdo do campo.
\end{itemize}
\end{itemize}

\end{list}
\paragraph*{Get{\_}MaskEdit{\_}LCL}\hspace*{\fill}

\begin{list}{}{
\settowidth{\tmplength}{\textbf{Declaração}}
\setlength{\itemindent}{0cm}
\setlength{\listparindent}{0cm}
\setlength{\leftmargin}{\evensidemargin}
\addtolength{\leftmargin}{\tmplength}
\settowidth{\labelsep}{X}
\addtolength{\leftmargin}{\labelsep}
\setlength{\labelwidth}{\tmplength}
}
\begin{flushleft}
\item[\textbf{Declaração}\hfill]
\begin{ttfamily}
public Function Get{\_}MaskEdit{\_}LCL(aTemplate : ShortString; out Size{\_}TypeFld, aLength{\_}Buffer : SmallWord; out OkMask : Boolean) : AnsiString; overload;\end{ttfamily}


\end{flushleft}
\par
\item[\textbf{Descrição}]
O método \textbf{\begin{ttfamily}Get{\_}MaskEdit{\_}LCL\end{ttfamily}} receber a máscara do DmxScroller e retorna a mascara do componente LCL.

\begin{itemize}
\item \textbf{Nota} \begin{itemize}
\item Em \textbf{Size{\_}TypeFld} retorno o tamanho do tipo de dados da mascara;
\item Em \textbf{OkMask} retorna \textbf{true} se tiver mascara e \textbf{false} caso contrário
\end{itemize}
\end{itemize}

\end{list}
\paragraph*{Get{\_}MaskEdit{\_}LCL}\hspace*{\fill}

\begin{list}{}{
\settowidth{\tmplength}{\textbf{Declaração}}
\setlength{\itemindent}{0cm}
\setlength{\listparindent}{0cm}
\setlength{\leftmargin}{\evensidemargin}
\addtolength{\leftmargin}{\tmplength}
\settowidth{\labelsep}{X}
\addtolength{\leftmargin}{\labelsep}
\setlength{\labelwidth}{\tmplength}
}
\begin{flushleft}
\item[\textbf{Declaração}\hfill]
\begin{ttfamily}
public Function Get{\_}MaskEdit{\_}LCL(aTemplate : ShortString; out OkMask : Boolean) : AnsiString; overload;\end{ttfamily}


\end{flushleft}
\par
\item[\textbf{Descrição}]
O método \textbf{\begin{ttfamily}Get{\_}MaskEdit{\_}LCL\end{ttfamily}} receber a máscara do Dmx e retorna a mascara do componente LCL.

\end{list}
\paragraph*{DoAddRec}\hspace*{\fill}

\begin{list}{}{
\settowidth{\tmplength}{\textbf{Declaração}}
\setlength{\itemindent}{0cm}
\setlength{\listparindent}{0cm}
\setlength{\leftmargin}{\evensidemargin}
\addtolength{\leftmargin}{\tmplength}
\settowidth{\labelsep}{X}
\addtolength{\leftmargin}{\labelsep}
\setlength{\labelwidth}{\tmplength}
}
\begin{flushleft}
\item[\textbf{Declaração}\hfill]
\begin{ttfamily}
public Function DoAddRec:Boolean; virtual;\end{ttfamily}


\end{flushleft}
\end{list}
\paragraph*{IfEqual}\hspace*{\fill}

\begin{list}{}{
\settowidth{\tmplength}{\textbf{Declaração}}
\setlength{\itemindent}{0cm}
\setlength{\listparindent}{0cm}
\setlength{\leftmargin}{\evensidemargin}
\addtolength{\leftmargin}{\tmplength}
\settowidth{\labelsep}{X}
\addtolength{\leftmargin}{\labelsep}
\setlength{\labelwidth}{\tmplength}
}
\begin{flushleft}
\item[\textbf{Declaração}\hfill]
\begin{ttfamily}
public Function IfEqual(Const Ofset{\_}Inicial:Word;Const PAnt,PAtu : Pointer; Const Len:Word):Boolean;\end{ttfamily}


\end{flushleft}
\par
\item[\textbf{Descrição}]
O atributo \textbf{\begin{ttfamily}IfEqual\end{ttfamily}} retorna true se o buffer apontado por PAnt for igual ao buffer apontado por PAtu.

\end{list}
\paragraph*{RecordAltered}\hspace*{\fill}

\begin{list}{}{
\settowidth{\tmplength}{\textbf{Declaração}}
\setlength{\itemindent}{0cm}
\setlength{\listparindent}{0cm}
\setlength{\leftmargin}{\evensidemargin}
\addtolength{\leftmargin}{\tmplength}
\settowidth{\labelsep}{X}
\addtolength{\leftmargin}{\labelsep}
\setlength{\labelwidth}{\tmplength}
}
\begin{flushleft}
\item[\textbf{Declaração}\hfill]
\begin{ttfamily}
public function RecordAltered: Boolean ;\end{ttfamily}


\end{flushleft}
\par
\item[\textbf{Descrição}]
O método \textbf{\begin{ttfamily}RecordAltered\end{ttfamily}} retorna true se o registro atual for diferente do registro anterior

\end{list}
\paragraph*{CreateExecAction}\hspace*{\fill}

\begin{list}{}{
\settowidth{\tmplength}{\textbf{Declaração}}
\setlength{\itemindent}{0cm}
\setlength{\listparindent}{0cm}
\setlength{\leftmargin}{\evensidemargin}
\addtolength{\leftmargin}{\tmplength}
\settowidth{\labelsep}{X}
\addtolength{\leftmargin}{\labelsep}
\setlength{\labelwidth}{\tmplength}
}
\begin{flushleft}
\item[\textbf{Declaração}\hfill]
\begin{ttfamily}
public class function CreateExecAction(Const aFieldName:AnsiString;const aExecAction: AnsiString) : AnsiString;\end{ttfamily}


\end{flushleft}
\par
\item[\textbf{Descrição}]
A classe método \textbf{\begin{ttfamily}CreateExecAction\end{ttfamily}} é usado para adicionar a chamada de um procedimento quando a tecla F7 é pressionada;

\end{list}
\paragraph*{add}\hspace*{\fill}

\begin{list}{}{
\settowidth{\tmplength}{\textbf{Declaração}}
\setlength{\itemindent}{0cm}
\setlength{\listparindent}{0cm}
\setlength{\leftmargin}{\evensidemargin}
\addtolength{\leftmargin}{\tmplength}
\settowidth{\labelsep}{X}
\addtolength{\leftmargin}{\labelsep}
\setlength{\labelwidth}{\tmplength}
}
\begin{flushleft}
\item[\textbf{Declaração}\hfill]
\begin{ttfamily}
public procedure add(aTemplate:AnsiString);\end{ttfamily}


\end{flushleft}
\end{list}
\paragraph*{SetLocked}\hspace*{\fill}

\begin{list}{}{
\settowidth{\tmplength}{\textbf{Declaração}}
\setlength{\itemindent}{0cm}
\setlength{\listparindent}{0cm}
\setlength{\leftmargin}{\evensidemargin}
\addtolength{\leftmargin}{\tmplength}
\settowidth{\labelsep}{X}
\addtolength{\leftmargin}{\labelsep}
\setlength{\labelwidth}{\tmplength}
}
\begin{flushleft}
\item[\textbf{Declaração}\hfill]
\begin{ttfamily}
protected Procedure SetLocked(aLocked:Boolean); Virtual;\end{ttfamily}


\end{flushleft}
\end{list}
\section{Tipos}
\subsection*{PSItem}
\begin{list}{}{
\settowidth{\tmplength}{\textbf{Declaração}}
\setlength{\itemindent}{0cm}
\setlength{\listparindent}{0cm}
\setlength{\leftmargin}{\evensidemargin}
\addtolength{\leftmargin}{\tmplength}
\settowidth{\labelsep}{X}
\addtolength{\leftmargin}{\labelsep}
\setlength{\labelwidth}{\tmplength}
}
\begin{flushleft}
\item[\textbf{Declaração}\hfill]
\begin{ttfamily}
PSItem =  TUiMethods.PSItem;\end{ttfamily}


\end{flushleft}
\end{list}
\subsection*{tString}
\begin{list}{}{
\settowidth{\tmplength}{\textbf{Declaração}}
\setlength{\itemindent}{0cm}
\setlength{\listparindent}{0cm}
\setlength{\leftmargin}{\evensidemargin}
\addtolength{\leftmargin}{\tmplength}
\settowidth{\labelsep}{X}
\addtolength{\leftmargin}{\labelsep}
\setlength{\labelwidth}{\tmplength}
}
\begin{flushleft}
\item[\textbf{Declaração}\hfill]
\begin{ttfamily}
tString =  TUiMethods.tString;\end{ttfamily}


\end{flushleft}
\end{list}
\subsection*{ptString}
\begin{list}{}{
\settowidth{\tmplength}{\textbf{Declaração}}
\setlength{\itemindent}{0cm}
\setlength{\listparindent}{0cm}
\setlength{\leftmargin}{\evensidemargin}
\addtolength{\leftmargin}{\tmplength}
\settowidth{\labelsep}{X}
\addtolength{\leftmargin}{\labelsep}
\setlength{\labelwidth}{\tmplength}
}
\begin{flushleft}
\item[\textbf{Declaração}\hfill]
\begin{ttfamily}
ptString = TUiMethods.Ptstring;\end{ttfamily}


\end{flushleft}
\end{list}
\subsection*{TDates}
\begin{list}{}{
\settowidth{\tmplength}{\textbf{Declaração}}
\setlength{\itemindent}{0cm}
\setlength{\listparindent}{0cm}
\setlength{\leftmargin}{\evensidemargin}
\addtolength{\leftmargin}{\tmplength}
\settowidth{\labelsep}{X}
\addtolength{\leftmargin}{\labelsep}
\setlength{\labelwidth}{\tmplength}
}
\begin{flushleft}
\item[\textbf{Declaração}\hfill]
\begin{ttfamily}
TDates = TUiMethods.TDates;\end{ttfamily}


\end{flushleft}
\end{list}
\subsection*{PValue}
\begin{list}{}{
\settowidth{\tmplength}{\textbf{Declaração}}
\setlength{\itemindent}{0cm}
\setlength{\listparindent}{0cm}
\setlength{\leftmargin}{\evensidemargin}
\addtolength{\leftmargin}{\tmplength}
\settowidth{\labelsep}{X}
\addtolength{\leftmargin}{\labelsep}
\setlength{\labelwidth}{\tmplength}
}
\begin{flushleft}
\item[\textbf{Declaração}\hfill]
\begin{ttfamily}
PValue = TUiMethods.PValue;\end{ttfamily}


\end{flushleft}
\end{list}
\subsection*{TValue}
\begin{list}{}{
\settowidth{\tmplength}{\textbf{Declaração}}
\setlength{\itemindent}{0cm}
\setlength{\listparindent}{0cm}
\setlength{\leftmargin}{\evensidemargin}
\addtolength{\leftmargin}{\tmplength}
\settowidth{\labelsep}{X}
\addtolength{\leftmargin}{\labelsep}
\setlength{\labelwidth}{\tmplength}
}
\begin{flushleft}
\item[\textbf{Declaração}\hfill]
\begin{ttfamily}
TValue = TUiMethods.TValue;\end{ttfamily}


\end{flushleft}
\end{list}
\subsection*{TOnGetTemplate}
\begin{list}{}{
\settowidth{\tmplength}{\textbf{Declaração}}
\setlength{\itemindent}{0cm}
\setlength{\listparindent}{0cm}
\setlength{\leftmargin}{\evensidemargin}
\addtolength{\leftmargin}{\tmplength}
\settowidth{\labelsep}{X}
\addtolength{\leftmargin}{\labelsep}
\setlength{\labelwidth}{\tmplength}
}
\begin{flushleft}
\item[\textbf{Declaração}\hfill]
\begin{ttfamily}
TOnGetTemplate = function (aNext: PSItem) : PSItem of Object unimplemented;\end{ttfamily}


\end{flushleft}
\par
\item[\textbf{Descrição}]
Usado para criar modelos de formulários dinamicamente usando como parâmetro listas de PSItems.

\end{list}
\subsection*{TOnAddTemplate}
\begin{list}{}{
\settowidth{\tmplength}{\textbf{Declaração}}
\setlength{\itemindent}{0cm}
\setlength{\listparindent}{0cm}
\setlength{\leftmargin}{\evensidemargin}
\addtolength{\leftmargin}{\tmplength}
\settowidth{\labelsep}{X}
\addtolength{\leftmargin}{\labelsep}
\setlength{\labelwidth}{\tmplength}
}
\begin{flushleft}
\item[\textbf{Declaração}\hfill]
\begin{ttfamily}
TOnAddTemplate = Procedure(const aUiDmxScroller:TUiDmxScroller) of Object unimplemented;\end{ttfamily}


\end{flushleft}
\par
\item[\textbf{Descrição}]
O tipo \begin{ttfamily}TOnAddTemplate\end{ttfamily} é usado para criar modelos de formulários dinamicamente usando o método add

\begin{itemize}
\item \textbf{EXEMPLO}

\texttt{\\\nopagebreak[3]
\\\nopagebreak[3]
}\textbf{Procedure}\texttt{~AddTemplate(}\textbf{const}\texttt{~aUiDmxScroller:TUiDmxScroller);\\\nopagebreak[3]
}\textbf{begin}\texttt{\\\nopagebreak[3]
~~}\textbf{with}\texttt{~aUiDmxScroller~}\textbf{do}\texttt{\\\nopagebreak[3]
~~}\textbf{begin}\texttt{\\\nopagebreak[3]
~~~~add('~EXEMPLO~DE~TEMPLATE~');\\\nopagebreak[3]
~~~~add('');\\\nopagebreak[3]
~~~~add('~Alfanumérico~maiúscula~com~15~posições:~{\textbackslash}SSSSSSSSSSSSSSS');\\\nopagebreak[3]
~~~~add('~Alfanumérico~maiúscula~e~minuscula~com~30~posições:~');\\\nopagebreak[3]
~~~~add('~~{\textbackslash}ssssssssssssssssssssssssssssssssssssss');\\\nopagebreak[3]
~~~~add('~Alfanumérico~com~a~primeira~letra~maiúscula:~{\textbackslash}Sssssssssssssss');\\\nopagebreak[3]
~~~~add('~Valor~double.......:~{\textbackslash}RRR,RRR.RR');\\\nopagebreak[3]
~~~~add('~Valor~SmalInt......:~{\textbackslash}II,III');\\\nopagebreak[3]
~~~~add('~Valor~Byte.........:~{\textbackslash}BBB');\\\nopagebreak[3]
~~~~add('~Valor~Smallword....:~{\textbackslash}WW,WWW');\\\nopagebreak[3]
~~~~add('~Sexo...............:~'+~CreateEnumField(TRUE,~accNormal,~0,\\\nopagebreak[3]
~~~~~~~~~~~~~~~~~~~~~~~~~~~~~~~~~~NewSItem('~indefinido~',\\\nopagebreak[3]
~~~~~~~~~~~~~~~~~~~~~~~~~~~~~~~~~~NewSItem('~Masculino',\\\nopagebreak[3]
~~~~~~~~~~~~~~~~~~~~~~~~~~~~~~~~~~NewSItem('~Feminino',\\\nopagebreak[3]
~~~~~~~~~~~~~~~~~~~~~~~~~~~~~~~~~~~~~~~~~~}\textbf{nil}\texttt{)))));\\\nopagebreak[3]
~~~~add('~Estado~Civil~~~~~~~~~~~~~~~{\textbackslash}KA~Indefinido~~'+chFN+'Sexo');\\\nopagebreak[3]
~~~~add('~~{\textbackslash}X~Casado?~~~~~~~~~~~~~~~~{\textbackslash}KA~Masculino~~~~');\\\nopagebreak[3]
~~~~add('~~{\textbackslash}X~Pretende~se~divorciar?~{\textbackslash}KA~Feminino~~~~~');\\\nopagebreak[3]
~~~~add('~~{\textbackslash}X~Tens~filhos?~~~~~~~~~~');\\\nopagebreak[3]
~~~~add('');\\\nopagebreak[3]
~~}\textbf{end}\texttt{;\\\nopagebreak[3]
}\textbf{end}\texttt{;\\\nopagebreak[3]
\\\nopagebreak[3]
}\textbf{procedure}\texttt{~TForm1.DmxScroller{\_}Form1AddTemplate(}\textbf{const}\texttt{~aUiDmxScroller:~TUiDmxScroller);\\\nopagebreak[3]
}\textbf{begin}\texttt{\\\nopagebreak[3]
~~AddTemplate(aUiDmxScroller);\\\nopagebreak[3]
}\textbf{end}\texttt{;\\
}
\end{itemize}

\end{list}
\subsection*{pDmxFieldRec}
\begin{list}{}{
\settowidth{\tmplength}{\textbf{Declaração}}
\setlength{\itemindent}{0cm}
\setlength{\listparindent}{0cm}
\setlength{\leftmargin}{\evensidemargin}
\addtolength{\leftmargin}{\tmplength}
\settowidth{\labelsep}{X}
\addtolength{\leftmargin}{\labelsep}
\setlength{\labelwidth}{\tmplength}
}
\begin{flushleft}
\item[\textbf{Declaração}\hfill]
\begin{ttfamily}
pDmxFieldRec = {\^{}}TDmxFieldRec;\end{ttfamily}


\end{flushleft}
\par
\item[\textbf{Descrição}]
O tipo \textbf{\begin{ttfamily}pDmxFieldRec\end{ttfamily}} aponta para o campo do tipo \begin{ttfamily}TDmxFieldRec\end{ttfamily}(\ref{mi_rtl_ui_Dmxscroller.TDmxFieldRec})

\end{list}
\subsection*{TEndProc}
\begin{list}{}{
\settowidth{\tmplength}{\textbf{Declaração}}
\setlength{\itemindent}{0cm}
\setlength{\listparindent}{0cm}
\setlength{\leftmargin}{\evensidemargin}
\addtolength{\leftmargin}{\tmplength}
\settowidth{\labelsep}{X}
\addtolength{\leftmargin}{\labelsep}
\setlength{\labelwidth}{\tmplength}
}
\begin{flushleft}
\item[\textbf{Declaração}\hfill]
\begin{ttfamily}
TEndProc = Procedure(Const AOwner:TUiDmxScroller; Const ADmxFieldRec:PDmxFieldRec);\end{ttfamily}


\end{flushleft}
\par
\item[\textbf{Descrição}]
O tipo \textbf{\begin{ttfamily}TEndProc\end{ttfamily}} é usado para fazer pesquisa genérica no banco de dados quando a tecla F7 é pressionada.

\end{list}
\subsection*{TOnEnter}
\begin{list}{}{
\settowidth{\tmplength}{\textbf{Declaração}}
\setlength{\itemindent}{0cm}
\setlength{\listparindent}{0cm}
\setlength{\leftmargin}{\evensidemargin}
\addtolength{\leftmargin}{\tmplength}
\settowidth{\labelsep}{X}
\addtolength{\leftmargin}{\labelsep}
\setlength{\labelwidth}{\tmplength}
}
\begin{flushleft}
\item[\textbf{Declaração}\hfill]
\begin{ttfamily}
TOnEnter = Procedure(aDmxScroller:TUiDmxScroller) of Object;\end{ttfamily}


\end{flushleft}
\par
\item[\textbf{Descrição}]
O tipo \textbf{\begin{ttfamily}TOnEnter\end{ttfamily}} é usado para implementar evento onEnter da classe \begin{ttfamily}TUiDmxScroller\end{ttfamily}(\ref{mi_rtl_ui_Dmxscroller.TUiDmxScroller})

\end{list}
\subsection*{TOnExit}
\begin{list}{}{
\settowidth{\tmplength}{\textbf{Declaração}}
\setlength{\itemindent}{0cm}
\setlength{\listparindent}{0cm}
\setlength{\leftmargin}{\evensidemargin}
\addtolength{\leftmargin}{\tmplength}
\settowidth{\labelsep}{X}
\addtolength{\leftmargin}{\labelsep}
\setlength{\labelwidth}{\tmplength}
}
\begin{flushleft}
\item[\textbf{Declaração}\hfill]
\begin{ttfamily}
TOnExit = Procedure(aDmxScroller:TUiDmxScroller) of Object;\end{ttfamily}


\end{flushleft}
\par
\item[\textbf{Descrição}]
O tipo \textbf{\begin{ttfamily}TOnExit\end{ttfamily}} é usado para implementar evento onExit da classe \begin{ttfamily}TUiDmxScroller\end{ttfamily}(\ref{mi_rtl_ui_Dmxscroller.TUiDmxScroller})

\end{list}
\subsection*{TOnNewRecord}
\begin{list}{}{
\settowidth{\tmplength}{\textbf{Declaração}}
\setlength{\itemindent}{0cm}
\setlength{\listparindent}{0cm}
\setlength{\leftmargin}{\evensidemargin}
\addtolength{\leftmargin}{\tmplength}
\settowidth{\labelsep}{X}
\addtolength{\leftmargin}{\labelsep}
\setlength{\labelwidth}{\tmplength}
}
\begin{flushleft}
\item[\textbf{Declaração}\hfill]
\begin{ttfamily}
TOnNewRecord = Procedure(aDmxScroller:TUiDmxScroller) of Object;\end{ttfamily}


\end{flushleft}
\par
\item[\textbf{Descrição}]
O tipo \textbf{\begin{ttfamily}TOnNewRecord\end{ttfamily}} é usado para implementar evento onNewRecord da classe \begin{ttfamily}TUiDmxScroller\end{ttfamily}(\ref{mi_rtl_ui_Dmxscroller.TUiDmxScroller})

\end{list}
\subsection*{TOnCloseQuery}
\begin{list}{}{
\settowidth{\tmplength}{\textbf{Declaração}}
\setlength{\itemindent}{0cm}
\setlength{\listparindent}{0cm}
\setlength{\leftmargin}{\evensidemargin}
\addtolength{\leftmargin}{\tmplength}
\settowidth{\labelsep}{X}
\addtolength{\leftmargin}{\labelsep}
\setlength{\labelwidth}{\tmplength}
}
\begin{flushleft}
\item[\textbf{Declaração}\hfill]
\begin{ttfamily}
TOnCloseQuery = Procedure(aDmxScroller:TUiDmxScroller; var CanClose:boolean) of Object;\end{ttfamily}


\end{flushleft}
\par
\item[\textbf{Descrição}]
O tipo \textbf{\begin{ttfamily}TOnCloseQuery\end{ttfamily}} é usado para implementar evento OnCloseQuery da classe \begin{ttfamily}TUiDmxScroller\end{ttfamily}(\ref{mi_rtl_ui_Dmxscroller.TUiDmxScroller})

\begin{itemize}
\item \textbf{NOTA* \begin{itemize}
\item Este evento é disparado antes de desativar a classe **\begin{ttfamily}TUiDmxScroller\end{ttfamily}(\ref{mi_rtl_ui_Dmxscroller.TUiDmxScroller})
\end{itemize}}. \begin{itemize}
\item Obs: Se o parâmetro \textbf{CanClose} for \textbf{false}, então a classe \textbf{\begin{ttfamily}TUiDmxScroller\end{ttfamily}(\ref{mi_rtl_ui_Dmxscroller.TUiDmxScroller})} não é desativado.
\end{itemize}
\end{itemize}

\end{list}
\subsection*{TOnEnterField}
\begin{list}{}{
\settowidth{\tmplength}{\textbf{Declaração}}
\setlength{\itemindent}{0cm}
\setlength{\listparindent}{0cm}
\setlength{\leftmargin}{\evensidemargin}
\addtolength{\leftmargin}{\tmplength}
\settowidth{\labelsep}{X}
\addtolength{\leftmargin}{\labelsep}
\setlength{\labelwidth}{\tmplength}
}
\begin{flushleft}
\item[\textbf{Declaração}\hfill]
\begin{ttfamily}
TOnEnterField = Procedure(aField:pDmxFieldRec) of Object;\end{ttfamily}


\end{flushleft}
\par
\item[\textbf{Descrição}]
O tipo \textbf{\begin{ttfamily}TOnEnterField\end{ttfamily}} é usado no evento OnEnterField

\end{list}
\subsection*{TOnExitField}
\begin{list}{}{
\settowidth{\tmplength}{\textbf{Declaração}}
\setlength{\itemindent}{0cm}
\setlength{\listparindent}{0cm}
\setlength{\leftmargin}{\evensidemargin}
\addtolength{\leftmargin}{\tmplength}
\settowidth{\labelsep}{X}
\addtolength{\leftmargin}{\labelsep}
\setlength{\labelwidth}{\tmplength}
}
\begin{flushleft}
\item[\textbf{Declaração}\hfill]
\begin{ttfamily}
TOnExitField = Procedure(aField:pDmxFieldRec) of Object;\end{ttfamily}


\end{flushleft}
\par
\item[\textbf{Descrição}]
O tipo \textbf{\begin{ttfamily}TOnExitField\end{ttfamily}} é usado no evento OnExitField

\end{list}
\subsection*{SmallWord}
\begin{list}{}{
\settowidth{\tmplength}{\textbf{Declaração}}
\setlength{\itemindent}{0cm}
\setlength{\listparindent}{0cm}
\setlength{\leftmargin}{\evensidemargin}
\addtolength{\leftmargin}{\tmplength}
\settowidth{\labelsep}{X}
\addtolength{\leftmargin}{\labelsep}
\setlength{\labelwidth}{\tmplength}
}
\begin{flushleft}
\item[\textbf{Declaração}\hfill]
\begin{ttfamily}
SmallWord  = TUiDmxScroller.SmallWord;\end{ttfamily}


\end{flushleft}
\end{list}
\section{Constantes}
\subsection*{AccNormal}
\begin{list}{}{
\settowidth{\tmplength}{\textbf{Declaração}}
\setlength{\itemindent}{0cm}
\setlength{\listparindent}{0cm}
\setlength{\leftmargin}{\evensidemargin}
\addtolength{\leftmargin}{\tmplength}
\settowidth{\labelsep}{X}
\addtolength{\leftmargin}{\labelsep}
\setlength{\labelwidth}{\tmplength}
}
\begin{flushleft}
\item[\textbf{Declaração}\hfill]
\begin{ttfamily}
AccNormal  = TUiMethods.AccNormal;\end{ttfamily}


\end{flushleft}
\end{list}
\subsection*{LF}
\begin{list}{}{
\settowidth{\tmplength}{\textbf{Declaração}}
\setlength{\itemindent}{0cm}
\setlength{\listparindent}{0cm}
\setlength{\leftmargin}{\evensidemargin}
\addtolength{\leftmargin}{\tmplength}
\settowidth{\labelsep}{X}
\addtolength{\leftmargin}{\labelsep}
\setlength{\labelwidth}{\tmplength}
}
\begin{flushleft}
\item[\textbf{Declaração}\hfill]
\begin{ttfamily}
LF        = TConsts.LF;\end{ttfamily}


\end{flushleft}
\end{list}
\chapter{Unit mi{\_}rtl{\_}ui{\_}DmxScroller{\_}Buttons}
\section{Descrição}
A unit \textbf{\begin{ttfamily}mi{\_}rtl{\_}ui{\_}DmxScroller{\_}Buttons\end{ttfamily}} implementa a classe \begin{ttfamily}TUiDmxScroller{\_}Buttons\end{ttfamily}(\ref{mi_rtl_ui_DmxScroller_Buttons.TUiDmxScroller_Buttons}).

\begin{itemize}
\item \textbf{VERSÃO} \begin{itemize}
\item Alpha {-} 0.5.0.687
\end{itemize}
\item \textbf{CÓDIGO FONTE}: \begin{itemize}
\item 
\item \textbf{PENDÊNCIAS}
\item \textbf{CONCLUÍDO} \begin{itemize}
\item Criar classe \begin{ttfamily}TUiDmxScroller{\_}Buttons\end{ttfamily}(\ref{mi_rtl_ui_DmxScroller_Buttons.TUiDmxScroller_Buttons}) ✅️
\item Criar constructor Create.✅️
\item Criar Function Create{\_}RCommands. ✅️
\item Commands{\_}Buttons{\_}High : Byte; ✅️
\item Commands{\_}Buttons : Array[0..Max{\_}List{\_}Buttons] of TRCommand; ✅️
\item Max{\_}List{\_}Buttons = sizeof(Longint); ✅️
\item Commands{\_}Buttons{\_}Mb : Longint; ✅️
\item Function Add{\_}RCommands{\_}Buttons; ✅️
\item Function Create{\_}RCommands{\_}Buttons; ✅️
\item Function Set{\_}Commands{\_}Buttons{\_}Mb(Const aMb{\_}Bits:Longint):Longint;✅️
\item Documentar os atributos abaixo: ️ \begin{itemize}
\item Commands{\_}Buttons{\_}High : Byte; ✅️
\item Commands{\_}Buttons : Array[0..Max{\_}List{\_}Buttons] of TRCommand; ✅️
\end{itemize}
\end{itemize}
\item \textbf{HISTÓRICOS} \begin{itemize}
\item \textbf{DIAS ANTERIORES} \begin{itemize}
\item 
\end{itemize}
\item \textbf{DO DIA} \begin{itemize}
\item \textbf{2022{-}07{-}07} \begin{itemize}
\item \textbf{09:45} \begin{itemize}
\item Documentar o método \textbf{Add{\_}RCommands{\_}Buttons}
\end{itemize}
\item \textbf{14:55} \begin{itemize}
\item Documentar o método \textbf{Add{\_}RCommands{\_}Buttons}
\end{itemize}
\end{itemize}
\end{itemize}
\end{itemize}
\end{itemize}
\end{itemize}
\section{Uses}
\begin{itemize}
\item \begin{ttfamily}Classes\end{ttfamily}\item \begin{ttfamily}SysUtils\end{ttfamily}\item \begin{ttfamily}mi{\_}rtl{\_}ui{\_}methods\end{ttfamily}(\ref{mi_rtl_ui_methods})\item \begin{ttfamily}mi{\_}rtl{\_}ui{\_}Dmxscroller\end{ttfamily}(\ref{mi_rtl_ui_Dmxscroller})\end{itemize}
\section{Visão Geral}
\begin{description}
\item[\texttt{\begin{ttfamily}TUiDmxScroller{\_}Buttons\end{ttfamily} Classe}]
\end{description}
\section{Classes, Interfaces, Objetos e Registros}
\subsection*{TUiDmxScroller{\_}Buttons Classe}
\subsubsection*{\large{\textbf{Hierarquia}}\normalsize\hspace{1ex}\hfill}
TUiDmxScroller{\_}Buttons {$>$} \begin{ttfamily}TUiMethods\end{ttfamily}(\ref{mi_rtl_ui_methods.TUiMethods}) {$>$} 
TUiConsts
\subsubsection*{\large{\textbf{Descrição}}\normalsize\hspace{1ex}\hfill}
A classe \textbf{\begin{ttfamily}TUiDmxScroller{\_}Buttons\end{ttfamily}} tem como objetivo registrar os dados necessários para criar os botões de navegação e edição de uma tabela quando TDataSource for {$<$}{$>$} nil.

\begin{itemize}
\item \textbf{EXEMPLO USO}

\texttt{}
\end{itemize}\subsubsection*{\large{\textbf{Propriedades}}\normalsize\hspace{1ex}\hfill}
\paragraph*{Commands{\_}Buttons{\_}High}\hspace*{\fill}

\begin{list}{}{
\settowidth{\tmplength}{\textbf{Declaração}}
\setlength{\itemindent}{0cm}
\setlength{\listparindent}{0cm}
\setlength{\leftmargin}{\evensidemargin}
\addtolength{\leftmargin}{\tmplength}
\settowidth{\labelsep}{X}
\addtolength{\leftmargin}{\labelsep}
\setlength{\labelwidth}{\tmplength}
}
\begin{flushleft}
\item[\textbf{Declaração}\hfill]
\begin{ttfamily}
public property Commands{\_}Buttons{\_}High : Byte Read {\_}Commands{\_}Buttons{\_}High;\end{ttfamily}


\end{flushleft}
\par
\item[\textbf{Descrição}]
A propriedade \textbf{\begin{ttfamily}Commands{\_}Buttons{\_}High\end{ttfamily}} contém o número de linhas inicializadas da matriz \textbf{\begin{ttfamily}Commands{\_}Buttons\end{ttfamily}(\ref{mi_rtl_ui_DmxScroller_Buttons.TUiDmxScroller_Buttons-Commands_Buttons})}, ou seja: é igual o número de linhas criadas em: \textbf{\begin{ttfamily}Create{\_}RCommands{\_}Buttons\end{ttfamily}(\ref{mi_rtl_ui_DmxScroller_Buttons.TUiDmxScroller_Buttons-Create_RCommands_Buttons})}.

\end{list}
\paragraph*{Commands{\_}Buttons{\_}Mb}\hspace*{\fill}

\begin{list}{}{
\settowidth{\tmplength}{\textbf{Declaração}}
\setlength{\itemindent}{0cm}
\setlength{\listparindent}{0cm}
\setlength{\leftmargin}{\evensidemargin}
\addtolength{\leftmargin}{\tmplength}
\settowidth{\labelsep}{X}
\addtolength{\leftmargin}{\labelsep}
\setlength{\labelwidth}{\tmplength}
}
\begin{flushleft}
\item[\textbf{Declaração}\hfill]
\begin{ttfamily}
public property Commands{\_}Buttons{\_}Mb  : Longint read {\_}Commands{\_}Buttons{\_}Mb;\end{ttfamily}


\end{flushleft}
\par
\item[\textbf{Descrição}]
O atributo \textbf{\begin{ttfamily}Commands{\_}Buttons{\_}Mb\end{ttfamily}} contém o mapa de bits dos botões que serão criados no formulário.

\begin{itemize}
\item \textbf{NOTA} \begin{itemize}
\item O mapa de bits é do tipo longint (4 bytes) por isso pode conter no máximo (4x8=32) botões.
\end{itemize}
\end{itemize}

\end{list}
\subsubsection*{\large{\textbf{Campos}}\normalsize\hspace{1ex}\hfill}
\paragraph*{UiDmxScroller}\hspace*{\fill}

\begin{list}{}{
\settowidth{\tmplength}{\textbf{Declaração}}
\setlength{\itemindent}{0cm}
\setlength{\listparindent}{0cm}
\setlength{\leftmargin}{\evensidemargin}
\addtolength{\leftmargin}{\tmplength}
\settowidth{\labelsep}{X}
\addtolength{\leftmargin}{\labelsep}
\setlength{\labelwidth}{\tmplength}
}
\begin{flushleft}
\item[\textbf{Declaração}\hfill]
\begin{ttfamily}
public UiDmxScroller: TUiDmxScroller;\end{ttfamily}


\end{flushleft}
\par
\item[\textbf{Descrição}]
O atributo \textbf{\begin{ttfamily}UiDmxScroller\end{ttfamily}} deve ser passado em constructor \begin{ttfamily}create\end{ttfamily}(\ref{mi_rtl_ui_DmxScroller_Buttons.TUiDmxScroller_Buttons-Create})

\end{list}
\paragraph*{OkCmmNewRecord}\hspace*{\fill}

\begin{list}{}{
\settowidth{\tmplength}{\textbf{Declaração}}
\setlength{\itemindent}{0cm}
\setlength{\listparindent}{0cm}
\setlength{\leftmargin}{\evensidemargin}
\addtolength{\leftmargin}{\tmplength}
\settowidth{\labelsep}{X}
\addtolength{\leftmargin}{\labelsep}
\setlength{\labelwidth}{\tmplength}
}
\begin{flushleft}
\item[\textbf{Declaração}\hfill]
\begin{ttfamily}
public Var OkCmmNewRecord: boolean;\end{ttfamily}


\end{flushleft}
\par
\item[\textbf{Descrição}]
O atributo \textbf{\begin{ttfamily}OkCmmNewRecord\end{ttfamily}} indica se o registro pode ser incluído ou não, ou seja: é o estado inicial da ação incluir informada pelo usuário.

\begin{itemize}
\item \textbf{NOTA} \begin{itemize}
\item True : O registro pode ser incluído. Obs: DataSet.Append pode ser executado.
\item False: O registro não pode ser incluído. Obs: DataSet.Append não pode ser executado.
\item Esse atributo é usado nos seguintes métodos: \begin{itemize}
\item Create{\_}RCommands{\_}Edit
\item No método DoOnNewRecord
\item Action Novo
\end{itemize}
\end{itemize}
\item \textbf{EXEMPLO}

\texttt{\\\nopagebreak[3]
\textit{//~Tirado~do~código:~Function~TRecord.Create{\_}RCommands{\_}Edit}\\\nopagebreak[3]
\\\nopagebreak[3]
OkCmmNewRecord~:=~Application.FileOptions{\_}CommandEnabled(Module,aCmNovo);\\\nopagebreak[3]
}\textbf{if}\texttt{~aCmNovo{$<$}=255~}\textbf{then}\texttt{\\\nopagebreak[3]
~~}\textbf{if}\texttt{~OkCmmNewRecord\\\nopagebreak[3]
~~}\textbf{then}\texttt{~Application.EnableCommands([aCmNovo])\\\nopagebreak[3]
~~}\textbf{Else}\texttt{~Application.DisableCommands([aCmNovo]);\\
}
\end{itemize}

\end{list}
\paragraph*{OkCmmDbLocaliza}\hspace*{\fill}

\begin{list}{}{
\settowidth{\tmplength}{\textbf{Declaração}}
\setlength{\itemindent}{0cm}
\setlength{\listparindent}{0cm}
\setlength{\leftmargin}{\evensidemargin}
\addtolength{\leftmargin}{\tmplength}
\settowidth{\labelsep}{X}
\addtolength{\leftmargin}{\labelsep}
\setlength{\labelwidth}{\tmplength}
}
\begin{flushleft}
\item[\textbf{Declaração}\hfill]
\begin{ttfamily}
public Var OkCmmDbLocaliza: boolean;\end{ttfamily}


\end{flushleft}
\par
\item[\textbf{Descrição}]
O atributo \textbf{\begin{ttfamily}OkCmmDbLocaliza\end{ttfamily}} indica se o registro pode ser localizado ou não, ou seja: é o estado inicial da ação \textbf{pesquisar} informada pelo usuário.

\begin{itemize}
\item \textbf{NOTA} \begin{itemize}
\item True : O registro pode ser pesquisado. DataSet.Locate pode ser executado.
\item False: O registro não pode ser pesquisado. DataSet.Locate não pode ser executado.
\item Esse atributo é usado nos seguintes métodos: \begin{itemize}
\item Create{\_}RCommands{\_}Edit
\item No método DoOnNewRecord
\item Action Pesquisa
\end{itemize}
\end{itemize}
\item \textbf{EXEMPLO}

\texttt{\\\nopagebreak[3]
\textit{//~Tirado~do~código:~Function~TRecord.Create{\_}RCommands{\_}Edit}\\\nopagebreak[3]
\\\nopagebreak[3]
}\textbf{if}\texttt{~OkCmmNewRecord~}\textbf{or}\texttt{~OkCmmEvaluateRecord~}\textbf{or}\texttt{~OkCmmZeroizeRecord\\\nopagebreak[3]
}\textbf{then}\texttt{~}\textbf{begin}\texttt{\\\nopagebreak[3]
~~~~~~~self.OkCmmDbLocaliza~:=~true;\\\nopagebreak[3]
~~~~~~~self.Locked~:=~false;\\\nopagebreak[3]
~~~~~}\textbf{end}\texttt{\\\nopagebreak[3]
}\textbf{else}\texttt{~}\textbf{begin}\texttt{\\\nopagebreak[3]
~~~~~~~~self.OkCmmDbLocaliza~:=~Application.FileOptions{\_}CommandEnabled(Module,ACmLocaliza);\\\nopagebreak[3]
~~~~~~~~self.Locked~:=~True;\\\nopagebreak[3]
~~~~~~}\textbf{end}\texttt{;\\
}
\end{itemize}

\end{list}
\paragraph*{OkCmmZeroizeRecord}\hspace*{\fill}

\begin{list}{}{
\settowidth{\tmplength}{\textbf{Declaração}}
\setlength{\itemindent}{0cm}
\setlength{\listparindent}{0cm}
\setlength{\leftmargin}{\evensidemargin}
\addtolength{\leftmargin}{\tmplength}
\settowidth{\labelsep}{X}
\addtolength{\leftmargin}{\labelsep}
\setlength{\labelwidth}{\tmplength}
}
\begin{flushleft}
\item[\textbf{Declaração}\hfill]
\begin{ttfamily}
public Var OkCmmZeroizeRecord: boolean;\end{ttfamily}


\end{flushleft}
\par
\item[\textbf{Descrição}]
O atributo \textbf{\begin{ttfamily}OkCmmZeroizeRecord\end{ttfamily}} indica se o registro pode ser excluído ou não, ou seja: é o estado inicial da ação excluir informada pelo usuário.

\begin{itemize}
\item \textbf{NOTA} \begin{itemize}
\item \textbf{True} : O registro pode ser deletado.
\item \textbf{False}: O registro não pode ser deletado.
\item Esse atributo é usado nos seguintes métodos: \begin{itemize}
\item Create{\_}RCommands{\_}Edit
\item Nos métodos DeleteRec
\item Action Delete
\end{itemize}
\end{itemize}
\item \textbf{EXEMPLO}

\texttt{\\\nopagebreak[3]
\textit{//~Tirado~do~código:~Function~TRecord.Create{\_}RCommands{\_}Edit}\\\nopagebreak[3]
\\\nopagebreak[3]
OkCmmZeroizeRecord~~:=~Application.FileOptions{\_}CommandEnabled(Module,ACmExclusao);\\\nopagebreak[3]
}\textbf{if}\texttt{~ACmExclusao{$<$}=255~}\textbf{then}\texttt{\\\nopagebreak[3]
~~}\textbf{if}\texttt{~OkCmmZeroizeRecord\\\nopagebreak[3]
~~}\textbf{then}\texttt{~Application.EnableCommands([ACmExclusao])\\\nopagebreak[3]
~~}\textbf{Else}\texttt{~Application.DisableCommands([ACmExclusao]);\\
}
\end{itemize}

\end{list}
\paragraph*{Max{\_}List{\_}Buttons}\hspace*{\fill}

\begin{list}{}{
\settowidth{\tmplength}{\textbf{Declaração}}
\setlength{\itemindent}{0cm}
\setlength{\listparindent}{0cm}
\setlength{\leftmargin}{\evensidemargin}
\addtolength{\leftmargin}{\tmplength}
\settowidth{\labelsep}{X}
\addtolength{\leftmargin}{\labelsep}
\setlength{\labelwidth}{\tmplength}
}
\begin{flushleft}
\item[\textbf{Declaração}\hfill]
\begin{ttfamily}
public const Max{\_}List{\_}Buttons = sizeof(Longint)*8;\end{ttfamily}


\end{flushleft}
\par
\item[\textbf{Descrição}]
A constante \textbf{\begin{ttfamily}Max{\_}List{\_}Buttons\end{ttfamily}} contém o número máximo de comandos da matriz \begin{ttfamily}Commands{\_}Buttons\end{ttfamily}(\ref{mi_rtl_ui_DmxScroller_Buttons.TUiDmxScroller_Buttons-Commands_Buttons})

\end{list}
\paragraph*{Commands{\_}Buttons}\hspace*{\fill}

\begin{list}{}{
\settowidth{\tmplength}{\textbf{Declaração}}
\setlength{\itemindent}{0cm}
\setlength{\listparindent}{0cm}
\setlength{\leftmargin}{\evensidemargin}
\addtolength{\leftmargin}{\tmplength}
\settowidth{\labelsep}{X}
\addtolength{\leftmargin}{\labelsep}
\setlength{\labelwidth}{\tmplength}
}
\begin{flushleft}
\item[\textbf{Declaração}\hfill]
\begin{ttfamily}
public Commands{\_}Buttons: Array[0..Max{\_}List{\_}Buttons] of TRCommand;\end{ttfamily}


\end{flushleft}
\par
\item[\textbf{Descrição}]
O atributo \textbf{\begin{ttfamily}Commands{\_}Buttons\end{ttfamily}} contém os dados necessários para criar os botões de ações da classe de acesso a arquivos.

\begin{itemize}
\item \textbf{EXEMPLO DE USO DESTA MATRIZ}

\texttt{\\\nopagebreak[3]
Commands{\_}Buttons[1]~:=~Create{\_}RCommand(CmOk,'Ok'~~,'',KbEnter,AHelpCtx,Flag){\^{}};\\\nopagebreak[3]
Commands{\_}Buttons[2]~:=~Create{\_}RCommand(CmOk,'Next','',Kbno~~~,AHelpCtx,Flag)^;\\
}
\end{itemize}

\end{list}
\subsubsection*{\large{\textbf{Métodos}}\normalsize\hspace{1ex}\hfill}
\paragraph*{Create}\hspace*{\fill}

\begin{list}{}{
\settowidth{\tmplength}{\textbf{Declaração}}
\setlength{\itemindent}{0cm}
\setlength{\listparindent}{0cm}
\setlength{\leftmargin}{\evensidemargin}
\addtolength{\leftmargin}{\tmplength}
\settowidth{\labelsep}{X}
\addtolength{\leftmargin}{\labelsep}
\setlength{\labelwidth}{\tmplength}
}
\begin{flushleft}
\item[\textbf{Declaração}\hfill]
\begin{ttfamily}
public constructor Create(aOwner:TComponent); Override;\end{ttfamily}


\end{flushleft}
\par
\item[\textbf{Descrição}]
O construtor \textbf{\begin{ttfamily}Create\end{ttfamily}} é usado para iniciar o atributo \textbf{{\_}UiDmxScroller} com o cast (aOwner as \begin{ttfamily}TUiDmxScroller\end{ttfamily}(\ref{mi_rtl_ui_Dmxscroller.TUiDmxScroller}))

\end{list}
\paragraph*{Set{\_}Commands{\_}Buttons{\_}Mb}\hspace*{\fill}

\begin{list}{}{
\settowidth{\tmplength}{\textbf{Declaração}}
\setlength{\itemindent}{0cm}
\setlength{\listparindent}{0cm}
\setlength{\leftmargin}{\evensidemargin}
\addtolength{\leftmargin}{\tmplength}
\settowidth{\labelsep}{X}
\addtolength{\leftmargin}{\labelsep}
\setlength{\labelwidth}{\tmplength}
}
\begin{flushleft}
\item[\textbf{Declaração}\hfill]
\begin{ttfamily}
public Function Set{\_}Commands{\_}Buttons{\_}Mb(Const aMb{\_}Bits:Longint):Longint;\end{ttfamily}


\end{flushleft}
\par
\item[\textbf{Descrição}]
O Método \textbf{\begin{ttfamily}Set{\_}Commands{\_}Buttons{\_}Mb\end{ttfamily}} seta {\_}Commands{\_}Buttons{\_}Mb e retorna o mapa de bits \textbf{\begin{ttfamily}Commands{\_}Buttons{\_}Mb\end{ttfamily}(\ref{mi_rtl_ui_DmxScroller_Buttons.TUiDmxScroller_Buttons-Commands_Buttons_Mb})} anterior.

\textbf{EXEMPLO DE USO}

\texttt{\\\nopagebreak[3]
\\\nopagebreak[3]
\textit{//***~Seta~as~propriedades~do~fornecedor~***}\\\nopagebreak[3]
}\textbf{With}\texttt{~ArqFornecedor~}\textbf{do}\texttt{\\\nopagebreak[3]
}\textbf{Begin}\texttt{\\\nopagebreak[3]
~~Alias~:=~sgc('Parâmetros~para~pesquisa~de~duplicatas');\\\nopagebreak[3]
~~SetExpandable(False);~\textit{//Não~permite~Inclusões}\\\nopagebreak[3]
~~SetLocked(False);~\textit{//false~=~Não~travado~porque~a~janela~filha~pode~ser~alterada~e~expandida}\\\nopagebreak[3]
~~SetOkWriteRec(False);~~\textit{//Desabilita~a~alteração.}\\\nopagebreak[3]
~~Set{\_}Commands{\_}Buttons{\_}Mb(Mb{\_}Cm{\_}Bof{\_}Prev{\_}Next{\_}Eof);\\\nopagebreak[3]
}\textbf{end}\texttt{;\\
}

\end{list}
\paragraph*{Create{\_}RCommand}\hspace*{\fill}

\begin{list}{}{
\settowidth{\tmplength}{\textbf{Declaração}}
\setlength{\itemindent}{0cm}
\setlength{\listparindent}{0cm}
\setlength{\leftmargin}{\evensidemargin}
\addtolength{\leftmargin}{\tmplength}
\settowidth{\labelsep}{X}
\addtolength{\leftmargin}{\labelsep}
\setlength{\labelwidth}{\tmplength}
}
\begin{flushleft}
\item[\textbf{Declaração}\hfill]
\begin{ttfamily}
public Procedure Create{\_}RCommand(Const aStrCommand:tString; Const aName,aParam :AnsiString; Const aKeyCode:Word; Const aAHelpCtx:Word; Const aFlag : Byte; Const aMb{\_}Bits : Longint; Const aFlags{\_}Buttons : Byte; var RCommand{\_}Temp : TRCommand );\end{ttfamily}


\end{flushleft}
\par
\item[\textbf{Descrição}]
O método \textbf{\begin{ttfamily}Create{\_}RCommand\end{ttfamily}} é usado para iniciar os elementos da matriz \textbf{\begin{ttfamily}Commands{\_}Buttons\end{ttfamily}(\ref{mi_rtl_ui_DmxScroller_Buttons.TUiDmxScroller_Buttons-Commands_Buttons})}

\begin{itemize}
\item \textbf{EXEMPLO E USO}

\texttt{\\\nopagebreak[3]
\\\nopagebreak[3]
}\textbf{Function}\texttt{~Create{\_}RCommands{\_}Buttons(}\textbf{Var}\texttt{~aCommands~:~}\textbf{Array}\texttt{~}\textbf{of}\texttt{~TRCommand):SmallWord;\\\nopagebreak[3]
}\textbf{Begin}\texttt{\\\nopagebreak[3]
~~}\textbf{If}\texttt{~High(aCommands)~{$<$}~2\\\nopagebreak[3]
~~}\textbf{Then}\texttt{~}\textbf{Raise}\texttt{~TException.Create(self,'Create{\_}RCommands{\_}Buttons()',ParametroInvalido);\\\nopagebreak[3]
\\\nopagebreak[3]
~~Create{\_}RCommand('CmDbGoBof'~~,CmDbGoBof~~~,'{\&}Inicio'~~~,'',kbNoKey,0,0,Mb{\_}Cm{\_}Bof~~~,bfNormal~,aCommands[1]);\\\nopagebreak[3]
~~Create{\_}RCommand('CmDbPrevRec',CmDbPrevRec~,'{\&}Anterior'~,'',kbNoKey,0,0,Mb{\_}Cm{\_}Prev~~,bfNormal~,aCommands[2]);\\\nopagebreak[3]
\\\nopagebreak[3]
}\textbf{end}\texttt{;\\
}
\end{itemize}

\end{list}
\paragraph*{Create{\_}RCommands{\_}Buttons}\hspace*{\fill}

\begin{list}{}{
\settowidth{\tmplength}{\textbf{Declaração}}
\setlength{\itemindent}{0cm}
\setlength{\listparindent}{0cm}
\setlength{\leftmargin}{\evensidemargin}
\addtolength{\leftmargin}{\tmplength}
\settowidth{\labelsep}{X}
\addtolength{\leftmargin}{\labelsep}
\setlength{\labelwidth}{\tmplength}
}
\begin{flushleft}
\item[\textbf{Declaração}\hfill]
\begin{ttfamily}
protected Function Create{\_}RCommands{\_}Buttons(aCmNovoStr:AnsiString;aCmAlteracaoStr:AnsiString;aCmExclusaoStr:AnsiString;ACmLocalizaStr:AnsiString):Word; overload; Virtual;\end{ttfamily}


\end{flushleft}
\par
\item[\textbf{Descrição}]
O método \textbf{\begin{ttfamily}Create{\_}RCommands{\_}Buttons\end{ttfamily}} retorna em \textbf{aCommands} a matriz aberta de TRCommand e em \textbf{result} retorna o número de elementos adicionados em \textbf{aCommands}.

\begin{itemize}
\item \textbf{EXEMPLO E USO}

\texttt{\\\nopagebreak[3]
\\\nopagebreak[3]
~}\textbf{Var}\texttt{\\\nopagebreak[3]
~~~Commands{\_}Buttons{\_}High~:~Byte;~\textit{//Numero~de~comandos~de~Commands}\\\nopagebreak[3]
~~~Commands{\_}Buttons~~~~~~:~}\textbf{Array}\texttt{[0..Max{\_}List{\_}Buttons]~}\textbf{of}\texttt{~TRCommand;\\\nopagebreak[3]
}\textbf{Begin}\texttt{\\\nopagebreak[3]
~~Commands{\_}Buttons{\_}High~:=~Create{\_}RCommands{\_}Buttons(Commands{\_}Buttons);\\\nopagebreak[3]
}\textbf{end}\texttt{;\\
}
\end{itemize}

\end{list}
\paragraph*{Add{\_}RCommands{\_}Buttons}\hspace*{\fill}

\begin{list}{}{
\settowidth{\tmplength}{\textbf{Declaração}}
\setlength{\itemindent}{0cm}
\setlength{\listparindent}{0cm}
\setlength{\leftmargin}{\evensidemargin}
\addtolength{\leftmargin}{\tmplength}
\settowidth{\labelsep}{X}
\addtolength{\leftmargin}{\labelsep}
\setlength{\labelwidth}{\tmplength}
}
\begin{flushleft}
\item[\textbf{Declaração}\hfill]
\begin{ttfamily}
protected function Add{\_}RCommands{\_}Buttons(aStrCommand: tString; aName: AnsiString; aParam: tString; aKeyCode: Word; aAHelpCtx: Word; aState: Byte; aFlags{\_}Buttons: Byte): Longint;\end{ttfamily}


\end{flushleft}
\par
\item[\textbf{Descrição}]
O método \textbf{\begin{ttfamily}Add{\_}RCommands{\_}Buttons\end{ttfamily}} adiciona um botão na posição \begin{ttfamily}Commands{\_}Buttons\end{ttfamily}(\ref{mi_rtl_ui_DmxScroller_Buttons.TUiDmxScroller_Buttons-Commands_Buttons})[\begin{ttfamily}Commands{\_}Buttons{\_}High\end{ttfamily}(\ref{mi_rtl_ui_DmxScroller_Buttons.TUiDmxScroller_Buttons-Commands_Buttons_High})+1].

\end{list}
\paragraph*{Length{\_}Button{\_}Name{\_}Actives}\hspace*{\fill}

\begin{list}{}{
\settowidth{\tmplength}{\textbf{Declaração}}
\setlength{\itemindent}{0cm}
\setlength{\listparindent}{0cm}
\setlength{\leftmargin}{\evensidemargin}
\addtolength{\leftmargin}{\tmplength}
\settowidth{\labelsep}{X}
\addtolength{\leftmargin}{\labelsep}
\setlength{\labelwidth}{\tmplength}
}
\begin{flushleft}
\item[\textbf{Declaração}\hfill]
\begin{ttfamily}
protected Function Length{\_}Button{\_}Name{\_}Actives:Smallint;\end{ttfamily}


\end{flushleft}
\par
\item[\textbf{Descrição}]
O método \textbf{\begin{ttfamily}Length{\_}Button{\_}Name{\_}Actives\end{ttfamily}} retorna a soma do número de caracteres do campo \textbf{TRCommand.name} dos botões ativos

\end{list}
\paragraph*{Get{\_}Commands{\_}Mb{\_}i}\hspace*{\fill}

\begin{list}{}{
\settowidth{\tmplength}{\textbf{Declaração}}
\setlength{\itemindent}{0cm}
\setlength{\listparindent}{0cm}
\setlength{\leftmargin}{\evensidemargin}
\addtolength{\leftmargin}{\tmplength}
\settowidth{\labelsep}{X}
\addtolength{\leftmargin}{\labelsep}
\setlength{\labelwidth}{\tmplength}
}
\begin{flushleft}
\item[\textbf{Declaração}\hfill]
\begin{ttfamily}
public Function Get{\_}Commands{\_}Mb{\_}i(Const aMb{\_}Bits:Longint):Longint;\end{ttfamily}


\end{flushleft}
\par
\item[\textbf{Descrição}]
O método \textbf{\begin{ttfamily}Get{\_}Commands{\_}Mb{\_}i\end{ttfamily}} retorna a posição na matriz \begin{ttfamily}Commands{\_}Buttons\end{ttfamily}(\ref{mi_rtl_ui_DmxScroller_Buttons.TUiDmxScroller_Buttons-Commands_Buttons}) do mapa de bit passado por aMb{\_}Bits.

\begin{itemize}
\item \textbf{Nota} \begin{itemize}
\item A posição deve ser a mesma do Mapa: {\_}Commands{\_}Buttons{\_}Mb
\end{itemize}
\end{itemize}

\end{list}
\paragraph*{Get{\_}Commands{\_}Mb{\_}StrCommand}\hspace*{\fill}

\begin{list}{}{
\settowidth{\tmplength}{\textbf{Declaração}}
\setlength{\itemindent}{0cm}
\setlength{\listparindent}{0cm}
\setlength{\leftmargin}{\evensidemargin}
\addtolength{\leftmargin}{\tmplength}
\settowidth{\labelsep}{X}
\addtolength{\leftmargin}{\labelsep}
\setlength{\labelwidth}{\tmplength}
}
\begin{flushleft}
\item[\textbf{Declaração}\hfill]
\begin{ttfamily}
public Function Get{\_}Commands{\_}Mb{\_}StrCommand(Const aMb{\_}Bits:Longint):AnsiString;\end{ttfamily}


\end{flushleft}
\par
\item[\textbf{Descrição}]
Retorna o nome do comando passado per aMb{\_}Bits.

\end{list}
\chapter{Unit mi{\_}rtl{\_}ui{\_}dmxscroller{\_}form}
\section{Descrição}
A unit \textbf{\begin{ttfamily}mi{\_}rtl{\_}ui{\_}dmxscroller{\_}form\end{ttfamily}} implementa a classe \begin{ttfamily}TDmxScroller{\_}Form\end{ttfamily}(\ref{mi_rtl_ui_dmxscroller_form.TDmxScroller_Form}).

\begin{itemize}
\item Primeiro autor: Paulo Sérgio da Silva Pacheco paulosspacheco@yahoo.com.br)
\item \textbf{VERSÃO} \begin{itemize}
\item Alpha {-} 0.5.0.687
\end{itemize}
\item \textbf{CÓDIGO FONTE}: \begin{itemize}
\item 
\end{itemize}
\item \textbf{HISTÓRICO}: \begin{itemize}
\item 
\end{itemize}
\item \textbf{PENDÊNCIAS} \begin{itemize}
\item T12 Documentar a unit.
\item T12 Criar propriedade UiDmxScroller{\_}Buttons:\begin{ttfamily}TUiDmxScroller{\_}Buttons\end{ttfamily}(\ref{mi_rtl_ui_DmxScroller_Buttons.TUiDmxScroller_Buttons})
\end{itemize}
\item \textbf{CONCLUÍDO} \begin{itemize}
\item Criar atributo private FirstDataRow : integer; ✅
\item Criar atributo private PrevRec : integer; ✅
\item Criar atributo protected DMXFields : TFPList; ✅
\item Criar atributo protected FldRadioButtonsAdicionados:TStringList;✅
\item Criar atributo Public Function SetHelpCtx{\_}Hint ✅
\item Criar atributo Public Procedure SetHelpCtx{\_}Hint ✅
\item Criar constructor Create(aOwner:TComponent);Override; ✅
\item Criar método public procedure AfterConstruction; override; ✅
\item Criar public destructor destroy;override; ✅
\item Criar método protected procedure ShowControlState;override; ✅
\item Criar método protected procedure CreateStruct ✅
\item Criar método Protected procedure DestroyStruct; Override; ✅
\item Criar método procedure Scroll{\_}it{\_}inview{\_}LCL ✅
\item Criar método public procedure Scroll{\_}it{\_}inview ✅
\item Criar método protected procedure CreateFormLCL ✅
\item Criar método public function GetTemplate(aNext: \begin{ttfamily}PSItem\end{ttfamily}(\ref{mi_rtl_ui_Dmxscroller-PSItem})) ✅
\item Criar método protected procedure UpdateBuffers{\_}Controls;virtual; ✅
\item Criar método public procedure UpdateBuffers;override; ✅
\item Criar método public procedure Refresh;override; ✅
\item Criar método protected procedure SetActiveTarget(aActive : Boolean);override; ✅
\item Criar método protected procedure SetActive(aActive : Boolean);override; ✅
\end{itemize}
\end{itemize}
\section{Uses}
\begin{itemize}
\item \begin{ttfamily}Classes\end{ttfamily}\item \begin{ttfamily}SysUtils\end{ttfamily}\item \begin{ttfamily}typInfo\end{ttfamily}\item \begin{ttfamily}mi.rtl.Consts\end{ttfamily}(\ref{mi.rtl.Consts})\item \begin{ttfamily}mi{\_}rtl{\_}ui{\_}Dmxscroller\end{ttfamily}(\ref{mi_rtl_ui_Dmxscroller})\end{itemize}
\section{Visão Geral}
\begin{description}
\item[\texttt{\begin{ttfamily}TDmxScroller{\_}Form{\_}Atributos\end{ttfamily} Classe}]
\item[\texttt{\begin{ttfamily}TDmxScroller{\_}Form\end{ttfamily} Classe}]
\end{description}
\begin{description}
\item[\texttt{Register}]
\end{description}
\section{Classes, Interfaces, Objetos e Registros}
\subsection*{TDmxScroller{\_}Form{\_}Atributos Classe}
\subsubsection*{\large{\textbf{Hierarquia}}\normalsize\hspace{1ex}\hfill}
TDmxScroller{\_}Form{\_}Atributos {$>$} \begin{ttfamily}TUiDmxScroller\end{ttfamily}(\ref{mi_rtl_ui_Dmxscroller.TUiDmxScroller}) {$>$} \begin{ttfamily}TUiMethods\end{ttfamily}(\ref{mi_rtl_ui_methods.TUiMethods}) {$>$} 
TUiConsts
\subsubsection*{\large{\textbf{Descrição}}\normalsize\hspace{1ex}\hfill}
no description available, TUiDmxScroller description followsA classe \textbf{\begin{ttfamily}TUiDmxScroller\end{ttfamily}} tem como objetivo criar um formulário baseado em uma lista do tipo ShortString.

\begin{itemize}
\item \textbf{NOTAS} \begin{itemize}
\item O método \begin{ttfamily}createStruct\end{ttfamily}(\ref{mi_rtl_ui_Dmxscroller.TUiDmxScroller-CreateStruct}) criar uma lista de campo tipo \begin{ttfamily}TDmxFieldRec\end{ttfamily}(\ref{mi_rtl_ui_Dmxscroller.TDmxFieldRec}) com todas as informações necessárias para criar uma tabela ou um formulário.
\item O formulário é criado com apena uma linha.
\end{itemize}
\item \textbf{EXEMPLO}: \begin{itemize}
\item Template := '~Nome~{\textbackslash}SSSSSSSSSSSSSSSSSSSS ~Idade:~{\textbackslash}BB' \begin{itemize}
\item A classe cria a lista de campos: \begin{itemize}
\item Label1: Nome
\item Field1: campo ShortString com 20 posições maiúsculas
\item Label2: Idade
\item Field2: Campo byte com duas posições
\end{itemize}
\end{itemize}
\end{itemize}
\end{itemize}\subsubsection*{\large{\textbf{Campos}}\normalsize\hspace{1ex}\hfill}
\paragraph*{DMXFields}\hspace*{\fill}

\begin{list}{}{
\settowidth{\tmplength}{\textbf{Declaração}}
\setlength{\itemindent}{0cm}
\setlength{\listparindent}{0cm}
\setlength{\leftmargin}{\evensidemargin}
\addtolength{\leftmargin}{\tmplength}
\settowidth{\labelsep}{X}
\addtolength{\leftmargin}{\labelsep}
\setlength{\labelwidth}{\tmplength}
}
\begin{flushleft}
\item[\textbf{Declaração}\hfill]
\begin{ttfamily}
protected DMXFields: TFPList;\end{ttfamily}


\end{flushleft}
\par
\item[\textbf{Descrição}]
O atributo \textbf{\begin{ttfamily}DMXFields\end{ttfamily}} salva todos os rótulos e campos da lista de Templates

\begin{itemize}
\item \textbf{MOTIVO} \begin{itemize}
\item A classe mãe \begin{ttfamily}TUiDmxScroller\end{ttfamily}(\ref{mi_rtl_ui_Dmxscroller.TUiDmxScroller}) mãe da classe \begin{ttfamily}TDmxScroller{\_}Form\end{ttfamily}(\ref{mi_rtl_ui_dmxscroller_form.TDmxScroller_Form}) cria Template de apenas uma linha, a lista \textbf{\begin{ttfamily}DMXFields\end{ttfamily}} salva todas as linhas para geração de um Template do tipo formulário.
\end{itemize}
\end{itemize}

\end{list}
\paragraph*{FldRadioButtonsAdicionados}\hspace*{\fill}

\begin{list}{}{
\settowidth{\tmplength}{\textbf{Declaração}}
\setlength{\itemindent}{0cm}
\setlength{\listparindent}{0cm}
\setlength{\leftmargin}{\evensidemargin}
\addtolength{\leftmargin}{\tmplength}
\settowidth{\labelsep}{X}
\addtolength{\leftmargin}{\labelsep}
\setlength{\labelwidth}{\tmplength}
}
\begin{flushleft}
\item[\textbf{Declaração}\hfill]
\begin{ttfamily}
protected FldRadioButtonsAdicionados:TStringList;\end{ttfamily}


\end{flushleft}
\par
\item[\textbf{Descrição}]
Usado para evitar que RadiosButton sejam adicionados mais de uma vês em radiosgroups diferentes. \begin{itemize}
\item Mais informações veja campos FldRadioGrous.
\end{itemize}

\end{list}
\subsubsection*{\large{\textbf{Métodos}}\normalsize\hspace{1ex}\hfill}
\paragraph*{SetHelpCtx{\_}Hint}\hspace*{\fill}

\begin{list}{}{
\settowidth{\tmplength}{\textbf{Declaração}}
\setlength{\itemindent}{0cm}
\setlength{\listparindent}{0cm}
\setlength{\leftmargin}{\evensidemargin}
\addtolength{\leftmargin}{\tmplength}
\settowidth{\labelsep}{X}
\addtolength{\leftmargin}{\labelsep}
\setlength{\labelwidth}{\tmplength}
}
\begin{flushleft}
\item[\textbf{Declaração}\hfill]
\begin{ttfamily}
public Function SetHelpCtx{\_}Hint(aFldNum:Integer;a{\_}HelpCtx{\_}Hint:AnsiString):pDmxFieldRec; override;\end{ttfamily}


\end{flushleft}
\par
\item[\textbf{Descrição}]
O método \textbf{\begin{ttfamily}SetHelpCtx{\_}Hint\end{ttfamily}} inicia a documentação resumida do campo. aFldNum

\end{list}
\paragraph*{SetHelpCtx{\_}Hint}\hspace*{\fill}

\begin{list}{}{
\settowidth{\tmplength}{\textbf{Declaração}}
\setlength{\itemindent}{0cm}
\setlength{\listparindent}{0cm}
\setlength{\leftmargin}{\evensidemargin}
\addtolength{\leftmargin}{\tmplength}
\settowidth{\labelsep}{X}
\addtolength{\leftmargin}{\labelsep}
\setlength{\labelwidth}{\tmplength}
}
\begin{flushleft}
\item[\textbf{Declaração}\hfill]
\begin{ttfamily}
public Procedure SetHelpCtx{\_}Hint(apDmxFieldRec:pDmxFieldRec;a{\_}HelpCtx{\_}Hint:AnsiString); override; overload;\end{ttfamily}


\end{flushleft}
\par
\item[\textbf{Descrição}]
O método \textbf{\begin{ttfamily}SetHelpCtx{\_}Hint\end{ttfamily}} inicia a documentação resumida do campo passado em :apDmxFieldRec

\end{list}
\subsection*{TDmxScroller{\_}Form Classe}
\subsubsection*{\large{\textbf{Hierarquia}}\normalsize\hspace{1ex}\hfill}
TDmxScroller{\_}Form {$>$} \begin{ttfamily}TDmxScroller{\_}Form{\_}Atributos\end{ttfamily}(\ref{mi_rtl_ui_dmxscroller_form.TDmxScroller_Form_Atributos}) {$>$} \begin{ttfamily}TUiDmxScroller\end{ttfamily}(\ref{mi_rtl_ui_Dmxscroller.TUiDmxScroller}) {$>$} \begin{ttfamily}TUiMethods\end{ttfamily}(\ref{mi_rtl_ui_methods.TUiMethods}) {$>$} 
TUiConsts
\subsubsection*{\large{\textbf{Descrição}}\normalsize\hspace{1ex}\hfill}
A classe \textbf{\begin{ttfamily}TDmxScroller{\_}Form\end{ttfamily}} implementa a construção de formulários usando uma lista de Templates do tipo TDmxScroller\subsubsection*{\large{\textbf{Propriedades}}\normalsize\hspace{1ex}\hfill}
\paragraph*{name}\hspace*{\fill}

\begin{list}{}{
\settowidth{\tmplength}{\textbf{Declaração}}
\setlength{\itemindent}{0cm}
\setlength{\listparindent}{0cm}
\setlength{\leftmargin}{\evensidemargin}
\addtolength{\leftmargin}{\tmplength}
\settowidth{\labelsep}{X}
\addtolength{\leftmargin}{\labelsep}
\setlength{\labelwidth}{\tmplength}
}
\begin{flushleft}
\item[\textbf{Declaração}\hfill]
\begin{ttfamily}
published property name;\end{ttfamily}


\end{flushleft}
\end{list}
\paragraph*{Alias}\hspace*{\fill}

\begin{list}{}{
\settowidth{\tmplength}{\textbf{Declaração}}
\setlength{\itemindent}{0cm}
\setlength{\listparindent}{0cm}
\setlength{\leftmargin}{\evensidemargin}
\addtolength{\leftmargin}{\tmplength}
\settowidth{\labelsep}{X}
\addtolength{\leftmargin}{\labelsep}
\setlength{\labelwidth}{\tmplength}
}
\begin{flushleft}
\item[\textbf{Declaração}\hfill]
\begin{ttfamily}
published property Alias;\end{ttfamily}


\end{flushleft}
\end{list}
\paragraph*{Strings}\hspace*{\fill}

\begin{list}{}{
\settowidth{\tmplength}{\textbf{Declaração}}
\setlength{\itemindent}{0cm}
\setlength{\listparindent}{0cm}
\setlength{\leftmargin}{\evensidemargin}
\addtolength{\leftmargin}{\tmplength}
\settowidth{\labelsep}{X}
\addtolength{\leftmargin}{\labelsep}
\setlength{\labelwidth}{\tmplength}
}
\begin{flushleft}
\item[\textbf{Declaração}\hfill]
\begin{ttfamily}
published property Strings;\end{ttfamily}


\end{flushleft}
\end{list}
\paragraph*{OnAddTemplate}\hspace*{\fill}

\begin{list}{}{
\settowidth{\tmplength}{\textbf{Declaração}}
\setlength{\itemindent}{0cm}
\setlength{\listparindent}{0cm}
\setlength{\leftmargin}{\evensidemargin}
\addtolength{\leftmargin}{\tmplength}
\settowidth{\labelsep}{X}
\addtolength{\leftmargin}{\labelsep}
\setlength{\labelwidth}{\tmplength}
}
\begin{flushleft}
\item[\textbf{Declaração}\hfill]
\begin{ttfamily}
published property OnAddTemplate;\end{ttfamily}


\end{flushleft}
\end{list}
\paragraph*{OnNewRecord}\hspace*{\fill}

\begin{list}{}{
\settowidth{\tmplength}{\textbf{Declaração}}
\setlength{\itemindent}{0cm}
\setlength{\listparindent}{0cm}
\setlength{\leftmargin}{\evensidemargin}
\addtolength{\leftmargin}{\tmplength}
\settowidth{\labelsep}{X}
\addtolength{\leftmargin}{\labelsep}
\setlength{\labelwidth}{\tmplength}
}
\begin{flushleft}
\item[\textbf{Declaração}\hfill]
\begin{ttfamily}
published property OnNewRecord;\end{ttfamily}


\end{flushleft}
\end{list}
\paragraph*{onCloseQuery}\hspace*{\fill}

\begin{list}{}{
\settowidth{\tmplength}{\textbf{Declaração}}
\setlength{\itemindent}{0cm}
\setlength{\listparindent}{0cm}
\setlength{\leftmargin}{\evensidemargin}
\addtolength{\leftmargin}{\tmplength}
\settowidth{\labelsep}{X}
\addtolength{\leftmargin}{\labelsep}
\setlength{\labelwidth}{\tmplength}
}
\begin{flushleft}
\item[\textbf{Declaração}\hfill]
\begin{ttfamily}
published property onCloseQuery;\end{ttfamily}


\end{flushleft}
\end{list}
\paragraph*{onEnter}\hspace*{\fill}

\begin{list}{}{
\settowidth{\tmplength}{\textbf{Declaração}}
\setlength{\itemindent}{0cm}
\setlength{\listparindent}{0cm}
\setlength{\leftmargin}{\evensidemargin}
\addtolength{\leftmargin}{\tmplength}
\settowidth{\labelsep}{X}
\addtolength{\leftmargin}{\labelsep}
\setlength{\labelwidth}{\tmplength}
}
\begin{flushleft}
\item[\textbf{Declaração}\hfill]
\begin{ttfamily}
published property onEnter;\end{ttfamily}


\end{flushleft}
\end{list}
\paragraph*{onExit}\hspace*{\fill}

\begin{list}{}{
\settowidth{\tmplength}{\textbf{Declaração}}
\setlength{\itemindent}{0cm}
\setlength{\listparindent}{0cm}
\setlength{\leftmargin}{\evensidemargin}
\addtolength{\leftmargin}{\tmplength}
\settowidth{\labelsep}{X}
\addtolength{\leftmargin}{\labelsep}
\setlength{\labelwidth}{\tmplength}
}
\begin{flushleft}
\item[\textbf{Declaração}\hfill]
\begin{ttfamily}
published property onExit;\end{ttfamily}


\end{flushleft}
\end{list}
\paragraph*{onEnterField}\hspace*{\fill}

\begin{list}{}{
\settowidth{\tmplength}{\textbf{Declaração}}
\setlength{\itemindent}{0cm}
\setlength{\listparindent}{0cm}
\setlength{\leftmargin}{\evensidemargin}
\addtolength{\leftmargin}{\tmplength}
\settowidth{\labelsep}{X}
\addtolength{\leftmargin}{\labelsep}
\setlength{\labelwidth}{\tmplength}
}
\begin{flushleft}
\item[\textbf{Declaração}\hfill]
\begin{ttfamily}
published property onEnterField;\end{ttfamily}


\end{flushleft}
\end{list}
\paragraph*{onExitField}\hspace*{\fill}

\begin{list}{}{
\settowidth{\tmplength}{\textbf{Declaração}}
\setlength{\itemindent}{0cm}
\setlength{\listparindent}{0cm}
\setlength{\leftmargin}{\evensidemargin}
\addtolength{\leftmargin}{\tmplength}
\settowidth{\labelsep}{X}
\addtolength{\leftmargin}{\labelsep}
\setlength{\labelwidth}{\tmplength}
}
\begin{flushleft}
\item[\textbf{Declaração}\hfill]
\begin{ttfamily}
published property onExitField;\end{ttfamily}


\end{flushleft}
\end{list}
\paragraph*{onGetTemplate}\hspace*{\fill}

\begin{list}{}{
\settowidth{\tmplength}{\textbf{Declaração}}
\setlength{\itemindent}{0cm}
\setlength{\listparindent}{0cm}
\setlength{\leftmargin}{\evensidemargin}
\addtolength{\leftmargin}{\tmplength}
\settowidth{\labelsep}{X}
\addtolength{\leftmargin}{\labelsep}
\setlength{\labelwidth}{\tmplength}
}
\begin{flushleft}
\item[\textbf{Declaração}\hfill]
\begin{ttfamily}
published property onGetTemplate;\end{ttfamily}


\end{flushleft}
\end{list}
\paragraph*{Active}\hspace*{\fill}

\begin{list}{}{
\settowidth{\tmplength}{\textbf{Declaração}}
\setlength{\itemindent}{0cm}
\setlength{\listparindent}{0cm}
\setlength{\leftmargin}{\evensidemargin}
\addtolength{\leftmargin}{\tmplength}
\settowidth{\labelsep}{X}
\addtolength{\leftmargin}{\labelsep}
\setlength{\labelwidth}{\tmplength}
}
\begin{flushleft}
\item[\textbf{Declaração}\hfill]
\begin{ttfamily}
published property Active;\end{ttfamily}


\end{flushleft}
\end{list}
\paragraph*{AlignmentLabels}\hspace*{\fill}

\begin{list}{}{
\settowidth{\tmplength}{\textbf{Declaração}}
\setlength{\itemindent}{0cm}
\setlength{\listparindent}{0cm}
\setlength{\leftmargin}{\evensidemargin}
\addtolength{\leftmargin}{\tmplength}
\settowidth{\labelsep}{X}
\addtolength{\leftmargin}{\labelsep}
\setlength{\labelwidth}{\tmplength}
}
\begin{flushleft}
\item[\textbf{Declaração}\hfill]
\begin{ttfamily}
published property AlignmentLabels;\end{ttfamily}


\end{flushleft}
\end{list}
\subsubsection*{\large{\textbf{Métodos}}\normalsize\hspace{1ex}\hfill}
\paragraph*{Create}\hspace*{\fill}

\begin{list}{}{
\settowidth{\tmplength}{\textbf{Declaração}}
\setlength{\itemindent}{0cm}
\setlength{\listparindent}{0cm}
\setlength{\leftmargin}{\evensidemargin}
\addtolength{\leftmargin}{\tmplength}
\settowidth{\labelsep}{X}
\addtolength{\leftmargin}{\labelsep}
\setlength{\labelwidth}{\tmplength}
}
\begin{flushleft}
\item[\textbf{Declaração}\hfill]
\begin{ttfamily}
public constructor Create(aOwner:TComponent); Override;\end{ttfamily}


\end{flushleft}
\par
\item[\textbf{Descrição}]
Constrói o componente

\end{list}
\paragraph*{AfterConstruction}\hspace*{\fill}

\begin{list}{}{
\settowidth{\tmplength}{\textbf{Declaração}}
\setlength{\itemindent}{0cm}
\setlength{\listparindent}{0cm}
\setlength{\leftmargin}{\evensidemargin}
\addtolength{\leftmargin}{\tmplength}
\settowidth{\labelsep}{X}
\addtolength{\leftmargin}{\labelsep}
\setlength{\labelwidth}{\tmplength}
}
\begin{flushleft}
\item[\textbf{Declaração}\hfill]
\begin{ttfamily}
public procedure AfterConstruction; override;\end{ttfamily}


\end{flushleft}
\end{list}
\paragraph*{destroy}\hspace*{\fill}

\begin{list}{}{
\settowidth{\tmplength}{\textbf{Declaração}}
\setlength{\itemindent}{0cm}
\setlength{\listparindent}{0cm}
\setlength{\leftmargin}{\evensidemargin}
\addtolength{\leftmargin}{\tmplength}
\settowidth{\labelsep}{X}
\addtolength{\leftmargin}{\labelsep}
\setlength{\labelwidth}{\tmplength}
}
\begin{flushleft}
\item[\textbf{Declaração}\hfill]
\begin{ttfamily}
public destructor destroy; override;\end{ttfamily}


\end{flushleft}
\par
\item[\textbf{Descrição}]
Destrói o componente

\end{list}
\paragraph*{CreateStruct}\hspace*{\fill}

\begin{list}{}{
\settowidth{\tmplength}{\textbf{Declaração}}
\setlength{\itemindent}{0cm}
\setlength{\listparindent}{0cm}
\setlength{\leftmargin}{\evensidemargin}
\addtolength{\leftmargin}{\tmplength}
\settowidth{\labelsep}{X}
\addtolength{\leftmargin}{\labelsep}
\setlength{\labelwidth}{\tmplength}
}
\begin{flushleft}
\item[\textbf{Declaração}\hfill]
\begin{ttfamily}
protected procedure CreateStruct(var ATemplate : PSItem); override; overload;\end{ttfamily}


\end{flushleft}
\par
\item[\textbf{Descrição}]
O método \textbf{\begin{ttfamily}CreateStruct\end{ttfamily}} interpreta uma lista de \begin{ttfamily}strings\end{ttfamily}(\ref{mi_rtl_ui_dmxscroller_form.TDmxScroller_Form-Strings}) do tipo \textbf{\begin{ttfamily}PSItem\end{ttfamily}(\ref{mi_rtl_ui_Dmxscroller-PSItem})} e adiciona os layout de cada campo em \begin{ttfamily}pDmxFieldRec\end{ttfamily}(\ref{mi_rtl_ui_dmxscroller_form-pDmxFieldRec}), em seguida adiciona \textbf{\begin{ttfamily}pDmxFieldRec\end{ttfamily}(\ref{mi_rtl_ui_dmxscroller_form-pDmxFieldRec})} em \textbf{\begin{ttfamily}DMXFields\end{ttfamily}(\ref{mi_rtl_ui_dmxscroller_form.TDmxScroller_Form_Atributos-DMXFields})} com todos os campos de formação de formulário visual.

\end{list}
\paragraph*{DestroyStruct}\hspace*{\fill}

\begin{list}{}{
\settowidth{\tmplength}{\textbf{Declaração}}
\setlength{\itemindent}{0cm}
\setlength{\listparindent}{0cm}
\setlength{\leftmargin}{\evensidemargin}
\addtolength{\leftmargin}{\tmplength}
\settowidth{\labelsep}{X}
\addtolength{\leftmargin}{\labelsep}
\setlength{\labelwidth}{\tmplength}
}
\begin{flushleft}
\item[\textbf{Declaração}\hfill]
\begin{ttfamily}
protected procedure DestroyStruct; Override;\end{ttfamily}


\end{flushleft}
\par
\item[\textbf{Descrição}]
O método \textbf{\begin{ttfamily}DestroyStruct\end{ttfamily}} destrói os dados criados em \textbf{\begin{ttfamily}CreateStruct\end{ttfamily}(\ref{mi_rtl_ui_dmxscroller_form.TDmxScroller_Form-CreateStruct})()}.

\end{list}
\paragraph*{GetTemplate}\hspace*{\fill}

\begin{list}{}{
\settowidth{\tmplength}{\textbf{Declaração}}
\setlength{\itemindent}{0cm}
\setlength{\listparindent}{0cm}
\setlength{\leftmargin}{\evensidemargin}
\addtolength{\leftmargin}{\tmplength}
\settowidth{\labelsep}{X}
\addtolength{\leftmargin}{\labelsep}
\setlength{\labelwidth}{\tmplength}
}
\begin{flushleft}
\item[\textbf{Declaração}\hfill]
\begin{ttfamily}
public function GetTemplate(aNext: PSItem) : PSItem; overload; override;\end{ttfamily}


\end{flushleft}
\par
\item[\textbf{Descrição}]
O método \textbf{\begin{ttfamily}GetTemplate\end{ttfamily}} retorna uma lista de \textbf{\begin{ttfamily}PSItem\end{ttfamily}(\ref{mi_rtl_ui_Dmxscroller-PSItem})} (Lista de \begin{ttfamily}strings\end{ttfamily}(\ref{mi_rtl_ui_dmxscroller_form.TDmxScroller_Form-Strings})) com o modelo usado para criar a tela.

\begin{itemize}
\item \textbf{NOTA} \begin{itemize}
\item O Evento \begin{ttfamily}onGetTemplate\end{ttfamily}(\ref{mi_rtl_ui_dmxscroller_form.TDmxScroller_Form-onGetTemplate}) só é iniciado em tempo de execução, por isso o formulário não pode ser criado em tempo de desenho do aplicativo.
\item Caso o evento \begin{ttfamily}onGetTemplate\end{ttfamily}(\ref{mi_rtl_ui_dmxscroller_form.TDmxScroller_Form-onGetTemplate}) seja nil, então não posso ativar a tela.
\item Esse método pode ser anulado, caso se queira ignorar o evento \begin{ttfamily}onGetTemplate\end{ttfamily}(\ref{mi_rtl_ui_dmxscroller_form.TDmxScroller_Form-onGetTemplate}) e definir o Template em uma método pai herdado desta classe.
\item O modelo cria um formulário usando os tipos de dados primitivos.
\end{itemize}
\item \textbf{EXEMPLO}

\texttt{\\\nopagebreak[3]
\\\nopagebreak[3]
\textit{//~Implementação~de~um~modelo~no~Alvo~LCL}\\\nopagebreak[3]
}\textbf{function}\texttt{~TDMAlunos.DmxScroller{\_}Form{\_}AlunoGetTemplate(aNext:~PSItem):~PSItem;\\\nopagebreak[3]
}\textbf{begin}\texttt{\\\nopagebreak[3]
~~}\textbf{with}\texttt{~DmxScroller{\_}Form{\_}Aluno~}\textbf{do}\texttt{\\\nopagebreak[3]
~~}\textbf{begin}\texttt{\\\nopagebreak[3]
~~~~\textit{//~AlignmentLabels:=~taCenter;}\\\nopagebreak[3]
~~~~AlignmentLabels~:=~taLeftJustify;\\\nopagebreak[3]
~~~~\textit{//~AlignmentLabels~:=~taRightJustify~;}\\\nopagebreak[3]
~~~~Result~:=\\\nopagebreak[3]
~~~~~~NewSItem('~~~~~~Matrícula~~{\textbackslash}LLLLL'+CharFieldName+'matricula'+CharAccReadOnly+CharPfInKeyPrimary+CharPfInAutoIncrement,\\\nopagebreak[3]
~~~~~~NewSItem('~Nome~do~aluno:~~{\textbackslash}ssssssssssssssssssss`sssssss'+CharFieldName+'Nome'+CharPfInKey,\\\nopagebreak[3]
~~~~~~NewSItem('',\\\nopagebreak[3]
~~~~~~NewSItem('~~~~~~Endereço:~~{\textbackslash}ssssssssssssssssssss`sssssssssss'+CharFieldName+'Endereco',\\\nopagebreak[3]
~~~~~~NewSItem('~P.~Referência:~~{\textbackslash}ssssssssssssssssssss`sssss'+CharFieldName+'PontoDeReferencia',\\\nopagebreak[3]
~~~~~~NewSItem('~~~~~~~~~~~Cep:~~{\textbackslash}{\#}{\#}.{\#}{\#}{\#}-{\#}{\#}{\#}'+CharFieldName+'cep',\\\nopagebreak[3]
~~~~~~NewSItem('~~~~~~~~Estado:~~{\textbackslash}SS'+CharFieldName+'Estado'+CharForeignKeyN{\_}Um{\_}false+'Estados,Estado',\\\nopagebreak[3]
~~~~~~NewSItem('~~~~~~~~Cidade:~~{\textbackslash}ssssssssssssssssssss`sssss'+CharFieldName+'cidade'+CharForeignKeyN{\_}Um{\_}false+'Cidades,Estado;Cidade',\\\nopagebreak[3]
~~~~~~NewSItem('',\\\nopagebreak[3]
~~~~~~aNext)))))))));\\\nopagebreak[3]
~~}\textbf{end}\texttt{;\\\nopagebreak[3]
}\textbf{end}\texttt{;\\
}
\end{itemize}

\end{list}
\paragraph*{UpdateBuffers{\_}Controls}\hspace*{\fill}

\begin{list}{}{
\settowidth{\tmplength}{\textbf{Declaração}}
\setlength{\itemindent}{0cm}
\setlength{\listparindent}{0cm}
\setlength{\leftmargin}{\evensidemargin}
\addtolength{\leftmargin}{\tmplength}
\settowidth{\labelsep}{X}
\addtolength{\leftmargin}{\labelsep}
\setlength{\labelwidth}{\tmplength}
}
\begin{flushleft}
\item[\textbf{Declaração}\hfill]
\begin{ttfamily}
protected procedure UpdateBuffers{\_}Controls; virtual;\end{ttfamily}


\end{flushleft}
\end{list}
\paragraph*{UpdateBuffers}\hspace*{\fill}

\begin{list}{}{
\settowidth{\tmplength}{\textbf{Declaração}}
\setlength{\itemindent}{0cm}
\setlength{\listparindent}{0cm}
\setlength{\leftmargin}{\evensidemargin}
\addtolength{\leftmargin}{\tmplength}
\settowidth{\labelsep}{X}
\addtolength{\leftmargin}{\labelsep}
\setlength{\labelwidth}{\tmplength}
}
\begin{flushleft}
\item[\textbf{Declaração}\hfill]
\begin{ttfamily}
public procedure UpdateBuffers; override;\end{ttfamily}


\end{flushleft}
\end{list}
\paragraph*{SetActiveTarget}\hspace*{\fill}

\begin{list}{}{
\settowidth{\tmplength}{\textbf{Declaração}}
\setlength{\itemindent}{0cm}
\setlength{\listparindent}{0cm}
\setlength{\leftmargin}{\evensidemargin}
\addtolength{\leftmargin}{\tmplength}
\settowidth{\labelsep}{X}
\addtolength{\leftmargin}{\labelsep}
\setlength{\labelwidth}{\tmplength}
}
\begin{flushleft}
\item[\textbf{Declaração}\hfill]
\begin{ttfamily}
protected procedure SetActiveTarget(aActive: Boolean); virtual;\end{ttfamily}


\end{flushleft}
\par
\item[\textbf{Descrição}]
A procedure \textbf{\begin{ttfamily}SetActiveTarget\end{ttfamily}} seta a propriedade \begin{ttfamily}active\end{ttfamily}(\ref{mi_rtl_ui_dmxscroller_form.TDmxScroller_Form-Active}) e criar um formulário na plataforma alvo

\end{list}
\paragraph*{SetActive}\hspace*{\fill}

\begin{list}{}{
\settowidth{\tmplength}{\textbf{Declaração}}
\setlength{\itemindent}{0cm}
\setlength{\listparindent}{0cm}
\setlength{\leftmargin}{\evensidemargin}
\addtolength{\leftmargin}{\tmplength}
\settowidth{\labelsep}{X}
\addtolength{\leftmargin}{\labelsep}
\setlength{\labelwidth}{\tmplength}
}
\begin{flushleft}
\item[\textbf{Declaração}\hfill]
\begin{ttfamily}
protected procedure SetActive(aActive : Boolean); override;\end{ttfamily}


\end{flushleft}
\par
\item[\textbf{Descrição}]
A procedure \textbf{\begin{ttfamily}SetActive\end{ttfamily}} seta a propriedade \begin{ttfamily}active\end{ttfamily}(\ref{mi_rtl_ui_dmxscroller_form.TDmxScroller_Form-Active}) e criar um formulário LCL ou HTML dependendo do tipo de aplicação

\end{list}
\section{Funções e Procedimentos}
\subsection*{Register}
\begin{list}{}{
\settowidth{\tmplength}{\textbf{Declaração}}
\setlength{\itemindent}{0cm}
\setlength{\listparindent}{0cm}
\setlength{\leftmargin}{\evensidemargin}
\addtolength{\leftmargin}{\tmplength}
\settowidth{\labelsep}{X}
\addtolength{\leftmargin}{\labelsep}
\setlength{\labelwidth}{\tmplength}
}
\begin{flushleft}
\item[\textbf{Declaração}\hfill]
\begin{ttfamily}
procedure Register;\end{ttfamily}


\end{flushleft}
\end{list}
\section{Tipos}
\subsection*{TDmxFieldRec}
\begin{list}{}{
\settowidth{\tmplength}{\textbf{Declaração}}
\setlength{\itemindent}{0cm}
\setlength{\listparindent}{0cm}
\setlength{\leftmargin}{\evensidemargin}
\addtolength{\leftmargin}{\tmplength}
\settowidth{\labelsep}{X}
\addtolength{\leftmargin}{\labelsep}
\setlength{\labelwidth}{\tmplength}
}
\begin{flushleft}
\item[\textbf{Declaração}\hfill]
\begin{ttfamily}
TDmxFieldRec = mi{\_}rtl{\_}ui{\_}Dmxscroller.TDmxFieldRec;\end{ttfamily}


\end{flushleft}
\end{list}
\subsection*{pDmxFieldRec}
\begin{list}{}{
\settowidth{\tmplength}{\textbf{Declaração}}
\setlength{\itemindent}{0cm}
\setlength{\listparindent}{0cm}
\setlength{\leftmargin}{\evensidemargin}
\addtolength{\leftmargin}{\tmplength}
\settowidth{\labelsep}{X}
\addtolength{\leftmargin}{\labelsep}
\setlength{\labelwidth}{\tmplength}
}
\begin{flushleft}
\item[\textbf{Declaração}\hfill]
\begin{ttfamily}
pDmxFieldRec = mi{\_}rtl{\_}ui{\_}Dmxscroller.pDmxFieldRec;\end{ttfamily}


\end{flushleft}
\end{list}
\subsection*{SmallWord}
\begin{list}{}{
\settowidth{\tmplength}{\textbf{Declaração}}
\setlength{\itemindent}{0cm}
\setlength{\listparindent}{0cm}
\setlength{\leftmargin}{\evensidemargin}
\addtolength{\leftmargin}{\tmplength}
\settowidth{\labelsep}{X}
\addtolength{\leftmargin}{\labelsep}
\setlength{\labelwidth}{\tmplength}
}
\begin{flushleft}
\item[\textbf{Declaração}\hfill]
\begin{ttfamily}
SmallWord    = TUiDmxScroller.SmallWord;\end{ttfamily}


\end{flushleft}
\end{list}
\section{Constantes}
\subsection*{accDelimiter}
\begin{list}{}{
\settowidth{\tmplength}{\textbf{Declaração}}
\setlength{\itemindent}{0cm}
\setlength{\listparindent}{0cm}
\setlength{\leftmargin}{\evensidemargin}
\addtolength{\leftmargin}{\tmplength}
\settowidth{\labelsep}{X}
\addtolength{\leftmargin}{\labelsep}
\setlength{\labelwidth}{\tmplength}
}
\begin{flushleft}
\item[\textbf{Declaração}\hfill]
\begin{ttfamily}
accDelimiter          = TConsts.accDelimiter ;\end{ttfamily}


\end{flushleft}
\par
\item[\textbf{Descrição}]
A constante \textbf{\begin{ttfamily}accDelimiter\end{ttfamily}} informa que o campo é delimitador de campos no Template.

\end{list}
\subsection*{accHidden}
\begin{list}{}{
\settowidth{\tmplength}{\textbf{Declaração}}
\setlength{\itemindent}{0cm}
\setlength{\listparindent}{0cm}
\setlength{\leftmargin}{\evensidemargin}
\addtolength{\leftmargin}{\tmplength}
\settowidth{\labelsep}{X}
\addtolength{\leftmargin}{\labelsep}
\setlength{\labelwidth}{\tmplength}
}
\begin{flushleft}
\item[\textbf{Declaração}\hfill]
\begin{ttfamily}
accHidden             = TConsts.accHidden    ;\end{ttfamily}


\end{flushleft}
\par
\item[\textbf{Descrição}]
A constante \textbf{\begin{ttfamily}accHidden\end{ttfamily}} (Const \begin{ttfamily}accHidden\end{ttfamily} = 2;) é um mapa de bits usado para identificar o bit do campo \begin{ttfamily}TDmxFieldRec.access\end{ttfamily}(\ref{mi_rtl_ui_Dmxscroller.TDmxFieldRec-access}) que informa se o mesmo é invisível.

\begin{itemize}
\item \textbf{EXEMPLO} \begin{itemize}
\item Como usar o mapa de bits \begin{ttfamily}accHidden\end{ttfamily} para saber se o campo está invisível.

\texttt{\\\nopagebreak[3]
\\\nopagebreak[3]
}\textbf{with}\texttt{~pDmxFieldRec{\^{}}~}\textbf{do}\texttt{\\\nopagebreak[3]
~~}\textbf{If}\texttt{~(access~}\textbf{and}\texttt{~accHidden~{$<$}{$>$}~0)\\\nopagebreak[3]
~~}\textbf{then}\texttt{~}\textbf{begin}\texttt{\\\nopagebreak[3]
~~~~~~~~~ShowMessage(Format('O~campo~{\%}s~está~invisível'),[CharFieldName]);~\\\nopagebreak[3]
~~~~~~~}\textbf{end}\texttt{;\\
}
\end{itemize}
\end{itemize}

\end{list}
\subsection*{AccNormal}
\begin{list}{}{
\settowidth{\tmplength}{\textbf{Declaração}}
\setlength{\itemindent}{0cm}
\setlength{\listparindent}{0cm}
\setlength{\leftmargin}{\evensidemargin}
\addtolength{\leftmargin}{\tmplength}
\settowidth{\labelsep}{X}
\addtolength{\leftmargin}{\labelsep}
\setlength{\labelwidth}{\tmplength}
}
\begin{flushleft}
\item[\textbf{Declaração}\hfill]
\begin{ttfamily}
AccNormal             = TConsts.AccNormal;\end{ttfamily}


\end{flushleft}
\par
\item[\textbf{Descrição}]
A constante \textbf{\begin{ttfamily}AccNormal\end{ttfamily}} (Const \begin{ttfamily}AccNormal\end{ttfamily} = 0;) é um mapa de bits usado para identificar o bit do campo \begin{ttfamily}TDmxFieldRec.access\end{ttfamily}(\ref{mi_rtl_ui_Dmxscroller.TDmxFieldRec-access}) que informa se que o campo pode ser editado.

\begin{itemize}
\item \textbf{EXEMPLO} \begin{itemize}
\item Como usar o mapa de bits \begin{ttfamily}accNormal\end{ttfamily} para saber se o campo pode ser editado.

\texttt{\\\nopagebreak[3]
\\\nopagebreak[3]
}\textbf{with}\texttt{~pDmxFieldRec{\^{}}~}\textbf{do}\texttt{\\\nopagebreak[3]
~~}\textbf{If}\texttt{~(access~}\textbf{and}\texttt{~accNormal~{$<$}{$>$}~0)~\\\nopagebreak[3]
~~}\textbf{then}\texttt{~}\textbf{begin}\texttt{\\\nopagebreak[3]
~~~~~~~~~ShowMessage(Format('O~campo~{\%}s~pode~ser~editado'),[FieldName]);\\\nopagebreak[3]
~~~~~~~}\textbf{end}\texttt{;\\
}
\end{itemize}
\end{itemize}

\end{list}
\subsection*{accReadOnly}
\begin{list}{}{
\settowidth{\tmplength}{\textbf{Declaração}}
\setlength{\itemindent}{0cm}
\setlength{\listparindent}{0cm}
\setlength{\leftmargin}{\evensidemargin}
\addtolength{\leftmargin}{\tmplength}
\settowidth{\labelsep}{X}
\addtolength{\leftmargin}{\labelsep}
\setlength{\labelwidth}{\tmplength}
}
\begin{flushleft}
\item[\textbf{Declaração}\hfill]
\begin{ttfamily}
accReadOnly           = TConsts.accReadOnly  ;\end{ttfamily}


\end{flushleft}
\par
\item[\textbf{Descrição}]
A constante \textbf{\begin{ttfamily}accReadOnly\end{ttfamily}} (Const ReadOnly = 1;) é um mapa de bits usado para identificar o bit do campo \begin{ttfamily}TDmxFieldRec.access\end{ttfamily}(\ref{mi_rtl_ui_Dmxscroller.TDmxFieldRec-access}) que informa se o campo é somente para leitura. \begin{itemize}
\item \textbf{EXEMPLO} \begin{itemize}
\item Como usar o mapa de bits ReadOnly para saber se o campo não pode ser editado. \texttt{\\\nopagebreak[3]
}\textbf{with}\texttt{~pDmxFieldRec{\^{}}~}\textbf{do}\texttt{\\\nopagebreak[3]
~~}\textbf{If}\texttt{~(access~}\textbf{and}\texttt{~ReadOnly~{$<$}{$>$}~0)\\\nopagebreak[3]
~~}\textbf{then}\texttt{~}\textbf{begin}\texttt{\\\nopagebreak[3]
~~~~~~~~~ShowMessage(Format('O~campo~{\%}s~não~pode~ser~editado'),[FieldName]);\\\nopagebreak[3]
~~~~~~~}\textbf{end}\texttt{;\\
}
\end{itemize}
\end{itemize}

\end{list}
\subsection*{accSkip}
\begin{list}{}{
\settowidth{\tmplength}{\textbf{Declaração}}
\setlength{\itemindent}{0cm}
\setlength{\listparindent}{0cm}
\setlength{\leftmargin}{\evensidemargin}
\addtolength{\leftmargin}{\tmplength}
\settowidth{\labelsep}{X}
\addtolength{\leftmargin}{\labelsep}
\setlength{\labelwidth}{\tmplength}
}
\begin{flushleft}
\item[\textbf{Declaração}\hfill]
\begin{ttfamily}
accSkip               = TConsts.accSkip      ;\end{ttfamily}


\end{flushleft}
\par
\item[\textbf{Descrição}]
A constante \textbf{\begin{ttfamily}accSkip\end{ttfamily}} (Const \begin{ttfamily}accSkip\end{ttfamily} = 4;) é um mapa de bits usado para identificar o bit do campo \begin{ttfamily}TDmxFieldRec.access\end{ttfamily}(\ref{mi_rtl_ui_Dmxscroller.TDmxFieldRec-access}) que informa se o campo pode receber o focus. \begin{itemize}
\item \textbf{EXEMPLO} \begin{itemize}
\item Como usar o mapa de bits \begin{ttfamily}accSkip\end{ttfamily} para saber se o campo não pode receber o focus. \texttt{\\\nopagebreak[3]
}\textbf{with}\texttt{~pDmxFieldRec{\^{}}~}\textbf{do}\texttt{\\\nopagebreak[3]
~~}\textbf{If}\texttt{~(access~}\textbf{and}\texttt{~accSkip~{$<$}{$>$}~0)\\\nopagebreak[3]
~~}\textbf{then}\texttt{~}\textbf{begin}\texttt{\\\nopagebreak[3]
~~~~~~~~~ShowMessage(Format('O~campo~{\%}s~não~pode~receber~o~focus'),[FieldName]);\\\nopagebreak[3]
~~~~~~~}\textbf{end}\texttt{;\\
}
\end{itemize}
\end{itemize}

\end{list}
\subsection*{CharHint}
\begin{list}{}{
\settowidth{\tmplength}{\textbf{Declaração}}
\setlength{\itemindent}{0cm}
\setlength{\listparindent}{0cm}
\setlength{\leftmargin}{\evensidemargin}
\addtolength{\leftmargin}{\tmplength}
\settowidth{\labelsep}{X}
\addtolength{\leftmargin}{\labelsep}
\setlength{\labelwidth}{\tmplength}
}
\begin{flushleft}
\item[\textbf{Declaração}\hfill]
\begin{ttfamily}
CharHint              = TConsts.CharHint;\end{ttfamily}


\end{flushleft}
\par
\item[\textbf{Descrição}]
A constante \textbf{\begin{ttfamily}CharHint\end{ttfamily}} é usado para documentar o campo e indica que todo o texto até o próximo caractere de controle será o conteúdo do campo HelpCtx{\_}Hint.

\begin{itemize}
\item \textbf{EXEMPLO}

\texttt{\\\nopagebreak[3]
\\\nopagebreak[3]
}\textbf{Resourcestring}\texttt{\\\nopagebreak[3]
\\\nopagebreak[3]
~tmp{\_}Alunos{\_}Idade~=~'{\textbackslash}BB'+ChFN+'idade'+CharUpperlimit+{\#}64+\\\nopagebreak[3]
~~~~~~~~~~~~~~~~~~~~~CharHint+'A~idade~do~aluno.~Valores~válidos~1~a~64'+\\\nopagebreak[3]
~~~~~~~~~~~~~~~~~~~~~CharHintPorque+'Este~campo~é~necessário~para~que~se~agrupe~o~alunos~baseado~em~sua~faixa~etária'+\\\nopagebreak[3]
~~~~~~~~~~~~~~~~~~~~~CharHintOnde+'Ele~será~usado~pelo~coordenador~ao~classificar~a~turma';\\\nopagebreak[3]
\\\nopagebreak[3]
\\\nopagebreak[3]
~tmp{\_}Alunos{\_}Matricula~=~{\textbackslash}IIII'+ChFN+'matricula'+CharHint+'A~matricula~}\textbf{do}\texttt{~aluno~é~um~campo~sequencial~e~calculado~ao~incluir~o~registro';\\\nopagebreak[3]
\\\nopagebreak[3]
~tmp{\_}Alunos~=~'~~~~~~Idade:~{\%}s'+lf+\\\nopagebreak[3]
~~~~~~~~~~~~~~'~~Matricula:~{\%}s\\
}
\end{itemize}

\end{list}
\subsection*{CharHintOnde}
\begin{list}{}{
\settowidth{\tmplength}{\textbf{Declaração}}
\setlength{\itemindent}{0cm}
\setlength{\listparindent}{0cm}
\setlength{\leftmargin}{\evensidemargin}
\addtolength{\leftmargin}{\tmplength}
\settowidth{\labelsep}{X}
\addtolength{\leftmargin}{\labelsep}
\setlength{\labelwidth}{\tmplength}
}
\begin{flushleft}
\item[\textbf{Declaração}\hfill]
\begin{ttfamily}
CharHintOnde          = TConsts.CharHintOnde;\end{ttfamily}


\end{flushleft}
\par
\item[\textbf{Descrição}]
A contante \textbf{\begin{ttfamily}CharHintOnde\end{ttfamily}} informa que todo texto até o próximo delimitador contém informações para o campo HelpCtx{\_}Onde

\end{list}
\subsection*{CharHintPorque}
\begin{list}{}{
\settowidth{\tmplength}{\textbf{Declaração}}
\setlength{\itemindent}{0cm}
\setlength{\listparindent}{0cm}
\setlength{\leftmargin}{\evensidemargin}
\addtolength{\leftmargin}{\tmplength}
\settowidth{\labelsep}{X}
\addtolength{\leftmargin}{\labelsep}
\setlength{\labelwidth}{\tmplength}
}
\begin{flushleft}
\item[\textbf{Declaração}\hfill]
\begin{ttfamily}
CharHintPorque        = TConsts.CharHintPorque;\end{ttfamily}


\end{flushleft}
\par
\item[\textbf{Descrição}]
A contante \textbf{\begin{ttfamily}CharHintPorque\end{ttfamily}} informa que todo texto até o próximo delimitador contém informações para o campo HelpCtx{\_}Porque

\end{list}
\subsection*{fld{\_}LData}
\begin{list}{}{
\settowidth{\tmplength}{\textbf{Declaração}}
\setlength{\itemindent}{0cm}
\setlength{\listparindent}{0cm}
\setlength{\leftmargin}{\evensidemargin}
\addtolength{\leftmargin}{\tmplength}
\settowidth{\labelsep}{X}
\addtolength{\leftmargin}{\labelsep}
\setlength{\labelwidth}{\tmplength}
}
\begin{flushleft}
\item[\textbf{Declaração}\hfill]
\begin{ttfamily}
fld{\_}LData             = TConsts.fld{\_}LData          ;\end{ttfamily}


\end{flushleft}
\par
\item[\textbf{Descrição}]
A constante \textbf{\begin{ttfamily}fld{\_}LData\end{ttfamily}} é do tipo TDateTime e guarda a data compactada 'dd/dd/dd'

\end{list}
\subsection*{fld{\_}LHora}
\begin{list}{}{
\settowidth{\tmplength}{\textbf{Declaração}}
\setlength{\itemindent}{0cm}
\setlength{\listparindent}{0cm}
\setlength{\leftmargin}{\evensidemargin}
\addtolength{\leftmargin}{\tmplength}
\settowidth{\labelsep}{X}
\addtolength{\leftmargin}{\labelsep}
\setlength{\labelwidth}{\tmplength}
}
\begin{flushleft}
\item[\textbf{Declaração}\hfill]
\begin{ttfamily}
fld{\_}LHora             = TConsts.fld{\_}LHora          ;\end{ttfamily}


\end{flushleft}
\par
\item[\textbf{Descrição}]
A constante \textbf{\begin{ttfamily}fld{\_}LHora\end{ttfamily}} é do tipo TDateTime e guarda a hora compactada {\#}{\#}:{\#}{\#}:{\#}{\#}

\end{list}
\subsection*{fldAnsiChar}
\begin{list}{}{
\settowidth{\tmplength}{\textbf{Declaração}}
\setlength{\itemindent}{0cm}
\setlength{\listparindent}{0cm}
\setlength{\leftmargin}{\evensidemargin}
\addtolength{\leftmargin}{\tmplength}
\settowidth{\labelsep}{X}
\addtolength{\leftmargin}{\labelsep}
\setlength{\labelwidth}{\tmplength}
}
\begin{flushleft}
\item[\textbf{Declaração}\hfill]
\begin{ttfamily}
fldAnsiChar           = TConsts.fldAnsiChar          ;\end{ttfamily}


\end{flushleft}
\par
\item[\textbf{Descrição}]
A constante \textbf{\begin{ttfamily}fldAnsiChar\end{ttfamily}} (Const \begin{ttfamily}fldAnsiChar\end{ttfamily} = 'C') usado na máscara do Template, informa ao componente \textbf{\begin{ttfamily}TUiDmxScroller\end{ttfamily}(\ref{mi_rtl_ui_Dmxscroller.TUiDmxScroller})} que a sequência de caracteres 'C' após o caractere \textbf{"{\textbackslash}"} representa no buffer do formulário um tipo AnsiString que só aceita caractere maiúsculo.

\begin{itemize}
\item \textbf{EXEMPLO} \begin{itemize}
\item Representação de um AnsiString de 10 dígitos em um buffer de 11 bytes onde o ultimo byte contém o caractere {\#}0 informando o fim da string;

\texttt{\\\nopagebreak[3]
\\\nopagebreak[3]
}\textbf{Const}\texttt{\\\nopagebreak[3]
~~Nome~:=~'{\textbackslash}CCCCCCCCCC';~\textit{//PAULO~SÉRG}\\
}
\end{itemize}
\end{itemize}

\end{list}
\subsection*{fldAnsiChar{\_}Minuscula}
\begin{list}{}{
\settowidth{\tmplength}{\textbf{Declaração}}
\setlength{\itemindent}{0cm}
\setlength{\listparindent}{0cm}
\setlength{\leftmargin}{\evensidemargin}
\addtolength{\leftmargin}{\tmplength}
\settowidth{\labelsep}{X}
\addtolength{\leftmargin}{\labelsep}
\setlength{\labelwidth}{\tmplength}
}
\begin{flushleft}
\item[\textbf{Declaração}\hfill]
\begin{ttfamily}
fldAnsiChar{\_}Minuscula = TConsts.fldAnsiChar{\_}Minuscula;\end{ttfamily}


\end{flushleft}
\par
\item[\textbf{Descrição}]
A constante \textbf{\begin{ttfamily}fldAnsiChar{\_}Minuscula\end{ttfamily}} (Const \begin{ttfamily}fldAnsiChar\end{ttfamily}(\ref{mi_rtl_ui_dmxscroller_form-fldAnsiChar}) = 'c') usado na máscara do Template, informa ao componente \textbf{\begin{ttfamily}TUiDmxScroller\end{ttfamily}(\ref{mi_rtl_ui_Dmxscroller.TUiDmxScroller})} que a sequência de caracteres 'c' após o caractere \textbf{"{\textbackslash}"} representa no buffer do formulário um tipo AnsiString que só aceita caractere minúsculo.

\begin{itemize}
\item \textbf{EXEMPLO} \begin{itemize}
\item Representação de um AnsiString de 10 dígitos em um buffer de 11 bytes onde o ultimo byte contém o caractere {\#}0 informando o fim da string;

\texttt{\\\nopagebreak[3]
\\\nopagebreak[3]
}\textbf{Const}\texttt{\\\nopagebreak[3]
~~Nome~:=~'{\textbackslash}cccccccccc';~\textit{//paulo~Sérg}\\\nopagebreak[3]
~~Nome~:=~'{\textbackslash}Cccccccccc';~\textit{//Paulo~Sérg}\\
}
\end{itemize}
\end{itemize}

\end{list}
\subsection*{fldAnsiCharNUM}
\begin{list}{}{
\settowidth{\tmplength}{\textbf{Declaração}}
\setlength{\itemindent}{0cm}
\setlength{\listparindent}{0cm}
\setlength{\leftmargin}{\evensidemargin}
\addtolength{\leftmargin}{\tmplength}
\settowidth{\labelsep}{X}
\addtolength{\leftmargin}{\labelsep}
\setlength{\labelwidth}{\tmplength}
}
\begin{flushleft}
\item[\textbf{Declaração}\hfill]
\begin{ttfamily}
fldAnsiCharNUM        = TConsts.fldAnsiCharNUM     ;\end{ttfamily}


\end{flushleft}
\par
\item[\textbf{Descrição}]
A constante \textbf{\begin{ttfamily}fldAnsiCharNUM\end{ttfamily}} (Const \begin{ttfamily}fldAnsiChar\end{ttfamily}(\ref{mi_rtl_ui_dmxscroller_form-fldAnsiChar}) = '0') usado na máscara do Template, informa ao componente \textbf{\begin{ttfamily}TUiDmxScroller\end{ttfamily}(\ref{mi_rtl_ui_Dmxscroller.TUiDmxScroller})} que a sequência de caracteres '0' após o caractere \textbf{"{\textbackslash}"} representa no buffer do formulário um tipo AnsiString que só aceita caractere numérico ['0'..'9']] .

\begin{itemize}
\item \textbf{EXEMPLO} \begin{itemize}
\item Representação de um AnsiString de 11 dígitos em um buffer de 12 bytes onde o ultimo byte contém o caractere {\#}0 informando o fim da string;

\texttt{\\\nopagebreak[3]
\\\nopagebreak[3]
}\textbf{Const}\texttt{\\\nopagebreak[3]
\\\nopagebreak[3]
~~telefone~:=~'{\textbackslash}(00)~0~0000-0000'~\textit{//85~9~9702~4498}\\
}
\end{itemize}
\end{itemize}

\end{list}
\subsection*{fldAnsiCharVAL}
\begin{list}{}{
\settowidth{\tmplength}{\textbf{Declaração}}
\setlength{\itemindent}{0cm}
\setlength{\listparindent}{0cm}
\setlength{\leftmargin}{\evensidemargin}
\addtolength{\leftmargin}{\tmplength}
\settowidth{\labelsep}{X}
\addtolength{\leftmargin}{\labelsep}
\setlength{\labelwidth}{\tmplength}
}
\begin{flushleft}
\item[\textbf{Declaração}\hfill]
\begin{ttfamily}
fldAnsiCharVAL        = TConsts.fldAnsiCharVAL     ;\end{ttfamily}


\end{flushleft}
\par
\item[\textbf{Descrição}]
A constante \textbf{\begin{ttfamily}fldAnsiCharVAL\end{ttfamily}} (Const \begin{ttfamily}fldAnsiChar\end{ttfamily}(\ref{mi_rtl_ui_dmxscroller_form-fldAnsiChar}) = '0') usado na máscara do Template, informa ao componente \textbf{\begin{ttfamily}TUiDmxScroller\end{ttfamily}(\ref{mi_rtl_ui_Dmxscroller.TUiDmxScroller})} que a sequência de caracteres '0' após o caractere \textbf{"{\textbackslash}"} representa no buffer do formulário um tipo AnsiString que só aceita caractere numérico ['0'..'9']] com formatação dbase.

\begin{itemize}
\item \textbf{EXEMPLO} \begin{itemize}
\item Representação de um AnsiString de 11 dígitos em um buffer de 12 bytes onde o ultimo byte contém o caractere {\#}0 informando o fim da string;

\texttt{\\\nopagebreak[3]
\\\nopagebreak[3]
}\textbf{Const}\texttt{\\\nopagebreak[3]
\\\nopagebreak[3]
~~telefone~:=~'{\textbackslash}(NN)~N~NNNN-NNNN'~\textit{//85~9~9702~4498}\\
}
\end{itemize}
\end{itemize}

\end{list}
\subsection*{fldAPPEND}
\begin{list}{}{
\settowidth{\tmplength}{\textbf{Declaração}}
\setlength{\itemindent}{0cm}
\setlength{\listparindent}{0cm}
\setlength{\leftmargin}{\evensidemargin}
\addtolength{\leftmargin}{\tmplength}
\settowidth{\labelsep}{X}
\addtolength{\leftmargin}{\labelsep}
\setlength{\labelwidth}{\tmplength}
}
\begin{flushleft}
\item[\textbf{Declaração}\hfill]
\begin{ttfamily}
fldAPPEND             = TConsts.fldAPPEND          ;\end{ttfamily}


\end{flushleft}
\par
\item[\textbf{Descrição}]
A constante \textbf{\begin{ttfamily}fldAPPEND\end{ttfamily}} é usada para concatenar duas listas do tipo \begin{ttfamily}PSItem\end{ttfamily}(\ref{mi_rtl_ui_Dmxscroller-PSItem}).

\begin{itemize}
\item A constante \textbf{\begin{ttfamily}fldAPPEND\end{ttfamily}} é necessário porque DmxScroller trabalha com string curta e a mesma tem um tamanho de 255 caracteres, onde o tamanho está na posição 0.
\item Como usar a constante \textbf{\begin{ttfamily}fldAPPEND\end{ttfamily}}:

\begin{itemize}
\item A função \textbf{CreateAppendFields} retorna a constante \textbf{\begin{ttfamily}fldAPPEND\end{ttfamily}} mais o endereço da string a ser concatenada.

\begin{itemize}
\item \textbf{EXEMPLO}

\texttt{\\\nopagebreak[3]
\\\nopagebreak[3]
}\textbf{procedure}\texttt{~Template~:~ShortString;\\\nopagebreak[3]
~~}\textbf{Var}\texttt{\\\nopagebreak[3]
~~~~S1,s2,Template~:~TString;\\\nopagebreak[3]
}\textbf{begin}\texttt{\\\nopagebreak[3]
~~S1~:=~'~Nome~do~Aluno....:~{\textbackslash}ssssssssssssssssssssssssssssssssss';\\\nopagebreak[3]
~~s2~:=~'~Endereço~do~aluno:~{\textbackslash}sssssssssssssssssssssssss';\\\nopagebreak[3]
~~result~:=~S1+CreateAppendFields(s2);\\\nopagebreak[3]
}\textbf{end}\texttt{;\\
}
\item \textbf{NOTA} \begin{itemize}
\item A contante \textbf{\begin{ttfamily}fldAPPEND\end{ttfamily}} foi criada porque o projeto inicial foi para turbo pascal e ambiente console.
\item A versão atual podemos usar AnsiString visto que o limite do mesmo é a memória.
\item Para usar AnsiString é necessário converter para \begin{ttfamily}PSitem\end{ttfamily}(\ref{mi_rtl_ui_Dmxscroller-PSItem}) com a função: \textbf{StringToSItem}.

\begin{itemize}
\item \textbf{EXEMPLO:}

\texttt{\\\nopagebreak[3]
\\\nopagebreak[3]
}\textbf{function}\texttt{~TMI{\_}UI{\_}InputBox.DmxScroller{\_}Form1GetTemplate(aNext:~PSItem):~PSItem;\\\nopagebreak[3]
}\textbf{begin}\texttt{\\\nopagebreak[3]
~~}\textbf{with}\texttt{~DmxScroller{\_}Form1~}\textbf{do}\texttt{\\\nopagebreak[3]
~~}\textbf{begin}\texttt{\\\nopagebreak[3]
~~~~}\textbf{if}\texttt{~{\_}Template~~{$<$}{$>$}~''\\\nopagebreak[3]
~~~~}\textbf{then}\texttt{~Result~:=~StringToSItem({\_}Template,~80);\\\nopagebreak[3]
\\\nopagebreak[3]
\textit{//~~~~Result~:=~StringToSItem({\_}Template,~40,TObjectsTypes.TAlinhamento.Alinhamento{\_}Esquerda)}\\\nopagebreak[3]
\textit{//~~~~Result~:=~StringToSItem({\_}Template,~40,TObjectsTypes.TAlinhamento.Alinhamento{\_}Central)}\\\nopagebreak[3]
\textit{//~~~~Result~:=~StringToSItem({\_}Template,~40,TObjectsTypes.TAlinhamento.Alinhamento{\_}Direita)}\\\nopagebreak[3]
\textit{//~~~~Result~:=~StringToSItem({\_}Template,~80,TObjectsTypes.TAlinhamento.Alinhamento{\_}Justificado)}\\\nopagebreak[3]
\\\nopagebreak[3]
~~~~}\textbf{else}\texttt{~result~:=~}\textbf{nil}\texttt{;\\\nopagebreak[3]
~~}\textbf{end}\texttt{;\\\nopagebreak[3]
}\textbf{end}\texttt{;\\
}
\end{itemize}
\end{itemize}
\end{itemize}
\end{itemize}
\end{itemize}

\end{list}
\subsection*{fldBLOb}
\begin{list}{}{
\settowidth{\tmplength}{\textbf{Declaração}}
\setlength{\itemindent}{0cm}
\setlength{\listparindent}{0cm}
\setlength{\leftmargin}{\evensidemargin}
\addtolength{\leftmargin}{\tmplength}
\settowidth{\labelsep}{X}
\addtolength{\leftmargin}{\labelsep}
\setlength{\labelwidth}{\tmplength}
}
\begin{flushleft}
\item[\textbf{Declaração}\hfill]
\begin{ttfamily}
fldBLOb               = TConsts.fldBLOb            ;\end{ttfamily}


\end{flushleft}
\par
\item[\textbf{Descrição}]
A constante \textbf{\begin{ttfamily}fldBLOb\end{ttfamily}} indica que o campo é não formatado podendo ser um Record, porém a edição do mesmo será feito por outros meios.

\begin{itemize}
\item \textbf{NOTA} \begin{itemize}
\item Para informar ao buffer do registro que o campo é \textbf{\begin{ttfamily}fldBLOb\end{ttfamily}}, a função \textbf{CreateBlobField} é necessário.
\item A \textbf{class function \begin{ttfamily}TUiMethods.CreateBlobField\end{ttfamily}(\ref{mi_rtl_ui_methods.TUiMethods-CreateBlobField})(Len: integer; AccMode,Default: byte) : DmxIDstr;} reserva espaço para o mesmo.
\item Pendência: Preciso criar um exemplo de uso deste tipo de informação.
\end{itemize}
\end{itemize}

\end{list}
\subsection*{fldBYTE}
\begin{list}{}{
\settowidth{\tmplength}{\textbf{Declaração}}
\setlength{\itemindent}{0cm}
\setlength{\listparindent}{0cm}
\setlength{\leftmargin}{\evensidemargin}
\addtolength{\leftmargin}{\tmplength}
\settowidth{\labelsep}{X}
\addtolength{\leftmargin}{\labelsep}
\setlength{\labelwidth}{\tmplength}
}
\begin{flushleft}
\item[\textbf{Declaração}\hfill]
\begin{ttfamily}
fldBYTE               = TConsts.fldBYTE            ;\end{ttfamily}


\end{flushleft}
\par
\item[\textbf{Descrição}]
A constante \textbf{\begin{ttfamily}fldBYTE\end{ttfamily}} (Const \begin{ttfamily}fldBYTE\end{ttfamily} = 'B') usado na máscara do Template, informa ao componente \textbf{\begin{ttfamily}TUiDmxScroller\end{ttfamily}(\ref{mi_rtl_ui_Dmxscroller.TUiDmxScroller})} que a sequência de caracteres 'B' após o caractere \textbf{"{\textbackslash}"} representa no buffer do formulário um tipo byte.

\begin{itemize}
\item \textbf{EXEMPLO}

\texttt{\\\nopagebreak[3]
\\\nopagebreak[3]
}\textbf{Const}\texttt{\\\nopagebreak[3]
~~~idade~:=~'{\textbackslash}BB'~\textit{//Os~dois~dígitos~estarão~em~um~buffer~de~1~byte;}\\
}
\end{itemize}

\end{list}
\subsection*{fldBoolean}
\begin{list}{}{
\settowidth{\tmplength}{\textbf{Declaração}}
\setlength{\itemindent}{0cm}
\setlength{\listparindent}{0cm}
\setlength{\leftmargin}{\evensidemargin}
\addtolength{\leftmargin}{\tmplength}
\settowidth{\labelsep}{X}
\addtolength{\leftmargin}{\labelsep}
\setlength{\labelwidth}{\tmplength}
}
\begin{flushleft}
\item[\textbf{Declaração}\hfill]
\begin{ttfamily}
fldBoolean          =   TConsts.fldBoolean;\end{ttfamily}


\end{flushleft}
\par
\item[\textbf{Descrição}]
A constante \textbf{\begin{ttfamily}fldBoolean\end{ttfamily}} (\begin{ttfamily}fldBoolean\end{ttfamily} = 'X') indica que o campo é do tipo byte e só pode ter dois valores.

\begin{itemize}
\item \textbf{NOTA} \begin{itemize}
\item Valores possíveis: \begin{itemize}
\item 0 {-} False; não
\item 1 = True; sim
\end{itemize}
\item A forma de editá{-}los deve ser com o componente checkbox.
\end{itemize}
\item \textbf{EXEMPLO}

\texttt{\\\nopagebreak[3]
\\\nopagebreak[3]
}\textbf{Resourcestring}\texttt{\\\nopagebreak[3]
~~tmp{\_}Aceita~=~'{\textbackslash}[X]'+ChFN+'Aceita{\_}contrato'+CharHint+'Aceita~os~termos~do~contrato?';\\\nopagebreak[3]
~~Template~=~tmp{\_}Aceita+'~Aceita~os~termos~do~contrato~';\\
}
\end{itemize}

\end{list}
\subsection*{fldCONTRACTION}
\begin{list}{}{
\settowidth{\tmplength}{\textbf{Declaração}}
\setlength{\itemindent}{0cm}
\setlength{\listparindent}{0cm}
\setlength{\leftmargin}{\evensidemargin}
\addtolength{\leftmargin}{\tmplength}
\settowidth{\labelsep}{X}
\addtolength{\leftmargin}{\labelsep}
\setlength{\labelwidth}{\tmplength}
}
\begin{flushleft}
\item[\textbf{Declaração}\hfill]
\begin{ttfamily}
fldCONTRACTION        = TConsts.fldCONTRACTION     ;\end{ttfamily}


\end{flushleft}
\par
\item[\textbf{Descrição}]
A constante \textbf{\begin{ttfamily}fldCONTRACTION\end{ttfamily}} ...

\end{list}
\subsection*{fldData}
\begin{list}{}{
\settowidth{\tmplength}{\textbf{Declaração}}
\setlength{\itemindent}{0cm}
\setlength{\listparindent}{0cm}
\setlength{\leftmargin}{\evensidemargin}
\addtolength{\leftmargin}{\tmplength}
\settowidth{\labelsep}{X}
\addtolength{\leftmargin}{\labelsep}
\setlength{\labelwidth}{\tmplength}
}
\begin{flushleft}
\item[\textbf{Declaração}\hfill]
\begin{ttfamily}
fldData               = TConsts.fldData            ;\end{ttfamily}


\end{flushleft}
\par
\item[\textbf{Descrição}]
A constante \textbf{\begin{ttfamily}fldData\end{ttfamily}} ...

\end{list}
\subsection*{FldDateTimeDos}
\begin{list}{}{
\settowidth{\tmplength}{\textbf{Declaração}}
\setlength{\itemindent}{0cm}
\setlength{\listparindent}{0cm}
\setlength{\leftmargin}{\evensidemargin}
\addtolength{\leftmargin}{\tmplength}
\settowidth{\labelsep}{X}
\addtolength{\leftmargin}{\labelsep}
\setlength{\labelwidth}{\tmplength}
}
\begin{flushleft}
\item[\textbf{Declaração}\hfill]
\begin{ttfamily}
FldDateTimeDos        = TConsts.FldDateTimeDos     ;\end{ttfamily}


\end{flushleft}
\par
\item[\textbf{Descrição}]
A constante \textbf{\begin{ttfamily}FldDateTimeDos\end{ttfamily}} ...

\end{list}
\subsection*{fldENUM}
\begin{list}{}{
\settowidth{\tmplength}{\textbf{Declaração}}
\setlength{\itemindent}{0cm}
\setlength{\listparindent}{0cm}
\setlength{\leftmargin}{\evensidemargin}
\addtolength{\leftmargin}{\tmplength}
\settowidth{\labelsep}{X}
\addtolength{\leftmargin}{\labelsep}
\setlength{\labelwidth}{\tmplength}
}
\begin{flushleft}
\item[\textbf{Declaração}\hfill]
\begin{ttfamily}
fldENUM               = TConsts.fldENUM            ;\end{ttfamily}


\end{flushleft}
\par
\item[\textbf{Descrição}]
A constante \textbf{\begin{ttfamily}fldENUM\end{ttfamily}} (\begin{ttfamily}fldENUM\end{ttfamily}={\^{}}E) é um campo do tipo byte(0..255) que contém uma lista de string que são selecionadas por um componente tipo ComboBox.

\begin{itemize}
\item \textbf{EXEMPLO}

\texttt{\\\nopagebreak[3]
\\\nopagebreak[3]
~~}\textbf{Const}\texttt{~tmpMidia~:~PSitem~=~}\textbf{nil}\texttt{;\\\nopagebreak[3]
}\textbf{begin}\texttt{\\\nopagebreak[3]
~~tmpMidia~:=~CreateEnumField(TRUE,~accNormal,~0,\\\nopagebreak[3]
~~~~~~~~~~~~~~~~~~~~~~~~~~~~~~NewSItem('~indefinido~',\\\nopagebreak[3]
~~~~~~~~~~~~~~~~~~~~~~~~~~~~~~NewSItem('~PenDriver~~',\\\nopagebreak[3]
~~~~~~~~~~~~~~~~~~~~~~~~~~~~~~NewSItem('~SSD~~~~~~~~',\\\nopagebreak[3]
~~~~~~~~~~~~~~~~~~~~~~~~~~~~~~}\textbf{nil}\texttt{))))+CharFieldName+'Midia;\\\nopagebreak[3]
\\\nopagebreak[3]
~~Template~=~NewSItem('~~~Eu~uso~~'+~tmpMidia~+~'~~em~meu~computador.~\\
}
\end{itemize}

\end{list}
\subsection*{CharExecAction}
\begin{list}{}{
\settowidth{\tmplength}{\textbf{Declaração}}
\setlength{\itemindent}{0cm}
\setlength{\listparindent}{0cm}
\setlength{\leftmargin}{\evensidemargin}
\addtolength{\leftmargin}{\tmplength}
\settowidth{\labelsep}{X}
\addtolength{\leftmargin}{\labelsep}
\setlength{\labelwidth}{\tmplength}
}
\begin{flushleft}
\item[\textbf{Declaração}\hfill]
\begin{ttfamily}
CharExecAction        = TConsts.CharExecAction        ;\end{ttfamily}


\end{flushleft}
\par
\item[\textbf{Descrição}]
A contante \textbf{\begin{ttfamily}CharExecAction\end{ttfamily}} é usado para associar ao campo atual uma classe \textbf{TAction}.

\begin{itemize}
\item \textbf{NOTA} \begin{itemize}
\item O interpretador de Templates associa a ação do Template ao corrente campo.
\end{itemize}
\item \textbf{EXEMPLO DE USO DE AÇÕES NO TEMPLATE} \begin{enumerate}
\setcounter{enumi}{0} \setcounter{enumii}{0} \setcounter{enumiii}{0} \setcounter{enumiv}{0} 
\item Se o atributo \textbf{Fieldnum} do campo for diferente de zero, então o \textbf{rótulo} do botão associado a ação será o caracteres 🔍 e a ação pode atualizar o buffer do campo. \begin{itemize}
\item No exemplo a seguir um rótulo e um campo de cliente:

\texttt{\\\nopagebreak[3]
\\\nopagebreak[3]
NewSItem('~Cliente:~'+'{\textbackslash}LLLLL'+CharExecAction+CreateExecAction(Action{\_}pesquisa)\\
}
\end{itemize}
\setcounter{enumi}{1} \setcounter{enumii}{1} \setcounter{enumiii}{1} \setcounter{enumiv}{1} 
\item Se o atributo \textbf{Fieldnum} do campo for igual a zero, então a rótulo do botão será o rótulo do campo. \begin{itemize}
\item No exemplo a seguir um rótulo de novo cliente (icons 🆕) e um botão ok (icons 🆗)

\texttt{\\\nopagebreak[3]
\\\nopagebreak[3]
NewSItem('~~🆕~{\&}Novo~cliente:~'+CharExecAction+CreateExecAction(Action{\_}Novo)+\\\nopagebreak[3]
~~~~~~~~~'~~~~~~~🆗~~'+CharExecAction+CreateExecAction(Action{\_}Ok)\\
}
\end{itemize}
\end{enumerate}
\end{itemize}

\end{list}
\subsection*{fldExtended}
\begin{list}{}{
\settowidth{\tmplength}{\textbf{Declaração}}
\setlength{\itemindent}{0cm}
\setlength{\listparindent}{0cm}
\setlength{\leftmargin}{\evensidemargin}
\addtolength{\leftmargin}{\tmplength}
\settowidth{\labelsep}{X}
\addtolength{\leftmargin}{\labelsep}
\setlength{\labelwidth}{\tmplength}
}
\begin{flushleft}
\item[\textbf{Declaração}\hfill]
\begin{ttfamily}
fldExtended           = TConsts.fldExtended        ;\end{ttfamily}


\end{flushleft}
\par
\item[\textbf{Descrição}]
A constante \textbf{\begin{ttfamily}fldExtended\end{ttfamily}} (\begin{ttfamily}fldExtended\end{ttfamily}='E') usado na máscara do Template, informa ao componente \textbf{\begin{ttfamily}TUiDmxScroller\end{ttfamily}(\ref{mi_rtl_ui_Dmxscroller.TUiDmxScroller})} que a sequência de caracteres 'E' após o caractere \textbf{"{\textbackslash}"} representa no buffer do formulário um tipo Extended.

\begin{itemize}
\item \textbf{EXEMPLO}

\texttt{\\\nopagebreak[3]
\\\nopagebreak[3]
}\textbf{Const}\texttt{\\\nopagebreak[3]
~~~Valor~:=~'{\textbackslash}EEE,EEE,EEE,EEE,EE'~\textit{//Todos~os~números~editados~nesta}\\\nopagebreak[3]
~~~~~~~~~~~~~~~~~~~~~~~~~~~~~~~~~~\textit{//mascara,~estarão~em~um~buffer~de~10~bytes;}\\
}
\end{itemize}

\end{list}
\subsection*{CharFieldName}
\begin{list}{}{
\settowidth{\tmplength}{\textbf{Declaração}}
\setlength{\itemindent}{0cm}
\setlength{\listparindent}{0cm}
\setlength{\leftmargin}{\evensidemargin}
\addtolength{\leftmargin}{\tmplength}
\settowidth{\labelsep}{X}
\addtolength{\leftmargin}{\labelsep}
\setlength{\labelwidth}{\tmplength}
}
\begin{flushleft}
\item[\textbf{Declaração}\hfill]
\begin{ttfamily}
CharFieldName          = TConsts.CharFieldName       ;\end{ttfamily}


\end{flushleft}
\par
\item[\textbf{Descrição}]
A constante \textbf{\begin{ttfamily}CharFieldName\end{ttfamily}} usada para informar o nome do campo informado antes deste caractere.

\begin{itemize}
\item \textbf{EXEMPLO}

\texttt{\\\nopagebreak[3]
\\\nopagebreak[3]
}\textbf{Const}\texttt{\\\nopagebreak[3]
~~idade~:=~'{\textbackslash}BB'+CharFieldName+'idade'\\
}
\end{itemize}

\end{list}
\subsection*{ChFN}
\begin{list}{}{
\settowidth{\tmplength}{\textbf{Declaração}}
\setlength{\itemindent}{0cm}
\setlength{\listparindent}{0cm}
\setlength{\leftmargin}{\evensidemargin}
\addtolength{\leftmargin}{\tmplength}
\settowidth{\labelsep}{X}
\addtolength{\leftmargin}{\labelsep}
\setlength{\labelwidth}{\tmplength}
}
\begin{flushleft}
\item[\textbf{Declaração}\hfill]
\begin{ttfamily}
ChFN                 = TConsts.ChFN              ;\end{ttfamily}


\end{flushleft}
\par
\item[\textbf{Descrição}]
A constante \textbf{\begin{ttfamily}ChFN\end{ttfamily}} é igual a \begin{ttfamily}CharFieldName\end{ttfamily}(\ref{mi_rtl_ui_dmxscroller_form-CharFieldName}). Foi criada para facilitar seu uso.

\end{list}
\subsection*{fldHexValue}
\begin{list}{}{
\settowidth{\tmplength}{\textbf{Declaração}}
\setlength{\itemindent}{0cm}
\setlength{\listparindent}{0cm}
\setlength{\leftmargin}{\evensidemargin}
\addtolength{\leftmargin}{\tmplength}
\settowidth{\labelsep}{X}
\addtolength{\leftmargin}{\labelsep}
\setlength{\labelwidth}{\tmplength}
}
\begin{flushleft}
\item[\textbf{Declaração}\hfill]
\begin{ttfamily}
fldHexValue           = TConsts.fldHexValue        ;\end{ttfamily}


\end{flushleft}
\par
\item[\textbf{Descrição}]
A constante \textbf{\begin{ttfamily}fldHexValue\end{ttfamily}} ...

\end{list}
\subsection*{fldLData}
\begin{list}{}{
\settowidth{\tmplength}{\textbf{Declaração}}
\setlength{\itemindent}{0cm}
\setlength{\listparindent}{0cm}
\setlength{\leftmargin}{\evensidemargin}
\addtolength{\leftmargin}{\tmplength}
\settowidth{\labelsep}{X}
\addtolength{\leftmargin}{\labelsep}
\setlength{\labelwidth}{\tmplength}
}
\begin{flushleft}
\item[\textbf{Declaração}\hfill]
\begin{ttfamily}
fldLData              = TConsts.fldLData           ;\end{ttfamily}


\end{flushleft}
\par
\item[\textbf{Descrição}]
A constante \textbf{\begin{ttfamily}fldLData\end{ttfamily}} ...

\end{list}
\subsection*{fldLHora}
\begin{list}{}{
\settowidth{\tmplength}{\textbf{Declaração}}
\setlength{\itemindent}{0cm}
\setlength{\listparindent}{0cm}
\setlength{\leftmargin}{\evensidemargin}
\addtolength{\leftmargin}{\tmplength}
\settowidth{\labelsep}{X}
\addtolength{\leftmargin}{\labelsep}
\setlength{\labelwidth}{\tmplength}
}
\begin{flushleft}
\item[\textbf{Declaração}\hfill]
\begin{ttfamily}
fldLHora              = TConsts.fldLHora           ;\end{ttfamily}


\end{flushleft}
\par
\item[\textbf{Descrição}]
A constante \textbf{\begin{ttfamily}fldLHora\end{ttfamily}} ...

\end{list}
\subsection*{fldLONGINT}
\begin{list}{}{
\settowidth{\tmplength}{\textbf{Declaração}}
\setlength{\itemindent}{0cm}
\setlength{\listparindent}{0cm}
\setlength{\leftmargin}{\evensidemargin}
\addtolength{\leftmargin}{\tmplength}
\settowidth{\labelsep}{X}
\addtolength{\leftmargin}{\labelsep}
\setlength{\labelwidth}{\tmplength}
}
\begin{flushleft}
\item[\textbf{Declaração}\hfill]
\begin{ttfamily}
fldLONGINT            = TConsts.fldLONGINT         ;\end{ttfamily}


\end{flushleft}
\par
\item[\textbf{Descrição}]
A constante \textbf{\begin{ttfamily}fldLONGINT\end{ttfamily}} ...

\end{list}
\subsection*{FldMemo}
\begin{list}{}{
\settowidth{\tmplength}{\textbf{Declaração}}
\setlength{\itemindent}{0cm}
\setlength{\listparindent}{0cm}
\setlength{\leftmargin}{\evensidemargin}
\addtolength{\leftmargin}{\tmplength}
\settowidth{\labelsep}{X}
\addtolength{\leftmargin}{\labelsep}
\setlength{\labelwidth}{\tmplength}
}
\begin{flushleft}
\item[\textbf{Declaração}\hfill]
\begin{ttfamily}
FldMemo               = TConsts.FldMemo            ;\end{ttfamily}


\end{flushleft}
\par
\item[\textbf{Descrição}]
A constante \textbf{\begin{ttfamily}FldMemo\end{ttfamily}} ...

\end{list}
\subsection*{FldOperador}
\begin{list}{}{
\settowidth{\tmplength}{\textbf{Declaração}}
\setlength{\itemindent}{0cm}
\setlength{\listparindent}{0cm}
\setlength{\leftmargin}{\evensidemargin}
\addtolength{\leftmargin}{\tmplength}
\settowidth{\labelsep}{X}
\addtolength{\leftmargin}{\labelsep}
\setlength{\labelwidth}{\tmplength}
}
\begin{flushleft}
\item[\textbf{Declaração}\hfill]
\begin{ttfamily}
FldOperador           = TConsts.FldOperador        ;\end{ttfamily}


\end{flushleft}
\par
\item[\textbf{Descrição}]
A constante \textbf{\begin{ttfamily}FldOperador\end{ttfamily}} ...

\end{list}
\subsection*{FldRadioButton}
\begin{list}{}{
\settowidth{\tmplength}{\textbf{Declaração}}
\setlength{\itemindent}{0cm}
\setlength{\listparindent}{0cm}
\setlength{\leftmargin}{\evensidemargin}
\addtolength{\leftmargin}{\tmplength}
\settowidth{\labelsep}{X}
\addtolength{\leftmargin}{\labelsep}
\setlength{\labelwidth}{\tmplength}
}
\begin{flushleft}
\item[\textbf{Declaração}\hfill]
\begin{ttfamily}
FldRadioButton        = TConsts.FldRadioButton     ;\end{ttfamily}


\end{flushleft}
\par
\item[\textbf{Descrição}]
A constante \textbf{\begin{ttfamily}FldRadioButton\end{ttfamily}} ...

\end{list}
\subsection*{fldReal4}
\begin{list}{}{
\settowidth{\tmplength}{\textbf{Declaração}}
\setlength{\itemindent}{0cm}
\setlength{\listparindent}{0cm}
\setlength{\leftmargin}{\evensidemargin}
\addtolength{\leftmargin}{\tmplength}
\settowidth{\labelsep}{X}
\addtolength{\leftmargin}{\labelsep}
\setlength{\labelwidth}{\tmplength}
}
\begin{flushleft}
\item[\textbf{Declaração}\hfill]
\begin{ttfamily}
fldReal4              = TConsts.fldReal4           ;\end{ttfamily}


\end{flushleft}
\par
\item[\textbf{Descrição}]
A constante \textbf{\begin{ttfamily}fldReal4\end{ttfamily}} ...

\end{list}
\subsection*{fldReal4P}
\begin{list}{}{
\settowidth{\tmplength}{\textbf{Declaração}}
\setlength{\itemindent}{0cm}
\setlength{\listparindent}{0cm}
\setlength{\leftmargin}{\evensidemargin}
\addtolength{\leftmargin}{\tmplength}
\settowidth{\labelsep}{X}
\addtolength{\leftmargin}{\labelsep}
\setlength{\labelwidth}{\tmplength}
}
\begin{flushleft}
\item[\textbf{Declaração}\hfill]
\begin{ttfamily}
fldReal4P             = TConsts.fldReal4P          ;\end{ttfamily}


\end{flushleft}
\par
\item[\textbf{Descrição}]
A constante \textbf{\begin{ttfamily}fldReal4P\end{ttfamily}} ...

\end{list}
\subsection*{fldReal4Positivo}
\begin{list}{}{
\settowidth{\tmplength}{\textbf{Declaração}}
\setlength{\itemindent}{0cm}
\setlength{\listparindent}{0cm}
\setlength{\leftmargin}{\evensidemargin}
\addtolength{\leftmargin}{\tmplength}
\settowidth{\labelsep}{X}
\addtolength{\leftmargin}{\labelsep}
\setlength{\labelwidth}{\tmplength}
}
\begin{flushleft}
\item[\textbf{Declaração}\hfill]
\begin{ttfamily}
fldReal4Positivo      = TConsts.fldReal4Positivo   ;\end{ttfamily}


\end{flushleft}
\par
\item[\textbf{Descrição}]
A constante \textbf{\begin{ttfamily}fldReal4Positivo\end{ttfamily}} ...

\end{list}
\subsection*{fldReal4PPositivo}
\begin{list}{}{
\settowidth{\tmplength}{\textbf{Declaração}}
\setlength{\itemindent}{0cm}
\setlength{\listparindent}{0cm}
\setlength{\leftmargin}{\evensidemargin}
\addtolength{\leftmargin}{\tmplength}
\settowidth{\labelsep}{X}
\addtolength{\leftmargin}{\labelsep}
\setlength{\labelwidth}{\tmplength}
}
\begin{flushleft}
\item[\textbf{Declaração}\hfill]
\begin{ttfamily}
fldReal4PPositivo     = TConsts.fldReal4PPositivo  ;\end{ttfamily}


\end{flushleft}
\par
\item[\textbf{Descrição}]
A constante \textbf{\begin{ttfamily}fldReal4PPositivo\end{ttfamily}} ...

\end{list}
\subsection*{fldRealNum}
\begin{list}{}{
\settowidth{\tmplength}{\textbf{Declaração}}
\setlength{\itemindent}{0cm}
\setlength{\listparindent}{0cm}
\setlength{\leftmargin}{\evensidemargin}
\addtolength{\leftmargin}{\tmplength}
\settowidth{\labelsep}{X}
\addtolength{\leftmargin}{\labelsep}
\setlength{\labelwidth}{\tmplength}
}
\begin{flushleft}
\item[\textbf{Declaração}\hfill]
\begin{ttfamily}
fldRealNum            = TConsts.fldRealNum         ;\end{ttfamily}


\end{flushleft}
\par
\item[\textbf{Descrição}]
A constante \textbf{\begin{ttfamily}fldRealNum\end{ttfamily}} ...

\end{list}
\subsection*{fldRealNum{\_}Positivo}
\begin{list}{}{
\settowidth{\tmplength}{\textbf{Declaração}}
\setlength{\itemindent}{0cm}
\setlength{\listparindent}{0cm}
\setlength{\leftmargin}{\evensidemargin}
\addtolength{\leftmargin}{\tmplength}
\settowidth{\labelsep}{X}
\addtolength{\leftmargin}{\labelsep}
\setlength{\labelwidth}{\tmplength}
}
\begin{flushleft}
\item[\textbf{Declaração}\hfill]
\begin{ttfamily}
fldRealNum{\_}Positivo   = TConsts.fldRealNum{\_}Positivo;\end{ttfamily}


\end{flushleft}
\par
\item[\textbf{Descrição}]
A constante \textbf{\begin{ttfamily}fldRealNum{\_}Positivo\end{ttfamily}} ...

\end{list}
\subsection*{FldSData}
\begin{list}{}{
\settowidth{\tmplength}{\textbf{Declaração}}
\setlength{\itemindent}{0cm}
\setlength{\listparindent}{0cm}
\setlength{\leftmargin}{\evensidemargin}
\addtolength{\leftmargin}{\tmplength}
\settowidth{\labelsep}{X}
\addtolength{\leftmargin}{\labelsep}
\setlength{\labelwidth}{\tmplength}
}
\begin{flushleft}
\item[\textbf{Declaração}\hfill]
\begin{ttfamily}
FldSData              = TConsts.FldSData           ;\end{ttfamily}


\end{flushleft}
\par
\item[\textbf{Descrição}]
A constante \textbf{\begin{ttfamily}FldSData\end{ttfamily}} ...

\end{list}
\subsection*{FldSDateTimeDos}
\begin{list}{}{
\settowidth{\tmplength}{\textbf{Declaração}}
\setlength{\itemindent}{0cm}
\setlength{\listparindent}{0cm}
\setlength{\leftmargin}{\evensidemargin}
\addtolength{\leftmargin}{\tmplength}
\settowidth{\labelsep}{X}
\addtolength{\leftmargin}{\labelsep}
\setlength{\labelwidth}{\tmplength}
}
\begin{flushleft}
\item[\textbf{Declaração}\hfill]
\begin{ttfamily}
FldSDateTimeDos       = TConsts.FldSDateTimeDos    ;\end{ttfamily}


\end{flushleft}
\par
\item[\textbf{Descrição}]
A constante \textbf{\begin{ttfamily}FldSDateTimeDos\end{ttfamily}} ...

\end{list}
\subsection*{FldSHora}
\begin{list}{}{
\settowidth{\tmplength}{\textbf{Declaração}}
\setlength{\itemindent}{0cm}
\setlength{\listparindent}{0cm}
\setlength{\leftmargin}{\evensidemargin}
\addtolength{\leftmargin}{\tmplength}
\settowidth{\labelsep}{X}
\addtolength{\leftmargin}{\labelsep}
\setlength{\labelwidth}{\tmplength}
}
\begin{flushleft}
\item[\textbf{Declaração}\hfill]
\begin{ttfamily}
FldSHora              = TConsts.FldSHora           ;\end{ttfamily}


\end{flushleft}
\par
\item[\textbf{Descrição}]
A constante \textbf{\begin{ttfamily}FldSHora\end{ttfamily}} ...

\end{list}
\subsection*{fldSHORTINT}
\begin{list}{}{
\settowidth{\tmplength}{\textbf{Declaração}}
\setlength{\itemindent}{0cm}
\setlength{\listparindent}{0cm}
\setlength{\leftmargin}{\evensidemargin}
\addtolength{\leftmargin}{\tmplength}
\settowidth{\labelsep}{X}
\addtolength{\leftmargin}{\labelsep}
\setlength{\labelwidth}{\tmplength}
}
\begin{flushleft}
\item[\textbf{Declaração}\hfill]
\begin{ttfamily}
fldSHORTINT           = TConsts.fldSHORTINT        ;\end{ttfamily}


\end{flushleft}
\par
\item[\textbf{Descrição}]
A constante \textbf{\begin{ttfamily}fldSHORTINT\end{ttfamily}} ...

\end{list}
\subsection*{fldSItems}
\begin{list}{}{
\settowidth{\tmplength}{\textbf{Declaração}}
\setlength{\itemindent}{0cm}
\setlength{\listparindent}{0cm}
\setlength{\leftmargin}{\evensidemargin}
\addtolength{\leftmargin}{\tmplength}
\settowidth{\labelsep}{X}
\addtolength{\leftmargin}{\labelsep}
\setlength{\labelwidth}{\tmplength}
}
\begin{flushleft}
\item[\textbf{Declaração}\hfill]
\begin{ttfamily}
fldSItems             = TConsts.fldSItems          ;\end{ttfamily}


\end{flushleft}
\par
\item[\textbf{Descrição}]
A constante \textbf{\begin{ttfamily}fldSItems\end{ttfamily}} ...

\end{list}
\subsection*{fldSmallInt}
\begin{list}{}{
\settowidth{\tmplength}{\textbf{Declaração}}
\setlength{\itemindent}{0cm}
\setlength{\listparindent}{0cm}
\setlength{\leftmargin}{\evensidemargin}
\addtolength{\leftmargin}{\tmplength}
\settowidth{\labelsep}{X}
\addtolength{\leftmargin}{\labelsep}
\setlength{\labelwidth}{\tmplength}
}
\begin{flushleft}
\item[\textbf{Declaração}\hfill]
\begin{ttfamily}
fldSmallInt           = TConsts.fldSmallInt        ;\end{ttfamily}


\end{flushleft}
\par
\item[\textbf{Descrição}]
A constante \textbf{\begin{ttfamily}fldSmallInt\end{ttfamily}} ...

\end{list}
\subsection*{fldSmallWORD}
\begin{list}{}{
\settowidth{\tmplength}{\textbf{Declaração}}
\setlength{\itemindent}{0cm}
\setlength{\listparindent}{0cm}
\setlength{\leftmargin}{\evensidemargin}
\addtolength{\leftmargin}{\tmplength}
\settowidth{\labelsep}{X}
\addtolength{\leftmargin}{\labelsep}
\setlength{\labelwidth}{\tmplength}
}
\begin{flushleft}
\item[\textbf{Declaração}\hfill]
\begin{ttfamily}
fldSmallWORD          = TConsts.fldSmallWORD       ;\end{ttfamily}


\end{flushleft}
\par
\item[\textbf{Descrição}]
A constante \textbf{\begin{ttfamily}fldSmallWORD\end{ttfamily}} ...

\end{list}
\subsection*{fldSTR}
\begin{list}{}{
\settowidth{\tmplength}{\textbf{Declaração}}
\setlength{\itemindent}{0cm}
\setlength{\listparindent}{0cm}
\setlength{\leftmargin}{\evensidemargin}
\addtolength{\leftmargin}{\tmplength}
\settowidth{\labelsep}{X}
\addtolength{\leftmargin}{\labelsep}
\setlength{\labelwidth}{\tmplength}
}
\begin{flushleft}
\item[\textbf{Declaração}\hfill]
\begin{ttfamily}
fldSTR                = TConsts.fldSTR          ;\end{ttfamily}


\end{flushleft}
\par
\item[\textbf{Descrição}]
A constante \textbf{\begin{ttfamily}fldSTR\end{ttfamily}} (Const \begin{ttfamily}fldStr\end{ttfamily} = 'S') usado na máscara do Template, informa ao componente \textbf{\begin{ttfamily}TUiDmxScroller\end{ttfamily}(\ref{mi_rtl_ui_Dmxscroller.TUiDmxScroller})} que a sequência de caracteres 'S' após o caractere \textbf{"{\textbackslash}"} representa no buffer do formulário um tipo ShortString que só aceita caractere maiúsculo.

\begin{itemize}
\item \textbf{EXEMPLO} \begin{itemize}
\item Representação de um string de 10 dígitos em um buffer de 11 bytes onde o byte zero contém o tamanho da string;

\texttt{\\\nopagebreak[3]
\\\nopagebreak[3]
}\textbf{Const}\texttt{\\\nopagebreak[3]
~~Nome~:=~'{\textbackslash}SSSSSSSSSS'~~\textit{//PAULO~SÉRG}\\
}
\end{itemize}
\end{itemize}

\end{list}
\subsection*{fldSTR{\_}Minuscula}
\begin{list}{}{
\settowidth{\tmplength}{\textbf{Declaração}}
\setlength{\itemindent}{0cm}
\setlength{\listparindent}{0cm}
\setlength{\leftmargin}{\evensidemargin}
\addtolength{\leftmargin}{\tmplength}
\settowidth{\labelsep}{X}
\addtolength{\leftmargin}{\labelsep}
\setlength{\labelwidth}{\tmplength}
}
\begin{flushleft}
\item[\textbf{Declaração}\hfill]
\begin{ttfamily}
fldSTR{\_}Minuscula      = TConsts.fldSTR{\_}Minuscula;\end{ttfamily}


\end{flushleft}
\par
\item[\textbf{Descrição}]
A constante \textbf{\begin{ttfamily}fldSTR{\_}Minuscula\end{ttfamily}} (Const \begin{ttfamily}fldSTR{\_}Minuscula\end{ttfamily} = 's') usado na máscara do Template, informa ao componente \textbf{\begin{ttfamily}TUiDmxScroller\end{ttfamily}(\ref{mi_rtl_ui_Dmxscroller.TUiDmxScroller})} que a sequência de caracteres 's' após o caractere \textbf{"{\textbackslash}"} representa no buffer do formulário um tipo ShortString que só aceita caractere minúscula.

\begin{itemize}
\item \textbf{EXEMPLO} \begin{itemize}
\item Representação de um string de 10 dígitos em um buffer de 11 bytes onde o byte zero contém o tamanho da string;

\texttt{\\\nopagebreak[3]
\\\nopagebreak[3]
}\textbf{Const}\texttt{\\\nopagebreak[3]
~~Nome~:=~'{\textbackslash}ssssssssss'~\textit{//paulo~sérg}\\\nopagebreak[3]
~~Nome~:=~'{\textbackslash}Ssssssssss'~\textit{//Paulo~sérg}\\
}
\end{itemize}
\end{itemize}

\end{list}
\subsection*{fldSTRNUM}
\begin{list}{}{
\settowidth{\tmplength}{\textbf{Declaração}}
\setlength{\itemindent}{0cm}
\setlength{\listparindent}{0cm}
\setlength{\leftmargin}{\evensidemargin}
\addtolength{\leftmargin}{\tmplength}
\settowidth{\labelsep}{X}
\addtolength{\leftmargin}{\labelsep}
\setlength{\labelwidth}{\tmplength}
}
\begin{flushleft}
\item[\textbf{Declaração}\hfill]
\begin{ttfamily}
fldSTRNUM             = TConsts.fldSTRNUM       ;\end{ttfamily}


\end{flushleft}
\par
\item[\textbf{Descrição}]
A constante \textbf{\begin{ttfamily}fldSTRNUM\end{ttfamily}} (Const \begin{ttfamily}fldSTRNUM\end{ttfamily} = '{\#}') usado na máscara do Template, informa ao componente \textbf{\begin{ttfamily}TUiDmxScroller\end{ttfamily}(\ref{mi_rtl_ui_Dmxscroller.TUiDmxScroller})} que a sequência de caracteres '{\#}' após o caractere \textbf{"{\textbackslash}"} representa no buffer do formulário um tipo ShortString que só aceita caractere numérico.

\begin{itemize}
\item \textbf{EXEMPLO} \begin{itemize}
\item Representação de um string de 11 dígitos em um buffer de 12 bytes onde o byte zero contém o tamanho da string;

\texttt{\\\nopagebreak[3]
\\\nopagebreak[3]
}\textbf{Const}\texttt{\\\nopagebreak[3]
~~telefone~:=~'{\textbackslash}({\#}{\#})~{\#}~{\#}{\#}{\#}{\#}-{\#}{\#}{\#}{\#}'~\textit{//85~9~9702~4498}\\
}
\end{itemize}
\end{itemize}

\end{list}
\subsection*{CharUpperlimit}
\begin{list}{}{
\settowidth{\tmplength}{\textbf{Declaração}}
\setlength{\itemindent}{0cm}
\setlength{\listparindent}{0cm}
\setlength{\leftmargin}{\evensidemargin}
\addtolength{\leftmargin}{\tmplength}
\settowidth{\labelsep}{X}
\addtolength{\leftmargin}{\labelsep}
\setlength{\labelwidth}{\tmplength}
}
\begin{flushleft}
\item[\textbf{Declaração}\hfill]
\begin{ttfamily}
CharUpperlimit         = TConsts.CharUpperlimit      ;\end{ttfamily}


\end{flushleft}
\par
\item[\textbf{Descrição}]
A constante \textbf{\begin{ttfamily}CharUpperlimit\end{ttfamily}} (\begin{ttfamily}CharUpperlimit\end{ttfamily}={\^{}}U) permite informar um limite superior para campos do tipo byte.

\begin{itemize}
\item O gerador de formulário deve usar o conteúdo do campo pDmxFieldRec.Upperlimit para criticar se o valor do campo está na faixa entre 1 e pDmxFieldRec.Upperlimit.
\item O valor zero significa que o campo está nulo.
\item \textbf{EXEMPLO} \begin{itemize}
\item Um campo onde o seu conteúdo não ultrapasse um byte, pode ser informado no Template da seguinte forma:

\texttt{\\\nopagebreak[3]
\\\nopagebreak[3]
}\textbf{Const}\texttt{\\\nopagebreak[3]
~~idade~:=~'{\textbackslash}BBB+CharUpperlimit+{\#}130+CharHint+'Não~existe~humanos~com~a~idade~superior~a~130~anos.\\
}
\end{itemize}
\end{itemize}

\end{list}
\subsection*{fldZEROMOD}
\begin{list}{}{
\settowidth{\tmplength}{\textbf{Declaração}}
\setlength{\itemindent}{0cm}
\setlength{\listparindent}{0cm}
\setlength{\leftmargin}{\evensidemargin}
\addtolength{\leftmargin}{\tmplength}
\settowidth{\labelsep}{X}
\addtolength{\leftmargin}{\labelsep}
\setlength{\labelwidth}{\tmplength}
}
\begin{flushleft}
\item[\textbf{Declaração}\hfill]
\begin{ttfamily}
fldZEROMOD            = TConsts.fldZEROMOD         ;\end{ttfamily}


\end{flushleft}
\par
\item[\textbf{Descrição}]
A constante \textbf{\begin{ttfamily}fldZEROMOD\end{ttfamily}} ...

\end{list}
\subsection*{CharShowPassword}
\begin{list}{}{
\settowidth{\tmplength}{\textbf{Declaração}}
\setlength{\itemindent}{0cm}
\setlength{\listparindent}{0cm}
\setlength{\leftmargin}{\evensidemargin}
\addtolength{\leftmargin}{\tmplength}
\settowidth{\labelsep}{X}
\addtolength{\leftmargin}{\labelsep}
\setlength{\labelwidth}{\tmplength}
}
\begin{flushleft}
\item[\textbf{Declaração}\hfill]
\begin{ttfamily}
CharShowPassword          = TConsts.CharShowPassword       ;\end{ttfamily}


\end{flushleft}
\par
\item[\textbf{Descrição}]
A constante \textbf{\begin{ttfamily}CharShowPassword\end{ttfamily}} informa para controle que os caracteres não devem ser visualizado.

\begin{itemize}
\item \textbf{NOTA} \begin{itemize}
\item Usados no campos tipo senha.
\end{itemize}
\end{itemize}

\end{list}
\subsection*{CharShowPasswordChar}
\begin{list}{}{
\settowidth{\tmplength}{\textbf{Declaração}}
\setlength{\itemindent}{0cm}
\setlength{\listparindent}{0cm}
\setlength{\leftmargin}{\evensidemargin}
\addtolength{\leftmargin}{\tmplength}
\settowidth{\labelsep}{X}
\addtolength{\leftmargin}{\labelsep}
\setlength{\labelwidth}{\tmplength}
}
\begin{flushleft}
\item[\textbf{Declaração}\hfill]
\begin{ttfamily}
CharShowPasswordChar      = TConsts.CharShowPasswordChar   ;\end{ttfamily}


\end{flushleft}
\par
\item[\textbf{Descrição}]
A constante \textbf{\begin{ttfamily}CharShowPasswordChar\end{ttfamily}} ...

\end{list}
\subsection*{TypeDate}
\begin{list}{}{
\settowidth{\tmplength}{\textbf{Declaração}}
\setlength{\itemindent}{0cm}
\setlength{\listparindent}{0cm}
\setlength{\leftmargin}{\evensidemargin}
\addtolength{\leftmargin}{\tmplength}
\settowidth{\labelsep}{X}
\addtolength{\leftmargin}{\labelsep}
\setlength{\labelwidth}{\tmplength}
}
\begin{flushleft}
\item[\textbf{Declaração}\hfill]
\begin{ttfamily}
TypeDate              = TConsts.TypeDate           ;\end{ttfamily}


\end{flushleft}
\par
\item[\textbf{Descrição}]
A constante \textbf{\begin{ttfamily}TypeDate\end{ttfamily}} ...

\end{list}
\subsection*{{\_}TypeDate}
\begin{list}{}{
\settowidth{\tmplength}{\textbf{Declaração}}
\setlength{\itemindent}{0cm}
\setlength{\listparindent}{0cm}
\setlength{\leftmargin}{\evensidemargin}
\addtolength{\leftmargin}{\tmplength}
\settowidth{\labelsep}{X}
\addtolength{\leftmargin}{\labelsep}
\setlength{\labelwidth}{\tmplength}
}
\begin{flushleft}
\item[\textbf{Declaração}\hfill]
\begin{ttfamily}
{\_}TypeDate             = TConsts.{\_}TypeDate          ;\end{ttfamily}


\end{flushleft}
\par
\item[\textbf{Descrição}]
A constante \textbf{\begin{ttfamily}{\_}TypeDate\end{ttfamily}} ...

\end{list}
\subsection*{TypeHora}
\begin{list}{}{
\settowidth{\tmplength}{\textbf{Declaração}}
\setlength{\itemindent}{0cm}
\setlength{\listparindent}{0cm}
\setlength{\leftmargin}{\evensidemargin}
\addtolength{\leftmargin}{\tmplength}
\settowidth{\labelsep}{X}
\addtolength{\leftmargin}{\labelsep}
\setlength{\labelwidth}{\tmplength}
}
\begin{flushleft}
\item[\textbf{Declaração}\hfill]
\begin{ttfamily}
TypeHora              = TConsts.TypeHora           ;\end{ttfamily}


\end{flushleft}
\par
\item[\textbf{Descrição}]
A constante \textbf{\begin{ttfamily}TypeHora\end{ttfamily}} ...

\end{list}
\subsection*{TypeMemo}
\begin{list}{}{
\settowidth{\tmplength}{\textbf{Declaração}}
\setlength{\itemindent}{0cm}
\setlength{\listparindent}{0cm}
\setlength{\leftmargin}{\evensidemargin}
\addtolength{\leftmargin}{\tmplength}
\settowidth{\labelsep}{X}
\addtolength{\leftmargin}{\labelsep}
\setlength{\labelwidth}{\tmplength}
}
\begin{flushleft}
\item[\textbf{Declaração}\hfill]
\begin{ttfamily}
TypeMemo              = TConsts.TypeMemo           ;\end{ttfamily}


\end{flushleft}
\par
\item[\textbf{Descrição}]
A constante \textbf{\begin{ttfamily}TypeMemo\end{ttfamily}} ...

\end{list}
\chapter{Unit mi{\_}rtl{\_}ui{\_}Dmxscroller{\_}sql}
\section{Descrição}
A unit \textbf{\begin{ttfamily}mi{\_}rtl{\_}ui{\_}Dmxscroller{\_}sql\end{ttfamily}} implementa a classe \begin{ttfamily}TUiDmxScroller{\_}sql\end{ttfamily}(\ref{mi_rtl_ui_Dmxscroller_sql.TUiDmxScroller_sql}).

\begin{itemize}
\item \textbf{VERSÃO} \begin{itemize}
\item Alpha {-} 0.5.0.687
\end{itemize}
\item \textbf{CÓDIGO FONTE}: \begin{itemize}
\item 
\item \textbf{PENDÊNCIAS} \begin{itemize}
\item T12 Falta implementar chave estrangeira em createTable;
\item T12 Em \begin{ttfamily}TUiDmxScroller{\_}sql.DoOnNewRecord\end{ttfamily}(\ref{mi_rtl_ui_Dmxscroller_sql.TUiDmxScroller_sql-DoOnNewRecord}); está executando o método (CustomBufDataset as TSQLQuery).Append; antes do componenente \begin{ttfamily}TUiDmxScroller{\_}sql\end{ttfamily}(\ref{mi_rtl_ui_Dmxscroller_sql.TUiDmxScroller_sql}) está visível e isto está gerando exceção.
\item T12 ANÁLISE \begin{itemize}
\item [Estudar os procedimentos armazenados](https://www.w3schools.com/sql/sql{\_}stored{\_}procedures.asp)
\item [Estudar as restrições SQL](https://www.w3schools.com/sql/sql{\_}constraints.asp)
\item Como saber se um campo é uma chave que liga outra tabela? \begin{itemize}
\item https://www.w3schools.com/sql/sql{\_}foreignkey.asp (SQL FOREIGN KEY Constraint)

\texttt{\\\nopagebreak[3]
\\\nopagebreak[3]
/*Não,~podemos~permitir~que~os~registros~das~pessoas~que~possuim~camisetas\\\nopagebreak[3]
~~lavando~sejam~apagados,~para~garantir~a~integridade~da~informação.\\\nopagebreak[3]
~~Para~isso~devemos~utilizar~o~}\textbf{as}\texttt{~chaves~estrangeiras~que~acusarão\\\nopagebreak[3]
~~um~erro~quando~tentarmos~deletar~uma~pessoa~que~possuir~camisetas.\\\nopagebreak[3]
~~Veja~em~código:\\\nopagebreak[3]
*/\\\nopagebreak[3]
\\\nopagebreak[3]
CREATE~TABLE~Pessoa(\\\nopagebreak[3]
~~~~IdPessoa~INT~}\textbf{NOT}\texttt{~NULL~PRIMARY~KEY~IDENTITY(1,1),\\\nopagebreak[3]
~~~~Nome~VARCHAR(20)~}\textbf{NOT}\texttt{~NULL\\\nopagebreak[3]
)\\\nopagebreak[3]
\\\nopagebreak[3]
CREATE~TABLE~Camiseta(\\\nopagebreak[3]
~~~~IdCamiseta~INT~}\textbf{NOT}\texttt{~NULL~PRIMARY~KEY~IDENTITY(1,1),\\\nopagebreak[3]
~~~~Descrição~VARCHAR(20)~}\textbf{NOT}\texttt{~NULL,\\\nopagebreak[3]
~~~~IdPessoa~INT~}\textbf{NOT}\texttt{~NULL\\\nopagebreak[3]
~~~~CONSTRAINT~FK{\_}Camiseta{\_}Pessoa~FOREIGN~~KEY(IdPessoa)~REFERENCES~Pessoa(IdPessoa)\\\nopagebreak[3]
)\\\nopagebreak[3]
\\\nopagebreak[3]
INSERT~INTO~Pessoa~VALUES~('HeyJoe')\\\nopagebreak[3]
INSERT~INTO~Pessoa~VALUES~('Caique')\\\nopagebreak[3]
\\\nopagebreak[3]
\\\nopagebreak[3]
INSERT~INTO~Camiseta~VALUES~('Azul',~1)\\\nopagebreak[3]
INSERT~INTO~Camiseta~VALUES~('Amarela',~1)\\\nopagebreak[3]
INSERT~INTO~Camiseta~VALUES~('Preta',~2)\\\nopagebreak[3]
\\\nopagebreak[3]
SELECT~*~FROM~Pessoa,~Camiseta~WHERE~Pessoa.IdPessoa~=~Camiseta.IdPessoa\\
}
\end{itemize}
\item Como saber o tipo de relacionamento que os campos de outra tabela tem com a tabela atual?
\end{itemize}
\item T12 A opção CreateTable está dando mensagem de erro quando a coluna já existe. \begin{itemize}
\item Encontrar uma forma de não gerar exceção ou ignorar as exceções nesta rotina.
\end{itemize}
\item T12 Em \begin{ttfamily}TUiDmxScroller{\_}sql.AlterTable\end{ttfamily}(\ref{mi_rtl_ui_Dmxscroller_sql.TUiDmxScroller_sql-AlterTable}) checar: \begin{itemize}
\item T12 Criar código para todos os tipos reconhecidos por marIcaraí.
\item T12 Debugar para saber se está tudo funcionando.
\item T12 Permitir adicionar uma nova coluna mesmo que a tabela já exista.
\end{itemize}
\item T12 Em SetTableName(aTableName:String) criticar o nome aTableName é um nome válido para a tabela.
\end{itemize}
\item \textbf{HISTÓRICO} \begin{itemize}
\item Criado por: Paulo Sérgio da Silva Pacheco paulosspacheco@yahoo.com.br) ✅
\item \textbf{2022{-}03{-}13} \begin{itemize}
\item \textbf{09:00} \begin{itemize}
\item T12 Implementar a criação de código SQL baseado nos dados de \begin{ttfamily}TUiDmxScroller\end{ttfamily}(\ref{mi_rtl_ui_Dmxscroller.TUiDmxScroller}). \begin{itemize}
\item \textbf{ANÁLISE:} \begin{itemize}
\item Onde pegar o nome da tabela ou consulta? ✅ \begin{itemize}
\item TUiDmxScroller.CustomBufDataset.FileName;
\end{itemize}
\item Onde pegar o nome dos campos da tabela CustomBufDataset.Filename? ✅ \begin{itemize}
\item A lista de campos encontra{-}se em : TUiDmxScroller{\_}Atributos.Fields : TFPList;
\end{itemize}
\item Como saber se \begin{ttfamily}TUiDmxScroller\end{ttfamily}(\ref{mi_rtl_ui_Dmxscroller.TUiDmxScroller}) é uma tabela ou a uma consulta? \begin{itemize}
\item Se todos os TUiDmxScroller{\_}Atributos.Fields[].FieldName não contém '|' é porque é FileName é uma tabela. ✅
\item Se pelo menos um TUiDmxScroller{\_}Atributos.Fields[].FieldName contém '|' é porque é FileName é uma consulta envolvendo mais de uma tabela. ✅
\end{itemize}
\item Como saber se uma tabela ou consulta existe do banco de dados? \begin{itemize}
\item O SQL do \textbf{postegres} e do \textbf{sqlite3} tem a clausula \textbf{IN NOT EXISTS} no comando CREATE TABLE: \begin{itemize}
\item EXEMPLO:

\texttt{\\\nopagebreak[3]
CREATE~TABLE~}\textbf{IF}\texttt{~}\textbf{NOT}\texttt{~EXISTS~TEST01~();~~✅\\
}
\end{itemize}
\end{itemize}
\end{itemize}
\item \textbf{REFERÊNCIAS} \begin{itemize}
\item https://en.wikipedia.org/wiki/SQL:2016 (SQL:2016) \begin{itemize}
\item (PostgresSQL aceita 160 das 169 especificação 2016)(https://www.postgresql.org/docs/12/features.html)
\item [Comparativo entre os bancos de dados x conformidade SQL](https://en.wikipedia.org/wiki/SQL{\_}compliance)
\item [Clientes de bancos de dados opensource](https://medevel.com/17-sql-client-open-source/)
\item https://dbeaver.io/ (Instalei programa cliente SQL DBeaver) \begin{itemize}
\item Obs: Não deu certo. Ele é escrito em java e não funcionou o básico.
\end{itemize}
\item [sqlite create database if not exists](https://www.codegrepper.com/code-examples/sql/sqlite+create+database+if+not+exists)
\end{itemize}
\end{itemize}
\end{itemize}
\end{itemize}
\end{itemize}
\item \textbf{2022{-}03{-}14} \begin{itemize}
\item \textbf{08:22} \begin{itemize}
\item T12 Criar a unit mi{\_}ui{\_}Dmxscroller{\_}sql.pas com a classe \textbf{\begin{ttfamily}TUiDmxScroller{\_}sql\end{ttfamily}(\ref{mi_rtl_ui_Dmxscroller_sql.TUiDmxScroller_sql})} com objetivo de concentrar a integração do TDmxScroller com o componente \textbf{TSQLQuery} ✅
\end{itemize}
\item \textbf{20:00} \begin{itemize}
\item T12 Na Construção de TFields atualizar a propriedade \textbf{TField.ProviderFlags} com o tipo de acesso definido em \begin{ttfamily}TDmxFieldRec.Access\end{ttfamily}(\ref{mi_rtl_ui_Dmxscroller.TDmxFieldRec-access}) ✅
\end{itemize}
\item \textbf{21:12} \begin{itemize}
\item T12 Criar propriedade \textbf{TableName} ✅
\end{itemize}
\item \textbf{21:27} \begin{itemize}
\item T12 Criar Function SetSqlCustomBufDataset:Boolean;Virtual; \begin{itemize}
\item CustomBufDataset.SQL := \textbf{SELECT * FROM X } onde X será definido pela propriedade \textbf{TableName} ✅
\end{itemize}
\end{itemize}
\end{itemize}
\item \textbf{2022{-}03{-}15} \begin{itemize}
\item \textbf{09:11} \begin{itemize}
\item Depurar o que fiz ontem para fazer funciona a atualização do banco de dados SQL. ✅
\end{itemize}
\item \textbf{11:36} \begin{itemize}
\item Criar método \begin{ttfamily}TUiDmxScroller{\_}sql.AlterTable\end{ttfamily}(\ref{mi_rtl_ui_Dmxscroller_sql.TUiDmxScroller_sql-AlterTable}) : Boolean; ✅
\end{itemize}
\item \textbf{14:38} \begin{itemize}
\item T12 Atualizar TSQLQury.TFields.ProviderFlags com TUiDmxScroller.MiProviderFlags ✅
\end{itemize}
\end{itemize}
\item \textbf{2022{-}03{-}16} \begin{itemize}
\item \textbf{16:23} \begin{itemize}
\item T12 Em \begin{ttfamily}TUiDmxScroller{\_}sql.CreateCustomBufDataset{\_}FieldDefs\end{ttfamily}(\ref{mi_rtl_ui_Dmxscroller_sql.TUiDmxScroller_sql-CreateCustomBufDataset_FieldDefs}), atualizar \textbf{TField.ProviderFlags} com os dados do campo \textbf{\begin{ttfamily}TDmxFieldRec.ProviderFlags\end{ttfamily}(\ref{mi_rtl_ui_Dmxscroller.TDmxFieldRec-ProviderFlags})}. ✅
\end{itemize}
\item \textbf{16:54} \begin{itemize}
\item Em \begin{ttfamily}TUiDmxScroller{\_}sql.AlterTable\end{ttfamily}(\ref{mi_rtl_ui_Dmxscroller_sql.TUiDmxScroller_sql-AlterTable}) usar os flags \begin{ttfamily}TDmxFieldRec.ProviderFlags\end{ttfamily}(\ref{mi_rtl_ui_Dmxscroller.TDmxFieldRec-ProviderFlags}) para criação da tabela. ✅
\end{itemize}
\end{itemize}
\item \textbf{2022{-}03{-}17} \begin{itemize}
\item \textbf{10:48} \begin{itemize}
\item T12 Os flags indicando que se trata de chave primária não está sendo atualizado em createStructor, por isso não está criando a chave primária. ✅
\end{itemize}
\end{itemize}
\item \textbf{2022{-}03{-}18} \begin{itemize}
\item \textbf{10:40} \begin{itemize}
\item T12 Ao criar uma tabela SQL em \textbf{AlterTable} adicionar colunas ao invés de criar a tabela toda. ✅ \begin{itemize}
\item \textbf{Motivo}: \begin{itemize}
\item Permitir que o banco de dados fique compatível com o Template.
\item Alterar um coluna de forma automática não é bom, porque o que está feito gera dependências que produzirão erros ao fazer essas alterações.
\end{itemize}
\end{itemize}
\end{itemize}
\end{itemize}
\item \textbf{2022{-}03{-}21} \begin{itemize}
\item \textbf{08:57} \begin{itemize}
\item T12 Criar function SQL{\_}AddkeysPrimaryKeyComposite(I : Integer):Boolean; ✅ \begin{itemize}
\item Esta função adiciona chave primária composta na tabela.
\item \textbf{REFERÊNCIA} \begin{itemize}
\item [Como adiconar chave primaria usando a expressão ALTER TABLE](https://www.techonthenet.com/postgresql/primary{\_}keys.php{\#}:~:text=In{\%}20PostgreSQL{\%}2C{\%}20a{\%}20primary{\%}20key{\%}20is{\%}20created{\%}20using{\%}20either{\%}20a,or{\%}20drop{\%}20a{\%}20primary{\%}20key.)
\end{itemize}
\end{itemize}
\end{itemize}
\item \textbf{15:40} \begin{itemize}
\item T12 Em AlterTable criar a restrição de chave estrangeira no TDmxScroller{\_}sql. ✅ \begin{itemize}
\item Nome da função: function AddKeyForeigns(I : Integer):Boolean;
\end{itemize}
\end{itemize}
\end{itemize}
\item \textbf{2022{-}03{-}22} \begin{itemize}
\item \textbf{09:00} \begin{itemize}
\item T12 Documentar as units \begin{ttfamily}TuiTypes\end{ttfamily}(\ref{mi_rtl_ui_types.TUiTypes}) e TUIConsts. ✅
\end{itemize}
\item \textbf{10:00} \begin{itemize}
\item T12 Criar os relacionamentos entre tabelas (restrições entre tabelas) ✅
\end{itemize}
\item \textbf{14:14} \begin{itemize}
\item T12 Depurar os relacionamentos entre tabelas. ✅
\end{itemize}
\item \textbf{18:47} \begin{itemize}
\item O Componente CustomBufDataset não está entrando no modo edit. ✅ \begin{itemize}
\item O problema estava nos eventos TScrollBoxDMX.DoOnEnter e TScrollBoxDMX.DoOnExit;]
\end{itemize}
\end{itemize}
\end{itemize}
\item \textbf{2022{-}03{-}22} \begin{itemize}
\item \textbf{20:27} \begin{itemize}
\item T12 Analisar como criar os comandos CmIncluir, cmAlterar, cmExcluir, cmConsulta para a tabela TDmxScroller \begin{itemize}
\item Criar os comandos: ✅ \begin{itemize}
\item Public Procedure DoOnNewRecord;overload;override; //Usado para inicializa os parametros de um novo registro
\item Public Procedure PutRec;Override;//Grava o buffer no arquivo memo
\item Public Procedure GetRec;Override;//O primeiro registro esta gravado em Value
\item Public Function DeleteRec:Boolean;Override;
\item Function UpdateRec: Boolean;Override;
\item Function UpdateRec{\_}if{\_}RecordAltered:Boolean;Override;
\item Function PrevRec : Boolean;overload;override;
\item Function NextRec : Boolean;overload;override;
\end{itemize}
\end{itemize}
\end{itemize}
\end{itemize}
\item \textbf{2022{-}03{-}23} \begin{itemize}
\item Criar método Public Function AddRec:Boolean;Override; ✅ \begin{itemize}
\item Para que DoAddrec possa adicionar o registro é necessário que o registro esteja selecionando, ou seja no modo edit.
\item Obs: Está com problema.
\end{itemize}
\end{itemize}
\item \textbf{2022{-}03{-}25} \begin{itemize}
\item https://wiki.freepascal.org/Firebird{\#}Creating{\_}objects{\_}programmatically (Estudar página sobre o banco de dados firebird) ✅
\end{itemize}
\item \textbf{2022{-}03{-}28} \begin{itemize}
\item Em \begin{ttfamily}TUiDmxScroller{\_}sql.DoOnNewRecord\end{ttfamily}(\ref{mi_rtl_ui_Dmxscroller_sql.TUiDmxScroller_sql-DoOnNewRecord}); está executando o método (CustomBufDataset as TSQLQuery).Append; antes do componenente \begin{ttfamily}TUiDmxScroller{\_}sql\end{ttfamily}(\ref{mi_rtl_ui_Dmxscroller_sql.TUiDmxScroller_sql}) está visível e isto está gerando exceção. {-}
\end{itemize}
\item \textbf{2022{-}03{-}30} \begin{itemize}
\item Implementar a conexão com o banco de dados usando o componente Mi{\_}Application.
\end{itemize}
\item \textbf{2022{-}04{-}14} \begin{itemize}
\item Debugar o método \textbf{\begin{ttfamily}TUiDmxScroller{\_}sql.AlterTable\end{ttfamily}(\ref{mi_rtl_ui_Dmxscroller_sql.TUiDmxScroller_sql-AlterTable})}.
\end{itemize}
\item \textbf{2022{-}04{-}15} \begin{itemize}
\item O método \textbf{\begin{ttfamily}TUiDmxScroller{\_}sql.AlterTable\end{ttfamily}(\ref{mi_rtl_ui_Dmxscroller_sql.TUiDmxScroller_sql-AlterTable})} precisa reconhecer a sintaxe do banco de dados selecionado. \begin{itemize}
\item O postgresSQL sintaxe: \begin{itemize}
\item CREATE TABLE [IF NOT EXISTS] table{\_}name ( column1 datatype(length) column{\_}contraint, column2 datatype(length) column{\_}contraint, column3 datatype(length) column{\_}contraint, table{\_}constraints );
\item \textbf{REFERÊNCIA} \begin{itemize}
\item [postgresql{-}create{-}table](https://www.postgresqltutorial.com/postgresql-tutorial/postgresql-create-table/)
\end{itemize}
\end{itemize}
\item O sqLite3 sintaxe: \begin{itemize}
\item CREATE TABLE [IF NOT EXISTS] [schema{\_}name].table{\_}name ( column{\_}1 data{\_}type PRIMARY KEY, column{\_}2 data{\_}type NOT NULL, column{\_}3 data{\_}type DEFAULT 0,table{\_}constraints) [WITHOUT ROWID];
\item \textbf{REFERÊNCIA:} \begin{itemize}
\item [sqlite{-}create{-}table](https://www.sqlitetutorial.net/sqlite-create-table/)
\end{itemize}
\end{itemize}
\end{itemize}
\end{itemize}
\end{itemize}
\end{itemize}
\end{itemize}
\section{Uses}
\begin{itemize}
\item \begin{ttfamily}Classes\end{ttfamily}\item \begin{ttfamily}SysUtils\end{ttfamily}\item \begin{ttfamily}BufDataset\end{ttfamily}\item \begin{ttfamily}db\end{ttfamily}\item \begin{ttfamily}SqlDb\end{ttfamily}\item \begin{ttfamily}mi.rtl.Types\end{ttfamily}(\ref{mi.rtl.Types})\item \begin{ttfamily}mi{\_}rtl{\_}ui{\_}types\end{ttfamily}(\ref{mi_rtl_ui_types})\item \begin{ttfamily}mi{\_}rtl{\_}ui{\_}consts\end{ttfamily}\item \begin{ttfamily}mi{\_}rtl{\_}ui{\_}Dmxscroller\end{ttfamily}(\ref{mi_rtl_ui_Dmxscroller})\item \begin{ttfamily}umi{\_}rtl{\_}ui{\_}custom{\_}application\end{ttfamily}\end{itemize}
\section{Visão Geral}
\begin{description}
\item[\texttt{\begin{ttfamily}TDmxScroller{\_}sql{\_}Atributos\end{ttfamily} Classe}]
\item[\texttt{\begin{ttfamily}TUiDmxScroller{\_}sql\end{ttfamily} Classe}]
\end{description}
\section{Classes, Interfaces, Objetos e Registros}
\subsection*{TDmxScroller{\_}sql{\_}Atributos Classe}
\subsubsection*{\large{\textbf{Hierarquia}}\normalsize\hspace{1ex}\hfill}
TDmxScroller{\_}sql{\_}Atributos {$>$} \begin{ttfamily}TUiDmxScroller\end{ttfamily}(\ref{mi_rtl_ui_Dmxscroller.TUiDmxScroller}) {$>$} \begin{ttfamily}TUiMethods\end{ttfamily}(\ref{mi_rtl_ui_methods.TUiMethods}) {$>$} 
TUiConsts
\subsubsection*{\large{\textbf{Descrição}}\normalsize\hspace{1ex}\hfill}
A class \textbf{\begin{ttfamily}TDmxScroller{\_}sql{\_}Atributos\end{ttfamily}} contém os atributos da class TDmxScroller{\_}sql\subsubsection*{\large{\textbf{Campos}}\normalsize\hspace{1ex}\hfill}
\paragraph*{CustomBufDataset}\hspace*{\fill}

\begin{list}{}{
\settowidth{\tmplength}{\textbf{Declaração}}
\setlength{\itemindent}{0cm}
\setlength{\listparindent}{0cm}
\setlength{\leftmargin}{\evensidemargin}
\addtolength{\leftmargin}{\tmplength}
\settowidth{\labelsep}{X}
\addtolength{\leftmargin}{\labelsep}
\setlength{\labelwidth}{\tmplength}
}
\begin{flushleft}
\item[\textbf{Declaração}\hfill]
\begin{ttfamily}
public CustomBufDataset: TCustomBufDataset;\end{ttfamily}


\end{flushleft}
\par
\item[\textbf{Descrição}]
O atributo pública \textbf{\begin{ttfamily}CustomBufDataset\end{ttfamily}} é definida em \textbf{CreateCustomBufDataset{\_}FieldDefs} que é executado em \textbf{TDmxScroller.CreateData} baseado na estrutura do Template passado por \begin{ttfamily}GetTemplate\end{ttfamily}(\ref{mi_rtl_ui_Dmxscroller.TUiDmxScroller-GetTemplate}).

\begin{itemize}
\item \textbf{NOTA} \begin{itemize}
\item O atributo \textbf{\begin{ttfamily}CustomBufDataset\end{ttfamily}} deve ser passado por \textbf{DataSource.DataSet}.
\item Em \textbf{CreateCustomBufDataset{\_}FieldDefs} é criado os campo da propriedade \textbf{\begin{ttfamily}CustomBufDataset\end{ttfamily}} se a propriedade (\begin{ttfamily}DataSource\end{ttfamily}(\ref{mi_rtl_ui_Dmxscroller.TUiDmxScroller-DataSource}){$<$}{$>$}nil) e (DataSource.DataSet {$<$}{$>$} nil).
\item Se a propriedade DataSource.DataSet = nil então a propriedade \textbf{\begin{ttfamily}CustomBufDataset\end{ttfamily}=nil}
\item O método \textbf{CreateCustomBufDataset{\_}FieldDefs} reconhece duas possibilidade para os descendentes de \begin{ttfamily}CustomBufDataset\end{ttfamily} quais sejam: \begin{enumerate}
\setcounter{enumi}{0} \setcounter{enumii}{0} \setcounter{enumiii}{0} \setcounter{enumiv}{0} 
\item [TBufDataset](https://www.freepascal.org/docs-html/fcl/bufdataset/tbufdataset.html)
\setcounter{enumi}{1} \setcounter{enumii}{1} \setcounter{enumiii}{1} \setcounter{enumiv}{1} 
\item https://www.freepascal.org/docs-html/fcl/sqldb/tcustomsqlquery.html (TCustomSQLQuery) \begin{itemize}
\item Preciso das propriedades de acesso a banco de dados SQL.
\item O evento \begin{ttfamily}OnGetTemplate\end{ttfamily}(\ref{mi_rtl_ui_Dmxscroller.TUiDmxScroller-onGetTemplate}) deve setar as propriedades customizadas de \textbf{TCustomSQLQuery}.
\end{itemize}
\end{enumerate}
\end{itemize}
\item \textbf{REFERẼNCIA}: \begin{itemize}
\item [tcustombufdataset](https://www.freepascal.org/daily/packages/fcl-db/bufdataset/tcustombufdataset-14.html)
\item [tcustomsqlquery](https://www.freepascal.org/docs-html/fcl/sqldb/tcustomsqlquery.html)
\item https://www.freepascal.org/docs-html/fcl/bufdataset/tcustombufdataset.html (TCustomBufDataset);
\item [TBufDataSet](https://wiki.freepascal.org/How{\_}to{\_}write{\_}in-memory{\_}database{\_}applications{\_}in{\_}Lazarus/FPC{\#}TBufDataSet)
\item [tstatementtype.html](https://www.freepascal.org/docs-html/fcl/sqltypes/tstatementtype.html)
\item [tsqlquery](https://www.freepascal.org/docs-html/fcl/sqldb/tsqlquery.html)
\item [tdatasetstate](https://www.freepascal.org/docs-html/fcl/db/tdatasetstate.html)
\item [How{\_}to{\_}connect{\_}to{\_}a{\_}database{\_}server](https://wiki.freepascal.org/SqlDBHowto{\#}How{\_}to{\_}connect{\_}to{\_}a{\_}database{\_}server.3F)
\item [Example:{\_}reading{\_}data{\_}from{\_}a{\_}table](https://wiki.freepascal.org/SqlDBHowto{\#}Example:{\_}reading{\_}data{\_}from{\_}a{\_}table)
\item [How{\_}to{\_}execute{\_}direct{\_}queries.2Fmake{\_}a{\_}table](https://wiki.freepascal.org/SqlDBHowto{\#}How{\_}to{\_}execute{\_}direct{\_}queries.2Fmake{\_}a{\_}table.3F)
\item [How{\_}to{\_}read{\_}data{\_}from{\_}a{\_}table](https://wiki.freepascal.org/SqlDBHowto{\#}How{\_}to{\_}read{\_}data{\_}from{\_}a{\_}table.3F)
\item [Why{\_}does{\_}TSQLQuery.RecordCount{\_}always{\_}return](https://wiki.freepascal.org/SqlDBHowto{\#}Why{\_}does{\_}TSQLQuery.RecordCount{\_}always{\_}return{\_}10.3F)
\item [Como usar SQLDb no Lazarus](https://wiki.freepascal.org/SqlDBHowto{\#}Lazarus)
\item [Trabalhando com tabelas relacionadas](https://wiki.freepascal.org/MasterDetail)
\item [How{\_}to{\_}change{\_}data{\_}in{\_}a{\_}table](https://wiki.freepascal.org/SqlDBHowto{\#}How{\_}to{\_}change{\_}data{\_}in{\_}a{\_}table.3F)
\item [How{\_}does{\_}SqlDB{\_}send{\_}the{\_}changes{\_}to{\_}the{\_}database{\_}server](https://wiki.freepascal.org/SqlDBHowto{\#}How{\_}does{\_}SqlDB{\_}send{\_}the{\_}changes{\_}to{\_}the{\_}database{\_}server.3F)
\item [How{\_}to{\_}handle{\_}Errors](https://wiki.freepascal.org/SqlDBHowto{\#}How{\_}to{\_}handle{\_}Errors)
\item [How{\_}to{\_}execute{\_}a{\_}query{\_}using{\_}TSQLQuery](https://wiki.freepascal.org/SqlDBHowto{\#}How{\_}to{\_}execute{\_}a{\_}query{\_}using{\_}TSQLQuery.3F)
\item [How{\_}to{\_}use{\_}parameters{\_}in{\_}a{\_}query](https://wiki.freepascal.org/SqlDBHowto{\#}How{\_}to{\_}use{\_}parameters{\_}in{\_}a{\_}query.3F)
\item [Select{\_}query](https://wiki.freepascal.org/SqlDBHowto{\#}Select{\_}query)
\item [Exemplo de SQLQuery com parãmetros](https://wiki.freepascal.org/SqlDBHowto{\#}Example)
\item https://wiki.freepascal.org/SqlDBHowto{\#}Troubleshooting:{\_}TSQLConnection{\_}logging (Troubleshooting:{\_}TSQLConnection{\_}logging) \begin{itemize}
\item [Exemplo de log](https://wiki.freepascal.org/SqlDBHowto{\#}FPC{\_}.28or:{\_}the{\_}manual{\_}way.29)
\end{itemize}
\end{itemize}
\end{itemize}

\end{list}
\subsection*{TUiDmxScroller{\_}sql Classe}
\subsubsection*{\large{\textbf{Hierarquia}}\normalsize\hspace{1ex}\hfill}
TUiDmxScroller{\_}sql {$>$} \begin{ttfamily}TDmxScroller{\_}sql{\_}Atributos\end{ttfamily}(\ref{mi_rtl_ui_Dmxscroller_sql.TDmxScroller_sql_Atributos}) {$>$} \begin{ttfamily}TUiDmxScroller\end{ttfamily}(\ref{mi_rtl_ui_Dmxscroller.TUiDmxScroller}) {$>$} \begin{ttfamily}TUiMethods\end{ttfamily}(\ref{mi_rtl_ui_methods.TUiMethods}) {$>$} 
TUiConsts
\subsubsection*{\large{\textbf{Descrição}}\normalsize\hspace{1ex}\hfill}
A classe \textbf{\begin{ttfamily}TUiDmxScroller{\_}sql\end{ttfamily}} implementa o acesso ao banco de dados usando o atributo \textbf{\begin{ttfamily}CustomBufDataset\end{ttfamily}(\ref{mi_rtl_ui_Dmxscroller_sql.TDmxScroller_sql_Atributos-CustomBufDataset})}

\begin{itemize}
\item \textbf{NOTA} \begin{itemize}
\item O atributo \textbf{\begin{ttfamily}CustomBufDataset\end{ttfamily}(\ref{mi_rtl_ui_Dmxscroller_sql.TDmxScroller_sql_Atributos-CustomBufDataset})} pode ser \textbf{TBufDataset} não conectado a banco de dados sql e \textbf{TCustomSQLQuery} conectado ao banco de dados SQL.
\end{itemize}
\item \textbf{REFERÊNCIA} \begin{itemize}
\item [Working{\_}With{\_}TSQLQuery](https://wiki.freepascal.org/Working{\_}With{\_}TSQLQuery)
\item [Parameters{\_}in{\_}TSQLQuery](https://wiki.freepascal.org/Working{\_}With{\_}TSQLQuery{\#}Parameters{\_}in{\_}TSQLQuery.SQL)
\item [sql{-}basico](https://www.devmedia.com.br/sql-basico/28877)
\end{itemize}
\end{itemize}\subsubsection*{\large{\textbf{Propriedades}}\normalsize\hspace{1ex}\hfill}
\paragraph*{DataSource}\hspace*{\fill}

\begin{list}{}{
\settowidth{\tmplength}{\textbf{Declaração}}
\setlength{\itemindent}{0cm}
\setlength{\listparindent}{0cm}
\setlength{\leftmargin}{\evensidemargin}
\addtolength{\leftmargin}{\tmplength}
\settowidth{\labelsep}{X}
\addtolength{\leftmargin}{\labelsep}
\setlength{\labelwidth}{\tmplength}
}
\begin{flushleft}
\item[\textbf{Declaração}\hfill]
\begin{ttfamily}
published property DataSource : TDataSource Read {\_}DataSource   Write  {\_}DataSource;\end{ttfamily}


\end{flushleft}
\par
\item[\textbf{Descrição}]
A propriedade \textbf{\begin{ttfamily}DataSource\end{ttfamily}} permite que controles da \textbf{LCL} (Lazarus Componenents Library) possam usar os dados do componenente \textbf{TDmxScroller}.

\begin{itemize}
\item \textbf{NOTA} \begin{itemize}
\item Essa integração permite que \textbf{TDmxScroller} utilize todos os componentes de banco de dados do Free Pascal.
\end{itemize}
\end{itemize}

\end{list}
\subsubsection*{\large{\textbf{Campos}}\normalsize\hspace{1ex}\hfill}
\paragraph*{{\_}DataSource}\hspace*{\fill}

\begin{list}{}{
\settowidth{\tmplength}{\textbf{Declaração}}
\setlength{\itemindent}{0cm}
\setlength{\listparindent}{0cm}
\setlength{\leftmargin}{\evensidemargin}
\addtolength{\leftmargin}{\tmplength}
\settowidth{\labelsep}{X}
\addtolength{\leftmargin}{\labelsep}
\setlength{\labelwidth}{\tmplength}
}
\begin{flushleft}
\item[\textbf{Declaração}\hfill]
\begin{ttfamily}
protected {\_}DataSource: TDataSource;\end{ttfamily}


\end{flushleft}
\end{list}
\subsubsection*{\large{\textbf{Métodos}}\normalsize\hspace{1ex}\hfill}
\paragraph*{SetDataBase}\hspace*{\fill}

\begin{list}{}{
\settowidth{\tmplength}{\textbf{Declaração}}
\setlength{\itemindent}{0cm}
\setlength{\listparindent}{0cm}
\setlength{\leftmargin}{\evensidemargin}
\addtolength{\leftmargin}{\tmplength}
\settowidth{\labelsep}{X}
\addtolength{\leftmargin}{\labelsep}
\setlength{\labelwidth}{\tmplength}
}
\begin{flushleft}
\item[\textbf{Declaração}\hfill]
\begin{ttfamily}
protected procedure SetDataBase;\end{ttfamily}


\end{flushleft}
\end{list}
\paragraph*{Create}\hspace*{\fill}

\begin{list}{}{
\settowidth{\tmplength}{\textbf{Declaração}}
\setlength{\itemindent}{0cm}
\setlength{\listparindent}{0cm}
\setlength{\leftmargin}{\evensidemargin}
\addtolength{\leftmargin}{\tmplength}
\settowidth{\labelsep}{X}
\addtolength{\leftmargin}{\labelsep}
\setlength{\labelwidth}{\tmplength}
}
\begin{flushleft}
\item[\textbf{Declaração}\hfill]
\begin{ttfamily}
public constructor Create(aOwner:TComponent); Override;\end{ttfamily}


\end{flushleft}
\par
\item[\textbf{Descrição}]
Constrói o componente

\end{list}
\paragraph*{GetkeysPrimaryComposite}\hspace*{\fill}

\begin{list}{}{
\settowidth{\tmplength}{\textbf{Declaração}}
\setlength{\itemindent}{0cm}
\setlength{\listparindent}{0cm}
\setlength{\leftmargin}{\evensidemargin}
\addtolength{\leftmargin}{\tmplength}
\settowidth{\labelsep}{X}
\addtolength{\leftmargin}{\labelsep}
\setlength{\labelwidth}{\tmplength}
}
\begin{flushleft}
\item[\textbf{Declaração}\hfill]
\begin{ttfamily}
public function GetkeysPrimaryComposite(I : Integer):AnsiString;\end{ttfamily}


\end{flushleft}
\par
\item[\textbf{Descrição}]
O método \textbf{\begin{ttfamily}GetkeysPrimaryComposite\end{ttfamily}} retorna a lista de campos pertencentes a chave composta primária.

\end{list}
\paragraph*{GetKeysPrimary}\hspace*{\fill}

\begin{list}{}{
\settowidth{\tmplength}{\textbf{Declaração}}
\setlength{\itemindent}{0cm}
\setlength{\listparindent}{0cm}
\setlength{\leftmargin}{\evensidemargin}
\addtolength{\leftmargin}{\tmplength}
\settowidth{\labelsep}{X}
\addtolength{\leftmargin}{\labelsep}
\setlength{\labelwidth}{\tmplength}
}
\begin{flushleft}
\item[\textbf{Declaração}\hfill]
\begin{ttfamily}
public function GetKeysPrimary:AnsiString;\end{ttfamily}


\end{flushleft}
\par
\item[\textbf{Descrição}]
A função \textbf{\begin{ttfamily}GetKeysPrimary\end{ttfamily}} retorna a chave primária composta ou não na tabela.

\begin{itemize}
\item \textbf{Como TSQLQuery trata os campos de chave primária} \begin{itemize}
\item Ao atualizar registros, TSQLQuery precisa saber quais campos compõem a chave primária que pode ser usada para atualizar o registro e quais campos devem ser atualizados: com base nessas informações, ele constrói um comando SQL UPDATE, INSERT ou DELETE.
\item A construção da instrução SQL é controlada pela propriedade UsePrimaryKeyAsKey e pelas propriedades ProviderFlags .
\item A propriedade Providerflags é um conjunto de 3 sinalizadores: \begin{itemize}
\item pfInkey : O campo faz parte da chave primária
\item pfInWhere : O campo deve ser utilizado na cláusula WHERE das instruções SQL.
\item pfInUpdate : Atualizações ou inserções devem incluir este campo. Por padrão, ProviderFlags consiste apenas em pfInUpdate .
\item \textbf{NOTA* \begin{itemize}
\item Se sua tabela tiver uma chave primária (conforme descrito acima), você só precisará definir a propriedade **UsePrimaryKeyAsKey
\end{itemize}} como True e tudo será feito para você. Isso definirá o sinalizador pfInKey para os campos de chave primária.
\end{itemize}
\end{itemize}
\item \textbf{REFERÊNCIA} \begin{itemize}
\item [Working With TSQLQuery e Primary{\_}key{\_}Fields](https://wiki.freepascal.org/Working{\_}With{\_}TSQLQuery{\#}Primary{\_}key{\_}Fields)
\end{itemize}
\end{itemize}

\end{list}
\paragraph*{CreateTable}\hspace*{\fill}

\begin{list}{}{
\settowidth{\tmplength}{\textbf{Declaração}}
\setlength{\itemindent}{0cm}
\setlength{\listparindent}{0cm}
\setlength{\leftmargin}{\evensidemargin}
\addtolength{\leftmargin}{\tmplength}
\settowidth{\labelsep}{X}
\addtolength{\leftmargin}{\labelsep}
\setlength{\labelwidth}{\tmplength}
}
\begin{flushleft}
\item[\textbf{Declaração}\hfill]
\begin{ttfamily}
public Function CreateTable: Boolean;\end{ttfamily}


\end{flushleft}
\par
\item[\textbf{Descrição}]
A função \textbf{\begin{ttfamily}CreateTable\end{ttfamily}} cria a tabela se a mesma não existir

\end{list}
\paragraph*{AlterTable}\hspace*{\fill}

\begin{list}{}{
\settowidth{\tmplength}{\textbf{Declaração}}
\setlength{\itemindent}{0cm}
\setlength{\listparindent}{0cm}
\setlength{\leftmargin}{\evensidemargin}
\addtolength{\leftmargin}{\tmplength}
\settowidth{\labelsep}{X}
\addtolength{\leftmargin}{\labelsep}
\setlength{\labelwidth}{\tmplength}
}
\begin{flushleft}
\item[\textbf{Declaração}\hfill]
\begin{ttfamily}
public Function AlterTable: Boolean; Virtual;\end{ttfamily}


\end{flushleft}
\par
\item[\textbf{Descrição}]
O método \textbf{\begin{ttfamily}AlterTable\end{ttfamily}} cria a tabela ou consulta \textbf{\begin{ttfamily}TableName\end{ttfamily}(\ref{mi_rtl_ui_Dmxscroller.TUiDmxScroller-TableName})} no banco de dados caso a propriedade \textbf{\begin{ttfamily}TableName\end{ttfamily}(\ref{mi_rtl_ui_Dmxscroller.TUiDmxScroller-TableName})} não existe no banco de dados e \textbf{\begin{ttfamily}TableName\end{ttfamily}(\ref{mi_rtl_ui_Dmxscroller.TUiDmxScroller-TableName})} seja diferente de vazio.

\begin{itemize}
\item O método \textbf{\begin{ttfamily}TUiDmxScroller{\_}sql.AlterTable\end{ttfamily}} precisa reconhecer a sintaxe do banco de dados selecionado. \begin{itemize}
\item O postgresSQL sintaxe: \begin{itemize}
\item \begin{ttfamily}CREATE\end{ttfamily}(\ref{mi_rtl_ui_Dmxscroller_sql.TUiDmxScroller_sql-Create}) TABLE [IF NOT EXISTS] table{\_}name ( column1 datatype(length) column{\_}contraint, column2 datatype(length) column{\_}contraint, column3 datatype(length) column{\_}contraint, table{\_}constraints );
\item \textbf{REFERÊNCIA} \begin{itemize}
\item [postgresql{-}\begin{ttfamily}create\end{ttfamily}(\ref{mi_rtl_ui_Dmxscroller_sql.TUiDmxScroller_sql-Create}){-}table](https://www.postgresqltutorial.com/postgresql-tutorial/postgresql-create-table/)
\end{itemize}
\end{itemize}
\item O sqLite3 sintaxe: \begin{itemize}
\item \begin{ttfamily}CREATE\end{ttfamily}(\ref{mi_rtl_ui_Dmxscroller_sql.TUiDmxScroller_sql-Create}) TABLE [IF NOT EXISTS] [schema{\_}name].table{\_}name ( column{\_}1 data{\_}type PRIMARY KEY, column{\_}2 data{\_}type NOT NULL, column{\_}3 data{\_}type DEFAULT 0,table{\_}constraints) [WITHOUT ROWID];
\item \textbf{REFERÊNCIA:} \begin{itemize}
\item [lang{\_}createtable.html](https://www.sqlite.org/lang{\_}createtable.html)
\item [sqlite{-}\begin{ttfamily}create\end{ttfamily}(\ref{mi_rtl_ui_Dmxscroller_sql.TUiDmxScroller_sql-Create}){-}table](https://www.sqlitetutorial.net/sqlite-create-table/)
\item [lang{\_}createtable.html](https://www.sqlite.org/lang{\_}createtable.html)
\end{itemize}
\end{itemize}
\end{itemize}
\item \textbf{NOTAS} \begin{itemize}
\item As tabelas só são criadas automaticamente caso a constante AlterTableQL = true.
\item Ao adiciona uma coluna que já exista no banco de dados o sistema trata a exceção e tenta adicionar a próxima coluna. Motivo: Poder expandir a tabela dinâmicamente.
\item O comportamento do Banco de dados SqLite ao criar uma tabela é diferente do postgres. \begin{itemize}
\item O sqLite não permite criar tabela vazia.
\end{itemize}
\end{itemize}
\end{itemize}

\end{list}
\paragraph*{SetSqlCustomBufDataset}\hspace*{\fill}

\begin{list}{}{
\settowidth{\tmplength}{\textbf{Declaração}}
\setlength{\itemindent}{0cm}
\setlength{\listparindent}{0cm}
\setlength{\leftmargin}{\evensidemargin}
\addtolength{\leftmargin}{\tmplength}
\settowidth{\labelsep}{X}
\addtolength{\leftmargin}{\labelsep}
\setlength{\labelwidth}{\tmplength}
}
\begin{flushleft}
\item[\textbf{Declaração}\hfill]
\begin{ttfamily}
public Function SetSqlCustomBufDataset:Boolean; Virtual;\end{ttfamily}


\end{flushleft}
\par
\item[\textbf{Descrição}]
O método \textbf{\begin{ttfamily}SetSqlCustomBufDataset\end{ttfamily}} inicializa as propriedades SQLs de \begin{ttfamily}CustomBufDataset\end{ttfamily}(\ref{mi_rtl_ui_Dmxscroller_sql.TDmxScroller_sql_Atributos-CustomBufDataset})

\begin{itemize}
\item \textbf{PROPRIEDADES OBRIGATÓRIAS SEREM INICIALIZADAS:} \begin{itemize}
\item CustomBufDataset.SQL;
\end{itemize}
\item \textbf{PROPRIEDADES OPCIONAIS SEREM INICIALIZADAS:} \begin{itemize}
\item CustomBufDataset.InsertSQL;
\item CustomBufDataset.UpdataSQL;
\item CustomBufDataset.DeleteSQL;
\item CustomBufDataset.RefreshSQL;
\end{itemize}
\item \textbf{GERAÇÃO AUTOMÁTICA DE INSTRUÇÃO SQL DE ATUALIZAÇÃO} \begin{itemize}
\item O \textbf{SqlDb} (mais em particular, \textbf{TSQLQuery} ) pode gerar automaticamente instruções de atualização para os dados que busca. Para isso, ele irá varrer a instrução propriedade \textbf{CustomBufDataset.SQL} e determinar a tabela principal na consulta: esta é a \textbf{primeira tabela} encontrada na parte \textbf{FROM} da instrução \textbf{SELECT} . \begin{itemize}
\item Exemplo:

\texttt{\\\nopagebreak[3]
\\\nopagebreak[3]
SELECT~*~FROM~ALUNOS\\
}

\begin{itemize}
\item Alunos será a tabela selecionada para uso dos campos de https://www.freepascal.org/docs-html/fcl/db/tField.html (TField).
\end{itemize}
\end{itemize}
\item Para operações \textbf{INSERT} e \textbf{UPDATE}, a propriedade instrução SQL gerada inserirá e atualizará todos os campos que possuim \textbf{pfInUpdate} em sua propriedade \textbf{TField.ProviderFlags}. \begin{itemize}
\item Os campos somente leitura não serão adicionados à instrução SQL.
\item Os campos que são NULL não serão adicionados a uma consulta de inserção, o que significa que o servidor de banco de dados inserirá o que estiver na cláusula DEFAULT da definição de campo correspondente.
\end{itemize}
\item O campos de chave primária \begin{itemize}
\item Ao atualizar registros, \textbf{TSQLQuery} precisa saber quais campos compõem a chave primária que pode ser usada para atualizar o registro e quais campos devem ser atualizados: com base nessas informações, ele constrói os comandos \textbf{SQL UPDATE, INSERT ou DELETE}.
\item A construção da instrução \textbf{SQL} é controlada pela propriedade \textbf{UsePrimaryKeyAsKey} e pelas propriedades \textbf{ProviderFlags}.
\item A propriedade TField.ProviderFlag é um conjunto de 6 sinalizadores: \begin{itemize}
\item \textbf{pfInUpdate} : As alterações no campo devem ser propagadas para o banco de dados..
\item \textbf{pfInWhere} : O campo deve ser usado na cláusula WHERE de uma instrução de atualização no caso de upWhereChanged.
\item \textbf{pfInKey} : Campo é um campo chave e usado na cláusula WHERE de uma instrução de atualização.
\item \textbf{pfHidden} : O valor deste campo deve ser atualizado após a inserção.
\item \textbf{pfRefreshOnInsert} : O valor deste campo deve ser atualizado após a inserção.
\item \textbf{pfRefreshOnUpdate} : O valor deste campo deve ser atualizado após a atualização.
\end{itemize}
\end{itemize}
\end{itemize}
\item \textbf{REFERẼNCIAS} \begin{itemize}
\item [TSQLQuery Introdução](https://wiki.freepascal.org/Working{\_}With{\_}TSQLQuery{\#}General)
\item [TSQLQuery exemplos](https://wiki.freepascal.org/TSQLQuery)
\item https://www.freepascal.org/docs-html/fcl/sqldb/tsqlquery.html
\item https://wiki.freepascal.org/Working{\_}With{\_}TSQLQuery (Trabalhando com TSQLQuery);
\item https://www.freepascal.org/docs-html/fcl/sqldb/updatesqls.html (updatesqls.html);
\end{itemize}
\end{itemize}

\end{list}
\paragraph*{CreateCustomBufDataset{\_}FieldDefs}\hspace*{\fill}

\begin{list}{}{
\settowidth{\tmplength}{\textbf{Declaração}}
\setlength{\itemindent}{0cm}
\setlength{\listparindent}{0cm}
\setlength{\leftmargin}{\evensidemargin}
\addtolength{\leftmargin}{\tmplength}
\settowidth{\labelsep}{X}
\addtolength{\leftmargin}{\labelsep}
\setlength{\labelwidth}{\tmplength}
}
\begin{flushleft}
\item[\textbf{Declaração}\hfill]
\begin{ttfamily}
public Procedure CreateCustomBufDataset{\_}FieldDefs; override;\end{ttfamily}


\end{flushleft}
\par
\item[\textbf{Descrição}]
O método \textbf{\begin{ttfamily}CreateCustomBufDataset{\_}FieldDefs\end{ttfamily}} é usado para criar os campos de \textbf{\begin{ttfamily}CustomBufDataset\end{ttfamily}(\ref{mi_rtl_ui_Dmxscroller_sql.TDmxScroller_sql_Atributos-CustomBufDataset})}

\end{list}
\paragraph*{GetTemplate}\hspace*{\fill}

\begin{list}{}{
\settowidth{\tmplength}{\textbf{Declaração}}
\setlength{\itemindent}{0cm}
\setlength{\listparindent}{0cm}
\setlength{\leftmargin}{\evensidemargin}
\addtolength{\leftmargin}{\tmplength}
\settowidth{\labelsep}{X}
\addtolength{\leftmargin}{\labelsep}
\setlength{\labelwidth}{\tmplength}
}
\begin{flushleft}
\item[\textbf{Declaração}\hfill]
\begin{ttfamily}
public function GetTemplate(aNext: PSItem) : PSItem; overload; override;\end{ttfamily}


\end{flushleft}
\par
\item[\textbf{Descrição}]
O método \textbf{\begin{ttfamily}GetTemplate\end{ttfamily}} retorna uma lista de \textbf{\begin{ttfamily}PSItem\end{ttfamily}(\ref{mi_rtl_ui_Dmxscroller-PSItem})} (Lista de \begin{ttfamily}strings\end{ttfamily}(\ref{mi_rtl_ui_Dmxscroller.TUiDmxScroller-Strings})) com o modelo usado para criar a tela.

\begin{itemize}
\item \textbf{NOTA} \begin{itemize}
\item O Evento \begin{ttfamily}onGetTemplate\end{ttfamily}(\ref{mi_rtl_ui_Dmxscroller.TUiDmxScroller-onGetTemplate}) só é iniciado em tempo de execução, por isso o formulário não pode ser criado em tempo de desenho do aplicativo.
\item Caso o evento \begin{ttfamily}onGetTemplate\end{ttfamily}(\ref{mi_rtl_ui_Dmxscroller.TUiDmxScroller-onGetTemplate}) seja nil, então não posso ativar a tela.
\item Esse método pode ser anulado, caso se queira ignorar o evento \begin{ttfamily}onGetTemplate\end{ttfamily}(\ref{mi_rtl_ui_Dmxscroller.TUiDmxScroller-onGetTemplate}) e definir o Template em uma método pai herdado desta classe.
\end{itemize}
\end{itemize}

\end{list}
\paragraph*{GetBuffers}\hspace*{\fill}

\begin{list}{}{
\settowidth{\tmplength}{\textbf{Declaração}}
\setlength{\itemindent}{0cm}
\setlength{\listparindent}{0cm}
\setlength{\leftmargin}{\evensidemargin}
\addtolength{\leftmargin}{\tmplength}
\settowidth{\labelsep}{X}
\addtolength{\leftmargin}{\labelsep}
\setlength{\labelwidth}{\tmplength}
}
\begin{flushleft}
\item[\textbf{Declaração}\hfill]
\begin{ttfamily}
public function GetBuffers:Boolean; Override;\end{ttfamily}


\end{flushleft}
\par
\item[\textbf{Descrição}]
O método \textbf{\begin{ttfamily}GetBuffers\end{ttfamily}} ler o buffer dos campos dos arquivos associados a classe \textbf{\begin{ttfamily}TUiDmxScroller{\_}sql\end{ttfamily}(\ref{mi_rtl_ui_Dmxscroller_sql.TUiDmxScroller_sql})} para o buffer dos campos da classe \textbf{\begin{ttfamily}TUiDmxScroller\end{ttfamily}(\ref{mi_rtl_ui_Dmxscroller.TUiDmxScroller})}

\end{list}
\paragraph*{PutBuffers}\hspace*{\fill}

\begin{list}{}{
\settowidth{\tmplength}{\textbf{Declaração}}
\setlength{\itemindent}{0cm}
\setlength{\listparindent}{0cm}
\setlength{\leftmargin}{\evensidemargin}
\addtolength{\leftmargin}{\tmplength}
\settowidth{\labelsep}{X}
\addtolength{\leftmargin}{\labelsep}
\setlength{\labelwidth}{\tmplength}
}
\begin{flushleft}
\item[\textbf{Declaração}\hfill]
\begin{ttfamily}
public function PutBuffers:Boolean; override;\end{ttfamily}


\end{flushleft}
\par
\item[\textbf{Descrição}]
O método \textbf{\begin{ttfamily}PutBuffers\end{ttfamily}} grava o buffer dos campos da classe \textbf{\begin{ttfamily}TUiDmxScroller{\_}sql\end{ttfamily}(\ref{mi_rtl_ui_Dmxscroller_sql.TUiDmxScroller_sql})} para o buffer dos campos dos arquivos associados a classe \textbf{\begin{ttfamily}TUiDmxScroller{\_}sql\end{ttfamily}(\ref{mi_rtl_ui_Dmxscroller_sql.TUiDmxScroller_sql})}

\end{list}
\paragraph*{SetActiveLCL}\hspace*{\fill}

\begin{list}{}{
\settowidth{\tmplength}{\textbf{Declaração}}
\setlength{\itemindent}{0cm}
\setlength{\listparindent}{0cm}
\setlength{\leftmargin}{\evensidemargin}
\addtolength{\leftmargin}{\tmplength}
\settowidth{\labelsep}{X}
\addtolength{\leftmargin}{\labelsep}
\setlength{\labelwidth}{\tmplength}
}
\begin{flushleft}
\item[\textbf{Declaração}\hfill]
\begin{ttfamily}
public procedure SetActiveLCL(aActive: Boolean); override;\end{ttfamily}


\end{flushleft}
\end{list}
\paragraph*{DoOnNewRecord}\hspace*{\fill}

\begin{list}{}{
\settowidth{\tmplength}{\textbf{Declaração}}
\setlength{\itemindent}{0cm}
\setlength{\listparindent}{0cm}
\setlength{\leftmargin}{\evensidemargin}
\addtolength{\leftmargin}{\tmplength}
\settowidth{\labelsep}{X}
\addtolength{\leftmargin}{\labelsep}
\setlength{\labelwidth}{\tmplength}
}
\begin{flushleft}
\item[\textbf{Declaração}\hfill]
\begin{ttfamily}
public Procedure DoOnNewRecord; Override;\end{ttfamily}


\end{flushleft}
\par
\item[\textbf{Descrição}]
O método \textbf{\begin{ttfamily}DoOnNewRecord\end{ttfamily}} seleciona o registro para adição de um novo registro \begin{itemize}
\item NOTA \begin{itemize}
\item Está gerando exceção.?????
\end{itemize}
\end{itemize}

\end{list}
\paragraph*{DoAddRec}\hspace*{\fill}

\begin{list}{}{
\settowidth{\tmplength}{\textbf{Declaração}}
\setlength{\itemindent}{0cm}
\setlength{\listparindent}{0cm}
\setlength{\leftmargin}{\evensidemargin}
\addtolength{\leftmargin}{\tmplength}
\settowidth{\labelsep}{X}
\addtolength{\leftmargin}{\labelsep}
\setlength{\labelwidth}{\tmplength}
}
\begin{flushleft}
\item[\textbf{Declaração}\hfill]
\begin{ttfamily}
public Function DoAddRec:Boolean; override;\end{ttfamily}


\end{flushleft}
\par
\item[\textbf{Descrição}]
O método \textbf{\begin{ttfamily}DoAddRec\end{ttfamily}} adicione o registro editado no banco de dados. = \textbf{OBSERVAÇÂO} \begin{itemize}
\item O método \textbf{\begin{ttfamily}DoAddRec\end{ttfamily}} só funciona se o registro atender as seguintes condições: \begin{itemize}
\item \begin{ttfamily}appending\end{ttfamily}(\ref{mi_rtl_ui_Dmxscroller.TUiDmxScroller-Appending}) =true;
\item Mb{\_}St{\_}Insert habilidado
\item \begin{ttfamily}CustomBufDataset\end{ttfamily}(\ref{mi_rtl_ui_Dmxscroller_sql.TDmxScroller_sql_Atributos-CustomBufDataset}) {$<$}{$>$} nil
\item CustomBufDataset.Active = true;
\end{itemize}
\end{itemize}\begin{itemize}
\item \textbf{REFERÊNCIA} \begin{itemize}
\item [tsqlquery.options](https://www.freepascal.org/docs-html/fcl/sqldb/tsqlquery.options.html)
\end{itemize}
\end{itemize}

\end{list}
\chapter{Unit mi{\_}rtl{\_}ui{\_}interfaces}
\section{Descrição}
\begin{itemize}
\item A unit \textbf{\begin{ttfamily}mi{\_}rtl{\_}ui{\_}interfaces\end{ttfamily}} é usada para implementar as interfaces do pacote mi.ui com propopósito de permitir que se possa criar as interfaces com usuário independente do pacote gráfico instalado.

\begin{itemize}
\item \textbf{NOTA} \begin{itemize}
\item O IDE Lazarus cria automaticamente o número da interface. Tecla: \textbf{Crt+Alt+G}
\end{itemize}
\end{itemize}
\end{itemize}
\section{Tipos}
\subsection*{TEnum{\_}HelpCtx{\_}StrCurrentCommand{\_}Topic{\_}Content{\_}run}
\begin{list}{}{
\settowidth{\tmplength}{\textbf{Declaração}}
\setlength{\itemindent}{0cm}
\setlength{\listparindent}{0cm}
\setlength{\leftmargin}{\evensidemargin}
\addtolength{\leftmargin}{\tmplength}
\settowidth{\labelsep}{X}
\addtolength{\leftmargin}{\labelsep}
\setlength{\labelwidth}{\tmplength}
}
\begin{flushleft}
\item[\textbf{Declaração}\hfill]
\begin{ttfamily}
TEnum{\_}HelpCtx{\_}StrCurrentCommand{\_}Topic{\_}Content{\_}run = (...);\end{ttfamily}


\end{flushleft}
\par
\item[\textbf{Descrição}]
 \item[\textbf{Valores}]
\begin{description}
\item[\texttt{HelpCtx{\_}StrCurrentCommand{\_}Topic{\_}Content{\_}run{\_}Parameter{\_}Indefinido}]  
\item[\texttt{HelpCtx{\_}StrCurrentCommand{\_}Topic{\_}Content{\_}run{\_}Parameter{\_}indicator}]  
\item[\texttt{HelpCtx{\_}StrCurrentCommand{\_}Topic{\_}Content{\_}run{\_}Parameter{\_}File}]  
\end{description}


\end{list}
\section{Constantes}
\subsection*{IITable}
\begin{list}{}{
\settowidth{\tmplength}{\textbf{Declaração}}
\setlength{\itemindent}{0cm}
\setlength{\listparindent}{0cm}
\setlength{\leftmargin}{\evensidemargin}
\addtolength{\leftmargin}{\tmplength}
\settowidth{\labelsep}{X}
\addtolength{\leftmargin}{\labelsep}
\setlength{\labelwidth}{\tmplength}
}
\begin{flushleft}
\item[\textbf{Declaração}\hfill]
\begin{ttfamily}
IITable : TGUID = '{\{}937B4AC1-A9B5-437C-A2ED-7EFF6CEEA919{\}}';\end{ttfamily}


\end{flushleft}
\end{list}
\subsection*{IIInputText}
\begin{list}{}{
\settowidth{\tmplength}{\textbf{Declaração}}
\setlength{\itemindent}{0cm}
\setlength{\listparindent}{0cm}
\setlength{\leftmargin}{\evensidemargin}
\addtolength{\leftmargin}{\tmplength}
\settowidth{\labelsep}{X}
\addtolength{\leftmargin}{\labelsep}
\setlength{\labelwidth}{\tmplength}
}
\begin{flushleft}
\item[\textbf{Declaração}\hfill]
\begin{ttfamily}
IIInputText : TGUID = '{\{}CBEFA72F-A283-4374-AED4-8A62C05335D9{\}}';\end{ttfamily}


\end{flushleft}
\end{list}
\chapter{Unit mi{\_}rtl{\_}ui{\_}methods}
\section{Uses}
\begin{itemize}
\item \begin{ttfamily}Classes\end{ttfamily}\item \begin{ttfamily}SysUtils\end{ttfamily}\item \begin{ttfamily}db\end{ttfamily}\item \begin{ttfamily}Variants\end{ttfamily}\item \begin{ttfamily}UTF8Process\end{ttfamily}\item \begin{ttfamily}System.UITypes\end{ttfamily}\item \begin{ttfamily}mi{\_}rtl{\_}ui{\_}consts\end{ttfamily}\end{itemize}
\section{Visão Geral}
\begin{description}
\item[\texttt{\begin{ttfamily}TUiMethods\end{ttfamily} Classe}]
\end{description}
\section{Classes, Interfaces, Objetos e Registros}
\subsection*{TUiMethods Classe}
\subsubsection*{\large{\textbf{Hierarquia}}\normalsize\hspace{1ex}\hfill}
TUiMethods {$>$} TUiConsts
%%%%Descrição
\subsubsection*{\large{\textbf{Métodos}}\normalsize\hspace{1ex}\hfill}
\paragraph*{CreateAppendFields}\hspace*{\fill}

\begin{list}{}{
\settowidth{\tmplength}{\textbf{Declaração}}
\setlength{\itemindent}{0cm}
\setlength{\listparindent}{0cm}
\setlength{\leftmargin}{\evensidemargin}
\addtolength{\leftmargin}{\tmplength}
\settowidth{\labelsep}{X}
\addtolength{\leftmargin}{\labelsep}
\setlength{\labelwidth}{\tmplength}
}
\begin{flushleft}
\item[\textbf{Declaração}\hfill]
\begin{ttfamily}
public class function CreateAppendFields(ATemplate: ptString) : DmxIDstr;\end{ttfamily}


\end{flushleft}
\par
\item[\textbf{Descrição}]
A class function \textbf{\begin{ttfamily}CreateAppendFields\end{ttfamily}} é usado para encandear Templates do tipo \begin{ttfamily}TString\end{ttfamily}(\ref{mi_rtl_ui_Dmxscroller-tString})

\end{list}
\paragraph*{CreateBlobField}\hspace*{\fill}

\begin{list}{}{
\settowidth{\tmplength}{\textbf{Declaração}}
\setlength{\itemindent}{0cm}
\setlength{\listparindent}{0cm}
\setlength{\leftmargin}{\evensidemargin}
\addtolength{\leftmargin}{\tmplength}
\settowidth{\labelsep}{X}
\addtolength{\leftmargin}{\labelsep}
\setlength{\labelwidth}{\tmplength}
}
\begin{flushleft}
\item[\textbf{Declaração}\hfill]
\begin{ttfamily}
public class function CreateBlobField(Len: integer; AccMode,Default: byte) : DmxIDstr;\end{ttfamily}


\end{flushleft}
\par
\item[\textbf{Descrição}]
A class function \textbf{\begin{ttfamily}CreateBlobField\end{ttfamily}} é usado para encandear campos do tipo blob

\end{list}
\paragraph*{CreateEnumField}\hspace*{\fill}

\begin{list}{}{
\settowidth{\tmplength}{\textbf{Declaração}}
\setlength{\itemindent}{0cm}
\setlength{\listparindent}{0cm}
\setlength{\leftmargin}{\evensidemargin}
\addtolength{\leftmargin}{\tmplength}
\settowidth{\labelsep}{X}
\addtolength{\leftmargin}{\labelsep}
\setlength{\labelwidth}{\tmplength}
}
\begin{flushleft}
\item[\textbf{Declaração}\hfill]
\begin{ttfamily}
public class function CreateEnumField(ShowZ: boolean; AccMode,Default: LongInt;AItems: PSItem) : DmxIDstr;\end{ttfamily}


\end{flushleft}
\par
\item[\textbf{Descrição}]
A class function \textbf{\begin{ttfamily}CreateEnumField\end{ttfamily}} é usado para encandear Templates do tipo enumerado

\end{list}
\paragraph*{CreateCheckBoxField}\hspace*{\fill}

\begin{list}{}{
\settowidth{\tmplength}{\textbf{Declaração}}
\setlength{\itemindent}{0cm}
\setlength{\listparindent}{0cm}
\setlength{\leftmargin}{\evensidemargin}
\addtolength{\leftmargin}{\tmplength}
\settowidth{\labelsep}{X}
\addtolength{\leftmargin}{\labelsep}
\setlength{\labelwidth}{\tmplength}
}
\begin{flushleft}
\item[\textbf{Declaração}\hfill]
\begin{ttfamily}
public class function CreateCheckBoxField(CharNumberField: AnsiChar;ShowZ: boolean; AccMode,Default: byte;AItems: PSItem) : AnsiString;\end{ttfamily}


\end{flushleft}
\par
\item[\textbf{Descrição}]
A class function \textbf{\begin{ttfamily}CreateCheckBoxField\end{ttfamily}} é usado para encandear Templates do tipo checkbox

\end{list}
\paragraph*{CreateTSItemFields}\hspace*{\fill}

\begin{list}{}{
\settowidth{\tmplength}{\textbf{Declaração}}
\setlength{\itemindent}{0cm}
\setlength{\listparindent}{0cm}
\setlength{\leftmargin}{\evensidemargin}
\addtolength{\leftmargin}{\tmplength}
\settowidth{\labelsep}{X}
\addtolength{\leftmargin}{\labelsep}
\setlength{\labelwidth}{\tmplength}
}
\begin{flushleft}
\item[\textbf{Declaração}\hfill]
\begin{ttfamily}
public class function CreateTSItemFields(ATemplates: PSItem) : DmxIDstr;\end{ttfamily}


\end{flushleft}
\par
\item[\textbf{Descrição}]
A class function \textbf{\begin{ttfamily}CreateTSItemFields\end{ttfamily}} é usado para encandear Templates do tipo \begin{ttfamily}PSItem\end{ttfamily}(\ref{mi_rtl_ui_Dmxscroller-PSItem})

\end{list}
\paragraph*{CreateOptions}\hspace*{\fill}

\begin{list}{}{
\settowidth{\tmplength}{\textbf{Declaração}}
\setlength{\itemindent}{0cm}
\setlength{\listparindent}{0cm}
\setlength{\leftmargin}{\evensidemargin}
\addtolength{\leftmargin}{\tmplength}
\settowidth{\labelsep}{X}
\addtolength{\leftmargin}{\labelsep}
\setlength{\labelwidth}{\tmplength}
}
\begin{flushleft}
\item[\textbf{Declaração}\hfill]
\begin{ttfamily}
public class function CreateOptions(Default: LongInt;AItems: PSItem) : DmxIDstr;\end{ttfamily}


\end{flushleft}
\par
\item[\textbf{Descrição}]
A class function \textbf{\begin{ttfamily}CreateOptions\end{ttfamily}} é usado para informar uma lista de opções para o campo.

\begin{itemize}
\item \textbf{NOTA} O campo que pode receber uma lista pode ser de qualquer tipo, exceto os tipos: \begin{itemize}
\item \begin{ttfamily}FldEnum\end{ttfamily}(\ref{mi_rtl_ui_dmxscroller_form-fldENUM}),\begin{ttfamily}FldBoolean\end{ttfamily}(\ref{mi_rtl_ui_dmxscroller_form-fldBoolean}) e \begin{ttfamily}FldRadioButton\end{ttfamily}(\ref{mi_rtl_ui_dmxscroller_form-FldRadioButton}).
\end{itemize}
\item \textbf{EXEMPLO DE USO}

\texttt{\\\nopagebreak[3]
\\\nopagebreak[3]
}\textbf{with}\texttt{~aUiDmxScroller~}\textbf{do}\texttt{\\\nopagebreak[3]
}\textbf{begin}\texttt{\\\nopagebreak[3]
~~add('~{\_}EXEMPLO~DE~TEMPLATE{\_}{\_}{\_}{\_}{\_}{\_}{\_}{\_}{\_}{\_}{\_}{\_}{\_}{\_}{\_}{\_}{\_}{\_}{\_}{\_}{\_}{\_}{\_}{\_}{\_}{\_}{\_}{\_}{\_}{\_}{\_}{\_}{\_}{\_}{\_}{\_}{\_}{\_}{\_}{\_}~');\\\nopagebreak[3]
~~add('');\\\nopagebreak[3]
~~add('~Vencimento:~{\textbackslash}Ssssss'+ChFN+'Vencimento'+CreateOptions(1,NewSItem('Dia~10',\\\nopagebreak[3]
~~~~~~~~~~~~~~~~~~~~~~~~~~~~~~~~~~~~~~~~~~~~~~~~~~~~~~~~~~~~~~~NewSItem('Dia~15',\\\nopagebreak[3]
~~~~~~~~~~~~~~~~~~~~~~~~~~~~~~~~~~~~~~~~~~~~~~~~~~~~~~~~~~~~~~~NewSItem('Dia~20',\\\nopagebreak[3]
~~~~~~~~~~~~~~~~~~~~~~~~~~~~~~~~~~~~~~~~~~~~~~~~~~~~~~~~~~~~~~~NewSItem('Dia~25',\\\nopagebreak[3]
~~~~~~~~~~~~~~~~~~~~~~~~~~~~~~~~~~~~~~~~~~~~~~~~~~~~~~~~~~~~~~~}\textbf{nil}\texttt{)))))+'~~dias~');\\\nopagebreak[3]
\\\nopagebreak[3]
~~add('~~~~~~Prazo:~{\textbackslash}BB'+ChFN+'Dias'+CreateOptions(2,NewSItem('30',\\\nopagebreak[3]
~~~~~~~~~~~~~~~~~~~~~~~~~~~~~~~~~~~~~~~~~~~~~~~~~~~~~NewSItem('60',\\\nopagebreak[3]
~~~~~~~~~~~~~~~~~~~~~~~~~~~~~~~~~~~~~~~~~~~~~~~~~~~~~NewSItem('90',\\\nopagebreak[3]
~~~~~~~~~~~~~~~~~~~~~~~~~~~~~~~~~~~~~~~~~~~~~~~~~~~~~NewSItem('120',\\\nopagebreak[3]
~~~~~~~~~~~~~~~~~~~~~~~~~~~~~~~~~~~~~~~~~~~~~~~~~~~~~}\textbf{nil}\texttt{)))))+'~~dias~');\\\nopagebreak[3]
\\\nopagebreak[3]
~~add('');\\\nopagebreak[3]
}\textbf{end}\texttt{;\\
}
\end{itemize}

\end{list}
\paragraph*{GetMaxTViRect}\hspace*{\fill}

\begin{list}{}{
\settowidth{\tmplength}{\textbf{Declaração}}
\setlength{\itemindent}{0cm}
\setlength{\listparindent}{0cm}
\setlength{\leftmargin}{\evensidemargin}
\addtolength{\leftmargin}{\tmplength}
\settowidth{\labelsep}{X}
\addtolength{\leftmargin}{\labelsep}
\setlength{\labelwidth}{\tmplength}
}
\begin{flushleft}
\item[\textbf{Declaração}\hfill]
\begin{ttfamily}
public class Function GetMaxTViRect: TViRect;\end{ttfamily}


\end{flushleft}
\end{list}
\paragraph*{AnsiString{\_}to{\_}TCollectionString}\hspace*{\fill}

\begin{list}{}{
\settowidth{\tmplength}{\textbf{Declaração}}
\setlength{\itemindent}{0cm}
\setlength{\listparindent}{0cm}
\setlength{\leftmargin}{\evensidemargin}
\addtolength{\leftmargin}{\tmplength}
\settowidth{\labelsep}{X}
\addtolength{\leftmargin}{\labelsep}
\setlength{\labelwidth}{\tmplength}
}
\begin{flushleft}
\item[\textbf{Declaração}\hfill]
\begin{ttfamily}
public class Function AnsiString{\_}to{\_}TCollectionString(Msg: AnsiString): TCollectionString;\end{ttfamily}


\end{flushleft}
\end{list}
\paragraph*{MsgDlgButtons{\_}To{\_}MsgDlgBtn}\hspace*{\fill}

\begin{list}{}{
\settowidth{\tmplength}{\textbf{Declaração}}
\setlength{\itemindent}{0cm}
\setlength{\listparindent}{0cm}
\setlength{\leftmargin}{\evensidemargin}
\addtolength{\leftmargin}{\tmplength}
\settowidth{\labelsep}{X}
\addtolength{\leftmargin}{\labelsep}
\setlength{\labelwidth}{\tmplength}
}
\begin{flushleft}
\item[\textbf{Declaração}\hfill]
\begin{ttfamily}
public class Function MsgDlgButtons{\_}To{\_}MsgDlgBtn(Buttons: TMI{\_}MsgBox.TMsgDlgButtons): TMI{\_}MsgBox.TArray{\_}MsgDlgBtn;\end{ttfamily}


\end{flushleft}
\end{list}
\paragraph*{FStrSelection}\hspace*{\fill}

\begin{list}{}{
\settowidth{\tmplength}{\textbf{Declaração}}
\setlength{\itemindent}{0cm}
\setlength{\listparindent}{0cm}
\setlength{\leftmargin}{\evensidemargin}
\addtolength{\leftmargin}{\tmplength}
\settowidth{\labelsep}{X}
\addtolength{\leftmargin}{\labelsep}
\setlength{\labelwidth}{\tmplength}
}
\begin{flushleft}
\item[\textbf{Declaração}\hfill]
\begin{ttfamily}
public class function FStrSelection(S:AnsiString):AnsiString;\end{ttfamily}


\end{flushleft}
\end{list}
\paragraph*{EliminaTilDeTodasAsStrings}\hspace*{\fill}

\begin{list}{}{
\settowidth{\tmplength}{\textbf{Declaração}}
\setlength{\itemindent}{0cm}
\setlength{\listparindent}{0cm}
\setlength{\leftmargin}{\evensidemargin}
\addtolength{\leftmargin}{\tmplength}
\settowidth{\labelsep}{X}
\addtolength{\leftmargin}{\labelsep}
\setlength{\labelwidth}{\tmplength}
}
\begin{flushleft}
\item[\textbf{Declaração}\hfill]
\begin{ttfamily}
public class Procedure EliminaTilDeTodasAsStrings(ATCollectionString: TCollectionString;Var aFrist{\_}Item{\_}Valid : Integer);\end{ttfamily}


\end{flushleft}
\end{list}
\paragraph*{GetModalResult}\hspace*{\fill}

\begin{list}{}{
\settowidth{\tmplength}{\textbf{Declaração}}
\setlength{\itemindent}{0cm}
\setlength{\listparindent}{0cm}
\setlength{\leftmargin}{\evensidemargin}
\addtolength{\leftmargin}{\tmplength}
\settowidth{\labelsep}{X}
\addtolength{\leftmargin}{\labelsep}
\setlength{\labelwidth}{\tmplength}
}
\begin{flushleft}
\item[\textbf{Declaração}\hfill]
\begin{ttfamily}
public class function GetModalResult(ButtonDefault: TMI{\_}MsgBox.TMsgDlgBtn):TModalResult;\end{ttfamily}


\end{flushleft}
\end{list}
\paragraph*{isValueDbChanged}\hspace*{\fill}

\begin{list}{}{
\settowidth{\tmplength}{\textbf{Declaração}}
\setlength{\itemindent}{0cm}
\setlength{\listparindent}{0cm}
\setlength{\leftmargin}{\evensidemargin}
\addtolength{\leftmargin}{\tmplength}
\settowidth{\labelsep}{X}
\addtolength{\leftmargin}{\labelsep}
\setlength{\labelwidth}{\tmplength}
}
\begin{flushleft}
\item[\textbf{Declaração}\hfill]
\begin{ttfamily}
public class function isValueDbChanged(Sender: TComponent): Boolean; virtual;\end{ttfamily}


\end{flushleft}
\par
\item[\textbf{Descrição}]
O método \textbf{\begin{ttfamily}isValueDbChanged\end{ttfamily}} verifica se o componente fornecido tem uma relação com \textbf{db} e seu conteúdo foi alterado

\end{list}
\paragraph*{FMb{\_}Bits}\hspace*{\fill}

\begin{list}{}{
\settowidth{\tmplength}{\textbf{Declaração}}
\setlength{\itemindent}{0cm}
\setlength{\listparindent}{0cm}
\setlength{\leftmargin}{\evensidemargin}
\addtolength{\leftmargin}{\tmplength}
\settowidth{\labelsep}{X}
\addtolength{\leftmargin}{\labelsep}
\setlength{\labelwidth}{\tmplength}
}
\begin{flushleft}
\item[\textbf{Declaração}\hfill]
\begin{ttfamily}
public function FMb{\_}Bits(const aBit: Byte): Longint;\end{ttfamily}


\end{flushleft}
\par
\item[\textbf{Descrição}]
O método \textbf{\begin{ttfamily}FMb{\_}Bits\end{ttfamily}} retorna o mapa de bits da posição aBit. Ou seja: a função move o bit para a esquerda aBits posição.

\begin{itemize}
\item \textbf{NOTA} \begin{itemize}
\item Como o mapa de bits possui 4 bytes este método gera exceção se aBit for maior que 32.
\end{itemize}
\item Example: \begin{itemize}
\item Command is: 00000100 shl 2 (shift left 2 bits)
\item Action is: 00000100 {$<$}{-} 00 (00 gets added to the right of the value; left 00 "disappears")
\item Result is: 00010000
\end{itemize}
\end{itemize}

\end{list}
\chapter{Unit mi{\_}rtl{\_}ui{\_}types}
\section{Descrição}
A unit \textbf{\begin{ttfamily}mi{\_}rtl{\_}ui{\_}types\end{ttfamily}} implementa a classe \begin{ttfamily}TUiTypes\end{ttfamily}(\ref{mi_rtl_ui_types.TUiTypes}).

\begin{itemize}
\item \textbf{VERSÃO} \begin{itemize}
\item Alpha {-} 0.5.0.687
\end{itemize}
\item \textbf{CÓDIGO FONTE}: \begin{itemize}
\item 
\item \textbf{HISTÓRICO} \begin{itemize}
\item Criado por: Paulo Sérgio da Silva Pacheco paulosspacheco@yahoo.com.br) ✅
\item \textbf{2022{-}03{-}16} \begin{itemize}
\item \textbf{19:49} \begin{itemize}
\item Criar o tipo TStrSQL com objetivo de criar sql para qualquer banco de dados conhecido pelo sistema. ✅
\end{itemize}
\end{itemize}
\end{itemize}
\end{itemize}
\end{itemize}
\section{Uses}
\begin{itemize}
\item \begin{ttfamily}Classes\end{ttfamily}\item \begin{ttfamily}SysUtils\end{ttfamily}\item \begin{ttfamily}db\end{ttfamily}\item \begin{ttfamily}mi.rtl.types\end{ttfamily}(\ref{mi.rtl.Types})\item \begin{ttfamily}mi.rtl.Consts\end{ttfamily}(\ref{mi.rtl.Consts})\item \begin{ttfamily}mi.rtl.files\end{ttfamily}(\ref{mi.rtl.files})\item \begin{ttfamily}mi.rtl.objects.consts.mi{\_}msgbox\end{ttfamily}\item \begin{ttfamily}mi.rtl.Objects.Methods\end{ttfamily}(\ref{mi.rtl.Objects.Methods})\item \begin{ttfamily}mi.rtl.objects.Methods.dates\end{ttfamily}(\ref{mi.rtl.objects.Methods.dates})\item \begin{ttfamily}mi.rtl.Objects.Methods.ParamExecucao.Application\end{ttfamily}(\ref{mi.rtl.Objects.Methods.Paramexecucao.Application})\item \begin{ttfamily}mi.rtl.Objects.Methods.Exception\end{ttfamily}(\ref{mi.rtl.Objects.Methods.Exception})\item \begin{ttfamily}mi.rtl.Objectss\end{ttfamily}(\ref{mi.rtl.Objectss})\end{itemize}
\section{Visão Geral}
\begin{description}
\item[\texttt{\begin{ttfamily}TUiTypes\end{ttfamily} Classe}]
\end{description}
\section{Classes, Interfaces, Objetos e Registros}
\subsection*{TUiTypes Classe}
\subsubsection*{\large{\textbf{Hierarquia}}\normalsize\hspace{1ex}\hfill}
TUiTypes {$>$} \begin{ttfamily}TObjectss\end{ttfamily}(\ref{mi.rtl.Objectss.TObjectss}) {$>$} 
\subsubsection*{\large{\textbf{Descrição}}\normalsize\hspace{1ex}\hfill}
A class \textbf{\begin{ttfamily}TUiTypes\end{ttfamily}} concentra todos os tipo do pacote mi.ui.\chapter{Unit mi{\_}ui{\_}Dmxscroller{\_}sql}
\section{Descrição}
A unit \textbf{\begin{ttfamily}mi{\_}ui{\_}Dmxscroller{\_}sql\end{ttfamily}} implementa a classe \begin{ttfamily}TUiDmxScroller{\_}sql\end{ttfamily}(\ref{mi_ui_Dmxscroller_sql.TUiDmxScroller_sql}).

\begin{itemize}
\item \textbf{VERSÃO} \begin{itemize}
\item Alpha {-} 0.5.0.687
\end{itemize}
\item \textbf{CÓDIGO FONTE}: \begin{itemize}
\item 
\item \textbf{PENDÊNCIAS} \begin{itemize}
\item T12 Falta implementar chave estrangeira em createTable;
\item T12 Em \begin{ttfamily}TUiDmxScroller{\_}sql.DoOnNewRecord\end{ttfamily}(\ref{mi_rtl_ui_Dmxscroller_sql.TUiDmxScroller_sql-DoOnNewRecord}); está executando o método (CustomBufDataset as TSQLQuery).Append; antes do componenente \begin{ttfamily}TUiDmxScroller{\_}sql\end{ttfamily}(\ref{mi_ui_Dmxscroller_sql.TUiDmxScroller_sql}) está visível e isto está gerando exceção.
\item T12 ANÁLISE \begin{itemize}
\item [Estudar os procedimentos armazenados](https://www.w3schools.com/sql/sql{\_}stored{\_}procedures.asp)
\item [Estudar as restrições SQL](https://www.w3schools.com/sql/sql{\_}constraints.asp)
\item Como saber se um campo é uma chave que liga outra tabela? \begin{itemize}
\item https://www.w3schools.com/sql/sql{\_}foreignkey.asp (SQL FOREIGN KEY Constraint)

\texttt{\\\nopagebreak[3]
\\\nopagebreak[3]
/*Não,~podemos~permitir~que~os~registros~das~pessoas~que~possuim~camisetas\\\nopagebreak[3]
~~lavando~sejam~apagados,~para~garantir~a~integridade~da~informação.\\\nopagebreak[3]
~~Para~isso~devemos~utilizar~o~}\textbf{as}\texttt{~chaves~estrangeiras~que~acusarão\\\nopagebreak[3]
~~um~erro~quando~tentarmos~deletar~uma~pessoa~que~possuir~camisetas.\\\nopagebreak[3]
~~Veja~em~código:\\\nopagebreak[3]
*/\\\nopagebreak[3]
\\\nopagebreak[3]
CREATE~TABLE~Pessoa(\\\nopagebreak[3]
~~~~IdPessoa~INT~}\textbf{NOT}\texttt{~NULL~PRIMARY~KEY~IDENTITY(1,1),\\\nopagebreak[3]
~~~~Nome~VARCHAR(20)~}\textbf{NOT}\texttt{~NULL\\\nopagebreak[3]
)\\\nopagebreak[3]
\\\nopagebreak[3]
CREATE~TABLE~Camiseta(\\\nopagebreak[3]
~~~~IdCamiseta~INT~}\textbf{NOT}\texttt{~NULL~PRIMARY~KEY~IDENTITY(1,1),\\\nopagebreak[3]
~~~~Descrição~VARCHAR(20)~}\textbf{NOT}\texttt{~NULL,\\\nopagebreak[3]
~~~~IdPessoa~INT~}\textbf{NOT}\texttt{~NULL\\\nopagebreak[3]
~~~~CONSTRAINT~FK{\_}Camiseta{\_}Pessoa~FOREIGN~~KEY(IdPessoa)~REFERENCES~Pessoa(IdPessoa)\\\nopagebreak[3]
)\\\nopagebreak[3]
\\\nopagebreak[3]
INSERT~INTO~Pessoa~VALUES~('HeyJoe')\\\nopagebreak[3]
INSERT~INTO~Pessoa~VALUES~('Caique')\\\nopagebreak[3]
\\\nopagebreak[3]
\\\nopagebreak[3]
INSERT~INTO~Camiseta~VALUES~('Azul',~1)\\\nopagebreak[3]
INSERT~INTO~Camiseta~VALUES~('Amarela',~1)\\\nopagebreak[3]
INSERT~INTO~Camiseta~VALUES~('Preta',~2)\\\nopagebreak[3]
\\\nopagebreak[3]
SELECT~*~FROM~Pessoa,~Camiseta~WHERE~Pessoa.IdPessoa~=~Camiseta.IdPessoa\\
}
\end{itemize}
\item Como saber o tipo de relacionamento que os campos de outra tabela tem com a tabela atual?
\end{itemize}
\item T12 A opção CreateTable está dando mensagem de erro quando a coluna já existe. \begin{itemize}
\item Encontrar uma forma de não gerar exceção ou ignorar as exceções nesta rotina.
\end{itemize}
\item T12 Em \begin{ttfamily}TUiDmxScroller{\_}sql.AlterTable\end{ttfamily}(\ref{mi_rtl_ui_Dmxscroller_sql.TUiDmxScroller_sql-AlterTable}) checar: \begin{itemize}
\item T12 Criar código para todos os tipos reconhecidos por marIcaraí.
\item T12 Debugar para saber se está tudo funcionando.
\item T12 Permitir adicionar uma nova coluna mesmo que a tabela já exista.
\end{itemize}
\item T12 Em SetTableName(aTableName:String) criticar o nome aTableName é um nome válido para a tabela.
\end{itemize}
\item \textbf{HISTÓRICO} \begin{itemize}
\item Criado por: Paulo Sérgio da Silva Pacheco paulosspacheco@yahoo.com.br) ✅
\item \textbf{2022{-}03{-}13} \begin{itemize}
\item \textbf{09:00} \begin{itemize}
\item T12 Implementar a criação de código SQL baseado nos dados de \begin{ttfamily}TUiDmxScroller\end{ttfamily}(\ref{mi_rtl_ui_Dmxscroller.TUiDmxScroller}). \begin{itemize}
\item \textbf{ANÁLISE:} \begin{itemize}
\item Onde pegar o nome da tabela ou consulta? ✅ \begin{itemize}
\item TUiDmxScroller.CustomBufDataset.FileName;
\end{itemize}
\item Onde pegar o nome dos campos da tabela CustomBufDataset.Filename? ✅ \begin{itemize}
\item A lista de campos encontra{-}se em : TUiDmxScroller{\_}Atributos.Fields : TFPList;
\end{itemize}
\item Como saber se \begin{ttfamily}TUiDmxScroller\end{ttfamily}(\ref{mi_rtl_ui_Dmxscroller.TUiDmxScroller}) é uma tabela ou a uma consulta? \begin{itemize}
\item Se todos os TUiDmxScroller{\_}Atributos.Fields[].FieldName não contém '|' é porque é FileName é uma tabela. ✅
\item Se pelo menos um TUiDmxScroller{\_}Atributos.Fields[].FieldName contém '|' é porque é FileName é uma consulta envolvendo mais de uma tabela. ✅
\end{itemize}
\item Como saber se uma tabela ou consulta existe do banco de dados? \begin{itemize}
\item O SQL do \textbf{postegres} e do \textbf{sqlite3} tem a clausula \textbf{IN NOT EXISTS} no comando CREATE TABLE: \begin{itemize}
\item EXEMPLO:

\texttt{\\\nopagebreak[3]
CREATE~TABLE~}\textbf{IF}\texttt{~}\textbf{NOT}\texttt{~EXISTS~TEST01~();~~✅\\
}
\end{itemize}
\end{itemize}
\end{itemize}
\item \textbf{REFERÊNCIAS} \begin{itemize}
\item https://en.wikipedia.org/wiki/SQL:2016 (SQL:2016) \begin{itemize}
\item (PostgresSQL aceita 160 das 169 especificação 2016)(https://www.postgresql.org/docs/12/features.html)
\item [Comparativo entre os bancos de dados x conformidade SQL](https://en.wikipedia.org/wiki/SQL{\_}compliance)
\item [Clientes de bancos de dados opensource](https://medevel.com/17-sql-client-open-source/)
\item https://dbeaver.io/ (Instalei programa cliente SQL DBeaver) \begin{itemize}
\item Obs: Não deu certo. Ele é escrito em java e não funcionou o básico.
\end{itemize}
\item [sqlite create database if not exists](https://www.codegrepper.com/code-examples/sql/sqlite+create+database+if+not+exists)
\end{itemize}
\end{itemize}
\end{itemize}
\end{itemize}
\end{itemize}
\item \textbf{2022{-}03{-}14} \begin{itemize}
\item \textbf{08:22} \begin{itemize}
\item T12 Criar a unit mi{\_}ui{\_}Dmxscroller{\_}sql.pas com a classe \textbf{\begin{ttfamily}TUiDmxScroller{\_}sql\end{ttfamily}(\ref{mi_ui_Dmxscroller_sql.TUiDmxScroller_sql})} com objetivo de concentrar a integração do TDmxScroller com o componente \textbf{TSQLQuery} ✅
\end{itemize}
\item \textbf{20:00} \begin{itemize}
\item T12 Na Construção de TFields atualizar a propriedade \textbf{TField.ProviderFlags} com o tipo de acesso definido em \begin{ttfamily}TDmxFieldRec.Access\end{ttfamily}(\ref{mi_rtl_ui_Dmxscroller.TDmxFieldRec-access}) ✅
\end{itemize}
\item \textbf{21:12} \begin{itemize}
\item T12 Criar propriedade \textbf{TableName} ✅
\end{itemize}
\item \textbf{21:27} \begin{itemize}
\item T12 Criar Function SetSqlCustomBufDataset:Boolean;Virtual; \begin{itemize}
\item CustomBufDataset.SQL := \textbf{SELECT * FROM X } onde X será definido pela propriedade \textbf{TableName} ✅
\end{itemize}
\end{itemize}
\end{itemize}
\item \textbf{2022{-}03{-}15} \begin{itemize}
\item \textbf{09:11} \begin{itemize}
\item Depurar o que fiz ontem para fazer funciona a atualização do banco de dados SQL. ✅
\end{itemize}
\item \textbf{11:36} \begin{itemize}
\item Criar método \begin{ttfamily}TUiDmxScroller{\_}sql.AlterTable\end{ttfamily}(\ref{mi_rtl_ui_Dmxscroller_sql.TUiDmxScroller_sql-AlterTable}) : Boolean; ✅
\end{itemize}
\item \textbf{14:38} \begin{itemize}
\item T12 Atualizar TSQLQury.TFields.ProviderFlags com TUiDmxScroller.MiProviderFlags ✅
\end{itemize}
\end{itemize}
\item \textbf{2022{-}03{-}16} \begin{itemize}
\item \textbf{16:23} \begin{itemize}
\item T12 Em \begin{ttfamily}TUiDmxScroller{\_}sql.CreateCustomBufDataset{\_}FieldDefs\end{ttfamily}(\ref{mi_rtl_ui_Dmxscroller_sql.TUiDmxScroller_sql-CreateCustomBufDataset_FieldDefs}), atualizar \textbf{TField.ProviderFlags} com os dados do campo \textbf{\begin{ttfamily}TDmxFieldRec.ProviderFlags\end{ttfamily}(\ref{mi_rtl_ui_Dmxscroller.TDmxFieldRec-ProviderFlags})}. ✅
\end{itemize}
\item \textbf{16:54} \begin{itemize}
\item Em \begin{ttfamily}TUiDmxScroller{\_}sql.AlterTable\end{ttfamily}(\ref{mi_rtl_ui_Dmxscroller_sql.TUiDmxScroller_sql-AlterTable}) usar os flags \begin{ttfamily}TDmxFieldRec.ProviderFlags\end{ttfamily}(\ref{mi_rtl_ui_Dmxscroller.TDmxFieldRec-ProviderFlags}) para criação da tabela. ✅
\end{itemize}
\end{itemize}
\item \textbf{2022{-}03{-}17} \begin{itemize}
\item \textbf{10:48} \begin{itemize}
\item T12 Os flags indicando que se trata de chave primária não está sendo atualizado em createStructor, por isso não está criando a chave primária. ✅
\end{itemize}
\end{itemize}
\item \textbf{2022{-}03{-}18} \begin{itemize}
\item \textbf{10:40} \begin{itemize}
\item T12 Ao criar uma tabela SQL em \textbf{AlterTable} adicionar colunas ao invés de criar a tabela toda. ✅ \begin{itemize}
\item \textbf{Motivo}: \begin{itemize}
\item Permitir que o banco de dados fique compatível com o Template.
\item Alterar um coluna de forma automática não é bom, porque o que está feito gera dependências que produzirão erros ao fazer essas alterações.
\end{itemize}
\end{itemize}
\end{itemize}
\end{itemize}
\item \textbf{2022{-}03{-}21} \begin{itemize}
\item \textbf{08:57} \begin{itemize}
\item T12 Criar function SQL{\_}AddkeysPrimaryKeyComposite(I : Integer):Boolean; ✅ \begin{itemize}
\item Esta função adiciona chave primária composta na tabela.
\item \textbf{REFERÊNCIA} \begin{itemize}
\item [Como adiconar chave primaria usando a expressão ALTER TABLE](https://www.techonthenet.com/postgresql/primary{\_}keys.php{\#}:~:text=In{\%}20PostgreSQL{\%}2C{\%}20a{\%}20primary{\%}20key{\%}20is{\%}20created{\%}20using{\%}20either{\%}20a,or{\%}20drop{\%}20a{\%}20primary{\%}20key.)
\end{itemize}
\end{itemize}
\end{itemize}
\item \textbf{15:40} \begin{itemize}
\item T12 Em AlterTable criar a restrição de chave estrangeira no TDmxScroller{\_}sql. ✅ \begin{itemize}
\item Nome da função: function AddKeyForeigns(I : Integer):Boolean;
\end{itemize}
\end{itemize}
\end{itemize}
\item \textbf{2022{-}03{-}22} \begin{itemize}
\item \textbf{09:00} \begin{itemize}
\item T12 Documentar as units \begin{ttfamily}TuiTypes\end{ttfamily}(\ref{mi_rtl_ui_types.TUiTypes}) e TUIConsts. ✅
\end{itemize}
\item \textbf{10:00} \begin{itemize}
\item T12 Criar os relacionamentos entre tabelas (restrições entre tabelas) ✅
\end{itemize}
\item \textbf{14:14} \begin{itemize}
\item T12 Depurar os relacionamentos entre tabelas. ✅
\end{itemize}
\item \textbf{18:47} \begin{itemize}
\item O Componente CustomBufDataset não está entrando no modo edit. ✅ \begin{itemize}
\item O problema estava nos eventos TScrollBoxDMX.DoOnEnter e TScrollBoxDMX.DoOnExit;]
\end{itemize}
\end{itemize}
\end{itemize}
\item \textbf{2022{-}03{-}22} \begin{itemize}
\item \textbf{20:27} \begin{itemize}
\item T12 Analisar como criar os comandos CmIncluir, cmAlterar, cmExcluir, cmConsulta para a tabela TDmxScroller \begin{itemize}
\item Criar os comandos: ✅ \begin{itemize}
\item Public Procedure DoOnNewRecord;overload;override; //Usado para inicializa os parametros de um novo registro
\item Public Procedure PutRec;Override;//Grava o buffer no arquivo memo
\item Public Procedure GetRec;Override;//O primeiro registro esta gravado em Value
\item Public Function DeleteRec:Boolean;Override;
\item Function UpdateRec: Boolean;Override;
\item Function UpdateRec{\_}if{\_}RecordAltered:Boolean;Override;
\item Function PrevRec : Boolean;overload;override;
\item Function NextRec : Boolean;overload;override;
\end{itemize}
\end{itemize}
\end{itemize}
\end{itemize}
\item \textbf{2022{-}03{-}23} \begin{itemize}
\item Criar método Public Function AddRec:Boolean;Override; ✅ \begin{itemize}
\item Para que DoAddrec possa adicionar o registro é necessário que o registro esteja selecionando, ou seja no modo edit.
\item Obs: Está com problema.
\end{itemize}
\end{itemize}
\item \textbf{2022{-}03{-}25} \begin{itemize}
\item https://wiki.freepascal.org/Firebird{\#}Creating{\_}objects{\_}programmatically (Estudar página sobre o banco de dados firebird) ✅
\end{itemize}
\item \textbf{2022{-}03{-}28} \begin{itemize}
\item Em \begin{ttfamily}TUiDmxScroller{\_}sql.DoOnNewRecord\end{ttfamily}(\ref{mi_rtl_ui_Dmxscroller_sql.TUiDmxScroller_sql-DoOnNewRecord}); está executando o método (CustomBufDataset as TSQLQuery).Append; antes do componenente \begin{ttfamily}TUiDmxScroller{\_}sql\end{ttfamily}(\ref{mi_ui_Dmxscroller_sql.TUiDmxScroller_sql}) está visível e isto está gerando exceção. {-}
\end{itemize}
\item \textbf{2022{-}03{-}30} \begin{itemize}
\item Implementar a conexão com o banco de dados usando o componente Mi{\_}Application.
\end{itemize}
\item \textbf{2022{-}04{-}14} \begin{itemize}
\item Debugar o método \textbf{\begin{ttfamily}TUiDmxScroller{\_}sql.AlterTable\end{ttfamily}(\ref{mi_rtl_ui_Dmxscroller_sql.TUiDmxScroller_sql-AlterTable})}.
\end{itemize}
\item \textbf{2022{-}04{-}15} \begin{itemize}
\item O método \textbf{\begin{ttfamily}TUiDmxScroller{\_}sql.AlterTable\end{ttfamily}(\ref{mi_rtl_ui_Dmxscroller_sql.TUiDmxScroller_sql-AlterTable})} precisa reconhecer a sintaxe do banco de dados selecionado. \begin{itemize}
\item O postgresSQL sintaxe: \begin{itemize}
\item CREATE TABLE [IF NOT EXISTS] table{\_}name ( column1 datatype(length) column{\_}contraint, column2 datatype(length) column{\_}contraint, column3 datatype(length) column{\_}contraint, table{\_}constraints );
\item \textbf{REFERÊNCIA} \begin{itemize}
\item [postgresql{-}create{-}table](https://www.postgresqltutorial.com/postgresql-tutorial/postgresql-create-table/)
\end{itemize}
\end{itemize}
\item O sqLite3 sintaxe: \begin{itemize}
\item CREATE TABLE [IF NOT EXISTS] [schema{\_}name].table{\_}name ( column{\_}1 data{\_}type PRIMARY KEY, column{\_}2 data{\_}type NOT NULL, column{\_}3 data{\_}type DEFAULT 0,table{\_}constraints) [WITHOUT ROWID];
\item \textbf{REFERÊNCIA:} \begin{itemize}
\item [sqlite{-}create{-}table](https://www.sqlitetutorial.net/sqlite-create-table/)
\end{itemize}
\end{itemize}
\end{itemize}
\end{itemize}
\end{itemize}
\end{itemize}
\end{itemize}
\section{Uses}
\begin{itemize}
\item \begin{ttfamily}Classes\end{ttfamily}\item \begin{ttfamily}SysUtils\end{ttfamily}\item \begin{ttfamily}BufDataset\end{ttfamily}\item \begin{ttfamily}db\end{ttfamily}\item \begin{ttfamily}SqlDb\end{ttfamily}\item \begin{ttfamily}mi.rtl.Types\end{ttfamily}(\ref{mi.rtl.Types})\item \begin{ttfamily}mi{\_}ui{\_}types\end{ttfamily}\item \begin{ttfamily}mi{\_}ui{\_}consts\end{ttfamily}\item \begin{ttfamily}mi{\_}ui{\_}Dmxscroller\end{ttfamily}\item \begin{ttfamily}uMi{\_}ui{\_}custom{\_}application\end{ttfamily}\end{itemize}
\section{Visão Geral}
\begin{description}
\item[\texttt{\begin{ttfamily}TDmxScroller{\_}sql{\_}Atributos\end{ttfamily} Classe}]
\item[\texttt{\begin{ttfamily}TUiDmxScroller{\_}sql\end{ttfamily} Classe}]
\end{description}
\section{Classes, Interfaces, Objetos e Registros}
\subsection*{TDmxScroller{\_}sql{\_}Atributos Classe}
\subsubsection*{\large{\textbf{Hierarquia}}\normalsize\hspace{1ex}\hfill}
TDmxScroller{\_}sql{\_}Atributos {$>$} \begin{ttfamily}TUiDmxScroller\end{ttfamily}(\ref{mi_rtl_ui_Dmxscroller.TUiDmxScroller}) {$>$} \begin{ttfamily}TUiMethods\end{ttfamily}(\ref{mi_rtl_ui_methods.TUiMethods}) {$>$} 
TUiConsts
\subsubsection*{\large{\textbf{Descrição}}\normalsize\hspace{1ex}\hfill}
A class \textbf{\begin{ttfamily}TDmxScroller{\_}sql{\_}Atributos\end{ttfamily}} contém os atributos da class TDmxScroller{\_}sql\subsubsection*{\large{\textbf{Campos}}\normalsize\hspace{1ex}\hfill}
\paragraph*{CustomBufDataset}\hspace*{\fill}

\begin{list}{}{
\settowidth{\tmplength}{\textbf{Declaração}}
\setlength{\itemindent}{0cm}
\setlength{\listparindent}{0cm}
\setlength{\leftmargin}{\evensidemargin}
\addtolength{\leftmargin}{\tmplength}
\settowidth{\labelsep}{X}
\addtolength{\leftmargin}{\labelsep}
\setlength{\labelwidth}{\tmplength}
}
\begin{flushleft}
\item[\textbf{Declaração}\hfill]
\begin{ttfamily}
public CustomBufDataset: TCustomBufDataset;\end{ttfamily}


\end{flushleft}
\par
\item[\textbf{Descrição}]
O atributo pública \textbf{\begin{ttfamily}CustomBufDataset\end{ttfamily}} é definida em \textbf{CreateCustomBufDataset{\_}FieldDefs} que é executado em \textbf{TDmxScroller.CreateData} baseado na estrutura do Template passado por \begin{ttfamily}GetTemplate\end{ttfamily}(\ref{mi_rtl_ui_Dmxscroller.TUiDmxScroller-GetTemplate}).

\begin{itemize}
\item \textbf{NOTA} \begin{itemize}
\item O atributo \textbf{\begin{ttfamily}CustomBufDataset\end{ttfamily}} deve ser passado por \textbf{DataSource.DataSet}.
\item Em \textbf{CreateCustomBufDataset{\_}FieldDefs} é criado os campo da propriedade \textbf{\begin{ttfamily}CustomBufDataset\end{ttfamily}} se a propriedade (\begin{ttfamily}DataSource\end{ttfamily}(\ref{mi_rtl_ui_Dmxscroller.TUiDmxScroller-DataSource}){$<$}{$>$}nil) e (DataSource.DataSet {$<$}{$>$} nil).
\item Se a propriedade DataSource.DataSet = nil então a propriedade \textbf{\begin{ttfamily}CustomBufDataset\end{ttfamily}=nil}
\item O método \textbf{CreateCustomBufDataset{\_}FieldDefs} reconhece duas possibilidade para os descendentes de \begin{ttfamily}CustomBufDataset\end{ttfamily} quais sejam: \begin{enumerate}
\setcounter{enumi}{0} \setcounter{enumii}{0} \setcounter{enumiii}{0} \setcounter{enumiv}{0} 
\item [TBufDataset](https://www.freepascal.org/docs-html/fcl/bufdataset/tbufdataset.html)
\setcounter{enumi}{1} \setcounter{enumii}{1} \setcounter{enumiii}{1} \setcounter{enumiv}{1} 
\item https://www.freepascal.org/docs-html/fcl/sqldb/tcustomsqlquery.html (TCustomSQLQuery) \begin{itemize}
\item Preciso das propriedades de acesso a banco de dados SQL.
\item O evento \begin{ttfamily}OnGetTemplate\end{ttfamily}(\ref{mi_rtl_ui_Dmxscroller.TUiDmxScroller-onGetTemplate}) deve setar as propriedades customizadas de \textbf{TCustomSQLQuery}.
\end{itemize}
\end{enumerate}
\end{itemize}
\item \textbf{REFERẼNCIA}: \begin{itemize}
\item [tcustombufdataset](https://www.freepascal.org/daily/packages/fcl-db/bufdataset/tcustombufdataset-14.html)
\item [tcustomsqlquery](https://www.freepascal.org/docs-html/fcl/sqldb/tcustomsqlquery.html)
\item https://www.freepascal.org/docs-html/fcl/bufdataset/tcustombufdataset.html (TCustomBufDataset);
\item [TBufDataSet](https://wiki.freepascal.org/How{\_}to{\_}write{\_}in-memory{\_}database{\_}applications{\_}in{\_}Lazarus/FPC{\#}TBufDataSet)
\item [tstatementtype.html](https://www.freepascal.org/docs-html/fcl/sqltypes/tstatementtype.html)
\item [tsqlquery](https://www.freepascal.org/docs-html/fcl/sqldb/tsqlquery.html)
\item [tdatasetstate](https://www.freepascal.org/docs-html/fcl/db/tdatasetstate.html)
\item [How{\_}to{\_}connect{\_}to{\_}a{\_}database{\_}server](https://wiki.freepascal.org/SqlDBHowto{\#}How{\_}to{\_}connect{\_}to{\_}a{\_}database{\_}server.3F)
\item [Example:{\_}reading{\_}data{\_}from{\_}a{\_}table](https://wiki.freepascal.org/SqlDBHowto{\#}Example:{\_}reading{\_}data{\_}from{\_}a{\_}table)
\item [How{\_}to{\_}execute{\_}direct{\_}queries.2Fmake{\_}a{\_}table](https://wiki.freepascal.org/SqlDBHowto{\#}How{\_}to{\_}execute{\_}direct{\_}queries.2Fmake{\_}a{\_}table.3F)
\item [How{\_}to{\_}read{\_}data{\_}from{\_}a{\_}table](https://wiki.freepascal.org/SqlDBHowto{\#}How{\_}to{\_}read{\_}data{\_}from{\_}a{\_}table.3F)
\item [Why{\_}does{\_}TSQLQuery.RecordCount{\_}always{\_}return](https://wiki.freepascal.org/SqlDBHowto{\#}Why{\_}does{\_}TSQLQuery.RecordCount{\_}always{\_}return{\_}10.3F)
\item [Como usar SQLDb no Lazarus](https://wiki.freepascal.org/SqlDBHowto{\#}Lazarus)
\item [Trabalhando com tabelas relacionadas](https://wiki.freepascal.org/MasterDetail)
\item [How{\_}to{\_}change{\_}data{\_}in{\_}a{\_}table](https://wiki.freepascal.org/SqlDBHowto{\#}How{\_}to{\_}change{\_}data{\_}in{\_}a{\_}table.3F)
\item [How{\_}does{\_}SqlDB{\_}send{\_}the{\_}changes{\_}to{\_}the{\_}database{\_}server](https://wiki.freepascal.org/SqlDBHowto{\#}How{\_}does{\_}SqlDB{\_}send{\_}the{\_}changes{\_}to{\_}the{\_}database{\_}server.3F)
\item [How{\_}to{\_}handle{\_}Errors](https://wiki.freepascal.org/SqlDBHowto{\#}How{\_}to{\_}handle{\_}Errors)
\item [How{\_}to{\_}execute{\_}a{\_}query{\_}using{\_}TSQLQuery](https://wiki.freepascal.org/SqlDBHowto{\#}How{\_}to{\_}execute{\_}a{\_}query{\_}using{\_}TSQLQuery.3F)
\item [How{\_}to{\_}use{\_}parameters{\_}in{\_}a{\_}query](https://wiki.freepascal.org/SqlDBHowto{\#}How{\_}to{\_}use{\_}parameters{\_}in{\_}a{\_}query.3F)
\item [Select{\_}query](https://wiki.freepascal.org/SqlDBHowto{\#}Select{\_}query)
\item [Exemplo de SQLQuery com parãmetros](https://wiki.freepascal.org/SqlDBHowto{\#}Example)
\item https://wiki.freepascal.org/SqlDBHowto{\#}Troubleshooting:{\_}TSQLConnection{\_}logging (Troubleshooting:{\_}TSQLConnection{\_}logging) \begin{itemize}
\item [Exemplo de log](https://wiki.freepascal.org/SqlDBHowto{\#}FPC{\_}.28or:{\_}the{\_}manual{\_}way.29)
\end{itemize}
\end{itemize}
\end{itemize}

\end{list}
\subsection*{TUiDmxScroller{\_}sql Classe}
\subsubsection*{\large{\textbf{Hierarquia}}\normalsize\hspace{1ex}\hfill}
TUiDmxScroller{\_}sql {$>$} \begin{ttfamily}TDmxScroller{\_}sql{\_}Atributos\end{ttfamily}(\ref{mi_ui_Dmxscroller_sql.TDmxScroller_sql_Atributos}) {$>$} \begin{ttfamily}TUiDmxScroller\end{ttfamily}(\ref{mi_rtl_ui_Dmxscroller.TUiDmxScroller}) {$>$} \begin{ttfamily}TUiMethods\end{ttfamily}(\ref{mi_rtl_ui_methods.TUiMethods}) {$>$} 
TUiConsts
\subsubsection*{\large{\textbf{Descrição}}\normalsize\hspace{1ex}\hfill}
A classe \textbf{\begin{ttfamily}TUiDmxScroller{\_}sql\end{ttfamily}} implementa o acesso ao banco de dados usando o atributo \textbf{\begin{ttfamily}CustomBufDataset\end{ttfamily}(\ref{mi_ui_Dmxscroller_sql.TDmxScroller_sql_Atributos-CustomBufDataset})}

\begin{itemize}
\item \textbf{NOTA} \begin{itemize}
\item O atributo \textbf{\begin{ttfamily}CustomBufDataset\end{ttfamily}(\ref{mi_ui_Dmxscroller_sql.TDmxScroller_sql_Atributos-CustomBufDataset})} pode ser \textbf{TBufDataset} não conectado a banco de dados sql e \textbf{TCustomSQLQuery} conectado ao banco de dados SQL.
\end{itemize}
\item \textbf{REFERÊNCIA} \begin{itemize}
\item [Working{\_}With{\_}TSQLQuery](https://wiki.freepascal.org/Working{\_}With{\_}TSQLQuery)
\item [Parameters{\_}in{\_}TSQLQuery](https://wiki.freepascal.org/Working{\_}With{\_}TSQLQuery{\#}Parameters{\_}in{\_}TSQLQuery.SQL)
\item [sql{-}basico](https://www.devmedia.com.br/sql-basico/28877)
\end{itemize}
\end{itemize}\subsubsection*{\large{\textbf{Propriedades}}\normalsize\hspace{1ex}\hfill}
\paragraph*{DataSource}\hspace*{\fill}

\begin{list}{}{
\settowidth{\tmplength}{\textbf{Declaração}}
\setlength{\itemindent}{0cm}
\setlength{\listparindent}{0cm}
\setlength{\leftmargin}{\evensidemargin}
\addtolength{\leftmargin}{\tmplength}
\settowidth{\labelsep}{X}
\addtolength{\leftmargin}{\labelsep}
\setlength{\labelwidth}{\tmplength}
}
\begin{flushleft}
\item[\textbf{Declaração}\hfill]
\begin{ttfamily}
published property DataSource : TDataSource Read {\_}DataSource   Write  {\_}DataSource;\end{ttfamily}


\end{flushleft}
\par
\item[\textbf{Descrição}]
A propriedade \textbf{\begin{ttfamily}DataSource\end{ttfamily}} permite que controles da \textbf{LCL} (Lazarus Componenents Library) possam usar os dados do componenente \textbf{TDmxScroller}.

\begin{itemize}
\item \textbf{NOTA} \begin{itemize}
\item Essa integração permite que \textbf{TDmxScroller} utilize todos os componentes de banco de dados do Free Pascal.
\end{itemize}
\end{itemize}

\end{list}
\subsubsection*{\large{\textbf{Campos}}\normalsize\hspace{1ex}\hfill}
\paragraph*{{\_}DataSource}\hspace*{\fill}

\begin{list}{}{
\settowidth{\tmplength}{\textbf{Declaração}}
\setlength{\itemindent}{0cm}
\setlength{\listparindent}{0cm}
\setlength{\leftmargin}{\evensidemargin}
\addtolength{\leftmargin}{\tmplength}
\settowidth{\labelsep}{X}
\addtolength{\leftmargin}{\labelsep}
\setlength{\labelwidth}{\tmplength}
}
\begin{flushleft}
\item[\textbf{Declaração}\hfill]
\begin{ttfamily}
protected {\_}DataSource: TDataSource;\end{ttfamily}


\end{flushleft}
\end{list}
\subsubsection*{\large{\textbf{Métodos}}\normalsize\hspace{1ex}\hfill}
\paragraph*{SetDataBase}\hspace*{\fill}

\begin{list}{}{
\settowidth{\tmplength}{\textbf{Declaração}}
\setlength{\itemindent}{0cm}
\setlength{\listparindent}{0cm}
\setlength{\leftmargin}{\evensidemargin}
\addtolength{\leftmargin}{\tmplength}
\settowidth{\labelsep}{X}
\addtolength{\leftmargin}{\labelsep}
\setlength{\labelwidth}{\tmplength}
}
\begin{flushleft}
\item[\textbf{Declaração}\hfill]
\begin{ttfamily}
protected procedure SetDataBase;\end{ttfamily}


\end{flushleft}
\end{list}
\paragraph*{Create}\hspace*{\fill}

\begin{list}{}{
\settowidth{\tmplength}{\textbf{Declaração}}
\setlength{\itemindent}{0cm}
\setlength{\listparindent}{0cm}
\setlength{\leftmargin}{\evensidemargin}
\addtolength{\leftmargin}{\tmplength}
\settowidth{\labelsep}{X}
\addtolength{\leftmargin}{\labelsep}
\setlength{\labelwidth}{\tmplength}
}
\begin{flushleft}
\item[\textbf{Declaração}\hfill]
\begin{ttfamily}
public constructor Create(aOwner:TComponent); Override;\end{ttfamily}


\end{flushleft}
\par
\item[\textbf{Descrição}]
Constrói o componente

\end{list}
\paragraph*{GetkeysPrimaryComposite}\hspace*{\fill}

\begin{list}{}{
\settowidth{\tmplength}{\textbf{Declaração}}
\setlength{\itemindent}{0cm}
\setlength{\listparindent}{0cm}
\setlength{\leftmargin}{\evensidemargin}
\addtolength{\leftmargin}{\tmplength}
\settowidth{\labelsep}{X}
\addtolength{\leftmargin}{\labelsep}
\setlength{\labelwidth}{\tmplength}
}
\begin{flushleft}
\item[\textbf{Declaração}\hfill]
\begin{ttfamily}
public function GetkeysPrimaryComposite(I : Integer):AnsiString;\end{ttfamily}


\end{flushleft}
\par
\item[\textbf{Descrição}]
O método \textbf{\begin{ttfamily}GetkeysPrimaryComposite\end{ttfamily}} retorna a lista de campos pertencentes a chave composta primária.

\end{list}
\paragraph*{GetKeysPrimary}\hspace*{\fill}

\begin{list}{}{
\settowidth{\tmplength}{\textbf{Declaração}}
\setlength{\itemindent}{0cm}
\setlength{\listparindent}{0cm}
\setlength{\leftmargin}{\evensidemargin}
\addtolength{\leftmargin}{\tmplength}
\settowidth{\labelsep}{X}
\addtolength{\leftmargin}{\labelsep}
\setlength{\labelwidth}{\tmplength}
}
\begin{flushleft}
\item[\textbf{Declaração}\hfill]
\begin{ttfamily}
public function GetKeysPrimary:AnsiString;\end{ttfamily}


\end{flushleft}
\par
\item[\textbf{Descrição}]
A função \textbf{\begin{ttfamily}GetKeysPrimary\end{ttfamily}} retorna a chave primária composta ou não na tabela.

\begin{itemize}
\item \textbf{Como TSQLQuery trata os campos de chave primária} \begin{itemize}
\item Ao atualizar registros, TSQLQuery precisa saber quais campos compõem a chave primária que pode ser usada para atualizar o registro e quais campos devem ser atualizados: com base nessas informações, ele constrói um comando SQL UPDATE, INSERT ou DELETE.
\item A construção da instrução SQL é controlada pela propriedade UsePrimaryKeyAsKey e pelas propriedades ProviderFlags .
\item A propriedade Providerflags é um conjunto de 3 sinalizadores: \begin{itemize}
\item pfInkey : O campo faz parte da chave primária
\item pfInWhere : O campo deve ser utilizado na cláusula WHERE das instruções SQL.
\item pfInUpdate : Atualizações ou inserções devem incluir este campo. Por padrão, ProviderFlags consiste apenas em pfInUpdate .
\item \textbf{NOTA* \begin{itemize}
\item Se sua tabela tiver uma chave primária (conforme descrito acima), você só precisará definir a propriedade **UsePrimaryKeyAsKey
\end{itemize}} como True e tudo será feito para você. Isso definirá o sinalizador pfInKey para os campos de chave primária.
\end{itemize}
\end{itemize}
\item \textbf{REFERÊNCIA} \begin{itemize}
\item [Working With TSQLQuery e Primary{\_}key{\_}Fields](https://wiki.freepascal.org/Working{\_}With{\_}TSQLQuery{\#}Primary{\_}key{\_}Fields)
\end{itemize}
\end{itemize}

\end{list}
\paragraph*{CreateTable}\hspace*{\fill}

\begin{list}{}{
\settowidth{\tmplength}{\textbf{Declaração}}
\setlength{\itemindent}{0cm}
\setlength{\listparindent}{0cm}
\setlength{\leftmargin}{\evensidemargin}
\addtolength{\leftmargin}{\tmplength}
\settowidth{\labelsep}{X}
\addtolength{\leftmargin}{\labelsep}
\setlength{\labelwidth}{\tmplength}
}
\begin{flushleft}
\item[\textbf{Declaração}\hfill]
\begin{ttfamily}
public Function CreateTable: Boolean;\end{ttfamily}


\end{flushleft}
\par
\item[\textbf{Descrição}]
A função \textbf{\begin{ttfamily}CreateTable\end{ttfamily}} cria a tabela se a mesma não existir

\end{list}
\paragraph*{AlterTable}\hspace*{\fill}

\begin{list}{}{
\settowidth{\tmplength}{\textbf{Declaração}}
\setlength{\itemindent}{0cm}
\setlength{\listparindent}{0cm}
\setlength{\leftmargin}{\evensidemargin}
\addtolength{\leftmargin}{\tmplength}
\settowidth{\labelsep}{X}
\addtolength{\leftmargin}{\labelsep}
\setlength{\labelwidth}{\tmplength}
}
\begin{flushleft}
\item[\textbf{Declaração}\hfill]
\begin{ttfamily}
public Function AlterTable: Boolean; Virtual;\end{ttfamily}


\end{flushleft}
\par
\item[\textbf{Descrição}]
O método \textbf{\begin{ttfamily}AlterTable\end{ttfamily}} cria a tabela ou consulta \textbf{\begin{ttfamily}TableName\end{ttfamily}(\ref{mi_rtl_ui_Dmxscroller.TUiDmxScroller-TableName})} no banco de dados caso a propriedade \textbf{\begin{ttfamily}TableName\end{ttfamily}(\ref{mi_rtl_ui_Dmxscroller.TUiDmxScroller-TableName})} não existe no banco de dados e \textbf{\begin{ttfamily}TableName\end{ttfamily}(\ref{mi_rtl_ui_Dmxscroller.TUiDmxScroller-TableName})} seja diferente de vazio.

\begin{itemize}
\item O método \textbf{\begin{ttfamily}TUiDmxScroller{\_}sql.AlterTable\end{ttfamily}} precisa reconhecer a sintaxe do banco de dados selecionado. \begin{itemize}
\item O postgresSQL sintaxe: \begin{itemize}
\item \begin{ttfamily}CREATE\end{ttfamily}(\ref{mi_ui_Dmxscroller_sql.TUiDmxScroller_sql-Create}) TABLE [IF NOT EXISTS] table{\_}name ( column1 datatype(length) column{\_}contraint, column2 datatype(length) column{\_}contraint, column3 datatype(length) column{\_}contraint, table{\_}constraints );
\item \textbf{REFERÊNCIA} \begin{itemize}
\item [postgresql{-}\begin{ttfamily}create\end{ttfamily}(\ref{mi_ui_Dmxscroller_sql.TUiDmxScroller_sql-Create}){-}table](https://www.postgresqltutorial.com/postgresql-tutorial/postgresql-create-table/)
\end{itemize}
\end{itemize}
\item O sqLite3 sintaxe: \begin{itemize}
\item \begin{ttfamily}CREATE\end{ttfamily}(\ref{mi_ui_Dmxscroller_sql.TUiDmxScroller_sql-Create}) TABLE [IF NOT EXISTS] [schema{\_}name].table{\_}name ( column{\_}1 data{\_}type PRIMARY KEY, column{\_}2 data{\_}type NOT NULL, column{\_}3 data{\_}type DEFAULT 0,table{\_}constraints) [WITHOUT ROWID];
\item \textbf{REFERÊNCIA:} \begin{itemize}
\item [lang{\_}createtable.html](https://www.sqlite.org/lang{\_}createtable.html)
\item [sqlite{-}\begin{ttfamily}create\end{ttfamily}(\ref{mi_ui_Dmxscroller_sql.TUiDmxScroller_sql-Create}){-}table](https://www.sqlitetutorial.net/sqlite-create-table/)
\item [lang{\_}createtable.html](https://www.sqlite.org/lang{\_}createtable.html)
\end{itemize}
\end{itemize}
\end{itemize}
\item \textbf{NOTAS} \begin{itemize}
\item As tabelas só são criadas automaticamente caso a constante AlterTableQL = true.
\item Ao adiciona uma coluna que já exista no banco de dados o sistema trata a exceção e tenta adicionar a próxima coluna. Motivo: Poder expandir a tabela dinâmicamente.
\item O comportamento do Banco de dados SqLite ao criar uma tabela é diferente do postgres. \begin{itemize}
\item O sqLite não permite criar tabela vazia.
\end{itemize}
\end{itemize}
\end{itemize}

\end{list}
\paragraph*{SetSqlCustomBufDataset}\hspace*{\fill}

\begin{list}{}{
\settowidth{\tmplength}{\textbf{Declaração}}
\setlength{\itemindent}{0cm}
\setlength{\listparindent}{0cm}
\setlength{\leftmargin}{\evensidemargin}
\addtolength{\leftmargin}{\tmplength}
\settowidth{\labelsep}{X}
\addtolength{\leftmargin}{\labelsep}
\setlength{\labelwidth}{\tmplength}
}
\begin{flushleft}
\item[\textbf{Declaração}\hfill]
\begin{ttfamily}
public Function SetSqlCustomBufDataset:Boolean; Virtual;\end{ttfamily}


\end{flushleft}
\par
\item[\textbf{Descrição}]
O método \textbf{\begin{ttfamily}SetSqlCustomBufDataset\end{ttfamily}} inicializa as propriedades SQLs de \begin{ttfamily}CustomBufDataset\end{ttfamily}(\ref{mi_ui_Dmxscroller_sql.TDmxScroller_sql_Atributos-CustomBufDataset})

\begin{itemize}
\item \textbf{PROPRIEDADES OBRIGATÓRIAS SEREM INICIALIZADAS:} \begin{itemize}
\item CustomBufDataset.SQL;
\end{itemize}
\item \textbf{PROPRIEDADES OPCIONAIS SEREM INICIALIZADAS:} \begin{itemize}
\item CustomBufDataset.InsertSQL;
\item CustomBufDataset.UpdataSQL;
\item CustomBufDataset.DeleteSQL;
\item CustomBufDataset.RefreshSQL;
\end{itemize}
\item \textbf{GERAÇÃO AUTOMÁTICA DE INSTRUÇÃO SQL DE ATUALIZAÇÃO} \begin{itemize}
\item O \textbf{SqlDb} (mais em particular, \textbf{TSQLQuery} ) pode gerar automaticamente instruções de atualização para os dados que busca. Para isso, ele irá varrer a instrução propriedade \textbf{CustomBufDataset.SQL} e determinar a tabela principal na consulta: esta é a \textbf{primeira tabela} encontrada na parte \textbf{FROM} da instrução \textbf{SELECT} . \begin{itemize}
\item Exemplo:

\texttt{\\\nopagebreak[3]
\\\nopagebreak[3]
SELECT~*~FROM~ALUNOS\\
}

\begin{itemize}
\item Alunos será a tabela selecionada para uso dos campos de https://www.freepascal.org/docs-html/fcl/db/tField.html (TField).
\end{itemize}
\end{itemize}
\item Para operações \textbf{INSERT} e \textbf{UPDATE}, a propriedade instrução SQL gerada inserirá e atualizará todos os campos que possuim \textbf{pfInUpdate} em sua propriedade \textbf{TField.ProviderFlags}. \begin{itemize}
\item Os campos somente leitura não serão adicionados à instrução SQL.
\item Os campos que são NULL não serão adicionados a uma consulta de inserção, o que significa que o servidor de banco de dados inserirá o que estiver na cláusula DEFAULT da definição de campo correspondente.
\end{itemize}
\item O campos de chave primária \begin{itemize}
\item Ao atualizar registros, \textbf{TSQLQuery} precisa saber quais campos compõem a chave primária que pode ser usada para atualizar o registro e quais campos devem ser atualizados: com base nessas informações, ele constrói os comandos \textbf{SQL UPDATE, INSERT ou DELETE}.
\item A construção da instrução \textbf{SQL} é controlada pela propriedade \textbf{UsePrimaryKeyAsKey} e pelas propriedades \textbf{ProviderFlags}.
\item A propriedade TField.ProviderFlag é um conjunto de 6 sinalizadores: \begin{itemize}
\item \textbf{pfInUpdate} : As alterações no campo devem ser propagadas para o banco de dados..
\item \textbf{pfInWhere} : O campo deve ser usado na cláusula WHERE de uma instrução de atualização no caso de upWhereChanged.
\item \textbf{pfInKey} : Campo é um campo chave e usado na cláusula WHERE de uma instrução de atualização.
\item \textbf{pfHidden} : O valor deste campo deve ser atualizado após a inserção.
\item \textbf{pfRefreshOnInsert} : O valor deste campo deve ser atualizado após a inserção.
\item \textbf{pfRefreshOnUpdate} : O valor deste campo deve ser atualizado após a atualização.
\end{itemize}
\end{itemize}
\end{itemize}
\item \textbf{REFERẼNCIAS} \begin{itemize}
\item [TSQLQuery Introdução](https://wiki.freepascal.org/Working{\_}With{\_}TSQLQuery{\#}General)
\item [TSQLQuery exemplos](https://wiki.freepascal.org/TSQLQuery)
\item https://www.freepascal.org/docs-html/fcl/sqldb/tsqlquery.html
\item https://wiki.freepascal.org/Working{\_}With{\_}TSQLQuery (Trabalhando com TSQLQuery);
\item https://www.freepascal.org/docs-html/fcl/sqldb/updatesqls.html (updatesqls.html);
\end{itemize}
\end{itemize}

\end{list}
\paragraph*{CreateCustomBufDataset{\_}FieldDefs}\hspace*{\fill}

\begin{list}{}{
\settowidth{\tmplength}{\textbf{Declaração}}
\setlength{\itemindent}{0cm}
\setlength{\listparindent}{0cm}
\setlength{\leftmargin}{\evensidemargin}
\addtolength{\leftmargin}{\tmplength}
\settowidth{\labelsep}{X}
\addtolength{\leftmargin}{\labelsep}
\setlength{\labelwidth}{\tmplength}
}
\begin{flushleft}
\item[\textbf{Declaração}\hfill]
\begin{ttfamily}
public Procedure CreateCustomBufDataset{\_}FieldDefs; override;\end{ttfamily}


\end{flushleft}
\par
\item[\textbf{Descrição}]
O método \textbf{\begin{ttfamily}CreateCustomBufDataset{\_}FieldDefs\end{ttfamily}} é usado para criar os campos de \textbf{\begin{ttfamily}CustomBufDataset\end{ttfamily}(\ref{mi_ui_Dmxscroller_sql.TDmxScroller_sql_Atributos-CustomBufDataset})}

\end{list}
\paragraph*{GetTemplate}\hspace*{\fill}

\begin{list}{}{
\settowidth{\tmplength}{\textbf{Declaração}}
\setlength{\itemindent}{0cm}
\setlength{\listparindent}{0cm}
\setlength{\leftmargin}{\evensidemargin}
\addtolength{\leftmargin}{\tmplength}
\settowidth{\labelsep}{X}
\addtolength{\leftmargin}{\labelsep}
\setlength{\labelwidth}{\tmplength}
}
\begin{flushleft}
\item[\textbf{Declaração}\hfill]
\begin{ttfamily}
public function GetTemplate(aNext: PSItem) : PSItem; overload; override;\end{ttfamily}


\end{flushleft}
\par
\item[\textbf{Descrição}]
O método \textbf{\begin{ttfamily}GetTemplate\end{ttfamily}} retorna uma lista de \textbf{\begin{ttfamily}PSItem\end{ttfamily}(\ref{mi_rtl_ui_Dmxscroller-PSItem})} (Lista de \begin{ttfamily}strings\end{ttfamily}(\ref{mi_rtl_ui_Dmxscroller.TUiDmxScroller-Strings})) com o modelo usado para criar a tela.

\begin{itemize}
\item \textbf{NOTA} \begin{itemize}
\item O Evento \begin{ttfamily}onGetTemplate\end{ttfamily}(\ref{mi_rtl_ui_Dmxscroller.TUiDmxScroller-onGetTemplate}) só é iniciado em tempo de execução, por isso o formulário não pode ser criado em tempo de desenho do aplicativo.
\item Caso o evento \begin{ttfamily}onGetTemplate\end{ttfamily}(\ref{mi_rtl_ui_Dmxscroller.TUiDmxScroller-onGetTemplate}) seja nil, então não posso ativar a tela.
\item Esse método pode ser anulado, caso se queira ignorar o evento \begin{ttfamily}onGetTemplate\end{ttfamily}(\ref{mi_rtl_ui_Dmxscroller.TUiDmxScroller-onGetTemplate}) e definir o Template em uma método pai herdado desta classe.
\end{itemize}
\end{itemize}

\end{list}
\paragraph*{GetBuffers}\hspace*{\fill}

\begin{list}{}{
\settowidth{\tmplength}{\textbf{Declaração}}
\setlength{\itemindent}{0cm}
\setlength{\listparindent}{0cm}
\setlength{\leftmargin}{\evensidemargin}
\addtolength{\leftmargin}{\tmplength}
\settowidth{\labelsep}{X}
\addtolength{\leftmargin}{\labelsep}
\setlength{\labelwidth}{\tmplength}
}
\begin{flushleft}
\item[\textbf{Declaração}\hfill]
\begin{ttfamily}
public function GetBuffers:Boolean; Override;\end{ttfamily}


\end{flushleft}
\par
\item[\textbf{Descrição}]
O método \textbf{\begin{ttfamily}GetBuffers\end{ttfamily}} ler o buffer dos campos dos arquivos associados a classe \textbf{\begin{ttfamily}TUiDmxScroller{\_}sql\end{ttfamily}(\ref{mi_ui_Dmxscroller_sql.TUiDmxScroller_sql})} para o buffer dos campos da classe \textbf{\begin{ttfamily}TUiDmxScroller\end{ttfamily}(\ref{mi_rtl_ui_Dmxscroller.TUiDmxScroller})}

\end{list}
\paragraph*{PutBuffers}\hspace*{\fill}

\begin{list}{}{
\settowidth{\tmplength}{\textbf{Declaração}}
\setlength{\itemindent}{0cm}
\setlength{\listparindent}{0cm}
\setlength{\leftmargin}{\evensidemargin}
\addtolength{\leftmargin}{\tmplength}
\settowidth{\labelsep}{X}
\addtolength{\leftmargin}{\labelsep}
\setlength{\labelwidth}{\tmplength}
}
\begin{flushleft}
\item[\textbf{Declaração}\hfill]
\begin{ttfamily}
public function PutBuffers:Boolean; override;\end{ttfamily}


\end{flushleft}
\par
\item[\textbf{Descrição}]
O método \textbf{\begin{ttfamily}PutBuffers\end{ttfamily}} grava o buffer dos campos da classe \textbf{\begin{ttfamily}TUiDmxScroller{\_}sql\end{ttfamily}(\ref{mi_ui_Dmxscroller_sql.TUiDmxScroller_sql})} para o buffer dos campos dos arquivos associados a classe \textbf{\begin{ttfamily}TUiDmxScroller{\_}sql\end{ttfamily}(\ref{mi_ui_Dmxscroller_sql.TUiDmxScroller_sql})}

\end{list}
\paragraph*{SetActiveLCL}\hspace*{\fill}

\begin{list}{}{
\settowidth{\tmplength}{\textbf{Declaração}}
\setlength{\itemindent}{0cm}
\setlength{\listparindent}{0cm}
\setlength{\leftmargin}{\evensidemargin}
\addtolength{\leftmargin}{\tmplength}
\settowidth{\labelsep}{X}
\addtolength{\leftmargin}{\labelsep}
\setlength{\labelwidth}{\tmplength}
}
\begin{flushleft}
\item[\textbf{Declaração}\hfill]
\begin{ttfamily}
public procedure SetActiveLCL(aActive: Boolean); override;\end{ttfamily}


\end{flushleft}
\end{list}
\paragraph*{DoOnNewRecord}\hspace*{\fill}

\begin{list}{}{
\settowidth{\tmplength}{\textbf{Declaração}}
\setlength{\itemindent}{0cm}
\setlength{\listparindent}{0cm}
\setlength{\leftmargin}{\evensidemargin}
\addtolength{\leftmargin}{\tmplength}
\settowidth{\labelsep}{X}
\addtolength{\leftmargin}{\labelsep}
\setlength{\labelwidth}{\tmplength}
}
\begin{flushleft}
\item[\textbf{Declaração}\hfill]
\begin{ttfamily}
public Procedure DoOnNewRecord; Override;\end{ttfamily}


\end{flushleft}
\par
\item[\textbf{Descrição}]
O método \textbf{\begin{ttfamily}DoOnNewRecord\end{ttfamily}} seleciona o registro para adição de um novo registro \begin{itemize}
\item NOTA \begin{itemize}
\item Está gerando exceção.?????
\end{itemize}
\end{itemize}

\end{list}
\paragraph*{DoAddRec}\hspace*{\fill}

\begin{list}{}{
\settowidth{\tmplength}{\textbf{Declaração}}
\setlength{\itemindent}{0cm}
\setlength{\listparindent}{0cm}
\setlength{\leftmargin}{\evensidemargin}
\addtolength{\leftmargin}{\tmplength}
\settowidth{\labelsep}{X}
\addtolength{\leftmargin}{\labelsep}
\setlength{\labelwidth}{\tmplength}
}
\begin{flushleft}
\item[\textbf{Declaração}\hfill]
\begin{ttfamily}
public Function DoAddRec:Boolean; override;\end{ttfamily}


\end{flushleft}
\par
\item[\textbf{Descrição}]
O método \textbf{\begin{ttfamily}DoAddRec\end{ttfamily}} adicione o registro editado no banco de dados. = \textbf{OBSERVAÇÂO} \begin{itemize}
\item O método \textbf{\begin{ttfamily}DoAddRec\end{ttfamily}} só funciona se o registro atender as seguintes condições: \begin{itemize}
\item \begin{ttfamily}appending\end{ttfamily}(\ref{mi_rtl_ui_Dmxscroller.TUiDmxScroller-Appending}) =true;
\item Mb{\_}St{\_}Insert habilidado
\item \begin{ttfamily}CustomBufDataset\end{ttfamily}(\ref{mi_ui_Dmxscroller_sql.TDmxScroller_sql_Atributos-CustomBufDataset}) {$<$}{$>$} nil
\item CustomBufDataset.Active = true;
\end{itemize}
\end{itemize}\begin{itemize}
\item \textbf{REFERÊNCIA} \begin{itemize}
\item [tsqlquery.options](https://www.freepascal.org/docs-html/fcl/sqldb/tsqlquery.options.html)
\end{itemize}
\end{itemize}

\end{list}
\chapter{Unit mi{\_}ui{\_}mi{\_}msgbox{\_}dm}
\section{Uses}
\begin{itemize}
\item \begin{ttfamily}Classes\end{ttfamily}\item \begin{ttfamily}SysUtils\end{ttfamily}\item \begin{ttfamily}Dialogs\end{ttfamily}\item \begin{ttfamily}Graphics\end{ttfamily}\item \begin{ttfamily}StdCtrls\end{ttfamily}\item \begin{ttfamily}System.UITypes\end{ttfamily}\item \begin{ttfamily}mi.rtl.objects.consts.mi{\_}msgbox\end{ttfamily}\end{itemize}
\section{Visão Geral}
\begin{description}
\item[\texttt{\begin{ttfamily}TMi{\_}ui{\_}mi{\_}msgBox\end{ttfamily} Classe}]
\end{description}
\section{Classes, Interfaces, Objetos e Registros}
\subsection*{TMi{\_}ui{\_}mi{\_}msgBox Classe}
\subsubsection*{\large{\textbf{Hierarquia}}\normalsize\hspace{1ex}\hfill}
TMi{\_}ui{\_}mi{\_}msgBox {$>$} TDataModule
\subsubsection*{\large{\textbf{Descrição}}\normalsize\hspace{1ex}\hfill}
\begin{ttfamily}TMi{\_}ui{\_}mi{\_}msgBox\end{ttfamily}

\begin{itemize}
\item EXEMPLO

\texttt{\\\nopagebreak[3]
\\\nopagebreak[3]
}\textbf{Var}\texttt{\\\nopagebreak[3]
~~S~:~}\textbf{String}\texttt{[10]~=~'';\\\nopagebreak[3]
}\textbf{begin}\texttt{\\\nopagebreak[3]
~~~}\textbf{if}\texttt{~MI{\_}MsgBox.InputBox('InputBox~Test','Gual~a~sua~indade?~',s,'ssssssssss')~=~Mrok\\\nopagebreak[3]
~~~}\textbf{then}\texttt{~ShowMessage('Sua~idade~é:~'s);\\
}
\end{itemize}\subsubsection*{\large{\textbf{Campos}}\normalsize\hspace{1ex}\hfill}
\paragraph*{MI{\_}MsgBox1}\hspace*{\fill}

\begin{list}{}{
\settowidth{\tmplength}{\textbf{Declaração}}
\setlength{\itemindent}{0cm}
\setlength{\listparindent}{0cm}
\setlength{\leftmargin}{\evensidemargin}
\addtolength{\leftmargin}{\tmplength}
\settowidth{\labelsep}{X}
\addtolength{\leftmargin}{\labelsep}
\setlength{\labelwidth}{\tmplength}
}
\begin{flushleft}
\item[\textbf{Declaração}\hfill]
\begin{ttfamily}
public MI{\_}MsgBox1: TMI{\_}MsgBox;\end{ttfamily}


\end{flushleft}
\end{list}
\subsubsection*{\large{\textbf{Métodos}}\normalsize\hspace{1ex}\hfill}
\paragraph*{MI{\_}MsgBox1InputBox}\hspace*{\fill}

\begin{list}{}{
\settowidth{\tmplength}{\textbf{Declaração}}
\setlength{\itemindent}{0cm}
\setlength{\listparindent}{0cm}
\setlength{\leftmargin}{\evensidemargin}
\addtolength{\leftmargin}{\tmplength}
\settowidth{\labelsep}{X}
\addtolength{\leftmargin}{\labelsep}
\setlength{\labelwidth}{\tmplength}
}
\begin{flushleft}
\item[\textbf{Declaração}\hfill]
\begin{ttfamily}
public function MI{\_}MsgBox1InputBox(const aTitle, ALabel: AnsiString; var Buff; Template: AnsiString): TModalResult;\end{ttfamily}


\end{flushleft}
\end{list}
\paragraph*{MI{\_}MsgBox1InputPassword}\hspace*{\fill}

\begin{list}{}{
\settowidth{\tmplength}{\textbf{Declaração}}
\setlength{\itemindent}{0cm}
\setlength{\listparindent}{0cm}
\setlength{\leftmargin}{\evensidemargin}
\addtolength{\leftmargin}{\tmplength}
\settowidth{\labelsep}{X}
\addtolength{\leftmargin}{\labelsep}
\setlength{\labelwidth}{\tmplength}
}
\begin{flushleft}
\item[\textbf{Declaração}\hfill]
\begin{ttfamily}
public function MI{\_}MsgBox1InputPassword(const aTitle: AnsiString; var aPassword: AnsiString): TModalResult;\end{ttfamily}


\end{flushleft}
\end{list}
\paragraph*{MI{\_}MsgBox1InputValue}\hspace*{\fill}

\begin{list}{}{
\settowidth{\tmplength}{\textbf{Declaração}}
\setlength{\itemindent}{0cm}
\setlength{\listparindent}{0cm}
\setlength{\leftmargin}{\evensidemargin}
\addtolength{\leftmargin}{\tmplength}
\settowidth{\labelsep}{X}
\addtolength{\leftmargin}{\labelsep}
\setlength{\labelwidth}{\tmplength}
}
\begin{flushleft}
\item[\textbf{Declaração}\hfill]
\begin{ttfamily}
public function MI{\_}MsgBox1InputValue(const aTitle, aLabel: AnsiString; var aValue: Variant): TModalResult;\end{ttfamily}


\end{flushleft}
\end{list}
\paragraph*{MI{\_}MsgBox1MessageBox}\hspace*{\fill}

\begin{list}{}{
\settowidth{\tmplength}{\textbf{Declaração}}
\setlength{\itemindent}{0cm}
\setlength{\listparindent}{0cm}
\setlength{\leftmargin}{\evensidemargin}
\addtolength{\leftmargin}{\tmplength}
\settowidth{\labelsep}{X}
\addtolength{\leftmargin}{\labelsep}
\setlength{\labelwidth}{\tmplength}
}
\begin{flushleft}
\item[\textbf{Declaração}\hfill]
\begin{ttfamily}
public function MI{\_}MsgBox1MessageBox(const aMsg: AnsiString): TModalResult;\end{ttfamily}


\end{flushleft}
\end{list}
\paragraph*{MI{\_}MsgBox1MessageBox{\_}03}\hspace*{\fill}

\begin{list}{}{
\settowidth{\tmplength}{\textbf{Declaração}}
\setlength{\itemindent}{0cm}
\setlength{\listparindent}{0cm}
\setlength{\leftmargin}{\evensidemargin}
\addtolength{\leftmargin}{\tmplength}
\settowidth{\labelsep}{X}
\addtolength{\leftmargin}{\labelsep}
\setlength{\labelwidth}{\tmplength}
}
\begin{flushleft}
\item[\textbf{Declaração}\hfill]
\begin{ttfamily}
protected function MI{\_}MsgBox1MessageBox{\_}03(const aMsg: AnsiString;DlgType: TMsgDlgType; Buttons: TMsgDlgButtons): TModalResult;\end{ttfamily}


\end{flushleft}
\end{list}
\paragraph*{MI{\_}MsgBox1MessageBox{\_}04}\hspace*{\fill}

\begin{list}{}{
\settowidth{\tmplength}{\textbf{Declaração}}
\setlength{\itemindent}{0cm}
\setlength{\listparindent}{0cm}
\setlength{\leftmargin}{\evensidemargin}
\addtolength{\leftmargin}{\tmplength}
\settowidth{\labelsep}{X}
\addtolength{\leftmargin}{\labelsep}
\setlength{\labelwidth}{\tmplength}
}
\begin{flushleft}
\item[\textbf{Declaração}\hfill]
\begin{ttfamily}
protected function MI{\_}MsgBox1MessageBox{\_}04(aMsg: AnsiString; DlgType: TMsgDlgType; Buttons: TMsgDlgButtons; ButtonDefault: TMsgDlgBtn): TModalResult;\end{ttfamily}


\end{flushleft}
\end{list}
\paragraph*{MI{\_}MsgBox1MessageBox{\_}04{\_}PSItem}\hspace*{\fill}

\begin{list}{}{
\settowidth{\tmplength}{\textbf{Declaração}}
\setlength{\itemindent}{0cm}
\setlength{\listparindent}{0cm}
\setlength{\leftmargin}{\evensidemargin}
\addtolength{\leftmargin}{\tmplength}
\settowidth{\labelsep}{X}
\addtolength{\leftmargin}{\labelsep}
\setlength{\labelwidth}{\tmplength}
}
\begin{flushleft}
\item[\textbf{Declaração}\hfill]
\begin{ttfamily}
protected function MI{\_}MsgBox1MessageBox{\_}04{\_}PSItem(aPSItem: TMI{\_}MsgBoxTypes.PSItem; DlgType: TMsgDlgType; Buttons: TMsgDlgButtons; ButtonDefault: TMsgDlgBtn ): TModalResult;\end{ttfamily}


\end{flushleft}
\end{list}
\paragraph*{MI{\_}MsgBox1MessageBox{\_}05}\hspace*{\fill}

\begin{list}{}{
\settowidth{\tmplength}{\textbf{Declaração}}
\setlength{\itemindent}{0cm}
\setlength{\listparindent}{0cm}
\setlength{\leftmargin}{\evensidemargin}
\addtolength{\leftmargin}{\tmplength}
\settowidth{\labelsep}{X}
\addtolength{\leftmargin}{\labelsep}
\setlength{\labelwidth}{\tmplength}
}
\begin{flushleft}
\item[\textbf{Declaração}\hfill]
\begin{ttfamily}
protected function MI{\_}MsgBox1MessageBox{\_}05(ATitle: AnsiString; aMsg: AnsiString; DlgType: TMsgDlgType; Buttons: TMsgDlgButtons; ButtonDefault: TMsgDlgBtn ): TModalResult;\end{ttfamily}


\end{flushleft}
\end{list}
\paragraph*{MI{\_}MsgBox1MessageBox{\_}ListBoxRec{\_}PSItem}\hspace*{\fill}

\begin{list}{}{
\settowidth{\tmplength}{\textbf{Declaração}}
\setlength{\itemindent}{0cm}
\setlength{\listparindent}{0cm}
\setlength{\leftmargin}{\evensidemargin}
\addtolength{\leftmargin}{\tmplength}
\settowidth{\labelsep}{X}
\addtolength{\leftmargin}{\labelsep}
\setlength{\labelwidth}{\tmplength}
}
\begin{flushleft}
\item[\textbf{Declaração}\hfill]
\begin{ttfamily}
protected function MI{\_}MsgBox1MessageBox{\_}ListBoxRec{\_}PSItem(Atitulo: AnsiString; APSItem: TMI{\_}MsgBoxTypes.PSItem; itemSelection: longint; DlgType: TMsgDlgType; Buttons: TMsgDlgButtons; ButtonDefault: TMsgDlgBtn ): TModalResult;\end{ttfamily}


\end{flushleft}
\end{list}
\paragraph*{create}\hspace*{\fill}

\begin{list}{}{
\settowidth{\tmplength}{\textbf{Declaração}}
\setlength{\itemindent}{0cm}
\setlength{\listparindent}{0cm}
\setlength{\leftmargin}{\evensidemargin}
\addtolength{\leftmargin}{\tmplength}
\settowidth{\labelsep}{X}
\addtolength{\leftmargin}{\labelsep}
\setlength{\labelwidth}{\tmplength}
}
\begin{flushleft}
\item[\textbf{Declaração}\hfill]
\begin{ttfamily}
public constructor create(aOwner:TComponent); Override;\end{ttfamily}


\end{flushleft}
\end{list}
\section{Variáveis}
\subsection*{Mi{\_}ui{\_}mi{\_}msgBox}
\begin{list}{}{
\settowidth{\tmplength}{\textbf{Declaração}}
\setlength{\itemindent}{0cm}
\setlength{\listparindent}{0cm}
\setlength{\leftmargin}{\evensidemargin}
\addtolength{\leftmargin}{\tmplength}
\settowidth{\labelsep}{X}
\addtolength{\leftmargin}{\labelsep}
\setlength{\labelwidth}{\tmplength}
}
\begin{flushleft}
\item[\textbf{Declaração}\hfill]
\begin{ttfamily}
Mi{\_}ui{\_}mi{\_}msgBox: TMi{\_}ui{\_}mi{\_}msgBox;\end{ttfamily}


\end{flushleft}
\end{list}
\chapter{Program project1}
\section{Uses}
\begin{itemize}
\item \begin{ttfamily}Interfaces\end{ttfamily}\item \begin{ttfamily}Forms\end{ttfamily}\item \begin{ttfamily}Unit1\end{ttfamily}(\ref{Unit1})\end{itemize}
\chapter{Unit testForm}
\section{Descrição}
A unit \textbf{\begin{ttfamily}testForm\end{ttfamily}} implementa o formulário \begin{ttfamily}TMI{\_}UI{\_}InputBox\end{ttfamily}(\ref{uMI_UI_InputBox.TMI_UI_InputBox}) usado para criar formulário baseado em Template \begin{ttfamily}PSITem\end{ttfamily}(\ref{mi_rtl_ui_Dmxscroller-PSItem}).

\begin{itemize}
\item \textbf{VERSÃO} \begin{itemize}
\item Alpha {-} 0.5.0.693
\end{itemize}
\item \textbf{CÓDIGO FONTE}: \begin{itemize}
\item 
\end{itemize}
\item \textbf{PENDÊNCIAS} \begin{itemize}
\item T12 A classe Mi{\_}ScrollBox{\_}LCL1 deve ser criada em tempo de execução para quenão tenha problema na instalação. ✅️
\item T12 A a classe DmxScroller{\_}Form{\_}Lcl1 deve ser criada em tempo de execução para que não tenha problema na instalação. ✅️
\item T12 A a classe ButtonPanel1 deve ser criada em tempo de execução para que não tenha problema na instalação.
\item T12 A propriedade autosize deve ser true após o form for criado.
\end{itemize}
\end{itemize}\begin{itemize}
\item \textbf{HISTÓRICO} \begin{itemize}
\item Criado por: Paulo Sérgio da Silva Pacheco paulosspacheco@yahoo.com.br)
\item \textbf{2022{-}05{-}17} \begin{itemize}
\item T12 Análise de como será a classe \begin{ttfamily}TMI{\_}UI{\_}InputBox\end{ttfamily}(\ref{uMI_UI_InputBox.TMI_UI_InputBox}). ✅️
\item T12 Criar a unit \textbf{\begin{ttfamily}testForm\end{ttfamily}}. ✅️
\item T12 Criar formulário \begin{ttfamily}TMI{\_}UI{\_}InputBox\end{ttfamily}(\ref{uMI_UI_InputBox.TMI_UI_InputBox}); ✅️
\item T12 Adicionar o componente ButtonPanel1 e habilitar os botões ok e cancel; ✅️
\item T12 Adicionar o componente Mi{\_}ScrollBox{\_}LCL1 ; ✅️
\item T12 Criar evento: function DmxScroller{\_}Form{\_}Lcl1GetTemplate; ✅️
\item T12 Criar atributo protected {\_}FormSItem : \begin{ttfamily}PSitem\end{ttfamily}(\ref{mi_rtl_ui_Dmxscroller-PSItem}); ✅️
\item T12 Criar propriedade Template:AnsiString; ✅️ \begin{itemize}
\item Criar método Set{\_}Template(aTemplate:AnsiString); ✅️
\end{itemize}
\end{itemize}
\item \textbf{2022{-}05{-}18} \begin{itemize}
\item \textbf{10:51} \begin{itemize}
\item As alterações que fiz ontem no método \begin{ttfamily}TObjectsMethods.StringToSItem\end{ttfamily}(\ref{mi.rtl.Objects.Methods.TObjectsMethods-StringToSItem})() criou efeito colateral. \begin{itemize}
\item Corrigido. ✅️
\end{itemize}
\end{itemize}
\item \textbf{14:28} \begin{itemize}
\item Criar função: \begin{itemize}
\item function \begin{ttfamily}InputBox\end{ttfamily}(\ref{uMI_UI_InputBox-InputBox})(): TModalResult;
\end{itemize}
\end{itemize}
\end{itemize}
\item \textbf{2022{-}05{-}19} \begin{itemize}
\item \textbf{11:13} \begin{itemize}
\item Criar os eventos \begin{itemize}
\item OnEnterLocal ✅️
\item OnExistLocal ✅️
\item onEnterFieldLocal ✅️
\item OnExitFieldLocal ✅️
\end{itemize}
\item Criar função: \begin{itemize}
\item function MI{\_}MsgBox1MessageBox{\_}04{\_}PSItem(aPSItem: TMI{\_}MsgBoxTypes.PSItem; DlgType: TMsgDlgType; Buttons: TMsgDlgButtons; ButtonDefault: TMsgDlgBtn ): TModalResult;
\end{itemize}
\end{itemize}
\end{itemize}
\item \textbf{2022{-}06{-}27} \begin{itemize}
\item \textbf{09:30} \begin{itemize}
\item T12 A classe Mi{\_}ScrollBox{\_}LCL1 deve ser criada em tempo de execução para quenão tenha problema na instalação. ✅️.
\end{itemize}
\item \textbf{10:25} \begin{itemize}
\item T12 A a classe DmxScroller{\_}Form{\_}Lcl1 deve ser criada em tempo de execução para que não tenha problema na instalação. ✅️
\end{itemize}
\item \textbf{10:41} \begin{itemize}
\item T12 A a classe ButtonPanel1 deve ser criada em tempo de execução para que não tenha problema na instalação.
\end{itemize}
\end{itemize}
\item \textbf{2022{-}06{-}28} \begin{itemize}
\item \textbf{15:52} \begin{itemize}
\item Criar unit \textbf{\begin{ttfamily}umi{\_}ui{\_}inputbox{\_}lcl{\_}test\end{ttfamily}(\ref{umi_ui_inputbox_lcl_test})} para desmostrar o uso de \begin{ttfamily}InputBox\end{ttfamily}(\ref{uMI_UI_InputBox-InputBox}).
\end{itemize}
\end{itemize}
\end{itemize}
\end{itemize}
\section{Uses}
\begin{itemize}
\item \begin{ttfamily}Classes\end{ttfamily}\item \begin{ttfamily}SysUtils\end{ttfamily}\item \begin{ttfamily}Forms\end{ttfamily}\end{itemize}
\section{Visão Geral}
\begin{description}
\item[\texttt{ShowFormF}]
\end{description}
\section{Funções e Procedimentos}
\subsection*{ShowFormF}
\begin{list}{}{
\settowidth{\tmplength}{\textbf{Declaração}}
\setlength{\itemindent}{0cm}
\setlength{\listparindent}{0cm}
\setlength{\leftmargin}{\evensidemargin}
\addtolength{\leftmargin}{\tmplength}
\settowidth{\labelsep}{X}
\addtolength{\leftmargin}{\labelsep}
\setlength{\labelwidth}{\tmplength}
}
\begin{flushleft}
\item[\textbf{Declaração}\hfill]
\begin{ttfamily}
Procedure ShowFormF;\end{ttfamily}


\end{flushleft}
\end{list}
\section{Variáveis}
\subsection*{f}
\begin{list}{}{
\settowidth{\tmplength}{\textbf{Declaração}}
\setlength{\itemindent}{0cm}
\setlength{\listparindent}{0cm}
\setlength{\leftmargin}{\evensidemargin}
\addtolength{\leftmargin}{\tmplength}
\settowidth{\labelsep}{X}
\addtolength{\leftmargin}{\labelsep}
\setlength{\labelwidth}{\tmplength}
}
\begin{flushleft}
\item[\textbf{Declaração}\hfill]
\begin{ttfamily}
f:Tform;\end{ttfamily}


\end{flushleft}
\end{list}
\chapter{Unit uDmxScroller{\_}Form{\_}Lcl{\_}add{\_}test}
\section{Descrição}
A unit \textbf{\begin{ttfamily}uDmxScroller{\_}Form{\_}Lcl{\_}add{\_}test\end{ttfamily}} implementa a classe TTForm1 do pacote mi.ui.
\section{Uses}
\begin{itemize}
\item \begin{ttfamily}Classes\end{ttfamily}\item \begin{ttfamily}SysUtils\end{ttfamily}\item \begin{ttfamily}Forms\end{ttfamily}\item \begin{ttfamily}Controls\end{ttfamily}\item \begin{ttfamily}Graphics\end{ttfamily}\item \begin{ttfamily}Dialogs\end{ttfamily}\item \begin{ttfamily}LMessages\end{ttfamily}\item \begin{ttfamily}StdCtrls\end{ttfamily}\item \begin{ttfamily}DBCtrls\end{ttfamily}\item \begin{ttfamily}ExtCtrls\end{ttfamily}\item \begin{ttfamily}ButtonPanel\end{ttfamily}\item \begin{ttfamily}uMi{\_}ui{\_}DmxScroller{\_}Form{\_}Lcl\end{ttfamily}\item \begin{ttfamily}uMi{\_}ui{\_}scrollbox{\_}lcl\end{ttfamily}(\ref{uMi_ui_scrollbox_lcl})\item \begin{ttfamily}mi{\_}rtl{\_}ui{\_}Dmxscroller\end{ttfamily}(\ref{mi_rtl_ui_Dmxscroller})\end{itemize}
\section{Visão Geral}
\begin{description}
\item[\texttt{\begin{ttfamily}TDmxScroller{\_}Form{\_}Lcl{\_}add{\_}test\end{ttfamily} Classe}]
\end{description}
\section{Classes, Interfaces, Objetos e Registros}
\subsection*{TDmxScroller{\_}Form{\_}Lcl{\_}add{\_}test Classe}
\subsubsection*{\large{\textbf{Hierarquia}}\normalsize\hspace{1ex}\hfill}
TDmxScroller{\_}Form{\_}Lcl{\_}add{\_}test {$>$} TForm
%%%%Descrição
\subsubsection*{\large{\textbf{Campos}}\normalsize\hspace{1ex}\hfill}
\paragraph*{ButtonPanel1}\hspace*{\fill}

\begin{list}{}{
\settowidth{\tmplength}{\textbf{Declaração}}
\setlength{\itemindent}{0cm}
\setlength{\listparindent}{0cm}
\setlength{\leftmargin}{\evensidemargin}
\addtolength{\leftmargin}{\tmplength}
\settowidth{\labelsep}{X}
\addtolength{\leftmargin}{\labelsep}
\setlength{\labelwidth}{\tmplength}
}
\begin{flushleft}
\item[\textbf{Declaração}\hfill]
\begin{ttfamily}
public ButtonPanel1: TButtonPanel;\end{ttfamily}


\end{flushleft}
\end{list}
\paragraph*{DmxScroller{\_}Form{\_}Lcl1}\hspace*{\fill}

\begin{list}{}{
\settowidth{\tmplength}{\textbf{Declaração}}
\setlength{\itemindent}{0cm}
\setlength{\listparindent}{0cm}
\setlength{\leftmargin}{\evensidemargin}
\addtolength{\leftmargin}{\tmplength}
\settowidth{\labelsep}{X}
\addtolength{\leftmargin}{\labelsep}
\setlength{\labelwidth}{\tmplength}
}
\begin{flushleft}
\item[\textbf{Declaração}\hfill]
\begin{ttfamily}
public DmxScroller{\_}Form{\_}Lcl1: TDmxScroller{\_}Form{\_}Lcl;\end{ttfamily}


\end{flushleft}
\end{list}
\paragraph*{Mi{\_}ScrollBox{\_}LCL1}\hspace*{\fill}

\begin{list}{}{
\settowidth{\tmplength}{\textbf{Declaração}}
\setlength{\itemindent}{0cm}
\setlength{\listparindent}{0cm}
\setlength{\leftmargin}{\evensidemargin}
\addtolength{\leftmargin}{\tmplength}
\settowidth{\labelsep}{X}
\addtolength{\leftmargin}{\labelsep}
\setlength{\labelwidth}{\tmplength}
}
\begin{flushleft}
\item[\textbf{Declaração}\hfill]
\begin{ttfamily}
public Mi{\_}ScrollBox{\_}LCL1: TMi{\_}ScrollBox{\_}LCL;\end{ttfamily}


\end{flushleft}
\end{list}
\subsubsection*{\large{\textbf{Métodos}}\normalsize\hspace{1ex}\hfill}
\paragraph*{ButtonPanel1Click}\hspace*{\fill}

\begin{list}{}{
\settowidth{\tmplength}{\textbf{Declaração}}
\setlength{\itemindent}{0cm}
\setlength{\listparindent}{0cm}
\setlength{\leftmargin}{\evensidemargin}
\addtolength{\leftmargin}{\tmplength}
\settowidth{\labelsep}{X}
\addtolength{\leftmargin}{\labelsep}
\setlength{\labelwidth}{\tmplength}
}
\begin{flushleft}
\item[\textbf{Declaração}\hfill]
\begin{ttfamily}
public procedure ButtonPanel1Click(Sender: TObject);\end{ttfamily}


\end{flushleft}
\end{list}
\paragraph*{DmxScroller{\_}Form{\_}Lcl1AddTemplate}\hspace*{\fill}

\begin{list}{}{
\settowidth{\tmplength}{\textbf{Declaração}}
\setlength{\itemindent}{0cm}
\setlength{\listparindent}{0cm}
\setlength{\leftmargin}{\evensidemargin}
\addtolength{\leftmargin}{\tmplength}
\settowidth{\labelsep}{X}
\addtolength{\leftmargin}{\labelsep}
\setlength{\labelwidth}{\tmplength}
}
\begin{flushleft}
\item[\textbf{Declaração}\hfill]
\begin{ttfamily}
public procedure DmxScroller{\_}Form{\_}Lcl1AddTemplate(const aUiDmxScroller: TUiDmxScroller );\end{ttfamily}


\end{flushleft}
\end{list}
\paragraph*{DmxScroller{\_}Form{\_}Lcl1Enter}\hspace*{\fill}

\begin{list}{}{
\settowidth{\tmplength}{\textbf{Declaração}}
\setlength{\itemindent}{0cm}
\setlength{\listparindent}{0cm}
\setlength{\leftmargin}{\evensidemargin}
\addtolength{\leftmargin}{\tmplength}
\settowidth{\labelsep}{X}
\addtolength{\leftmargin}{\labelsep}
\setlength{\labelwidth}{\tmplength}
}
\begin{flushleft}
\item[\textbf{Declaração}\hfill]
\begin{ttfamily}
public procedure DmxScroller{\_}Form{\_}Lcl1Enter(aDmxScroller: TUiDmxScroller);\end{ttfamily}


\end{flushleft}
\end{list}
\paragraph*{DmxScroller{\_}Form{\_}Lcl1EnterField}\hspace*{\fill}

\begin{list}{}{
\settowidth{\tmplength}{\textbf{Declaração}}
\setlength{\itemindent}{0cm}
\setlength{\listparindent}{0cm}
\setlength{\leftmargin}{\evensidemargin}
\addtolength{\leftmargin}{\tmplength}
\settowidth{\labelsep}{X}
\addtolength{\leftmargin}{\labelsep}
\setlength{\labelwidth}{\tmplength}
}
\begin{flushleft}
\item[\textbf{Declaração}\hfill]
\begin{ttfamily}
public procedure DmxScroller{\_}Form{\_}Lcl1EnterField(aField: pDmxFieldRec);\end{ttfamily}


\end{flushleft}
\end{list}
\paragraph*{FormCreate}\hspace*{\fill}

\begin{list}{}{
\settowidth{\tmplength}{\textbf{Declaração}}
\setlength{\itemindent}{0cm}
\setlength{\listparindent}{0cm}
\setlength{\leftmargin}{\evensidemargin}
\addtolength{\leftmargin}{\tmplength}
\settowidth{\labelsep}{X}
\addtolength{\leftmargin}{\labelsep}
\setlength{\labelwidth}{\tmplength}
}
\begin{flushleft}
\item[\textbf{Declaração}\hfill]
\begin{ttfamily}
public procedure FormCreate(Sender: TObject);\end{ttfamily}


\end{flushleft}
\end{list}
\paragraph*{destroy}\hspace*{\fill}

\begin{list}{}{
\settowidth{\tmplength}{\textbf{Declaração}}
\setlength{\itemindent}{0cm}
\setlength{\listparindent}{0cm}
\setlength{\leftmargin}{\evensidemargin}
\addtolength{\leftmargin}{\tmplength}
\settowidth{\labelsep}{X}
\addtolength{\leftmargin}{\labelsep}
\setlength{\labelwidth}{\tmplength}
}
\begin{flushleft}
\item[\textbf{Declaração}\hfill]
\begin{ttfamily}
public destructor destroy; override;\end{ttfamily}


\end{flushleft}
\end{list}
\section{Variáveis}
\subsection*{DmxScroller{\_}Form{\_}Lcl{\_}add{\_}test}
\begin{list}{}{
\settowidth{\tmplength}{\textbf{Declaração}}
\setlength{\itemindent}{0cm}
\setlength{\listparindent}{0cm}
\setlength{\leftmargin}{\evensidemargin}
\addtolength{\leftmargin}{\tmplength}
\settowidth{\labelsep}{X}
\addtolength{\leftmargin}{\labelsep}
\setlength{\labelwidth}{\tmplength}
}
\begin{flushleft}
\item[\textbf{Declaração}\hfill]
\begin{ttfamily}
DmxScroller{\_}Form{\_}Lcl{\_}add{\_}test: TDmxScroller{\_}Form{\_}Lcl{\_}add{\_}test;\end{ttfamily}


\end{flushleft}
\end{list}
\chapter{Unit uDmxScroller{\_}Form{\_}Lcl{\_}add{\_}test2}
\section{Uses}
\begin{itemize}
\item \begin{ttfamily}Classes\end{ttfamily}\item \begin{ttfamily}SysUtils\end{ttfamily}\item \begin{ttfamily}Forms\end{ttfamily}\item \begin{ttfamily}Controls\end{ttfamily}\item \begin{ttfamily}Graphics\end{ttfamily}\item \begin{ttfamily}Dialogs\end{ttfamily}\item \begin{ttfamily}ButtonPanel\end{ttfamily}\item \begin{ttfamily}uMi{\_}ui{\_}scrollbox{\_}lcl\end{ttfamily}(\ref{uMi_ui_scrollbox_lcl})\item \begin{ttfamily}uMi{\_}ui{\_}DmxScroller{\_}Form{\_}Lcl\end{ttfamily}\item \begin{ttfamily}mi{\_}rtl{\_}ui{\_}Dmxscroller\end{ttfamily}(\ref{mi_rtl_ui_Dmxscroller})\end{itemize}
\section{Visão Geral}
\begin{description}
\item[\texttt{\begin{ttfamily}TDmxScroller{\_}Form{\_}Lcl{\_}add{\_}test2\end{ttfamily} Classe}]
\end{description}
\section{Classes, Interfaces, Objetos e Registros}
\subsection*{TDmxScroller{\_}Form{\_}Lcl{\_}add{\_}test2 Classe}
\subsubsection*{\large{\textbf{Hierarquia}}\normalsize\hspace{1ex}\hfill}
TDmxScroller{\_}Form{\_}Lcl{\_}add{\_}test2 {$>$} TForm
%%%%Descrição
\subsubsection*{\large{\textbf{Campos}}\normalsize\hspace{1ex}\hfill}
\paragraph*{ButtonPanel1}\hspace*{\fill}

\begin{list}{}{
\settowidth{\tmplength}{\textbf{Declaração}}
\setlength{\itemindent}{0cm}
\setlength{\listparindent}{0cm}
\setlength{\leftmargin}{\evensidemargin}
\addtolength{\leftmargin}{\tmplength}
\settowidth{\labelsep}{X}
\addtolength{\leftmargin}{\labelsep}
\setlength{\labelwidth}{\tmplength}
}
\begin{flushleft}
\item[\textbf{Declaração}\hfill]
\begin{ttfamily}
public ButtonPanel1: TButtonPanel;\end{ttfamily}


\end{flushleft}
\end{list}
\paragraph*{DmxScroller{\_}Form{\_}Lcl1}\hspace*{\fill}

\begin{list}{}{
\settowidth{\tmplength}{\textbf{Declaração}}
\setlength{\itemindent}{0cm}
\setlength{\listparindent}{0cm}
\setlength{\leftmargin}{\evensidemargin}
\addtolength{\leftmargin}{\tmplength}
\settowidth{\labelsep}{X}
\addtolength{\leftmargin}{\labelsep}
\setlength{\labelwidth}{\tmplength}
}
\begin{flushleft}
\item[\textbf{Declaração}\hfill]
\begin{ttfamily}
public DmxScroller{\_}Form{\_}Lcl1: TDmxScroller{\_}Form{\_}Lcl;\end{ttfamily}


\end{flushleft}
\end{list}
\paragraph*{Mi{\_}ScrollBox{\_}LCL1}\hspace*{\fill}

\begin{list}{}{
\settowidth{\tmplength}{\textbf{Declaração}}
\setlength{\itemindent}{0cm}
\setlength{\listparindent}{0cm}
\setlength{\leftmargin}{\evensidemargin}
\addtolength{\leftmargin}{\tmplength}
\settowidth{\labelsep}{X}
\addtolength{\leftmargin}{\labelsep}
\setlength{\labelwidth}{\tmplength}
}
\begin{flushleft}
\item[\textbf{Declaração}\hfill]
\begin{ttfamily}
public Mi{\_}ScrollBox{\_}LCL1: TMi{\_}ScrollBox{\_}LCL;\end{ttfamily}


\end{flushleft}
\end{list}
\subsubsection*{\large{\textbf{Métodos}}\normalsize\hspace{1ex}\hfill}
\paragraph*{DmxScroller{\_}Form{\_}Lcl1GetTemplate}\hspace*{\fill}

\begin{list}{}{
\settowidth{\tmplength}{\textbf{Declaração}}
\setlength{\itemindent}{0cm}
\setlength{\listparindent}{0cm}
\setlength{\leftmargin}{\evensidemargin}
\addtolength{\leftmargin}{\tmplength}
\settowidth{\labelsep}{X}
\addtolength{\leftmargin}{\labelsep}
\setlength{\labelwidth}{\tmplength}
}
\begin{flushleft}
\item[\textbf{Declaração}\hfill]
\begin{ttfamily}
public function DmxScroller{\_}Form{\_}Lcl1GetTemplate(aNext: PSItem): PSItem;\end{ttfamily}


\end{flushleft}
\end{list}
\paragraph*{FormCreate}\hspace*{\fill}

\begin{list}{}{
\settowidth{\tmplength}{\textbf{Declaração}}
\setlength{\itemindent}{0cm}
\setlength{\listparindent}{0cm}
\setlength{\leftmargin}{\evensidemargin}
\addtolength{\leftmargin}{\tmplength}
\settowidth{\labelsep}{X}
\addtolength{\leftmargin}{\labelsep}
\setlength{\labelwidth}{\tmplength}
}
\begin{flushleft}
\item[\textbf{Declaração}\hfill]
\begin{ttfamily}
public procedure FormCreate(Sender: TObject);\end{ttfamily}


\end{flushleft}
\end{list}
\section{Variáveis}
\subsection*{DmxScroller{\_}Form{\_}Lcl{\_}add{\_}test2}
\begin{list}{}{
\settowidth{\tmplength}{\textbf{Declaração}}
\setlength{\itemindent}{0cm}
\setlength{\listparindent}{0cm}
\setlength{\leftmargin}{\evensidemargin}
\addtolength{\leftmargin}{\tmplength}
\settowidth{\labelsep}{X}
\addtolength{\leftmargin}{\labelsep}
\setlength{\labelwidth}{\tmplength}
}
\begin{flushleft}
\item[\textbf{Declaração}\hfill]
\begin{ttfamily}
DmxScroller{\_}Form{\_}Lcl{\_}add{\_}test2: TDmxScroller{\_}Form{\_}Lcl{\_}add{\_}test2;\end{ttfamily}


\end{flushleft}
\end{list}
\chapter{Unit uDmxScroller{\_}Form{\_}Lcl{\_}test}
\section{Descrição}
A unit \textbf{\begin{ttfamily}uDmxScroller{\_}Form{\_}Lcl{\_}test\end{ttfamily}} implementa o teste dos componentes TUiConsts.MI{\_}MsgBox, mi{\_}scrollbox{\_}LCL1 e TDmxScroller{\_}Form{\_}Lcl onde os mesmos são ligados no evento \textbf{\begin{ttfamily}TDmxScroller{\_}Form{\_}Lcl{\_}test.FormCreate\end{ttfamily}(\ref{uDmxScroller_Form_Lcl_test.TDmxScroller_Form_Lcl_test-FormCreate})}

\begin{itemize}
\item \textbf{NOTAS} \begin{itemize}
\item A constante \textbf{TUiConsts.MI{\_}MsgBox} precisa se iniciada com o atributo \textbf{\begin{ttfamily}TMi{\_}ui{\_}mi{\_}msgBox.MI{\_}MsgBox1\end{ttfamily}(\ref{mi_ui_mi_msgbox_dm.TMi_ui_mi_msgBox-MI_MsgBox1})} da unit \textbf{\begin{ttfamily}mi{\_}ui{\_}mi{\_}msgbox{\_}dm\end{ttfamily}(\ref{mi_ui_mi_msgbox_dm})} para que os diálogos internos do componente \textbf{DmxScroller{\_}Form{\_}Lcl1} possa gerar mensagens sem depender diretamente da \textbf{LCL}, ou seja: Será possível implementar dialogs em outros frameworks visuais tais como \textbf{html}, \textbf{angula 4}, etc alterando o método SetActive(). \begin{itemize}
\item O método SetActive seleciona os método DmxScroller{\_}Form{\_}Lcl1.CreateFormLCL ou o método DmxScroller{\_}Form{\_}Lcl1.CreateFormHTML conforme o tipo de aplicação.
\end{itemize}
\item O evento DmxScroller{\_}Form{\_}Lcl1.onGetTemplate precisa se iniciado em OnCreate do form porque a propriedade \textbf{onGetTemplate} ainda não foi lida do arquivo de recursos e precisamos da mesma para executar o método DmxScroller{\_}Form{\_}Lcl1.SetParentLcl.
\end{itemize}
\item \textbf{CÓDIGO PASCAL}

\texttt{\\\nopagebreak[3]
\\\nopagebreak[3]
}\textbf{procedure}\texttt{~TForm{\_}Mi{\_}Ui{\_}Test.FormCreate(Sender:~TObject);\\\nopagebreak[3]
}\textbf{begin}\texttt{\\\nopagebreak[3]
~~~TUiConsts.MI{\_}MsgBox~:=~get{\_}MI{\_}MsgBox.MI{\_}MsgBox1;\\\nopagebreak[3]
~~~DmxScroller{\_}Form{\_}Lcl1.onGetTemplate:=~DmxScroller{\_}Form{\_}Lcl1GetTemplate;\\\nopagebreak[3]
~~~DmxScroller{\_}Form{\_}Lcl1.SetParentLcl(mi{\_}scrollbox{\_}LCL1);\\\nopagebreak[3]
}\textbf{end}\texttt{;\\
}
\end{itemize}
\section{Uses}
\begin{itemize}
\item \begin{ttfamily}Classes\end{ttfamily}\item \begin{ttfamily}SysUtils\end{ttfamily}\item \begin{ttfamily}DB\end{ttfamily}\item \begin{ttfamily}BufDataset\end{ttfamily}\item \begin{ttfamily}memds\end{ttfamily}\item \begin{ttfamily}Forms\end{ttfamily}\item \begin{ttfamily}Controls\end{ttfamily}\item \begin{ttfamily}Graphics\end{ttfamily}\item \begin{ttfamily}Dialogs\end{ttfamily}\item \begin{ttfamily}typInfo\end{ttfamily}\item \begin{ttfamily}MaskEdit\end{ttfamily}\item \begin{ttfamily}StdCtrls\end{ttfamily}\item \begin{ttfamily}ExtCtrls\end{ttfamily}\item \begin{ttfamily}DBGrids\end{ttfamily}\item \begin{ttfamily}ButtonPanel\end{ttfamily}\item \begin{ttfamily}ActnList\end{ttfamily}\item \begin{ttfamily}DBCtrls\end{ttfamily}\item \begin{ttfamily}Spin\end{ttfamily}\item \begin{ttfamily}Buttons\end{ttfamily}\item \begin{ttfamily}DBExtCtrls\end{ttfamily}\item \begin{ttfamily}EditBtn\end{ttfamily}\item \begin{ttfamily}SpinEx\end{ttfamily}\item \begin{ttfamily}SynEdit\end{ttfamily}\item \begin{ttfamily}TAChartExtentLink\end{ttfamily}\item \begin{ttfamily}SQLite3Conn\end{ttfamily}\item \begin{ttfamily}SqlDb\end{ttfamily}\item \begin{ttfamily}mi.rtl.Types\end{ttfamily}(\ref{mi.rtl.Types})\item \begin{ttfamily}mi{\_}rtl{\_}ui{\_}consts\end{ttfamily}\item \begin{ttfamily}mi{\_}rtl{\_}ui{\_}Dmxscroller\end{ttfamily}(\ref{mi_rtl_ui_Dmxscroller})\item \begin{ttfamily}uMi{\_}ui{\_}scrollbox{\_}lcl\end{ttfamily}(\ref{uMi_ui_scrollbox_lcl})\item \begin{ttfamily}uMi{\_}Ui{\_}DbComboBox{\_}lcl\end{ttfamily}(\ref{uMi_Ui_DbComboBox_lcl})\item \begin{ttfamily}uMI{\_}ui{\_}DbEdit{\_}LCL\end{ttfamily}(\ref{uMI_ui_DbEdit_LCL})\item \begin{ttfamily}uMi{\_}ui{\_}maskedit{\_}lcl\end{ttfamily}(\ref{uMi_ui_maskedit_lcl})\item \begin{ttfamily}uMi{\_}ui{\_}ComboBox{\_}LCL\end{ttfamily}(\ref{uMi_ui_ComboBox_lcl})\item \begin{ttfamily}uMi{\_}BitBtn{\_}LCL\end{ttfamily}\item \begin{ttfamily}uMi{\_}ui{\_}DmxScroller{\_}Form{\_}Lcl\end{ttfamily}\item \begin{ttfamily}uMi{\_}ui{\_}mi{\_}msgbox{\_}dm\end{ttfamily}(\ref{umi_ui_mi_msgbox_dm})\item \begin{ttfamily}uMi{\_}ui{\_}DmxScroller{\_}Form{\_}Lcl{\_}DS{\_}Test\end{ttfamily}(\ref{uMi_ui_DmxScroller_Form_Lcl_ds_test})\item \begin{ttfamily}umi{\_}ui{\_}InputBox{\_}lcl\end{ttfamily}(\ref{umi_ui_InputBox_lcl})\item \begin{ttfamily}uDmxScroller{\_}Form{\_}Lcl{\_}add{\_}test\end{ttfamily}(\ref{uDmxScroller_Form_Lcl_add_test})\item \begin{ttfamily}uMi{\_}ui{\_}DmxScroller{\_}Form{\_}Lcl{\_}ds{\_}test2\end{ttfamily}(\ref{uMi_ui_DmxScroller_Form_Lcl_ds_test2})\item \begin{ttfamily}umi{\_}ui{\_}inputbox{\_}lcl{\_}test\end{ttfamily}(\ref{umi_ui_inputbox_lcl_test})\item \begin{ttfamily}uDmxScroller{\_}Form{\_}Lcl{\_}add{\_}test2\end{ttfamily}(\ref{uDmxScroller_Form_Lcl_add_test2})\end{itemize}
\section{Visão Geral}
\begin{description}
\item[\texttt{\begin{ttfamily}TDmxScroller{\_}Form{\_}Lcl{\_}test\end{ttfamily} Classe}]
\end{description}
\section{Classes, Interfaces, Objetos e Registros}
\subsection*{TDmxScroller{\_}Form{\_}Lcl{\_}test Classe}
\subsubsection*{\large{\textbf{Hierarquia}}\normalsize\hspace{1ex}\hfill}
TDmxScroller{\_}Form{\_}Lcl{\_}test {$>$} TForm
%%%%Descrição
\subsubsection*{\large{\textbf{Campos}}\normalsize\hspace{1ex}\hfill}
\paragraph*{Form{\_}ds{\_}test2}\hspace*{\fill}

\begin{list}{}{
\settowidth{\tmplength}{\textbf{Declaração}}
\setlength{\itemindent}{0cm}
\setlength{\listparindent}{0cm}
\setlength{\leftmargin}{\evensidemargin}
\addtolength{\leftmargin}{\tmplength}
\settowidth{\labelsep}{X}
\addtolength{\leftmargin}{\labelsep}
\setlength{\labelwidth}{\tmplength}
}
\begin{flushleft}
\item[\textbf{Declaração}\hfill]
\begin{ttfamily}
public Form{\_}ds{\_}test2: TAction;\end{ttfamily}


\end{flushleft}
\end{list}
\paragraph*{Action{\_}Form{\_}ds{\_}test}\hspace*{\fill}

\begin{list}{}{
\settowidth{\tmplength}{\textbf{Declaração}}
\setlength{\itemindent}{0cm}
\setlength{\listparindent}{0cm}
\setlength{\leftmargin}{\evensidemargin}
\addtolength{\leftmargin}{\tmplength}
\settowidth{\labelsep}{X}
\addtolength{\leftmargin}{\labelsep}
\setlength{\labelwidth}{\tmplength}
}
\begin{flushleft}
\item[\textbf{Declaração}\hfill]
\begin{ttfamily}
public Action{\_}Form{\_}ds{\_}test: TAction;\end{ttfamily}


\end{flushleft}
\end{list}
\paragraph*{AddTemplate}\hspace*{\fill}

\begin{list}{}{
\settowidth{\tmplength}{\textbf{Declaração}}
\setlength{\itemindent}{0cm}
\setlength{\listparindent}{0cm}
\setlength{\leftmargin}{\evensidemargin}
\addtolength{\leftmargin}{\tmplength}
\settowidth{\labelsep}{X}
\addtolength{\leftmargin}{\labelsep}
\setlength{\labelwidth}{\tmplength}
}
\begin{flushleft}
\item[\textbf{Declaração}\hfill]
\begin{ttfamily}
public AddTemplate: TButton;\end{ttfamily}


\end{flushleft}
\end{list}
\paragraph*{Button1}\hspace*{\fill}

\begin{list}{}{
\settowidth{\tmplength}{\textbf{Declaração}}
\setlength{\itemindent}{0cm}
\setlength{\listparindent}{0cm}
\setlength{\leftmargin}{\evensidemargin}
\addtolength{\leftmargin}{\tmplength}
\settowidth{\labelsep}{X}
\addtolength{\leftmargin}{\labelsep}
\setlength{\labelwidth}{\tmplength}
}
\begin{flushleft}
\item[\textbf{Declaração}\hfill]
\begin{ttfamily}
public Button1: TButton;\end{ttfamily}


\end{flushleft}
\end{list}
\paragraph*{GetTemplate}\hspace*{\fill}

\begin{list}{}{
\settowidth{\tmplength}{\textbf{Declaração}}
\setlength{\itemindent}{0cm}
\setlength{\listparindent}{0cm}
\setlength{\leftmargin}{\evensidemargin}
\addtolength{\leftmargin}{\tmplength}
\settowidth{\labelsep}{X}
\addtolength{\leftmargin}{\labelsep}
\setlength{\labelwidth}{\tmplength}
}
\begin{flushleft}
\item[\textbf{Declaração}\hfill]
\begin{ttfamily}
public GetTemplate: TButton;\end{ttfamily}


\end{flushleft}
\end{list}
\paragraph*{Novo}\hspace*{\fill}

\begin{list}{}{
\settowidth{\tmplength}{\textbf{Declaração}}
\setlength{\itemindent}{0cm}
\setlength{\listparindent}{0cm}
\setlength{\leftmargin}{\evensidemargin}
\addtolength{\leftmargin}{\tmplength}
\settowidth{\labelsep}{X}
\addtolength{\leftmargin}{\labelsep}
\setlength{\labelwidth}{\tmplength}
}
\begin{flushleft}
\item[\textbf{Declaração}\hfill]
\begin{ttfamily}
public Novo: TAction;\end{ttfamily}


\end{flushleft}
\end{list}
\paragraph*{Gravar}\hspace*{\fill}

\begin{list}{}{
\settowidth{\tmplength}{\textbf{Declaração}}
\setlength{\itemindent}{0cm}
\setlength{\listparindent}{0cm}
\setlength{\leftmargin}{\evensidemargin}
\addtolength{\leftmargin}{\tmplength}
\settowidth{\labelsep}{X}
\addtolength{\leftmargin}{\labelsep}
\setlength{\labelwidth}{\tmplength}
}
\begin{flushleft}
\item[\textbf{Declaração}\hfill]
\begin{ttfamily}
public Gravar: TAction;\end{ttfamily}


\end{flushleft}
\end{list}
\paragraph*{Excluir}\hspace*{\fill}

\begin{list}{}{
\settowidth{\tmplength}{\textbf{Declaração}}
\setlength{\itemindent}{0cm}
\setlength{\listparindent}{0cm}
\setlength{\leftmargin}{\evensidemargin}
\addtolength{\leftmargin}{\tmplength}
\settowidth{\labelsep}{X}
\addtolength{\leftmargin}{\labelsep}
\setlength{\labelwidth}{\tmplength}
}
\begin{flushleft}
\item[\textbf{Declaração}\hfill]
\begin{ttfamily}
public Excluir: TAction;\end{ttfamily}


\end{flushleft}
\end{list}
\paragraph*{Pesquisar}\hspace*{\fill}

\begin{list}{}{
\settowidth{\tmplength}{\textbf{Declaração}}
\setlength{\itemindent}{0cm}
\setlength{\listparindent}{0cm}
\setlength{\leftmargin}{\evensidemargin}
\addtolength{\leftmargin}{\tmplength}
\settowidth{\labelsep}{X}
\addtolength{\leftmargin}{\labelsep}
\setlength{\labelwidth}{\tmplength}
}
\begin{flushleft}
\item[\textbf{Declaração}\hfill]
\begin{ttfamily}
public Pesquisar: TAction;\end{ttfamily}


\end{flushleft}
\end{list}
\paragraph*{Pesquisa}\hspace*{\fill}

\begin{list}{}{
\settowidth{\tmplength}{\textbf{Declaração}}
\setlength{\itemindent}{0cm}
\setlength{\listparindent}{0cm}
\setlength{\leftmargin}{\evensidemargin}
\addtolength{\leftmargin}{\tmplength}
\settowidth{\labelsep}{X}
\addtolength{\leftmargin}{\labelsep}
\setlength{\labelwidth}{\tmplength}
}
\begin{flushleft}
\item[\textbf{Declaração}\hfill]
\begin{ttfamily}
public Pesquisa: TAction;\end{ttfamily}


\end{flushleft}
\end{list}
\paragraph*{ActionList1}\hspace*{\fill}

\begin{list}{}{
\settowidth{\tmplength}{\textbf{Declaração}}
\setlength{\itemindent}{0cm}
\setlength{\listparindent}{0cm}
\setlength{\leftmargin}{\evensidemargin}
\addtolength{\leftmargin}{\tmplength}
\settowidth{\labelsep}{X}
\addtolength{\leftmargin}{\labelsep}
\setlength{\labelwidth}{\tmplength}
}
\begin{flushleft}
\item[\textbf{Declaração}\hfill]
\begin{ttfamily}
public ActionList1: TActionList;\end{ttfamily}


\end{flushleft}
\end{list}
\paragraph*{Button{\_}ModifyFontsAll{\_}LCL}\hspace*{\fill}

\begin{list}{}{
\settowidth{\tmplength}{\textbf{Declaração}}
\setlength{\itemindent}{0cm}
\setlength{\listparindent}{0cm}
\setlength{\leftmargin}{\evensidemargin}
\addtolength{\leftmargin}{\tmplength}
\settowidth{\labelsep}{X}
\addtolength{\leftmargin}{\labelsep}
\setlength{\labelwidth}{\tmplength}
}
\begin{flushleft}
\item[\textbf{Declaração}\hfill]
\begin{ttfamily}
public Button{\_}ModifyFontsAll{\_}LCL: TButton;\end{ttfamily}


\end{flushleft}
\end{list}
\paragraph*{InputBox}\hspace*{\fill}

\begin{list}{}{
\settowidth{\tmplength}{\textbf{Declaração}}
\setlength{\itemindent}{0cm}
\setlength{\listparindent}{0cm}
\setlength{\leftmargin}{\evensidemargin}
\addtolength{\leftmargin}{\tmplength}
\settowidth{\labelsep}{X}
\addtolength{\leftmargin}{\labelsep}
\setlength{\labelwidth}{\tmplength}
}
\begin{flushleft}
\item[\textbf{Declaração}\hfill]
\begin{ttfamily}
public InputBox: TButton;\end{ttfamily}


\end{flushleft}
\end{list}
\paragraph*{form{\_}ds{\_}Test}\hspace*{\fill}

\begin{list}{}{
\settowidth{\tmplength}{\textbf{Declaração}}
\setlength{\itemindent}{0cm}
\setlength{\listparindent}{0cm}
\setlength{\leftmargin}{\evensidemargin}
\addtolength{\leftmargin}{\tmplength}
\settowidth{\labelsep}{X}
\addtolength{\leftmargin}{\labelsep}
\setlength{\labelwidth}{\tmplength}
}
\begin{flushleft}
\item[\textbf{Declaração}\hfill]
\begin{ttfamily}
public form{\_}ds{\_}Test: TButton;\end{ttfamily}


\end{flushleft}
\end{list}
\paragraph*{Button{\_}Cidades}\hspace*{\fill}

\begin{list}{}{
\settowidth{\tmplength}{\textbf{Declaração}}
\setlength{\itemindent}{0cm}
\setlength{\listparindent}{0cm}
\setlength{\leftmargin}{\evensidemargin}
\addtolength{\leftmargin}{\tmplength}
\settowidth{\labelsep}{X}
\addtolength{\leftmargin}{\labelsep}
\setlength{\labelwidth}{\tmplength}
}
\begin{flushleft}
\item[\textbf{Declaração}\hfill]
\begin{ttfamily}
public Button{\_}Cidades: TButton;\end{ttfamily}


\end{flushleft}
\end{list}
\paragraph*{ButtonPanel1}\hspace*{\fill}

\begin{list}{}{
\settowidth{\tmplength}{\textbf{Declaração}}
\setlength{\itemindent}{0cm}
\setlength{\listparindent}{0cm}
\setlength{\leftmargin}{\evensidemargin}
\addtolength{\leftmargin}{\tmplength}
\settowidth{\labelsep}{X}
\addtolength{\leftmargin}{\labelsep}
\setlength{\labelwidth}{\tmplength}
}
\begin{flushleft}
\item[\textbf{Declaração}\hfill]
\begin{ttfamily}
public ButtonPanel1: TButtonPanel;\end{ttfamily}


\end{flushleft}
\end{list}
\paragraph*{DmxScroller{\_}Form{\_}Lcl1}\hspace*{\fill}

\begin{list}{}{
\settowidth{\tmplength}{\textbf{Declaração}}
\setlength{\itemindent}{0cm}
\setlength{\listparindent}{0cm}
\setlength{\leftmargin}{\evensidemargin}
\addtolength{\leftmargin}{\tmplength}
\settowidth{\labelsep}{X}
\addtolength{\leftmargin}{\labelsep}
\setlength{\labelwidth}{\tmplength}
}
\begin{flushleft}
\item[\textbf{Declaração}\hfill]
\begin{ttfamily}
public DmxScroller{\_}Form{\_}Lcl1: TDmxScroller{\_}Form{\_}Lcl;\end{ttfamily}


\end{flushleft}
\end{list}
\paragraph*{GroupBox1}\hspace*{\fill}

\begin{list}{}{
\settowidth{\tmplength}{\textbf{Declaração}}
\setlength{\itemindent}{0cm}
\setlength{\listparindent}{0cm}
\setlength{\leftmargin}{\evensidemargin}
\addtolength{\leftmargin}{\tmplength}
\settowidth{\labelsep}{X}
\addtolength{\leftmargin}{\labelsep}
\setlength{\labelwidth}{\tmplength}
}
\begin{flushleft}
\item[\textbf{Declaração}\hfill]
\begin{ttfamily}
public GroupBox1: TGroupBox;\end{ttfamily}


\end{flushleft}
\end{list}
\paragraph*{Mi{\_}ScrollBox{\_}LCL1}\hspace*{\fill}

\begin{list}{}{
\settowidth{\tmplength}{\textbf{Declaração}}
\setlength{\itemindent}{0cm}
\setlength{\listparindent}{0cm}
\setlength{\leftmargin}{\evensidemargin}
\addtolength{\leftmargin}{\tmplength}
\settowidth{\labelsep}{X}
\addtolength{\leftmargin}{\labelsep}
\setlength{\labelwidth}{\tmplength}
}
\begin{flushleft}
\item[\textbf{Declaração}\hfill]
\begin{ttfamily}
public Mi{\_}ScrollBox{\_}LCL1: TMi{\_}ScrollBox{\_}LCL;\end{ttfamily}


\end{flushleft}
\end{list}
\paragraph*{Panel1}\hspace*{\fill}

\begin{list}{}{
\settowidth{\tmplength}{\textbf{Declaração}}
\setlength{\itemindent}{0cm}
\setlength{\listparindent}{0cm}
\setlength{\leftmargin}{\evensidemargin}
\addtolength{\leftmargin}{\tmplength}
\settowidth{\labelsep}{X}
\addtolength{\leftmargin}{\labelsep}
\setlength{\labelwidth}{\tmplength}
}
\begin{flushleft}
\item[\textbf{Declaração}\hfill]
\begin{ttfamily}
public Panel1: TPanel;\end{ttfamily}


\end{flushleft}
\end{list}
\paragraph*{StaticText1}\hspace*{\fill}

\begin{list}{}{
\settowidth{\tmplength}{\textbf{Declaração}}
\setlength{\itemindent}{0cm}
\setlength{\listparindent}{0cm}
\setlength{\leftmargin}{\evensidemargin}
\addtolength{\leftmargin}{\tmplength}
\settowidth{\labelsep}{X}
\addtolength{\leftmargin}{\labelsep}
\setlength{\labelwidth}{\tmplength}
}
\begin{flushleft}
\item[\textbf{Declaração}\hfill]
\begin{ttfamily}
public StaticText1: TStaticText;\end{ttfamily}


\end{flushleft}
\end{list}
\subsubsection*{\large{\textbf{Métodos}}\normalsize\hspace{1ex}\hfill}
\paragraph*{Form{\_}ds{\_}test2Execute}\hspace*{\fill}

\begin{list}{}{
\settowidth{\tmplength}{\textbf{Declaração}}
\setlength{\itemindent}{0cm}
\setlength{\listparindent}{0cm}
\setlength{\leftmargin}{\evensidemargin}
\addtolength{\leftmargin}{\tmplength}
\settowidth{\labelsep}{X}
\addtolength{\leftmargin}{\labelsep}
\setlength{\labelwidth}{\tmplength}
}
\begin{flushleft}
\item[\textbf{Declaração}\hfill]
\begin{ttfamily}
public procedure Form{\_}ds{\_}test2Execute(Sender: TObject);\end{ttfamily}


\end{flushleft}
\end{list}
\paragraph*{Action{\_}Form{\_}ds{\_}testExecute}\hspace*{\fill}

\begin{list}{}{
\settowidth{\tmplength}{\textbf{Declaração}}
\setlength{\itemindent}{0cm}
\setlength{\listparindent}{0cm}
\setlength{\leftmargin}{\evensidemargin}
\addtolength{\leftmargin}{\tmplength}
\settowidth{\labelsep}{X}
\addtolength{\leftmargin}{\labelsep}
\setlength{\labelwidth}{\tmplength}
}
\begin{flushleft}
\item[\textbf{Declaração}\hfill]
\begin{ttfamily}
public procedure Action{\_}Form{\_}ds{\_}testExecute(Sender: TObject);\end{ttfamily}


\end{flushleft}
\end{list}
\paragraph*{AddTemplateClick}\hspace*{\fill}

\begin{list}{}{
\settowidth{\tmplength}{\textbf{Declaração}}
\setlength{\itemindent}{0cm}
\setlength{\listparindent}{0cm}
\setlength{\leftmargin}{\evensidemargin}
\addtolength{\leftmargin}{\tmplength}
\settowidth{\labelsep}{X}
\addtolength{\leftmargin}{\labelsep}
\setlength{\labelwidth}{\tmplength}
}
\begin{flushleft}
\item[\textbf{Declaração}\hfill]
\begin{ttfamily}
public procedure AddTemplateClick(Sender: TObject);\end{ttfamily}


\end{flushleft}
\end{list}
\paragraph*{GetTemplateClick}\hspace*{\fill}

\begin{list}{}{
\settowidth{\tmplength}{\textbf{Declaração}}
\setlength{\itemindent}{0cm}
\setlength{\listparindent}{0cm}
\setlength{\leftmargin}{\evensidemargin}
\addtolength{\leftmargin}{\tmplength}
\settowidth{\labelsep}{X}
\addtolength{\leftmargin}{\labelsep}
\setlength{\labelwidth}{\tmplength}
}
\begin{flushleft}
\item[\textbf{Declaração}\hfill]
\begin{ttfamily}
public procedure GetTemplateClick(Sender: TObject);\end{ttfamily}


\end{flushleft}
\end{list}
\paragraph*{Button{\_}ModifyFontsAll{\_}LCLClick}\hspace*{\fill}

\begin{list}{}{
\settowidth{\tmplength}{\textbf{Declaração}}
\setlength{\itemindent}{0cm}
\setlength{\listparindent}{0cm}
\setlength{\leftmargin}{\evensidemargin}
\addtolength{\leftmargin}{\tmplength}
\settowidth{\labelsep}{X}
\addtolength{\leftmargin}{\labelsep}
\setlength{\labelwidth}{\tmplength}
}
\begin{flushleft}
\item[\textbf{Declaração}\hfill]
\begin{ttfamily}
public procedure Button{\_}ModifyFontsAll{\_}LCLClick(Sender: TObject);\end{ttfamily}


\end{flushleft}
\end{list}
\paragraph*{form{\_}ds{\_}TestClick}\hspace*{\fill}

\begin{list}{}{
\settowidth{\tmplength}{\textbf{Declaração}}
\setlength{\itemindent}{0cm}
\setlength{\listparindent}{0cm}
\setlength{\leftmargin}{\evensidemargin}
\addtolength{\leftmargin}{\tmplength}
\settowidth{\labelsep}{X}
\addtolength{\leftmargin}{\labelsep}
\setlength{\labelwidth}{\tmplength}
}
\begin{flushleft}
\item[\textbf{Declaração}\hfill]
\begin{ttfamily}
public procedure form{\_}ds{\_}TestClick(Sender: TObject);\end{ttfamily}


\end{flushleft}
\end{list}
\paragraph*{InputBoxClick}\hspace*{\fill}

\begin{list}{}{
\settowidth{\tmplength}{\textbf{Declaração}}
\setlength{\itemindent}{0cm}
\setlength{\listparindent}{0cm}
\setlength{\leftmargin}{\evensidemargin}
\addtolength{\leftmargin}{\tmplength}
\settowidth{\labelsep}{X}
\addtolength{\leftmargin}{\labelsep}
\setlength{\labelwidth}{\tmplength}
}
\begin{flushleft}
\item[\textbf{Declaração}\hfill]
\begin{ttfamily}
public procedure InputBoxClick(Sender: TObject);\end{ttfamily}


\end{flushleft}
\par
\item[\textbf{Descrição}]
O método \textbf{\begin{ttfamily}InputBoxClick\end{ttfamily}} demonstra o uso da função MsgBox{\_}Form

\end{list}
\paragraph*{DmxScroller{\_}Form{\_}Lcl1Enter}\hspace*{\fill}

\begin{list}{}{
\settowidth{\tmplength}{\textbf{Declaração}}
\setlength{\itemindent}{0cm}
\setlength{\listparindent}{0cm}
\setlength{\leftmargin}{\evensidemargin}
\addtolength{\leftmargin}{\tmplength}
\settowidth{\labelsep}{X}
\addtolength{\leftmargin}{\labelsep}
\setlength{\labelwidth}{\tmplength}
}
\begin{flushleft}
\item[\textbf{Declaração}\hfill]
\begin{ttfamily}
public procedure DmxScroller{\_}Form{\_}Lcl1Enter(aDmxScroller: TUiDmxScroller);\end{ttfamily}


\end{flushleft}
\end{list}
\paragraph*{DmxScroller{\_}Form{\_}Lcl1EnterField}\hspace*{\fill}

\begin{list}{}{
\settowidth{\tmplength}{\textbf{Declaração}}
\setlength{\itemindent}{0cm}
\setlength{\listparindent}{0cm}
\setlength{\leftmargin}{\evensidemargin}
\addtolength{\leftmargin}{\tmplength}
\settowidth{\labelsep}{X}
\addtolength{\leftmargin}{\labelsep}
\setlength{\labelwidth}{\tmplength}
}
\begin{flushleft}
\item[\textbf{Declaração}\hfill]
\begin{ttfamily}
public procedure DmxScroller{\_}Form{\_}Lcl1EnterField(aField: pDmxFieldRec);\end{ttfamily}


\end{flushleft}
\end{list}
\paragraph*{DmxScroller{\_}Form{\_}Lcl1Exit}\hspace*{\fill}

\begin{list}{}{
\settowidth{\tmplength}{\textbf{Declaração}}
\setlength{\itemindent}{0cm}
\setlength{\listparindent}{0cm}
\setlength{\leftmargin}{\evensidemargin}
\addtolength{\leftmargin}{\tmplength}
\settowidth{\labelsep}{X}
\addtolength{\leftmargin}{\labelsep}
\setlength{\labelwidth}{\tmplength}
}
\begin{flushleft}
\item[\textbf{Declaração}\hfill]
\begin{ttfamily}
public procedure DmxScroller{\_}Form{\_}Lcl1Exit(aDmxScroller: TUiDmxScroller);\end{ttfamily}


\end{flushleft}
\end{list}
\paragraph*{DmxScroller{\_}Form{\_}Lcl1ExitField}\hspace*{\fill}

\begin{list}{}{
\settowidth{\tmplength}{\textbf{Declaração}}
\setlength{\itemindent}{0cm}
\setlength{\listparindent}{0cm}
\setlength{\leftmargin}{\evensidemargin}
\addtolength{\leftmargin}{\tmplength}
\settowidth{\labelsep}{X}
\addtolength{\leftmargin}{\labelsep}
\setlength{\labelwidth}{\tmplength}
}
\begin{flushleft}
\item[\textbf{Declaração}\hfill]
\begin{ttfamily}
public procedure DmxScroller{\_}Form{\_}Lcl1ExitField(aField: pDmxFieldRec);\end{ttfamily}


\end{flushleft}
\end{list}
\paragraph*{DmxScroller{\_}Form{\_}Lcl1GetTemplate}\hspace*{\fill}

\begin{list}{}{
\settowidth{\tmplength}{\textbf{Declaração}}
\setlength{\itemindent}{0cm}
\setlength{\listparindent}{0cm}
\setlength{\leftmargin}{\evensidemargin}
\addtolength{\leftmargin}{\tmplength}
\settowidth{\labelsep}{X}
\addtolength{\leftmargin}{\labelsep}
\setlength{\labelwidth}{\tmplength}
}
\begin{flushleft}
\item[\textbf{Declaração}\hfill]
\begin{ttfamily}
public function DmxScroller{\_}Form{\_}Lcl1GetTemplate(aNext: PSItem): PSItem;\end{ttfamily}


\end{flushleft}
\end{list}
\paragraph*{DmxScroller{\_}Form{\_}Lcl1NewRecord}\hspace*{\fill}

\begin{list}{}{
\settowidth{\tmplength}{\textbf{Declaração}}
\setlength{\itemindent}{0cm}
\setlength{\listparindent}{0cm}
\setlength{\leftmargin}{\evensidemargin}
\addtolength{\leftmargin}{\tmplength}
\settowidth{\labelsep}{X}
\addtolength{\leftmargin}{\labelsep}
\setlength{\labelwidth}{\tmplength}
}
\begin{flushleft}
\item[\textbf{Declaração}\hfill]
\begin{ttfamily}
public procedure DmxScroller{\_}Form{\_}Lcl1NewRecord(aDmxScroller: TUiDmxScroller);\end{ttfamily}


\end{flushleft}
\end{list}
\paragraph*{ExcluirExecute}\hspace*{\fill}

\begin{list}{}{
\settowidth{\tmplength}{\textbf{Declaração}}
\setlength{\itemindent}{0cm}
\setlength{\listparindent}{0cm}
\setlength{\leftmargin}{\evensidemargin}
\addtolength{\leftmargin}{\tmplength}
\settowidth{\labelsep}{X}
\addtolength{\leftmargin}{\labelsep}
\setlength{\labelwidth}{\tmplength}
}
\begin{flushleft}
\item[\textbf{Declaração}\hfill]
\begin{ttfamily}
public procedure ExcluirExecute(Sender: TObject);\end{ttfamily}


\end{flushleft}
\end{list}
\paragraph*{FormClose}\hspace*{\fill}

\begin{list}{}{
\settowidth{\tmplength}{\textbf{Declaração}}
\setlength{\itemindent}{0cm}
\setlength{\listparindent}{0cm}
\setlength{\leftmargin}{\evensidemargin}
\addtolength{\leftmargin}{\tmplength}
\settowidth{\labelsep}{X}
\addtolength{\leftmargin}{\labelsep}
\setlength{\labelwidth}{\tmplength}
}
\begin{flushleft}
\item[\textbf{Declaração}\hfill]
\begin{ttfamily}
public procedure FormClose(Sender: TObject; var CloseAction: TCloseAction);\end{ttfamily}


\end{flushleft}
\end{list}
\paragraph*{FormCreate}\hspace*{\fill}

\begin{list}{}{
\settowidth{\tmplength}{\textbf{Declaração}}
\setlength{\itemindent}{0cm}
\setlength{\listparindent}{0cm}
\setlength{\leftmargin}{\evensidemargin}
\addtolength{\leftmargin}{\tmplength}
\settowidth{\labelsep}{X}
\addtolength{\leftmargin}{\labelsep}
\setlength{\labelwidth}{\tmplength}
}
\begin{flushleft}
\item[\textbf{Declaração}\hfill]
\begin{ttfamily}
public procedure FormCreate(Sender: TObject);\end{ttfamily}


\end{flushleft}
\end{list}
\paragraph*{GravarExecute}\hspace*{\fill}

\begin{list}{}{
\settowidth{\tmplength}{\textbf{Declaração}}
\setlength{\itemindent}{0cm}
\setlength{\listparindent}{0cm}
\setlength{\leftmargin}{\evensidemargin}
\addtolength{\leftmargin}{\tmplength}
\settowidth{\labelsep}{X}
\addtolength{\leftmargin}{\labelsep}
\setlength{\labelwidth}{\tmplength}
}
\begin{flushleft}
\item[\textbf{Declaração}\hfill]
\begin{ttfamily}
public procedure GravarExecute(Sender: TObject);\end{ttfamily}


\end{flushleft}
\end{list}
\paragraph*{mi{\_}scrollbox{\_}LCL1Enter}\hspace*{\fill}

\begin{list}{}{
\settowidth{\tmplength}{\textbf{Declaração}}
\setlength{\itemindent}{0cm}
\setlength{\listparindent}{0cm}
\setlength{\leftmargin}{\evensidemargin}
\addtolength{\leftmargin}{\tmplength}
\settowidth{\labelsep}{X}
\addtolength{\leftmargin}{\labelsep}
\setlength{\labelwidth}{\tmplength}
}
\begin{flushleft}
\item[\textbf{Declaração}\hfill]
\begin{ttfamily}
public procedure mi{\_}scrollbox{\_}LCL1Enter(Sender: TObject);\end{ttfamily}


\end{flushleft}
\end{list}
\paragraph*{NovoExecute}\hspace*{\fill}

\begin{list}{}{
\settowidth{\tmplength}{\textbf{Declaração}}
\setlength{\itemindent}{0cm}
\setlength{\listparindent}{0cm}
\setlength{\leftmargin}{\evensidemargin}
\addtolength{\leftmargin}{\tmplength}
\settowidth{\labelsep}{X}
\addtolength{\leftmargin}{\labelsep}
\setlength{\labelwidth}{\tmplength}
}
\begin{flushleft}
\item[\textbf{Declaração}\hfill]
\begin{ttfamily}
public procedure NovoExecute(Sender: TObject);\end{ttfamily}


\end{flushleft}
\end{list}
\paragraph*{PesquisarExecute}\hspace*{\fill}

\begin{list}{}{
\settowidth{\tmplength}{\textbf{Declaração}}
\setlength{\itemindent}{0cm}
\setlength{\listparindent}{0cm}
\setlength{\leftmargin}{\evensidemargin}
\addtolength{\leftmargin}{\tmplength}
\settowidth{\labelsep}{X}
\addtolength{\leftmargin}{\labelsep}
\setlength{\labelwidth}{\tmplength}
}
\begin{flushleft}
\item[\textbf{Declaração}\hfill]
\begin{ttfamily}
public procedure PesquisarExecute(Sender: TObject);\end{ttfamily}


\end{flushleft}
\end{list}
\paragraph*{PesquisaExecute}\hspace*{\fill}

\begin{list}{}{
\settowidth{\tmplength}{\textbf{Declaração}}
\setlength{\itemindent}{0cm}
\setlength{\listparindent}{0cm}
\setlength{\leftmargin}{\evensidemargin}
\addtolength{\leftmargin}{\tmplength}
\settowidth{\labelsep}{X}
\addtolength{\leftmargin}{\labelsep}
\setlength{\labelwidth}{\tmplength}
}
\begin{flushleft}
\item[\textbf{Declaração}\hfill]
\begin{ttfamily}
public procedure PesquisaExecute(Sender: TObject);\end{ttfamily}


\end{flushleft}
\end{list}
\paragraph*{StaticText1Click}\hspace*{\fill}

\begin{list}{}{
\settowidth{\tmplength}{\textbf{Declaração}}
\setlength{\itemindent}{0cm}
\setlength{\listparindent}{0cm}
\setlength{\leftmargin}{\evensidemargin}
\addtolength{\leftmargin}{\tmplength}
\settowidth{\labelsep}{X}
\addtolength{\leftmargin}{\labelsep}
\setlength{\labelwidth}{\tmplength}
}
\begin{flushleft}
\item[\textbf{Declaração}\hfill]
\begin{ttfamily}
public procedure StaticText1Click(Sender: TObject);\end{ttfamily}


\end{flushleft}
\end{list}
\paragraph*{StaticText2Click}\hspace*{\fill}

\begin{list}{}{
\settowidth{\tmplength}{\textbf{Declaração}}
\setlength{\itemindent}{0cm}
\setlength{\listparindent}{0cm}
\setlength{\leftmargin}{\evensidemargin}
\addtolength{\leftmargin}{\tmplength}
\settowidth{\labelsep}{X}
\addtolength{\leftmargin}{\labelsep}
\setlength{\labelwidth}{\tmplength}
}
\begin{flushleft}
\item[\textbf{Declaração}\hfill]
\begin{ttfamily}
public procedure StaticText2Click(Sender: TObject);\end{ttfamily}


\end{flushleft}
\end{list}
\paragraph*{destroy}\hspace*{\fill}

\begin{list}{}{
\settowidth{\tmplength}{\textbf{Declaração}}
\setlength{\itemindent}{0cm}
\setlength{\listparindent}{0cm}
\setlength{\leftmargin}{\evensidemargin}
\addtolength{\leftmargin}{\tmplength}
\settowidth{\labelsep}{X}
\addtolength{\leftmargin}{\labelsep}
\setlength{\labelwidth}{\tmplength}
}
\begin{flushleft}
\item[\textbf{Declaração}\hfill]
\begin{ttfamily}
public destructor destroy; override;\end{ttfamily}


\end{flushleft}
\end{list}
\section{Constantes}
\subsection*{tmp{\_}Alunos{\_}Idade}
\begin{list}{}{
\settowidth{\tmplength}{\textbf{Declaração}}
\setlength{\itemindent}{0cm}
\setlength{\listparindent}{0cm}
\setlength{\leftmargin}{\evensidemargin}
\addtolength{\leftmargin}{\tmplength}
\settowidth{\labelsep}{X}
\addtolength{\leftmargin}{\labelsep}
\setlength{\labelwidth}{\tmplength}
}
\begin{flushleft}
\item[\textbf{Declaração}\hfill]
\begin{ttfamily}
tmp{\_}Alunos{\_}Idade = '{\textbackslash}BB'+ChFN+'idade'+CharUpperlimit+{\#}64+
                     CharHint+'A idade do aluno. Valores válidos 1 a 64'+
                     CharHintPorque+'Este campo é necessário para que se agrupe o alunos baseado em sua faixa etária'+
                     CharHintOnde+'Ele será usado pelo coordenador ao classificar a turma';\end{ttfamily}


\end{flushleft}
\end{list}
\subsection*{tmp{\_}Alunos{\_}Matricula}
\begin{list}{}{
\settowidth{\tmplength}{\textbf{Declaração}}
\setlength{\itemindent}{0cm}
\setlength{\listparindent}{0cm}
\setlength{\leftmargin}{\evensidemargin}
\addtolength{\leftmargin}{\tmplength}
\settowidth{\labelsep}{X}
\addtolength{\leftmargin}{\labelsep}
\setlength{\labelwidth}{\tmplength}
}
\begin{flushleft}
\item[\textbf{Declaração}\hfill]
\begin{ttfamily}
tmp{\_}Alunos{\_}Matricula = '{\textbackslash}IIII'+ChFN+'matricula'+CharHint+'A matricula do aluno é um campo sequencial e calculado ao incluir o registro';\end{ttfamily}


\end{flushleft}
\end{list}
\subsection*{tmp{\_}Alunos}
\begin{list}{}{
\settowidth{\tmplength}{\textbf{Declaração}}
\setlength{\itemindent}{0cm}
\setlength{\listparindent}{0cm}
\setlength{\leftmargin}{\evensidemargin}
\addtolength{\leftmargin}{\tmplength}
\settowidth{\labelsep}{X}
\addtolength{\leftmargin}{\labelsep}
\setlength{\labelwidth}{\tmplength}
}
\begin{flushleft}
\item[\textbf{Declaração}\hfill]
\begin{ttfamily}
tmp{\_}Alunos = '~     Idade:~ {\%}s'+TDmxScroller{\_}Form{\_}Lcl.lf+
               '~ Matricula:~ {\%}s'+TDmxScroller{\_}Form{\_}Lcl.lf;\end{ttfamily}


\end{flushleft}
\end{list}
\section{Variáveis}
\subsection*{DmxScroller{\_}Form{\_}Lcl{\_}test}
\begin{list}{}{
\settowidth{\tmplength}{\textbf{Declaração}}
\setlength{\itemindent}{0cm}
\setlength{\listparindent}{0cm}
\setlength{\leftmargin}{\evensidemargin}
\addtolength{\leftmargin}{\tmplength}
\settowidth{\labelsep}{X}
\addtolength{\leftmargin}{\labelsep}
\setlength{\labelwidth}{\tmplength}
}
\begin{flushleft}
\item[\textbf{Declaração}\hfill]
\begin{ttfamily}
DmxScroller{\_}Form{\_}Lcl{\_}test: TDmxScroller{\_}Form{\_}Lcl{\_}test;\end{ttfamily}


\end{flushleft}
\end{list}
\chapter{Unit umi{\_}ui{\_}bitbtn{\_}lcl}
\section{Uses}
\begin{itemize}
\item \begin{ttfamily}Classes\end{ttfamily}\item \begin{ttfamily}SysUtils\end{ttfamily}\item \begin{ttfamily}LResources\end{ttfamily}\item \begin{ttfamily}Forms\end{ttfamily}\item \begin{ttfamily}Controls\end{ttfamily}\item \begin{ttfamily}Graphics\end{ttfamily}\item \begin{ttfamily}Dialogs\end{ttfamily}\item \begin{ttfamily}Buttons\end{ttfamily}\item \begin{ttfamily}ActnList\end{ttfamily}\item \begin{ttfamily}mi{\_}rtl{\_}ui{\_}DmxScroller{\_}Form\end{ttfamily}(\ref{mi_rtl_ui_dmxscroller_form})\item \begin{ttfamily}umi{\_}ui{\_}dmxscroller{\_}form{\_}lcl{\_}attributes\end{ttfamily}(\ref{umi_ui_dmxscroller_form_lcl_attributes})\end{itemize}
\section{Visão Geral}
\begin{description}
\item[\texttt{\begin{ttfamily}TMi{\_}BitBtn{\_}LCL\end{ttfamily} Classe}]
\end{description}
\begin{description}
\item[\texttt{Register}]
\end{description}
\section{Classes, Interfaces, Objetos e Registros}
\subsection*{TMi{\_}BitBtn{\_}LCL Classe}
\subsubsection*{\large{\textbf{Hierarquia}}\normalsize\hspace{1ex}\hfill}
TMi{\_}BitBtn{\_}LCL {$>$} TBitBtn
\subsubsection*{\large{\textbf{Descrição}}\normalsize\hspace{1ex}\hfill}
A classe \textbf{\begin{ttfamily}TMi{\_}BitBtn{\_}LCL\end{ttfamily}} é necessária para que se possa selecionar o controle associado ao botão criado pelo método: \begin{ttfamily}pDmxFieldRec\end{ttfamily}(\ref{mi_rtl_ui_dmxscroller_form-pDmxFieldRec}){\^{}}.createExecAction.\subsubsection*{\large{\textbf{Propriedades}}\normalsize\hspace{1ex}\hfill}
\paragraph*{DmxScroller{\_}Form{\_}Lcl{\_}attributes}\hspace*{\fill}

\begin{list}{}{
\settowidth{\tmplength}{\textbf{Declaração}}
\setlength{\itemindent}{0cm}
\setlength{\listparindent}{0cm}
\setlength{\leftmargin}{\evensidemargin}
\addtolength{\leftmargin}{\tmplength}
\settowidth{\labelsep}{X}
\addtolength{\leftmargin}{\labelsep}
\setlength{\labelwidth}{\tmplength}
}
\begin{flushleft}
\item[\textbf{Declaração}\hfill]
\begin{ttfamily}
published property DmxScroller{\_}Form{\_}Lcl{\_}attributes : TDmxScroller{\_}Form{\_}Lcl{\_}attributes Read {\_}DmxScroller{\_}Form{\_}Lcl{\_}attributes  write SetDmxScroller{\_}Form{\_}Lcl{\_}attributes;\end{ttfamily}


\end{flushleft}
\par
\item[\textbf{Descrição}]
A propriedade \textbf{\begin{ttfamily}DmxScroller{\_}Form{\_}Lcl{\_}attributes\end{ttfamily}} contém o modelo e os cálculos do formulário

\end{list}
\paragraph*{DmxFieldRec}\hspace*{\fill}

\begin{list}{}{
\settowidth{\tmplength}{\textbf{Declaração}}
\setlength{\itemindent}{0cm}
\setlength{\listparindent}{0cm}
\setlength{\leftmargin}{\evensidemargin}
\addtolength{\leftmargin}{\tmplength}
\settowidth{\labelsep}{X}
\addtolength{\leftmargin}{\labelsep}
\setlength{\labelwidth}{\tmplength}
}
\begin{flushleft}
\item[\textbf{Declaração}\hfill]
\begin{ttfamily}
public property DmxFieldRec: pDmxFieldRec Read {\_}pDmxFieldRec   Write  SeTDmxFieldRec;\end{ttfamily}


\end{flushleft}
\par
\item[\textbf{Descrição}]
A propriedade \textbf{\begin{ttfamily}DmxFieldRec\end{ttfamily}} fornece os dados necessários para criar o componente \begin{ttfamily}TMI{\_}BitBtn{\_}LCL\end{ttfamily}(\ref{umi_ui_bitbtn_lcl.TMi_BitBtn_LCL}).

\begin{itemize}
\item \textbf{NOTA} \begin{itemize}
\item Esses dados devem ser criados pelo método DmxScroller{\_}Form{\_}Lcl{\_}attributesr.CreateStruct(var ATemplate : \begin{ttfamily}TString\end{ttfamily}(\ref{mi_rtl_ui_Dmxscroller-tString}))
\end{itemize}
\end{itemize}

\end{list}
\subsubsection*{\large{\textbf{Métodos}}\normalsize\hspace{1ex}\hfill}
\paragraph*{DoOnEnter}\hspace*{\fill}

\begin{list}{}{
\settowidth{\tmplength}{\textbf{Declaração}}
\setlength{\itemindent}{0cm}
\setlength{\listparindent}{0cm}
\setlength{\leftmargin}{\evensidemargin}
\addtolength{\leftmargin}{\tmplength}
\settowidth{\labelsep}{X}
\addtolength{\leftmargin}{\labelsep}
\setlength{\labelwidth}{\tmplength}
}
\begin{flushleft}
\item[\textbf{Declaração}\hfill]
\begin{ttfamily}
protected procedure DoOnEnter(Sender: TObject);\end{ttfamily}


\end{flushleft}
\end{list}
\section{Funções e Procedimentos}
\subsection*{Register}
\begin{list}{}{
\settowidth{\tmplength}{\textbf{Declaração}}
\setlength{\itemindent}{0cm}
\setlength{\listparindent}{0cm}
\setlength{\leftmargin}{\evensidemargin}
\addtolength{\leftmargin}{\tmplength}
\settowidth{\labelsep}{X}
\addtolength{\leftmargin}{\labelsep}
\setlength{\labelwidth}{\tmplength}
}
\begin{flushleft}
\item[\textbf{Declaração}\hfill]
\begin{ttfamily}
procedure Register;\end{ttfamily}


\end{flushleft}
\end{list}
\chapter{Unit umi{\_}ui{\_}button{\_}lcl}
\section{Uses}
\begin{itemize}
\item \begin{ttfamily}Classes\end{ttfamily}\item \begin{ttfamily}SysUtils\end{ttfamily}\item \begin{ttfamily}LResources\end{ttfamily}\item \begin{ttfamily}Forms\end{ttfamily}\item \begin{ttfamily}Controls\end{ttfamily}\item \begin{ttfamily}Graphics\end{ttfamily}\item \begin{ttfamily}Dialogs\end{ttfamily}\item \begin{ttfamily}StdCtrls\end{ttfamily}\item \begin{ttfamily}ActnList\end{ttfamily}\item \begin{ttfamily}mi{\_}rtl{\_}ui{\_}DmxScroller{\_}Form\end{ttfamily}(\ref{mi_rtl_ui_dmxscroller_form})\item \begin{ttfamily}umi{\_}ui{\_}dmxscroller{\_}form{\_}lcl{\_}attributes\end{ttfamily}(\ref{umi_ui_dmxscroller_form_lcl_attributes})\end{itemize}
\section{Visão Geral}
\begin{description}
\item[\texttt{\begin{ttfamily}TMI{\_}Button{\_}LCL\end{ttfamily} Classe}]
\end{description}
\begin{description}
\item[\texttt{Register}]
\end{description}
\section{Classes, Interfaces, Objetos e Registros}
\subsection*{TMI{\_}Button{\_}LCL Classe}
\subsubsection*{\large{\textbf{Hierarquia}}\normalsize\hspace{1ex}\hfill}
TMI{\_}Button{\_}LCL {$>$} TButton
\subsubsection*{\large{\textbf{Descrição}}\normalsize\hspace{1ex}\hfill}
A classe \textbf{\begin{ttfamily}TMI{\_}Button{\_}LCL\end{ttfamily}} é necessária para que se possa selecionar o controle associado ao botão criado pelo método: \begin{ttfamily}pDmxFieldRec\end{ttfamily}(\ref{mi_rtl_ui_dmxscroller_form-pDmxFieldRec}){\^{}}.createExecAction.\subsubsection*{\large{\textbf{Propriedades}}\normalsize\hspace{1ex}\hfill}
\paragraph*{DmxScroller{\_}Form{\_}Lcl{\_}attributes}\hspace*{\fill}

\begin{list}{}{
\settowidth{\tmplength}{\textbf{Declaração}}
\setlength{\itemindent}{0cm}
\setlength{\listparindent}{0cm}
\setlength{\leftmargin}{\evensidemargin}
\addtolength{\leftmargin}{\tmplength}
\settowidth{\labelsep}{X}
\addtolength{\leftmargin}{\labelsep}
\setlength{\labelwidth}{\tmplength}
}
\begin{flushleft}
\item[\textbf{Declaração}\hfill]
\begin{ttfamily}
published property DmxScroller{\_}Form{\_}Lcl{\_}attributes : TDmxScroller{\_}Form{\_}Lcl{\_}attributes Read {\_}DmxScroller{\_}Form{\_}Lcl{\_}attributes  write SetDmxScroller{\_}Form{\_}Lcl{\_}attributes;\end{ttfamily}


\end{flushleft}
\par
\item[\textbf{Descrição}]
A propriedade \textbf{\begin{ttfamily}DmxScroller{\_}Form{\_}Lcl{\_}attributes\end{ttfamily}} contém o modelo e os cálculos do formulário

\end{list}
\paragraph*{DmxFieldRec}\hspace*{\fill}

\begin{list}{}{
\settowidth{\tmplength}{\textbf{Declaração}}
\setlength{\itemindent}{0cm}
\setlength{\listparindent}{0cm}
\setlength{\leftmargin}{\evensidemargin}
\addtolength{\leftmargin}{\tmplength}
\settowidth{\labelsep}{X}
\addtolength{\leftmargin}{\labelsep}
\setlength{\labelwidth}{\tmplength}
}
\begin{flushleft}
\item[\textbf{Declaração}\hfill]
\begin{ttfamily}
public property DmxFieldRec: pDmxFieldRec Read {\_}pDmxFieldRec   Write  SeTDmxFieldRec;\end{ttfamily}


\end{flushleft}
\par
\item[\textbf{Descrição}]
A propriedade \textbf{\begin{ttfamily}DmxFieldRec\end{ttfamily}} fornece os dados necessários para criar o componente \begin{ttfamily}TMI{\_}Button{\_}LCL\end{ttfamily}(\ref{umi_ui_button_lcl.TMI_Button_LCL}).

\begin{itemize}
\item \textbf{NOTA} \begin{itemize}
\item Esses dados devem ser criados pelo método DmxScroller{\_}Form{\_}Lcl{\_}attributesr.CreateStruct(var ATemplate : \begin{ttfamily}TString\end{ttfamily}(\ref{mi_rtl_ui_Dmxscroller-tString}))
\end{itemize}
\end{itemize}

\end{list}
\subsubsection*{\large{\textbf{Métodos}}\normalsize\hspace{1ex}\hfill}
\paragraph*{DoOnEnter}\hspace*{\fill}

\begin{list}{}{
\settowidth{\tmplength}{\textbf{Declaração}}
\setlength{\itemindent}{0cm}
\setlength{\listparindent}{0cm}
\setlength{\leftmargin}{\evensidemargin}
\addtolength{\leftmargin}{\tmplength}
\settowidth{\labelsep}{X}
\addtolength{\leftmargin}{\labelsep}
\setlength{\labelwidth}{\tmplength}
}
\begin{flushleft}
\item[\textbf{Declaração}\hfill]
\begin{ttfamily}
protected procedure DoOnEnter(Sender: TObject);\end{ttfamily}


\end{flushleft}
\end{list}
\section{Funções e Procedimentos}
\subsection*{Register}
\begin{list}{}{
\settowidth{\tmplength}{\textbf{Declaração}}
\setlength{\itemindent}{0cm}
\setlength{\listparindent}{0cm}
\setlength{\leftmargin}{\evensidemargin}
\addtolength{\leftmargin}{\tmplength}
\settowidth{\labelsep}{X}
\addtolength{\leftmargin}{\labelsep}
\setlength{\labelwidth}{\tmplength}
}
\begin{flushleft}
\item[\textbf{Declaração}\hfill]
\begin{ttfamily}
procedure Register;\end{ttfamily}


\end{flushleft}
\end{list}
\chapter{Unit umi{\_}ui{\_}checkbox{\_}lcl}
\section{Uses}
\begin{itemize}
\item \begin{ttfamily}Classes\end{ttfamily}\item \begin{ttfamily}SysUtils\end{ttfamily}\item \begin{ttfamily}LResources\end{ttfamily}\item \begin{ttfamily}Forms\end{ttfamily}\item \begin{ttfamily}Controls\end{ttfamily}\item \begin{ttfamily}Graphics\end{ttfamily}\item \begin{ttfamily}Dialogs\end{ttfamily}\item \begin{ttfamily}StdCtrls\end{ttfamily}\item \begin{ttfamily}ActnList\end{ttfamily}\item \begin{ttfamily}mi{\_}rtl{\_}ui{\_}DmxScroller\end{ttfamily}(\ref{mi_rtl_ui_Dmxscroller})\item \begin{ttfamily}mi{\_}rtl{\_}ui{\_}DmxScroller{\_}Form\end{ttfamily}(\ref{mi_rtl_ui_dmxscroller_form})\item \begin{ttfamily}umi{\_}ui{\_}dmxscroller{\_}form{\_}lcl{\_}attributes\end{ttfamily}(\ref{umi_ui_dmxscroller_form_lcl_attributes})\end{itemize}
\section{Visão Geral}
\begin{description}
\item[\texttt{\begin{ttfamily}TMI{\_}CheckBox{\_}LCL\end{ttfamily} Classe}]
\end{description}
\begin{description}
\item[\texttt{Register}]
\end{description}
\section{Classes, Interfaces, Objetos e Registros}
\subsection*{TMI{\_}CheckBox{\_}LCL Classe}
\subsubsection*{\large{\textbf{Hierarquia}}\normalsize\hspace{1ex}\hfill}
TMI{\_}CheckBox{\_}LCL {$>$} TCheckBox
%%%%Descrição
\subsubsection*{\large{\textbf{Propriedades}}\normalsize\hspace{1ex}\hfill}
\paragraph*{DmxScroller{\_}Form{\_}Lcl{\_}attributes}\hspace*{\fill}

\begin{list}{}{
\settowidth{\tmplength}{\textbf{Declaração}}
\setlength{\itemindent}{0cm}
\setlength{\listparindent}{0cm}
\setlength{\leftmargin}{\evensidemargin}
\addtolength{\leftmargin}{\tmplength}
\settowidth{\labelsep}{X}
\addtolength{\leftmargin}{\labelsep}
\setlength{\labelwidth}{\tmplength}
}
\begin{flushleft}
\item[\textbf{Declaração}\hfill]
\begin{ttfamily}
published property DmxScroller{\_}Form{\_}Lcl{\_}attributes : TDmxScroller{\_}Form{\_}Lcl{\_}attributes Read {\_}DmxScroller{\_}Form{\_}Lcl{\_}attributes  write SetDmxScroller{\_}Form{\_}Lcl{\_}attributes;\end{ttfamily}


\end{flushleft}
\par
\item[\textbf{Descrição}]
A propriedade \textbf{\begin{ttfamily}DmxScroller{\_}Form{\_}Lcl{\_}attributes\end{ttfamily}} contém o modelo e os cálculos do formulário

\end{list}
\paragraph*{DmxFieldRec}\hspace*{\fill}

\begin{list}{}{
\settowidth{\tmplength}{\textbf{Declaração}}
\setlength{\itemindent}{0cm}
\setlength{\listparindent}{0cm}
\setlength{\leftmargin}{\evensidemargin}
\addtolength{\leftmargin}{\tmplength}
\settowidth{\labelsep}{X}
\addtolength{\leftmargin}{\labelsep}
\setlength{\labelwidth}{\tmplength}
}
\begin{flushleft}
\item[\textbf{Declaração}\hfill]
\begin{ttfamily}
public property DmxFieldRec: pDmxFieldRec Read {\_}pDmxFieldRec   Write  SeTDmxFieldRec;\end{ttfamily}


\end{flushleft}
\par
\item[\textbf{Descrição}]
A propriedade \textbf{\begin{ttfamily}DmxFieldRec\end{ttfamily}} fornece os dados necessários para criar o componente \begin{ttfamily}TMI{\_}Button{\_}LCL\end{ttfamily}(\ref{umi_ui_button_lcl.TMI_Button_LCL}).

\begin{itemize}
\item \textbf{NOTA} \begin{itemize}
\item Esses dados devem ser criados pelo método DmxScroller{\_}Form{\_}Lcl{\_}attributesr.CreateStruct(var ATemplate : \begin{ttfamily}TString\end{ttfamily}(\ref{mi_rtl_ui_Dmxscroller-tString}))
\end{itemize}
\end{itemize}

\end{list}
\subsubsection*{\large{\textbf{Métodos}}\normalsize\hspace{1ex}\hfill}
\paragraph*{PutBuffer}\hspace*{\fill}

\begin{list}{}{
\settowidth{\tmplength}{\textbf{Declaração}}
\setlength{\itemindent}{0cm}
\setlength{\listparindent}{0cm}
\setlength{\leftmargin}{\evensidemargin}
\addtolength{\leftmargin}{\tmplength}
\settowidth{\labelsep}{X}
\addtolength{\leftmargin}{\labelsep}
\setlength{\labelwidth}{\tmplength}
}
\begin{flushleft}
\item[\textbf{Declaração}\hfill]
\begin{ttfamily}
public Procedure PutBuffer;\end{ttfamily}


\end{flushleft}
\par
\item[\textbf{Descrição}]
O método \textbf{\begin{ttfamily}PutBuffer\end{ttfamily}} salva os dados do controle (Self) para a propriedade \begin{ttfamily}pDmxFieldRec\end{ttfamily}(\ref{mi_rtl_ui_Dmxscroller-pDmxFieldRec})

\end{list}
\paragraph*{GetBuffer}\hspace*{\fill}

\begin{list}{}{
\settowidth{\tmplength}{\textbf{Declaração}}
\setlength{\itemindent}{0cm}
\setlength{\listparindent}{0cm}
\setlength{\leftmargin}{\evensidemargin}
\addtolength{\leftmargin}{\tmplength}
\settowidth{\labelsep}{X}
\addtolength{\leftmargin}{\labelsep}
\setlength{\labelwidth}{\tmplength}
}
\begin{flushleft}
\item[\textbf{Declaração}\hfill]
\begin{ttfamily}
public Procedure GetBuffer;\end{ttfamily}


\end{flushleft}
\par
\item[\textbf{Descrição}]
O método \textbf{\begin{ttfamily}GetBuffer\end{ttfamily}} ler os dados da propriedade \begin{ttfamily}pDmxFieldRec\end{ttfamily}(\ref{mi_rtl_ui_Dmxscroller-pDmxFieldRec}) para o controle (Self).

\end{list}
\paragraph*{DoOnEnter}\hspace*{\fill}

\begin{list}{}{
\settowidth{\tmplength}{\textbf{Declaração}}
\setlength{\itemindent}{0cm}
\setlength{\listparindent}{0cm}
\setlength{\leftmargin}{\evensidemargin}
\addtolength{\leftmargin}{\tmplength}
\settowidth{\labelsep}{X}
\addtolength{\leftmargin}{\labelsep}
\setlength{\labelwidth}{\tmplength}
}
\begin{flushleft}
\item[\textbf{Declaração}\hfill]
\begin{ttfamily}
protected procedure DoOnEnter(Sender: TObject);\end{ttfamily}


\end{flushleft}
\end{list}
\paragraph*{DoOnExit}\hspace*{\fill}

\begin{list}{}{
\settowidth{\tmplength}{\textbf{Declaração}}
\setlength{\itemindent}{0cm}
\setlength{\listparindent}{0cm}
\setlength{\leftmargin}{\evensidemargin}
\addtolength{\leftmargin}{\tmplength}
\settowidth{\labelsep}{X}
\addtolength{\leftmargin}{\labelsep}
\setlength{\labelwidth}{\tmplength}
}
\begin{flushleft}
\item[\textbf{Declaração}\hfill]
\begin{ttfamily}
protected procedure DoOnExit(Sender: TObject);\end{ttfamily}


\end{flushleft}
\par
\item[\textbf{Descrição}]
O método \textbf{\begin{ttfamily}DoOnExit\end{ttfamily}} ao perder o foco executa os métodos PuttBuffer e pDmxFieldRec.DoOnExit(Self).

\end{list}
\section{Funções e Procedimentos}
\subsection*{Register}
\begin{list}{}{
\settowidth{\tmplength}{\textbf{Declaração}}
\setlength{\itemindent}{0cm}
\setlength{\listparindent}{0cm}
\setlength{\leftmargin}{\evensidemargin}
\addtolength{\leftmargin}{\tmplength}
\settowidth{\labelsep}{X}
\addtolength{\leftmargin}{\labelsep}
\setlength{\labelwidth}{\tmplength}
}
\begin{flushleft}
\item[\textbf{Declaração}\hfill]
\begin{ttfamily}
procedure Register;\end{ttfamily}


\end{flushleft}
\end{list}
\chapter{Unit uMi{\_}ui{\_}ComboBox{\_}lcl}
\section{Uses}
\begin{itemize}
\item \begin{ttfamily}Windows\end{ttfamily}\item \begin{ttfamily}Messages\end{ttfamily}\item \begin{ttfamily}SysUtils\end{ttfamily}\item \begin{ttfamily}Classes\end{ttfamily}\item \begin{ttfamily}Graphics\end{ttfamily}\item \begin{ttfamily}Controls\end{ttfamily}\item \begin{ttfamily}Forms\end{ttfamily}\item \begin{ttfamily}Dialogs\end{ttfamily}\item \begin{ttfamily}StdCtrls\end{ttfamily}\item \begin{ttfamily}LResources\end{ttfamily}\item \begin{ttfamily}mi{\_}rtl{\_}ui{\_}DmxScroller\end{ttfamily}(\ref{mi_rtl_ui_Dmxscroller})\item \begin{ttfamily}mi{\_}rtl{\_}ui{\_}DmxScroller{\_}Form\end{ttfamily}(\ref{mi_rtl_ui_dmxscroller_form})\item \begin{ttfamily}umi{\_}ui{\_}dmxscroller{\_}form{\_}lcl{\_}attributes\end{ttfamily}(\ref{umi_ui_dmxscroller_form_lcl_attributes})\end{itemize}
\section{Visão Geral}
\begin{description}
\item[\texttt{\begin{ttfamily}TMI{\_}ComboBox{\_}LCL\end{ttfamily} Classe}]
\end{description}
\begin{description}
\item[\texttt{Register}]
\end{description}
\section{Classes, Interfaces, Objetos e Registros}
\subsection*{TMI{\_}ComboBox{\_}LCL Classe}
\subsubsection*{\large{\textbf{Hierarquia}}\normalsize\hspace{1ex}\hfill}
TMI{\_}ComboBox{\_}LCL {$>$} TComboBox
\subsubsection*{\large{\textbf{Descrição}}\normalsize\hspace{1ex}\hfill}
A classe \textbf{\begin{ttfamily}TMI{\_}ComboBox{\_}LCL\end{ttfamily}} permite edita um campo enumerado do componente \textbf{\begin{ttfamily}TDmxFieldRec\end{ttfamily}(\ref{mi_rtl_ui_Dmxscroller.TDmxFieldRec})}

\begin{itemize}
\item \textbf{NOTA} \begin{itemize}
\item O item zero cont�m a string selecionada e caso a mesma seja editada o valor digitado passa ser o filtro de pesquisa.
\end{itemize}
\end{itemize}\subsubsection*{\large{\textbf{Propriedades}}\normalsize\hspace{1ex}\hfill}
\paragraph*{DmxScroller{\_}Form{\_}Lcl{\_}attributes}\hspace*{\fill}

\begin{list}{}{
\settowidth{\tmplength}{\textbf{Declaração}}
\setlength{\itemindent}{0cm}
\setlength{\listparindent}{0cm}
\setlength{\leftmargin}{\evensidemargin}
\addtolength{\leftmargin}{\tmplength}
\settowidth{\labelsep}{X}
\addtolength{\leftmargin}{\labelsep}
\setlength{\labelwidth}{\tmplength}
}
\begin{flushleft}
\item[\textbf{Declaração}\hfill]
\begin{ttfamily}
published property DmxScroller{\_}Form{\_}Lcl{\_}attributes: TDmxScroller{\_}Form{\_}Lcl{\_}attributes Read {\_}DmxScroller{\_}Form{\_}Lcl{\_}attributes write SetDmxScroller{\_}Form{\_}Lcl{\_}attributes;\end{ttfamily}


\end{flushleft}
\end{list}
\paragraph*{DmxFieldRec}\hspace*{\fill}

\begin{list}{}{
\settowidth{\tmplength}{\textbf{Declaração}}
\setlength{\itemindent}{0cm}
\setlength{\listparindent}{0cm}
\setlength{\leftmargin}{\evensidemargin}
\addtolength{\leftmargin}{\tmplength}
\settowidth{\labelsep}{X}
\addtolength{\leftmargin}{\labelsep}
\setlength{\labelwidth}{\tmplength}
}
\begin{flushleft}
\item[\textbf{Declaração}\hfill]
\begin{ttfamily}
public property DmxFieldRec: pDmxFieldRec Read {\_}pDmxFieldRec   Write  SeTDmxFieldRec;\end{ttfamily}


\end{flushleft}
\par
\item[\textbf{Descrição}]
O atributo \textbf{\begin{ttfamily}DmxFieldRec\end{ttfamily}} fornece os dados necess�rios para criar o componente \begin{ttfamily}TMI{\_}MaskEdit{\_}LCL\end{ttfamily}(\ref{uMi_ui_maskedit_lcl.TMI_MaskEdit_LCL}).

\begin{itemize}
\item \textbf{NOTA} \begin{itemize}
\item Esses dados devem ser criados pelo m�todo TDmxScroller{\_}Form{\_}Lcl{\_}attributes.CreateStruct(var ATemplate : \begin{ttfamily}TString\end{ttfamily}(\ref{mi_rtl_ui_Dmxscroller-tString}))
\end{itemize}
\end{itemize}

\end{list}
\paragraph*{Value}\hspace*{\fill}

\begin{list}{}{
\settowidth{\tmplength}{\textbf{Declaração}}
\setlength{\itemindent}{0cm}
\setlength{\listparindent}{0cm}
\setlength{\leftmargin}{\evensidemargin}
\addtolength{\leftmargin}{\tmplength}
\settowidth{\labelsep}{X}
\addtolength{\leftmargin}{\labelsep}
\setlength{\labelwidth}{\tmplength}
}
\begin{flushleft}
\item[\textbf{Declaração}\hfill]
\begin{ttfamily}
public property Value: String read GetValue write SetValue;\end{ttfamily}


\end{flushleft}
\end{list}
\paragraph*{ImgIndexes}\hspace*{\fill}

\begin{list}{}{
\settowidth{\tmplength}{\textbf{Declaração}}
\setlength{\itemindent}{0cm}
\setlength{\listparindent}{0cm}
\setlength{\leftmargin}{\evensidemargin}
\addtolength{\leftmargin}{\tmplength}
\settowidth{\labelsep}{X}
\addtolength{\leftmargin}{\labelsep}
\setlength{\labelwidth}{\tmplength}
}
\begin{flushleft}
\item[\textbf{Declaração}\hfill]
\begin{ttfamily}
published property ImgIndexes: TStringList read FImgIndexes write SetImgIndexes;\end{ttfamily}


\end{flushleft}
\end{list}
\paragraph*{Images}\hspace*{\fill}

\begin{list}{}{
\settowidth{\tmplength}{\textbf{Declaração}}
\setlength{\itemindent}{0cm}
\setlength{\listparindent}{0cm}
\setlength{\leftmargin}{\evensidemargin}
\addtolength{\leftmargin}{\tmplength}
\settowidth{\labelsep}{X}
\addtolength{\leftmargin}{\labelsep}
\setlength{\labelwidth}{\tmplength}
}
\begin{flushleft}
\item[\textbf{Declaração}\hfill]
\begin{ttfamily}
published property Images: TImageList read FImages write SetImages;\end{ttfamily}


\end{flushleft}
\end{list}
\paragraph*{ShowImages}\hspace*{\fill}

\begin{list}{}{
\settowidth{\tmplength}{\textbf{Declaração}}
\setlength{\itemindent}{0cm}
\setlength{\listparindent}{0cm}
\setlength{\leftmargin}{\evensidemargin}
\addtolength{\leftmargin}{\tmplength}
\settowidth{\labelsep}{X}
\addtolength{\leftmargin}{\labelsep}
\setlength{\labelwidth}{\tmplength}
}
\begin{flushleft}
\item[\textbf{Declaração}\hfill]
\begin{ttfamily}
published property ShowImages: Boolean read FShowImages write SetShowImages;\end{ttfamily}


\end{flushleft}
\end{list}
\paragraph*{Color}\hspace*{\fill}

\begin{list}{}{
\settowidth{\tmplength}{\textbf{Declaração}}
\setlength{\itemindent}{0cm}
\setlength{\listparindent}{0cm}
\setlength{\leftmargin}{\evensidemargin}
\addtolength{\leftmargin}{\tmplength}
\settowidth{\labelsep}{X}
\addtolength{\leftmargin}{\labelsep}
\setlength{\labelwidth}{\tmplength}
}
\begin{flushleft}
\item[\textbf{Declaração}\hfill]
\begin{ttfamily}
published property Color;\end{ttfamily}


\end{flushleft}
\end{list}
\paragraph*{Align}\hspace*{\fill}

\begin{list}{}{
\settowidth{\tmplength}{\textbf{Declaração}}
\setlength{\itemindent}{0cm}
\setlength{\listparindent}{0cm}
\setlength{\leftmargin}{\evensidemargin}
\addtolength{\leftmargin}{\tmplength}
\settowidth{\labelsep}{X}
\addtolength{\leftmargin}{\labelsep}
\setlength{\labelwidth}{\tmplength}
}
\begin{flushleft}
\item[\textbf{Declaração}\hfill]
\begin{ttfamily}
published property Align;\end{ttfamily}


\end{flushleft}
\end{list}
\paragraph*{AutoComplete}\hspace*{\fill}

\begin{list}{}{
\settowidth{\tmplength}{\textbf{Declaração}}
\setlength{\itemindent}{0cm}
\setlength{\listparindent}{0cm}
\setlength{\leftmargin}{\evensidemargin}
\addtolength{\leftmargin}{\tmplength}
\settowidth{\labelsep}{X}
\addtolength{\leftmargin}{\labelsep}
\setlength{\labelwidth}{\tmplength}
}
\begin{flushleft}
\item[\textbf{Declaração}\hfill]
\begin{ttfamily}
published property AutoComplete;\end{ttfamily}


\end{flushleft}
\end{list}
\paragraph*{AutoDropDown}\hspace*{\fill}

\begin{list}{}{
\settowidth{\tmplength}{\textbf{Declaração}}
\setlength{\itemindent}{0cm}
\setlength{\listparindent}{0cm}
\setlength{\leftmargin}{\evensidemargin}
\addtolength{\leftmargin}{\tmplength}
\settowidth{\labelsep}{X}
\addtolength{\leftmargin}{\labelsep}
\setlength{\labelwidth}{\tmplength}
}
\begin{flushleft}
\item[\textbf{Declaração}\hfill]
\begin{ttfamily}
published property AutoDropDown;\end{ttfamily}


\end{flushleft}
\end{list}
\paragraph*{AutoSelect}\hspace*{\fill}

\begin{list}{}{
\settowidth{\tmplength}{\textbf{Declaração}}
\setlength{\itemindent}{0cm}
\setlength{\listparindent}{0cm}
\setlength{\leftmargin}{\evensidemargin}
\addtolength{\leftmargin}{\tmplength}
\settowidth{\labelsep}{X}
\addtolength{\leftmargin}{\labelsep}
\setlength{\labelwidth}{\tmplength}
}
\begin{flushleft}
\item[\textbf{Declaração}\hfill]
\begin{ttfamily}
published property AutoSelect;\end{ttfamily}


\end{flushleft}
\end{list}
\paragraph*{OnEditingDone}\hspace*{\fill}

\begin{list}{}{
\settowidth{\tmplength}{\textbf{Declaração}}
\setlength{\itemindent}{0cm}
\setlength{\listparindent}{0cm}
\setlength{\leftmargin}{\evensidemargin}
\addtolength{\leftmargin}{\tmplength}
\settowidth{\labelsep}{X}
\addtolength{\leftmargin}{\labelsep}
\setlength{\labelwidth}{\tmplength}
}
\begin{flushleft}
\item[\textbf{Declaração}\hfill]
\begin{ttfamily}
published property OnEditingDone;\end{ttfamily}


\end{flushleft}
\end{list}
\paragraph*{AutoSize}\hspace*{\fill}

\begin{list}{}{
\settowidth{\tmplength}{\textbf{Declaração}}
\setlength{\itemindent}{0cm}
\setlength{\listparindent}{0cm}
\setlength{\leftmargin}{\evensidemargin}
\addtolength{\leftmargin}{\tmplength}
\settowidth{\labelsep}{X}
\addtolength{\leftmargin}{\labelsep}
\setlength{\labelwidth}{\tmplength}
}
\begin{flushleft}
\item[\textbf{Declaração}\hfill]
\begin{ttfamily}
published property AutoSize;\end{ttfamily}


\end{flushleft}
\end{list}
\paragraph*{text}\hspace*{\fill}

\begin{list}{}{
\settowidth{\tmplength}{\textbf{Declaração}}
\setlength{\itemindent}{0cm}
\setlength{\listparindent}{0cm}
\setlength{\leftmargin}{\evensidemargin}
\addtolength{\leftmargin}{\tmplength}
\settowidth{\labelsep}{X}
\addtolength{\leftmargin}{\labelsep}
\setlength{\labelwidth}{\tmplength}
}
\begin{flushleft}
\item[\textbf{Declaração}\hfill]
\begin{ttfamily}
published property text;\end{ttfamily}


\end{flushleft}
\end{list}
\paragraph*{ItemIndex}\hspace*{\fill}

\begin{list}{}{
\settowidth{\tmplength}{\textbf{Declaração}}
\setlength{\itemindent}{0cm}
\setlength{\listparindent}{0cm}
\setlength{\leftmargin}{\evensidemargin}
\addtolength{\leftmargin}{\tmplength}
\settowidth{\labelsep}{X}
\addtolength{\leftmargin}{\labelsep}
\setlength{\labelwidth}{\tmplength}
}
\begin{flushleft}
\item[\textbf{Declaração}\hfill]
\begin{ttfamily}
published property ItemIndex;\end{ttfamily}


\end{flushleft}
\end{list}
\paragraph*{DragMode}\hspace*{\fill}

\begin{list}{}{
\settowidth{\tmplength}{\textbf{Declaração}}
\setlength{\itemindent}{0cm}
\setlength{\listparindent}{0cm}
\setlength{\leftmargin}{\evensidemargin}
\addtolength{\leftmargin}{\tmplength}
\settowidth{\labelsep}{X}
\addtolength{\leftmargin}{\labelsep}
\setlength{\labelwidth}{\tmplength}
}
\begin{flushleft}
\item[\textbf{Declaração}\hfill]
\begin{ttfamily}
published property DragMode;\end{ttfamily}


\end{flushleft}
\end{list}
\paragraph*{DragCursor}\hspace*{\fill}

\begin{list}{}{
\settowidth{\tmplength}{\textbf{Declaração}}
\setlength{\itemindent}{0cm}
\setlength{\listparindent}{0cm}
\setlength{\leftmargin}{\evensidemargin}
\addtolength{\leftmargin}{\tmplength}
\settowidth{\labelsep}{X}
\addtolength{\leftmargin}{\labelsep}
\setlength{\labelwidth}{\tmplength}
}
\begin{flushleft}
\item[\textbf{Declaração}\hfill]
\begin{ttfamily}
published property DragCursor;\end{ttfamily}


\end{flushleft}
\end{list}
\paragraph*{DropDownCount}\hspace*{\fill}

\begin{list}{}{
\settowidth{\tmplength}{\textbf{Declaração}}
\setlength{\itemindent}{0cm}
\setlength{\listparindent}{0cm}
\setlength{\leftmargin}{\evensidemargin}
\addtolength{\leftmargin}{\tmplength}
\settowidth{\labelsep}{X}
\addtolength{\leftmargin}{\labelsep}
\setlength{\labelwidth}{\tmplength}
}
\begin{flushleft}
\item[\textbf{Declaração}\hfill]
\begin{ttfamily}
published property DropDownCount;\end{ttfamily}


\end{flushleft}
\end{list}
\paragraph*{Enabled}\hspace*{\fill}

\begin{list}{}{
\settowidth{\tmplength}{\textbf{Declaração}}
\setlength{\itemindent}{0cm}
\setlength{\listparindent}{0cm}
\setlength{\leftmargin}{\evensidemargin}
\addtolength{\leftmargin}{\tmplength}
\settowidth{\labelsep}{X}
\addtolength{\leftmargin}{\labelsep}
\setlength{\labelwidth}{\tmplength}
}
\begin{flushleft}
\item[\textbf{Declaração}\hfill]
\begin{ttfamily}
published property Enabled;\end{ttfamily}


\end{flushleft}
\end{list}
\paragraph*{Font}\hspace*{\fill}

\begin{list}{}{
\settowidth{\tmplength}{\textbf{Declaração}}
\setlength{\itemindent}{0cm}
\setlength{\listparindent}{0cm}
\setlength{\leftmargin}{\evensidemargin}
\addtolength{\leftmargin}{\tmplength}
\settowidth{\labelsep}{X}
\addtolength{\leftmargin}{\labelsep}
\setlength{\labelwidth}{\tmplength}
}
\begin{flushleft}
\item[\textbf{Declaração}\hfill]
\begin{ttfamily}
published property Font;\end{ttfamily}


\end{flushleft}
\end{list}
\paragraph*{ItemHeight}\hspace*{\fill}

\begin{list}{}{
\settowidth{\tmplength}{\textbf{Declaração}}
\setlength{\itemindent}{0cm}
\setlength{\listparindent}{0cm}
\setlength{\leftmargin}{\evensidemargin}
\addtolength{\leftmargin}{\tmplength}
\settowidth{\labelsep}{X}
\addtolength{\leftmargin}{\labelsep}
\setlength{\labelwidth}{\tmplength}
}
\begin{flushleft}
\item[\textbf{Declaração}\hfill]
\begin{ttfamily}
published property ItemHeight;\end{ttfamily}


\end{flushleft}
\end{list}
\paragraph*{Items}\hspace*{\fill}

\begin{list}{}{
\settowidth{\tmplength}{\textbf{Declaração}}
\setlength{\itemindent}{0cm}
\setlength{\listparindent}{0cm}
\setlength{\leftmargin}{\evensidemargin}
\addtolength{\leftmargin}{\tmplength}
\settowidth{\labelsep}{X}
\addtolength{\leftmargin}{\labelsep}
\setlength{\labelwidth}{\tmplength}
}
\begin{flushleft}
\item[\textbf{Declaração}\hfill]
\begin{ttfamily}
published property Items;\end{ttfamily}


\end{flushleft}
\end{list}
\paragraph*{MaxLength}\hspace*{\fill}

\begin{list}{}{
\settowidth{\tmplength}{\textbf{Declaração}}
\setlength{\itemindent}{0cm}
\setlength{\listparindent}{0cm}
\setlength{\leftmargin}{\evensidemargin}
\addtolength{\leftmargin}{\tmplength}
\settowidth{\labelsep}{X}
\addtolength{\leftmargin}{\labelsep}
\setlength{\labelwidth}{\tmplength}
}
\begin{flushleft}
\item[\textbf{Declaração}\hfill]
\begin{ttfamily}
published property MaxLength;\end{ttfamily}


\end{flushleft}
\end{list}
\paragraph*{ParentColor}\hspace*{\fill}

\begin{list}{}{
\settowidth{\tmplength}{\textbf{Declaração}}
\setlength{\itemindent}{0cm}
\setlength{\listparindent}{0cm}
\setlength{\leftmargin}{\evensidemargin}
\addtolength{\leftmargin}{\tmplength}
\settowidth{\labelsep}{X}
\addtolength{\leftmargin}{\labelsep}
\setlength{\labelwidth}{\tmplength}
}
\begin{flushleft}
\item[\textbf{Declaração}\hfill]
\begin{ttfamily}
published property ParentColor;\end{ttfamily}


\end{flushleft}
\end{list}
\paragraph*{ParentFont}\hspace*{\fill}

\begin{list}{}{
\settowidth{\tmplength}{\textbf{Declaração}}
\setlength{\itemindent}{0cm}
\setlength{\listparindent}{0cm}
\setlength{\leftmargin}{\evensidemargin}
\addtolength{\leftmargin}{\tmplength}
\settowidth{\labelsep}{X}
\addtolength{\leftmargin}{\labelsep}
\setlength{\labelwidth}{\tmplength}
}
\begin{flushleft}
\item[\textbf{Declaração}\hfill]
\begin{ttfamily}
published property ParentFont;\end{ttfamily}


\end{flushleft}
\end{list}
\paragraph*{ParentShowHint}\hspace*{\fill}

\begin{list}{}{
\settowidth{\tmplength}{\textbf{Declaração}}
\setlength{\itemindent}{0cm}
\setlength{\listparindent}{0cm}
\setlength{\leftmargin}{\evensidemargin}
\addtolength{\leftmargin}{\tmplength}
\settowidth{\labelsep}{X}
\addtolength{\leftmargin}{\labelsep}
\setlength{\labelwidth}{\tmplength}
}
\begin{flushleft}
\item[\textbf{Declaração}\hfill]
\begin{ttfamily}
published property ParentShowHint;\end{ttfamily}


\end{flushleft}
\end{list}
\paragraph*{PopupMenu}\hspace*{\fill}

\begin{list}{}{
\settowidth{\tmplength}{\textbf{Declaração}}
\setlength{\itemindent}{0cm}
\setlength{\listparindent}{0cm}
\setlength{\leftmargin}{\evensidemargin}
\addtolength{\leftmargin}{\tmplength}
\settowidth{\labelsep}{X}
\addtolength{\leftmargin}{\labelsep}
\setlength{\labelwidth}{\tmplength}
}
\begin{flushleft}
\item[\textbf{Declaração}\hfill]
\begin{ttfamily}
published property PopupMenu;\end{ttfamily}


\end{flushleft}
\end{list}
\paragraph*{ShowHint}\hspace*{\fill}

\begin{list}{}{
\settowidth{\tmplength}{\textbf{Declaração}}
\setlength{\itemindent}{0cm}
\setlength{\listparindent}{0cm}
\setlength{\leftmargin}{\evensidemargin}
\addtolength{\leftmargin}{\tmplength}
\settowidth{\labelsep}{X}
\addtolength{\leftmargin}{\labelsep}
\setlength{\labelwidth}{\tmplength}
}
\begin{flushleft}
\item[\textbf{Declaração}\hfill]
\begin{ttfamily}
published property ShowHint;\end{ttfamily}


\end{flushleft}
\end{list}
\paragraph*{Sorted}\hspace*{\fill}

\begin{list}{}{
\settowidth{\tmplength}{\textbf{Declaração}}
\setlength{\itemindent}{0cm}
\setlength{\listparindent}{0cm}
\setlength{\leftmargin}{\evensidemargin}
\addtolength{\leftmargin}{\tmplength}
\settowidth{\labelsep}{X}
\addtolength{\leftmargin}{\labelsep}
\setlength{\labelwidth}{\tmplength}
}
\begin{flushleft}
\item[\textbf{Declaração}\hfill]
\begin{ttfamily}
published property Sorted;\end{ttfamily}


\end{flushleft}
\end{list}
\paragraph*{TabOrder}\hspace*{\fill}

\begin{list}{}{
\settowidth{\tmplength}{\textbf{Declaração}}
\setlength{\itemindent}{0cm}
\setlength{\listparindent}{0cm}
\setlength{\leftmargin}{\evensidemargin}
\addtolength{\leftmargin}{\tmplength}
\settowidth{\labelsep}{X}
\addtolength{\leftmargin}{\labelsep}
\setlength{\labelwidth}{\tmplength}
}
\begin{flushleft}
\item[\textbf{Declaração}\hfill]
\begin{ttfamily}
published property TabOrder;\end{ttfamily}


\end{flushleft}
\end{list}
\paragraph*{TabStop}\hspace*{\fill}

\begin{list}{}{
\settowidth{\tmplength}{\textbf{Declaração}}
\setlength{\itemindent}{0cm}
\setlength{\listparindent}{0cm}
\setlength{\leftmargin}{\evensidemargin}
\addtolength{\leftmargin}{\tmplength}
\settowidth{\labelsep}{X}
\addtolength{\leftmargin}{\labelsep}
\setlength{\labelwidth}{\tmplength}
}
\begin{flushleft}
\item[\textbf{Declaração}\hfill]
\begin{ttfamily}
published property TabStop;\end{ttfamily}


\end{flushleft}
\end{list}
\paragraph*{Visible}\hspace*{\fill}

\begin{list}{}{
\settowidth{\tmplength}{\textbf{Declaração}}
\setlength{\itemindent}{0cm}
\setlength{\listparindent}{0cm}
\setlength{\leftmargin}{\evensidemargin}
\addtolength{\leftmargin}{\tmplength}
\settowidth{\labelsep}{X}
\addtolength{\leftmargin}{\labelsep}
\setlength{\labelwidth}{\tmplength}
}
\begin{flushleft}
\item[\textbf{Declaração}\hfill]
\begin{ttfamily}
published property Visible;\end{ttfamily}


\end{flushleft}
\end{list}
\paragraph*{OnChange}\hspace*{\fill}

\begin{list}{}{
\settowidth{\tmplength}{\textbf{Declaração}}
\setlength{\itemindent}{0cm}
\setlength{\listparindent}{0cm}
\setlength{\leftmargin}{\evensidemargin}
\addtolength{\leftmargin}{\tmplength}
\settowidth{\labelsep}{X}
\addtolength{\leftmargin}{\labelsep}
\setlength{\labelwidth}{\tmplength}
}
\begin{flushleft}
\item[\textbf{Declaração}\hfill]
\begin{ttfamily}
published property OnChange;\end{ttfamily}


\end{flushleft}
\end{list}
\paragraph*{OnClick}\hspace*{\fill}

\begin{list}{}{
\settowidth{\tmplength}{\textbf{Declaração}}
\setlength{\itemindent}{0cm}
\setlength{\listparindent}{0cm}
\setlength{\leftmargin}{\evensidemargin}
\addtolength{\leftmargin}{\tmplength}
\settowidth{\labelsep}{X}
\addtolength{\leftmargin}{\labelsep}
\setlength{\labelwidth}{\tmplength}
}
\begin{flushleft}
\item[\textbf{Declaração}\hfill]
\begin{ttfamily}
published property OnClick;\end{ttfamily}


\end{flushleft}
\end{list}
\paragraph*{OnDblClick}\hspace*{\fill}

\begin{list}{}{
\settowidth{\tmplength}{\textbf{Declaração}}
\setlength{\itemindent}{0cm}
\setlength{\listparindent}{0cm}
\setlength{\leftmargin}{\evensidemargin}
\addtolength{\leftmargin}{\tmplength}
\settowidth{\labelsep}{X}
\addtolength{\leftmargin}{\labelsep}
\setlength{\labelwidth}{\tmplength}
}
\begin{flushleft}
\item[\textbf{Declaração}\hfill]
\begin{ttfamily}
published property OnDblClick;\end{ttfamily}


\end{flushleft}
\end{list}
\paragraph*{OnDragDrop}\hspace*{\fill}

\begin{list}{}{
\settowidth{\tmplength}{\textbf{Declaração}}
\setlength{\itemindent}{0cm}
\setlength{\listparindent}{0cm}
\setlength{\leftmargin}{\evensidemargin}
\addtolength{\leftmargin}{\tmplength}
\settowidth{\labelsep}{X}
\addtolength{\leftmargin}{\labelsep}
\setlength{\labelwidth}{\tmplength}
}
\begin{flushleft}
\item[\textbf{Declaração}\hfill]
\begin{ttfamily}
published property OnDragDrop;\end{ttfamily}


\end{flushleft}
\end{list}
\paragraph*{OnDragOver}\hspace*{\fill}

\begin{list}{}{
\settowidth{\tmplength}{\textbf{Declaração}}
\setlength{\itemindent}{0cm}
\setlength{\listparindent}{0cm}
\setlength{\leftmargin}{\evensidemargin}
\addtolength{\leftmargin}{\tmplength}
\settowidth{\labelsep}{X}
\addtolength{\leftmargin}{\labelsep}
\setlength{\labelwidth}{\tmplength}
}
\begin{flushleft}
\item[\textbf{Declaração}\hfill]
\begin{ttfamily}
published property OnDragOver;\end{ttfamily}


\end{flushleft}
\end{list}
\paragraph*{OnDrawItem}\hspace*{\fill}

\begin{list}{}{
\settowidth{\tmplength}{\textbf{Declaração}}
\setlength{\itemindent}{0cm}
\setlength{\listparindent}{0cm}
\setlength{\leftmargin}{\evensidemargin}
\addtolength{\leftmargin}{\tmplength}
\settowidth{\labelsep}{X}
\addtolength{\leftmargin}{\labelsep}
\setlength{\labelwidth}{\tmplength}
}
\begin{flushleft}
\item[\textbf{Declaração}\hfill]
\begin{ttfamily}
published property OnDrawItem;\end{ttfamily}


\end{flushleft}
\end{list}
\paragraph*{OnDropDown}\hspace*{\fill}

\begin{list}{}{
\settowidth{\tmplength}{\textbf{Declaração}}
\setlength{\itemindent}{0cm}
\setlength{\listparindent}{0cm}
\setlength{\leftmargin}{\evensidemargin}
\addtolength{\leftmargin}{\tmplength}
\settowidth{\labelsep}{X}
\addtolength{\leftmargin}{\labelsep}
\setlength{\labelwidth}{\tmplength}
}
\begin{flushleft}
\item[\textbf{Declaração}\hfill]
\begin{ttfamily}
published property OnDropDown;\end{ttfamily}


\end{flushleft}
\end{list}
\paragraph*{OnEndDrag}\hspace*{\fill}

\begin{list}{}{
\settowidth{\tmplength}{\textbf{Declaração}}
\setlength{\itemindent}{0cm}
\setlength{\listparindent}{0cm}
\setlength{\leftmargin}{\evensidemargin}
\addtolength{\leftmargin}{\tmplength}
\settowidth{\labelsep}{X}
\addtolength{\leftmargin}{\labelsep}
\setlength{\labelwidth}{\tmplength}
}
\begin{flushleft}
\item[\textbf{Declaração}\hfill]
\begin{ttfamily}
published property OnEndDrag;\end{ttfamily}


\end{flushleft}
\end{list}
\paragraph*{OnEnter}\hspace*{\fill}

\begin{list}{}{
\settowidth{\tmplength}{\textbf{Declaração}}
\setlength{\itemindent}{0cm}
\setlength{\listparindent}{0cm}
\setlength{\leftmargin}{\evensidemargin}
\addtolength{\leftmargin}{\tmplength}
\settowidth{\labelsep}{X}
\addtolength{\leftmargin}{\labelsep}
\setlength{\labelwidth}{\tmplength}
}
\begin{flushleft}
\item[\textbf{Declaração}\hfill]
\begin{ttfamily}
published property OnEnter;\end{ttfamily}


\end{flushleft}
\end{list}
\paragraph*{OnExit}\hspace*{\fill}

\begin{list}{}{
\settowidth{\tmplength}{\textbf{Declaração}}
\setlength{\itemindent}{0cm}
\setlength{\listparindent}{0cm}
\setlength{\leftmargin}{\evensidemargin}
\addtolength{\leftmargin}{\tmplength}
\settowidth{\labelsep}{X}
\addtolength{\leftmargin}{\labelsep}
\setlength{\labelwidth}{\tmplength}
}
\begin{flushleft}
\item[\textbf{Declaração}\hfill]
\begin{ttfamily}
published property OnExit;\end{ttfamily}


\end{flushleft}
\end{list}
\paragraph*{OnKeyDown}\hspace*{\fill}

\begin{list}{}{
\settowidth{\tmplength}{\textbf{Declaração}}
\setlength{\itemindent}{0cm}
\setlength{\listparindent}{0cm}
\setlength{\leftmargin}{\evensidemargin}
\addtolength{\leftmargin}{\tmplength}
\settowidth{\labelsep}{X}
\addtolength{\leftmargin}{\labelsep}
\setlength{\labelwidth}{\tmplength}
}
\begin{flushleft}
\item[\textbf{Declaração}\hfill]
\begin{ttfamily}
published property OnKeyDown;\end{ttfamily}


\end{flushleft}
\end{list}
\paragraph*{OnKeyPress}\hspace*{\fill}

\begin{list}{}{
\settowidth{\tmplength}{\textbf{Declaração}}
\setlength{\itemindent}{0cm}
\setlength{\listparindent}{0cm}
\setlength{\leftmargin}{\evensidemargin}
\addtolength{\leftmargin}{\tmplength}
\settowidth{\labelsep}{X}
\addtolength{\leftmargin}{\labelsep}
\setlength{\labelwidth}{\tmplength}
}
\begin{flushleft}
\item[\textbf{Declaração}\hfill]
\begin{ttfamily}
published property OnKeyPress;\end{ttfamily}


\end{flushleft}
\end{list}
\paragraph*{OnKeyUp}\hspace*{\fill}

\begin{list}{}{
\settowidth{\tmplength}{\textbf{Declaração}}
\setlength{\itemindent}{0cm}
\setlength{\listparindent}{0cm}
\setlength{\leftmargin}{\evensidemargin}
\addtolength{\leftmargin}{\tmplength}
\settowidth{\labelsep}{X}
\addtolength{\leftmargin}{\labelsep}
\setlength{\labelwidth}{\tmplength}
}
\begin{flushleft}
\item[\textbf{Declaração}\hfill]
\begin{ttfamily}
published property OnKeyUp;\end{ttfamily}


\end{flushleft}
\end{list}
\paragraph*{OnMeasureItem}\hspace*{\fill}

\begin{list}{}{
\settowidth{\tmplength}{\textbf{Declaração}}
\setlength{\itemindent}{0cm}
\setlength{\listparindent}{0cm}
\setlength{\leftmargin}{\evensidemargin}
\addtolength{\leftmargin}{\tmplength}
\settowidth{\labelsep}{X}
\addtolength{\leftmargin}{\labelsep}
\setlength{\labelwidth}{\tmplength}
}
\begin{flushleft}
\item[\textbf{Declaração}\hfill]
\begin{ttfamily}
published property OnMeasureItem;\end{ttfamily}


\end{flushleft}
\end{list}
\paragraph*{OnStartDrag}\hspace*{\fill}

\begin{list}{}{
\settowidth{\tmplength}{\textbf{Declaração}}
\setlength{\itemindent}{0cm}
\setlength{\listparindent}{0cm}
\setlength{\leftmargin}{\evensidemargin}
\addtolength{\leftmargin}{\tmplength}
\settowidth{\labelsep}{X}
\addtolength{\leftmargin}{\labelsep}
\setlength{\labelwidth}{\tmplength}
}
\begin{flushleft}
\item[\textbf{Declaração}\hfill]
\begin{ttfamily}
published property OnStartDrag;\end{ttfamily}


\end{flushleft}
\end{list}
\paragraph*{Anchors}\hspace*{\fill}

\begin{list}{}{
\settowidth{\tmplength}{\textbf{Declaração}}
\setlength{\itemindent}{0cm}
\setlength{\listparindent}{0cm}
\setlength{\leftmargin}{\evensidemargin}
\addtolength{\leftmargin}{\tmplength}
\settowidth{\labelsep}{X}
\addtolength{\leftmargin}{\labelsep}
\setlength{\labelwidth}{\tmplength}
}
\begin{flushleft}
\item[\textbf{Declaração}\hfill]
\begin{ttfamily}
published property Anchors;\end{ttfamily}


\end{flushleft}
\end{list}
\subsubsection*{\large{\textbf{Métodos}}\normalsize\hspace{1ex}\hfill}
\paragraph*{DrawItem}\hspace*{\fill}

\begin{list}{}{
\settowidth{\tmplength}{\textbf{Declaração}}
\setlength{\itemindent}{0cm}
\setlength{\listparindent}{0cm}
\setlength{\leftmargin}{\evensidemargin}
\addtolength{\leftmargin}{\tmplength}
\settowidth{\labelsep}{X}
\addtolength{\leftmargin}{\labelsep}
\setlength{\labelwidth}{\tmplength}
}
\begin{flushleft}
\item[\textbf{Declaração}\hfill]
\begin{ttfamily}
protected procedure DrawItem(Index: Integer; Rect: TRect; State: TOwnerDrawState); override;\end{ttfamily}


\end{flushleft}
\end{list}
\paragraph*{Create}\hspace*{\fill}

\begin{list}{}{
\settowidth{\tmplength}{\textbf{Declaração}}
\setlength{\itemindent}{0cm}
\setlength{\listparindent}{0cm}
\setlength{\leftmargin}{\evensidemargin}
\addtolength{\leftmargin}{\tmplength}
\settowidth{\labelsep}{X}
\addtolength{\leftmargin}{\labelsep}
\setlength{\labelwidth}{\tmplength}
}
\begin{flushleft}
\item[\textbf{Declaração}\hfill]
\begin{ttfamily}
public constructor Create(AOwner:TComponent); override; overload;\end{ttfamily}


\end{flushleft}
\end{list}
\paragraph*{Create}\hspace*{\fill}

\begin{list}{}{
\settowidth{\tmplength}{\textbf{Declaração}}
\setlength{\itemindent}{0cm}
\setlength{\listparindent}{0cm}
\setlength{\leftmargin}{\evensidemargin}
\addtolength{\leftmargin}{\tmplength}
\settowidth{\labelsep}{X}
\addtolength{\leftmargin}{\labelsep}
\setlength{\labelwidth}{\tmplength}
}
\begin{flushleft}
\item[\textbf{Declaração}\hfill]
\begin{ttfamily}
public constructor Create(aOwner:TComponent;aDmxScroller{\_}Form{\_}Lcl{\_}attributes: TDmxScroller{\_}Form{\_}Lcl{\_}attributes); overload; overload;\end{ttfamily}


\end{flushleft}
\end{list}
\paragraph*{Destroy}\hspace*{\fill}

\begin{list}{}{
\settowidth{\tmplength}{\textbf{Declaração}}
\setlength{\itemindent}{0cm}
\setlength{\listparindent}{0cm}
\setlength{\leftmargin}{\evensidemargin}
\addtolength{\leftmargin}{\tmplength}
\settowidth{\labelsep}{X}
\addtolength{\leftmargin}{\labelsep}
\setlength{\labelwidth}{\tmplength}
}
\begin{flushleft}
\item[\textbf{Declaração}\hfill]
\begin{ttfamily}
public destructor Destroy; override;\end{ttfamily}


\end{flushleft}
\end{list}
\paragraph*{PutBuffer}\hspace*{\fill}

\begin{list}{}{
\settowidth{\tmplength}{\textbf{Declaração}}
\setlength{\itemindent}{0cm}
\setlength{\listparindent}{0cm}
\setlength{\leftmargin}{\evensidemargin}
\addtolength{\leftmargin}{\tmplength}
\settowidth{\labelsep}{X}
\addtolength{\leftmargin}{\labelsep}
\setlength{\labelwidth}{\tmplength}
}
\begin{flushleft}
\item[\textbf{Declaração}\hfill]
\begin{ttfamily}
public Procedure PutBuffer;\end{ttfamily}


\end{flushleft}
\par
\item[\textbf{Descrição}]
O método \textbf{\begin{ttfamily}PutBuffer\end{ttfamily}} transfere os dados do controle para o componente TMI{\_}rtl{\_}ui{\_}DmxScroller. \begin{itemize}
\item \textbf{NOTA} \begin{itemize}
\item Quando \textbf{pDmxFieldRec.ListComboBox} {$<$}{$>$} nil usar \textbf{\begin{ttfamily}Value\end{ttfamily}(\ref{uMi_ui_ComboBox_lcl.TMI_ComboBox_LCL-Value})} e se \textbf{pDmxFieldRec.ListComboBox}=nil usar \begin{ttfamily}ItemIndex\end{ttfamily}(\ref{uMi_ui_ComboBox_lcl.TMI_ComboBox_LCL-ItemIndex}).
\item A propriedade \textbf{\begin{ttfamily}Value\end{ttfamily}(\ref{uMi_ui_ComboBox_lcl.TMI_ComboBox_LCL-Value})} pode ser qualquer valor.
\end{itemize}
\end{itemize}

\end{list}
\paragraph*{GetBuffer}\hspace*{\fill}

\begin{list}{}{
\settowidth{\tmplength}{\textbf{Declaração}}
\setlength{\itemindent}{0cm}
\setlength{\listparindent}{0cm}
\setlength{\leftmargin}{\evensidemargin}
\addtolength{\leftmargin}{\tmplength}
\settowidth{\labelsep}{X}
\addtolength{\leftmargin}{\labelsep}
\setlength{\labelwidth}{\tmplength}
}
\begin{flushleft}
\item[\textbf{Declaração}\hfill]
\begin{ttfamily}
public Procedure GetBuffer;\end{ttfamily}


\end{flushleft}
\par
\item[\textbf{Descrição}]
O método \textbf{\begin{ttfamily}GetBuffer\end{ttfamily}} transfere os dados do controle para o componente TMI{\_}rtl{\_}ui{\_}DmxScroller. \begin{itemize}
\item \textbf{NOTA} \begin{itemize}
\item Quando \textbf{pDmxFieldRec.ListComboBox} {$<$}{$>$} nil usar \textbf{\begin{ttfamily}Value\end{ttfamily}(\ref{uMi_ui_ComboBox_lcl.TMI_ComboBox_LCL-Value})} e se \textbf{pDmxFieldRec.ListComboBox}=nil usar \begin{ttfamily}ItemIndex\end{ttfamily}(\ref{uMi_ui_ComboBox_lcl.TMI_ComboBox_LCL-ItemIndex}).
\end{itemize}
\end{itemize}

\end{list}
\paragraph*{DoOnMouseDown}\hspace*{\fill}

\begin{list}{}{
\settowidth{\tmplength}{\textbf{Declaração}}
\setlength{\itemindent}{0cm}
\setlength{\listparindent}{0cm}
\setlength{\leftmargin}{\evensidemargin}
\addtolength{\leftmargin}{\tmplength}
\settowidth{\labelsep}{X}
\addtolength{\leftmargin}{\labelsep}
\setlength{\labelwidth}{\tmplength}
}
\begin{flushleft}
\item[\textbf{Declaração}\hfill]
\begin{ttfamily}
protected procedure DoOnMouseDown(Sender: TObject; Button: TMouseButton; Shift: TShiftState; X, Y: Integer);\end{ttfamily}


\end{flushleft}
\end{list}
\paragraph*{DoOnEnter}\hspace*{\fill}

\begin{list}{}{
\settowidth{\tmplength}{\textbf{Declaração}}
\setlength{\itemindent}{0cm}
\setlength{\listparindent}{0cm}
\setlength{\leftmargin}{\evensidemargin}
\addtolength{\leftmargin}{\tmplength}
\settowidth{\labelsep}{X}
\addtolength{\leftmargin}{\labelsep}
\setlength{\labelwidth}{\tmplength}
}
\begin{flushleft}
\item[\textbf{Declaração}\hfill]
\begin{ttfamily}
protected procedure DoOnEnter(Sender: TObject);\end{ttfamily}


\end{flushleft}
\end{list}
\paragraph*{DoOnExit}\hspace*{\fill}

\begin{list}{}{
\settowidth{\tmplength}{\textbf{Declaração}}
\setlength{\itemindent}{0cm}
\setlength{\listparindent}{0cm}
\setlength{\leftmargin}{\evensidemargin}
\addtolength{\leftmargin}{\tmplength}
\settowidth{\labelsep}{X}
\addtolength{\leftmargin}{\labelsep}
\setlength{\labelwidth}{\tmplength}
}
\begin{flushleft}
\item[\textbf{Declaração}\hfill]
\begin{ttfamily}
protected procedure DoOnExit(Sender: TObject);\end{ttfamily}


\end{flushleft}
\end{list}
\paragraph*{Select}\hspace*{\fill}

\begin{list}{}{
\settowidth{\tmplength}{\textbf{Declaração}}
\setlength{\itemindent}{0cm}
\setlength{\listparindent}{0cm}
\setlength{\leftmargin}{\evensidemargin}
\addtolength{\leftmargin}{\tmplength}
\settowidth{\labelsep}{X}
\addtolength{\leftmargin}{\labelsep}
\setlength{\labelwidth}{\tmplength}
}
\begin{flushleft}
\item[\textbf{Declaração}\hfill]
\begin{ttfamily}
protected procedure Select; override;\end{ttfamily}


\end{flushleft}
\end{list}
\paragraph*{DoOnKeyPress}\hspace*{\fill}

\begin{list}{}{
\settowidth{\tmplength}{\textbf{Declaração}}
\setlength{\itemindent}{0cm}
\setlength{\listparindent}{0cm}
\setlength{\leftmargin}{\evensidemargin}
\addtolength{\leftmargin}{\tmplength}
\settowidth{\labelsep}{X}
\addtolength{\leftmargin}{\labelsep}
\setlength{\labelwidth}{\tmplength}
}
\begin{flushleft}
\item[\textbf{Declaração}\hfill]
\begin{ttfamily}
protected procedure DoOnKeyPress(Sender: TObject; var Key: system.Char);\end{ttfamily}


\end{flushleft}
\end{list}
\paragraph*{Clear}\hspace*{\fill}

\begin{list}{}{
\settowidth{\tmplength}{\textbf{Declaração}}
\setlength{\itemindent}{0cm}
\setlength{\listparindent}{0cm}
\setlength{\leftmargin}{\evensidemargin}
\addtolength{\leftmargin}{\tmplength}
\settowidth{\labelsep}{X}
\addtolength{\leftmargin}{\labelsep}
\setlength{\labelwidth}{\tmplength}
}
\begin{flushleft}
\item[\textbf{Declaração}\hfill]
\begin{ttfamily}
public procedure Clear; Override;\end{ttfamily}


\end{flushleft}
\end{list}
\paragraph*{AddValue}\hspace*{\fill}

\begin{list}{}{
\settowidth{\tmplength}{\textbf{Declaração}}
\setlength{\itemindent}{0cm}
\setlength{\listparindent}{0cm}
\setlength{\leftmargin}{\evensidemargin}
\addtolength{\leftmargin}{\tmplength}
\settowidth{\labelsep}{X}
\addtolength{\leftmargin}{\labelsep}
\setlength{\labelwidth}{\tmplength}
}
\begin{flushleft}
\item[\textbf{Declaração}\hfill]
\begin{ttfamily}
public procedure AddValue(aString:String);\end{ttfamily}


\end{flushleft}
\end{list}
\paragraph*{WMPaint}\hspace*{\fill}

\begin{list}{}{
\settowidth{\tmplength}{\textbf{Declaração}}
\setlength{\itemindent}{0cm}
\setlength{\listparindent}{0cm}
\setlength{\leftmargin}{\evensidemargin}
\addtolength{\leftmargin}{\tmplength}
\settowidth{\labelsep}{X}
\addtolength{\leftmargin}{\labelsep}
\setlength{\labelwidth}{\tmplength}
}
\begin{flushleft}
\item[\textbf{Declaração}\hfill]
\begin{ttfamily}
protected procedure WMPaint(var Message: TLMPaint); message LM{\_}PAINT;\end{ttfamily}


\end{flushleft}
\end{list}
\section{Funções e Procedimentos}
\subsection*{Register}
\begin{list}{}{
\settowidth{\tmplength}{\textbf{Declaração}}
\setlength{\itemindent}{0cm}
\setlength{\listparindent}{0cm}
\setlength{\leftmargin}{\evensidemargin}
\addtolength{\leftmargin}{\tmplength}
\settowidth{\labelsep}{X}
\addtolength{\leftmargin}{\labelsep}
\setlength{\labelwidth}{\tmplength}
}
\begin{flushleft}
\item[\textbf{Declaração}\hfill]
\begin{ttfamily}
procedure Register;\end{ttfamily}


\end{flushleft}
\end{list}
\chapter{Unit uMi{\_}Ui{\_}DBCheckBox{\_}Lcl}
\section{Uses}
\begin{itemize}
\item \begin{ttfamily}Classes\end{ttfamily}\item \begin{ttfamily}SysUtils\end{ttfamily}\item \begin{ttfamily}LResources\end{ttfamily}\item \begin{ttfamily}Forms\end{ttfamily}\item \begin{ttfamily}Controls\end{ttfamily}\item \begin{ttfamily}Graphics\end{ttfamily}\item \begin{ttfamily}Dialogs\end{ttfamily}\item \begin{ttfamily}DBCtrls\end{ttfamily}\item \begin{ttfamily}ActnList\end{ttfamily}\item \begin{ttfamily}mi{\_}rtl{\_}ui{\_}DmxScroller\end{ttfamily}(\ref{mi_rtl_ui_Dmxscroller})\item \begin{ttfamily}mi{\_}rtl{\_}ui{\_}DmxScroller{\_}Form\end{ttfamily}(\ref{mi_rtl_ui_dmxscroller_form})\item \begin{ttfamily}umi{\_}ui{\_}dmxscroller{\_}form{\_}lcl{\_}attributes\end{ttfamily}(\ref{umi_ui_dmxscroller_form_lcl_attributes})\end{itemize}
\section{Visão Geral}
\begin{description}
\item[\texttt{\begin{ttfamily}TMi{\_}Ui{\_}DBCheckBox{\_}Lcl\end{ttfamily} Classe}]
\end{description}
\begin{description}
\item[\texttt{Register}]
\end{description}
\section{Classes, Interfaces, Objetos e Registros}
\subsection*{TMi{\_}Ui{\_}DBCheckBox{\_}Lcl Classe}
\subsubsection*{\large{\textbf{Hierarquia}}\normalsize\hspace{1ex}\hfill}
TMi{\_}Ui{\_}DBCheckBox{\_}Lcl {$>$} TDBCheckBox
\subsubsection*{\large{\textbf{Descrição}}\normalsize\hspace{1ex}\hfill}
A classe \textbf{\begin{ttfamily}TMi{\_}Ui{\_}DBCheckBox{\_}Lcl\end{ttfamily}} permite edita um campo boolean do registro \textbf{\begin{ttfamily}TDmxFieldRec\end{ttfamily}(\ref{mi_rtl_ui_Dmxscroller.TDmxFieldRec})}

\begin{itemize}
\item \textbf{NOTA} {-}
\end{itemize}\subsubsection*{\large{\textbf{Propriedades}}\normalsize\hspace{1ex}\hfill}
\paragraph*{DmxScroller{\_}Form{\_}Lcl{\_}attributes}\hspace*{\fill}

\begin{list}{}{
\settowidth{\tmplength}{\textbf{Declaração}}
\setlength{\itemindent}{0cm}
\setlength{\listparindent}{0cm}
\setlength{\leftmargin}{\evensidemargin}
\addtolength{\leftmargin}{\tmplength}
\settowidth{\labelsep}{X}
\addtolength{\leftmargin}{\labelsep}
\setlength{\labelwidth}{\tmplength}
}
\begin{flushleft}
\item[\textbf{Declaração}\hfill]
\begin{ttfamily}
published property DmxScroller{\_}Form{\_}Lcl{\_}attributes : TDmxScroller{\_}Form{\_}Lcl{\_}attributes Read {\_}DmxScroller{\_}Form{\_}Lcl{\_}attributes write SetDmxScroller{\_}Form{\_}Lcl{\_}attributes;\end{ttfamily}


\end{flushleft}
\end{list}
\paragraph*{DmxFieldRec}\hspace*{\fill}

\begin{list}{}{
\settowidth{\tmplength}{\textbf{Declaração}}
\setlength{\itemindent}{0cm}
\setlength{\listparindent}{0cm}
\setlength{\leftmargin}{\evensidemargin}
\addtolength{\leftmargin}{\tmplength}
\settowidth{\labelsep}{X}
\addtolength{\leftmargin}{\labelsep}
\setlength{\labelwidth}{\tmplength}
}
\begin{flushleft}
\item[\textbf{Declaração}\hfill]
\begin{ttfamily}
public property DmxFieldRec: pDmxFieldRec Read {\_}pDmxFieldRec   Write  SeTDmxFieldRec;\end{ttfamily}


\end{flushleft}
\par
\item[\textbf{Descrição}]
O atributo \textbf{\begin{ttfamily}DmxFieldRec\end{ttfamily}} fornece os dados necessários para criar o componente \begin{ttfamily}TMI{\_}MaskEdit{\_}LCL\end{ttfamily}(\ref{uMi_ui_maskedit_lcl.TMI_MaskEdit_LCL}).

\begin{itemize}
\item \textbf{NOTA} \begin{itemize}
\item Esses dados devem ser criados pelo método TDmxScroller{\_}Form{\_}Lcl{\_}attributes.CreateStruct(var ATemplate : \begin{ttfamily}TString\end{ttfamily}(\ref{mi_rtl_ui_Dmxscroller-tString}))
\end{itemize}
\end{itemize}

\end{list}
\subsubsection*{\large{\textbf{Métodos}}\normalsize\hspace{1ex}\hfill}
\paragraph*{Create}\hspace*{\fill}

\begin{list}{}{
\settowidth{\tmplength}{\textbf{Declaração}}
\setlength{\itemindent}{0cm}
\setlength{\listparindent}{0cm}
\setlength{\leftmargin}{\evensidemargin}
\addtolength{\leftmargin}{\tmplength}
\settowidth{\labelsep}{X}
\addtolength{\leftmargin}{\labelsep}
\setlength{\labelwidth}{\tmplength}
}
\begin{flushleft}
\item[\textbf{Declaração}\hfill]
\begin{ttfamily}
public constructor Create(AOwner:TComponent); override;\end{ttfamily}


\end{flushleft}
\end{list}
\paragraph*{DoOnEnter}\hspace*{\fill}

\begin{list}{}{
\settowidth{\tmplength}{\textbf{Declaração}}
\setlength{\itemindent}{0cm}
\setlength{\listparindent}{0cm}
\setlength{\leftmargin}{\evensidemargin}
\addtolength{\leftmargin}{\tmplength}
\settowidth{\labelsep}{X}
\addtolength{\leftmargin}{\labelsep}
\setlength{\labelwidth}{\tmplength}
}
\begin{flushleft}
\item[\textbf{Declaração}\hfill]
\begin{ttfamily}
protected procedure DoOnEnter(Sender: TObject);\end{ttfamily}


\end{flushleft}
\end{list}
\paragraph*{DoOnExit}\hspace*{\fill}

\begin{list}{}{
\settowidth{\tmplength}{\textbf{Declaração}}
\setlength{\itemindent}{0cm}
\setlength{\listparindent}{0cm}
\setlength{\leftmargin}{\evensidemargin}
\addtolength{\leftmargin}{\tmplength}
\settowidth{\labelsep}{X}
\addtolength{\leftmargin}{\labelsep}
\setlength{\labelwidth}{\tmplength}
}
\begin{flushleft}
\item[\textbf{Declaração}\hfill]
\begin{ttfamily}
protected procedure DoOnExit(Sender: TObject);\end{ttfamily}


\end{flushleft}
\par
\item[\textbf{Descrição}]
O método \textbf{\begin{ttfamily}DoOnExit\end{ttfamily}} ao perder o foco executa os métodos PuttBuffer e pDmxFieldRec.DoOnExit(Self).

\end{list}
\paragraph*{PutBuffer}\hspace*{\fill}

\begin{list}{}{
\settowidth{\tmplength}{\textbf{Declaração}}
\setlength{\itemindent}{0cm}
\setlength{\listparindent}{0cm}
\setlength{\leftmargin}{\evensidemargin}
\addtolength{\leftmargin}{\tmplength}
\settowidth{\labelsep}{X}
\addtolength{\leftmargin}{\labelsep}
\setlength{\labelwidth}{\tmplength}
}
\begin{flushleft}
\item[\textbf{Declaração}\hfill]
\begin{ttfamily}
public Procedure PutBuffer;\end{ttfamily}


\end{flushleft}
\par
\item[\textbf{Descrição}]
O método \textbf{\begin{ttfamily}PutBuffer\end{ttfamily}} salva os dados do controle (Self) para a propriedade \begin{ttfamily}pDmxFieldRec\end{ttfamily}(\ref{mi_rtl_ui_Dmxscroller-pDmxFieldRec})

\end{list}
\paragraph*{GetBuffer}\hspace*{\fill}

\begin{list}{}{
\settowidth{\tmplength}{\textbf{Declaração}}
\setlength{\itemindent}{0cm}
\setlength{\listparindent}{0cm}
\setlength{\leftmargin}{\evensidemargin}
\addtolength{\leftmargin}{\tmplength}
\settowidth{\labelsep}{X}
\addtolength{\leftmargin}{\labelsep}
\setlength{\labelwidth}{\tmplength}
}
\begin{flushleft}
\item[\textbf{Declaração}\hfill]
\begin{ttfamily}
public Procedure GetBuffer;\end{ttfamily}


\end{flushleft}
\par
\item[\textbf{Descrição}]
O método \textbf{\begin{ttfamily}GetBuffer\end{ttfamily}} ler os dados da propriedade \begin{ttfamily}pDmxFieldRec\end{ttfamily}(\ref{mi_rtl_ui_Dmxscroller-pDmxFieldRec}) para o controle (Self).

\end{list}
\section{Funções e Procedimentos}
\subsection*{Register}
\begin{list}{}{
\settowidth{\tmplength}{\textbf{Declaração}}
\setlength{\itemindent}{0cm}
\setlength{\listparindent}{0cm}
\setlength{\leftmargin}{\evensidemargin}
\addtolength{\leftmargin}{\tmplength}
\settowidth{\labelsep}{X}
\addtolength{\leftmargin}{\labelsep}
\setlength{\labelwidth}{\tmplength}
}
\begin{flushleft}
\item[\textbf{Declaração}\hfill]
\begin{ttfamily}
procedure Register;\end{ttfamily}


\end{flushleft}
\end{list}
\chapter{Unit uMi{\_}Ui{\_}DbComboBox{\_}lcl}
\section{Uses}
\begin{itemize}
\item \begin{ttfamily}Windows\end{ttfamily}\item \begin{ttfamily}Classes\end{ttfamily}\item \begin{ttfamily}SysUtils\end{ttfamily}\item \begin{ttfamily}LResources\end{ttfamily}\item \begin{ttfamily}Forms\end{ttfamily}\item \begin{ttfamily}Controls\end{ttfamily}\item \begin{ttfamily}Graphics\end{ttfamily}\item \begin{ttfamily}Dialogs\end{ttfamily}\item \begin{ttfamily}DBCtrls\end{ttfamily}\item \begin{ttfamily}StdCtrls\end{ttfamily}\item \begin{ttfamily}mi{\_}rtl{\_}ui{\_}DmxScroller\end{ttfamily}(\ref{mi_rtl_ui_Dmxscroller})\item \begin{ttfamily}mi{\_}rtl{\_}ui{\_}DmxScroller{\_}Form\end{ttfamily}(\ref{mi_rtl_ui_dmxscroller_form})\item \begin{ttfamily}umi{\_}ui{\_}dmxscroller{\_}form{\_}lcl{\_}attributes\end{ttfamily}(\ref{umi_ui_dmxscroller_form_lcl_attributes})\end{itemize}
\section{Visão Geral}
\begin{description}
\item[\texttt{\begin{ttfamily}TMi{\_}DbComboBox{\_}LCL\end{ttfamily} Classe}]
\end{description}
\begin{description}
\item[\texttt{Register}]
\end{description}
\section{Classes, Interfaces, Objetos e Registros}
\subsection*{TMi{\_}DbComboBox{\_}LCL Classe}
\subsubsection*{\large{\textbf{Hierarquia}}\normalsize\hspace{1ex}\hfill}
TMi{\_}DbComboBox{\_}LCL {$>$} TDBComboBox
\subsubsection*{\large{\textbf{Descrição}}\normalsize\hspace{1ex}\hfill}
A classe \textbf{\begin{ttfamily}TMi{\_}DbComboBox{\_}LCL\end{ttfamily}} permite edita um campo enumerado do registro \textbf{\begin{ttfamily}TDmxFieldRec\end{ttfamily}(\ref{mi_rtl_ui_Dmxscroller.TDmxFieldRec})}

\begin{itemize}
\item \textbf{NOTA} \begin{itemize}
\item O item zero contém a string selecionada e caso a mesma seja editada o valor digitado passa ser o filtro de pesquisa.
\end{itemize}
\end{itemize}\subsubsection*{\large{\textbf{Propriedades}}\normalsize\hspace{1ex}\hfill}
\paragraph*{DmxScroller{\_}Form{\_}Lcl{\_}attributes}\hspace*{\fill}

\begin{list}{}{
\settowidth{\tmplength}{\textbf{Declaração}}
\setlength{\itemindent}{0cm}
\setlength{\listparindent}{0cm}
\setlength{\leftmargin}{\evensidemargin}
\addtolength{\leftmargin}{\tmplength}
\settowidth{\labelsep}{X}
\addtolength{\leftmargin}{\labelsep}
\setlength{\labelwidth}{\tmplength}
}
\begin{flushleft}
\item[\textbf{Declaração}\hfill]
\begin{ttfamily}
published property DmxScroller{\_}Form{\_}Lcl{\_}attributes : TDmxScroller{\_}Form{\_}Lcl{\_}attributes Read {\_}DmxScroller{\_}Form{\_}Lcl{\_}attributes write SetDmxScroller{\_}Form{\_}Lcl{\_}attributes;\end{ttfamily}


\end{flushleft}
\end{list}
\paragraph*{DmxFieldRec}\hspace*{\fill}

\begin{list}{}{
\settowidth{\tmplength}{\textbf{Declaração}}
\setlength{\itemindent}{0cm}
\setlength{\listparindent}{0cm}
\setlength{\leftmargin}{\evensidemargin}
\addtolength{\leftmargin}{\tmplength}
\settowidth{\labelsep}{X}
\addtolength{\leftmargin}{\labelsep}
\setlength{\labelwidth}{\tmplength}
}
\begin{flushleft}
\item[\textbf{Declaração}\hfill]
\begin{ttfamily}
public property DmxFieldRec: pDmxFieldRec Read {\_}pDmxFieldRec   Write  SeTDmxFieldRec;\end{ttfamily}


\end{flushleft}
\par
\item[\textbf{Descrição}]
O atributo \textbf{\begin{ttfamily}DmxFieldRec\end{ttfamily}} fornece os dados necessários para criar o componente \begin{ttfamily}TMI{\_}MaskEdit{\_}LCL\end{ttfamily}(\ref{uMi_ui_maskedit_lcl.TMI_MaskEdit_LCL}).

\begin{itemize}
\item \textbf{NOTA} \begin{itemize}
\item Esses dados devem ser criados pelo método TDmxScroller{\_}Form{\_}Lcl{\_}attributes.CreateStruct(var ATemplate : \begin{ttfamily}TString\end{ttfamily}(\ref{mi_rtl_ui_Dmxscroller-tString}))
\end{itemize}
\end{itemize}

\end{list}
\paragraph*{Value}\hspace*{\fill}

\begin{list}{}{
\settowidth{\tmplength}{\textbf{Declaração}}
\setlength{\itemindent}{0cm}
\setlength{\listparindent}{0cm}
\setlength{\leftmargin}{\evensidemargin}
\addtolength{\leftmargin}{\tmplength}
\settowidth{\labelsep}{X}
\addtolength{\leftmargin}{\labelsep}
\setlength{\labelwidth}{\tmplength}
}
\begin{flushleft}
\item[\textbf{Declaração}\hfill]
\begin{ttfamily}
public property Value: String read GetValue write SetValue;\end{ttfamily}


\end{flushleft}
\end{list}
\paragraph*{ImgIndexes}\hspace*{\fill}

\begin{list}{}{
\settowidth{\tmplength}{\textbf{Declaração}}
\setlength{\itemindent}{0cm}
\setlength{\listparindent}{0cm}
\setlength{\leftmargin}{\evensidemargin}
\addtolength{\leftmargin}{\tmplength}
\settowidth{\labelsep}{X}
\addtolength{\leftmargin}{\labelsep}
\setlength{\labelwidth}{\tmplength}
}
\begin{flushleft}
\item[\textbf{Declaração}\hfill]
\begin{ttfamily}
published property ImgIndexes: TStringList read FImgIndexes write SetImgIndexes;\end{ttfamily}


\end{flushleft}
\end{list}
\paragraph*{Images}\hspace*{\fill}

\begin{list}{}{
\settowidth{\tmplength}{\textbf{Declaração}}
\setlength{\itemindent}{0cm}
\setlength{\listparindent}{0cm}
\setlength{\leftmargin}{\evensidemargin}
\addtolength{\leftmargin}{\tmplength}
\settowidth{\labelsep}{X}
\addtolength{\leftmargin}{\labelsep}
\setlength{\labelwidth}{\tmplength}
}
\begin{flushleft}
\item[\textbf{Declaração}\hfill]
\begin{ttfamily}
published property Images: TImageList read FImages write SetImages;\end{ttfamily}


\end{flushleft}
\end{list}
\paragraph*{ShowImages}\hspace*{\fill}

\begin{list}{}{
\settowidth{\tmplength}{\textbf{Declaração}}
\setlength{\itemindent}{0cm}
\setlength{\listparindent}{0cm}
\setlength{\leftmargin}{\evensidemargin}
\addtolength{\leftmargin}{\tmplength}
\settowidth{\labelsep}{X}
\addtolength{\leftmargin}{\labelsep}
\setlength{\labelwidth}{\tmplength}
}
\begin{flushleft}
\item[\textbf{Declaração}\hfill]
\begin{ttfamily}
published property ShowImages: Boolean read FShowImages write SetShowImages;\end{ttfamily}


\end{flushleft}
\end{list}
\paragraph*{Color}\hspace*{\fill}

\begin{list}{}{
\settowidth{\tmplength}{\textbf{Declaração}}
\setlength{\itemindent}{0cm}
\setlength{\listparindent}{0cm}
\setlength{\leftmargin}{\evensidemargin}
\addtolength{\leftmargin}{\tmplength}
\settowidth{\labelsep}{X}
\addtolength{\leftmargin}{\labelsep}
\setlength{\labelwidth}{\tmplength}
}
\begin{flushleft}
\item[\textbf{Declaração}\hfill]
\begin{ttfamily}
published property Color;\end{ttfamily}


\end{flushleft}
\end{list}
\paragraph*{Align}\hspace*{\fill}

\begin{list}{}{
\settowidth{\tmplength}{\textbf{Declaração}}
\setlength{\itemindent}{0cm}
\setlength{\listparindent}{0cm}
\setlength{\leftmargin}{\evensidemargin}
\addtolength{\leftmargin}{\tmplength}
\settowidth{\labelsep}{X}
\addtolength{\leftmargin}{\labelsep}
\setlength{\labelwidth}{\tmplength}
}
\begin{flushleft}
\item[\textbf{Declaração}\hfill]
\begin{ttfamily}
published property Align;\end{ttfamily}


\end{flushleft}
\end{list}
\paragraph*{AutoComplete}\hspace*{\fill}

\begin{list}{}{
\settowidth{\tmplength}{\textbf{Declaração}}
\setlength{\itemindent}{0cm}
\setlength{\listparindent}{0cm}
\setlength{\leftmargin}{\evensidemargin}
\addtolength{\leftmargin}{\tmplength}
\settowidth{\labelsep}{X}
\addtolength{\leftmargin}{\labelsep}
\setlength{\labelwidth}{\tmplength}
}
\begin{flushleft}
\item[\textbf{Declaração}\hfill]
\begin{ttfamily}
published property AutoComplete;\end{ttfamily}


\end{flushleft}
\end{list}
\paragraph*{AutoDropDown}\hspace*{\fill}

\begin{list}{}{
\settowidth{\tmplength}{\textbf{Declaração}}
\setlength{\itemindent}{0cm}
\setlength{\listparindent}{0cm}
\setlength{\leftmargin}{\evensidemargin}
\addtolength{\leftmargin}{\tmplength}
\settowidth{\labelsep}{X}
\addtolength{\leftmargin}{\labelsep}
\setlength{\labelwidth}{\tmplength}
}
\begin{flushleft}
\item[\textbf{Declaração}\hfill]
\begin{ttfamily}
published property AutoDropDown;\end{ttfamily}


\end{flushleft}
\end{list}
\paragraph*{AutoSelect}\hspace*{\fill}

\begin{list}{}{
\settowidth{\tmplength}{\textbf{Declaração}}
\setlength{\itemindent}{0cm}
\setlength{\listparindent}{0cm}
\setlength{\leftmargin}{\evensidemargin}
\addtolength{\leftmargin}{\tmplength}
\settowidth{\labelsep}{X}
\addtolength{\leftmargin}{\labelsep}
\setlength{\labelwidth}{\tmplength}
}
\begin{flushleft}
\item[\textbf{Declaração}\hfill]
\begin{ttfamily}
published property AutoSelect;\end{ttfamily}


\end{flushleft}
\end{list}
\paragraph*{OnEditingDone}\hspace*{\fill}

\begin{list}{}{
\settowidth{\tmplength}{\textbf{Declaração}}
\setlength{\itemindent}{0cm}
\setlength{\listparindent}{0cm}
\setlength{\leftmargin}{\evensidemargin}
\addtolength{\leftmargin}{\tmplength}
\settowidth{\labelsep}{X}
\addtolength{\leftmargin}{\labelsep}
\setlength{\labelwidth}{\tmplength}
}
\begin{flushleft}
\item[\textbf{Declaração}\hfill]
\begin{ttfamily}
published property OnEditingDone;\end{ttfamily}


\end{flushleft}
\end{list}
\paragraph*{AutoSize}\hspace*{\fill}

\begin{list}{}{
\settowidth{\tmplength}{\textbf{Declaração}}
\setlength{\itemindent}{0cm}
\setlength{\listparindent}{0cm}
\setlength{\leftmargin}{\evensidemargin}
\addtolength{\leftmargin}{\tmplength}
\settowidth{\labelsep}{X}
\addtolength{\leftmargin}{\labelsep}
\setlength{\labelwidth}{\tmplength}
}
\begin{flushleft}
\item[\textbf{Declaração}\hfill]
\begin{ttfamily}
published property AutoSize;\end{ttfamily}


\end{flushleft}
\end{list}
\paragraph*{text}\hspace*{\fill}

\begin{list}{}{
\settowidth{\tmplength}{\textbf{Declaração}}
\setlength{\itemindent}{0cm}
\setlength{\listparindent}{0cm}
\setlength{\leftmargin}{\evensidemargin}
\addtolength{\leftmargin}{\tmplength}
\settowidth{\labelsep}{X}
\addtolength{\leftmargin}{\labelsep}
\setlength{\labelwidth}{\tmplength}
}
\begin{flushleft}
\item[\textbf{Declaração}\hfill]
\begin{ttfamily}
published property text;\end{ttfamily}


\end{flushleft}
\end{list}
\paragraph*{ItemIndex}\hspace*{\fill}

\begin{list}{}{
\settowidth{\tmplength}{\textbf{Declaração}}
\setlength{\itemindent}{0cm}
\setlength{\listparindent}{0cm}
\setlength{\leftmargin}{\evensidemargin}
\addtolength{\leftmargin}{\tmplength}
\settowidth{\labelsep}{X}
\addtolength{\leftmargin}{\labelsep}
\setlength{\labelwidth}{\tmplength}
}
\begin{flushleft}
\item[\textbf{Declaração}\hfill]
\begin{ttfamily}
published property ItemIndex;\end{ttfamily}


\end{flushleft}
\end{list}
\paragraph*{DragMode}\hspace*{\fill}

\begin{list}{}{
\settowidth{\tmplength}{\textbf{Declaração}}
\setlength{\itemindent}{0cm}
\setlength{\listparindent}{0cm}
\setlength{\leftmargin}{\evensidemargin}
\addtolength{\leftmargin}{\tmplength}
\settowidth{\labelsep}{X}
\addtolength{\leftmargin}{\labelsep}
\setlength{\labelwidth}{\tmplength}
}
\begin{flushleft}
\item[\textbf{Declaração}\hfill]
\begin{ttfamily}
published property DragMode;\end{ttfamily}


\end{flushleft}
\end{list}
\paragraph*{DragCursor}\hspace*{\fill}

\begin{list}{}{
\settowidth{\tmplength}{\textbf{Declaração}}
\setlength{\itemindent}{0cm}
\setlength{\listparindent}{0cm}
\setlength{\leftmargin}{\evensidemargin}
\addtolength{\leftmargin}{\tmplength}
\settowidth{\labelsep}{X}
\addtolength{\leftmargin}{\labelsep}
\setlength{\labelwidth}{\tmplength}
}
\begin{flushleft}
\item[\textbf{Declaração}\hfill]
\begin{ttfamily}
published property DragCursor;\end{ttfamily}


\end{flushleft}
\end{list}
\paragraph*{DropDownCount}\hspace*{\fill}

\begin{list}{}{
\settowidth{\tmplength}{\textbf{Declaração}}
\setlength{\itemindent}{0cm}
\setlength{\listparindent}{0cm}
\setlength{\leftmargin}{\evensidemargin}
\addtolength{\leftmargin}{\tmplength}
\settowidth{\labelsep}{X}
\addtolength{\leftmargin}{\labelsep}
\setlength{\labelwidth}{\tmplength}
}
\begin{flushleft}
\item[\textbf{Declaração}\hfill]
\begin{ttfamily}
published property DropDownCount;\end{ttfamily}


\end{flushleft}
\end{list}
\paragraph*{Enabled}\hspace*{\fill}

\begin{list}{}{
\settowidth{\tmplength}{\textbf{Declaração}}
\setlength{\itemindent}{0cm}
\setlength{\listparindent}{0cm}
\setlength{\leftmargin}{\evensidemargin}
\addtolength{\leftmargin}{\tmplength}
\settowidth{\labelsep}{X}
\addtolength{\leftmargin}{\labelsep}
\setlength{\labelwidth}{\tmplength}
}
\begin{flushleft}
\item[\textbf{Declaração}\hfill]
\begin{ttfamily}
published property Enabled;\end{ttfamily}


\end{flushleft}
\end{list}
\paragraph*{Font}\hspace*{\fill}

\begin{list}{}{
\settowidth{\tmplength}{\textbf{Declaração}}
\setlength{\itemindent}{0cm}
\setlength{\listparindent}{0cm}
\setlength{\leftmargin}{\evensidemargin}
\addtolength{\leftmargin}{\tmplength}
\settowidth{\labelsep}{X}
\addtolength{\leftmargin}{\labelsep}
\setlength{\labelwidth}{\tmplength}
}
\begin{flushleft}
\item[\textbf{Declaração}\hfill]
\begin{ttfamily}
published property Font;\end{ttfamily}


\end{flushleft}
\end{list}
\paragraph*{ItemHeight}\hspace*{\fill}

\begin{list}{}{
\settowidth{\tmplength}{\textbf{Declaração}}
\setlength{\itemindent}{0cm}
\setlength{\listparindent}{0cm}
\setlength{\leftmargin}{\evensidemargin}
\addtolength{\leftmargin}{\tmplength}
\settowidth{\labelsep}{X}
\addtolength{\leftmargin}{\labelsep}
\setlength{\labelwidth}{\tmplength}
}
\begin{flushleft}
\item[\textbf{Declaração}\hfill]
\begin{ttfamily}
published property ItemHeight;\end{ttfamily}


\end{flushleft}
\end{list}
\paragraph*{Items}\hspace*{\fill}

\begin{list}{}{
\settowidth{\tmplength}{\textbf{Declaração}}
\setlength{\itemindent}{0cm}
\setlength{\listparindent}{0cm}
\setlength{\leftmargin}{\evensidemargin}
\addtolength{\leftmargin}{\tmplength}
\settowidth{\labelsep}{X}
\addtolength{\leftmargin}{\labelsep}
\setlength{\labelwidth}{\tmplength}
}
\begin{flushleft}
\item[\textbf{Declaração}\hfill]
\begin{ttfamily}
published property Items;\end{ttfamily}


\end{flushleft}
\end{list}
\paragraph*{MaxLength}\hspace*{\fill}

\begin{list}{}{
\settowidth{\tmplength}{\textbf{Declaração}}
\setlength{\itemindent}{0cm}
\setlength{\listparindent}{0cm}
\setlength{\leftmargin}{\evensidemargin}
\addtolength{\leftmargin}{\tmplength}
\settowidth{\labelsep}{X}
\addtolength{\leftmargin}{\labelsep}
\setlength{\labelwidth}{\tmplength}
}
\begin{flushleft}
\item[\textbf{Declaração}\hfill]
\begin{ttfamily}
published property MaxLength;\end{ttfamily}


\end{flushleft}
\end{list}
\paragraph*{ParentColor}\hspace*{\fill}

\begin{list}{}{
\settowidth{\tmplength}{\textbf{Declaração}}
\setlength{\itemindent}{0cm}
\setlength{\listparindent}{0cm}
\setlength{\leftmargin}{\evensidemargin}
\addtolength{\leftmargin}{\tmplength}
\settowidth{\labelsep}{X}
\addtolength{\leftmargin}{\labelsep}
\setlength{\labelwidth}{\tmplength}
}
\begin{flushleft}
\item[\textbf{Declaração}\hfill]
\begin{ttfamily}
published property ParentColor;\end{ttfamily}


\end{flushleft}
\end{list}
\paragraph*{ParentFont}\hspace*{\fill}

\begin{list}{}{
\settowidth{\tmplength}{\textbf{Declaração}}
\setlength{\itemindent}{0cm}
\setlength{\listparindent}{0cm}
\setlength{\leftmargin}{\evensidemargin}
\addtolength{\leftmargin}{\tmplength}
\settowidth{\labelsep}{X}
\addtolength{\leftmargin}{\labelsep}
\setlength{\labelwidth}{\tmplength}
}
\begin{flushleft}
\item[\textbf{Declaração}\hfill]
\begin{ttfamily}
published property ParentFont;\end{ttfamily}


\end{flushleft}
\end{list}
\paragraph*{ParentShowHint}\hspace*{\fill}

\begin{list}{}{
\settowidth{\tmplength}{\textbf{Declaração}}
\setlength{\itemindent}{0cm}
\setlength{\listparindent}{0cm}
\setlength{\leftmargin}{\evensidemargin}
\addtolength{\leftmargin}{\tmplength}
\settowidth{\labelsep}{X}
\addtolength{\leftmargin}{\labelsep}
\setlength{\labelwidth}{\tmplength}
}
\begin{flushleft}
\item[\textbf{Declaração}\hfill]
\begin{ttfamily}
published property ParentShowHint;\end{ttfamily}


\end{flushleft}
\end{list}
\paragraph*{PopupMenu}\hspace*{\fill}

\begin{list}{}{
\settowidth{\tmplength}{\textbf{Declaração}}
\setlength{\itemindent}{0cm}
\setlength{\listparindent}{0cm}
\setlength{\leftmargin}{\evensidemargin}
\addtolength{\leftmargin}{\tmplength}
\settowidth{\labelsep}{X}
\addtolength{\leftmargin}{\labelsep}
\setlength{\labelwidth}{\tmplength}
}
\begin{flushleft}
\item[\textbf{Declaração}\hfill]
\begin{ttfamily}
published property PopupMenu;\end{ttfamily}


\end{flushleft}
\end{list}
\paragraph*{ShowHint}\hspace*{\fill}

\begin{list}{}{
\settowidth{\tmplength}{\textbf{Declaração}}
\setlength{\itemindent}{0cm}
\setlength{\listparindent}{0cm}
\setlength{\leftmargin}{\evensidemargin}
\addtolength{\leftmargin}{\tmplength}
\settowidth{\labelsep}{X}
\addtolength{\leftmargin}{\labelsep}
\setlength{\labelwidth}{\tmplength}
}
\begin{flushleft}
\item[\textbf{Declaração}\hfill]
\begin{ttfamily}
published property ShowHint;\end{ttfamily}


\end{flushleft}
\end{list}
\paragraph*{Sorted}\hspace*{\fill}

\begin{list}{}{
\settowidth{\tmplength}{\textbf{Declaração}}
\setlength{\itemindent}{0cm}
\setlength{\listparindent}{0cm}
\setlength{\leftmargin}{\evensidemargin}
\addtolength{\leftmargin}{\tmplength}
\settowidth{\labelsep}{X}
\addtolength{\leftmargin}{\labelsep}
\setlength{\labelwidth}{\tmplength}
}
\begin{flushleft}
\item[\textbf{Declaração}\hfill]
\begin{ttfamily}
published property Sorted;\end{ttfamily}


\end{flushleft}
\end{list}
\paragraph*{TabOrder}\hspace*{\fill}

\begin{list}{}{
\settowidth{\tmplength}{\textbf{Declaração}}
\setlength{\itemindent}{0cm}
\setlength{\listparindent}{0cm}
\setlength{\leftmargin}{\evensidemargin}
\addtolength{\leftmargin}{\tmplength}
\settowidth{\labelsep}{X}
\addtolength{\leftmargin}{\labelsep}
\setlength{\labelwidth}{\tmplength}
}
\begin{flushleft}
\item[\textbf{Declaração}\hfill]
\begin{ttfamily}
published property TabOrder;\end{ttfamily}


\end{flushleft}
\end{list}
\paragraph*{TabStop}\hspace*{\fill}

\begin{list}{}{
\settowidth{\tmplength}{\textbf{Declaração}}
\setlength{\itemindent}{0cm}
\setlength{\listparindent}{0cm}
\setlength{\leftmargin}{\evensidemargin}
\addtolength{\leftmargin}{\tmplength}
\settowidth{\labelsep}{X}
\addtolength{\leftmargin}{\labelsep}
\setlength{\labelwidth}{\tmplength}
}
\begin{flushleft}
\item[\textbf{Declaração}\hfill]
\begin{ttfamily}
published property TabStop;\end{ttfamily}


\end{flushleft}
\end{list}
\paragraph*{Visible}\hspace*{\fill}

\begin{list}{}{
\settowidth{\tmplength}{\textbf{Declaração}}
\setlength{\itemindent}{0cm}
\setlength{\listparindent}{0cm}
\setlength{\leftmargin}{\evensidemargin}
\addtolength{\leftmargin}{\tmplength}
\settowidth{\labelsep}{X}
\addtolength{\leftmargin}{\labelsep}
\setlength{\labelwidth}{\tmplength}
}
\begin{flushleft}
\item[\textbf{Declaração}\hfill]
\begin{ttfamily}
published property Visible;\end{ttfamily}


\end{flushleft}
\end{list}
\paragraph*{OnChange}\hspace*{\fill}

\begin{list}{}{
\settowidth{\tmplength}{\textbf{Declaração}}
\setlength{\itemindent}{0cm}
\setlength{\listparindent}{0cm}
\setlength{\leftmargin}{\evensidemargin}
\addtolength{\leftmargin}{\tmplength}
\settowidth{\labelsep}{X}
\addtolength{\leftmargin}{\labelsep}
\setlength{\labelwidth}{\tmplength}
}
\begin{flushleft}
\item[\textbf{Declaração}\hfill]
\begin{ttfamily}
published property OnChange;\end{ttfamily}


\end{flushleft}
\end{list}
\paragraph*{OnClick}\hspace*{\fill}

\begin{list}{}{
\settowidth{\tmplength}{\textbf{Declaração}}
\setlength{\itemindent}{0cm}
\setlength{\listparindent}{0cm}
\setlength{\leftmargin}{\evensidemargin}
\addtolength{\leftmargin}{\tmplength}
\settowidth{\labelsep}{X}
\addtolength{\leftmargin}{\labelsep}
\setlength{\labelwidth}{\tmplength}
}
\begin{flushleft}
\item[\textbf{Declaração}\hfill]
\begin{ttfamily}
published property OnClick;\end{ttfamily}


\end{flushleft}
\end{list}
\paragraph*{OnDblClick}\hspace*{\fill}

\begin{list}{}{
\settowidth{\tmplength}{\textbf{Declaração}}
\setlength{\itemindent}{0cm}
\setlength{\listparindent}{0cm}
\setlength{\leftmargin}{\evensidemargin}
\addtolength{\leftmargin}{\tmplength}
\settowidth{\labelsep}{X}
\addtolength{\leftmargin}{\labelsep}
\setlength{\labelwidth}{\tmplength}
}
\begin{flushleft}
\item[\textbf{Declaração}\hfill]
\begin{ttfamily}
published property OnDblClick;\end{ttfamily}


\end{flushleft}
\end{list}
\paragraph*{OnDragDrop}\hspace*{\fill}

\begin{list}{}{
\settowidth{\tmplength}{\textbf{Declaração}}
\setlength{\itemindent}{0cm}
\setlength{\listparindent}{0cm}
\setlength{\leftmargin}{\evensidemargin}
\addtolength{\leftmargin}{\tmplength}
\settowidth{\labelsep}{X}
\addtolength{\leftmargin}{\labelsep}
\setlength{\labelwidth}{\tmplength}
}
\begin{flushleft}
\item[\textbf{Declaração}\hfill]
\begin{ttfamily}
published property OnDragDrop;\end{ttfamily}


\end{flushleft}
\end{list}
\paragraph*{OnDragOver}\hspace*{\fill}

\begin{list}{}{
\settowidth{\tmplength}{\textbf{Declaração}}
\setlength{\itemindent}{0cm}
\setlength{\listparindent}{0cm}
\setlength{\leftmargin}{\evensidemargin}
\addtolength{\leftmargin}{\tmplength}
\settowidth{\labelsep}{X}
\addtolength{\leftmargin}{\labelsep}
\setlength{\labelwidth}{\tmplength}
}
\begin{flushleft}
\item[\textbf{Declaração}\hfill]
\begin{ttfamily}
published property OnDragOver;\end{ttfamily}


\end{flushleft}
\end{list}
\paragraph*{OnDrawItem}\hspace*{\fill}

\begin{list}{}{
\settowidth{\tmplength}{\textbf{Declaração}}
\setlength{\itemindent}{0cm}
\setlength{\listparindent}{0cm}
\setlength{\leftmargin}{\evensidemargin}
\addtolength{\leftmargin}{\tmplength}
\settowidth{\labelsep}{X}
\addtolength{\leftmargin}{\labelsep}
\setlength{\labelwidth}{\tmplength}
}
\begin{flushleft}
\item[\textbf{Declaração}\hfill]
\begin{ttfamily}
published property OnDrawItem;\end{ttfamily}


\end{flushleft}
\end{list}
\paragraph*{OnDropDown}\hspace*{\fill}

\begin{list}{}{
\settowidth{\tmplength}{\textbf{Declaração}}
\setlength{\itemindent}{0cm}
\setlength{\listparindent}{0cm}
\setlength{\leftmargin}{\evensidemargin}
\addtolength{\leftmargin}{\tmplength}
\settowidth{\labelsep}{X}
\addtolength{\leftmargin}{\labelsep}
\setlength{\labelwidth}{\tmplength}
}
\begin{flushleft}
\item[\textbf{Declaração}\hfill]
\begin{ttfamily}
published property OnDropDown;\end{ttfamily}


\end{flushleft}
\end{list}
\paragraph*{OnEndDrag}\hspace*{\fill}

\begin{list}{}{
\settowidth{\tmplength}{\textbf{Declaração}}
\setlength{\itemindent}{0cm}
\setlength{\listparindent}{0cm}
\setlength{\leftmargin}{\evensidemargin}
\addtolength{\leftmargin}{\tmplength}
\settowidth{\labelsep}{X}
\addtolength{\leftmargin}{\labelsep}
\setlength{\labelwidth}{\tmplength}
}
\begin{flushleft}
\item[\textbf{Declaração}\hfill]
\begin{ttfamily}
published property OnEndDrag;\end{ttfamily}


\end{flushleft}
\end{list}
\paragraph*{OnEnter}\hspace*{\fill}

\begin{list}{}{
\settowidth{\tmplength}{\textbf{Declaração}}
\setlength{\itemindent}{0cm}
\setlength{\listparindent}{0cm}
\setlength{\leftmargin}{\evensidemargin}
\addtolength{\leftmargin}{\tmplength}
\settowidth{\labelsep}{X}
\addtolength{\leftmargin}{\labelsep}
\setlength{\labelwidth}{\tmplength}
}
\begin{flushleft}
\item[\textbf{Declaração}\hfill]
\begin{ttfamily}
published property OnEnter;\end{ttfamily}


\end{flushleft}
\end{list}
\paragraph*{OnExit}\hspace*{\fill}

\begin{list}{}{
\settowidth{\tmplength}{\textbf{Declaração}}
\setlength{\itemindent}{0cm}
\setlength{\listparindent}{0cm}
\setlength{\leftmargin}{\evensidemargin}
\addtolength{\leftmargin}{\tmplength}
\settowidth{\labelsep}{X}
\addtolength{\leftmargin}{\labelsep}
\setlength{\labelwidth}{\tmplength}
}
\begin{flushleft}
\item[\textbf{Declaração}\hfill]
\begin{ttfamily}
published property OnExit;\end{ttfamily}


\end{flushleft}
\end{list}
\paragraph*{OnKeyDown}\hspace*{\fill}

\begin{list}{}{
\settowidth{\tmplength}{\textbf{Declaração}}
\setlength{\itemindent}{0cm}
\setlength{\listparindent}{0cm}
\setlength{\leftmargin}{\evensidemargin}
\addtolength{\leftmargin}{\tmplength}
\settowidth{\labelsep}{X}
\addtolength{\leftmargin}{\labelsep}
\setlength{\labelwidth}{\tmplength}
}
\begin{flushleft}
\item[\textbf{Declaração}\hfill]
\begin{ttfamily}
published property OnKeyDown;\end{ttfamily}


\end{flushleft}
\end{list}
\paragraph*{OnKeyPress}\hspace*{\fill}

\begin{list}{}{
\settowidth{\tmplength}{\textbf{Declaração}}
\setlength{\itemindent}{0cm}
\setlength{\listparindent}{0cm}
\setlength{\leftmargin}{\evensidemargin}
\addtolength{\leftmargin}{\tmplength}
\settowidth{\labelsep}{X}
\addtolength{\leftmargin}{\labelsep}
\setlength{\labelwidth}{\tmplength}
}
\begin{flushleft}
\item[\textbf{Declaração}\hfill]
\begin{ttfamily}
published property OnKeyPress;\end{ttfamily}


\end{flushleft}
\end{list}
\paragraph*{OnKeyUp}\hspace*{\fill}

\begin{list}{}{
\settowidth{\tmplength}{\textbf{Declaração}}
\setlength{\itemindent}{0cm}
\setlength{\listparindent}{0cm}
\setlength{\leftmargin}{\evensidemargin}
\addtolength{\leftmargin}{\tmplength}
\settowidth{\labelsep}{X}
\addtolength{\leftmargin}{\labelsep}
\setlength{\labelwidth}{\tmplength}
}
\begin{flushleft}
\item[\textbf{Declaração}\hfill]
\begin{ttfamily}
published property OnKeyUp;\end{ttfamily}


\end{flushleft}
\end{list}
\paragraph*{OnMeasureItem}\hspace*{\fill}

\begin{list}{}{
\settowidth{\tmplength}{\textbf{Declaração}}
\setlength{\itemindent}{0cm}
\setlength{\listparindent}{0cm}
\setlength{\leftmargin}{\evensidemargin}
\addtolength{\leftmargin}{\tmplength}
\settowidth{\labelsep}{X}
\addtolength{\leftmargin}{\labelsep}
\setlength{\labelwidth}{\tmplength}
}
\begin{flushleft}
\item[\textbf{Declaração}\hfill]
\begin{ttfamily}
published property OnMeasureItem;\end{ttfamily}


\end{flushleft}
\end{list}
\paragraph*{OnStartDrag}\hspace*{\fill}

\begin{list}{}{
\settowidth{\tmplength}{\textbf{Declaração}}
\setlength{\itemindent}{0cm}
\setlength{\listparindent}{0cm}
\setlength{\leftmargin}{\evensidemargin}
\addtolength{\leftmargin}{\tmplength}
\settowidth{\labelsep}{X}
\addtolength{\leftmargin}{\labelsep}
\setlength{\labelwidth}{\tmplength}
}
\begin{flushleft}
\item[\textbf{Declaração}\hfill]
\begin{ttfamily}
published property OnStartDrag;\end{ttfamily}


\end{flushleft}
\end{list}
\paragraph*{Anchors}\hspace*{\fill}

\begin{list}{}{
\settowidth{\tmplength}{\textbf{Declaração}}
\setlength{\itemindent}{0cm}
\setlength{\listparindent}{0cm}
\setlength{\leftmargin}{\evensidemargin}
\addtolength{\leftmargin}{\tmplength}
\settowidth{\labelsep}{X}
\addtolength{\leftmargin}{\labelsep}
\setlength{\labelwidth}{\tmplength}
}
\begin{flushleft}
\item[\textbf{Declaração}\hfill]
\begin{ttfamily}
published property Anchors;\end{ttfamily}


\end{flushleft}
\end{list}
\subsubsection*{\large{\textbf{Métodos}}\normalsize\hspace{1ex}\hfill}
\paragraph*{DrawItem}\hspace*{\fill}

\begin{list}{}{
\settowidth{\tmplength}{\textbf{Declaração}}
\setlength{\itemindent}{0cm}
\setlength{\listparindent}{0cm}
\setlength{\leftmargin}{\evensidemargin}
\addtolength{\leftmargin}{\tmplength}
\settowidth{\labelsep}{X}
\addtolength{\leftmargin}{\labelsep}
\setlength{\labelwidth}{\tmplength}
}
\begin{flushleft}
\item[\textbf{Declaração}\hfill]
\begin{ttfamily}
protected procedure DrawItem(Index: Integer; Rect: TRect; State: TOwnerDrawState); override;\end{ttfamily}


\end{flushleft}
\end{list}
\paragraph*{Create}\hspace*{\fill}

\begin{list}{}{
\settowidth{\tmplength}{\textbf{Declaração}}
\setlength{\itemindent}{0cm}
\setlength{\listparindent}{0cm}
\setlength{\leftmargin}{\evensidemargin}
\addtolength{\leftmargin}{\tmplength}
\settowidth{\labelsep}{X}
\addtolength{\leftmargin}{\labelsep}
\setlength{\labelwidth}{\tmplength}
}
\begin{flushleft}
\item[\textbf{Declaração}\hfill]
\begin{ttfamily}
public constructor Create(AOwner:TComponent); override; overload;\end{ttfamily}


\end{flushleft}
\end{list}
\paragraph*{Create}\hspace*{\fill}

\begin{list}{}{
\settowidth{\tmplength}{\textbf{Declaração}}
\setlength{\itemindent}{0cm}
\setlength{\listparindent}{0cm}
\setlength{\leftmargin}{\evensidemargin}
\addtolength{\leftmargin}{\tmplength}
\settowidth{\labelsep}{X}
\addtolength{\leftmargin}{\labelsep}
\setlength{\labelwidth}{\tmplength}
}
\begin{flushleft}
\item[\textbf{Declaração}\hfill]
\begin{ttfamily}
public constructor Create(aOwner:TComponent;aDmxScroller{\_}Form{\_}Lcl{\_}attributes : TDmxScroller{\_}Form{\_}Lcl{\_}attributes); overload; overload;\end{ttfamily}


\end{flushleft}
\end{list}
\paragraph*{Destroy}\hspace*{\fill}

\begin{list}{}{
\settowidth{\tmplength}{\textbf{Declaração}}
\setlength{\itemindent}{0cm}
\setlength{\listparindent}{0cm}
\setlength{\leftmargin}{\evensidemargin}
\addtolength{\leftmargin}{\tmplength}
\settowidth{\labelsep}{X}
\addtolength{\leftmargin}{\labelsep}
\setlength{\labelwidth}{\tmplength}
}
\begin{flushleft}
\item[\textbf{Declaração}\hfill]
\begin{ttfamily}
public destructor Destroy; override;\end{ttfamily}


\end{flushleft}
\end{list}
\paragraph*{PutBuffer}\hspace*{\fill}

\begin{list}{}{
\settowidth{\tmplength}{\textbf{Declaração}}
\setlength{\itemindent}{0cm}
\setlength{\listparindent}{0cm}
\setlength{\leftmargin}{\evensidemargin}
\addtolength{\leftmargin}{\tmplength}
\settowidth{\labelsep}{X}
\addtolength{\leftmargin}{\labelsep}
\setlength{\labelwidth}{\tmplength}
}
\begin{flushleft}
\item[\textbf{Declaração}\hfill]
\begin{ttfamily}
public Procedure PutBuffer;\end{ttfamily}


\end{flushleft}
\end{list}
\paragraph*{GetBuffer}\hspace*{\fill}

\begin{list}{}{
\settowidth{\tmplength}{\textbf{Declaração}}
\setlength{\itemindent}{0cm}
\setlength{\listparindent}{0cm}
\setlength{\leftmargin}{\evensidemargin}
\addtolength{\leftmargin}{\tmplength}
\settowidth{\labelsep}{X}
\addtolength{\leftmargin}{\labelsep}
\setlength{\labelwidth}{\tmplength}
}
\begin{flushleft}
\item[\textbf{Declaração}\hfill]
\begin{ttfamily}
public Procedure GetBuffer;\end{ttfamily}


\end{flushleft}
\end{list}
\paragraph*{DoOnMouseDown}\hspace*{\fill}

\begin{list}{}{
\settowidth{\tmplength}{\textbf{Declaração}}
\setlength{\itemindent}{0cm}
\setlength{\listparindent}{0cm}
\setlength{\leftmargin}{\evensidemargin}
\addtolength{\leftmargin}{\tmplength}
\settowidth{\labelsep}{X}
\addtolength{\leftmargin}{\labelsep}
\setlength{\labelwidth}{\tmplength}
}
\begin{flushleft}
\item[\textbf{Declaração}\hfill]
\begin{ttfamily}
protected procedure DoOnMouseDown(Sender: TObject; Button: TMouseButton; Shift: TShiftState; X, Y: Integer);\end{ttfamily}


\end{flushleft}
\end{list}
\paragraph*{DoOnEnter}\hspace*{\fill}

\begin{list}{}{
\settowidth{\tmplength}{\textbf{Declaração}}
\setlength{\itemindent}{0cm}
\setlength{\listparindent}{0cm}
\setlength{\leftmargin}{\evensidemargin}
\addtolength{\leftmargin}{\tmplength}
\settowidth{\labelsep}{X}
\addtolength{\leftmargin}{\labelsep}
\setlength{\labelwidth}{\tmplength}
}
\begin{flushleft}
\item[\textbf{Declaração}\hfill]
\begin{ttfamily}
protected procedure DoOnEnter(Sender: TObject);\end{ttfamily}


\end{flushleft}
\end{list}
\paragraph*{DoOnExit}\hspace*{\fill}

\begin{list}{}{
\settowidth{\tmplength}{\textbf{Declaração}}
\setlength{\itemindent}{0cm}
\setlength{\listparindent}{0cm}
\setlength{\leftmargin}{\evensidemargin}
\addtolength{\leftmargin}{\tmplength}
\settowidth{\labelsep}{X}
\addtolength{\leftmargin}{\labelsep}
\setlength{\labelwidth}{\tmplength}
}
\begin{flushleft}
\item[\textbf{Declaração}\hfill]
\begin{ttfamily}
protected procedure DoOnExit(Sender: TObject);\end{ttfamily}


\end{flushleft}
\end{list}
\paragraph*{Select}\hspace*{\fill}

\begin{list}{}{
\settowidth{\tmplength}{\textbf{Declaração}}
\setlength{\itemindent}{0cm}
\setlength{\listparindent}{0cm}
\setlength{\leftmargin}{\evensidemargin}
\addtolength{\leftmargin}{\tmplength}
\settowidth{\labelsep}{X}
\addtolength{\leftmargin}{\labelsep}
\setlength{\labelwidth}{\tmplength}
}
\begin{flushleft}
\item[\textbf{Declaração}\hfill]
\begin{ttfamily}
protected procedure Select; override;\end{ttfamily}


\end{flushleft}
\end{list}
\paragraph*{DoOnKeyPress}\hspace*{\fill}

\begin{list}{}{
\settowidth{\tmplength}{\textbf{Declaração}}
\setlength{\itemindent}{0cm}
\setlength{\listparindent}{0cm}
\setlength{\leftmargin}{\evensidemargin}
\addtolength{\leftmargin}{\tmplength}
\settowidth{\labelsep}{X}
\addtolength{\leftmargin}{\labelsep}
\setlength{\labelwidth}{\tmplength}
}
\begin{flushleft}
\item[\textbf{Declaração}\hfill]
\begin{ttfamily}
protected procedure DoOnKeyPress(Sender: TObject; var Key: system.Char);\end{ttfamily}


\end{flushleft}
\end{list}
\paragraph*{Clear}\hspace*{\fill}

\begin{list}{}{
\settowidth{\tmplength}{\textbf{Declaração}}
\setlength{\itemindent}{0cm}
\setlength{\listparindent}{0cm}
\setlength{\leftmargin}{\evensidemargin}
\addtolength{\leftmargin}{\tmplength}
\settowidth{\labelsep}{X}
\addtolength{\leftmargin}{\labelsep}
\setlength{\labelwidth}{\tmplength}
}
\begin{flushleft}
\item[\textbf{Declaração}\hfill]
\begin{ttfamily}
public procedure Clear; Override;\end{ttfamily}


\end{flushleft}
\end{list}
\paragraph*{AddValue}\hspace*{\fill}

\begin{list}{}{
\settowidth{\tmplength}{\textbf{Declaração}}
\setlength{\itemindent}{0cm}
\setlength{\listparindent}{0cm}
\setlength{\leftmargin}{\evensidemargin}
\addtolength{\leftmargin}{\tmplength}
\settowidth{\labelsep}{X}
\addtolength{\leftmargin}{\labelsep}
\setlength{\labelwidth}{\tmplength}
}
\begin{flushleft}
\item[\textbf{Declaração}\hfill]
\begin{ttfamily}
public procedure AddValue(aString:String);\end{ttfamily}


\end{flushleft}
\end{list}
\paragraph*{WMPaint}\hspace*{\fill}

\begin{list}{}{
\settowidth{\tmplength}{\textbf{Declaração}}
\setlength{\itemindent}{0cm}
\setlength{\listparindent}{0cm}
\setlength{\leftmargin}{\evensidemargin}
\addtolength{\leftmargin}{\tmplength}
\settowidth{\labelsep}{X}
\addtolength{\leftmargin}{\labelsep}
\setlength{\labelwidth}{\tmplength}
}
\begin{flushleft}
\item[\textbf{Declaração}\hfill]
\begin{ttfamily}
protected procedure WMPaint(var Message: TLMPaint); message LM{\_}PAINT;\end{ttfamily}


\end{flushleft}
\end{list}
\section{Funções e Procedimentos}
\subsection*{Register}
\begin{list}{}{
\settowidth{\tmplength}{\textbf{Declaração}}
\setlength{\itemindent}{0cm}
\setlength{\listparindent}{0cm}
\setlength{\leftmargin}{\evensidemargin}
\addtolength{\leftmargin}{\tmplength}
\settowidth{\labelsep}{X}
\addtolength{\leftmargin}{\labelsep}
\setlength{\labelwidth}{\tmplength}
}
\begin{flushleft}
\item[\textbf{Declaração}\hfill]
\begin{ttfamily}
procedure Register;\end{ttfamily}


\end{flushleft}
\end{list}
\chapter{Unit uMI{\_}ui{\_}DbEdit{\_}LCL}
\section{Uses}
\begin{itemize}
\item \begin{ttfamily}Classes\end{ttfamily}\item \begin{ttfamily}SysUtils\end{ttfamily}\item \begin{ttfamily}LResources\end{ttfamily}\item \begin{ttfamily}Forms\end{ttfamily}\item \begin{ttfamily}Controls\end{ttfamily}\item \begin{ttfamily}Graphics\end{ttfamily}\item \begin{ttfamily}Dialogs\end{ttfamily}\item \begin{ttfamily}StdCtrls\end{ttfamily}\item \begin{ttfamily}DBCtrls\end{ttfamily}\item \begin{ttfamily}LMessages\end{ttfamily}\item \begin{ttfamily}LCLType\end{ttfamily}\item \begin{ttfamily}db\end{ttfamily}\item \begin{ttfamily}Variants\end{ttfamily}\item \begin{ttfamily}mi.rtl.objects.Methods.dates\end{ttfamily}(\ref{mi.rtl.objects.Methods.dates})\item \begin{ttfamily}mi{\_}rtl{\_}ui{\_}DmxScroller\end{ttfamily}(\ref{mi_rtl_ui_Dmxscroller})\item \begin{ttfamily}mi{\_}rtl{\_}ui{\_}DmxScroller{\_}Form\end{ttfamily}(\ref{mi_rtl_ui_dmxscroller_form})\item \begin{ttfamily}umi{\_}ui{\_}dmxscroller{\_}form{\_}lcl{\_}attributes\end{ttfamily}(\ref{umi_ui_dmxscroller_form_lcl_attributes})\end{itemize}
\section{Visão Geral}
\begin{description}
\item[\texttt{\begin{ttfamily}TMI{\_}DbEdit{\_}LCL\end{ttfamily} Classe}]
\end{description}
\begin{description}
\item[\texttt{Register}]
\end{description}
\section{Classes, Interfaces, Objetos e Registros}
\subsection*{TMI{\_}DbEdit{\_}LCL Classe}
\subsubsection*{\large{\textbf{Hierarquia}}\normalsize\hspace{1ex}\hfill}
TMI{\_}DbEdit{\_}LCL {$>$} TDBEdit
%%%%Descrição
\subsubsection*{\large{\textbf{Propriedades}}\normalsize\hspace{1ex}\hfill}
\paragraph*{DmxScroller{\_}Form{\_}Lcl{\_}attributes}\hspace*{\fill}

\begin{list}{}{
\settowidth{\tmplength}{\textbf{Declaração}}
\setlength{\itemindent}{0cm}
\setlength{\listparindent}{0cm}
\setlength{\leftmargin}{\evensidemargin}
\addtolength{\leftmargin}{\tmplength}
\settowidth{\labelsep}{X}
\addtolength{\leftmargin}{\labelsep}
\setlength{\labelwidth}{\tmplength}
}
\begin{flushleft}
\item[\textbf{Declaração}\hfill]
\begin{ttfamily}
public property DmxScroller{\_}Form{\_}Lcl{\_}attributes : TDmxScroller{\_}Form{\_}Lcl{\_}attributes Read {\_}DmxScroller{\_}Form{\_}Lcl{\_}attributes write SetDmxScroller{\_}Form{\_}Lcl{\_}attributes;\end{ttfamily}


\end{flushleft}
\par
\item[\textbf{Descrição}]
A propriedade \textbf{\begin{ttfamily}DmxScroller{\_}Form{\_}Lcl{\_}attributes\end{ttfamily}} contém o modelo e os cálculos do formulário a ser criado em owner

\end{list}
\paragraph*{DmxFieldRec}\hspace*{\fill}

\begin{list}{}{
\settowidth{\tmplength}{\textbf{Declaração}}
\setlength{\itemindent}{0cm}
\setlength{\listparindent}{0cm}
\setlength{\leftmargin}{\evensidemargin}
\addtolength{\leftmargin}{\tmplength}
\settowidth{\labelsep}{X}
\addtolength{\leftmargin}{\labelsep}
\setlength{\labelwidth}{\tmplength}
}
\begin{flushleft}
\item[\textbf{Declaração}\hfill]
\begin{ttfamily}
public property DmxFieldRec: pDmxFieldRec Read {\_}pDmxFieldRec   Write  SeTDmxFieldRec;\end{ttfamily}


\end{flushleft}
\par
\item[\textbf{Descrição}]
O atributo \textbf{\begin{ttfamily}DmxFieldRec\end{ttfamily}} fornece os dados necessários para criar o componente \begin{ttfamily}TMI{\_}DbEdit{\_}LCL\end{ttfamily}(\ref{uMI_ui_DbEdit_LCL.TMI_DbEdit_LCL}).

\begin{itemize}
\item \textbf{NOTA} \begin{itemize}
\item Esses dados devem ser criados pelo método DmxScroller{\_}Form{\_}Lcl{\_}attributesr.CreateStruct(var ATemplate : \begin{ttfamily}TString\end{ttfamily}(\ref{mi_rtl_ui_Dmxscroller-tString}))
\end{itemize}
\end{itemize}

\end{list}
\paragraph*{DisplayFormat}\hspace*{\fill}

\begin{list}{}{
\settowidth{\tmplength}{\textbf{Declaração}}
\setlength{\itemindent}{0cm}
\setlength{\listparindent}{0cm}
\setlength{\leftmargin}{\evensidemargin}
\addtolength{\leftmargin}{\tmplength}
\settowidth{\labelsep}{X}
\addtolength{\leftmargin}{\labelsep}
\setlength{\labelwidth}{\tmplength}
}
\begin{flushleft}
\item[\textbf{Declaração}\hfill]
\begin{ttfamily}
public property DisplayFormat : AnsiString Read {\_}DisplayFormat;\end{ttfamily}


\end{flushleft}
\par
\item[\textbf{Descrição}]
A propriedade \textbf{\begin{ttfamily}DisplayFormat\end{ttfamily}} a mascara de saída usada para setar Field.DisplayFormat

\end{list}
\paragraph*{MaskEdit}\hspace*{\fill}

\begin{list}{}{
\settowidth{\tmplength}{\textbf{Declaração}}
\setlength{\itemindent}{0cm}
\setlength{\listparindent}{0cm}
\setlength{\leftmargin}{\evensidemargin}
\addtolength{\leftmargin}{\tmplength}
\settowidth{\labelsep}{X}
\addtolength{\leftmargin}{\labelsep}
\setlength{\labelwidth}{\tmplength}
}
\begin{flushleft}
\item[\textbf{Declaração}\hfill]
\begin{ttfamily}
public property MaskEdit : AnsiString Read {\_}MaskEdit write SetMaskEdit;\end{ttfamily}


\end{flushleft}
\par
\item[\textbf{Descrição}]
A propriedade \textbf{\begin{ttfamily}MaskEdit\end{ttfamily}} contém o modelo e os cálculos do formulário a ser criado em owner

\end{list}
\subsubsection*{\large{\textbf{Métodos}}\normalsize\hspace{1ex}\hfill}
\paragraph*{Create}\hspace*{\fill}

\begin{list}{}{
\settowidth{\tmplength}{\textbf{Declaração}}
\setlength{\itemindent}{0cm}
\setlength{\listparindent}{0cm}
\setlength{\leftmargin}{\evensidemargin}
\addtolength{\leftmargin}{\tmplength}
\settowidth{\labelsep}{X}
\addtolength{\leftmargin}{\labelsep}
\setlength{\labelwidth}{\tmplength}
}
\begin{flushleft}
\item[\textbf{Declaração}\hfill]
\begin{ttfamily}
public constructor Create(AOwner:TComponent); overload; override;\end{ttfamily}


\end{flushleft}
\end{list}
\paragraph*{Create}\hspace*{\fill}

\begin{list}{}{
\settowidth{\tmplength}{\textbf{Declaração}}
\setlength{\itemindent}{0cm}
\setlength{\listparindent}{0cm}
\setlength{\leftmargin}{\evensidemargin}
\addtolength{\leftmargin}{\tmplength}
\settowidth{\labelsep}{X}
\addtolength{\leftmargin}{\labelsep}
\setlength{\labelwidth}{\tmplength}
}
\begin{flushleft}
\item[\textbf{Declaração}\hfill]
\begin{ttfamily}
public constructor Create(aOwner:TComponent;aDmxScroller{\_}Form{\_}Lcl{\_}attributes : TDmxScroller{\_}Form{\_}Lcl{\_}attributes); overload;\end{ttfamily}


\end{flushleft}
\end{list}
\paragraph*{SeTDmxFieldRec}\hspace*{\fill}

\begin{list}{}{
\settowidth{\tmplength}{\textbf{Declaração}}
\setlength{\itemindent}{0cm}
\setlength{\listparindent}{0cm}
\setlength{\leftmargin}{\evensidemargin}
\addtolength{\leftmargin}{\tmplength}
\settowidth{\labelsep}{X}
\addtolength{\leftmargin}{\labelsep}
\setlength{\labelwidth}{\tmplength}
}
\begin{flushleft}
\item[\textbf{Declaração}\hfill]
\begin{ttfamily}
protected Procedure SeTDmxFieldRec(apDmxFieldRec : pDmxFieldRec ); Overload;\end{ttfamily}


\end{flushleft}
\end{list}
\paragraph*{PutBuffer}\hspace*{\fill}

\begin{list}{}{
\settowidth{\tmplength}{\textbf{Declaração}}
\setlength{\itemindent}{0cm}
\setlength{\listparindent}{0cm}
\setlength{\leftmargin}{\evensidemargin}
\addtolength{\leftmargin}{\tmplength}
\settowidth{\labelsep}{X}
\addtolength{\leftmargin}{\labelsep}
\setlength{\labelwidth}{\tmplength}
}
\begin{flushleft}
\item[\textbf{Declaração}\hfill]
\begin{ttfamily}
public Procedure PutBuffer;\end{ttfamily}


\end{flushleft}
\par
\item[\textbf{Descrição}]
O método \textbf{\begin{ttfamily}PutBuffer\end{ttfamily}} salva os dados do controle (Self) para a propriedade \begin{ttfamily}pDmxFieldRec\end{ttfamily}(\ref{mi_rtl_ui_Dmxscroller-pDmxFieldRec})

\end{list}
\paragraph*{GetBuffer}\hspace*{\fill}

\begin{list}{}{
\settowidth{\tmplength}{\textbf{Declaração}}
\setlength{\itemindent}{0cm}
\setlength{\listparindent}{0cm}
\setlength{\leftmargin}{\evensidemargin}
\addtolength{\leftmargin}{\tmplength}
\settowidth{\labelsep}{X}
\addtolength{\leftmargin}{\labelsep}
\setlength{\labelwidth}{\tmplength}
}
\begin{flushleft}
\item[\textbf{Declaração}\hfill]
\begin{ttfamily}
public Procedure GetBuffer;\end{ttfamily}


\end{flushleft}
\par
\item[\textbf{Descrição}]
O método \textbf{\begin{ttfamily}GetBuffer\end{ttfamily}} ler os dados da propriedade \begin{ttfamily}pDmxFieldRec\end{ttfamily}(\ref{mi_rtl_ui_Dmxscroller-pDmxFieldRec}) para o controle (Self).

\end{list}
\paragraph*{DoOnEnter}\hspace*{\fill}

\begin{list}{}{
\settowidth{\tmplength}{\textbf{Declaração}}
\setlength{\itemindent}{0cm}
\setlength{\listparindent}{0cm}
\setlength{\leftmargin}{\evensidemargin}
\addtolength{\leftmargin}{\tmplength}
\settowidth{\labelsep}{X}
\addtolength{\leftmargin}{\labelsep}
\setlength{\labelwidth}{\tmplength}
}
\begin{flushleft}
\item[\textbf{Declaração}\hfill]
\begin{ttfamily}
protected procedure DoOnEnter(Sender: TObject);\end{ttfamily}


\end{flushleft}
\par
\item[\textbf{Descrição}]
O método \textbf{\begin{ttfamily}DoOnEnter\end{ttfamily}} ao receber o foco executa os métodos \begin{ttfamily}GetBuffer\end{ttfamily}(\ref{uMI_ui_DbEdit_LCL.TMI_DbEdit_LCL-GetBuffer}) e pDmxFieldRec.DoOnEnter(Self).

\end{list}
\paragraph*{DoOnExit}\hspace*{\fill}

\begin{list}{}{
\settowidth{\tmplength}{\textbf{Declaração}}
\setlength{\itemindent}{0cm}
\setlength{\listparindent}{0cm}
\setlength{\leftmargin}{\evensidemargin}
\addtolength{\leftmargin}{\tmplength}
\settowidth{\labelsep}{X}
\addtolength{\leftmargin}{\labelsep}
\setlength{\labelwidth}{\tmplength}
}
\begin{flushleft}
\item[\textbf{Declaração}\hfill]
\begin{ttfamily}
protected procedure DoOnExit(Sender: TObject);\end{ttfamily}


\end{flushleft}
\par
\item[\textbf{Descrição}]
O método \textbf{\begin{ttfamily}DoOnExit\end{ttfamily}} ao perder o foco executa os métodos PuttBuffer e pDmxFieldRec.DoOnExit(Self).

\end{list}
\paragraph*{DoEditNumberKeyPress}\hspace*{\fill}

\begin{list}{}{
\settowidth{\tmplength}{\textbf{Declaração}}
\setlength{\itemindent}{0cm}
\setlength{\listparindent}{0cm}
\setlength{\leftmargin}{\evensidemargin}
\addtolength{\leftmargin}{\tmplength}
\settowidth{\labelsep}{X}
\addtolength{\leftmargin}{\labelsep}
\setlength{\labelwidth}{\tmplength}
}
\begin{flushleft}
\item[\textbf{Declaração}\hfill]
\begin{ttfamily}
protected procedure DoEditNumberKeyPress(Sender: TObject; var Key: char);\end{ttfamily}


\end{flushleft}
\par
\item[\textbf{Descrição}]
O método \textbf{\begin{ttfamily}DoEditNumberKeyPress\end{ttfamily}} edita os campos números de 1 a 10 bytes

\end{list}
\paragraph*{GetHelpCtx{\_}Hint}\hspace*{\fill}

\begin{list}{}{
\settowidth{\tmplength}{\textbf{Declaração}}
\setlength{\itemindent}{0cm}
\setlength{\listparindent}{0cm}
\setlength{\leftmargin}{\evensidemargin}
\addtolength{\leftmargin}{\tmplength}
\settowidth{\labelsep}{X}
\addtolength{\leftmargin}{\labelsep}
\setlength{\labelwidth}{\tmplength}
}
\begin{flushleft}
\item[\textbf{Declaração}\hfill]
\begin{ttfamily}
protected FUNCTION GetHelpCtx{\_}Hint(): AnsiString;\end{ttfamily}


\end{flushleft}
\par
\item[\textbf{Descrição}]
O método \textbf{\begin{ttfamily}GetHelpCtx{\_}Hint\end{ttfamily}} captura a Idocumentação do campo definido na classe onde o campo for criado.

\begin{itemize}
\item Com o programa \textbf{pasdoc} a Idocumentação não precisa está no arquivo de recursos, por isso, para obter o link para o campo � preciso saber apenas o endereço do link.
\end{itemize}

\end{list}
\paragraph*{GetMaxLength}\hspace*{\fill}

\begin{list}{}{
\settowidth{\tmplength}{\textbf{Declaração}}
\setlength{\itemindent}{0cm}
\setlength{\listparindent}{0cm}
\setlength{\leftmargin}{\evensidemargin}
\addtolength{\leftmargin}{\tmplength}
\settowidth{\labelsep}{X}
\addtolength{\leftmargin}{\labelsep}
\setlength{\labelwidth}{\tmplength}
}
\begin{flushleft}
\item[\textbf{Declaração}\hfill]
\begin{ttfamily}
protected FUNCTION GetMaxLength(): Variant;\end{ttfamily}


\end{flushleft}
\end{list}
\paragraph*{SetMaxLength}\hspace*{\fill}

\begin{list}{}{
\settowidth{\tmplength}{\textbf{Declaração}}
\setlength{\itemindent}{0cm}
\setlength{\listparindent}{0cm}
\setlength{\leftmargin}{\evensidemargin}
\addtolength{\leftmargin}{\tmplength}
\settowidth{\labelsep}{X}
\addtolength{\leftmargin}{\labelsep}
\setlength{\labelwidth}{\tmplength}
}
\begin{flushleft}
\item[\textbf{Declaração}\hfill]
\begin{ttfamily}
protected PROCEDURE SetMaxLength(aMaxLength: Variant);\end{ttfamily}


\end{flushleft}
\end{list}
\paragraph*{GetSize}\hspace*{\fill}

\begin{list}{}{
\settowidth{\tmplength}{\textbf{Declaração}}
\setlength{\itemindent}{0cm}
\setlength{\listparindent}{0cm}
\setlength{\leftmargin}{\evensidemargin}
\addtolength{\leftmargin}{\tmplength}
\settowidth{\labelsep}{X}
\addtolength{\leftmargin}{\labelsep}
\setlength{\labelwidth}{\tmplength}
}
\begin{flushleft}
\item[\textbf{Declaração}\hfill]
\begin{ttfamily}
protected FUNCTION GetSize(): Variant;\end{ttfamily}


\end{flushleft}
\end{list}
\paragraph*{SetSize}\hspace*{\fill}

\begin{list}{}{
\settowidth{\tmplength}{\textbf{Declaração}}
\setlength{\itemindent}{0cm}
\setlength{\listparindent}{0cm}
\setlength{\leftmargin}{\evensidemargin}
\addtolength{\leftmargin}{\tmplength}
\settowidth{\labelsep}{X}
\addtolength{\leftmargin}{\labelsep}
\setlength{\labelwidth}{\tmplength}
}
\begin{flushleft}
\item[\textbf{Declaração}\hfill]
\begin{ttfamily}
protected PROCEDURE SetSize(aSize: Variant);\end{ttfamily}


\end{flushleft}
\end{list}
\paragraph*{SetAlias}\hspace*{\fill}

\begin{list}{}{
\settowidth{\tmplength}{\textbf{Declaração}}
\setlength{\itemindent}{0cm}
\setlength{\listparindent}{0cm}
\setlength{\leftmargin}{\evensidemargin}
\addtolength{\leftmargin}{\tmplength}
\settowidth{\labelsep}{X}
\addtolength{\leftmargin}{\labelsep}
\setlength{\labelwidth}{\tmplength}
}
\begin{flushleft}
\item[\textbf{Declaração}\hfill]
\begin{ttfamily}
protected procedure SetAlias(const aAlias: AnsiString);\end{ttfamily}


\end{flushleft}
\end{list}
\paragraph*{GetName}\hspace*{\fill}

\begin{list}{}{
\settowidth{\tmplength}{\textbf{Declaração}}
\setlength{\itemindent}{0cm}
\setlength{\listparindent}{0cm}
\setlength{\leftmargin}{\evensidemargin}
\addtolength{\leftmargin}{\tmplength}
\settowidth{\labelsep}{X}
\addtolength{\leftmargin}{\labelsep}
\setlength{\labelwidth}{\tmplength}
}
\begin{flushleft}
\item[\textbf{Declaração}\hfill]
\begin{ttfamily}
protected FUNCTION GetName(): AnsiString;\end{ttfamily}


\end{flushleft}
\end{list}
\paragraph*{GetAlias}\hspace*{\fill}

\begin{list}{}{
\settowidth{\tmplength}{\textbf{Declaração}}
\setlength{\itemindent}{0cm}
\setlength{\listparindent}{0cm}
\setlength{\leftmargin}{\evensidemargin}
\addtolength{\leftmargin}{\tmplength}
\settowidth{\labelsep}{X}
\addtolength{\leftmargin}{\labelsep}
\setlength{\labelwidth}{\tmplength}
}
\begin{flushleft}
\item[\textbf{Declaração}\hfill]
\begin{ttfamily}
protected FUNCTION GetAlias: AnsiString;\end{ttfamily}


\end{flushleft}
\end{list}
\paragraph*{WMSetFocus}\hspace*{\fill}

\begin{list}{}{
\settowidth{\tmplength}{\textbf{Declaração}}
\setlength{\itemindent}{0cm}
\setlength{\listparindent}{0cm}
\setlength{\leftmargin}{\evensidemargin}
\addtolength{\leftmargin}{\tmplength}
\settowidth{\labelsep}{X}
\addtolength{\leftmargin}{\labelsep}
\setlength{\labelwidth}{\tmplength}
}
\begin{flushleft}
\item[\textbf{Declaração}\hfill]
\begin{ttfamily}
protected procedure WMSetFocus(var Message: TLMSetFocus); message LM{\_}SETFOCUS;\end{ttfamily}


\end{flushleft}
\end{list}
\paragraph*{GetDataLink}\hspace*{\fill}

\begin{list}{}{
\settowidth{\tmplength}{\textbf{Declaração}}
\setlength{\itemindent}{0cm}
\setlength{\listparindent}{0cm}
\setlength{\leftmargin}{\evensidemargin}
\addtolength{\leftmargin}{\tmplength}
\settowidth{\labelsep}{X}
\addtolength{\leftmargin}{\labelsep}
\setlength{\labelwidth}{\tmplength}
}
\begin{flushleft}
\item[\textbf{Declaração}\hfill]
\begin{ttfamily}
protected function GetDataLink:TFieldDataLink;\end{ttfamily}


\end{flushleft}
\end{list}
\paragraph*{WMPaint}\hspace*{\fill}

\begin{list}{}{
\settowidth{\tmplength}{\textbf{Declaração}}
\setlength{\itemindent}{0cm}
\setlength{\listparindent}{0cm}
\setlength{\leftmargin}{\evensidemargin}
\addtolength{\leftmargin}{\tmplength}
\settowidth{\labelsep}{X}
\addtolength{\leftmargin}{\labelsep}
\setlength{\labelwidth}{\tmplength}
}
\begin{flushleft}
\item[\textbf{Declaração}\hfill]
\begin{ttfamily}
protected procedure WMPaint(var Message: TLMPaint); message LM{\_}PAINT;\end{ttfamily}


\end{flushleft}
\end{list}
\paragraph*{ValidateEdit}\hspace*{\fill}

\begin{list}{}{
\settowidth{\tmplength}{\textbf{Declaração}}
\setlength{\itemindent}{0cm}
\setlength{\listparindent}{0cm}
\setlength{\leftmargin}{\evensidemargin}
\addtolength{\leftmargin}{\tmplength}
\settowidth{\labelsep}{X}
\addtolength{\leftmargin}{\labelsep}
\setlength{\labelwidth}{\tmplength}
}
\begin{flushleft}
\item[\textbf{Declaração}\hfill]
\begin{ttfamily}
public procedure ValidateEdit; override;\end{ttfamily}


\end{flushleft}
\end{list}
\section{Funções e Procedimentos}
\subsection*{Register}
\begin{list}{}{
\settowidth{\tmplength}{\textbf{Declaração}}
\setlength{\itemindent}{0cm}
\setlength{\listparindent}{0cm}
\setlength{\leftmargin}{\evensidemargin}
\addtolength{\leftmargin}{\tmplength}
\settowidth{\labelsep}{X}
\addtolength{\leftmargin}{\labelsep}
\setlength{\labelwidth}{\tmplength}
}
\begin{flushleft}
\item[\textbf{Declaração}\hfill]
\begin{ttfamily}
procedure Register;\end{ttfamily}


\end{flushleft}
\end{list}
\chapter{Unit umi{\_}ui{\_}dblookupComboBox{\_}lcl}
\section{Uses}
\begin{itemize}
\item \begin{ttfamily}Classes\end{ttfamily}\item \begin{ttfamily}SysUtils\end{ttfamily}\item \begin{ttfamily}LResources\end{ttfamily}\item \begin{ttfamily}Forms\end{ttfamily}\item \begin{ttfamily}Controls\end{ttfamily}\item \begin{ttfamily}Graphics\end{ttfamily}\item \begin{ttfamily}Dialogs\end{ttfamily}\item \begin{ttfamily}DBCtrls\end{ttfamily}\item \begin{ttfamily}StdCtrls\end{ttfamily}\item \begin{ttfamily}db\end{ttfamily}\item \begin{ttfamily}mi{\_}rtl{\_}ui{\_}DmxScroller\end{ttfamily}(\ref{mi_rtl_ui_Dmxscroller})\item \begin{ttfamily}mi{\_}rtl{\_}ui{\_}DmxScroller{\_}Form\end{ttfamily}(\ref{mi_rtl_ui_dmxscroller_form})\item \begin{ttfamily}umi{\_}ui{\_}dmxscroller{\_}form{\_}lcl{\_}attributes\end{ttfamily}(\ref{umi_ui_dmxscroller_form_lcl_attributes})\end{itemize}
\section{Visão Geral}
\begin{description}
\item[\texttt{\begin{ttfamily}TMi{\_}ui{\_}DBLookupComboBox{\_}Lcl\end{ttfamily} Classe}]
\end{description}
\begin{description}
\item[\texttt{Register}]
\end{description}
\section{Classes, Interfaces, Objetos e Registros}
\subsection*{TMi{\_}ui{\_}DBLookupComboBox{\_}Lcl Classe}
\subsubsection*{\large{\textbf{Hierarquia}}\normalsize\hspace{1ex}\hfill}
TMi{\_}ui{\_}DBLookupComboBox{\_}Lcl {$>$} TDBLookupComboBox
%%%%Descrição
\subsubsection*{\large{\textbf{Propriedades}}\normalsize\hspace{1ex}\hfill}
\paragraph*{DmxScroller{\_}Form{\_}Lcl{\_}attributes}\hspace*{\fill}

\begin{list}{}{
\settowidth{\tmplength}{\textbf{Declaração}}
\setlength{\itemindent}{0cm}
\setlength{\listparindent}{0cm}
\setlength{\leftmargin}{\evensidemargin}
\addtolength{\leftmargin}{\tmplength}
\settowidth{\labelsep}{X}
\addtolength{\leftmargin}{\labelsep}
\setlength{\labelwidth}{\tmplength}
}
\begin{flushleft}
\item[\textbf{Declaração}\hfill]
\begin{ttfamily}
published property DmxScroller{\_}Form{\_}Lcl{\_}attributes : TDmxScroller{\_}Form{\_}Lcl{\_}attributes Read {\_}DmxScroller{\_}Form{\_}Lcl{\_}attributes write SetDmxScroller{\_}Form{\_}Lcl{\_}attributes;\end{ttfamily}


\end{flushleft}
\end{list}
\paragraph*{DmxFieldRec}\hspace*{\fill}

\begin{list}{}{
\settowidth{\tmplength}{\textbf{Declaração}}
\setlength{\itemindent}{0cm}
\setlength{\listparindent}{0cm}
\setlength{\leftmargin}{\evensidemargin}
\addtolength{\leftmargin}{\tmplength}
\settowidth{\labelsep}{X}
\addtolength{\leftmargin}{\labelsep}
\setlength{\labelwidth}{\tmplength}
}
\begin{flushleft}
\item[\textbf{Declaração}\hfill]
\begin{ttfamily}
public property DmxFieldRec: pDmxFieldRec Read {\_}pDmxFieldRec   Write  SeTDmxFieldRec;\end{ttfamily}


\end{flushleft}
\par
\item[\textbf{Descrição}]
O atributo \textbf{\begin{ttfamily}DmxFieldRec\end{ttfamily}} fornece os dados necessários para criar o componente \begin{ttfamily}TMI{\_}MaskEdit{\_}LCL\end{ttfamily}(\ref{uMi_ui_maskedit_lcl.TMI_MaskEdit_LCL}).

\begin{itemize}
\item \textbf{NOTA} \begin{itemize}
\item Esses dados devem ser criados pelo método TDmxScroller{\_}Form{\_}Lcl{\_}attributes.CreateStruct(var ATemplate : \begin{ttfamily}TString\end{ttfamily}(\ref{mi_rtl_ui_Dmxscroller-tString}))
\end{itemize}
\end{itemize}

\end{list}
\subsubsection*{\large{\textbf{Métodos}}\normalsize\hspace{1ex}\hfill}
\paragraph*{Create}\hspace*{\fill}

\begin{list}{}{
\settowidth{\tmplength}{\textbf{Declaração}}
\setlength{\itemindent}{0cm}
\setlength{\listparindent}{0cm}
\setlength{\leftmargin}{\evensidemargin}
\addtolength{\leftmargin}{\tmplength}
\settowidth{\labelsep}{X}
\addtolength{\leftmargin}{\labelsep}
\setlength{\labelwidth}{\tmplength}
}
\begin{flushleft}
\item[\textbf{Declaração}\hfill]
\begin{ttfamily}
public constructor Create(AOwner:TComponent); override; overload;\end{ttfamily}


\end{flushleft}
\end{list}
\paragraph*{Create}\hspace*{\fill}

\begin{list}{}{
\settowidth{\tmplength}{\textbf{Declaração}}
\setlength{\itemindent}{0cm}
\setlength{\listparindent}{0cm}
\setlength{\leftmargin}{\evensidemargin}
\addtolength{\leftmargin}{\tmplength}
\settowidth{\labelsep}{X}
\addtolength{\leftmargin}{\labelsep}
\setlength{\labelwidth}{\tmplength}
}
\begin{flushleft}
\item[\textbf{Declaração}\hfill]
\begin{ttfamily}
public constructor Create(aOwner:TComponent;aDmxScroller{\_}Form{\_}Lcl{\_}attributes : TDmxScroller{\_}Form{\_}Lcl{\_}attributes); overload; overload;\end{ttfamily}


\end{flushleft}
\end{list}
\paragraph*{Destroy}\hspace*{\fill}

\begin{list}{}{
\settowidth{\tmplength}{\textbf{Declaração}}
\setlength{\itemindent}{0cm}
\setlength{\listparindent}{0cm}
\setlength{\leftmargin}{\evensidemargin}
\addtolength{\leftmargin}{\tmplength}
\settowidth{\labelsep}{X}
\addtolength{\leftmargin}{\labelsep}
\setlength{\labelwidth}{\tmplength}
}
\begin{flushleft}
\item[\textbf{Declaração}\hfill]
\begin{ttfamily}
public destructor Destroy; override;\end{ttfamily}


\end{flushleft}
\end{list}
\paragraph*{PutBuffer}\hspace*{\fill}

\begin{list}{}{
\settowidth{\tmplength}{\textbf{Declaração}}
\setlength{\itemindent}{0cm}
\setlength{\listparindent}{0cm}
\setlength{\leftmargin}{\evensidemargin}
\addtolength{\leftmargin}{\tmplength}
\settowidth{\labelsep}{X}
\addtolength{\leftmargin}{\labelsep}
\setlength{\labelwidth}{\tmplength}
}
\begin{flushleft}
\item[\textbf{Declaração}\hfill]
\begin{ttfamily}
public Procedure PutBuffer;\end{ttfamily}


\end{flushleft}
\end{list}
\paragraph*{GetBuffer}\hspace*{\fill}

\begin{list}{}{
\settowidth{\tmplength}{\textbf{Declaração}}
\setlength{\itemindent}{0cm}
\setlength{\listparindent}{0cm}
\setlength{\leftmargin}{\evensidemargin}
\addtolength{\leftmargin}{\tmplength}
\settowidth{\labelsep}{X}
\addtolength{\leftmargin}{\labelsep}
\setlength{\labelwidth}{\tmplength}
}
\begin{flushleft}
\item[\textbf{Declaração}\hfill]
\begin{ttfamily}
public Procedure GetBuffer;\end{ttfamily}


\end{flushleft}
\end{list}
\paragraph*{DoOnEnter}\hspace*{\fill}

\begin{list}{}{
\settowidth{\tmplength}{\textbf{Declaração}}
\setlength{\itemindent}{0cm}
\setlength{\listparindent}{0cm}
\setlength{\leftmargin}{\evensidemargin}
\addtolength{\leftmargin}{\tmplength}
\settowidth{\labelsep}{X}
\addtolength{\leftmargin}{\labelsep}
\setlength{\labelwidth}{\tmplength}
}
\begin{flushleft}
\item[\textbf{Declaração}\hfill]
\begin{ttfamily}
protected procedure DoOnEnter(Sender: TObject);\end{ttfamily}


\end{flushleft}
\end{list}
\paragraph*{DoOnExit}\hspace*{\fill}

\begin{list}{}{
\settowidth{\tmplength}{\textbf{Declaração}}
\setlength{\itemindent}{0cm}
\setlength{\listparindent}{0cm}
\setlength{\leftmargin}{\evensidemargin}
\addtolength{\leftmargin}{\tmplength}
\settowidth{\labelsep}{X}
\addtolength{\leftmargin}{\labelsep}
\setlength{\labelwidth}{\tmplength}
}
\begin{flushleft}
\item[\textbf{Declaração}\hfill]
\begin{ttfamily}
protected procedure DoOnExit(Sender: TObject);\end{ttfamily}


\end{flushleft}
\end{list}
\section{Funções e Procedimentos}
\subsection*{Register}
\begin{list}{}{
\settowidth{\tmplength}{\textbf{Declaração}}
\setlength{\itemindent}{0cm}
\setlength{\listparindent}{0cm}
\setlength{\leftmargin}{\evensidemargin}
\addtolength{\leftmargin}{\tmplength}
\settowidth{\labelsep}{X}
\addtolength{\leftmargin}{\labelsep}
\setlength{\labelwidth}{\tmplength}
}
\begin{flushleft}
\item[\textbf{Declaração}\hfill]
\begin{ttfamily}
procedure Register;\end{ttfamily}


\end{flushleft}
\end{list}
\chapter{Unit uMI{\_}ui{\_}DbRadioGroup{\_}Lcl}
\section{Uses}
\begin{itemize}
\item \begin{ttfamily}Classes\end{ttfamily}\item \begin{ttfamily}SysUtils\end{ttfamily}\item \begin{ttfamily}LResources\end{ttfamily}\item \begin{ttfamily}Forms\end{ttfamily}\item \begin{ttfamily}Controls\end{ttfamily}\item \begin{ttfamily}Graphics\end{ttfamily}\item \begin{ttfamily}Dialogs\end{ttfamily}\item \begin{ttfamily}DBCtrls\end{ttfamily}\item \begin{ttfamily}ActnList\end{ttfamily}\item \begin{ttfamily}mi{\_}rtl{\_}ui{\_}DmxScroller\end{ttfamily}(\ref{mi_rtl_ui_Dmxscroller})\item \begin{ttfamily}mi{\_}rtl{\_}ui{\_}DmxScroller{\_}Form\end{ttfamily}(\ref{mi_rtl_ui_dmxscroller_form})\item \begin{ttfamily}umi{\_}ui{\_}dmxscroller{\_}form{\_}lcl{\_}attributes\end{ttfamily}(\ref{umi_ui_dmxscroller_form_lcl_attributes})\end{itemize}
\section{Visão Geral}
\begin{description}
\item[\texttt{\begin{ttfamily}TMI{\_}ui{\_}DbRadioGroup{\_}Lcl\end{ttfamily} Classe}]
\end{description}
\begin{description}
\item[\texttt{Register}]
\end{description}
\section{Classes, Interfaces, Objetos e Registros}
\subsection*{TMI{\_}ui{\_}DbRadioGroup{\_}Lcl Classe}
\subsubsection*{\large{\textbf{Hierarquia}}\normalsize\hspace{1ex}\hfill}
TMI{\_}ui{\_}DbRadioGroup{\_}Lcl {$>$} TDBRadioGroup
%%%%Descrição
\subsubsection*{\large{\textbf{Propriedades}}\normalsize\hspace{1ex}\hfill}
\paragraph*{DmxScroller{\_}Form{\_}Lcl{\_}attributes}\hspace*{\fill}

\begin{list}{}{
\settowidth{\tmplength}{\textbf{Declaração}}
\setlength{\itemindent}{0cm}
\setlength{\listparindent}{0cm}
\setlength{\leftmargin}{\evensidemargin}
\addtolength{\leftmargin}{\tmplength}
\settowidth{\labelsep}{X}
\addtolength{\leftmargin}{\labelsep}
\setlength{\labelwidth}{\tmplength}
}
\begin{flushleft}
\item[\textbf{Declaração}\hfill]
\begin{ttfamily}
published property DmxScroller{\_}Form{\_}Lcl{\_}attributes : TDmxScroller{\_}Form{\_}Lcl{\_}attributes Read {\_}DmxScroller{\_}Form{\_}Lcl{\_}attributes  write SetDmxScroller{\_}Form{\_}Lcl{\_}attributes;\end{ttfamily}


\end{flushleft}
\par
\item[\textbf{Descrição}]
A propriedade \textbf{\begin{ttfamily}DmxScroller{\_}Form{\_}Lcl{\_}attributes\end{ttfamily}} contém o modelo e os cálculos do formulário

\end{list}
\paragraph*{DmxFieldRec}\hspace*{\fill}

\begin{list}{}{
\settowidth{\tmplength}{\textbf{Declaração}}
\setlength{\itemindent}{0cm}
\setlength{\listparindent}{0cm}
\setlength{\leftmargin}{\evensidemargin}
\addtolength{\leftmargin}{\tmplength}
\settowidth{\labelsep}{X}
\addtolength{\leftmargin}{\labelsep}
\setlength{\labelwidth}{\tmplength}
}
\begin{flushleft}
\item[\textbf{Declaração}\hfill]
\begin{ttfamily}
public property DmxFieldRec: pDmxFieldRec Read {\_}pDmxFieldRec   Write  SeTDmxFieldRec;\end{ttfamily}


\end{flushleft}
\par
\item[\textbf{Descrição}]
A propriedade \textbf{\begin{ttfamily}DmxFieldRec\end{ttfamily}} fornece os dados necessários para criar o componente \begin{ttfamily}TMI{\_}Button{\_}LCL\end{ttfamily}(\ref{umi_ui_button_lcl.TMI_Button_LCL}).

\begin{itemize}
\item \textbf{NOTA} \begin{itemize}
\item Esses dados devem ser criados pelo método DmxScroller{\_}Form{\_}Lcl{\_}attributesr.CreateStruct(var ATemplate : \begin{ttfamily}TString\end{ttfamily}(\ref{mi_rtl_ui_Dmxscroller-tString}))
\end{itemize}
\end{itemize}

\end{list}
\subsubsection*{\large{\textbf{Métodos}}\normalsize\hspace{1ex}\hfill}
\paragraph*{GetBuffer}\hspace*{\fill}

\begin{list}{}{
\settowidth{\tmplength}{\textbf{Declaração}}
\setlength{\itemindent}{0cm}
\setlength{\listparindent}{0cm}
\setlength{\leftmargin}{\evensidemargin}
\addtolength{\leftmargin}{\tmplength}
\settowidth{\labelsep}{X}
\addtolength{\leftmargin}{\labelsep}
\setlength{\labelwidth}{\tmplength}
}
\begin{flushleft}
\item[\textbf{Declaração}\hfill]
\begin{ttfamily}
public Procedure GetBuffer;\end{ttfamily}


\end{flushleft}
\par
\item[\textbf{Descrição}]
O método \textbf{\begin{ttfamily}GetBuffer\end{ttfamily}} ler os dados da propriedade \begin{ttfamily}pDmxFieldRec\end{ttfamily}(\ref{mi_rtl_ui_Dmxscroller-pDmxFieldRec}) para o controle (Self).

\end{list}
\paragraph*{DoOnEnter}\hspace*{\fill}

\begin{list}{}{
\settowidth{\tmplength}{\textbf{Declaração}}
\setlength{\itemindent}{0cm}
\setlength{\listparindent}{0cm}
\setlength{\leftmargin}{\evensidemargin}
\addtolength{\leftmargin}{\tmplength}
\settowidth{\labelsep}{X}
\addtolength{\leftmargin}{\labelsep}
\setlength{\labelwidth}{\tmplength}
}
\begin{flushleft}
\item[\textbf{Declaração}\hfill]
\begin{ttfamily}
protected procedure DoOnEnter(Sender: TObject);\end{ttfamily}


\end{flushleft}
\end{list}
\paragraph*{PutBuffer}\hspace*{\fill}

\begin{list}{}{
\settowidth{\tmplength}{\textbf{Declaração}}
\setlength{\itemindent}{0cm}
\setlength{\listparindent}{0cm}
\setlength{\leftmargin}{\evensidemargin}
\addtolength{\leftmargin}{\tmplength}
\settowidth{\labelsep}{X}
\addtolength{\leftmargin}{\labelsep}
\setlength{\labelwidth}{\tmplength}
}
\begin{flushleft}
\item[\textbf{Declaração}\hfill]
\begin{ttfamily}
public Procedure PutBuffer;\end{ttfamily}


\end{flushleft}
\par
\item[\textbf{Descrição}]
O método \textbf{\begin{ttfamily}PutBuffer\end{ttfamily}} salva os dados do controle (Self) para a propriedade \begin{ttfamily}pDmxFieldRec\end{ttfamily}(\ref{mi_rtl_ui_Dmxscroller-pDmxFieldRec})

\end{list}
\paragraph*{DoOnExit}\hspace*{\fill}

\begin{list}{}{
\settowidth{\tmplength}{\textbf{Declaração}}
\setlength{\itemindent}{0cm}
\setlength{\listparindent}{0cm}
\setlength{\leftmargin}{\evensidemargin}
\addtolength{\leftmargin}{\tmplength}
\settowidth{\labelsep}{X}
\addtolength{\leftmargin}{\labelsep}
\setlength{\labelwidth}{\tmplength}
}
\begin{flushleft}
\item[\textbf{Declaração}\hfill]
\begin{ttfamily}
protected procedure DoOnExit(Sender: TObject);\end{ttfamily}


\end{flushleft}
\par
\item[\textbf{Descrição}]
O método \textbf{\begin{ttfamily}DoOnExit\end{ttfamily}} ao perder o foco executa os métodos PuttBuffer e \begin{ttfamily}pDmxFieldRec\end{ttfamily}(\ref{mi_rtl_ui_Dmxscroller-pDmxFieldRec}){\^{}}.DoOnExit(Self).

\end{list}
\section{Funções e Procedimentos}
\subsection*{Register}
\begin{list}{}{
\settowidth{\tmplength}{\textbf{Declaração}}
\setlength{\itemindent}{0cm}
\setlength{\listparindent}{0cm}
\setlength{\leftmargin}{\evensidemargin}
\addtolength{\leftmargin}{\tmplength}
\settowidth{\labelsep}{X}
\addtolength{\leftmargin}{\labelsep}
\setlength{\labelwidth}{\tmplength}
}
\begin{flushleft}
\item[\textbf{Declaração}\hfill]
\begin{ttfamily}
procedure Register;\end{ttfamily}


\end{flushleft}
\end{list}
\chapter{Unit umi{\_}ui{\_}dmxscroller{\_}form{\_}lcl{\_}attributes}
\section{Descrição}
A unit \textbf{\begin{ttfamily}umi{\_}ui{\_}dmxscroller{\_}form{\_}lcl{\_}attributes\end{ttfamily}} implementa a classe \begin{ttfamily}TDmxScroller{\_}Form\end{ttfamily}(\ref{mi_rtl_ui_dmxscroller_form.TDmxScroller_Form}).

\begin{itemize}
\item Primeiro autor: Paulo Sérgio da Silva Pacheco paulosspacheco@yahoo.com.br)
\item \textbf{VERSÃO} \begin{itemize}
\item Alpha {-} 0.5.0.687
\end{itemize}
\item \textbf{CÓDIGO FONTE}: \begin{itemize}
\item 
\end{itemize}
\item \textbf{HISTÓRICO}: \begin{itemize}
\item T12 Criado unit com objetivo de separar o pacote \textbf{\begin{ttfamily}mi.rtl\end{ttfamily}(\ref{mi.rtl})} do pacote \textbf{LCL} com objetivo de criar aplicação web independente de gráficos locais.
\end{itemize}
\item \textbf{PENDÊNCIAS} \begin{itemize}
\item T12 Documentar a unit.
\item T12 Criado unit com objetivo de separar o pacote \textbf{\begin{ttfamily}mi.rtl\end{ttfamily}(\ref{mi.rtl})} do pacote \textbf{LCL} com objetivo de criar aplicação web independente de gráficos locais.
\item T12 Criar classe \textbf{\begin{ttfamily}TDmxScroller{\_}Form{\_}Lcl{\_}attributes\end{ttfamily}(\ref{umi_ui_dmxscroller_form_lcl_attributes.TDmxScroller_Form_Lcl_attributes})} e mover todos os atributos da classe TDmxScroller{\_}Form{\_}Lcl para ela.
\end{itemize}
\end{itemize}
\section{Uses}
\begin{itemize}
\item \begin{ttfamily}Classes\end{ttfamily}\item \begin{ttfamily}SysUtils\end{ttfamily}\item \begin{ttfamily}controls\end{ttfamily}\item \begin{ttfamily}StdCtrls\end{ttfamily}\item \begin{ttfamily}forms\end{ttfamily}\item \begin{ttfamily}typInfo\end{ttfamily}\item \begin{ttfamily}types\end{ttfamily}\item \begin{ttfamily}Graphics\end{ttfamily}\item \begin{ttfamily}ActnList\end{ttfamily}\item \begin{ttfamily}Dialogs\end{ttfamily}\item \begin{ttfamily}mi.rtl.Consts\end{ttfamily}(\ref{mi.rtl.Consts})\item \begin{ttfamily}mi{\_}rtl{\_}ui{\_}DmxScroller{\_}Form\end{ttfamily}(\ref{mi_rtl_ui_dmxscroller_form})\item \begin{ttfamily}uMi{\_}ui{\_}scrollbox{\_}lcl\end{ttfamily}(\ref{uMi_ui_scrollbox_lcl})\end{itemize}
\section{Visão Geral}
\begin{description}
\item[\texttt{\begin{ttfamily}TDmxScroller{\_}Form{\_}Lcl{\_}attributes\end{ttfamily} Classe}]
\end{description}
\section{Classes, Interfaces, Objetos e Registros}
\subsection*{TDmxScroller{\_}Form{\_}Lcl{\_}attributes Classe}
\subsubsection*{\large{\textbf{Hierarquia}}\normalsize\hspace{1ex}\hfill}
TDmxScroller{\_}Form{\_}Lcl{\_}attributes {$>$} \begin{ttfamily}TDmxScroller{\_}Form\end{ttfamily}(\ref{mi_rtl_ui_dmxscroller_form.TDmxScroller_Form}) {$>$} \begin{ttfamily}TDmxScroller{\_}Form{\_}Atributos\end{ttfamily}(\ref{mi_rtl_ui_dmxscroller_form.TDmxScroller_Form_Atributos}) {$>$} \begin{ttfamily}TUiDmxScroller\end{ttfamily}(\ref{mi_rtl_ui_Dmxscroller.TUiDmxScroller}) {$>$} \begin{ttfamily}TUiMethods\end{ttfamily}(\ref{mi_rtl_ui_methods.TUiMethods}) {$>$} 
TUiConsts
\subsubsection*{\large{\textbf{Descrição}}\normalsize\hspace{1ex}\hfill}
no description available, TDmxScroller{\_}Form description followsA classe \textbf{\begin{ttfamily}TDmxScroller{\_}Form\end{ttfamily}} implementa a construção de formulários usando uma lista de Templates do tipo TDmxScroller\subsubsection*{\large{\textbf{Propriedades}}\normalsize\hspace{1ex}\hfill}
\paragraph*{ActionList}\hspace*{\fill}

\begin{list}{}{
\settowidth{\tmplength}{\textbf{Declaração}}
\setlength{\itemindent}{0cm}
\setlength{\listparindent}{0cm}
\setlength{\leftmargin}{\evensidemargin}
\addtolength{\leftmargin}{\tmplength}
\settowidth{\labelsep}{X}
\addtolength{\leftmargin}{\labelsep}
\setlength{\labelwidth}{\tmplength}
}
\begin{flushleft}
\item[\textbf{Declaração}\hfill]
\begin{ttfamily}
public property ActionList : TActionList Read {\_}ActionList   Write  {\_}ActionList;\end{ttfamily}


\end{flushleft}
\par
\item[\textbf{Descrição}]
A propriedade \textbf{\begin{ttfamily}ActionList\end{ttfamily}} permite que ações do formulário LCL possam ser executados a partir do componente \textbf{TDmxScroller}.

\begin{itemize}
\item \textbf{NOTA} \begin{itemize}
\item O interpretador de Template inicia o campo \textbf{\begin{ttfamily}TDmxFieldRec.ExecAction\end{ttfamily}(\ref{mi_rtl_ui_Dmxscroller.TDmxFieldRec-ExecAction})} e o campo LinkExecAction.
\item O gerador de formulário ao encontrar \textbf{\begin{ttfamily}TDmxFieldRec.ExecAction\end{ttfamily}(\ref{mi_rtl_ui_Dmxscroller.TDmxFieldRec-ExecAction})} cria um botão para que se possa executar a ação.
\end{itemize}
\item \textbf{EXEMPLO}

\texttt{\\\nopagebreak[3]
\\\nopagebreak[3]
}\textbf{with}\texttt{~DmxScroller{\_}Form1~}\textbf{do}\texttt{\\\nopagebreak[3]
}\textbf{begin}\texttt{\\\nopagebreak[3]
~~Result~:=~NewSItem('~~~~~~~~⏭️{\&}Primeiro~'+CharExecAction+Novo.}\textbf{Name}\texttt{+\\\nopagebreak[3]
~~~~~~~~~~~~~~~~~~~~~'~⏩~P{\&}róximo~'+CharExecAction+(Gravar.}\textbf{Name}\texttt{)+\\\nopagebreak[3]
~~~~~~~~~~~~~~~~~~~~~'~⏪️{\&}Anterior~'+CharExecAction+Pesquisar.}\textbf{Name}\texttt{+\\\nopagebreak[3]
~~~~~~~~~~~~~~~~~~~~~'~⏮️~{\&}Ultimo~'+CharExecAction+(Excluir.}\textbf{Name}\texttt{),}\textbf{nil}\texttt{);\\\nopagebreak[3]
}\textbf{end}\texttt{;\\
}
\end{itemize}

\end{list}
\paragraph*{ParentLCL}\hspace*{\fill}

\begin{list}{}{
\settowidth{\tmplength}{\textbf{Declaração}}
\setlength{\itemindent}{0cm}
\setlength{\listparindent}{0cm}
\setlength{\leftmargin}{\evensidemargin}
\addtolength{\leftmargin}{\tmplength}
\settowidth{\labelsep}{X}
\addtolength{\leftmargin}{\labelsep}
\setlength{\labelwidth}{\tmplength}
}
\begin{flushleft}
\item[\textbf{Declaração}\hfill]
\begin{ttfamily}
published property ParentLCL : TScrollingWinControl Read {\_}ParentLCL write SetParentLCL;\end{ttfamily}


\end{flushleft}
\par
\item[\textbf{Descrição}]
A propriedade \textbf{\begin{ttfamily}ParentLCL\end{ttfamily}} informa a janela que será desenhada o formulário

\end{list}
\subsubsection*{\large{\textbf{Campos}}\normalsize\hspace{1ex}\hfill}
\paragraph*{{\_}ActionList}\hspace*{\fill}

\begin{list}{}{
\settowidth{\tmplength}{\textbf{Declaração}}
\setlength{\itemindent}{0cm}
\setlength{\listparindent}{0cm}
\setlength{\leftmargin}{\evensidemargin}
\addtolength{\leftmargin}{\tmplength}
\settowidth{\labelsep}{X}
\addtolength{\leftmargin}{\labelsep}
\setlength{\labelwidth}{\tmplength}
}
\begin{flushleft}
\item[\textbf{Declaração}\hfill]
\begin{ttfamily}
protected {\_}ActionList: TActionList;\end{ttfamily}


\end{flushleft}
\end{list}
\chapter{Unit umi{\_}ui{\_}dmxscroller{\_}form{\_}lcl{\_}base}
\section{Uses}
\begin{itemize}
\item \begin{ttfamily}Classes\end{ttfamily}\item \begin{ttfamily}SysUtils\end{ttfamily}\end{itemize}
\chapter{Unit umi{\_}ui{\_}dmxscroller{\_}form{\_}lcl{\_}ds}
\section{Descrição}
A unit \textbf{\begin{ttfamily}umi{\_}ui{\_}dmxscroller{\_}form{\_}lcl{\_}ds\end{ttfamily}} implementa a classe TDmxScroller{\_}Form{\_}Lcl{\_}attributes{\_}Form{\_}ds.

\begin{itemize}
\item \textbf{VERSÃO} \begin{itemize}
\item Alpha {-} 0.5.0.687
\end{itemize}
\item \textbf{CÓDIGO FONTE}: \begin{itemize}
\item 
\item \textbf{PENDÊNCIAS} \begin{itemize}
\item T12 O Componente TMi{\_}ui{\_}DmxScroller{\_}Form{\_}ds quando inserido em um dataModule os campos checkBox e RadioButton não estão visíveis.
\item T12 Corrigir problema quando vinculados ao dbGrid os controles do tipo: \begin{itemize}
\item \begin{ttfamily}fldEnum\end{ttfamily}(\ref{mi_rtl_ui_dmxscroller_form-fldENUM});
\item \begin{ttfamily}FldBoolean\end{ttfamily}(\ref{mi_rtl_ui_dmxscroller_form-fldBoolean});
\item \begin{ttfamily}FldRadioButton\end{ttfamily}(\ref{mi_rtl_ui_dmxscroller_form-FldRadioButton});
\end{itemize}
\item T12 Criar Método procedure TDmxScroller{\_}Form.CreateFormLCL(aOwner: TScrollingWinControl); ✅ \begin{itemize}
\item Esse método deve usar as classes da Pallet Data Controls \begin{itemize}
\item TMI{\_}ui{\_}DbText{\_}LCL
\item TMI{\_}ui{\_}DbEdit{\_}LCL ✅
\item \begin{ttfamily}TMI{\_}ui{\_}DbLookupComboBox{\_}LCL\end{ttfamily}(\ref{umi_ui_dblookupComboBox_lcl.TMi_ui_DBLookupComboBox_Lcl}) ✅
\item TMI{\_}ui{\_}DbCheck{\_}LCL ✅
\item TMI{\_}ui{\_}DbRadioButton{\_}Lcl ✅
\item TMi{\_}ui{\_}DbData{\_}lcl
\item TMi{\_}ui{\_}DbHora{\_}lcl
\item TMi{\_}ui{\_}DbDataHora{\_}lcl
\end{itemize}
\end{itemize}
\item T12 HABILITAR OS EVENTOS DE TDataSource.dataset \begin{itemize}
\item T12 Executar os eventos do dataSet associado aos controles dbText e Db dbComboBox
\item T12 Dar opção global para habilitar e desabilitar as mascaras dos textos. \begin{itemize}
\item Criar opção global OkEditMask para usar com a propriedade: TMI{\_}DbEdit{\_}LCL.CustomEditMask;
\end{itemize}
\end{itemize}
\end{itemize}
\end{itemize} = \textbf{CONCLUÍDO} \begin{itemize}
\item T12 A classe deve herdar de \textbf{TDmxScroller{\_}Form{\_}DS} ✅
\end{itemize}\begin{itemize}
\item \textbf{HISTÓRICO} \begin{itemize}
\item Criado por: Paulo Sérgio da Silva Pacheco paulosspacheco@yahoo.com.br) ✅
\item \textbf{2022{-}04{-}27} \begin{itemize}
\item \textbf{14:10} \begin{itemize}
\item Análise de como implementar a unit;
\item Criar a classe TMI{\_}ui{\_}DbEdit{\_}LCL ✅
\end{itemize}
\end{itemize}
\item \textbf{2022{-}06{-}07} \begin{itemize}
\item Criar a classe TMI{\_}ui{\_}DbCheck{\_}LCL ✅ \begin{itemize}
\item Em CreateFormLCL inserir classe TMI{\_}ui{\_}DbCheck{\_}LCL se CurrentField for do tipo \begin{ttfamily}FldBoolean\end{ttfamily}(\ref{mi_rtl_ui_dmxscroller_form-fldBoolean}); ✅
\end{itemize}
\item Criar a classe \begin{ttfamily}TMi{\_}Ui{\_}DBRadioGroup{\_}Lcl\end{ttfamily}(\ref{uMI_ui_DbRadioGroup_Lcl.TMI_ui_DbRadioGroup_Lcl}) ✅ \begin{itemize}
\item Em CreateFormLCL inserir classe \begin{ttfamily}TMI{\_}ui{\_}DbRadioGroup{\_}LCL\end{ttfamily}(\ref{uMI_ui_DbRadioGroup_Lcl.TMI_ui_DbRadioGroup_Lcl}) se CurrentField for do tipo \begin{ttfamily}FldRadioButton\end{ttfamily}(\ref{mi_rtl_ui_dmxscroller_form-FldRadioButton}); \begin{itemize}
\item Criar função CreateRadioGroup;
\end{itemize}
\end{itemize}
\item Em CreateFormLCL inserir classe \begin{ttfamily}TMi{\_}Button{\_}LCL\end{ttfamily}(\ref{umi_ui_button_lcl.TMI_Button_LCL}) se DmxFieldRec.ExecAction{$<$}{$>$}''
\end{itemize}
\item \textbf{2022{-}07{-}04} \begin{itemize}
\item \textbf{09:38} \begin{itemize}
\item T12 O Componente TMi{\_}ui{\_}DmxScroller{\_}Form{\_}ds quando inserido em um dataModule os campos checkBox e RadioButton não estão visíveis.
\end{itemize}
\end{itemize}
\end{itemize}
\end{itemize}
\end{itemize}
\section{Uses}
\begin{itemize}
\item \begin{ttfamily}Classes\end{ttfamily}\item \begin{ttfamily}SysUtils\end{ttfamily}\item \begin{ttfamily}controls\end{ttfamily}\item \begin{ttfamily}StdCtrls\end{ttfamily}\item \begin{ttfamily}forms\end{ttfamily}\item \begin{ttfamily}db\end{ttfamily}\item \begin{ttfamily}mi{\_}rtl{\_}ui{\_}Dmxscroller\end{ttfamily}(\ref{mi_rtl_ui_Dmxscroller})\item \begin{ttfamily}uMi{\_}ui{\_}Dmxscroller{\_}form{\_}Lcl\end{ttfamily}\item \begin{ttfamily}uMI{\_}ui{\_}DbEdit{\_}LCL\end{ttfamily}(\ref{uMI_ui_DbEdit_LCL})\item \begin{ttfamily}uMi{\_}Ui{\_}DbLookupComboBox{\_}lcl\end{ttfamily}(\ref{umi_ui_dblookupComboBox_lcl})\item \begin{ttfamily}uMi{\_}Ui{\_}DbComboBox{\_}lcl\end{ttfamily}(\ref{uMi_Ui_DbComboBox_lcl})\item \begin{ttfamily}uMi{\_}Ui{\_}DBCheckBox{\_}Lcl\end{ttfamily}(\ref{uMi_Ui_DBCheckBox_Lcl})\item \begin{ttfamily}uMi{\_}Ui{\_}DbRadioGroup{\_}Lcl\end{ttfamily}(\ref{uMI_ui_DbRadioGroup_Lcl})\item \begin{ttfamily}umi{\_}ui{\_}button{\_}lcl\end{ttfamily}(\ref{umi_ui_button_lcl})\item \begin{ttfamily}uMi{\_}ui{\_}Label{\_}lcl\end{ttfamily}(\ref{uMi_ui_Label_lcl})\end{itemize}
\section{Visão Geral}
\begin{description}
\item[\texttt{\begin{ttfamily}TDmxScroller{\_}Form{\_}Lcl{\_}DS\end{ttfamily} Classe}]
\end{description}
\begin{description}
\item[\texttt{Register}]
\end{description}
\section{Classes, Interfaces, Objetos e Registros}
\subsection*{TDmxScroller{\_}Form{\_}Lcl{\_}DS Classe}
\subsubsection*{\large{\textbf{Hierarquia}}\normalsize\hspace{1ex}\hfill}
TDmxScroller{\_}Form{\_}Lcl{\_}DS {$>$} TDmxScroller_Form_Lcl
%%%%Descrição
\subsubsection*{\large{\textbf{Propriedades}}\normalsize\hspace{1ex}\hfill}
\paragraph*{TableName}\hspace*{\fill}

\begin{list}{}{
\settowidth{\tmplength}{\textbf{Declaração}}
\setlength{\itemindent}{0cm}
\setlength{\listparindent}{0cm}
\setlength{\leftmargin}{\evensidemargin}
\addtolength{\leftmargin}{\tmplength}
\settowidth{\labelsep}{X}
\addtolength{\leftmargin}{\labelsep}
\setlength{\labelwidth}{\tmplength}
}
\begin{flushleft}
\item[\textbf{Declaração}\hfill]
\begin{ttfamily}
published property TableName;\end{ttfamily}


\end{flushleft}
\end{list}
\paragraph*{DataSource}\hspace*{\fill}

\begin{list}{}{
\settowidth{\tmplength}{\textbf{Declaração}}
\setlength{\itemindent}{0cm}
\setlength{\listparindent}{0cm}
\setlength{\leftmargin}{\evensidemargin}
\addtolength{\leftmargin}{\tmplength}
\settowidth{\labelsep}{X}
\addtolength{\leftmargin}{\labelsep}
\setlength{\labelwidth}{\tmplength}
}
\begin{flushleft}
\item[\textbf{Declaração}\hfill]
\begin{ttfamily}
published property DataSource;\end{ttfamily}


\end{flushleft}
\end{list}
\subsubsection*{\large{\textbf{Métodos}}\normalsize\hspace{1ex}\hfill}
\paragraph*{CreateFormLCL}\hspace*{\fill}

\begin{list}{}{
\settowidth{\tmplength}{\textbf{Declaração}}
\setlength{\itemindent}{0cm}
\setlength{\listparindent}{0cm}
\setlength{\leftmargin}{\evensidemargin}
\addtolength{\leftmargin}{\tmplength}
\settowidth{\labelsep}{X}
\addtolength{\leftmargin}{\labelsep}
\setlength{\labelwidth}{\tmplength}
}
\begin{flushleft}
\item[\textbf{Declaração}\hfill]
\begin{ttfamily}
protected procedure CreateFormLCL(aOwner:TScrollingWinControl); override;\end{ttfamily}


\end{flushleft}
\par
\item[\textbf{Descrição}]
O método \textbf{\begin{ttfamily}CreateFormLCL\end{ttfamily}} desenha um formulário TScrollingWinControl usando várias linha. \begin{itemize}
\item O modelo cria um registro usando os tipos de dados primitivos.
\item \textbf{EXEMPLO}:

\texttt{\\\nopagebreak[3]
\\\nopagebreak[3]
}\textbf{function}\texttt{~TDMAlunos.DmxScroller{\_}Form{\_}AlunoGetTemplate(aNext:~PSItem):~PSItem;\\\nopagebreak[3]
}\textbf{begin}\texttt{\\\nopagebreak[3]
~~}\textbf{with}\texttt{~DmxScroller{\_}Form1{\_}DS~}\textbf{do}\texttt{\\\nopagebreak[3]
~~}\textbf{begin}\texttt{\\\nopagebreak[3]
~~~~\textit{//~AlignmentLabels:=~taCenter;}\\\nopagebreak[3]
~~~~AlignmentLabels~:=~taLeftJustify;\\\nopagebreak[3]
~~~~\textit{//~AlignmentLabels~:=~taRightJustify~;}\\\nopagebreak[3]
~~~~Result~:=\\\nopagebreak[3]
~~~~~~NewSItem('~~~~~~Matrícula~~{\textbackslash}LLLLL'+CharFieldName+'matricula'+CharAccReadOnly+CharPfInKeyPrimary+CharPfInAutoIncrement,\\\nopagebreak[3]
~~~~~~NewSItem('~Nome~do~aluno:~~{\textbackslash}ssssssssssssssssssss`sssssss'+CharFieldName+'Nome'+CharPfInKey,\\\nopagebreak[3]
~~~~~~NewSItem('',\\\nopagebreak[3]
~~~~~~NewSItem('~~~~~~Endereço:~~{\textbackslash}ssssssssssssssssssss`sssssssssss'+CharFieldName+'Endereco',\\\nopagebreak[3]
~~~~~~NewSItem('~P.~Referência:~~{\textbackslash}ssssssssssssssssssss`sssss'+CharFieldName+'PontoDeReferencia',\\\nopagebreak[3]
~~~~~~NewSItem('~~~~~~~~~~~Cep:~~{\textbackslash}{\#}{\#}.{\#}{\#}{\#}-{\#}{\#}{\#}'+CharFieldName+'cep',\\\nopagebreak[3]
~~~~~~NewSItem('~~~~~~~~Estado:~~{\textbackslash}SS'+CharFieldName+'Estado'+CharForeignKeyN{\_}Um{\_}false+'Estados,Estado',\\\nopagebreak[3]
~~~~~~NewSItem('~~~~~~~~Cidade:~~{\textbackslash}ssssssssssssssssssss`sssss'+CharFieldName+'cidade'+CharForeignKeyN{\_}Um{\_}false+'Cidades,Estado;Cidade',\\\nopagebreak[3]
\\\nopagebreak[3]
~~~~~~NewSItem('',\\\nopagebreak[3]
~~~~~~aNext)))))))));\\\nopagebreak[3]
\\\nopagebreak[3]
~~}\textbf{end}\texttt{;\\\nopagebreak[3]
}\textbf{end}\texttt{;\\
}
\end{itemize}

\end{list}
\paragraph*{UpdateBuffers{\_}Controls}\hspace*{\fill}

\begin{list}{}{
\settowidth{\tmplength}{\textbf{Declaração}}
\setlength{\itemindent}{0cm}
\setlength{\listparindent}{0cm}
\setlength{\leftmargin}{\evensidemargin}
\addtolength{\leftmargin}{\tmplength}
\settowidth{\labelsep}{X}
\addtolength{\leftmargin}{\labelsep}
\setlength{\labelwidth}{\tmplength}
}
\begin{flushleft}
\item[\textbf{Declaração}\hfill]
\begin{ttfamily}
protected procedure UpdateBuffers{\_}Controls; override;\end{ttfamily}


\end{flushleft}
\par
\item[\textbf{Descrição}]
O método \textbf{\begin{ttfamily}UpdateBuffers{\_}Controls\end{ttfamily}} ler o buffer dos campos dos arquivos associados a classe \textbf{TDmxScroller{\_}Form{\_}Lcl{\_}attributes{\_}sql} para o buffer dos campos da classe \textbf{\begin{ttfamily}TDmxScroller{\_}Form{\_}Lcl{\_}attributes\end{ttfamily}(\ref{umi_ui_dmxscroller_form_lcl_attributes.TDmxScroller_Form_Lcl_attributes})}

\end{list}
\paragraph*{SetActiveTarget}\hspace*{\fill}

\begin{list}{}{
\settowidth{\tmplength}{\textbf{Declaração}}
\setlength{\itemindent}{0cm}
\setlength{\listparindent}{0cm}
\setlength{\leftmargin}{\evensidemargin}
\addtolength{\leftmargin}{\tmplength}
\settowidth{\labelsep}{X}
\addtolength{\leftmargin}{\labelsep}
\setlength{\labelwidth}{\tmplength}
}
\begin{flushleft}
\item[\textbf{Declaração}\hfill]
\begin{ttfamily}
protected procedure SetActiveTarget(aActive: Boolean); override;\end{ttfamily}


\end{flushleft}
\end{list}
\section{Funções e Procedimentos}
\subsection*{Register}
\begin{list}{}{
\settowidth{\tmplength}{\textbf{Declaração}}
\setlength{\itemindent}{0cm}
\setlength{\listparindent}{0cm}
\setlength{\leftmargin}{\evensidemargin}
\addtolength{\leftmargin}{\tmplength}
\settowidth{\labelsep}{X}
\addtolength{\leftmargin}{\labelsep}
\setlength{\labelwidth}{\tmplength}
}
\begin{flushleft}
\item[\textbf{Declaração}\hfill]
\begin{ttfamily}
procedure Register;\end{ttfamily}


\end{flushleft}
\end{list}
\section{Tipos}
\subsection*{TDmxFieldRec}
\begin{list}{}{
\settowidth{\tmplength}{\textbf{Declaração}}
\setlength{\itemindent}{0cm}
\setlength{\listparindent}{0cm}
\setlength{\leftmargin}{\evensidemargin}
\addtolength{\leftmargin}{\tmplength}
\settowidth{\labelsep}{X}
\addtolength{\leftmargin}{\labelsep}
\setlength{\labelwidth}{\tmplength}
}
\begin{flushleft}
\item[\textbf{Declaração}\hfill]
\begin{ttfamily}
TDmxFieldRec = uMi{\_}ui{\_}Dmxscroller{\_}form{\_}Lcl.TDmxFieldRec;\end{ttfamily}


\end{flushleft}
\end{list}
\subsection*{pDmxFieldRec}
\begin{list}{}{
\settowidth{\tmplength}{\textbf{Declaração}}
\setlength{\itemindent}{0cm}
\setlength{\listparindent}{0cm}
\setlength{\leftmargin}{\evensidemargin}
\addtolength{\leftmargin}{\tmplength}
\settowidth{\labelsep}{X}
\addtolength{\leftmargin}{\labelsep}
\setlength{\labelwidth}{\tmplength}
}
\begin{flushleft}
\item[\textbf{Declaração}\hfill]
\begin{ttfamily}
pDmxFieldRec = uMi{\_}ui{\_}Dmxscroller{\_}form{\_}Lcl.pDmxFieldRec;\end{ttfamily}


\end{flushleft}
\end{list}
\subsection*{SmallWord}
\begin{list}{}{
\settowidth{\tmplength}{\textbf{Declaração}}
\setlength{\itemindent}{0cm}
\setlength{\listparindent}{0cm}
\setlength{\leftmargin}{\evensidemargin}
\addtolength{\leftmargin}{\tmplength}
\settowidth{\labelsep}{X}
\addtolength{\leftmargin}{\labelsep}
\setlength{\labelwidth}{\tmplength}
}
\begin{flushleft}
\item[\textbf{Declaração}\hfill]
\begin{ttfamily}
SmallWord    = uMi{\_}ui{\_}Dmxscroller{\_}form{\_}Lcl.SmallWord;\end{ttfamily}


\end{flushleft}
\end{list}
\section{Constantes}
\subsection*{AccNormal}
\begin{list}{}{
\settowidth{\tmplength}{\textbf{Declaração}}
\setlength{\itemindent}{0cm}
\setlength{\listparindent}{0cm}
\setlength{\leftmargin}{\evensidemargin}
\addtolength{\leftmargin}{\tmplength}
\settowidth{\labelsep}{X}
\addtolength{\leftmargin}{\labelsep}
\setlength{\labelwidth}{\tmplength}
}
\begin{flushleft}
\item[\textbf{Declaração}\hfill]
\begin{ttfamily}
AccNormal  = mi{\_}rtl{\_}ui{\_}Dmxscroller.AccNormal;\end{ttfamily}


\end{flushleft}
\end{list}
\chapter{Unit uMi{\_}ui{\_}DmxScroller{\_}Form{\_}Lcl{\_}ds{\_}test}
\section{Uses}
\begin{itemize}
\item \begin{ttfamily}mi.rtl.Objects.Consts\end{ttfamily}(\ref{mi.rtl.Objects.Consts})\item \begin{ttfamily}Classes\end{ttfamily}\item \begin{ttfamily}SysUtils\end{ttfamily}\item \begin{ttfamily}DB\end{ttfamily}\item \begin{ttfamily}Forms\end{ttfamily}\item \begin{ttfamily}Controls\end{ttfamily}\item \begin{ttfamily}Graphics\end{ttfamily}\item \begin{ttfamily}Dialogs\end{ttfamily}\item \begin{ttfamily}StdCtrls\end{ttfamily}\item \begin{ttfamily}DBGrids\end{ttfamily}\item \begin{ttfamily}DBCtrls\end{ttfamily}\item \begin{ttfamily}uMi{\_}ui{\_}scrollbox{\_}lcl\end{ttfamily}(\ref{uMi_ui_scrollbox_lcl})\item \begin{ttfamily}uMi{\_}ui{\_}DmxScroller{\_}Form{\_}Lcl{\_}ds\end{ttfamily}(\ref{umi_ui_dmxscroller_form_lcl_ds})\item \begin{ttfamily}uMi{\_}ui{\_}Dmxscroller{\_}form{\_}lcl\end{ttfamily}\item \begin{ttfamily}mi{\_}rtl{\_}ui{\_}Dmxscroller\end{ttfamily}(\ref{mi_rtl_ui_Dmxscroller})\item \begin{ttfamily}uMi{\_}Ui{\_}DBCheckBox{\_}Lcl\end{ttfamily}(\ref{uMi_Ui_DBCheckBox_Lcl})\end{itemize}
\section{Visão Geral}
\begin{description}
\item[\texttt{\begin{ttfamily}TMi{\_}ui{\_}DmxScroller{\_}Form{\_}Lcl{\_}ds{\_}test\end{ttfamily} Classe}]
\end{description}
\section{Classes, Interfaces, Objetos e Registros}
\subsection*{TMi{\_}ui{\_}DmxScroller{\_}Form{\_}Lcl{\_}ds{\_}test Classe}
\subsubsection*{\large{\textbf{Hierarquia}}\normalsize\hspace{1ex}\hfill}
TMi{\_}ui{\_}DmxScroller{\_}Form{\_}Lcl{\_}ds{\_}test {$>$} TForm
\subsubsection*{\large{\textbf{Descrição}}\normalsize\hspace{1ex}\hfill}
A class \textbf{\begin{ttfamily}TMi{\_}ui{\_}DmxScroller{\_}Form{\_}Lcl{\_}ds{\_}test\end{ttfamily}} demonstra a classe \textbf{\begin{ttfamily}TDmxScroller{\_}Form{\_}Lcl{\_}DS\end{ttfamily}(\ref{umi_ui_dmxscroller_form_lcl_ds.TDmxScroller_Form_Lcl_DS})} integrada aos controles associados a \begin{ttfamily}DataSource1\end{ttfamily}(\ref{uMi_ui_DmxScroller_Form_Lcl_ds_test.TMi_ui_DmxScroller_Form_Lcl_ds_test-DataSource1}).

\begin{itemize}
\item \textbf{NOTA} \begin{itemize}
\item A class \textbf{\begin{ttfamily}TMi{\_}ui{\_}DmxScroller{\_}Form{\_}Lcl{\_}ds{\_}test\end{ttfamily}} cria o dataset associado a \begin{ttfamily}DataSource1\end{ttfamily}(\ref{uMi_ui_DmxScroller_Form_Lcl_ds_test.TMi_ui_DmxScroller_Form_Lcl_ds_test-DataSource1}) caso DataSource1.DataSet seja nil.
\item Caso DataSource1.dataSet {$<$}{$>$} nil, então o mesmo precisa ter os mesmo campos passado pelo template. \begin{itemize}
\item Obs: Se o campos passados em DataSource1.DataSet não sejão iguais aos templates, o sistema vai haver execeção.
\end{itemize}
\end{itemize}
\end{itemize}\subsubsection*{\large{\textbf{Campos}}\normalsize\hspace{1ex}\hfill}
\paragraph*{DataSource1}\hspace*{\fill}

\begin{list}{}{
\settowidth{\tmplength}{\textbf{Declaração}}
\setlength{\itemindent}{0cm}
\setlength{\listparindent}{0cm}
\setlength{\leftmargin}{\evensidemargin}
\addtolength{\leftmargin}{\tmplength}
\settowidth{\labelsep}{X}
\addtolength{\leftmargin}{\labelsep}
\setlength{\labelwidth}{\tmplength}
}
\begin{flushleft}
\item[\textbf{Declaração}\hfill]
\begin{ttfamily}
public DataSource1: TDataSource;\end{ttfamily}


\end{flushleft}
\end{list}
\paragraph*{DBGrid1}\hspace*{\fill}

\begin{list}{}{
\settowidth{\tmplength}{\textbf{Declaração}}
\setlength{\itemindent}{0cm}
\setlength{\listparindent}{0cm}
\setlength{\leftmargin}{\evensidemargin}
\addtolength{\leftmargin}{\tmplength}
\settowidth{\labelsep}{X}
\addtolength{\leftmargin}{\labelsep}
\setlength{\labelwidth}{\tmplength}
}
\begin{flushleft}
\item[\textbf{Declaração}\hfill]
\begin{ttfamily}
public DBGrid1: TDBGrid;\end{ttfamily}


\end{flushleft}
\end{list}
\paragraph*{DmxScroller{\_}Form{\_}Lcl{\_}DS1}\hspace*{\fill}

\begin{list}{}{
\settowidth{\tmplength}{\textbf{Declaração}}
\setlength{\itemindent}{0cm}
\setlength{\listparindent}{0cm}
\setlength{\leftmargin}{\evensidemargin}
\addtolength{\leftmargin}{\tmplength}
\settowidth{\labelsep}{X}
\addtolength{\leftmargin}{\labelsep}
\setlength{\labelwidth}{\tmplength}
}
\begin{flushleft}
\item[\textbf{Declaração}\hfill]
\begin{ttfamily}
public DmxScroller{\_}Form{\_}Lcl{\_}DS1: TDmxScroller{\_}Form{\_}Lcl{\_}DS;\end{ttfamily}


\end{flushleft}
\end{list}
\paragraph*{GroupBox1}\hspace*{\fill}

\begin{list}{}{
\settowidth{\tmplength}{\textbf{Declaração}}
\setlength{\itemindent}{0cm}
\setlength{\listparindent}{0cm}
\setlength{\leftmargin}{\evensidemargin}
\addtolength{\leftmargin}{\tmplength}
\settowidth{\labelsep}{X}
\addtolength{\leftmargin}{\labelsep}
\setlength{\labelwidth}{\tmplength}
}
\begin{flushleft}
\item[\textbf{Declaração}\hfill]
\begin{ttfamily}
public GroupBox1: TGroupBox;\end{ttfamily}


\end{flushleft}
\end{list}
\paragraph*{Mi{\_}ScrollBox{\_}LCL1}\hspace*{\fill}

\begin{list}{}{
\settowidth{\tmplength}{\textbf{Declaração}}
\setlength{\itemindent}{0cm}
\setlength{\listparindent}{0cm}
\setlength{\leftmargin}{\evensidemargin}
\addtolength{\leftmargin}{\tmplength}
\settowidth{\labelsep}{X}
\addtolength{\leftmargin}{\labelsep}
\setlength{\labelwidth}{\tmplength}
}
\begin{flushleft}
\item[\textbf{Declaração}\hfill]
\begin{ttfamily}
public Mi{\_}ScrollBox{\_}LCL1: TMi{\_}ScrollBox{\_}LCL;\end{ttfamily}


\end{flushleft}
\end{list}
\subsubsection*{\large{\textbf{Métodos}}\normalsize\hspace{1ex}\hfill}
\paragraph*{DmxScroller{\_}Form{\_}Lcl{\_}DS1GetTemplate}\hspace*{\fill}

\begin{list}{}{
\settowidth{\tmplength}{\textbf{Declaração}}
\setlength{\itemindent}{0cm}
\setlength{\listparindent}{0cm}
\setlength{\leftmargin}{\evensidemargin}
\addtolength{\leftmargin}{\tmplength}
\settowidth{\labelsep}{X}
\addtolength{\leftmargin}{\labelsep}
\setlength{\labelwidth}{\tmplength}
}
\begin{flushleft}
\item[\textbf{Declaração}\hfill]
\begin{ttfamily}
public function DmxScroller{\_}Form{\_}Lcl{\_}DS1GetTemplate(aNext: PSItem): PSItem;\end{ttfamily}


\end{flushleft}
\end{list}
\paragraph*{FormCreate}\hspace*{\fill}

\begin{list}{}{
\settowidth{\tmplength}{\textbf{Declaração}}
\setlength{\itemindent}{0cm}
\setlength{\listparindent}{0cm}
\setlength{\leftmargin}{\evensidemargin}
\addtolength{\leftmargin}{\tmplength}
\settowidth{\labelsep}{X}
\addtolength{\leftmargin}{\labelsep}
\setlength{\labelwidth}{\tmplength}
}
\begin{flushleft}
\item[\textbf{Declaração}\hfill]
\begin{ttfamily}
public procedure FormCreate(Sender: TObject);\end{ttfamily}


\end{flushleft}
\end{list}
\section{Variáveis}
\subsection*{Mi{\_}ui{\_}DmxScroller{\_}Form{\_}Lcl{\_}ds{\_}test}
\begin{list}{}{
\settowidth{\tmplength}{\textbf{Declaração}}
\setlength{\itemindent}{0cm}
\setlength{\listparindent}{0cm}
\setlength{\leftmargin}{\evensidemargin}
\addtolength{\leftmargin}{\tmplength}
\settowidth{\labelsep}{X}
\addtolength{\leftmargin}{\labelsep}
\setlength{\labelwidth}{\tmplength}
}
\begin{flushleft}
\item[\textbf{Declaração}\hfill]
\begin{ttfamily}
Mi{\_}ui{\_}DmxScroller{\_}Form{\_}Lcl{\_}ds{\_}test: TMi{\_}ui{\_}DmxScroller{\_}Form{\_}Lcl{\_}ds{\_}test;\end{ttfamily}


\end{flushleft}
\end{list}
\chapter{Unit uMi{\_}ui{\_}DmxScroller{\_}Form{\_}Lcl{\_}ds{\_}test2}
\section{Uses}
\begin{itemize}
\item \begin{ttfamily}Classes\end{ttfamily}\item \begin{ttfamily}SysUtils\end{ttfamily}\item \begin{ttfamily}Forms\end{ttfamily}\item \begin{ttfamily}Controls\end{ttfamily}\item \begin{ttfamily}Graphics\end{ttfamily}\item \begin{ttfamily}Dialogs\end{ttfamily}\item \begin{ttfamily}DBGrids\end{ttfamily}\item \begin{ttfamily}uMi{\_}ui{\_}DmxScroller{\_}Form{\_}Lcl{\_}ds{\_}test2{\_}dm\end{ttfamily}(\ref{uMi_ui_DmxScroller_Form_Lcl_ds_test2_dm})\item \begin{ttfamily}uMi{\_}ui{\_}scrollbox{\_}lcl\end{ttfamily}(\ref{uMi_ui_scrollbox_lcl})\end{itemize}
\section{Visão Geral}
\begin{description}
\item[\texttt{\begin{ttfamily}TMi{\_}ui{\_}DmxScroller{\_}Form{\_}Lcl{\_}ds{\_}test2\end{ttfamily} Classe}]
\end{description}
\section{Classes, Interfaces, Objetos e Registros}
\subsection*{TMi{\_}ui{\_}DmxScroller{\_}Form{\_}Lcl{\_}ds{\_}test2 Classe}
\subsubsection*{\large{\textbf{Hierarquia}}\normalsize\hspace{1ex}\hfill}
TMi{\_}ui{\_}DmxScroller{\_}Form{\_}Lcl{\_}ds{\_}test2 {$>$} TForm
%%%%Descrição
\subsubsection*{\large{\textbf{Campos}}\normalsize\hspace{1ex}\hfill}
\paragraph*{DBGrid1}\hspace*{\fill}

\begin{list}{}{
\settowidth{\tmplength}{\textbf{Declaração}}
\setlength{\itemindent}{0cm}
\setlength{\listparindent}{0cm}
\setlength{\leftmargin}{\evensidemargin}
\addtolength{\leftmargin}{\tmplength}
\settowidth{\labelsep}{X}
\addtolength{\leftmargin}{\labelsep}
\setlength{\labelwidth}{\tmplength}
}
\begin{flushleft}
\item[\textbf{Declaração}\hfill]
\begin{ttfamily}
public DBGrid1: TDBGrid;\end{ttfamily}


\end{flushleft}
\end{list}
\paragraph*{Mi{\_}ScrollBox{\_}LCL1}\hspace*{\fill}

\begin{list}{}{
\settowidth{\tmplength}{\textbf{Declaração}}
\setlength{\itemindent}{0cm}
\setlength{\listparindent}{0cm}
\setlength{\leftmargin}{\evensidemargin}
\addtolength{\leftmargin}{\tmplength}
\settowidth{\labelsep}{X}
\addtolength{\leftmargin}{\labelsep}
\setlength{\labelwidth}{\tmplength}
}
\begin{flushleft}
\item[\textbf{Declaração}\hfill]
\begin{ttfamily}
public Mi{\_}ScrollBox{\_}LCL1: TMi{\_}ScrollBox{\_}LCL;\end{ttfamily}


\end{flushleft}
\end{list}
\subsubsection*{\large{\textbf{Métodos}}\normalsize\hspace{1ex}\hfill}
\paragraph*{FormCreate}\hspace*{\fill}

\begin{list}{}{
\settowidth{\tmplength}{\textbf{Declaração}}
\setlength{\itemindent}{0cm}
\setlength{\listparindent}{0cm}
\setlength{\leftmargin}{\evensidemargin}
\addtolength{\leftmargin}{\tmplength}
\settowidth{\labelsep}{X}
\addtolength{\leftmargin}{\labelsep}
\setlength{\labelwidth}{\tmplength}
}
\begin{flushleft}
\item[\textbf{Declaração}\hfill]
\begin{ttfamily}
public procedure FormCreate(Sender: TObject);\end{ttfamily}


\end{flushleft}
\end{list}
\section{Variáveis}
\subsection*{Mi{\_}ui{\_}DmxScroller{\_}Form{\_}Lcl{\_}ds{\_}test2}
\begin{list}{}{
\settowidth{\tmplength}{\textbf{Declaração}}
\setlength{\itemindent}{0cm}
\setlength{\listparindent}{0cm}
\setlength{\leftmargin}{\evensidemargin}
\addtolength{\leftmargin}{\tmplength}
\settowidth{\labelsep}{X}
\addtolength{\leftmargin}{\labelsep}
\setlength{\labelwidth}{\tmplength}
}
\begin{flushleft}
\item[\textbf{Declaração}\hfill]
\begin{ttfamily}
Mi{\_}ui{\_}DmxScroller{\_}Form{\_}Lcl{\_}ds{\_}test2: TMi{\_}ui{\_}DmxScroller{\_}Form{\_}Lcl{\_}ds{\_}test2;\end{ttfamily}


\end{flushleft}
\end{list}
\chapter{Unit uMi{\_}ui{\_}DmxScroller{\_}Form{\_}Lcl{\_}ds{\_}test2{\_}dm}
\section{Uses}
\begin{itemize}
\item \begin{ttfamily}Classes\end{ttfamily}\item \begin{ttfamily}SysUtils\end{ttfamily}\item \begin{ttfamily}ActnList\end{ttfamily}\item \begin{ttfamily}DB\end{ttfamily}\item \begin{ttfamily}uMi{\_}ui{\_}DmxScroller{\_}Form{\_}Lcl{\_}ds\end{ttfamily}(\ref{umi_ui_dmxscroller_form_lcl_ds})\item \begin{ttfamily}mi{\_}rtl{\_}ui{\_}Dmxscroller\end{ttfamily}(\ref{mi_rtl_ui_Dmxscroller})\end{itemize}
\section{Visão Geral}
\begin{description}
\item[\texttt{\begin{ttfamily}TMi{\_}ui{\_}DmxScroller{\_}Form{\_}Lcl{\_}ds{\_}test2{\_}dm\end{ttfamily} Classe}]
\end{description}
\section{Classes, Interfaces, Objetos e Registros}
\subsection*{TMi{\_}ui{\_}DmxScroller{\_}Form{\_}Lcl{\_}ds{\_}test2{\_}dm Classe}
\subsubsection*{\large{\textbf{Hierarquia}}\normalsize\hspace{1ex}\hfill}
TMi{\_}ui{\_}DmxScroller{\_}Form{\_}Lcl{\_}ds{\_}test2{\_}dm {$>$} TDataModule
%%%%Descrição
\subsubsection*{\large{\textbf{Campos}}\normalsize\hspace{1ex}\hfill}
\paragraph*{GoBof}\hspace*{\fill}

\begin{list}{}{
\settowidth{\tmplength}{\textbf{Declaração}}
\setlength{\itemindent}{0cm}
\setlength{\listparindent}{0cm}
\setlength{\leftmargin}{\evensidemargin}
\addtolength{\leftmargin}{\tmplength}
\settowidth{\labelsep}{X}
\addtolength{\leftmargin}{\labelsep}
\setlength{\labelwidth}{\tmplength}
}
\begin{flushleft}
\item[\textbf{Declaração}\hfill]
\begin{ttfamily}
public GoBof: TAction;\end{ttfamily}


\end{flushleft}
\end{list}
\paragraph*{Next}\hspace*{\fill}

\begin{list}{}{
\settowidth{\tmplength}{\textbf{Declaração}}
\setlength{\itemindent}{0cm}
\setlength{\listparindent}{0cm}
\setlength{\leftmargin}{\evensidemargin}
\addtolength{\leftmargin}{\tmplength}
\settowidth{\labelsep}{X}
\addtolength{\leftmargin}{\labelsep}
\setlength{\labelwidth}{\tmplength}
}
\begin{flushleft}
\item[\textbf{Declaração}\hfill]
\begin{ttfamily}
public Next: TAction;\end{ttfamily}


\end{flushleft}
\end{list}
\paragraph*{Prev}\hspace*{\fill}

\begin{list}{}{
\settowidth{\tmplength}{\textbf{Declaração}}
\setlength{\itemindent}{0cm}
\setlength{\listparindent}{0cm}
\setlength{\leftmargin}{\evensidemargin}
\addtolength{\leftmargin}{\tmplength}
\settowidth{\labelsep}{X}
\addtolength{\leftmargin}{\labelsep}
\setlength{\labelwidth}{\tmplength}
}
\begin{flushleft}
\item[\textbf{Declaração}\hfill]
\begin{ttfamily}
public Prev: TAction;\end{ttfamily}


\end{flushleft}
\end{list}
\paragraph*{GoEof}\hspace*{\fill}

\begin{list}{}{
\settowidth{\tmplength}{\textbf{Declaração}}
\setlength{\itemindent}{0cm}
\setlength{\listparindent}{0cm}
\setlength{\leftmargin}{\evensidemargin}
\addtolength{\leftmargin}{\tmplength}
\settowidth{\labelsep}{X}
\addtolength{\leftmargin}{\labelsep}
\setlength{\labelwidth}{\tmplength}
}
\begin{flushleft}
\item[\textbf{Declaração}\hfill]
\begin{ttfamily}
public GoEof: TAction;\end{ttfamily}


\end{flushleft}
\end{list}
\paragraph*{ActionList1}\hspace*{\fill}

\begin{list}{}{
\settowidth{\tmplength}{\textbf{Declaração}}
\setlength{\itemindent}{0cm}
\setlength{\listparindent}{0cm}
\setlength{\leftmargin}{\evensidemargin}
\addtolength{\leftmargin}{\tmplength}
\settowidth{\labelsep}{X}
\addtolength{\leftmargin}{\labelsep}
\setlength{\labelwidth}{\tmplength}
}
\begin{flushleft}
\item[\textbf{Declaração}\hfill]
\begin{ttfamily}
public ActionList1: TActionList;\end{ttfamily}


\end{flushleft}
\end{list}
\paragraph*{DataSource1}\hspace*{\fill}

\begin{list}{}{
\settowidth{\tmplength}{\textbf{Declaração}}
\setlength{\itemindent}{0cm}
\setlength{\listparindent}{0cm}
\setlength{\leftmargin}{\evensidemargin}
\addtolength{\leftmargin}{\tmplength}
\settowidth{\labelsep}{X}
\addtolength{\leftmargin}{\labelsep}
\setlength{\labelwidth}{\tmplength}
}
\begin{flushleft}
\item[\textbf{Declaração}\hfill]
\begin{ttfamily}
public DataSource1: TDataSource;\end{ttfamily}


\end{flushleft}
\end{list}
\paragraph*{DmxScroller{\_}Form{\_}Lcl{\_}DS1}\hspace*{\fill}

\begin{list}{}{
\settowidth{\tmplength}{\textbf{Declaração}}
\setlength{\itemindent}{0cm}
\setlength{\listparindent}{0cm}
\setlength{\leftmargin}{\evensidemargin}
\addtolength{\leftmargin}{\tmplength}
\settowidth{\labelsep}{X}
\addtolength{\leftmargin}{\labelsep}
\setlength{\labelwidth}{\tmplength}
}
\begin{flushleft}
\item[\textbf{Declaração}\hfill]
\begin{ttfamily}
public DmxScroller{\_}Form{\_}Lcl{\_}DS1: TDmxScroller{\_}Form{\_}Lcl{\_}DS;\end{ttfamily}


\end{flushleft}
\end{list}
\paragraph*{Excluir}\hspace*{\fill}

\begin{list}{}{
\settowidth{\tmplength}{\textbf{Declaração}}
\setlength{\itemindent}{0cm}
\setlength{\listparindent}{0cm}
\setlength{\leftmargin}{\evensidemargin}
\addtolength{\leftmargin}{\tmplength}
\settowidth{\labelsep}{X}
\addtolength{\leftmargin}{\labelsep}
\setlength{\labelwidth}{\tmplength}
}
\begin{flushleft}
\item[\textbf{Declaração}\hfill]
\begin{ttfamily}
public Excluir: TAction;\end{ttfamily}


\end{flushleft}
\end{list}
\paragraph*{Gravar}\hspace*{\fill}

\begin{list}{}{
\settowidth{\tmplength}{\textbf{Declaração}}
\setlength{\itemindent}{0cm}
\setlength{\listparindent}{0cm}
\setlength{\leftmargin}{\evensidemargin}
\addtolength{\leftmargin}{\tmplength}
\settowidth{\labelsep}{X}
\addtolength{\leftmargin}{\labelsep}
\setlength{\labelwidth}{\tmplength}
}
\begin{flushleft}
\item[\textbf{Declaração}\hfill]
\begin{ttfamily}
public Gravar: TAction;\end{ttfamily}


\end{flushleft}
\end{list}
\paragraph*{Novo}\hspace*{\fill}

\begin{list}{}{
\settowidth{\tmplength}{\textbf{Declaração}}
\setlength{\itemindent}{0cm}
\setlength{\listparindent}{0cm}
\setlength{\leftmargin}{\evensidemargin}
\addtolength{\leftmargin}{\tmplength}
\settowidth{\labelsep}{X}
\addtolength{\leftmargin}{\labelsep}
\setlength{\labelwidth}{\tmplength}
}
\begin{flushleft}
\item[\textbf{Declaração}\hfill]
\begin{ttfamily}
public Novo: TAction;\end{ttfamily}


\end{flushleft}
\end{list}
\paragraph*{Pesquisar}\hspace*{\fill}

\begin{list}{}{
\settowidth{\tmplength}{\textbf{Declaração}}
\setlength{\itemindent}{0cm}
\setlength{\listparindent}{0cm}
\setlength{\leftmargin}{\evensidemargin}
\addtolength{\leftmargin}{\tmplength}
\settowidth{\labelsep}{X}
\addtolength{\leftmargin}{\labelsep}
\setlength{\labelwidth}{\tmplength}
}
\begin{flushleft}
\item[\textbf{Declaração}\hfill]
\begin{ttfamily}
public Pesquisar: TAction;\end{ttfamily}


\end{flushleft}
\end{list}
\subsubsection*{\large{\textbf{Métodos}}\normalsize\hspace{1ex}\hfill}
\paragraph*{DataModuleCreate}\hspace*{\fill}

\begin{list}{}{
\settowidth{\tmplength}{\textbf{Declaração}}
\setlength{\itemindent}{0cm}
\setlength{\listparindent}{0cm}
\setlength{\leftmargin}{\evensidemargin}
\addtolength{\leftmargin}{\tmplength}
\settowidth{\labelsep}{X}
\addtolength{\leftmargin}{\labelsep}
\setlength{\labelwidth}{\tmplength}
}
\begin{flushleft}
\item[\textbf{Declaração}\hfill]
\begin{ttfamily}
public procedure DataModuleCreate(Sender: TObject);\end{ttfamily}


\end{flushleft}
\end{list}
\paragraph*{GoBofExecute}\hspace*{\fill}

\begin{list}{}{
\settowidth{\tmplength}{\textbf{Declaração}}
\setlength{\itemindent}{0cm}
\setlength{\listparindent}{0cm}
\setlength{\leftmargin}{\evensidemargin}
\addtolength{\leftmargin}{\tmplength}
\settowidth{\labelsep}{X}
\addtolength{\leftmargin}{\labelsep}
\setlength{\labelwidth}{\tmplength}
}
\begin{flushleft}
\item[\textbf{Declaração}\hfill]
\begin{ttfamily}
public procedure GoBofExecute(Sender: TObject);\end{ttfamily}


\end{flushleft}
\end{list}
\paragraph*{DmxScroller{\_}Form{\_}Lcl{\_}DS1AddTemplate}\hspace*{\fill}

\begin{list}{}{
\settowidth{\tmplength}{\textbf{Declaração}}
\setlength{\itemindent}{0cm}
\setlength{\listparindent}{0cm}
\setlength{\leftmargin}{\evensidemargin}
\addtolength{\leftmargin}{\tmplength}
\settowidth{\labelsep}{X}
\addtolength{\leftmargin}{\labelsep}
\setlength{\labelwidth}{\tmplength}
}
\begin{flushleft}
\item[\textbf{Declaração}\hfill]
\begin{ttfamily}
public procedure DmxScroller{\_}Form{\_}Lcl{\_}DS1AddTemplate( const aUiDmxScroller: TUiDmxScroller);\end{ttfamily}


\end{flushleft}
\end{list}
\paragraph*{ExcluirExecute}\hspace*{\fill}

\begin{list}{}{
\settowidth{\tmplength}{\textbf{Declaração}}
\setlength{\itemindent}{0cm}
\setlength{\listparindent}{0cm}
\setlength{\leftmargin}{\evensidemargin}
\addtolength{\leftmargin}{\tmplength}
\settowidth{\labelsep}{X}
\addtolength{\leftmargin}{\labelsep}
\setlength{\labelwidth}{\tmplength}
}
\begin{flushleft}
\item[\textbf{Declaração}\hfill]
\begin{ttfamily}
public procedure ExcluirExecute(Sender: TObject);\end{ttfamily}


\end{flushleft}
\end{list}
\paragraph*{GoEofExecute}\hspace*{\fill}

\begin{list}{}{
\settowidth{\tmplength}{\textbf{Declaração}}
\setlength{\itemindent}{0cm}
\setlength{\listparindent}{0cm}
\setlength{\leftmargin}{\evensidemargin}
\addtolength{\leftmargin}{\tmplength}
\settowidth{\labelsep}{X}
\addtolength{\leftmargin}{\labelsep}
\setlength{\labelwidth}{\tmplength}
}
\begin{flushleft}
\item[\textbf{Declaração}\hfill]
\begin{ttfamily}
public procedure GoEofExecute(Sender: TObject);\end{ttfamily}


\end{flushleft}
\end{list}
\paragraph*{GravarExecute}\hspace*{\fill}

\begin{list}{}{
\settowidth{\tmplength}{\textbf{Declaração}}
\setlength{\itemindent}{0cm}
\setlength{\listparindent}{0cm}
\setlength{\leftmargin}{\evensidemargin}
\addtolength{\leftmargin}{\tmplength}
\settowidth{\labelsep}{X}
\addtolength{\leftmargin}{\labelsep}
\setlength{\labelwidth}{\tmplength}
}
\begin{flushleft}
\item[\textbf{Declaração}\hfill]
\begin{ttfamily}
public procedure GravarExecute(Sender: TObject);\end{ttfamily}


\end{flushleft}
\end{list}
\paragraph*{NextExecute}\hspace*{\fill}

\begin{list}{}{
\settowidth{\tmplength}{\textbf{Declaração}}
\setlength{\itemindent}{0cm}
\setlength{\listparindent}{0cm}
\setlength{\leftmargin}{\evensidemargin}
\addtolength{\leftmargin}{\tmplength}
\settowidth{\labelsep}{X}
\addtolength{\leftmargin}{\labelsep}
\setlength{\labelwidth}{\tmplength}
}
\begin{flushleft}
\item[\textbf{Declaração}\hfill]
\begin{ttfamily}
public procedure NextExecute(Sender: TObject);\end{ttfamily}


\end{flushleft}
\end{list}
\paragraph*{NovoExecute}\hspace*{\fill}

\begin{list}{}{
\settowidth{\tmplength}{\textbf{Declaração}}
\setlength{\itemindent}{0cm}
\setlength{\listparindent}{0cm}
\setlength{\leftmargin}{\evensidemargin}
\addtolength{\leftmargin}{\tmplength}
\settowidth{\labelsep}{X}
\addtolength{\leftmargin}{\labelsep}
\setlength{\labelwidth}{\tmplength}
}
\begin{flushleft}
\item[\textbf{Declaração}\hfill]
\begin{ttfamily}
public procedure NovoExecute(Sender: TObject);\end{ttfamily}


\end{flushleft}
\end{list}
\paragraph*{PesquisarExecute}\hspace*{\fill}

\begin{list}{}{
\settowidth{\tmplength}{\textbf{Declaração}}
\setlength{\itemindent}{0cm}
\setlength{\listparindent}{0cm}
\setlength{\leftmargin}{\evensidemargin}
\addtolength{\leftmargin}{\tmplength}
\settowidth{\labelsep}{X}
\addtolength{\leftmargin}{\labelsep}
\setlength{\labelwidth}{\tmplength}
}
\begin{flushleft}
\item[\textbf{Declaração}\hfill]
\begin{ttfamily}
public procedure PesquisarExecute(Sender: TObject);\end{ttfamily}


\end{flushleft}
\end{list}
\paragraph*{PrevExecute}\hspace*{\fill}

\begin{list}{}{
\settowidth{\tmplength}{\textbf{Declaração}}
\setlength{\itemindent}{0cm}
\setlength{\listparindent}{0cm}
\setlength{\leftmargin}{\evensidemargin}
\addtolength{\leftmargin}{\tmplength}
\settowidth{\labelsep}{X}
\addtolength{\leftmargin}{\labelsep}
\setlength{\labelwidth}{\tmplength}
}
\begin{flushleft}
\item[\textbf{Declaração}\hfill]
\begin{ttfamily}
public procedure PrevExecute(Sender: TObject);\end{ttfamily}


\end{flushleft}
\end{list}
\section{Variáveis}
\subsection*{Mi{\_}ui{\_}DmxScroller{\_}Form{\_}Lcl{\_}ds{\_}test2{\_}dm}
\begin{list}{}{
\settowidth{\tmplength}{\textbf{Declaração}}
\setlength{\itemindent}{0cm}
\setlength{\listparindent}{0cm}
\setlength{\leftmargin}{\evensidemargin}
\addtolength{\leftmargin}{\tmplength}
\settowidth{\labelsep}{X}
\addtolength{\leftmargin}{\labelsep}
\setlength{\labelwidth}{\tmplength}
}
\begin{flushleft}
\item[\textbf{Declaração}\hfill]
\begin{ttfamily}
Mi{\_}ui{\_}DmxScroller{\_}Form{\_}Lcl{\_}ds{\_}test2{\_}dm: TMi{\_}ui{\_}DmxScroller{\_}Form{\_}Lcl{\_}ds{\_}test2{\_}dm;\end{ttfamily}


\end{flushleft}
\end{list}
\chapter{Unit uMi{\_}ui{\_}DmxScroller{\_}Form{\_}Lcl{\_}ds{\_}test{\_}dm}
\section{Uses}
\begin{itemize}
\item \begin{ttfamily}Classes\end{ttfamily}\item \begin{ttfamily}SysUtils\end{ttfamily}\item \begin{ttfamily}DB\end{ttfamily}\item \begin{ttfamily}uMi{\_}ui{\_}DmxScroller{\_}Form{\_}Lcl{\_}ds\end{ttfamily}(\ref{umi_ui_dmxscroller_form_lcl_ds})\item \begin{ttfamily}mi{\_}rtl{\_}ui{\_}Dmxscroller\end{ttfamily}(\ref{mi_rtl_ui_Dmxscroller})\end{itemize}
\section{Visão Geral}
\begin{description}
\item[\texttt{\begin{ttfamily}TMi{\_}ui{\_}DmxScroller{\_}Form{\_}Lcl{\_}ds{\_}test{\_}dm\end{ttfamily} Classe}]
\end{description}
\section{Classes, Interfaces, Objetos e Registros}
\subsection*{TMi{\_}ui{\_}DmxScroller{\_}Form{\_}Lcl{\_}ds{\_}test{\_}dm Classe}
\subsubsection*{\large{\textbf{Hierarquia}}\normalsize\hspace{1ex}\hfill}
TMi{\_}ui{\_}DmxScroller{\_}Form{\_}Lcl{\_}ds{\_}test{\_}dm {$>$} TDataModule
%%%%Descrição
\subsubsection*{\large{\textbf{Campos}}\normalsize\hspace{1ex}\hfill}
\paragraph*{DataSource1}\hspace*{\fill}

\begin{list}{}{
\settowidth{\tmplength}{\textbf{Declaração}}
\setlength{\itemindent}{0cm}
\setlength{\listparindent}{0cm}
\setlength{\leftmargin}{\evensidemargin}
\addtolength{\leftmargin}{\tmplength}
\settowidth{\labelsep}{X}
\addtolength{\leftmargin}{\labelsep}
\setlength{\labelwidth}{\tmplength}
}
\begin{flushleft}
\item[\textbf{Declaração}\hfill]
\begin{ttfamily}
public DataSource1: TDataSource;\end{ttfamily}


\end{flushleft}
\end{list}
\paragraph*{DmxScroller{\_}Form{\_}Lcl{\_}DS1}\hspace*{\fill}

\begin{list}{}{
\settowidth{\tmplength}{\textbf{Declaração}}
\setlength{\itemindent}{0cm}
\setlength{\listparindent}{0cm}
\setlength{\leftmargin}{\evensidemargin}
\addtolength{\leftmargin}{\tmplength}
\settowidth{\labelsep}{X}
\addtolength{\leftmargin}{\labelsep}
\setlength{\labelwidth}{\tmplength}
}
\begin{flushleft}
\item[\textbf{Declaração}\hfill]
\begin{ttfamily}
public DmxScroller{\_}Form{\_}Lcl{\_}DS1: TDmxScroller{\_}Form{\_}Lcl{\_}DS;\end{ttfamily}


\end{flushleft}
\end{list}
\subsubsection*{\large{\textbf{Métodos}}\normalsize\hspace{1ex}\hfill}
\paragraph*{DmxScroller{\_}Form{\_}Lcl{\_}DS1GetTemplate}\hspace*{\fill}

\begin{list}{}{
\settowidth{\tmplength}{\textbf{Declaração}}
\setlength{\itemindent}{0cm}
\setlength{\listparindent}{0cm}
\setlength{\leftmargin}{\evensidemargin}
\addtolength{\leftmargin}{\tmplength}
\settowidth{\labelsep}{X}
\addtolength{\leftmargin}{\labelsep}
\setlength{\labelwidth}{\tmplength}
}
\begin{flushleft}
\item[\textbf{Declaração}\hfill]
\begin{ttfamily}
public function DmxScroller{\_}Form{\_}Lcl{\_}DS1GetTemplate(aNext: PSItem): PSItem;\end{ttfamily}


\end{flushleft}
\end{list}
\section{Variáveis}
\subsection*{Mi{\_}ui{\_}DmxScroller{\_}Form{\_}Lcl{\_}ds{\_}test{\_}dm}
\begin{list}{}{
\settowidth{\tmplength}{\textbf{Declaração}}
\setlength{\itemindent}{0cm}
\setlength{\listparindent}{0cm}
\setlength{\leftmargin}{\evensidemargin}
\addtolength{\leftmargin}{\tmplength}
\settowidth{\labelsep}{X}
\addtolength{\leftmargin}{\labelsep}
\setlength{\labelwidth}{\tmplength}
}
\begin{flushleft}
\item[\textbf{Declaração}\hfill]
\begin{ttfamily}
Mi{\_}ui{\_}DmxScroller{\_}Form{\_}Lcl{\_}ds{\_}test{\_}dm: TMi{\_}ui{\_}DmxScroller{\_}Form{\_}Lcl{\_}ds{\_}test{\_}dm;\end{ttfamily}


\end{flushleft}
\end{list}
\chapter{Unit uMI{\_}UI{\_}InputBox}
\section{Descrição}
A unit \textbf{\begin{ttfamily}uMI{\_}UI{\_}InputBox\end{ttfamily}} implementa o formulário \begin{ttfamily}TMI{\_}UI{\_}InputBox\end{ttfamily}(\ref{uMI_UI_InputBox.TMI_UI_InputBox}) usado para criar formulário baseado em Template \begin{ttfamily}PSITem\end{ttfamily}(\ref{mi_rtl_ui_Dmxscroller-PSItem}).

\begin{itemize}
\item \textbf{VERSÃO} \begin{itemize}
\item Alpha {-} 0.5.0.693
\end{itemize}
\item \textbf{CÓDIGO FONTE}: \begin{itemize}
\item 
\end{itemize}
\item \textbf{PENDÊNCIAS}
\end{itemize}\begin{itemize}
\item \textbf{HISTÓRICO} \begin{itemize}
\item Criado por: Paulo Sérgio da Silva Pacheco paulosspacheco@yahoo.com.br)
\item \textbf{2022{-}05{-}17} \begin{itemize}
\item T12 Análise de como será a classe \begin{ttfamily}TMI{\_}UI{\_}InputBox\end{ttfamily}(\ref{uMI_UI_InputBox.TMI_UI_InputBox}). ✔
\item T12 Criar a unit \textbf{\begin{ttfamily}uMI{\_}UI{\_}InputBox\end{ttfamily}}. ✔
\item T12 Criar formulário \begin{ttfamily}TMI{\_}UI{\_}InputBox\end{ttfamily}(\ref{uMI_UI_InputBox.TMI_UI_InputBox}); ✔
\item T12 Adicionar o componente ButtonPanel1 e habilitar os botões ok e cancel; ✔
\item T12 Adicionar o componente Mi{\_}ScrollBox{\_}LCL1 ; ✔
\item T12 Criar evento: function DmxScroller{\_}Form{\_}Lcl1GetTemplate; ✔
\item T12 Criar atributo protected {\_}FormSItem : \begin{ttfamily}PSitem\end{ttfamily}(\ref{mi_rtl_ui_Dmxscroller-PSItem}); ✔
\item T12 Criar propriedade Template:AnsiString; ✔ \begin{itemize}
\item Criar método Set{\_}Template(aTemplate:AnsiString); ✔
\end{itemize}
\end{itemize}
\item \textbf{2022{-}05{-}18} \begin{itemize}
\item \textbf{10:51} \begin{itemize}
\item As alterações que fiz ontem no método \begin{ttfamily}TObjectsMethods.StringToSItem\end{ttfamily}(\ref{mi.rtl.Objects.Methods.TObjectsMethods-StringToSItem})() criou efeito colateral. \begin{itemize}
\item Corrigido. ✔
\end{itemize}
\end{itemize}
\item \textbf{14:28} \begin{itemize}
\item Criar função: \begin{itemize}
\item function \begin{ttfamily}InputBox\end{ttfamily}(\ref{uMI_UI_InputBox-InputBox})(): TModalResult;
\end{itemize}
\end{itemize}
\end{itemize}
\item \textbf{2022{-}05{-}19} \begin{itemize}
\item \textbf{11:13} \begin{itemize}
\item Criar os eventos \begin{itemize}
\item OnEnterLocal ✔
\item OnExistLocal ✔
\item onEnterFieldLocal ✔
\item OnExitDieldLocal ✔
\end{itemize}
\item Criar função: \begin{itemize}
\item function MI{\_}MsgBox1MessageBox{\_}04{\_}PSItem(aPSItem: TMI{\_}MsgBoxTypes.PSItem; DlgType: TMsgDlgType; Buttons: TMsgDlgButtons; ButtonDefault: TMsgDlgBtn ): TModalResult;
\end{itemize}
\end{itemize}
\end{itemize}
\end{itemize}
\end{itemize}
\section{Uses}
\begin{itemize}
\item \begin{ttfamily}Classes\end{ttfamily}\item \begin{ttfamily}SysUtils\end{ttfamily}\item \begin{ttfamily}Forms\end{ttfamily}\item \begin{ttfamily}Controls\end{ttfamily}\item \begin{ttfamily}Graphics\end{ttfamily}\item \begin{ttfamily}Dialogs\end{ttfamily}\item \begin{ttfamily}StdCtrls\end{ttfamily}\item \begin{ttfamily}ExtCtrls\end{ttfamily}\item \begin{ttfamily}ButtonPanel\end{ttfamily}\item \begin{ttfamily}uMi{\_}ui{\_}DmxScroller{\_}Form{\_}Lcl\end{ttfamily}\item \begin{ttfamily}mi{\_}rtl{\_}ui{\_}Dmxscroller\end{ttfamily}(\ref{mi_rtl_ui_Dmxscroller})\item \begin{ttfamily}uMi{\_}ui{\_}scrollbox{\_}lcl\end{ttfamily}(\ref{uMi_ui_scrollbox_lcl})\item \begin{ttfamily}uMi{\_}ui{\_}DmxScroller{\_}Form{\_}Lcl{\_}ds\end{ttfamily}(\ref{umi_ui_dmxscroller_form_lcl_ds})\end{itemize}
\section{Visão Geral}
\begin{description}
\item[\texttt{\begin{ttfamily}TMI{\_}UI{\_}InputBox\end{ttfamily} Classe}]
\end{description}
\begin{description}
\item[\texttt{InputBox}]
\end{description}
\section{Classes, Interfaces, Objetos e Registros}
\subsection*{TMI{\_}UI{\_}InputBox Classe}
\subsubsection*{\large{\textbf{Hierarquia}}\normalsize\hspace{1ex}\hfill}
TMI{\_}UI{\_}InputBox {$>$} TForm
\subsubsection*{\large{\textbf{Descrição}}\normalsize\hspace{1ex}\hfill}
A class \textbf{\begin{ttfamily}TMI{\_}UI{\_}InputBox\end{ttfamily}} edita uma formulário passado em forma de \begin{ttfamily}Template\end{ttfamily}(\ref{uMI_UI_InputBox.TMI_UI_InputBox-Template}) e devolve o formulário LCL caso o botão \textbf{OK} seja pressionando ou nil caso o botão \textbf{Cancelar} seja pressionando.

\begin{itemize}
\item \textbf{EXEMPLO} \begin{itemize}
\item O exemplo abaixo cria um formulário e permite interagir com com os eventos ao entrar e ao sair do formulário.
\end{itemize} \texttt{\\\nopagebreak[3]
\\\nopagebreak[3]
}\textbf{procedure}\texttt{~InputBoxOnEnter(aDmxScroller:~TUiDmxScroller);\\\nopagebreak[3]
~\textit{//Ao~entrar~no~formulário~este~evento~inicia~os~campos~nome,~endereço~e~cep}\\\nopagebreak[3]
\\\nopagebreak[3]
~~}\textbf{procedure}\texttt{~SetValue(Field:}\textbf{String}\texttt{;Value:}\textbf{String}\texttt{);\\\nopagebreak[3]
~~~~}\textbf{var}\texttt{\\\nopagebreak[3]
~~~~~~aField:~pDmxFieldRec;\\\nopagebreak[3]
~~}\textbf{begin}\texttt{\\\nopagebreak[3]
~~~~}\textbf{with}\texttt{~aDmxScroller~}\textbf{do}\texttt{\\\nopagebreak[3]
~~~~}\textbf{begin}\texttt{\\\nopagebreak[3]
~~~~~~aField~:=~FieldByName(Field);\\\nopagebreak[3]
~~~~~~}\textbf{if}\texttt{~aField{$<$}{$>$}}\textbf{nil}\texttt{\\\nopagebreak[3]
~~~~~~}\textbf{Then}\texttt{~aField.AsString:=~value;\\\nopagebreak[3]
~~~~}\textbf{end}\texttt{;\\\nopagebreak[3]
~~}\textbf{end}\texttt{;\\\nopagebreak[3]
\\\nopagebreak[3]
}\textbf{begin}\texttt{\\\nopagebreak[3]
~~}\textbf{with}\texttt{~aDmxScroller~}\textbf{do}\texttt{\\\nopagebreak[3]
~~}\textbf{begin}\texttt{\\\nopagebreak[3]
~~~~setValue('nome','Paulo~Henrique');\\\nopagebreak[3]
~~~~setValue('endereço','Rua~Francisco~de~Souza~Oliveira,~15');\\\nopagebreak[3]
~~~~setValue('cep','60310770');\\\nopagebreak[3]
~~}\textbf{end}\texttt{;\\\nopagebreak[3]
}\textbf{end}\texttt{;\\\nopagebreak[3]
\\\nopagebreak[3]
}\textbf{procedure}\texttt{~InputBoxOnEnterField(aField:~pDmxFieldRec);\\\nopagebreak[3]
~~~~\textit{//~Ao~entrar~no~campo~01~e~o~mesmo~for~vazio~inicializa-o~com~o~nome~Paulo~Sérgio}\\\nopagebreak[3]
\\\nopagebreak[3]
}\textbf{begin}\texttt{\\\nopagebreak[3]
~~}\textbf{Case}\texttt{~aField.Fieldnum~}\textbf{of}\texttt{\\\nopagebreak[3]
~~~~1~:~}\textbf{begin}\texttt{\\\nopagebreak[3]
~~~~~~~~~~}\textbf{if}\texttt{~aField.AsString~=~''\\\nopagebreak[3]
~~~~~~~~~~}\textbf{then}\texttt{~aField.AsString~:=~'Paulo~Sérgio';\\\nopagebreak[3]
~~~~~~~~}\textbf{end}\texttt{;\\\nopagebreak[3]
~~~~2~:~}\textbf{begin}\texttt{~}\textbf{end}\texttt{;\\\nopagebreak[3]
~~}\textbf{end}\texttt{;\\\nopagebreak[3]
\\\nopagebreak[3]
}\textbf{end}\texttt{;\\\nopagebreak[3]
\\\nopagebreak[3]
\\\nopagebreak[3]
}\textbf{Procedure}\texttt{~InputBoxOnCloseQuery~(aDmxScroller:TUiDmxScroller;~}\textbf{var}\texttt{~CanClose:boolean);\\\nopagebreak[3]
~~\textit{//Esta~função~permite~validar~o~formulário~ao~pressionar~o~botão~ok.}\\\nopagebreak[3]
\\\nopagebreak[3]
~~}\textbf{var}\texttt{\\\nopagebreak[3]
~~~~idade~:~byte;\\\nopagebreak[3]
~~~~s~:~}\textbf{string}\texttt{;\\\nopagebreak[3]
}\textbf{begin}\texttt{\\\nopagebreak[3]
~~s~:=~aDmxScroller.FieldByName('idade').AsString;\\\nopagebreak[3]
~~idade~:=~StrToInt(s);\\\nopagebreak[3]
~~}\textbf{if}\texttt{~~idade~{$<$}{$>$}~64\\\nopagebreak[3]
~~}\textbf{then}\texttt{~}\textbf{begin}\texttt{\\\nopagebreak[3]
~~~~~~~~~ShowMessage('O~campo~idade~{$<$}{$>$}~de~64!.');\\\nopagebreak[3]
~~~~~~~~~CanClose~:=~false;\\\nopagebreak[3]
~~~~~~~}\textbf{end}\texttt{\\\nopagebreak[3]
~~}\textbf{else}\texttt{~CanClose~:=~true;\\\nopagebreak[3]
}\textbf{end}\texttt{;\\\nopagebreak[3]
\\\nopagebreak[3]
\\\nopagebreak[3]
}\textbf{Procedure}\texttt{~TestInputBox;\\\nopagebreak[3]
~~~~\textit{//Criar~o~formulário}\\\nopagebreak[3]
\\\nopagebreak[3]
~~}\textbf{Var}\texttt{\\\nopagebreak[3]
~~~~MI{\_}UI{\_}InputBox~:~TMI{\_}UI{\_}InputBox~=~}\textbf{nil}\texttt{;\\\nopagebreak[3]
}\textbf{begin}\texttt{\\\nopagebreak[3]
~~}\textbf{with}\texttt{~TDmxScroller{\_}Form{\_}Lcl~}\textbf{do}\texttt{\\\nopagebreak[3]
~~}\textbf{if}\texttt{~InputBox('Dados~do~aluno',\\\nopagebreak[3]
~~~~~~~~~~~~~~~~~'~Nome~do~Aluno:~{\textbackslash}Sssssssssssssssssssssssss`ssssssssssssssss'+ChFN+'Nome'+lf+\\\nopagebreak[3]
~~~~~~~~~~~~~~~~~'~~~~~~Endereço:~{\textbackslash}Sssssssssssssssssssssssss`ssssssssssssssss'+ChFN+'endereco'+lf+\\\nopagebreak[3]
~~~~~~~~~~~~~~~~~'~~~~~~~~~~~Cep:~{\textbackslash}{\#}{\#}-{\#}{\#}{\#}-{\#}{\#}{\#}'+ChFN+'cep'+lf+\\\nopagebreak[3]
~~~~~~~~~~~~~~~~~'~~~~~~~~Bairro:~{\textbackslash}sssssssssssssssssssssssss'+ChFN+'bairro'+lf+\\\nopagebreak[3]
~~~~~~~~~~~~~~~~~'~~~~~~~~Cidade:~{\textbackslash}sssssssssssssssssssssssss'+ChFN+'cidade'+lf+\\\nopagebreak[3]
~~~~~~~~~~~~~~~~~'~~~~~~~~Estado:~{\textbackslash}SS'+ChFN+'estado'+lf+\\\nopagebreak[3]
~~~~~~~~~~~~~~~~~'~~~~~~~~~Idade:~{\textbackslash}BB'+ChFN+'idade'+lf+\\\nopagebreak[3]
~~~~~~~~~~~~~~~~~'~~~~~Matricula:~{\textbackslash}III'+ChFN+'matricula'+lf+\\\nopagebreak[3]
~~~~~~~~~~~~~~~~~'~~~~~Mensalidade:~{\textbackslash}R,RRR.RR'+ChFN+'mensalidade',\\\nopagebreak[3]
~~~~~~~~~~~~~~~~~InputBoxOnEnter,}\textbf{nil}\texttt{,\\\nopagebreak[3]
\\\nopagebreak[3]
~~~~~~~~~~~~~~~~~InputBoxOnEnterField,}\textbf{nil}\texttt{,\\\nopagebreak[3]
~~~~~~~~~~~~~~~~~InputBoxOnCloseQuery,\\\nopagebreak[3]
~~~~~~~~~~~~~~~~~MI{\_}UI{\_}InputBox\\\nopagebreak[3]
~~~~~~~~~~~~)~=~MrOk\\\nopagebreak[3]
~~}\textbf{then}\texttt{~}\textbf{with}\texttt{~MI{\_}UI{\_}InputBox.DmxScroller{\_}Form{\_}Lcl1~}\textbf{do}\texttt{\\\nopagebreak[3]
~~~~~~~}\textbf{begin}\texttt{\\\nopagebreak[3]
~~~~~~~~~}\textbf{if}\texttt{~FieldByName('nome').AsString~=~''\\\nopagebreak[3]
~~~~~~~~~}\textbf{then}\texttt{~}\textbf{begin}\texttt{\\\nopagebreak[3]
~~~~~~~~~~~~~~~~ShowMessage('Campo~"nome"~não~pode~ser~vazio');\\\nopagebreak[3]
~~~~~~~~~~~~~~}\textbf{end}\texttt{;\\\nopagebreak[3]
~~~~~~~~~MI{\_}UI{\_}InputBox.free;\\\nopagebreak[3]
~~~~~~~}\textbf{end}\texttt{;\\\nopagebreak[3]
}\textbf{end}\texttt{;\\\nopagebreak[3]
\\\nopagebreak[3]
}\textbf{procedure}\texttt{~TDmxScroller{\_}Form{\_}Lcl{\_}test.Button{\_}InputBoxClick(Sender:~TObject);\\\nopagebreak[3]
}\textbf{begin}\texttt{\\\nopagebreak[3]
~~TestInputBox;\\\nopagebreak[3]
}\textbf{end}\texttt{;\\
}
\end{itemize}\subsubsection*{\large{\textbf{Propriedades}}\normalsize\hspace{1ex}\hfill}
\paragraph*{Template}\hspace*{\fill}

\begin{list}{}{
\settowidth{\tmplength}{\textbf{Declaração}}
\setlength{\itemindent}{0cm}
\setlength{\listparindent}{0cm}
\setlength{\leftmargin}{\evensidemargin}
\addtolength{\leftmargin}{\tmplength}
\settowidth{\labelsep}{X}
\addtolength{\leftmargin}{\labelsep}
\setlength{\labelwidth}{\tmplength}
}
\begin{flushleft}
\item[\textbf{Declaração}\hfill]
\begin{ttfamily}
public property Template: AnsiString read {\_}Template write Set{\_}Template;\end{ttfamily}


\end{flushleft}
\par
\item[\textbf{Descrição}]
A propriedade \textbf{\begin{ttfamily}Template\end{ttfamily}} é usada para criar uma lista de \begin{ttfamily}PSItem\end{ttfamily}(\ref{mi_rtl_ui_Dmxscroller-PSItem}) para ser usada como modelo do formulário.

\begin{itemize}
\item \textbf{NOTAS} \begin{itemize}
\item \begin{ttfamily}Template\end{ttfamily} é um string comum, onde cada linha é separada com {\^{}}J.
\item \begin{ttfamily}Template\end{ttfamily} tem uma lista de string com formato Dmx. \begin{itemize}
\item Formato da propriedade \begin{ttfamily}Template\end{ttfamily}:

\texttt{\\\nopagebreak[3]
\\\nopagebreak[3]
Template~:=~'~Nome~do~Aluno:~{\textbackslash}Sssssssssssssssssssssssss`ssssssssssssssss'+ChFN+'Nome'+lf+\\\nopagebreak[3]
~~~~~~~~~~~~'~~~~~~Endereço:~{\textbackslash}Sssssssssssssssssssssssss`ssssssssssssssss'+ChFN+'endereco'+lf+\\\nopagebreak[3]
~~~~~~~~~~~~'~~~~~~~~~~~Cep:~{\textbackslash}{\#}{\#}-{\#}{\#}{\#}-{\#}{\#}{\#}'+ChFN+'cep'+lf+\\\nopagebreak[3]
~~~~~~~~~~~~'~~~~~~~~Bairro:~{\textbackslash}sssssssssssssssssssssssss'+ChFN+'bairro'+lf+\\\nopagebreak[3]
~~~~~~~~~~~~'~~~~~~~~Cidade:~{\textbackslash}sssssssssssssssssssssssss'+ChFN+'cidade'+lf+\\\nopagebreak[3]
~~~~~~~~~~~~'~~~~~~~~Estado:~{\textbackslash}SS'+ChFN+'estado'+lf+\\\nopagebreak[3]
~~~~~~~~~~~~'~~~~~~~~~Idade:~{\textbackslash}BB'+ChFN+'idade'+FldUpperLimit+{\#}18+lf+\\\nopagebreak[3]
~~~~~~~~~~~~'~~~~~Matricula:~{\textbackslash}III'+ChFN+'matricula'+lf+\\\nopagebreak[3]
~~~~~~~~~~~~'~~~~~~Valor~da~'+lf+\\\nopagebreak[3]
~~~~~~~~~~~~'~~~~~Mensalidade:~{\textbackslash}R,RRR.RR'+ChFN+'mensalidade';\\
}
\end{itemize}
\item \textbf{SINTAXE} \begin{itemize}
\item \textbf{~} (til) : Limitador de rótulos do formulário;
\item \textbf{s} (s minúsculo) : caracteres alfanumérico incluindo os maiúsculas, minusculas, números e símbolos;
\item \textbf{S} (S maiúsculo) : caracteres alfanumérico incluindo os maiúsculas, números e símbolos;
\item \textbf{{\#}} ({\#} cancela) : Aceita somente números de 0 a 99
\item \textbf{{-}} (literal ) : Separador de números
\item \textbf{B} (B maiúsculo): Campo do tipo byte
\item \textbf{FldUpperLimit} : O caractere seguinte indica o limite superior da variável. No exemplo acima é 18 anos;
\item \textbf{R} : Indica um caractere de um campo do tipo double;
\item \textbf{I} : Indica um caractere de um campo do tipo interger. Faixa: {-}32000 a +32000;
\end{itemize}
\end{itemize}
\end{itemize}

\end{list}
\subsubsection*{\large{\textbf{Campos}}\normalsize\hspace{1ex}\hfill}
\paragraph*{ButtonPanel1}\hspace*{\fill}

\begin{list}{}{
\settowidth{\tmplength}{\textbf{Declaração}}
\setlength{\itemindent}{0cm}
\setlength{\listparindent}{0cm}
\setlength{\leftmargin}{\evensidemargin}
\addtolength{\leftmargin}{\tmplength}
\settowidth{\labelsep}{X}
\addtolength{\leftmargin}{\labelsep}
\setlength{\labelwidth}{\tmplength}
}
\begin{flushleft}
\item[\textbf{Declaração}\hfill]
\begin{ttfamily}
public ButtonPanel1: TButtonPanel;\end{ttfamily}


\end{flushleft}
\end{list}
\paragraph*{DmxScroller{\_}Form{\_}Lcl1}\hspace*{\fill}

\begin{list}{}{
\settowidth{\tmplength}{\textbf{Declaração}}
\setlength{\itemindent}{0cm}
\setlength{\listparindent}{0cm}
\setlength{\leftmargin}{\evensidemargin}
\addtolength{\leftmargin}{\tmplength}
\settowidth{\labelsep}{X}
\addtolength{\leftmargin}{\labelsep}
\setlength{\labelwidth}{\tmplength}
}
\begin{flushleft}
\item[\textbf{Declaração}\hfill]
\begin{ttfamily}
public DmxScroller{\_}Form{\_}Lcl1: TDmxScroller{\_}Form{\_}Lcl;\end{ttfamily}


\end{flushleft}
\end{list}
\paragraph*{Mi{\_}ScrollBox{\_}LCL1}\hspace*{\fill}

\begin{list}{}{
\settowidth{\tmplength}{\textbf{Declaração}}
\setlength{\itemindent}{0cm}
\setlength{\listparindent}{0cm}
\setlength{\leftmargin}{\evensidemargin}
\addtolength{\leftmargin}{\tmplength}
\settowidth{\labelsep}{X}
\addtolength{\leftmargin}{\labelsep}
\setlength{\labelwidth}{\tmplength}
}
\begin{flushleft}
\item[\textbf{Declaração}\hfill]
\begin{ttfamily}
public Mi{\_}ScrollBox{\_}LCL1: TMi{\_}ScrollBox{\_}LCL;\end{ttfamily}


\end{flushleft}
\end{list}
\subsubsection*{\large{\textbf{Métodos}}\normalsize\hspace{1ex}\hfill}
\paragraph*{CancelButtonClick}\hspace*{\fill}

\begin{list}{}{
\settowidth{\tmplength}{\textbf{Declaração}}
\setlength{\itemindent}{0cm}
\setlength{\listparindent}{0cm}
\setlength{\leftmargin}{\evensidemargin}
\addtolength{\leftmargin}{\tmplength}
\settowidth{\labelsep}{X}
\addtolength{\leftmargin}{\labelsep}
\setlength{\labelwidth}{\tmplength}
}
\begin{flushleft}
\item[\textbf{Declaração}\hfill]
\begin{ttfamily}
public procedure CancelButtonClick(Sender: TObject);\end{ttfamily}


\end{flushleft}
\end{list}
\paragraph*{DmxScroller{\_}Form{\_}Lcl1AddTemplate}\hspace*{\fill}

\begin{list}{}{
\settowidth{\tmplength}{\textbf{Declaração}}
\setlength{\itemindent}{0cm}
\setlength{\listparindent}{0cm}
\setlength{\leftmargin}{\evensidemargin}
\addtolength{\leftmargin}{\tmplength}
\settowidth{\labelsep}{X}
\addtolength{\leftmargin}{\labelsep}
\setlength{\labelwidth}{\tmplength}
}
\begin{flushleft}
\item[\textbf{Declaração}\hfill]
\begin{ttfamily}
public procedure DmxScroller{\_}Form{\_}Lcl1AddTemplate(const aUiDmxScroller: TUiDmxScroller );\end{ttfamily}


\end{flushleft}
\end{list}
\paragraph*{DmxScroller{\_}Form{\_}Lcl1CloseQuery}\hspace*{\fill}

\begin{list}{}{
\settowidth{\tmplength}{\textbf{Declaração}}
\setlength{\itemindent}{0cm}
\setlength{\listparindent}{0cm}
\setlength{\leftmargin}{\evensidemargin}
\addtolength{\leftmargin}{\tmplength}
\settowidth{\labelsep}{X}
\addtolength{\leftmargin}{\labelsep}
\setlength{\labelwidth}{\tmplength}
}
\begin{flushleft}
\item[\textbf{Declaração}\hfill]
\begin{ttfamily}
public procedure DmxScroller{\_}Form{\_}Lcl1CloseQuery(aDmxScroller: TUiDmxScroller;var CanClose: boolean);\end{ttfamily}


\end{flushleft}
\end{list}
\paragraph*{DmxScroller{\_}Form{\_}Lcl1Enter}\hspace*{\fill}

\begin{list}{}{
\settowidth{\tmplength}{\textbf{Declaração}}
\setlength{\itemindent}{0cm}
\setlength{\listparindent}{0cm}
\setlength{\leftmargin}{\evensidemargin}
\addtolength{\leftmargin}{\tmplength}
\settowidth{\labelsep}{X}
\addtolength{\leftmargin}{\labelsep}
\setlength{\labelwidth}{\tmplength}
}
\begin{flushleft}
\item[\textbf{Declaração}\hfill]
\begin{ttfamily}
public procedure DmxScroller{\_}Form{\_}Lcl1Enter(aDmxScroller: TUiDmxScroller);\end{ttfamily}


\end{flushleft}
\end{list}
\paragraph*{DmxScroller{\_}Form{\_}Lcl1EnterField}\hspace*{\fill}

\begin{list}{}{
\settowidth{\tmplength}{\textbf{Declaração}}
\setlength{\itemindent}{0cm}
\setlength{\listparindent}{0cm}
\setlength{\leftmargin}{\evensidemargin}
\addtolength{\leftmargin}{\tmplength}
\settowidth{\labelsep}{X}
\addtolength{\leftmargin}{\labelsep}
\setlength{\labelwidth}{\tmplength}
}
\begin{flushleft}
\item[\textbf{Declaração}\hfill]
\begin{ttfamily}
public procedure DmxScroller{\_}Form{\_}Lcl1EnterField(aField: pDmxFieldRec);\end{ttfamily}


\end{flushleft}
\end{list}
\paragraph*{DmxScroller{\_}Form{\_}Lcl1Exit}\hspace*{\fill}

\begin{list}{}{
\settowidth{\tmplength}{\textbf{Declaração}}
\setlength{\itemindent}{0cm}
\setlength{\listparindent}{0cm}
\setlength{\leftmargin}{\evensidemargin}
\addtolength{\leftmargin}{\tmplength}
\settowidth{\labelsep}{X}
\addtolength{\leftmargin}{\labelsep}
\setlength{\labelwidth}{\tmplength}
}
\begin{flushleft}
\item[\textbf{Declaração}\hfill]
\begin{ttfamily}
public procedure DmxScroller{\_}Form{\_}Lcl1Exit(aDmxScroller: TUiDmxScroller);\end{ttfamily}


\end{flushleft}
\end{list}
\paragraph*{DmxScroller{\_}Form{\_}Lcl1ExitField}\hspace*{\fill}

\begin{list}{}{
\settowidth{\tmplength}{\textbf{Declaração}}
\setlength{\itemindent}{0cm}
\setlength{\listparindent}{0cm}
\setlength{\leftmargin}{\evensidemargin}
\addtolength{\leftmargin}{\tmplength}
\settowidth{\labelsep}{X}
\addtolength{\leftmargin}{\labelsep}
\setlength{\labelwidth}{\tmplength}
}
\begin{flushleft}
\item[\textbf{Declaração}\hfill]
\begin{ttfamily}
public procedure DmxScroller{\_}Form{\_}Lcl1ExitField(aField: pDmxFieldRec);\end{ttfamily}


\end{flushleft}
\end{list}
\paragraph*{DmxScroller{\_}Form{\_}Lcl1GetTemplate}\hspace*{\fill}

\begin{list}{}{
\settowidth{\tmplength}{\textbf{Declaração}}
\setlength{\itemindent}{0cm}
\setlength{\listparindent}{0cm}
\setlength{\leftmargin}{\evensidemargin}
\addtolength{\leftmargin}{\tmplength}
\settowidth{\labelsep}{X}
\addtolength{\leftmargin}{\labelsep}
\setlength{\labelwidth}{\tmplength}
}
\begin{flushleft}
\item[\textbf{Declaração}\hfill]
\begin{ttfamily}
public function DmxScroller{\_}Form{\_}Lcl1GetTemplate(aNext: PSItem): PSItem;\end{ttfamily}


\end{flushleft}
\end{list}
\paragraph*{DmxScroller{\_}Form{\_}Lcl1NewRecord}\hspace*{\fill}

\begin{list}{}{
\settowidth{\tmplength}{\textbf{Declaração}}
\setlength{\itemindent}{0cm}
\setlength{\listparindent}{0cm}
\setlength{\leftmargin}{\evensidemargin}
\addtolength{\leftmargin}{\tmplength}
\settowidth{\labelsep}{X}
\addtolength{\leftmargin}{\labelsep}
\setlength{\labelwidth}{\tmplength}
}
\begin{flushleft}
\item[\textbf{Declaração}\hfill]
\begin{ttfamily}
public procedure DmxScroller{\_}Form{\_}Lcl1NewRecord(aDmxScroller: TUiDmxScroller);\end{ttfamily}


\end{flushleft}
\end{list}
\paragraph*{FormCloseQuery}\hspace*{\fill}

\begin{list}{}{
\settowidth{\tmplength}{\textbf{Declaração}}
\setlength{\itemindent}{0cm}
\setlength{\listparindent}{0cm}
\setlength{\leftmargin}{\evensidemargin}
\addtolength{\leftmargin}{\tmplength}
\settowidth{\labelsep}{X}
\addtolength{\leftmargin}{\labelsep}
\setlength{\labelwidth}{\tmplength}
}
\begin{flushleft}
\item[\textbf{Declaração}\hfill]
\begin{ttfamily}
public procedure FormCloseQuery(Sender: TObject; var CanClose: Boolean);\end{ttfamily}


\end{flushleft}
\end{list}
\paragraph*{FormCreate}\hspace*{\fill}

\begin{list}{}{
\settowidth{\tmplength}{\textbf{Declaração}}
\setlength{\itemindent}{0cm}
\setlength{\listparindent}{0cm}
\setlength{\leftmargin}{\evensidemargin}
\addtolength{\leftmargin}{\tmplength}
\settowidth{\labelsep}{X}
\addtolength{\leftmargin}{\labelsep}
\setlength{\labelwidth}{\tmplength}
}
\begin{flushleft}
\item[\textbf{Declaração}\hfill]
\begin{ttfamily}
public procedure FormCreate(Sender: TObject);\end{ttfamily}


\end{flushleft}
\end{list}
\paragraph*{OKButtonClick}\hspace*{\fill}

\begin{list}{}{
\settowidth{\tmplength}{\textbf{Declaração}}
\setlength{\itemindent}{0cm}
\setlength{\listparindent}{0cm}
\setlength{\leftmargin}{\evensidemargin}
\addtolength{\leftmargin}{\tmplength}
\settowidth{\labelsep}{X}
\addtolength{\leftmargin}{\labelsep}
\setlength{\labelwidth}{\tmplength}
}
\begin{flushleft}
\item[\textbf{Declaração}\hfill]
\begin{ttfamily}
public procedure OKButtonClick(Sender: TObject);\end{ttfamily}


\end{flushleft}
\end{list}
\paragraph*{Set{\_}Template}\hspace*{\fill}

\begin{list}{}{
\settowidth{\tmplength}{\textbf{Declaração}}
\setlength{\itemindent}{0cm}
\setlength{\listparindent}{0cm}
\setlength{\leftmargin}{\evensidemargin}
\addtolength{\leftmargin}{\tmplength}
\settowidth{\labelsep}{X}
\addtolength{\leftmargin}{\labelsep}
\setlength{\labelwidth}{\tmplength}
}
\begin{flushleft}
\item[\textbf{Declaração}\hfill]
\begin{ttfamily}
protected Procedure Set{\_}Template(aTemplate:AnsiString);\end{ttfamily}


\end{flushleft}
\par
\item[\textbf{Descrição}]
O Método \textbf{\begin{ttfamily}Set{\_}Template\end{ttfamily}} inicia o atributo {\_}Template e criar a lista {\_}FormSItem : \begin{ttfamily}PSitem\end{ttfamily}(\ref{mi_rtl_ui_Dmxscroller-PSItem})

\begin{itemize}
\item \textbf{NOTAS} \begin{itemize}
\item Criar em TObjectss.TStringList o método
\end{itemize}
\end{itemize}

\end{list}
\paragraph*{SetAlias}\hspace*{\fill}

\begin{list}{}{
\settowidth{\tmplength}{\textbf{Declaração}}
\setlength{\itemindent}{0cm}
\setlength{\listparindent}{0cm}
\setlength{\leftmargin}{\evensidemargin}
\addtolength{\leftmargin}{\tmplength}
\settowidth{\labelsep}{X}
\addtolength{\leftmargin}{\labelsep}
\setlength{\labelwidth}{\tmplength}
}
\begin{flushleft}
\item[\textbf{Declaração}\hfill]
\begin{ttfamily}
public Procedure SetAlias(aTitle:AnsiString);\end{ttfamily}


\end{flushleft}
\end{list}
\section{Funções e Procedimentos}
\subsection*{InputBox}
\begin{list}{}{
\settowidth{\tmplength}{\textbf{Declaração}}
\setlength{\itemindent}{0cm}
\setlength{\listparindent}{0cm}
\setlength{\leftmargin}{\evensidemargin}
\addtolength{\leftmargin}{\tmplength}
\settowidth{\labelsep}{X}
\addtolength{\leftmargin}{\labelsep}
\setlength{\labelwidth}{\tmplength}
}
\begin{flushleft}
\item[\textbf{Declaração}\hfill]
\begin{ttfamily}
function InputBox( aTitle: AnsiString; aTemplate: AnsiString; aOnEnterLocal:TOnEnterLocal ; aOnExitLocal:TOnExitLocal; aOnEnterFieldLocal:TOnEnterFieldLocal; aOnExitFieldLocal:TOnExitFieldLocal; aOnCloseQueryLocal:TOnCloseQueryLocal; out aMi{\_}ui{\_}InputBox : TMI{\_}UI{\_}InputBox ): TModalResult;\end{ttfamily}


\end{flushleft}
\par
\item[\textbf{Descrição}]
A função \textbf{\begin{ttfamily}InputBox\end{ttfamily}} cria um formulário passado por Template e executa os eventos do formulário passado pelos parâmetros.

\begin{itemize}
\item \textbf{PARÂMETROS} \begin{itemize}
\item atitle; // Título do formulário;
\item aTemplate; // Modelo do formulário cuja a sintaxe segue abaixo:
\item aOnEnter; // Executado ao entrar no formulário criado baseado no Template;
\item aOnExit; // Executado ao sair do formulário criado baseado no Template;
\item aOnEnterField; // Executado ao entrar um campo focado;
\item aOnExitField; // Executado ao sair do campo focado;
\item aOnCloseQuery // Executado ao fechar o form se CanClose = true;
\end{itemize}
\item \textbf{SINTAXE DO MODELO} \begin{itemize}
\item Exemplo \texttt{\\\nopagebreak[3]
\\\nopagebreak[3]
}\textbf{const}\texttt{\\\nopagebreak[3]
~~Template~:=~'~Nome~do~Aluno:~{\textbackslash}sssssssssssssssssssssssss'+lf+\\\nopagebreak[3]
~~~~~~~~~~~~~~'~~~~~~Endereço:~{\textbackslash}sssssssssssssssssssssssss`ssssssssssssssss'+lf+\\\nopagebreak[3]
~~~~~~~~~~~~~~'~~~~~~~~~~~Cep:~{\textbackslash}{\#}{\#}-{\#}{\#}{\#}-{\#}{\#}{\#}'+lf+\\\nopagebreak[3]
~~~~~~~~~~~~~~'~~~~~~~~Bairro:~{\textbackslash}sssssssssssssssssssssssss'+lf+\\\nopagebreak[3]
~~~~~~~~~~~~~~'~~~~~~~~Cidade:~{\textbackslash}sssssssssssssssssssssssss'+lf+\\\nopagebreak[3]
~~~~~~~~~~~~~~'~~~~~~~~Estado:~{\textbackslash}sssssssssssssssssssssssss'+lf+\\\nopagebreak[3]
~~~~~~~~~~~~~~'~~~~~~~~~Idade:~{\textbackslash}BB'+lf+\\\nopagebreak[3]
~~~~~~~~~~~~~~'~~~~~Mensalidade:~{\textbackslash}R,RRR.RR';\\
}
\item Tipos de dados do formulário: \begin{itemize}
\item s = Char alfanumérico
\item {\#} = Char numérico
\item R = Double
\item B = Byte
\end{itemize}
\end{itemize}
\item Exemplo de chamada da função:

\texttt{\\\nopagebreak[3]
\\\nopagebreak[3]
}\textbf{if}\texttt{~InputBox('Dados~do~aluno',\\\nopagebreak[3]
~~~~~~~~~~~~~'~Nome~do~Aluno:~{\textbackslash}sssssssssssssssssssssssss'+lf+\\\nopagebreak[3]
~~~~~~~~~~~~~'~~~~~~Endereço:~{\textbackslash}sssssssssssssssssssssssss`ssssssssssssssss'+lf+\\\nopagebreak[3]
~~~~~~~~~~~~~'~~~~~~~~~~~Cep:~{\textbackslash}{\#}{\#}-{\#}{\#}{\#}-{\#}{\#}{\#}'+lf+\\\nopagebreak[3]
~~~~~~~~~~~~~'~~~~~~~~Bairro:~{\textbackslash}sssssssssssssssssssssssss'+lf+\\\nopagebreak[3]
~~~~~~~~~~~~~'~~~~~~~~Cidade:~{\textbackslash}sssssssssssssssssssssssss'+lf+\\\nopagebreak[3]
~~~~~~~~~~~~~'~~~~~~~~Estado:~{\textbackslash}sssssssssssssssssssssssss'+lf+\\\nopagebreak[3]
~~~~~~~~~~~~~'~~~~~~~~~Idade:~{\textbackslash}BB'+lf+\\\nopagebreak[3]
~~~~~~~~~~~~~'~~~~~Mensalidade:~{\textbackslash}R,RRR.RR',\\\nopagebreak[3]
~~~~~~~~~~~~~}\textbf{nil}\texttt{,}\textbf{nil}\texttt{,}\textbf{nil}\texttt{,}\textbf{nil}\texttt{,}\textbf{nil}\texttt{\\\nopagebreak[3]
~~~~~~~~)~=~MrOk\\\nopagebreak[3]
}\textbf{then}\texttt{~}\textbf{begin}\texttt{\\\nopagebreak[3]
~~~~~}\textbf{end}\texttt{;\\
}
\end{itemize}

\end{list}
\section{Tipos}
\subsection*{TOnEnterLocal}
\begin{list}{}{
\settowidth{\tmplength}{\textbf{Declaração}}
\setlength{\itemindent}{0cm}
\setlength{\listparindent}{0cm}
\setlength{\leftmargin}{\evensidemargin}
\addtolength{\leftmargin}{\tmplength}
\settowidth{\labelsep}{X}
\addtolength{\leftmargin}{\labelsep}
\setlength{\labelwidth}{\tmplength}
}
\begin{flushleft}
\item[\textbf{Declaração}\hfill]
\begin{ttfamily}
TOnEnterLocal = Procedure(aDmxScroller:TUiDmxScroller);\end{ttfamily}


\end{flushleft}
\par
\item[\textbf{Descrição}]
O tipo \textbf{\begin{ttfamily}TOnEnterLocal\end{ttfamily}} é usado para implementar evento onEnterLocal do atributo \textbf{Mi{\_}ScrollBox{\_}LCL1}

\end{list}
\subsection*{TOnExitLocal}
\begin{list}{}{
\settowidth{\tmplength}{\textbf{Declaração}}
\setlength{\itemindent}{0cm}
\setlength{\listparindent}{0cm}
\setlength{\leftmargin}{\evensidemargin}
\addtolength{\leftmargin}{\tmplength}
\settowidth{\labelsep}{X}
\addtolength{\leftmargin}{\labelsep}
\setlength{\labelwidth}{\tmplength}
}
\begin{flushleft}
\item[\textbf{Declaração}\hfill]
\begin{ttfamily}
TOnExitLocal = Procedure(aDmxScroller:TUiDmxScroller);\end{ttfamily}


\end{flushleft}
\par
\item[\textbf{Descrição}]
O tipo \textbf{\begin{ttfamily}TOnExitLocal\end{ttfamily}} é usado para implementar evento onExitLocal do atributo \textbf{Mi{\_}ScrollBox{\_}LCL1}

\end{list}
\subsection*{TOnEnterFieldLocal}
\begin{list}{}{
\settowidth{\tmplength}{\textbf{Declaração}}
\setlength{\itemindent}{0cm}
\setlength{\listparindent}{0cm}
\setlength{\leftmargin}{\evensidemargin}
\addtolength{\leftmargin}{\tmplength}
\settowidth{\labelsep}{X}
\addtolength{\leftmargin}{\labelsep}
\setlength{\labelwidth}{\tmplength}
}
\begin{flushleft}
\item[\textbf{Declaração}\hfill]
\begin{ttfamily}
TOnEnterFieldLocal = Procedure(aField:pDmxFieldRec);\end{ttfamily}


\end{flushleft}
\par
\item[\textbf{Descrição}]
O tipo \textbf{\begin{ttfamily}TOnEnterFieldLocal\end{ttfamily}} é usado para implementar evento OnEnterFieldLocal do atributo \textbf{Mi{\_}ScrollBox{\_}LCL1}

\end{list}
\subsection*{TOnExitFieldLocal}
\begin{list}{}{
\settowidth{\tmplength}{\textbf{Declaração}}
\setlength{\itemindent}{0cm}
\setlength{\listparindent}{0cm}
\setlength{\leftmargin}{\evensidemargin}
\addtolength{\leftmargin}{\tmplength}
\settowidth{\labelsep}{X}
\addtolength{\leftmargin}{\labelsep}
\setlength{\labelwidth}{\tmplength}
}
\begin{flushleft}
\item[\textbf{Declaração}\hfill]
\begin{ttfamily}
TOnExitFieldLocal = Procedure(aField:pDmxFieldRec);\end{ttfamily}


\end{flushleft}
\par
\item[\textbf{Descrição}]
O tipo \textbf{\begin{ttfamily}TOnExitFieldLocal\end{ttfamily}} é usado para implementar evento OnExitFieldLocal do atributo \textbf{Mi{\_}ScrollBox{\_}LCL1}

\end{list}
\subsection*{TOnCloseQueryLocal}
\begin{list}{}{
\settowidth{\tmplength}{\textbf{Declaração}}
\setlength{\itemindent}{0cm}
\setlength{\listparindent}{0cm}
\setlength{\leftmargin}{\evensidemargin}
\addtolength{\leftmargin}{\tmplength}
\settowidth{\labelsep}{X}
\addtolength{\leftmargin}{\labelsep}
\setlength{\labelwidth}{\tmplength}
}
\begin{flushleft}
\item[\textbf{Declaração}\hfill]
\begin{ttfamily}
TOnCloseQueryLocal = Procedure(aDmxScroller:TUiDmxScroller; var CanClose:boolean);\end{ttfamily}


\end{flushleft}
\par
\item[\textbf{Descrição}]
O tipo \textbf{\begin{ttfamily}TOnCloseQueryLocal\end{ttfamily}} é usado para implementar evento OnCloseQueryLocal do atributo \textbf{Mi{\_}ScrollBox{\_}LCL1}.

\begin{itemize}
\item \textbf{NOTA* \begin{itemize}
\item Este evento é disparado antes de desativar a classe **\begin{ttfamily}TUiDmxScroller\end{ttfamily}(\ref{mi_rtl_ui_Dmxscroller.TUiDmxScroller})
\end{itemize}}. \begin{itemize}
\item Obs: Se o parâmetro \textbf{CanClose} for \textbf{false}, então a classe \textbf{\begin{ttfamily}TUiDmxScroller\end{ttfamily}(\ref{mi_rtl_ui_Dmxscroller.TUiDmxScroller})} não é desativado.
\end{itemize}
\end{itemize}

\end{list}
\chapter{Unit umi{\_}ui{\_}InputBox{\_}lcl}
\section{Descrição}
A unit \textbf{\begin{ttfamily}umi{\_}ui{\_}InputBox{\_}lcl\end{ttfamily}} implementa o formulário \begin{ttfamily}TMI{\_}UI{\_}InputBox\end{ttfamily}(\ref{umi_ui_InputBox_lcl.TMI_UI_InputBox}) usado para criar formulário baseado em Template \begin{ttfamily}PSITem\end{ttfamily}(\ref{mi_rtl_ui_Dmxscroller-PSItem}).

\begin{itemize}
\item \textbf{VERSÃO} \begin{itemize}
\item Alpha {-} 0.5.0.687
\end{itemize}
\item \textbf{CÓDIGO FONTE}: \begin{itemize}
\item 
\end{itemize}
\item \textbf{PENDÊNCIAS} \begin{itemize}
\item T12 A propriedade autosize deve ser true após o form for criado.
\item T12 Dar
\end{itemize}
\end{itemize}\begin{itemize}
\item \textbf{HISTÓRICO} \begin{itemize}
\item Criado por: Paulo Sérgio da Silva Pacheco paulosspacheco@yahoo.com.br)
\item \textbf{2022{-}05{-}17} \begin{itemize}
\item T12 Análise de como será a classe \begin{ttfamily}TMI{\_}UI{\_}InputBox\end{ttfamily}(\ref{umi_ui_InputBox_lcl.TMI_UI_InputBox}). ✔
\item T12 Criar a unit \textbf{\begin{ttfamily}umi{\_}ui{\_}InputBox{\_}lcl\end{ttfamily}}. ✔
\item T12 Criar formulário \begin{ttfamily}TMI{\_}UI{\_}InputBox\end{ttfamily}(\ref{umi_ui_InputBox_lcl.TMI_UI_InputBox}); ✔
\item T12 Adicionar o componente ButtonPanel1 e habilitar os botões ok e cancel; ✔
\item T12 Adicionar o componente Mi{\_}ScrollBox{\_}LCL1 ; ✔
\item T12 Criar evento: function DmxScroller{\_}Form1GetTemplate; ✔
\item T12 Criar atributo protected {\_}FormSItem : \begin{ttfamily}PSitem\end{ttfamily}(\ref{mi_rtl_ui_Dmxscroller-PSItem}); ✔
\item T12 Criar propriedade Template:AnsiString; ✔ \begin{itemize}
\item Criar método Set{\_}Template(aTemplate:AnsiString); ✔
\end{itemize}
\end{itemize}
\item \textbf{2022{-}05{-}18} \begin{itemize}
\item \textbf{10:51} \begin{itemize}
\item As alterações que fiz ontem no método \begin{ttfamily}TObjectsMethods.StringToSItem\end{ttfamily}(\ref{mi.rtl.Objects.Methods.TObjectsMethods-StringToSItem})() criou efeito colateral. \begin{itemize}
\item Corrigido. ✔
\end{itemize}
\end{itemize}
\item \textbf{14:28} \begin{itemize}
\item Criar função: \begin{itemize}
\item function \begin{ttfamily}InputBox\end{ttfamily}(\ref{umi_ui_InputBox_lcl-InputBox})(): TModalResult;
\end{itemize}
\end{itemize}
\end{itemize}
\item \textbf{2022{-}05{-}19} \begin{itemize}
\item \textbf{11:13} \begin{itemize}
\item Criar os eventos \begin{itemize}
\item OnEnterLocal ✔
\item OnExistLocal ✔
\item onEnterFieldLocal ✔
\item OnExitFieldLocal ✔
\end{itemize}
\item Criar função: \begin{itemize}
\item function MI{\_}MsgBox1MessageBox{\_}04{\_}PSItem(aPSItem: TMI{\_}MsgBoxTypes.PSItem; DlgType: TMsgDlgType; Buttons: TMsgDlgButtons; ButtonDefault: TMsgDlgBtn ): TModalResult;
\end{itemize}
\end{itemize}
\end{itemize}
\end{itemize}
\end{itemize}
\section{Uses}
\begin{itemize}
\item \begin{ttfamily}Classes\end{ttfamily}\item \begin{ttfamily}SysUtils\end{ttfamily}\item \begin{ttfamily}Forms\end{ttfamily}\item \begin{ttfamily}Controls\end{ttfamily}\item \begin{ttfamily}Graphics\end{ttfamily}\item \begin{ttfamily}Dialogs\end{ttfamily}\item \begin{ttfamily}StdCtrls\end{ttfamily}\item \begin{ttfamily}ExtCtrls\end{ttfamily}\item \begin{ttfamily}ButtonPanel\end{ttfamily}\item \begin{ttfamily}uMi{\_}ui{\_}Dmxscroller{\_}form\end{ttfamily}\item \begin{ttfamily}mi{\_}rtl{\_}ui{\_}Dmxscroller\end{ttfamily}(\ref{mi_rtl_ui_Dmxscroller})\item \begin{ttfamily}uMi{\_}ui{\_}scrollbox{\_}lcl\end{ttfamily}(\ref{uMi_ui_scrollbox_lcl})\item \begin{ttfamily}uMi{\_}ui{\_}Dmxscroller{\_}form{\_}ds\end{ttfamily}\end{itemize}
\section{Visão Geral}
\begin{description}
\item[\texttt{\begin{ttfamily}TMI{\_}UI{\_}InputBox\end{ttfamily} Classe}]
\end{description}
\begin{description}
\item[\texttt{InputBox}]
\end{description}
\section{Classes, Interfaces, Objetos e Registros}
\subsection*{TMI{\_}UI{\_}InputBox Classe}
\subsubsection*{\large{\textbf{Hierarquia}}\normalsize\hspace{1ex}\hfill}
TMI{\_}UI{\_}InputBox {$>$} TForm
\subsubsection*{\large{\textbf{Descrição}}\normalsize\hspace{1ex}\hfill}
A class \textbf{\begin{ttfamily}TMI{\_}UI{\_}InputBox\end{ttfamily}} edita uma formulário passado em forma de \begin{ttfamily}Template\end{ttfamily}(\ref{umi_ui_InputBox_lcl.TMI_UI_InputBox-Template}) e devolve o formulário LCL caso o botão \textbf{OK} seja pressionando ou nil caso o botão \textbf{Cancelar} seja pressionando.

\begin{itemize}
\item \textbf{EXEMPLO} \begin{itemize}
\item O exemplo abaixo cria um formulário e permite interagir com com os eventos ao entrar e ao sair do formulário.
\end{itemize} \texttt{\\\nopagebreak[3]
\\\nopagebreak[3]
}\textbf{procedure}\texttt{~InputBoxOnEnter(aDmxScroller:~TUiDmxScroller);\\\nopagebreak[3]
~\textit{//Ao~entrar~no~formulário~este~evento~inicia~os~campos~nome,~endereço~e~cep}\\\nopagebreak[3]
\\\nopagebreak[3]
~~}\textbf{procedure}\texttt{~SetValue(Field:}\textbf{String}\texttt{;Value:}\textbf{String}\texttt{);\\\nopagebreak[3]
~~~~}\textbf{var}\texttt{\\\nopagebreak[3]
~~~~~~aField:~pDmxFieldRec;\\\nopagebreak[3]
~~}\textbf{begin}\texttt{\\\nopagebreak[3]
~~~~}\textbf{with}\texttt{~aDmxScroller~}\textbf{do}\texttt{\\\nopagebreak[3]
~~~~}\textbf{begin}\texttt{\\\nopagebreak[3]
~~~~~~aField~:=~FieldByName(Field);\\\nopagebreak[3]
~~~~~~}\textbf{if}\texttt{~aField{$<$}{$>$}}\textbf{nil}\texttt{\\\nopagebreak[3]
~~~~~~}\textbf{Then}\texttt{~aField.AsString:=~value;\\\nopagebreak[3]
~~~~}\textbf{end}\texttt{;\\\nopagebreak[3]
~~}\textbf{end}\texttt{;\\\nopagebreak[3]
\\\nopagebreak[3]
}\textbf{begin}\texttt{\\\nopagebreak[3]
~~}\textbf{with}\texttt{~aDmxScroller~}\textbf{do}\texttt{\\\nopagebreak[3]
~~}\textbf{begin}\texttt{\\\nopagebreak[3]
~~~~setValue('nome','Paulo~Henrique');\\\nopagebreak[3]
~~~~setValue('endereço','Rua~Francisco~de~Souza~Oliveira,~15');\\\nopagebreak[3]
~~~~setValue('cep','60310770');\\\nopagebreak[3]
~~}\textbf{end}\texttt{;\\\nopagebreak[3]
}\textbf{end}\texttt{;\\\nopagebreak[3]
\\\nopagebreak[3]
}\textbf{procedure}\texttt{~InputBoxOnEnterField(aField:~pDmxFieldRec);\\\nopagebreak[3]
~~~~\textit{//~Ao~entrar~no~campo~01~e~o~mesmo~for~vazio~inicializa-o~com~o~nome~Paulo~Sérgio}\\\nopagebreak[3]
\\\nopagebreak[3]
}\textbf{begin}\texttt{\\\nopagebreak[3]
~~}\textbf{Case}\texttt{~aField.Fieldnum~}\textbf{of}\texttt{\\\nopagebreak[3]
~~~~1~:~}\textbf{begin}\texttt{\\\nopagebreak[3]
~~~~~~~~~~}\textbf{if}\texttt{~aField.AsString~=~''\\\nopagebreak[3]
~~~~~~~~~~}\textbf{then}\texttt{~aField.AsString~:=~'Paulo~Sérgio';\\\nopagebreak[3]
~~~~~~~~}\textbf{end}\texttt{;\\\nopagebreak[3]
~~~~2~:~}\textbf{begin}\texttt{~}\textbf{end}\texttt{;\\\nopagebreak[3]
~~}\textbf{end}\texttt{;\\\nopagebreak[3]
\\\nopagebreak[3]
}\textbf{end}\texttt{;\\\nopagebreak[3]
\\\nopagebreak[3]
\\\nopagebreak[3]
}\textbf{Procedure}\texttt{~InputBoxOnCloseQuery~(aDmxScroller:TUiDmxScroller;~}\textbf{var}\texttt{~CanClose:boolean);\\\nopagebreak[3]
~~\textit{//Esta~função~permite~validar~o~formulário~ao~pressionar~o~botão~ok.}\\\nopagebreak[3]
\\\nopagebreak[3]
~~}\textbf{var}\texttt{\\\nopagebreak[3]
~~~~idade~:~byte;\\\nopagebreak[3]
~~~~s~:~}\textbf{string}\texttt{;\\\nopagebreak[3]
}\textbf{begin}\texttt{\\\nopagebreak[3]
~~s~:=~aDmxScroller.FieldByName('idade').AsString;\\\nopagebreak[3]
~~idade~:=~StrToInt(s);\\\nopagebreak[3]
~~}\textbf{if}\texttt{~~idade~{$<$}{$>$}~64\\\nopagebreak[3]
~~}\textbf{then}\texttt{~}\textbf{begin}\texttt{\\\nopagebreak[3]
~~~~~~~~~ShowMessage('O~campo~idade~{$<$}{$>$}~de~64!.');\\\nopagebreak[3]
~~~~~~~~~CanClose~:=~false;\\\nopagebreak[3]
~~~~~~~}\textbf{end}\texttt{\\\nopagebreak[3]
~~}\textbf{else}\texttt{~CanClose~:=~true;\\\nopagebreak[3]
}\textbf{end}\texttt{;\\\nopagebreak[3]
\\\nopagebreak[3]
\\\nopagebreak[3]
}\textbf{Procedure}\texttt{~TestInputBox;\\\nopagebreak[3]
~~~~\textit{//Criar~o~formulário}\\\nopagebreak[3]
\\\nopagebreak[3]
~~}\textbf{Var}\texttt{\\\nopagebreak[3]
~~~~MI{\_}UI{\_}InputBox~:~TMI{\_}UI{\_}InputBox~=~}\textbf{nil}\texttt{;\\\nopagebreak[3]
}\textbf{begin}\texttt{\\\nopagebreak[3]
~~}\textbf{with}\texttt{~TDmxScroller{\_}Form~}\textbf{do}\texttt{\\\nopagebreak[3]
~~}\textbf{if}\texttt{~InputBox('Dados~do~aluno',\\\nopagebreak[3]
~~~~~~~~~~~~~~~~~'~Nome~do~Aluno:~{\textbackslash}Sssssssssssssssssssssssss`ssssssssssssssss'+ChFN+'Nome'+lf+\\\nopagebreak[3]
~~~~~~~~~~~~~~~~~'~~~~~~Endereço:~{\textbackslash}Sssssssssssssssssssssssss`ssssssssssssssss'+ChFN+'endereco'+lf+\\\nopagebreak[3]
~~~~~~~~~~~~~~~~~'~~~~~~~~~~~Cep:~{\textbackslash}{\#}{\#}-{\#}{\#}{\#}-{\#}{\#}{\#}'+ChFN+'cep'+lf+\\\nopagebreak[3]
~~~~~~~~~~~~~~~~~'~~~~~~~~Bairro:~{\textbackslash}sssssssssssssssssssssssss'+ChFN+'bairro'+lf+\\\nopagebreak[3]
~~~~~~~~~~~~~~~~~'~~~~~~~~Cidade:~{\textbackslash}sssssssssssssssssssssssss'+ChFN+'cidade'+lf+\\\nopagebreak[3]
~~~~~~~~~~~~~~~~~'~~~~~~~~Estado:~{\textbackslash}SS'+ChFN+'estado'+lf+\\\nopagebreak[3]
~~~~~~~~~~~~~~~~~'~~~~~~~~~Idade:~{\textbackslash}BB'+ChFN+'idade'+lf+\\\nopagebreak[3]
~~~~~~~~~~~~~~~~~'~~~~~Matricula:~{\textbackslash}III'+ChFN+'matricula'+lf+\\\nopagebreak[3]
~~~~~~~~~~~~~~~~~'~~~Mensalidade:~{\textbackslash}R,RRR.RR'+ChFN+'mensalidade',\\\nopagebreak[3]
~~~~~~~~~~~~~~~~~InputBoxOnEnter,}\textbf{nil}\texttt{,\\\nopagebreak[3]
\\\nopagebreak[3]
~~~~~~~~~~~~~~~~~InputBoxOnEnterField,}\textbf{nil}\texttt{,\\\nopagebreak[3]
~~~~~~~~~~~~~~~~~InputBoxOnCloseQuery,\\\nopagebreak[3]
~~~~~~~~~~~~~~~~~MI{\_}UI{\_}InputBox\\\nopagebreak[3]
~~~~~~~~~~~~)~=~MrOk\\\nopagebreak[3]
~~}\textbf{then}\texttt{~}\textbf{with}\texttt{~MI{\_}UI{\_}InputBox.DmxScroller{\_}Form1~}\textbf{do}\texttt{\\\nopagebreak[3]
~~~~~~~}\textbf{begin}\texttt{\\\nopagebreak[3]
~~~~~~~~~}\textbf{if}\texttt{~FieldByName('nome').AsString~=~''\\\nopagebreak[3]
~~~~~~~~~}\textbf{then}\texttt{~}\textbf{begin}\texttt{\\\nopagebreak[3]
~~~~~~~~~~~~~~~~ShowMessage('Campo~"nome"~não~pode~ser~vazio');\\\nopagebreak[3]
~~~~~~~~~~~~~~}\textbf{end}\texttt{;\\\nopagebreak[3]
~~~~~~~~~MI{\_}UI{\_}InputBox.free;\\\nopagebreak[3]
~~~~~~~}\textbf{end}\texttt{;\\\nopagebreak[3]
}\textbf{end}\texttt{;\\\nopagebreak[3]
\\\nopagebreak[3]
}\textbf{procedure}\texttt{~TDmxscroller{\_}form{\_}test.Button{\_}InputBoxClick(Sender:~TObject);\\\nopagebreak[3]
}\textbf{begin}\texttt{\\\nopagebreak[3]
~~TestInputBox;\\\nopagebreak[3]
}\textbf{end}\texttt{;\\
}
\end{itemize}\subsubsection*{\large{\textbf{Propriedades}}\normalsize\hspace{1ex}\hfill}
\paragraph*{Template}\hspace*{\fill}

\begin{list}{}{
\settowidth{\tmplength}{\textbf{Declaração}}
\setlength{\itemindent}{0cm}
\setlength{\listparindent}{0cm}
\setlength{\leftmargin}{\evensidemargin}
\addtolength{\leftmargin}{\tmplength}
\settowidth{\labelsep}{X}
\addtolength{\leftmargin}{\labelsep}
\setlength{\labelwidth}{\tmplength}
}
\begin{flushleft}
\item[\textbf{Declaração}\hfill]
\begin{ttfamily}
public property Template: AnsiString read {\_}Template write Set{\_}Template;\end{ttfamily}


\end{flushleft}
\par
\item[\textbf{Descrição}]
A propriedade \textbf{\begin{ttfamily}Template\end{ttfamily}} é usada para criar uma lista de \begin{ttfamily}PSItem\end{ttfamily}(\ref{mi_rtl_ui_Dmxscroller-PSItem}) para ser usada como modelo do formulário.

\begin{itemize}
\item \textbf{NOTAS} \begin{itemize}
\item \begin{ttfamily}Template\end{ttfamily} é um string comum, onde cada linha é separada com {\^{}}J.
\item \begin{ttfamily}Template\end{ttfamily} tem uma lista de string com formato Dmx. \begin{itemize}
\item Formato da propriedade \begin{ttfamily}Template\end{ttfamily}:

\texttt{\\\nopagebreak[3]
\\\nopagebreak[3]
Template~:=~'~Nome~do~Aluno:~{\textbackslash}Sssssssssssssssssssssssss`ssssssssssssssss'+ChFN+'Nome'+lf+\\\nopagebreak[3]
~~~~~~~~~~~~'~~~~~~Endereço:~{\textbackslash}Sssssssssssssssssssssssss`ssssssssssssssss'+ChFN+'endereco'+lf+\\\nopagebreak[3]
~~~~~~~~~~~~'~~~~~~~~~~~Cep:~{\textbackslash}{\#}{\#}-{\#}{\#}{\#}-{\#}{\#}{\#}'+ChFN+'cep'+lf+\\\nopagebreak[3]
~~~~~~~~~~~~'~~~~~~~~Bairro:~{\textbackslash}sssssssssssssssssssssssss'+ChFN+'bairro'+lf+\\\nopagebreak[3]
~~~~~~~~~~~~'~~~~~~~~Cidade:~{\textbackslash}sssssssssssssssssssssssss'+ChFN+'cidade'+lf+\\\nopagebreak[3]
~~~~~~~~~~~~'~~~~~~~~Estado:~{\textbackslash}SS'+ChFN+'estado'+lf+\\\nopagebreak[3]
~~~~~~~~~~~~'~~~~~~~~~Idade:~{\textbackslash}BB'+ChFN+'idade'+FldUpperLimit+{\#}18+lf+\\\nopagebreak[3]
~~~~~~~~~~~~'~~~~~Matricula:~{\textbackslash}III'+ChFN+'matricula'+lf+\\\nopagebreak[3]
~~~~~~~~~~~~'~~~~~~Valor~da~'+lf+\\\nopagebreak[3]
~~~~~~~~~~~~'~~~~~Mensalidade:~{\textbackslash}R,RRR.RR'+ChFN+'mensalidade';\\
}
\end{itemize}
\item \textbf{SINTAXE} \begin{itemize}
\item \textbf{~} (til) : Limitador de rótulos do formulário;
\item \textbf{s} (s minúsculo) : caracteres alfanumérico incluindo os maiúsculas, minusculas, números e símbolos;
\item \textbf{S} (S maiúsculo) : caracteres alfanumérico incluindo os maiúsculas, números e símbolos;
\item \textbf{{\#}} ({\#} cancela) : Aceita somente números de 0 a 99
\item \textbf{{-}} (literal ) : Separador de números
\item \textbf{B} (B maiúsculo): Campo do tipo byte
\item \textbf{FldUpperLimit} : O caractere seguinte indica o limite superior da variável. No exemplo acima é 18 anos;
\item \textbf{R} : Indica um caractere de um campo do tipo double;
\item \textbf{I} : Indica um caractere de um campo do tipo interger. Faixa: {-}32000 a +32000;
\end{itemize}
\end{itemize}
\end{itemize}

\end{list}
\subsubsection*{\large{\textbf{Campos}}\normalsize\hspace{1ex}\hfill}
\paragraph*{ButtonPanel1}\hspace*{\fill}

\begin{list}{}{
\settowidth{\tmplength}{\textbf{Declaração}}
\setlength{\itemindent}{0cm}
\setlength{\listparindent}{0cm}
\setlength{\leftmargin}{\evensidemargin}
\addtolength{\leftmargin}{\tmplength}
\settowidth{\labelsep}{X}
\addtolength{\leftmargin}{\labelsep}
\setlength{\labelwidth}{\tmplength}
}
\begin{flushleft}
\item[\textbf{Declaração}\hfill]
\begin{ttfamily}
public ButtonPanel1: TButtonPanel;\end{ttfamily}


\end{flushleft}
\end{list}
\paragraph*{DmxScroller{\_}Form1}\hspace*{\fill}

\begin{list}{}{
\settowidth{\tmplength}{\textbf{Declaração}}
\setlength{\itemindent}{0cm}
\setlength{\listparindent}{0cm}
\setlength{\leftmargin}{\evensidemargin}
\addtolength{\leftmargin}{\tmplength}
\settowidth{\labelsep}{X}
\addtolength{\leftmargin}{\labelsep}
\setlength{\labelwidth}{\tmplength}
}
\begin{flushleft}
\item[\textbf{Declaração}\hfill]
\begin{ttfamily}
public DmxScroller{\_}Form1: TDmxScroller{\_}Form;\end{ttfamily}


\end{flushleft}
\end{list}
\paragraph*{Mi{\_}ScrollBox{\_}LCL1}\hspace*{\fill}

\begin{list}{}{
\settowidth{\tmplength}{\textbf{Declaração}}
\setlength{\itemindent}{0cm}
\setlength{\listparindent}{0cm}
\setlength{\leftmargin}{\evensidemargin}
\addtolength{\leftmargin}{\tmplength}
\settowidth{\labelsep}{X}
\addtolength{\leftmargin}{\labelsep}
\setlength{\labelwidth}{\tmplength}
}
\begin{flushleft}
\item[\textbf{Declaração}\hfill]
\begin{ttfamily}
public Mi{\_}ScrollBox{\_}LCL1: TMi{\_}ScrollBox{\_}LCL;\end{ttfamily}


\end{flushleft}
\end{list}
\subsubsection*{\large{\textbf{Métodos}}\normalsize\hspace{1ex}\hfill}
\paragraph*{CancelButtonClick}\hspace*{\fill}

\begin{list}{}{
\settowidth{\tmplength}{\textbf{Declaração}}
\setlength{\itemindent}{0cm}
\setlength{\listparindent}{0cm}
\setlength{\leftmargin}{\evensidemargin}
\addtolength{\leftmargin}{\tmplength}
\settowidth{\labelsep}{X}
\addtolength{\leftmargin}{\labelsep}
\setlength{\labelwidth}{\tmplength}
}
\begin{flushleft}
\item[\textbf{Declaração}\hfill]
\begin{ttfamily}
public procedure CancelButtonClick(Sender: TObject);\end{ttfamily}


\end{flushleft}
\end{list}
\paragraph*{CloseButtonClick}\hspace*{\fill}

\begin{list}{}{
\settowidth{\tmplength}{\textbf{Declaração}}
\setlength{\itemindent}{0cm}
\setlength{\listparindent}{0cm}
\setlength{\leftmargin}{\evensidemargin}
\addtolength{\leftmargin}{\tmplength}
\settowidth{\labelsep}{X}
\addtolength{\leftmargin}{\labelsep}
\setlength{\labelwidth}{\tmplength}
}
\begin{flushleft}
\item[\textbf{Declaração}\hfill]
\begin{ttfamily}
public procedure CloseButtonClick(Sender: TObject);\end{ttfamily}


\end{flushleft}
\end{list}
\paragraph*{DmxScroller{\_}Form1AddTemplate}\hspace*{\fill}

\begin{list}{}{
\settowidth{\tmplength}{\textbf{Declaração}}
\setlength{\itemindent}{0cm}
\setlength{\listparindent}{0cm}
\setlength{\leftmargin}{\evensidemargin}
\addtolength{\leftmargin}{\tmplength}
\settowidth{\labelsep}{X}
\addtolength{\leftmargin}{\labelsep}
\setlength{\labelwidth}{\tmplength}
}
\begin{flushleft}
\item[\textbf{Declaração}\hfill]
\begin{ttfamily}
public procedure DmxScroller{\_}Form1AddTemplate(const aUiDmxScroller: TUiDmxScroller );\end{ttfamily}


\end{flushleft}
\end{list}
\paragraph*{DmxScroller{\_}Form1CloseQuery}\hspace*{\fill}

\begin{list}{}{
\settowidth{\tmplength}{\textbf{Declaração}}
\setlength{\itemindent}{0cm}
\setlength{\listparindent}{0cm}
\setlength{\leftmargin}{\evensidemargin}
\addtolength{\leftmargin}{\tmplength}
\settowidth{\labelsep}{X}
\addtolength{\leftmargin}{\labelsep}
\setlength{\labelwidth}{\tmplength}
}
\begin{flushleft}
\item[\textbf{Declaração}\hfill]
\begin{ttfamily}
public procedure DmxScroller{\_}Form1CloseQuery(aDmxScroller: TUiDmxScroller;var CanClose: boolean);\end{ttfamily}


\end{flushleft}
\end{list}
\paragraph*{DmxScroller{\_}Form1Enter}\hspace*{\fill}

\begin{list}{}{
\settowidth{\tmplength}{\textbf{Declaração}}
\setlength{\itemindent}{0cm}
\setlength{\listparindent}{0cm}
\setlength{\leftmargin}{\evensidemargin}
\addtolength{\leftmargin}{\tmplength}
\settowidth{\labelsep}{X}
\addtolength{\leftmargin}{\labelsep}
\setlength{\labelwidth}{\tmplength}
}
\begin{flushleft}
\item[\textbf{Declaração}\hfill]
\begin{ttfamily}
public procedure DmxScroller{\_}Form1Enter(aDmxScroller: TUiDmxScroller);\end{ttfamily}


\end{flushleft}
\end{list}
\paragraph*{DmxScroller{\_}Form1EnterField}\hspace*{\fill}

\begin{list}{}{
\settowidth{\tmplength}{\textbf{Declaração}}
\setlength{\itemindent}{0cm}
\setlength{\listparindent}{0cm}
\setlength{\leftmargin}{\evensidemargin}
\addtolength{\leftmargin}{\tmplength}
\settowidth{\labelsep}{X}
\addtolength{\leftmargin}{\labelsep}
\setlength{\labelwidth}{\tmplength}
}
\begin{flushleft}
\item[\textbf{Declaração}\hfill]
\begin{ttfamily}
public procedure DmxScroller{\_}Form1EnterField(aField: pDmxFieldRec);\end{ttfamily}


\end{flushleft}
\end{list}
\paragraph*{DmxScroller{\_}Form1Exit}\hspace*{\fill}

\begin{list}{}{
\settowidth{\tmplength}{\textbf{Declaração}}
\setlength{\itemindent}{0cm}
\setlength{\listparindent}{0cm}
\setlength{\leftmargin}{\evensidemargin}
\addtolength{\leftmargin}{\tmplength}
\settowidth{\labelsep}{X}
\addtolength{\leftmargin}{\labelsep}
\setlength{\labelwidth}{\tmplength}
}
\begin{flushleft}
\item[\textbf{Declaração}\hfill]
\begin{ttfamily}
public procedure DmxScroller{\_}Form1Exit(aDmxScroller: TUiDmxScroller);\end{ttfamily}


\end{flushleft}
\end{list}
\paragraph*{DmxScroller{\_}Form1ExitField}\hspace*{\fill}

\begin{list}{}{
\settowidth{\tmplength}{\textbf{Declaração}}
\setlength{\itemindent}{0cm}
\setlength{\listparindent}{0cm}
\setlength{\leftmargin}{\evensidemargin}
\addtolength{\leftmargin}{\tmplength}
\settowidth{\labelsep}{X}
\addtolength{\leftmargin}{\labelsep}
\setlength{\labelwidth}{\tmplength}
}
\begin{flushleft}
\item[\textbf{Declaração}\hfill]
\begin{ttfamily}
public procedure DmxScroller{\_}Form1ExitField(aField: pDmxFieldRec);\end{ttfamily}


\end{flushleft}
\end{list}
\paragraph*{DmxScroller{\_}Form1GetTemplate}\hspace*{\fill}

\begin{list}{}{
\settowidth{\tmplength}{\textbf{Declaração}}
\setlength{\itemindent}{0cm}
\setlength{\listparindent}{0cm}
\setlength{\leftmargin}{\evensidemargin}
\addtolength{\leftmargin}{\tmplength}
\settowidth{\labelsep}{X}
\addtolength{\leftmargin}{\labelsep}
\setlength{\labelwidth}{\tmplength}
}
\begin{flushleft}
\item[\textbf{Declaração}\hfill]
\begin{ttfamily}
public function DmxScroller{\_}Form1GetTemplate(aNext: PSItem): PSItem;\end{ttfamily}


\end{flushleft}
\end{list}
\paragraph*{DmxScroller{\_}Form1NewRecord}\hspace*{\fill}

\begin{list}{}{
\settowidth{\tmplength}{\textbf{Declaração}}
\setlength{\itemindent}{0cm}
\setlength{\listparindent}{0cm}
\setlength{\leftmargin}{\evensidemargin}
\addtolength{\leftmargin}{\tmplength}
\settowidth{\labelsep}{X}
\addtolength{\leftmargin}{\labelsep}
\setlength{\labelwidth}{\tmplength}
}
\begin{flushleft}
\item[\textbf{Declaração}\hfill]
\begin{ttfamily}
public procedure DmxScroller{\_}Form1NewRecord(aDmxScroller: TUiDmxScroller);\end{ttfamily}


\end{flushleft}
\end{list}
\paragraph*{FormCloseQuery}\hspace*{\fill}

\begin{list}{}{
\settowidth{\tmplength}{\textbf{Declaração}}
\setlength{\itemindent}{0cm}
\setlength{\listparindent}{0cm}
\setlength{\leftmargin}{\evensidemargin}
\addtolength{\leftmargin}{\tmplength}
\settowidth{\labelsep}{X}
\addtolength{\leftmargin}{\labelsep}
\setlength{\labelwidth}{\tmplength}
}
\begin{flushleft}
\item[\textbf{Declaração}\hfill]
\begin{ttfamily}
public procedure FormCloseQuery(Sender: TObject; var CanClose: Boolean);\end{ttfamily}


\end{flushleft}
\end{list}
\paragraph*{FormCreate}\hspace*{\fill}

\begin{list}{}{
\settowidth{\tmplength}{\textbf{Declaração}}
\setlength{\itemindent}{0cm}
\setlength{\listparindent}{0cm}
\setlength{\leftmargin}{\evensidemargin}
\addtolength{\leftmargin}{\tmplength}
\settowidth{\labelsep}{X}
\addtolength{\leftmargin}{\labelsep}
\setlength{\labelwidth}{\tmplength}
}
\begin{flushleft}
\item[\textbf{Declaração}\hfill]
\begin{ttfamily}
public procedure FormCreate(Sender: TObject);\end{ttfamily}


\end{flushleft}
\end{list}
\paragraph*{OKButtonClick}\hspace*{\fill}

\begin{list}{}{
\settowidth{\tmplength}{\textbf{Declaração}}
\setlength{\itemindent}{0cm}
\setlength{\listparindent}{0cm}
\setlength{\leftmargin}{\evensidemargin}
\addtolength{\leftmargin}{\tmplength}
\settowidth{\labelsep}{X}
\addtolength{\leftmargin}{\labelsep}
\setlength{\labelwidth}{\tmplength}
}
\begin{flushleft}
\item[\textbf{Declaração}\hfill]
\begin{ttfamily}
public procedure OKButtonClick(Sender: TObject);\end{ttfamily}


\end{flushleft}
\end{list}
\paragraph*{Set{\_}Template}\hspace*{\fill}

\begin{list}{}{
\settowidth{\tmplength}{\textbf{Declaração}}
\setlength{\itemindent}{0cm}
\setlength{\listparindent}{0cm}
\setlength{\leftmargin}{\evensidemargin}
\addtolength{\leftmargin}{\tmplength}
\settowidth{\labelsep}{X}
\addtolength{\leftmargin}{\labelsep}
\setlength{\labelwidth}{\tmplength}
}
\begin{flushleft}
\item[\textbf{Declaração}\hfill]
\begin{ttfamily}
protected Procedure Set{\_}Template(aTemplate:AnsiString);\end{ttfamily}


\end{flushleft}
\par
\item[\textbf{Descrição}]
O Método \textbf{\begin{ttfamily}Set{\_}Template\end{ttfamily}} inicia o atributo {\_}Template e criar a lista {\_}FormSItem : \begin{ttfamily}PSitem\end{ttfamily}(\ref{mi_rtl_ui_Dmxscroller-PSItem})

\begin{itemize}
\item \textbf{NOTAS} \begin{itemize}
\item Criar em TObjectss.TStringList o método
\end{itemize}
\end{itemize}

\end{list}
\paragraph*{SetAlias}\hspace*{\fill}

\begin{list}{}{
\settowidth{\tmplength}{\textbf{Declaração}}
\setlength{\itemindent}{0cm}
\setlength{\listparindent}{0cm}
\setlength{\leftmargin}{\evensidemargin}
\addtolength{\leftmargin}{\tmplength}
\settowidth{\labelsep}{X}
\addtolength{\leftmargin}{\labelsep}
\setlength{\labelwidth}{\tmplength}
}
\begin{flushleft}
\item[\textbf{Declaração}\hfill]
\begin{ttfamily}
public Procedure SetAlias(aTitle:AnsiString);\end{ttfamily}


\end{flushleft}
\end{list}
\section{Funções e Procedimentos}
\subsection*{InputBox}
\begin{list}{}{
\settowidth{\tmplength}{\textbf{Declaração}}
\setlength{\itemindent}{0cm}
\setlength{\listparindent}{0cm}
\setlength{\leftmargin}{\evensidemargin}
\addtolength{\leftmargin}{\tmplength}
\settowidth{\labelsep}{X}
\addtolength{\leftmargin}{\labelsep}
\setlength{\labelwidth}{\tmplength}
}
\begin{flushleft}
\item[\textbf{Declaração}\hfill]
\begin{ttfamily}
function InputBox( aTitle: AnsiString; aTemplate: AnsiString; aOnEnterLocal:TOnEnterLocal ; aOnExitLocal:TOnExitLocal; aOnEnterFieldLocal:TOnEnterFieldLocal; aOnExitFieldLocal:TOnExitFieldLocal; aOnCloseQueryLocal:TOnCloseQueryLocal; out aMi{\_}ui{\_}InputBox : TMI{\_}UI{\_}InputBox ): TModalResult;\end{ttfamily}


\end{flushleft}
\par
\item[\textbf{Descrição}]
A função \textbf{\begin{ttfamily}InputBox\end{ttfamily}} cria um formulário passado por Template e executa os eventos do formulário passado pelos parâmetros.

\begin{itemize}
\item \textbf{PARÂMETROS} \begin{itemize}
\item atitle; // Título do formulário;
\item aTemplate; // Modelo do formulário cuja a sintaxe segue abaixo:
\item aOnEnter; // Executado ao entrar no formulário criado baseado no Template;
\item aOnExit; // Executado ao sair do formulário criado baseado no Template;
\item aOnEnterField; // Executado ao entrar um campo focado;
\item aOnExitField; // Executado ao sair do campo focado;
\item aOnCloseQuery // Executado ao fechar o form se CanClose = true;
\end{itemize}
\item \textbf{SINTAXE DO MODELO} \begin{itemize}
\item Exemplo \texttt{\\\nopagebreak[3]
\\\nopagebreak[3]
}\textbf{const}\texttt{\\\nopagebreak[3]
~~Template~:=~'~Nome~do~Aluno:~{\textbackslash}sssssssssssssssssssssssss'+lf+\\\nopagebreak[3]
~~~~~~~~~~~~~~'~~~~~~Endereço:~{\textbackslash}sssssssssssssssssssssssss`ssssssssssssssss'+lf+\\\nopagebreak[3]
~~~~~~~~~~~~~~'~~~~~~~~~~~Cep:~{\textbackslash}{\#}{\#}-{\#}{\#}{\#}-{\#}{\#}{\#}'+lf+\\\nopagebreak[3]
~~~~~~~~~~~~~~'~~~~~~~~Bairro:~{\textbackslash}sssssssssssssssssssssssss'+lf+\\\nopagebreak[3]
~~~~~~~~~~~~~~'~~~~~~~~Cidade:~{\textbackslash}sssssssssssssssssssssssss'+lf+\\\nopagebreak[3]
~~~~~~~~~~~~~~'~~~~~~~~Estado:~{\textbackslash}sssssssssssssssssssssssss'+lf+\\\nopagebreak[3]
~~~~~~~~~~~~~~'~~~~~~~~~Idade:~{\textbackslash}BB'+lf+\\\nopagebreak[3]
~~~~~~~~~~~~~~'~~~Mensalidade:~{\textbackslash}R,RRR.RR';\\
}
\item Tipos de dados do formulário: \begin{itemize}
\item s = Char alfanumérico
\item {\#} = Char numérico
\item R = Double
\item B = Byte
\end{itemize}
\end{itemize}
\item Exemplo de chamada da função:

\texttt{\\\nopagebreak[3]
\\\nopagebreak[3]
}\textbf{if}\texttt{~InputBox('Dados~do~aluno',\\\nopagebreak[3]
~~~~~~~~~~~~~'~Nome~do~Aluno:~{\textbackslash}sssssssssssssssssssssssss'+lf+\\\nopagebreak[3]
~~~~~~~~~~~~~'~~~~~~Endereço:~{\textbackslash}sssssssssssssssssssssssss`ssssssssssssssss'+lf+\\\nopagebreak[3]
~~~~~~~~~~~~~'~~~~~~~~~~~Cep:~{\textbackslash}{\#}{\#}-{\#}{\#}{\#}-{\#}{\#}{\#}'+lf+\\\nopagebreak[3]
~~~~~~~~~~~~~'~~~~~~~~Bairro:~{\textbackslash}sssssssssssssssssssssssss'+lf+\\\nopagebreak[3]
~~~~~~~~~~~~~'~~~~~~~~Cidade:~{\textbackslash}sssssssssssssssssssssssss'+lf+\\\nopagebreak[3]
~~~~~~~~~~~~~'~~~~~~~~Estado:~{\textbackslash}sssssssssssssssssssssssss'+lf+\\\nopagebreak[3]
~~~~~~~~~~~~~'~~~~~~~~~Idade:~{\textbackslash}BB'+lf+\\\nopagebreak[3]
~~~~~~~~~~~~~'~~~~~Mensalidade:~{\textbackslash}R,RRR.RR',\\\nopagebreak[3]
~~~~~~~~~~~~~}\textbf{nil}\texttt{,}\textbf{nil}\texttt{,}\textbf{nil}\texttt{,}\textbf{nil}\texttt{,}\textbf{nil}\texttt{\\\nopagebreak[3]
~~~~~~~~)~=~MrOk\\\nopagebreak[3]
}\textbf{then}\texttt{~}\textbf{begin}\texttt{\\\nopagebreak[3]
~~~~~}\textbf{end}\texttt{;\\
}
\end{itemize}

\end{list}
\section{Tipos}
\subsection*{TOnEnterLocal}
\begin{list}{}{
\settowidth{\tmplength}{\textbf{Declaração}}
\setlength{\itemindent}{0cm}
\setlength{\listparindent}{0cm}
\setlength{\leftmargin}{\evensidemargin}
\addtolength{\leftmargin}{\tmplength}
\settowidth{\labelsep}{X}
\addtolength{\leftmargin}{\labelsep}
\setlength{\labelwidth}{\tmplength}
}
\begin{flushleft}
\item[\textbf{Declaração}\hfill]
\begin{ttfamily}
TOnEnterLocal = Procedure(aDmxScroller:TUiDmxScroller);\end{ttfamily}


\end{flushleft}
\par
\item[\textbf{Descrição}]
O tipo \textbf{\begin{ttfamily}TOnEnterLocal\end{ttfamily}} é usado para implementar evento onEnterLocal do atributo \textbf{Mi{\_}ScrollBox{\_}LCL1}

\end{list}
\subsection*{TOnExitLocal}
\begin{list}{}{
\settowidth{\tmplength}{\textbf{Declaração}}
\setlength{\itemindent}{0cm}
\setlength{\listparindent}{0cm}
\setlength{\leftmargin}{\evensidemargin}
\addtolength{\leftmargin}{\tmplength}
\settowidth{\labelsep}{X}
\addtolength{\leftmargin}{\labelsep}
\setlength{\labelwidth}{\tmplength}
}
\begin{flushleft}
\item[\textbf{Declaração}\hfill]
\begin{ttfamily}
TOnExitLocal = Procedure(aDmxScroller:TUiDmxScroller);\end{ttfamily}


\end{flushleft}
\par
\item[\textbf{Descrição}]
O tipo \textbf{\begin{ttfamily}TOnExitLocal\end{ttfamily}} é usado para implementar evento onExitLocal do atributo \textbf{Mi{\_}ScrollBox{\_}LCL1}

\end{list}
\subsection*{TOnEnterFieldLocal}
\begin{list}{}{
\settowidth{\tmplength}{\textbf{Declaração}}
\setlength{\itemindent}{0cm}
\setlength{\listparindent}{0cm}
\setlength{\leftmargin}{\evensidemargin}
\addtolength{\leftmargin}{\tmplength}
\settowidth{\labelsep}{X}
\addtolength{\leftmargin}{\labelsep}
\setlength{\labelwidth}{\tmplength}
}
\begin{flushleft}
\item[\textbf{Declaração}\hfill]
\begin{ttfamily}
TOnEnterFieldLocal = Procedure(aField:pDmxFieldRec);\end{ttfamily}


\end{flushleft}
\par
\item[\textbf{Descrição}]
O tipo \textbf{\begin{ttfamily}TOnEnterFieldLocal\end{ttfamily}} é usado para implementar evento OnEnterFieldLocal do atributo \textbf{Mi{\_}ScrollBox{\_}LCL1}

\end{list}
\subsection*{TOnExitFieldLocal}
\begin{list}{}{
\settowidth{\tmplength}{\textbf{Declaração}}
\setlength{\itemindent}{0cm}
\setlength{\listparindent}{0cm}
\setlength{\leftmargin}{\evensidemargin}
\addtolength{\leftmargin}{\tmplength}
\settowidth{\labelsep}{X}
\addtolength{\leftmargin}{\labelsep}
\setlength{\labelwidth}{\tmplength}
}
\begin{flushleft}
\item[\textbf{Declaração}\hfill]
\begin{ttfamily}
TOnExitFieldLocal = Procedure(aField:pDmxFieldRec);\end{ttfamily}


\end{flushleft}
\par
\item[\textbf{Descrição}]
O tipo \textbf{\begin{ttfamily}TOnExitFieldLocal\end{ttfamily}} é usado para implementar evento OnExitFieldLocal do atributo \textbf{Mi{\_}ScrollBox{\_}LCL1}

\end{list}
\subsection*{TOnCloseQueryLocal}
\begin{list}{}{
\settowidth{\tmplength}{\textbf{Declaração}}
\setlength{\itemindent}{0cm}
\setlength{\listparindent}{0cm}
\setlength{\leftmargin}{\evensidemargin}
\addtolength{\leftmargin}{\tmplength}
\settowidth{\labelsep}{X}
\addtolength{\leftmargin}{\labelsep}
\setlength{\labelwidth}{\tmplength}
}
\begin{flushleft}
\item[\textbf{Declaração}\hfill]
\begin{ttfamily}
TOnCloseQueryLocal = Procedure(aDmxScroller:TUiDmxScroller; var CanClose:boolean);\end{ttfamily}


\end{flushleft}
\par
\item[\textbf{Descrição}]
O tipo \textbf{\begin{ttfamily}TOnCloseQueryLocal\end{ttfamily}} é usado para implementar evento OnCloseQueryLocal do atributo \textbf{Mi{\_}ScrollBox{\_}LCL1}.

\begin{itemize}
\item \textbf{NOTA* \begin{itemize}
\item Este evento é disparado antes de desativar a classe **\begin{ttfamily}TUiDmxScroller\end{ttfamily}(\ref{mi_rtl_ui_Dmxscroller.TUiDmxScroller})
\end{itemize}}. \begin{itemize}
\item Obs: Se o parâmetro \textbf{CanClose} for \textbf{false}, então a classe \textbf{\begin{ttfamily}TUiDmxScroller\end{ttfamily}(\ref{mi_rtl_ui_Dmxscroller.TUiDmxScroller})} não é desativado.
\end{itemize}
\end{itemize}

\end{list}
\chapter{Unit umi{\_}ui{\_}inputbox{\_}lcl{\_}test}
\section{Uses}
\begin{itemize}
\item \begin{ttfamily}Classes\end{ttfamily}\item \begin{ttfamily}SysUtils\end{ttfamily}\item \begin{ttfamily}umi{\_}ui{\_}InputBox{\_}lcl\end{ttfamily}(\ref{umi_ui_InputBox_lcl})\end{itemize}
\section{Visão Geral}
\begin{description}
\item[\texttt{TestinputBox}]
\end{description}
\section{Funções e Procedimentos}
\subsection*{TestinputBox}
\begin{list}{}{
\settowidth{\tmplength}{\textbf{Declaração}}
\setlength{\itemindent}{0cm}
\setlength{\listparindent}{0cm}
\setlength{\leftmargin}{\evensidemargin}
\addtolength{\leftmargin}{\tmplength}
\settowidth{\labelsep}{X}
\addtolength{\leftmargin}{\labelsep}
\setlength{\labelwidth}{\tmplength}
}
\begin{flushleft}
\item[\textbf{Declaração}\hfill]
\begin{ttfamily}
Procedure TestinputBox;\end{ttfamily}


\end{flushleft}
\end{list}
\chapter{Unit uMi{\_}ui{\_}Label{\_}lcl}
\section{Uses}
\begin{itemize}
\item \begin{ttfamily}Classes\end{ttfamily}\item \begin{ttfamily}SysUtils\end{ttfamily}\item \begin{ttfamily}LResources\end{ttfamily}\item \begin{ttfamily}Forms\end{ttfamily}\item \begin{ttfamily}Controls\end{ttfamily}\item \begin{ttfamily}Graphics\end{ttfamily}\item \begin{ttfamily}Dialogs\end{ttfamily}\item \begin{ttfamily}StdCtrls\end{ttfamily}\item \begin{ttfamily}ActnList\end{ttfamily}\item \begin{ttfamily}mi{\_}rtl{\_}ui{\_}DmxScroller\end{ttfamily}(\ref{mi_rtl_ui_Dmxscroller})\item \begin{ttfamily}mi{\_}rtl{\_}ui{\_}DmxScroller{\_}Form\end{ttfamily}(\ref{mi_rtl_ui_dmxscroller_form})\item \begin{ttfamily}umi{\_}ui{\_}dmxscroller{\_}form{\_}lcl{\_}attributes\end{ttfamily}(\ref{umi_ui_dmxscroller_form_lcl_attributes})\end{itemize}
\section{Visão Geral}
\begin{description}
\item[\texttt{\begin{ttfamily}TMi{\_}ui{\_}Label{\_}lcl\end{ttfamily} Classe}]
\end{description}
\begin{description}
\item[\texttt{Register}]
\end{description}
\section{Classes, Interfaces, Objetos e Registros}
\subsection*{TMi{\_}ui{\_}Label{\_}lcl Classe}
\subsubsection*{\large{\textbf{Hierarquia}}\normalsize\hspace{1ex}\hfill}
TMi{\_}ui{\_}Label{\_}lcl {$>$} TLabel
%%%%Descrição
\subsubsection*{\large{\textbf{Propriedades}}\normalsize\hspace{1ex}\hfill}
\paragraph*{DmxFieldRec}\hspace*{\fill}

\begin{list}{}{
\settowidth{\tmplength}{\textbf{Declaração}}
\setlength{\itemindent}{0cm}
\setlength{\listparindent}{0cm}
\setlength{\leftmargin}{\evensidemargin}
\addtolength{\leftmargin}{\tmplength}
\settowidth{\labelsep}{X}
\addtolength{\leftmargin}{\labelsep}
\setlength{\labelwidth}{\tmplength}
}
\begin{flushleft}
\item[\textbf{Declaração}\hfill]
\begin{ttfamily}
public property DmxFieldRec: pDmxFieldRec Read {\_}pDmxFieldRec   Write  SeTDmxFieldRec;\end{ttfamily}


\end{flushleft}
\par
\item[\textbf{Descrição}]
A propriedade \textbf{\begin{ttfamily}DmxFieldRec\end{ttfamily}} fornece os dados necessários para criar o componente \begin{ttfamily}TMI{\_}BitBtn{\_}LCL\end{ttfamily}(\ref{umi_ui_bitbtn_lcl.TMi_BitBtn_LCL}).

\begin{itemize}
\item \textbf{NOTA} \begin{itemize}
\item Esses dados devem ser criados pelo método DmxScroller{\_}Form{\_}Lcl{\_}attributesr.CreateStruct(var ATemplate : \begin{ttfamily}TString\end{ttfamily}(\ref{mi_rtl_ui_Dmxscroller-tString}))
\end{itemize}
\end{itemize}

\end{list}
\paragraph*{DmxScroller{\_}Form{\_}Lcl{\_}attributes}\hspace*{\fill}

\begin{list}{}{
\settowidth{\tmplength}{\textbf{Declaração}}
\setlength{\itemindent}{0cm}
\setlength{\listparindent}{0cm}
\setlength{\leftmargin}{\evensidemargin}
\addtolength{\leftmargin}{\tmplength}
\settowidth{\labelsep}{X}
\addtolength{\leftmargin}{\labelsep}
\setlength{\labelwidth}{\tmplength}
}
\begin{flushleft}
\item[\textbf{Declaração}\hfill]
\begin{ttfamily}
published property DmxScroller{\_}Form{\_}Lcl{\_}attributes : TDmxScroller{\_}Form{\_}Lcl{\_}attributes Read {\_}DmxScroller{\_}Form{\_}Lcl{\_}attributes  write SetDmxScroller{\_}Form{\_}Lcl{\_}attributes;\end{ttfamily}


\end{flushleft}
\par
\item[\textbf{Descrição}]
A propriedade \textbf{\begin{ttfamily}DmxScroller{\_}Form{\_}Lcl{\_}attributes\end{ttfamily}} contém o modelo e os cálculos do formulário

\end{list}
\subsubsection*{\large{\textbf{Métodos}}\normalsize\hspace{1ex}\hfill}
\paragraph*{DoOnClick}\hspace*{\fill}

\begin{list}{}{
\settowidth{\tmplength}{\textbf{Declaração}}
\setlength{\itemindent}{0cm}
\setlength{\listparindent}{0cm}
\setlength{\leftmargin}{\evensidemargin}
\addtolength{\leftmargin}{\tmplength}
\settowidth{\labelsep}{X}
\addtolength{\leftmargin}{\labelsep}
\setlength{\labelwidth}{\tmplength}
}
\begin{flushleft}
\item[\textbf{Declaração}\hfill]
\begin{ttfamily}
protected procedure DoOnClick(Sender: TObject);\end{ttfamily}


\end{flushleft}
\end{list}
\section{Funções e Procedimentos}
\subsection*{Register}
\begin{list}{}{
\settowidth{\tmplength}{\textbf{Declaração}}
\setlength{\itemindent}{0cm}
\setlength{\listparindent}{0cm}
\setlength{\leftmargin}{\evensidemargin}
\addtolength{\leftmargin}{\tmplength}
\settowidth{\labelsep}{X}
\addtolength{\leftmargin}{\labelsep}
\setlength{\labelwidth}{\tmplength}
}
\begin{flushleft}
\item[\textbf{Declaração}\hfill]
\begin{ttfamily}
procedure Register;\end{ttfamily}


\end{flushleft}
\end{list}
\chapter{Unit uMi{\_}ui{\_}maskedit{\_}lcl}
\section{Uses}
\begin{itemize}
\item \begin{ttfamily}Mask\end{ttfamily}\item \begin{ttfamily}Windows\end{ttfamily}\item \begin{ttfamily}SysUtils\end{ttfamily}\item \begin{ttfamily}Messages\end{ttfamily}\item \begin{ttfamily}Classes\end{ttfamily}\item \begin{ttfamily}Controls\end{ttfamily}\item \begin{ttfamily}StdCtrls\end{ttfamily}\item \begin{ttfamily}Forms\end{ttfamily}\item \begin{ttfamily}Grids\end{ttfamily}\item \begin{ttfamily}dialogs\end{ttfamily}\item \begin{ttfamily}LResources\end{ttfamily}\item \begin{ttfamily}mi.rtl.Consts.StrError\end{ttfamily}(\ref{mi.rtl.Consts.StrError})\item \begin{ttfamily}mi{\_}rtl{\_}ui{\_}DmxScroller\end{ttfamily}(\ref{mi_rtl_ui_Dmxscroller})\item \begin{ttfamily}mi{\_}rtl{\_}ui{\_}DmxScroller{\_}Form\end{ttfamily}(\ref{mi_rtl_ui_dmxscroller_form})\item \begin{ttfamily}umi{\_}ui{\_}dmxscroller{\_}form{\_}lcl{\_}attributes\end{ttfamily}(\ref{umi_ui_dmxscroller_form_lcl_attributes})\end{itemize}
\section{Visão Geral}
\begin{description}
\item[\texttt{\begin{ttfamily}TMI{\_}MaskEdit{\_}LCL\end{ttfamily} Classe}]
\end{description}
\begin{description}
\item[\texttt{Register}]
\end{description}
\section{Classes, Interfaces, Objetos e Registros}
\subsection*{TMI{\_}MaskEdit{\_}LCL Classe}
\subsubsection*{\large{\textbf{Hierarquia}}\normalsize\hspace{1ex}\hfill}
TMI{\_}MaskEdit{\_}LCL {$>$} TMaskEdit
%%%%Descrição
\subsubsection*{\large{\textbf{Propriedades}}\normalsize\hspace{1ex}\hfill}
\paragraph*{DmxScroller{\_}Form{\_}Lcl{\_}attributes}\hspace*{\fill}

\begin{list}{}{
\settowidth{\tmplength}{\textbf{Declaração}}
\setlength{\itemindent}{0cm}
\setlength{\listparindent}{0cm}
\setlength{\leftmargin}{\evensidemargin}
\addtolength{\leftmargin}{\tmplength}
\settowidth{\labelsep}{X}
\addtolength{\leftmargin}{\labelsep}
\setlength{\labelwidth}{\tmplength}
}
\begin{flushleft}
\item[\textbf{Declaração}\hfill]
\begin{ttfamily}
published property DmxScroller{\_}Form{\_}Lcl{\_}attributes : TDmxScroller{\_}Form{\_}Lcl{\_}attributes Read {\_}DmxScroller{\_}Form{\_}Lcl{\_}attributes  write SetDmxScroller{\_}Form{\_}Lcl{\_}attributes;\end{ttfamily}


\end{flushleft}
\par
\item[\textbf{Descrição}]
A propriedade \textbf{\begin{ttfamily}DmxScroller{\_}Form{\_}Lcl{\_}attributes\end{ttfamily}} contém o modelo e os cálculos do formulário

\end{list}
\paragraph*{DmxFieldRec}\hspace*{\fill}

\begin{list}{}{
\settowidth{\tmplength}{\textbf{Declaração}}
\setlength{\itemindent}{0cm}
\setlength{\listparindent}{0cm}
\setlength{\leftmargin}{\evensidemargin}
\addtolength{\leftmargin}{\tmplength}
\settowidth{\labelsep}{X}
\addtolength{\leftmargin}{\labelsep}
\setlength{\labelwidth}{\tmplength}
}
\begin{flushleft}
\item[\textbf{Declaração}\hfill]
\begin{ttfamily}
public property DmxFieldRec: pDmxFieldRec Read {\_}pDmxFieldRec   Write  SeTDmxFieldRec;\end{ttfamily}


\end{flushleft}
\par
\item[\textbf{Descrição}]
O atributo \textbf{\begin{ttfamily}DmxFieldRec\end{ttfamily}} fornece os dados necessários para criar o componente \begin{ttfamily}TMI{\_}MaskEdit{\_}LCL\end{ttfamily}(\ref{uMi_ui_maskedit_lcl.TMI_MaskEdit_LCL}).

\begin{itemize}
\item \textbf{NOTA} \begin{itemize}
\item Esses dados devem ser criados pelo método DmxScroller{\_}Form{\_}Lcl{\_}attributesr.CreateStruct(var ATemplate : \begin{ttfamily}TString\end{ttfamily}(\ref{mi_rtl_ui_Dmxscroller-tString}))
\end{itemize}
\end{itemize}

\end{list}
\subsubsection*{\large{\textbf{Campos}}\normalsize\hspace{1ex}\hfill}
\paragraph*{{\_}StringGrid}\hspace*{\fill}

\begin{list}{}{
\settowidth{\tmplength}{\textbf{Declaração}}
\setlength{\itemindent}{0cm}
\setlength{\listparindent}{0cm}
\setlength{\leftmargin}{\evensidemargin}
\addtolength{\leftmargin}{\tmplength}
\settowidth{\labelsep}{X}
\addtolength{\leftmargin}{\labelsep}
\setlength{\labelwidth}{\tmplength}
}
\begin{flushleft}
\item[\textbf{Declaração}\hfill]
\begin{ttfamily}
public var {\_}StringGrid: TStringGrid;\end{ttfamily}


\end{flushleft}
\end{list}
\subsubsection*{\large{\textbf{Métodos}}\normalsize\hspace{1ex}\hfill}
\paragraph*{Create}\hspace*{\fill}

\begin{list}{}{
\settowidth{\tmplength}{\textbf{Declaração}}
\setlength{\itemindent}{0cm}
\setlength{\listparindent}{0cm}
\setlength{\leftmargin}{\evensidemargin}
\addtolength{\leftmargin}{\tmplength}
\settowidth{\labelsep}{X}
\addtolength{\leftmargin}{\labelsep}
\setlength{\labelwidth}{\tmplength}
}
\begin{flushleft}
\item[\textbf{Declaração}\hfill]
\begin{ttfamily}
public constructor Create(AOwner:TComponent); override;\end{ttfamily}


\end{flushleft}
\end{list}
\paragraph*{PutBuffer}\hspace*{\fill}

\begin{list}{}{
\settowidth{\tmplength}{\textbf{Declaração}}
\setlength{\itemindent}{0cm}
\setlength{\listparindent}{0cm}
\setlength{\leftmargin}{\evensidemargin}
\addtolength{\leftmargin}{\tmplength}
\settowidth{\labelsep}{X}
\addtolength{\leftmargin}{\labelsep}
\setlength{\labelwidth}{\tmplength}
}
\begin{flushleft}
\item[\textbf{Declaração}\hfill]
\begin{ttfamily}
public Procedure PutBuffer;\end{ttfamily}


\end{flushleft}
\par
\item[\textbf{Descrição}]
O método \textbf{\begin{ttfamily}PutBuffer\end{ttfamily}} salva os dados do controle (Self) para a propriedade \begin{ttfamily}pDmxFieldRec\end{ttfamily}(\ref{mi_rtl_ui_Dmxscroller-pDmxFieldRec})

\end{list}
\paragraph*{GetBuffer}\hspace*{\fill}

\begin{list}{}{
\settowidth{\tmplength}{\textbf{Declaração}}
\setlength{\itemindent}{0cm}
\setlength{\listparindent}{0cm}
\setlength{\leftmargin}{\evensidemargin}
\addtolength{\leftmargin}{\tmplength}
\settowidth{\labelsep}{X}
\addtolength{\leftmargin}{\labelsep}
\setlength{\labelwidth}{\tmplength}
}
\begin{flushleft}
\item[\textbf{Declaração}\hfill]
\begin{ttfamily}
public Procedure GetBuffer;\end{ttfamily}


\end{flushleft}
\par
\item[\textbf{Descrição}]
O método \textbf{\begin{ttfamily}GetBuffer\end{ttfamily}} ler os dados da propriedade \begin{ttfamily}pDmxFieldRec\end{ttfamily}(\ref{mi_rtl_ui_Dmxscroller-pDmxFieldRec}) para o controle (Self).

\end{list}
\paragraph*{DoOnMouseDown}\hspace*{\fill}

\begin{list}{}{
\settowidth{\tmplength}{\textbf{Declaração}}
\setlength{\itemindent}{0cm}
\setlength{\listparindent}{0cm}
\setlength{\leftmargin}{\evensidemargin}
\addtolength{\leftmargin}{\tmplength}
\settowidth{\labelsep}{X}
\addtolength{\leftmargin}{\labelsep}
\setlength{\labelwidth}{\tmplength}
}
\begin{flushleft}
\item[\textbf{Declaração}\hfill]
\begin{ttfamily}
protected procedure DoOnMouseDown(Sender: TObject; Button: TMouseButton; Shift: TShiftState; X, Y: Integer);\end{ttfamily}


\end{flushleft}
\par
\item[\textbf{Descrição}]
O método \textbf{\begin{ttfamily}DoOnMouseDown\end{ttfamily}} seleciona todas as letras ou número do controle focado.

\end{list}
\paragraph*{DoOnEnter}\hspace*{\fill}

\begin{list}{}{
\settowidth{\tmplength}{\textbf{Declaração}}
\setlength{\itemindent}{0cm}
\setlength{\listparindent}{0cm}
\setlength{\leftmargin}{\evensidemargin}
\addtolength{\leftmargin}{\tmplength}
\settowidth{\labelsep}{X}
\addtolength{\leftmargin}{\labelsep}
\setlength{\labelwidth}{\tmplength}
}
\begin{flushleft}
\item[\textbf{Declaração}\hfill]
\begin{ttfamily}
protected procedure DoOnEnter(Sender: TObject);\end{ttfamily}


\end{flushleft}
\par
\item[\textbf{Descrição}]
O método \textbf{\begin{ttfamily}DoOnEnter\end{ttfamily}} ao receber o foco executa os métodos \begin{ttfamily}GetBuffer\end{ttfamily}(\ref{uMi_ui_maskedit_lcl.TMI_MaskEdit_LCL-GetBuffer}) e pDmxFieldRec.DoOnEnter(Self).

\end{list}
\paragraph*{DoOnExit}\hspace*{\fill}

\begin{list}{}{
\settowidth{\tmplength}{\textbf{Declaração}}
\setlength{\itemindent}{0cm}
\setlength{\listparindent}{0cm}
\setlength{\leftmargin}{\evensidemargin}
\addtolength{\leftmargin}{\tmplength}
\settowidth{\labelsep}{X}
\addtolength{\leftmargin}{\labelsep}
\setlength{\labelwidth}{\tmplength}
}
\begin{flushleft}
\item[\textbf{Declaração}\hfill]
\begin{ttfamily}
protected procedure DoOnExit(Sender: TObject);\end{ttfamily}


\end{flushleft}
\par
\item[\textbf{Descrição}]
O método \textbf{\begin{ttfamily}DoOnExit\end{ttfamily}} ao perder o foco executa os métodos PuttBuffer e pDmxFieldRec.DoOnExit(Self).

\end{list}
\paragraph*{DoEditNumberKeyPress}\hspace*{\fill}

\begin{list}{}{
\settowidth{\tmplength}{\textbf{Declaração}}
\setlength{\itemindent}{0cm}
\setlength{\listparindent}{0cm}
\setlength{\leftmargin}{\evensidemargin}
\addtolength{\leftmargin}{\tmplength}
\settowidth{\labelsep}{X}
\addtolength{\leftmargin}{\labelsep}
\setlength{\labelwidth}{\tmplength}
}
\begin{flushleft}
\item[\textbf{Declaração}\hfill]
\begin{ttfamily}
protected procedure DoEditNumberKeyPress(Sender: TObject; var Key: char);\end{ttfamily}


\end{flushleft}
\par
\item[\textbf{Descrição}]
O método \textbf{\begin{ttfamily}DoEditNumberKeyPress\end{ttfamily}} edita os campos números de 1 a 10 bytes

\end{list}
\paragraph*{DoOnKeyPress}\hspace*{\fill}

\begin{list}{}{
\settowidth{\tmplength}{\textbf{Declaração}}
\setlength{\itemindent}{0cm}
\setlength{\listparindent}{0cm}
\setlength{\leftmargin}{\evensidemargin}
\addtolength{\leftmargin}{\tmplength}
\settowidth{\labelsep}{X}
\addtolength{\leftmargin}{\labelsep}
\setlength{\labelwidth}{\tmplength}
}
\begin{flushleft}
\item[\textbf{Declaração}\hfill]
\begin{ttfamily}
protected procedure DoOnKeyPress(Sender: TObject; var Key: system.Char);\end{ttfamily}


\end{flushleft}
\par
\item[\textbf{Descrição}]
O método \textbf{\begin{ttfamily}DoOnKeyPress\end{ttfamily}} não usado por enquanto???

\end{list}
\paragraph*{GetHelpCtx{\_}Hint}\hspace*{\fill}

\begin{list}{}{
\settowidth{\tmplength}{\textbf{Declaração}}
\setlength{\itemindent}{0cm}
\setlength{\listparindent}{0cm}
\setlength{\leftmargin}{\evensidemargin}
\addtolength{\leftmargin}{\tmplength}
\settowidth{\labelsep}{X}
\addtolength{\leftmargin}{\labelsep}
\setlength{\labelwidth}{\tmplength}
}
\begin{flushleft}
\item[\textbf{Declaração}\hfill]
\begin{ttfamily}
protected FUNCTION GetHelpCtx{\_}Hint(): AnsiString;\end{ttfamily}


\end{flushleft}
\par
\item[\textbf{Descrição}]
O método \textbf{\begin{ttfamily}GetHelpCtx{\_}Hint\end{ttfamily}} captura a documentação do campo definido na classe onde o campo for criado.

\begin{itemize}
\item Com o programa \textbf{pasdoc} a documentação não precisa está no arquivo de recursos, por isso, para obter o link para o campo é preciso saber apenas o endereço do link.
\end{itemize}

\end{list}
\paragraph*{GetSize}\hspace*{\fill}

\begin{list}{}{
\settowidth{\tmplength}{\textbf{Declaração}}
\setlength{\itemindent}{0cm}
\setlength{\listparindent}{0cm}
\setlength{\leftmargin}{\evensidemargin}
\addtolength{\leftmargin}{\tmplength}
\settowidth{\labelsep}{X}
\addtolength{\leftmargin}{\labelsep}
\setlength{\labelwidth}{\tmplength}
}
\begin{flushleft}
\item[\textbf{Declaração}\hfill]
\begin{ttfamily}
protected FUNCTION GetSize(): Variant;\end{ttfamily}


\end{flushleft}
\end{list}
\paragraph*{SetSize}\hspace*{\fill}

\begin{list}{}{
\settowidth{\tmplength}{\textbf{Declaração}}
\setlength{\itemindent}{0cm}
\setlength{\listparindent}{0cm}
\setlength{\leftmargin}{\evensidemargin}
\addtolength{\leftmargin}{\tmplength}
\settowidth{\labelsep}{X}
\addtolength{\leftmargin}{\labelsep}
\setlength{\labelwidth}{\tmplength}
}
\begin{flushleft}
\item[\textbf{Declaração}\hfill]
\begin{ttfamily}
protected PROCEDURE SetSize(aSize: Variant);\end{ttfamily}


\end{flushleft}
\end{list}
\paragraph*{SetAlias}\hspace*{\fill}

\begin{list}{}{
\settowidth{\tmplength}{\textbf{Declaração}}
\setlength{\itemindent}{0cm}
\setlength{\listparindent}{0cm}
\setlength{\leftmargin}{\evensidemargin}
\addtolength{\leftmargin}{\tmplength}
\settowidth{\labelsep}{X}
\addtolength{\leftmargin}{\labelsep}
\setlength{\labelwidth}{\tmplength}
}
\begin{flushleft}
\item[\textbf{Declaração}\hfill]
\begin{ttfamily}
protected procedure SetAlias(const aAlias: AnsiString);\end{ttfamily}


\end{flushleft}
\end{list}
\paragraph*{GetName}\hspace*{\fill}

\begin{list}{}{
\settowidth{\tmplength}{\textbf{Declaração}}
\setlength{\itemindent}{0cm}
\setlength{\listparindent}{0cm}
\setlength{\leftmargin}{\evensidemargin}
\addtolength{\leftmargin}{\tmplength}
\settowidth{\labelsep}{X}
\addtolength{\leftmargin}{\labelsep}
\setlength{\labelwidth}{\tmplength}
}
\begin{flushleft}
\item[\textbf{Declaração}\hfill]
\begin{ttfamily}
protected FUNCTION GetName(): AnsiString;\end{ttfamily}


\end{flushleft}
\end{list}
\paragraph*{GetAlias}\hspace*{\fill}

\begin{list}{}{
\settowidth{\tmplength}{\textbf{Declaração}}
\setlength{\itemindent}{0cm}
\setlength{\listparindent}{0cm}
\setlength{\leftmargin}{\evensidemargin}
\addtolength{\leftmargin}{\tmplength}
\settowidth{\labelsep}{X}
\addtolength{\leftmargin}{\labelsep}
\setlength{\labelwidth}{\tmplength}
}
\begin{flushleft}
\item[\textbf{Declaração}\hfill]
\begin{ttfamily}
protected FUNCTION GetAlias: AnsiString;\end{ttfamily}


\end{flushleft}
\end{list}
\paragraph*{WMPaint}\hspace*{\fill}

\begin{list}{}{
\settowidth{\tmplength}{\textbf{Declaração}}
\setlength{\itemindent}{0cm}
\setlength{\listparindent}{0cm}
\setlength{\leftmargin}{\evensidemargin}
\addtolength{\leftmargin}{\tmplength}
\settowidth{\labelsep}{X}
\addtolength{\leftmargin}{\labelsep}
\setlength{\labelwidth}{\tmplength}
}
\begin{flushleft}
\item[\textbf{Declaração}\hfill]
\begin{ttfamily}
protected procedure WMPaint(var Message: TLMPaint); message LM{\_}PAINT;\end{ttfamily}


\end{flushleft}
\end{list}
\section{Funções e Procedimentos}
\subsection*{Register}
\begin{list}{}{
\settowidth{\tmplength}{\textbf{Declaração}}
\setlength{\itemindent}{0cm}
\setlength{\listparindent}{0cm}
\setlength{\leftmargin}{\evensidemargin}
\addtolength{\leftmargin}{\tmplength}
\settowidth{\labelsep}{X}
\addtolength{\leftmargin}{\labelsep}
\setlength{\labelwidth}{\tmplength}
}
\begin{flushleft}
\item[\textbf{Declaração}\hfill]
\begin{ttfamily}
procedure Register;\end{ttfamily}


\end{flushleft}
\end{list}
\chapter{Unit umi{\_}ui{\_}mi{\_}msgbox{\_}dm}
\section{Uses}
\begin{itemize}
\item \begin{ttfamily}Classes\end{ttfamily}\item \begin{ttfamily}SysUtils\end{ttfamily}\item \begin{ttfamily}Dialogs\end{ttfamily}\item \begin{ttfamily}Graphics\end{ttfamily}\item \begin{ttfamily}StdCtrls\end{ttfamily}\item \begin{ttfamily}System.UITypes\end{ttfamily}\item \begin{ttfamily}mi.rtl.objects.consts.mi{\_}msgbox\end{ttfamily}\end{itemize}
\section{Visão Geral}
\begin{description}
\item[\texttt{\begin{ttfamily}TMi{\_}ui{\_}mi{\_}msgBox\end{ttfamily} Classe}]
\end{description}
\begin{description}
\item[\texttt{get{\_}MI{\_}MsgBox}]
\end{description}
\section{Classes, Interfaces, Objetos e Registros}
\subsection*{TMi{\_}ui{\_}mi{\_}msgBox Classe}
\subsubsection*{\large{\textbf{Hierarquia}}\normalsize\hspace{1ex}\hfill}
TMi{\_}ui{\_}mi{\_}msgBox {$>$} TDataModule
%%%%Descrição
\subsubsection*{\large{\textbf{Campos}}\normalsize\hspace{1ex}\hfill}
\paragraph*{MI{\_}MsgBox1}\hspace*{\fill}

\begin{list}{}{
\settowidth{\tmplength}{\textbf{Declaração}}
\setlength{\itemindent}{0cm}
\setlength{\listparindent}{0cm}
\setlength{\leftmargin}{\evensidemargin}
\addtolength{\leftmargin}{\tmplength}
\settowidth{\labelsep}{X}
\addtolength{\leftmargin}{\labelsep}
\setlength{\labelwidth}{\tmplength}
}
\begin{flushleft}
\item[\textbf{Declaração}\hfill]
\begin{ttfamily}
public MI{\_}MsgBox1: TMI{\_}MsgBox;\end{ttfamily}


\end{flushleft}
\end{list}
\subsubsection*{\large{\textbf{Métodos}}\normalsize\hspace{1ex}\hfill}
\paragraph*{MI{\_}MsgBox1InputBox}\hspace*{\fill}

\begin{list}{}{
\settowidth{\tmplength}{\textbf{Declaração}}
\setlength{\itemindent}{0cm}
\setlength{\listparindent}{0cm}
\setlength{\leftmargin}{\evensidemargin}
\addtolength{\leftmargin}{\tmplength}
\settowidth{\labelsep}{X}
\addtolength{\leftmargin}{\labelsep}
\setlength{\labelwidth}{\tmplength}
}
\begin{flushleft}
\item[\textbf{Declaração}\hfill]
\begin{ttfamily}
public function MI{\_}MsgBox1InputBox(const aTitle, ALabel: AnsiString; var Buff; Template: AnsiString): TModalResult;\end{ttfamily}


\end{flushleft}
\end{list}
\paragraph*{MI{\_}MsgBox1InputPassword}\hspace*{\fill}

\begin{list}{}{
\settowidth{\tmplength}{\textbf{Declaração}}
\setlength{\itemindent}{0cm}
\setlength{\listparindent}{0cm}
\setlength{\leftmargin}{\evensidemargin}
\addtolength{\leftmargin}{\tmplength}
\settowidth{\labelsep}{X}
\addtolength{\leftmargin}{\labelsep}
\setlength{\labelwidth}{\tmplength}
}
\begin{flushleft}
\item[\textbf{Declaração}\hfill]
\begin{ttfamily}
public function MI{\_}MsgBox1InputPassword(const aTitle: AnsiString; var aPassword: AnsiString): TModalResult;\end{ttfamily}


\end{flushleft}
\end{list}
\paragraph*{MI{\_}MsgBox1InputValue}\hspace*{\fill}

\begin{list}{}{
\settowidth{\tmplength}{\textbf{Declaração}}
\setlength{\itemindent}{0cm}
\setlength{\listparindent}{0cm}
\setlength{\leftmargin}{\evensidemargin}
\addtolength{\leftmargin}{\tmplength}
\settowidth{\labelsep}{X}
\addtolength{\leftmargin}{\labelsep}
\setlength{\labelwidth}{\tmplength}
}
\begin{flushleft}
\item[\textbf{Declaração}\hfill]
\begin{ttfamily}
public function MI{\_}MsgBox1InputValue(const aTitle, aLabel: AnsiString; var aValue: Variant): TModalResult;\end{ttfamily}


\end{flushleft}
\end{list}
\paragraph*{MI{\_}MsgBox1MessageBox}\hspace*{\fill}

\begin{list}{}{
\settowidth{\tmplength}{\textbf{Declaração}}
\setlength{\itemindent}{0cm}
\setlength{\listparindent}{0cm}
\setlength{\leftmargin}{\evensidemargin}
\addtolength{\leftmargin}{\tmplength}
\settowidth{\labelsep}{X}
\addtolength{\leftmargin}{\labelsep}
\setlength{\labelwidth}{\tmplength}
}
\begin{flushleft}
\item[\textbf{Declaração}\hfill]
\begin{ttfamily}
public function MI{\_}MsgBox1MessageBox(const aMsg: AnsiString): TModalResult;\end{ttfamily}


\end{flushleft}
\end{list}
\paragraph*{MI{\_}MsgBox1MessageBox{\_}03}\hspace*{\fill}

\begin{list}{}{
\settowidth{\tmplength}{\textbf{Declaração}}
\setlength{\itemindent}{0cm}
\setlength{\listparindent}{0cm}
\setlength{\leftmargin}{\evensidemargin}
\addtolength{\leftmargin}{\tmplength}
\settowidth{\labelsep}{X}
\addtolength{\leftmargin}{\labelsep}
\setlength{\labelwidth}{\tmplength}
}
\begin{flushleft}
\item[\textbf{Declaração}\hfill]
\begin{ttfamily}
public function MI{\_}MsgBox1MessageBox{\_}03(const aMsg: AnsiString; DlgType: TMsgDlgType; Buttons: TMsgDlgButtons): TModalResult;\end{ttfamily}


\end{flushleft}
\end{list}
\paragraph*{MI{\_}MsgBox1MessageBox{\_}04}\hspace*{\fill}

\begin{list}{}{
\settowidth{\tmplength}{\textbf{Declaração}}
\setlength{\itemindent}{0cm}
\setlength{\listparindent}{0cm}
\setlength{\leftmargin}{\evensidemargin}
\addtolength{\leftmargin}{\tmplength}
\settowidth{\labelsep}{X}
\addtolength{\leftmargin}{\labelsep}
\setlength{\labelwidth}{\tmplength}
}
\begin{flushleft}
\item[\textbf{Declaração}\hfill]
\begin{ttfamily}
public function MI{\_}MsgBox1MessageBox{\_}04(aMsg: AnsiString; DlgType: TMsgDlgType; Buttons: TMsgDlgButtons; ButtonDefault: TMsgDlgBtn): TModalResult;\end{ttfamily}


\end{flushleft}
\end{list}
\paragraph*{MI{\_}MsgBox1MessageBox{\_}04{\_}PSItem}\hspace*{\fill}

\begin{list}{}{
\settowidth{\tmplength}{\textbf{Declaração}}
\setlength{\itemindent}{0cm}
\setlength{\listparindent}{0cm}
\setlength{\leftmargin}{\evensidemargin}
\addtolength{\leftmargin}{\tmplength}
\settowidth{\labelsep}{X}
\addtolength{\leftmargin}{\labelsep}
\setlength{\labelwidth}{\tmplength}
}
\begin{flushleft}
\item[\textbf{Declaração}\hfill]
\begin{ttfamily}
public function MI{\_}MsgBox1MessageBox{\_}04{\_}PSItem(aPSItem: TMI{\_}MsgBoxTypes.PSItem; DlgType: TMsgDlgType; Buttons: TMsgDlgButtons; ButtonDefault: TMsgDlgBtn ): TModalResult;\end{ttfamily}


\end{flushleft}
\end{list}
\paragraph*{MI{\_}MsgBox1MessageBox{\_}05}\hspace*{\fill}

\begin{list}{}{
\settowidth{\tmplength}{\textbf{Declaração}}
\setlength{\itemindent}{0cm}
\setlength{\listparindent}{0cm}
\setlength{\leftmargin}{\evensidemargin}
\addtolength{\leftmargin}{\tmplength}
\settowidth{\labelsep}{X}
\addtolength{\leftmargin}{\labelsep}
\setlength{\labelwidth}{\tmplength}
}
\begin{flushleft}
\item[\textbf{Declaração}\hfill]
\begin{ttfamily}
public function MI{\_}MsgBox1MessageBox{\_}05(ATitle: AnsiString; aMsg: AnsiString; DlgType: TMsgDlgType; Buttons: TMsgDlgButtons; ButtonDefault: TMsgDlgBtn ): TModalResult;\end{ttfamily}


\end{flushleft}
\end{list}
\paragraph*{Alert}\hspace*{\fill}

\begin{list}{}{
\settowidth{\tmplength}{\textbf{Declaração}}
\setlength{\itemindent}{0cm}
\setlength{\listparindent}{0cm}
\setlength{\leftmargin}{\evensidemargin}
\addtolength{\leftmargin}{\tmplength}
\settowidth{\labelsep}{X}
\addtolength{\leftmargin}{\labelsep}
\setlength{\labelwidth}{\tmplength}
}
\begin{flushleft}
\item[\textbf{Declaração}\hfill]
\begin{ttfamily}
public Procedure Alert(aTitle: AnsiString;aMsg:AnsiString);\end{ttfamily}


\end{flushleft}
\par
\item[\textbf{Descrição}]
\begin{itemize}
\item A procedure \textbf{\begin{ttfamily}Alert\end{ttfamily}} executa um dialogo com botão \textbf{OK}
\end{itemize}

\end{list}
\paragraph*{Confirm}\hspace*{\fill}

\begin{list}{}{
\settowidth{\tmplength}{\textbf{Declaração}}
\setlength{\itemindent}{0cm}
\setlength{\listparindent}{0cm}
\setlength{\leftmargin}{\evensidemargin}
\addtolength{\leftmargin}{\tmplength}
\settowidth{\labelsep}{X}
\addtolength{\leftmargin}{\labelsep}
\setlength{\labelwidth}{\tmplength}
}
\begin{flushleft}
\item[\textbf{Declaração}\hfill]
\begin{ttfamily}
public Function Confirm(aTitle: AnsiString;aPergunta:AnsiString):Boolean;\end{ttfamily}


\end{flushleft}
\par
\item[\textbf{Descrição}]
\begin{itemize}
\item A função \textbf{\begin{ttfamily}Confirm\end{ttfamily}} executa um dialogo com os botões \textbf{OK} e \textbf{Cancel} fazendo uma pergunta.

\begin{itemize}
\item \textbf{RETORNA:} \begin{itemize}
\item \textbf{True} : Se o botão \textbf{OK} foi pŕessionando;
\item \textbf{False} : Se o botão \textbf{Cancel} foi pŕessionando.
\end{itemize}
\end{itemize}
\end{itemize}

\end{list}
\paragraph*{Prompt}\hspace*{\fill}

\begin{list}{}{
\settowidth{\tmplength}{\textbf{Declaração}}
\setlength{\itemindent}{0cm}
\setlength{\listparindent}{0cm}
\setlength{\leftmargin}{\evensidemargin}
\addtolength{\leftmargin}{\tmplength}
\settowidth{\labelsep}{X}
\addtolength{\leftmargin}{\labelsep}
\setlength{\labelwidth}{\tmplength}
}
\begin{flushleft}
\item[\textbf{Declaração}\hfill]
\begin{ttfamily}
public Function Prompt(aTitle: AnsiString;aPergunta:AnsiString;Var aResult: Variant):Boolean;\end{ttfamily}


\end{flushleft}
\par
\item[\textbf{Descrição}]
\begin{itemize}
\item A função \textbf{\begin{ttfamily}Prompt\end{ttfamily}} mostra um dialogo com dois botões \textbf{OK} e \textbf{Cancel} e um campo input solicitando que o usuário digite um valor.

\begin{itemize}
\item \textbf{RETORNA:} \begin{itemize}
\item \textbf{True} : Se o botão \textbf{ok} foi pŕessionando;
\item \textbf{False} : Se o botão \textbf{cancel} foi pŕessionando.
\item \textbf{aResult} : Retorna a string digitada no formulário;
\end{itemize}
\end{itemize}
\end{itemize}

\end{list}
\paragraph*{InputPassword}\hspace*{\fill}

\begin{list}{}{
\settowidth{\tmplength}{\textbf{Declaração}}
\setlength{\itemindent}{0cm}
\setlength{\listparindent}{0cm}
\setlength{\leftmargin}{\evensidemargin}
\addtolength{\leftmargin}{\tmplength}
\settowidth{\labelsep}{X}
\addtolength{\leftmargin}{\labelsep}
\setlength{\labelwidth}{\tmplength}
}
\begin{flushleft}
\item[\textbf{Declaração}\hfill]
\begin{ttfamily}
public Function InputPassword(aTitle: AnsiString;out aUsername:AnsiString;out apassword:AnsiString):Boolean; Overload;\end{ttfamily}


\end{flushleft}
\par
\item[\textbf{Descrição}]
\begin{itemize}
\item A função \textbf{\begin{ttfamily}InputPassword\end{ttfamily}} mostra um dialogo solicitando o login do usuário e a senha e dois botões \textbf{OK} e \textbf{Cancel}

\begin{itemize}
\item \textbf{RETORNA:} \begin{itemize}
\item \textbf{True} : Se o botão \textbf{ok} foi pŕessionando;
\item \textbf{False} : Se o botão \textbf{cancel} foi pŕessionando.
\item \textbf{aUsername} : Retorna a string com nome do usuário.
\item \textbf{apassword} : Retorna a string com a senha do usuário.
\end{itemize}
\end{itemize}
\end{itemize}

\end{list}
\paragraph*{InputPassword}\hspace*{\fill}

\begin{list}{}{
\settowidth{\tmplength}{\textbf{Declaração}}
\setlength{\itemindent}{0cm}
\setlength{\listparindent}{0cm}
\setlength{\leftmargin}{\evensidemargin}
\addtolength{\leftmargin}{\tmplength}
\settowidth{\labelsep}{X}
\addtolength{\leftmargin}{\labelsep}
\setlength{\labelwidth}{\tmplength}
}
\begin{flushleft}
\item[\textbf{Declaração}\hfill]
\begin{ttfamily}
public Function InputPassword(aTitle: AnsiString;out apassword:AnsiString):Boolean; Overload;\end{ttfamily}


\end{flushleft}
\end{list}
\paragraph*{create}\hspace*{\fill}

\begin{list}{}{
\settowidth{\tmplength}{\textbf{Declaração}}
\setlength{\itemindent}{0cm}
\setlength{\listparindent}{0cm}
\setlength{\leftmargin}{\evensidemargin}
\addtolength{\leftmargin}{\tmplength}
\settowidth{\labelsep}{X}
\addtolength{\leftmargin}{\labelsep}
\setlength{\labelwidth}{\tmplength}
}
\begin{flushleft}
\item[\textbf{Declaração}\hfill]
\begin{ttfamily}
public constructor create(aOwner:TComponent); override;\end{ttfamily}


\end{flushleft}
\end{list}
\section{Funções e Procedimentos}
\subsection*{get{\_}MI{\_}MsgBox}
\begin{list}{}{
\settowidth{\tmplength}{\textbf{Declaração}}
\setlength{\itemindent}{0cm}
\setlength{\listparindent}{0cm}
\setlength{\leftmargin}{\evensidemargin}
\addtolength{\leftmargin}{\tmplength}
\settowidth{\labelsep}{X}
\addtolength{\leftmargin}{\labelsep}
\setlength{\labelwidth}{\tmplength}
}
\begin{flushleft}
\item[\textbf{Declaração}\hfill]
\begin{ttfamily}
function get{\_}MI{\_}MsgBox: TMi{\_}ui{\_}mi{\_}msgBox;\end{ttfamily}


\end{flushleft}
\end{list}
\chapter{Unit umi{\_}ui{\_}radiogroup{\_}lcl}
\section{Uses}
\begin{itemize}
\item \begin{ttfamily}Classes\end{ttfamily}\item \begin{ttfamily}SysUtils\end{ttfamily}\item \begin{ttfamily}LResources\end{ttfamily}\item \begin{ttfamily}Forms\end{ttfamily}\item \begin{ttfamily}Controls\end{ttfamily}\item \begin{ttfamily}Graphics\end{ttfamily}\item \begin{ttfamily}Dialogs\end{ttfamily}\item \begin{ttfamily}ExtCtrls\end{ttfamily}\item \begin{ttfamily}StrUtils\end{ttfamily}\item \begin{ttfamily}ActnList\end{ttfamily}\item \begin{ttfamily}mi{\_}rtl{\_}ui{\_}DmxScroller\end{ttfamily}(\ref{mi_rtl_ui_Dmxscroller})\item \begin{ttfamily}mi{\_}rtl{\_}ui{\_}DmxScroller{\_}Form\end{ttfamily}(\ref{mi_rtl_ui_dmxscroller_form})\item \begin{ttfamily}umi{\_}ui{\_}dmxscroller{\_}form{\_}lcl{\_}attributes\end{ttfamily}(\ref{umi_ui_dmxscroller_form_lcl_attributes})\end{itemize}
\section{Visão Geral}
\begin{description}
\item[\texttt{\begin{ttfamily}TMI{\_}RadioGroup{\_}LCL\end{ttfamily} Classe}]
\end{description}
\begin{description}
\item[\texttt{Register}]
\end{description}
\section{Classes, Interfaces, Objetos e Registros}
\subsection*{TMI{\_}RadioGroup{\_}LCL Classe}
\subsubsection*{\large{\textbf{Hierarquia}}\normalsize\hspace{1ex}\hfill}
TMI{\_}RadioGroup{\_}LCL {$>$} TRadioGroup
%%%%Descrição
\subsubsection*{\large{\textbf{Propriedades}}\normalsize\hspace{1ex}\hfill}
\paragraph*{DmxScroller{\_}Form{\_}Lcl{\_}attributes}\hspace*{\fill}

\begin{list}{}{
\settowidth{\tmplength}{\textbf{Declaração}}
\setlength{\itemindent}{0cm}
\setlength{\listparindent}{0cm}
\setlength{\leftmargin}{\evensidemargin}
\addtolength{\leftmargin}{\tmplength}
\settowidth{\labelsep}{X}
\addtolength{\leftmargin}{\labelsep}
\setlength{\labelwidth}{\tmplength}
}
\begin{flushleft}
\item[\textbf{Declaração}\hfill]
\begin{ttfamily}
published property DmxScroller{\_}Form{\_}Lcl{\_}attributes : TDmxScroller{\_}Form{\_}Lcl{\_}attributes Read {\_}DmxScroller{\_}Form{\_}Lcl{\_}attributes  write SetDmxScroller{\_}Form{\_}Lcl{\_}attributes;\end{ttfamily}


\end{flushleft}
\par
\item[\textbf{Descrição}]
A propriedade \textbf{\begin{ttfamily}DmxScroller{\_}Form{\_}Lcl{\_}attributes\end{ttfamily}} contém o modelo e os cálculos do formulário

\end{list}
\paragraph*{DmxFieldRec}\hspace*{\fill}

\begin{list}{}{
\settowidth{\tmplength}{\textbf{Declaração}}
\setlength{\itemindent}{0cm}
\setlength{\listparindent}{0cm}
\setlength{\leftmargin}{\evensidemargin}
\addtolength{\leftmargin}{\tmplength}
\settowidth{\labelsep}{X}
\addtolength{\leftmargin}{\labelsep}
\setlength{\labelwidth}{\tmplength}
}
\begin{flushleft}
\item[\textbf{Declaração}\hfill]
\begin{ttfamily}
public property DmxFieldRec: pDmxFieldRec Read {\_}pDmxFieldRec   Write  SeTDmxFieldRec;\end{ttfamily}


\end{flushleft}
\par
\item[\textbf{Descrição}]
A propriedade \textbf{\begin{ttfamily}DmxFieldRec\end{ttfamily}} fornece os dados necessários para criar o componente \begin{ttfamily}TMI{\_}Button{\_}LCL\end{ttfamily}(\ref{umi_ui_button_lcl.TMI_Button_LCL}).

\begin{itemize}
\item \textbf{NOTA} \begin{itemize}
\item Esses dados devem ser criados pelo método DmxScroller{\_}Form{\_}Lcl{\_}attributesr.CreateStruct(var ATemplate : \begin{ttfamily}TString\end{ttfamily}(\ref{mi_rtl_ui_Dmxscroller-tString}))
\end{itemize}
\end{itemize}

\end{list}
\subsubsection*{\large{\textbf{Métodos}}\normalsize\hspace{1ex}\hfill}
\paragraph*{GetBuffer}\hspace*{\fill}

\begin{list}{}{
\settowidth{\tmplength}{\textbf{Declaração}}
\setlength{\itemindent}{0cm}
\setlength{\listparindent}{0cm}
\setlength{\leftmargin}{\evensidemargin}
\addtolength{\leftmargin}{\tmplength}
\settowidth{\labelsep}{X}
\addtolength{\leftmargin}{\labelsep}
\setlength{\labelwidth}{\tmplength}
}
\begin{flushleft}
\item[\textbf{Declaração}\hfill]
\begin{ttfamily}
public Procedure GetBuffer;\end{ttfamily}


\end{flushleft}
\par
\item[\textbf{Descrição}]
O método \textbf{\begin{ttfamily}GetBuffer\end{ttfamily}} ler os dados da propriedade \begin{ttfamily}pDmxFieldRec\end{ttfamily}(\ref{mi_rtl_ui_Dmxscroller-pDmxFieldRec}) para o controle (Self).

\end{list}
\paragraph*{DoOnEnter}\hspace*{\fill}

\begin{list}{}{
\settowidth{\tmplength}{\textbf{Declaração}}
\setlength{\itemindent}{0cm}
\setlength{\listparindent}{0cm}
\setlength{\leftmargin}{\evensidemargin}
\addtolength{\leftmargin}{\tmplength}
\settowidth{\labelsep}{X}
\addtolength{\leftmargin}{\labelsep}
\setlength{\labelwidth}{\tmplength}
}
\begin{flushleft}
\item[\textbf{Declaração}\hfill]
\begin{ttfamily}
protected procedure DoOnEnter(Sender: TObject);\end{ttfamily}


\end{flushleft}
\end{list}
\paragraph*{PutBuffer}\hspace*{\fill}

\begin{list}{}{
\settowidth{\tmplength}{\textbf{Declaração}}
\setlength{\itemindent}{0cm}
\setlength{\listparindent}{0cm}
\setlength{\leftmargin}{\evensidemargin}
\addtolength{\leftmargin}{\tmplength}
\settowidth{\labelsep}{X}
\addtolength{\leftmargin}{\labelsep}
\setlength{\labelwidth}{\tmplength}
}
\begin{flushleft}
\item[\textbf{Declaração}\hfill]
\begin{ttfamily}
public Procedure PutBuffer;\end{ttfamily}


\end{flushleft}
\par
\item[\textbf{Descrição}]
O método \textbf{\begin{ttfamily}PutBuffer\end{ttfamily}} salva os dados do controle (Self) para a propriedade \begin{ttfamily}pDmxFieldRec\end{ttfamily}(\ref{mi_rtl_ui_Dmxscroller-pDmxFieldRec})

\end{list}
\paragraph*{DoOnExit}\hspace*{\fill}

\begin{list}{}{
\settowidth{\tmplength}{\textbf{Declaração}}
\setlength{\itemindent}{0cm}
\setlength{\listparindent}{0cm}
\setlength{\leftmargin}{\evensidemargin}
\addtolength{\leftmargin}{\tmplength}
\settowidth{\labelsep}{X}
\addtolength{\leftmargin}{\labelsep}
\setlength{\labelwidth}{\tmplength}
}
\begin{flushleft}
\item[\textbf{Declaração}\hfill]
\begin{ttfamily}
protected procedure DoOnExit(Sender: TObject);\end{ttfamily}


\end{flushleft}
\par
\item[\textbf{Descrição}]
O método \textbf{\begin{ttfamily}DoOnExit\end{ttfamily}} ao perder o foco executa os métodos PuttBuffer e \begin{ttfamily}pDmxFieldRec\end{ttfamily}(\ref{mi_rtl_ui_Dmxscroller-pDmxFieldRec}){\^{}}.DoOnExit(Self).

\end{list}
\section{Funções e Procedimentos}
\subsection*{Register}
\begin{list}{}{
\settowidth{\tmplength}{\textbf{Declaração}}
\setlength{\itemindent}{0cm}
\setlength{\listparindent}{0cm}
\setlength{\leftmargin}{\evensidemargin}
\addtolength{\leftmargin}{\tmplength}
\settowidth{\labelsep}{X}
\addtolength{\leftmargin}{\labelsep}
\setlength{\labelwidth}{\tmplength}
}
\begin{flushleft}
\item[\textbf{Declaração}\hfill]
\begin{ttfamily}
procedure Register;\end{ttfamily}


\end{flushleft}
\end{list}
\chapter{Unit uMi{\_}ui{\_}scrollbox{\_}lcl}
\section{Descrição}
A unit \textbf{\begin{ttfamily}uMi{\_}ui{\_}scrollbox{\_}lcl\end{ttfamily}} implementa a classe \begin{ttfamily}TMi{\_}ScrollBox{\_}LCL\end{ttfamily}(\ref{uMi_ui_scrollbox_lcl.TMi_ScrollBox_LCL}) para ser usado como container para UiDmxScroller.

\begin{itemize}
\item \textbf{VERSÃO} \begin{itemize}
\item Alpha {-} 0.5.0.687
\end{itemize}
\item \textbf{CÓDIGO FONTE}: \begin{itemize}
\item 
\end{itemize}
\item \textbf{PENDÊNCIAS} \begin{itemize}
\item O evento onGetTemplate não está funcionando de forma automática, preciso setá{-}lo.
\item Antes de executar onGetTemplate definir um nome de fonte padrão e checar se o mesmo precede a fonte editada na IDE.
\end{itemize}
\item \textbf{HISTÓRICO} \begin{itemize}
\item Criado por: Paulo Sérgio da Silva Pacheco paulosspacheco@yahoo.com.br) \begin{itemize}
\item \textbf{2022{-}02{-}21 } \begin{itemize}
\item Data em que essa unity foi criada. ✅
\end{itemize}
\item \textbf{2022{-}02{-}22 14:00} \begin{itemize}
\item Documentar a unidade. ✅ {-}
\end{itemize}
\item \textbf{2022{-}02{-}24 21:00} \begin{itemize}
\item Implementar os enventos OnEnter e OnExit de \begin{ttfamily}TMi{\_}ScrollBox{\_}LCL\end{ttfamily}(\ref{uMi_ui_scrollbox_lcl.TMi_ScrollBox_LCL}) para executar os eventos OnEnter e onExit de UiDmxScroller ✅
\end{itemize}
\end{itemize}\begin{itemize}
\item \textbf{2022{-}03{-}01 11:00} \begin{itemize}
\item Em \begin{ttfamily}TMi{\_}ScrollBox{\_}LCL.Create\end{ttfamily}(\ref{uMi_ui_scrollbox_lcl.TMi_ScrollBox_LCL-Create}) inicializar a fonte fixa \textbf{Courie New} se windows ou \textbf{DejaVu Sans Mono} se linux e em ambas as plataformas foi definido o tamanho da fonte em 12 px. ✅
\end{itemize}
\item \textbf{2022{-}03{-}22 20:00} \begin{itemize}
\item Os eventos onEnter e OnExit não estavam executando \begin{ttfamily}TUiDmxScroller.DoOnEnter\end{ttfamily}(\ref{mi_rtl_ui_Dmxscroller.TUiDmxScroller-DoOnEnter}) e \begin{ttfamily}TUiDmxScroller.OnExit\end{ttfamily}(\ref{mi_rtl_ui_Dmxscroller.TUiDmxScroller-onExit}) caso o usuário não tenha iniciado os eventos OnEnter e OnExit do formulário isso gerava problema em GetBuffer e SetBuffer. Corrigido. ✅
\end{itemize}
\item \textbf{2022{-}06{-}27 18:19} \begin{itemize}
\item Redefinir procedure ComputeScrollbars; virtual; \begin{itemize}
\item Implementei mais meus problema continuaram. Tem algo muito errado no controle ScrollBox do Lazarus.
\end{itemize}
\end{itemize}
\end{itemize}
\end{itemize}
\end{itemize}
\section{Uses}
\begin{itemize}
\item \begin{ttfamily}Classes\end{ttfamily}\item \begin{ttfamily}SysUtils\end{ttfamily}\item \begin{ttfamily}LResources\end{ttfamily}\item \begin{ttfamily}Forms\end{ttfamily}\item \begin{ttfamily}Controls\end{ttfamily}\item \begin{ttfamily}Graphics\end{ttfamily}\item \begin{ttfamily}Dialogs\end{ttfamily}\item \begin{ttfamily}LMessages\end{ttfamily}\item \begin{ttfamily}types\end{ttfamily}\item \begin{ttfamily}mi{\_}rtl{\_}Ui{\_}Methods\end{ttfamily}(\ref{mi_rtl_ui_methods})\item \begin{ttfamily}mi{\_}rtl{\_}ui{\_}Dmxscroller\end{ttfamily}(\ref{mi_rtl_ui_Dmxscroller})\end{itemize}
\section{Visão Geral}
\begin{description}
\item[\texttt{\begin{ttfamily}TMi{\_}ScrollBox{\_}LCL\end{ttfamily} Classe}]
\end{description}
\begin{description}
\item[\texttt{Register}]
\end{description}
\section{Classes, Interfaces, Objetos e Registros}
\subsection*{TMi{\_}ScrollBox{\_}LCL Classe}
\subsubsection*{\large{\textbf{Hierarquia}}\normalsize\hspace{1ex}\hfill}
TMi{\_}ScrollBox{\_}LCL {$>$} TScrollBox
\subsubsection*{\large{\textbf{Descrição}}\normalsize\hspace{1ex}\hfill}
A Classe \textbf{\begin{ttfamily}TMi{\_}ScrollBox{\_}LCL\end{ttfamily}} foi redefinida para que os eventos DmxScroller.OnEnter e DmxScroller.OnExit sejam executados quando o ScrollBox recebe e perde o foco.

\begin{itemize}
\item \textbf{REFERẼNCIAS} \begin{itemize}
\item [TScrollBox](https://lazarus-ccr.sourceforge.io/docs/lcl/forms/tscrollbox.html)
\item https://wiki.freepascal.org/Form{\_}Tutorial (Form{\_}Tutorial) {-}
\end{itemize}
\end{itemize}\subsubsection*{\large{\textbf{Propriedades}}\normalsize\hspace{1ex}\hfill}
\paragraph*{UiDmxScroller}\hspace*{\fill}

\begin{list}{}{
\settowidth{\tmplength}{\textbf{Declaração}}
\setlength{\itemindent}{0cm}
\setlength{\listparindent}{0cm}
\setlength{\leftmargin}{\evensidemargin}
\addtolength{\leftmargin}{\tmplength}
\settowidth{\labelsep}{X}
\addtolength{\leftmargin}{\labelsep}
\setlength{\labelwidth}{\tmplength}
}
\begin{flushleft}
\item[\textbf{Declaração}\hfill]
\begin{ttfamily}
public property UiDmxScroller : TUiDmxScroller read {\_}UiDmxScroller write SetUiDmxScroller;\end{ttfamily}


\end{flushleft}
\par
\item[\textbf{Descrição}]
A propriedade \textbf{\begin{ttfamily}UiDmxScroller\end{ttfamily}} foi crada para que os eventos DmxScroller.OnEnter e DmxScroller.OnExit sejam executados quando o ScrollBox recebe o foco e perde o foco.

\begin{itemize}
\item A propriedade \textbf{\begin{ttfamily}UiDmxScroller\end{ttfamily}} não deve ser published por que a mesma deve ser inicializada automatiocamente em TUiDmxScroller.SetParentLCL()
\end{itemize}

\end{list}
\subsubsection*{\large{\textbf{Métodos}}\normalsize\hspace{1ex}\hfill}
\paragraph*{DoOnEnter}\hspace*{\fill}

\begin{list}{}{
\settowidth{\tmplength}{\textbf{Declaração}}
\setlength{\itemindent}{0cm}
\setlength{\listparindent}{0cm}
\setlength{\leftmargin}{\evensidemargin}
\addtolength{\leftmargin}{\tmplength}
\settowidth{\labelsep}{X}
\addtolength{\leftmargin}{\labelsep}
\setlength{\labelwidth}{\tmplength}
}
\begin{flushleft}
\item[\textbf{Declaração}\hfill]
\begin{ttfamily}
protected procedure DoOnEnter(Sender: TObject);\end{ttfamily}


\end{flushleft}
\par
\item[\textbf{Descrição}]
A procedure \textbf{\begin{ttfamily}DoOnEnter\end{ttfamily}} é usado para executar o evento onEnter de DmxScroller

\end{list}
\paragraph*{DoOnExit}\hspace*{\fill}

\begin{list}{}{
\settowidth{\tmplength}{\textbf{Declaração}}
\setlength{\itemindent}{0cm}
\setlength{\listparindent}{0cm}
\setlength{\leftmargin}{\evensidemargin}
\addtolength{\leftmargin}{\tmplength}
\settowidth{\labelsep}{X}
\addtolength{\leftmargin}{\labelsep}
\setlength{\labelwidth}{\tmplength}
}
\begin{flushleft}
\item[\textbf{Declaração}\hfill]
\begin{ttfamily}
protected procedure DoOnExit(Sender: TObject);\end{ttfamily}


\end{flushleft}
\par
\item[\textbf{Descrição}]
A procedure \textbf{\begin{ttfamily}DoOnExit\end{ttfamily}} é usado para executar o evento onExit de DmxScroller

\end{list}
\paragraph*{Create}\hspace*{\fill}

\begin{list}{}{
\settowidth{\tmplength}{\textbf{Declaração}}
\setlength{\itemindent}{0cm}
\setlength{\listparindent}{0cm}
\setlength{\leftmargin}{\evensidemargin}
\addtolength{\leftmargin}{\tmplength}
\settowidth{\labelsep}{X}
\addtolength{\leftmargin}{\labelsep}
\setlength{\labelwidth}{\tmplength}
}
\begin{flushleft}
\item[\textbf{Declaração}\hfill]
\begin{ttfamily}
public Constructor Create(aOwner: TComponent); Override;\end{ttfamily}


\end{flushleft}
\par
\item[\textbf{Descrição}]
O constructor \textbf{\begin{ttfamily}Create\end{ttfamily}} foi redefinido para que os eventos OnEnter e OnExit sajam iniciados com \begin{ttfamily}DoOnEnter\end{ttfamily}(\ref{uMi_ui_scrollbox_lcl.TMi_ScrollBox_LCL-DoOnEnter}) e \begin{ttfamily}DoOnExit\end{ttfamily}(\ref{uMi_ui_scrollbox_lcl.TMi_ScrollBox_LCL-DoOnExit}) respectivamente.

\end{list}
\paragraph*{SetUiDmxScroller}\hspace*{\fill}

\begin{list}{}{
\settowidth{\tmplength}{\textbf{Declaração}}
\setlength{\itemindent}{0cm}
\setlength{\listparindent}{0cm}
\setlength{\leftmargin}{\evensidemargin}
\addtolength{\leftmargin}{\tmplength}
\settowidth{\labelsep}{X}
\addtolength{\leftmargin}{\labelsep}
\setlength{\labelwidth}{\tmplength}
}
\begin{flushleft}
\item[\textbf{Declaração}\hfill]
\begin{ttfamily}
protected Procedure SetUiDmxScroller(aUiDmxScroller:TUiDmxScroller);\end{ttfamily}


\end{flushleft}
\par
\item[\textbf{Descrição}]
O método \textbf{\begin{ttfamily}SetUiDmxScroller\end{ttfamily}} é usado para inicializar o atributo {\_}UiDmxScroller

\begin{itemize}
\item \textbf{NOTA} \begin{itemize}
\item Caso a propriedade DmxScroller.Active seja \textbf{true}, então deve{-}se toná{-}la \textbf{false} para que perca o vínculo com a visão anterior.
\end{itemize}
\end{itemize}

\end{list}
\paragraph*{Refresh}\hspace*{\fill}

\begin{list}{}{
\settowidth{\tmplength}{\textbf{Declaração}}
\setlength{\itemindent}{0cm}
\setlength{\listparindent}{0cm}
\setlength{\leftmargin}{\evensidemargin}
\addtolength{\leftmargin}{\tmplength}
\settowidth{\labelsep}{X}
\addtolength{\leftmargin}{\labelsep}
\setlength{\labelwidth}{\tmplength}
}
\begin{flushleft}
\item[\textbf{Declaração}\hfill]
\begin{ttfamily}
public Procedure Refresh;\end{ttfamily}


\end{flushleft}
\end{list}
\paragraph*{WidthChar}\hspace*{\fill}

\begin{list}{}{
\settowidth{\tmplength}{\textbf{Declaração}}
\setlength{\itemindent}{0cm}
\setlength{\listparindent}{0cm}
\setlength{\leftmargin}{\evensidemargin}
\addtolength{\leftmargin}{\tmplength}
\settowidth{\labelsep}{X}
\addtolength{\leftmargin}{\labelsep}
\setlength{\labelwidth}{\tmplength}
}
\begin{flushleft}
\item[\textbf{Declaração}\hfill]
\begin{ttfamily}
public function WidthChar:Byte;\end{ttfamily}


\end{flushleft}
\end{list}
\paragraph*{HeightChar}\hspace*{\fill}

\begin{list}{}{
\settowidth{\tmplength}{\textbf{Declaração}}
\setlength{\itemindent}{0cm}
\setlength{\listparindent}{0cm}
\setlength{\leftmargin}{\evensidemargin}
\addtolength{\leftmargin}{\tmplength}
\settowidth{\labelsep}{X}
\addtolength{\leftmargin}{\labelsep}
\setlength{\labelwidth}{\tmplength}
}
\begin{flushleft}
\item[\textbf{Declaração}\hfill]
\begin{ttfamily}
public function HeightChar:Byte;\end{ttfamily}


\end{flushleft}
\end{list}
\section{Funções e Procedimentos}
\subsection*{Register}
\begin{list}{}{
\settowidth{\tmplength}{\textbf{Declaração}}
\setlength{\itemindent}{0cm}
\setlength{\listparindent}{0cm}
\setlength{\leftmargin}{\evensidemargin}
\addtolength{\leftmargin}{\tmplength}
\settowidth{\labelsep}{X}
\addtolength{\leftmargin}{\labelsep}
\setlength{\labelwidth}{\tmplength}
}
\begin{flushleft}
\item[\textbf{Declaração}\hfill]
\begin{ttfamily}
procedure Register;\end{ttfamily}


\end{flushleft}
\end{list}
\section{Tipos}
\subsection*{PSItem}
\begin{list}{}{
\settowidth{\tmplength}{\textbf{Declaração}}
\setlength{\itemindent}{0cm}
\setlength{\listparindent}{0cm}
\setlength{\leftmargin}{\evensidemargin}
\addtolength{\leftmargin}{\tmplength}
\settowidth{\labelsep}{X}
\addtolength{\leftmargin}{\labelsep}
\setlength{\labelwidth}{\tmplength}
}
\begin{flushleft}
\item[\textbf{Declaração}\hfill]
\begin{ttfamily}
PSItem =  TUiMethods.PSItem;\end{ttfamily}


\end{flushleft}
\end{list}
\chapter{Unit Unit1}
\section{Uses}
\begin{itemize}
\item \begin{ttfamily}Classes\end{ttfamily}\item \begin{ttfamily}SysUtils\end{ttfamily}\item \begin{ttfamily}Forms\end{ttfamily}\item \begin{ttfamily}Controls\end{ttfamily}\item \begin{ttfamily}Graphics\end{ttfamily}\item \begin{ttfamily}Dialogs\end{ttfamily}\item \begin{ttfamily}uDmxScroller{\_}Form{\_}Lcl{\_}add{\_}test\end{ttfamily}(\ref{uDmxScroller_Form_Lcl_add_test})\end{itemize}
\section{Visão Geral}
\begin{description}
\item[\texttt{\begin{ttfamily}TDmxScroller{\_}Form{\_}Lcl{\_}add{\_}test1\end{ttfamily} Classe}]
\end{description}
\section{Classes, Interfaces, Objetos e Registros}
\subsection*{TDmxScroller{\_}Form{\_}Lcl{\_}add{\_}test1 Classe}
\subsubsection*{\large{\textbf{Hierarquia}}\normalsize\hspace{1ex}\hfill}
TDmxScroller{\_}Form{\_}Lcl{\_}add{\_}test1 {$>$} \begin{ttfamily}TDmxScroller{\_}Form{\_}Lcl{\_}add{\_}test\end{ttfamily}(\ref{uDmxScroller_Form_Lcl_add_test.TDmxScroller_Form_Lcl_add_test}) {$>$} 
TForm
%%%%Descrição
\section{Variáveis}
\subsection*{DmxScroller{\_}Form{\_}Lcl{\_}add{\_}test1}
\begin{list}{}{
\settowidth{\tmplength}{\textbf{Declaração}}
\setlength{\itemindent}{0cm}
\setlength{\listparindent}{0cm}
\setlength{\leftmargin}{\evensidemargin}
\addtolength{\leftmargin}{\tmplength}
\settowidth{\labelsep}{X}
\addtolength{\leftmargin}{\labelsep}
\setlength{\labelwidth}{\tmplength}
}
\begin{flushleft}
\item[\textbf{Declaração}\hfill]
\begin{ttfamily}
DmxScroller{\_}Form{\_}Lcl{\_}add{\_}test1: TDmxScroller{\_}Form{\_}Lcl{\_}add{\_}test1;\end{ttfamily}


\end{flushleft}
\end{list}
\end{document}
